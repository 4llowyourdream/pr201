\documentclass[12pt, addpoints, answers]{exam} % добавить или удалить answers в скобках, чтобы показать ответы
%\usepackage[T2A, T1]{fontenc}
%\usepackage[utf8x]{inputenc}
%\usepackage[greek, russian]{babel}
%\usepackage[OT1]{fontenc}


\usepackage{fontspec}
\usepackage{polyglossia}

\setdefaultlanguage{russian}
\setotherlanguages{english, greek}

\setmainfont[Ligatures=TeX]{Linux Libertine O}
% http://www.linuxlibertine.org/index.php?id=91&L=1

%\usepackage{mathtools}
\usepackage{comment}
\usepackage{booktabs}
\usepackage{amsmath}
\usepackage{tikz}
\usepackage{amsfonts}
\usepackage{amssymb}
\usepackage[left=1cm,right=1cm,top=2cm,bottom=2cm]{geometry}
\DeclareMathOperator{\E}{\mathbb{E}}
\DeclareMathOperator{\Var}{\mathbb{V}\mathrm{ar}}
\DeclareMathOperator{\Cov}{\mathbb{C}\mathrm{ov}}
\DeclareMathOperator{\Corr}{\mathbb{C}\mathrm{orr}}
\let\P\relax
\DeclareMathOperator{\P}{\mathbb{P}}
\newcommand{\cN}{\mathcal{N}}
\newcommand{\hbeta}{\hat{\beta}}

\usepackage{floatrow}
%\newfloatcommand{capbtabbox}{table}[][\FBwidth]

\begin{document}

\pagestyle{headandfoot}
\runningheadrule
\firstpageheader{Теория вероятностей}{Невероятная регата, кильки}{26 сентября 2015}
\firstpageheadrule
\runningheader{Теория вероятностей}{Невероятная регата, кильки}{26 сентября 2015}
\firstpagefooter{}{}{}
\runningfooter{}{}{}
\runningfootrule




\hqword{Задача}
\hpgword{Страница}
\hpword{Максимум}
\hsword{Баллы}
\htword{Итого}
\pointname{\%}
%\renewcommand{\solutiontitle}{\noindent\textbf{Решение:}\par\noindent}
\renewcommand{\solutiontitle}{}

%Таблица с результатами заполняется проверяющим работу. Пожалуйста, не делайте в ней пометок.

%\begin{center}
%  \gradetable[h][questions]
%\end{center}

\vspace{0.2in}

\makebox[\textwidth]{Команда:\enspace\hrulefill}


\begin{center}
ТУР ПЕРВЫЙ
\end{center}

\begin{questions}


\question
У пиратов выдалась удачная неделя. В первый и второй день им удалось подбить и ограбить по 15 торговых судов, а на третий день --- аж 20. Какова вероятность того, что корабль «Адмирал Крузенштерн» был сбит на третий день, если встречу пиратского корабля с торговым определяет фортуна?
\begin{solution}
 P(A)= 0.4
\end{solution}


\question В мешке лежали один шар белого цвета и один шар черного цвета. Из него извлекли один шар и положили в пустой ящик. Также в ящик положили еще один белый шар. Наконец, из ящика извлекли один шар, он оказался белым. Какова вероятность того, что оставшийся в ящике шар тоже белый?
\begin{solution}
 $2/3$
\end{solution}

\question
Пираты развлекаются как могут. В мешке 3 заряженных револьера, и 7 не заряженных. Пират достает револьер и радостно стреляет в капитана. С какой вероятностью после первого раунда капитан останется жив?
А теперь скажите, с какой вероятностью оставшийся в живых, но обезумевший, капитан пристрелит нахала, вытащив из всё того же мешка один из оставшихся револьверов?
\begin{solution}
 $\dfrac{7}{10}, \; \dfrac13$
\end{solution}

\question
Каюты драит очень необязательный пират. Первый день он точно отдежурит, второй --- с вероятность 1/2, третий --- с 1/4 и так далее.

С какой вероятностью пират отдежурит все 7 дней?
\begin{solution}
 $\dfrac12\cdot\dfrac14\cdot \dots \cdot\dfrac{1}{64}$
\end{solution}



\question  Во время очередного удачного рейда пиратами были добыты шашки и три бутылки рома. Из всех на борту правила в шашки знают только капитан и Некеша.


Каждый шторм, случившийся во время партии повышает вероятность капитана выиграть в два раза. Известно, что во-время последней партии шторм случился аж три раза, что сравняло шансы на победу капитана и попугая. Найдите вероятность победы попугая, если бы никаких штормов не происходило.
\begin{solution}
 $\frac{15}{16}$
\end{solution}

\question Взломщик Евгений перебирает по одному ключи из связки в 20 ключей, чтобы открыть сейф. Только один ключ из связки подходит к замку. Найдите вероятность того, что подойдет 19-ый по счёту ключ.
\begin{solution}
 $1/20$
\end{solution}

\question Известно, что $\P(\{\spadesuit\})=\P(\{\heartsuit\})=\P(\{\diamondsuit\})=\P(\{\clubsuit\})=1/4$.  Заданы события $A=\{\spadesuit, \heartsuit\}$ и $B=\{\diamondsuit, \clubsuit\}$. Верно ли, что $A$ и $B$ зависимы?
\begin{solution}
 да
\end{solution}


\question Существуют ли $n$ случайных событий, таких, что любые $n-1$ независимы, а все они вместе -- зависимы? 
\begin{solution}
существуют
\end{solution}

\question Зависимы ли случайные величины $\xi$ и $\arcsin(\xi)$?

\begin{solution}
зависимы
\end{solution}

\question Один небезызвестный модник В.С. решил побаловать себя новой парой кроссовок и заказать эксклюзивный экземпляр из US. Как не горестно, но помимо стоимости кроссовок,  265\$, придется заплатить еще и за доставку 25\$. B.C. планирует потратить всю зарплату ассистента, которая составляет 20000 без учета 13\% подоходного налога. Какова вероятность, что покупка удастся, если курс доллара к рублю распределен равномерно на интервале $[55,75]$.

\begin{solution}
 $\frac{1}{4}$
\end{solution}


\question Пусть $X_1, X_2, \dots, X_n$ --- случайная выборка размера $n$, где $X_i$ имеет функцию распределения $F_X$. Чему равна $P(\min \{ X_1, \ldots, X_n\} \le t)$?
\begin{solution}
 $1 - (1 - F_X(t))^n$
\end{solution}


\question Пусть $X_1, X_2, \dots, X_n$ --- случайная выборка размера $n$, где $X_i$ имеет функцию распределения $F_X$. Чему равна $P(\max \{ X_1, \ldots, X_n\} \le t)$?

\begin{solution}
 $[F_x(t)]^n$
\end{solution}



\question В гирлянде 5 лампочек. Каждая лампочка может не работать с вероятностью $p$. Найдите вероятность того, что не работают 1 и 5 лампочки, а все остальные работают.
\begin{solution}
 $p^{2}(1-p)^{3}$
\end{solution}


\question В урне 4 белых и 6 черных шаров. Из урны наудачу извлечены 2 шара. Найдите вероятность того, что они разного цвета.
\begin{solution}
 $(C_{4}^1 \cdot C_{6}^1)/ (C_{10}^2) = 8/15$
\end{solution}

\question Верно ли, что $\P(A|B) + \P(A|\bar{B}) \equiv 1$?
\begin{solution}
 неверно
\end{solution}

\question Сколько решений в целых числах имеет  уравнение $X_1 + X_2 + X_3 = 10$, где  $X_i \in [1,6] \: \: i \in \{1,2,3\}$?
\begin{solution}
 27
\end{solution}

\question  Возможно ли, что $P(A \cap \bar{B}) = P(B \cap \bar{A}) = \frac{4}{7}$?

\begin{solution}
 нет
\end{solution}

\question  Известно, что $P(A) = P(B) = 1$. Найдите $P(A\Delta B)$
\footnote{$A\Delta B = (A \cap \bar B) \cup ( \bar A \cap B)$}.

\begin{solution}
 0
\end{solution}

\question Найдите $\E(X)$, если $f_X(x) = \frac{1}{2}e^{-|x|}$, $x \in R$.

\begin{solution}
 0
\end{solution}

\question  Имеет место менделевское тригибридное скрещивание: гетерозиготный по всем трём признакам М (умный, красивый, на мерседесе) $\times$ гетерозиготная по всем трём признакам Ж (умная, красивая, на мерседесе). Какова вероятность, что их ребенок окажется тупым, непривлекательным, на жигули?\\
Гетерозигота здесь --- АаВвСс.\\
({\it Hint:} при данном скрещивании один ген каждого типа заимствуется у отца, второй --- у матери. Негативное качество реализуется только при наличии 2х рецессивный генов, к примеру, {\it aa}.)

\begin{solution}
 $\frac{1}{64}$
\end{solution}

\question Елена Прекрасная нашла не менее прекрасного котенка и решила, что завтра непременно придумает ему кличку, а именно, наугад вытянет любое из следующего перечня \{{\it Вейерштрасс, Пифагор, Супремум, Чевиан}\}. Однако с вероятностью $\frac{1}{3}$ завтра будет пасмурная погода, в которую Елена Прекрасная принципиально ничем важным не занимается, тогда на помощь придет ее служанка, которая, недолго думая, назовет котика именем из списка так, чтобы первая буква имени совпадала с сегодняшнем днем недели. Найдите вероятность, что котенка назовут Пифагором (все дни недели равновероятны).


\begin{solution}
 $\frac{11}{42}$
\end{solution}

\question Функция плотности случайной величины Х имеет вид

\begin{center}
	$f(x)= \begin{cases}
	\dfrac{1}{6}, x\in [3;9]\\
	0, x\notin [3;9]
\end{cases}$
\end{center}
 Найдите вероятность того, что $x\le 8 $.

\begin{solution}
 $5/6$
\end{solution} 



\question В тесте 20 вопросов с 4 вариантами ответа, 30 вопросов с 5 вариантами ответа и 15 вопросов с 3 вариантами. Вася совершенно не подготовился к тесту и ставит все ответы случайным образом. Каково математическое ожидание количества правильных решенных заданий, если в каждом вопросе только один правильный ответ.

\begin{solution}
 $16$
\end{solution}

\question У Кати 10 пар совершенно одинаковых чёрных перчаток, и лежат они все в одном ящике. Катя боится опоздать на автобус, поэтому берет две перчатки наугад. Найдите вероятность того, что она сможет их надеть.

\begin{solution}
 $10/19$
\end{solution}

\question Есть 15 шаров в корзине, 7 белых, 3 черных и 5 зеленых. Каждый раз вытаскивается шар, запоминается цвет и кладется обратно. Найдите вероятность того, что за 3 раза вытащим шары 3-х разных цветов?

\begin{solution}
 $105\cdot6/15^3$
\end{solution}

\question Случайная величина X принимает три значения 1, 2 и 3 с вероятностями 1/6, 2/6 и 3/6, соответственно. Найдите $P(X=3|X>1.5)$


\begin{solution}
 $3/5$
\end{solution}

\question Имеется 2 стандартные игральные кости с 6 гранями. Васе для победы нужно выкинуть в сумме 4 и больше. Какова вероятность, что Вася выиграет?

\begin{solution}
 $1-3/36=11/12$
\end{solution}



\question Василий очень любит дорогой коньяк, но еще больше он любит качественные автомобили. Приобретая новую машину, Вася решил быть предельно настойчивым и отыскать свою мечту цвета баклажан. Однако это и впрямь не просто, вероятность того, что в салоне окажется нужная марка автомобиля, в желаемой цветовой гамме, всего 0.09. Найдите вероятность того, что Василию в поисках своей мечты придется обойти ровно 12 автомобильных центров.

\begin{solution}
 $0.09\cdot(0.91)^{11}$
\end{solution}

\question
В револьвере 6 камор (ячеек для патронов). Для игры в русскую рулетку в смежные каморы кладут два патрона, барабан прокручивают и закрывают. После первого выстрела игрок остаётся в живых. Найдите вероятность того, что он останется жив и после второго выстрела (играет в одиночку).
\begin{solution}
$\dfrac{3}{4}$
\end{solution}

\question
Аня посещает семинар по макроэкономике с вероятностью $\dfrac{3}{4}$, Таня --- с вероятностью $\dfrac{1}{2}$. Каждая из девушек всегда пишет конспект, когда оказывается на семинаре. Дима ходит на все семинары, но пишет конспект с вероятностью всего лишь $\dfrac{1}{4}$. Перед контрольной Дима осознаёт, что ему нужны конспекты всех семинаров. Он может использовать свои или, в случае необходимости, взять конспект у Ани или у Тани. Найдите вероятность того, что у Димы будут конспекты всех семинаров.

\begin{solution}

$\left(\dfrac14 + \dfrac34\cdot\left(1 - \dfrac{1}{4}\cdot\dfrac12\right)\right)^{10} = \left(\dfrac{29}{32}\right)^{10}\approx 0.374$
\end{solution}


\question Если из события А следует событие В, то какой знак нужно поставить между $P(A|C)$ и $P(B|C)$.
\begin{solution}
 $\leq$
\end{solution}

\question В мешке $n$ шаров: $n/2$ белых и столько же чёрных. Какова вероятность вытащить набор из $n/2$ шаров, в котором будет одинаковое количество черных и белых?

\begin{solution} $\frac{(C^{n/4}_{n/2})^2}{C_{n}^{n/2}}$
\end{solution}

\question Маша придумала функцию распределения и показала её Васе. Вася заметил, что функция не убывает на участке от $a$ до $b$, и $F’[(a+b)/2] = 0$. Можно ли заключить, что Маша ошиблась в поиске функции распределения?
\begin{solution}
 Нет
\end{solution}


\end{questions}

\newpage
\vspace{0.2in}

\makebox[\textwidth]{Команда:\enspace\hrulefill}


\begin{center}
ТУР ВТОРОЙ
\end{center}


\begin{questions}

\question Даша забыла ключи от квартиры и стучится до тех пор, пока Вася не откроет. От каждого стука Вася проснётся (и сонным пойдёт открывать) с вероятностью $1/10$. Сколько в среднем нужно ударить Даше?

\begin{solution}
 10
\end{solution}

\question Андрей стреляет по мишени из лука по мишени. Расстояние в метрах от центра мишени распределено экспоненциально. Cреднее отклонение от центра равно 20 см. Какова вероятность, что два раза подряд стрела будет находится на расстоянии 1 метр от центра?

\begin{solution}
 0
\end{solution}

\question Если $P(A | B \cap C) = P(A | B)$, то чему равно $P(A|B)P(B|C)P(C)$?.


\begin{solution}
 $P(A \cap B \cap C)$
\end{solution}


\question Любой уважающий себя экономист должен быть и психологом и математиком. Ничто не определяет эти качество так выразительно, как игра в покер, а знание основных комбинаций в игре, умение посчитать вероятности их появления это залог победы (не говоря уже о том, что нужно спрогнозировать все это для своего оппонента). Какова вероятность появления Фул-Хауса (3 карты одного достоинства и 2 карты другого достоинства при выборе 5 карт из стандартной колоды в 52 карты).

\begin{solution}
 $\dfrac{13\cdot 12\cdot C_{4}^3 \cdot C_{4}^2}{ C_{52}^5}$
\end{solution}


\question В покере комбинацию из 5 карт называют флэш-рояль, если в ней есть все карты от 10 до Туза, и они одной масти.  Найдите вероятность, что после вытаскивания 5 карт из стандартной колоды (52 карты) образуется флэш рояль.

\begin{solution}
 $\frac{4}{C_{52}^5}$
\end{solution}

\question Андрей подбрасывает правильную монетку 4 раза. Найдите вероятность того, что количество орлов будет меньше либо равно 2.

\begin{solution}
 $\frac{1}{2}^4 + 4\cdot \frac{1}{2}^4 + 6\cdot \frac{1}{2}^4 = \frac{11}{16}$
\end{solution}


\question Петр где-то услышал, что у людей с рыжим цветом волос нет души, и забеспокоился. К счастью, это не точно: у каждого человека с рыжими волосами душа есть с вероятностью $\frac13$. Петр сам шатен, но все 10 его друзей рыжие ---  неудивительно, что его волнует этот вопрос. Какова вероятность того, что ровно у четверых из друзей Петра есть душа?


\begin{solution}
 $C_{10}^4\left(\frac{1}{3}\right)^4\left(\frac{2}{3}\right)^6$
\end{solution}

\question Александр умеет играть на гитаре 10 аккордов (неплохо!). Ольге, в которую Александр влюблен, нравится песня группы Сплин, в которой всего 6 аккордов (все 6 из тех, что знакомы молодому человеку, но он не знает, какие именно, поэтому выбирает наугад из тех, что знакомы). Александр хочет исполнить эту песню и завоевать сердце Ольги, но для этого он должен сыграть правильно хотя бы 5 аккордов из 6, тогда она не заметит подвоха. Какова вероятность, что ему это удастся?


\begin{solution}
 $\frac{12^{10}}{10!}e^{-12}$
\end{solution}






\question Известно что $\P(A) = 1$, $\P(B)=1$. Верно ли, что события А и B независимы?

\begin{solution}
 да
\end{solution}

\question Баба Маша и тётя Зина любят смотреть «Угадай мелодию». Баба Маша угадывает мелодию ровно с 3-х, 4-х или 5-ти нот с вероятностями 1/4, 1/2 и 1/4 соответственно. Функция распределения числа нот, необходимых тёте Зине для отгадывания, такова
\[
F(x)=\begin{cases}
0, \; x \leq4 \\
1/4, \; x \in[4,5)\\
1/2, \; x \in]5,6)\\
3/4, \; x \in[6,7)\\
1, \; x \geq7 \\
\end{cases}
\]
Найдите вероятность того, что одновременно тётя Зина отгадает мелодию ровно с 5-ти нот, а баба Маша с 4-х и более.



\begin{solution}
 3/16
\end{solution}

\question Кот Базилио нашёл мешок с 50 монетами. Однако хитрая лиса Алиса точно знает, что в мешке есть 5 фальшивых монет. Какова вероятность того, что среди 8 монет, которые кот Базилио сразу достанет из мешка, будет ровно 4 фальшивых?


\begin{solution}
 $ \dfrac{C_{5}^4C_{45}^4}{C_{50}^8}$
\end{solution}

\question Маша и Ваня играют в игру: ставят на бумаге точку наугад на отрезке от 0 до 100. Маша уже сделала свой ход. Оказалось, что она выбрала точку 40. Найдите вероятность того, что после хода Вани сумма их точек не будет превышать 60.


\begin{solution}
 0.2
\end{solution}

\question Маша, Ваня и Петя играют в игру: ставят на бумаге точку наугад в любом месте на отрезке от 0 до 100. Маша сделала ход, поставила точку. Оказалось, что она выбрала число 40. Найдите вероятность того, что после ходов Вани и Пети, сумма их чисел будет не более 60.

\begin{solution}
 0.02
\end{solution}

\question Какова вероятность того, что разница между корнями уравнения $ a^{2}\cdot x^{2}=b^{2} $ (a и b параметры) не более 4-х, a $\in [1,8] $, а b $\in [0,20] $?

\begin{solution}
 9/20
\end{solution}

\question Функция распределения случайной величины Х имеет следующий вид:
\[
F(x)=\begin{cases}
0, \; x<0 \\
1/4, \; x \in [0,1)\\
3/4, \; x \in [1,2)\\
1, \; x \geq 2\\
\end{cases}
\]

Найдите $\E(X)$

\begin{solution}
$\E[X]=1$
\end{solution}

\question Найдите вероятность того, что величина $X$ попадёт на отрезок $[0;4]$, если она равномерно распределена на отрезке $[2;18]$.


\begin{solution}
 $2/16=1/8=0.125$
\end{solution}

\question Функция плотности случайной величины $X$ имеет вид
\[
f(x)=\begin{cases}
0, \; x < 0 \\
\cos(x), \; x \in [0;\pi/2] \\
0, \; x > \pi/2
\end{cases}.
\]
Найдите $F(x)$.



\begin{solution}
$
F(x)=\begin{cases}
0, \; x\leq 0 \\
\sin(x), \; x \in (0;\pi/2] \\
1, \; x > \pi/2
\end{cases}.
$
\end{solution}

\question Допустим, в результате переписи обнаружено, что темноволосые матери с темноволосыми дочерьми составили 0.06 обследованных семей, темноволосые матери и светловолосые дочери --- 0.1, светловолосые матери и темноволосые дочери --- 0.14, а светловолосые матери и светловолосые дочери --- 0.7. Найдите условную вероятность того, что дочь имеет темные волосы, если мать темноволосая.

\begin{solution}
 $0.06/(0.06+0.1)=0.375$
\end{solution}

\question В первой урне 5 шаров, из них 3 белых; во второй урне 6 шаров, из них 3 белых; в третьей урне 30 шаров, среди них белых шаров 12 штук. Из каждой урны извлекли по шару, а затем из них наудачу взяли один. Найдите вероятность, что этот шар белый.


\begin{solution}
 $0.5$
\end{solution}

\question Числа $1, 2, \ldots , n$ расставлены случайным образом. Предполагая, что различные расположения чисел равновероятны, Найдите вероятность того, что числа 1, 2, 3 расположены в порядке
возрастания, но не обязательно рядом.


\begin{solution}
 $\frac{1}{6}$
\end{solution}

\question Стрелок A попадает в мишень с вероятностью 0.6, стрелок B --- с вероятностью 0.5, стрелок C --- с вероятностью 0.4. Стрелки дали залп по мишени. Какова вероятность, что ровно две пули попали в цель?

\begin{solution}
 $0.6\cdot0.5\cdot0.4 + 0.6\cdot0.5\cdot0.4+0.4\cdot0.5\cdot0.4=0.38$
\end{solution}

\question Математическое ожидание случайной величины $X$ равно 2. Найдите математическое ожидание величины $5X + 2$.

\begin{solution}
 $12$
\end{solution}



\question Рассеянная Маша любит кошек и всегда оставляет дверь открытой. Все Машины кошки придерживаются принципа: увидев утром открытую дверь, выбегать из дома с вероятностью $0.8$. Посчитать вероятность того, что в очередной день, когда Маша вернется домой, она найдет там хотя бы одну кошку, если до её ухода их было пять.


\begin{solution}
 $0.87$
\end{solution}



\question Если в семье двое детей, то какова вероятность того, что оба они мальчики, если известно, что хотя бы один из них мальчик? (Рождение девочки и мальчика считать равновероятными событиями.)


\begin{solution}
 $\frac{1}{3}$
\end{solution}

\question Вероятности для  случайной величины $X$ заданы функцией: $\P(X=x) = \frac{x}{c}$, и $X \in \{1,2,3,4\}$. Чему равна величина $c$?

\begin{solution}
 $10$
\end{solution}


\question Случайная величина $X$ распределена по Пуассону, причем $3P(X=1) = P(X=2)$.

Найдите $P(X=4)$.

\begin{solution}
 $e^{-6}\cdot\dfrac{6^4}{4!}=0.13$
\end{solution}


\question Маша и Ваня играют в такую игру: они по очереди подбрасывают монетку до тех пор, пока на ней не выпадет орел. Победителем объявляется тот игрок, на чей ход это произошло. Какова вероятность, что игру выиграет Маша, если первым бросает Ваня?


\begin{solution}
 $\frac{1}{3}$
\end{solution}

\question Вероятность того, что Петр Сергеевич поймает хотя бы одного окуня, забросив удочку 4 раза, равна 0.9984. Найдите, с какой вероятностью он поймает его с первого раза.

\begin{solution}
 $0.8$
\end{solution}

\question Пират Семен закопал три сундука с золотом на трех разных островах одного архипелага. Какова вероятность того, что кладоискатель Борис найдет два из них на первых же двух островах, которые он посетит случайным образом, если всего в архипелаге 18 островов.

\begin{solution}
 $1/51$
\end{solution}

\question Функция плотности случайной величины X имеет вид:
\[
f(x)=
\begin{cases}
0,\quad \quad \quad \quad  x \leqslant -4\\
a\sqrt{x+4},\quad x \in (-4,5]\\
0,\quad \quad \quad \quad x>5
\end{cases}
\]
Найдите значение параметра $a$.

\begin{solution}
 $ 1/18$
\end{solution}


\question В коробке номер один лежат 6 белых и 6 черных шариков. Наугад достается 2 шарика и перекладывается в коробку номер два, где изначально тоже находились 6 белых и 6 черных. Затем из коробки номер два наугад достается один шарик. Какова вероятность, что переложили два черных шарика, если в итоге вытащили белый?

\begin{solution}
 $\frac{15}{77}$
\end{solution}



\question На первом курсе учится очень красивый молодой человек по имени Архип. Люди в космос летают, придумывают умные часы, а вот он Архип. А лекция по риторике безусловно очень полезна, но невероятно скучна, поэтому вместо того чтобы слушать лектора, девушки глазеют по сторонам. Количество девушек, которые влюбляются в Архипа за фиксированный промежуток времени времени --- случайная величина, имеющая распределение Пуассона. В среднем, за пару по риторике, в Архипа влюбляется 8 девушек. Какова вероятность, что за пару в Архипа влюбятся по крайней мере две девушки?

\begin{solution}
 $1 - 9e^{-8}$
\end{solution}

\question Найдите функцию распределения случайной величины $X$, если она равномерно распределена на отрезке $[-2;11]$.

\begin{solution}
\[
F_X(x)=\begin{cases}
0, \; x< -2 \\
\dfrac{x+2}{13}, \; x \in [-2;11) \\
1, \; t \geq 11
\end{cases}
\]
\end{solution}

\end{questions}

\newpage

\vspace{0.2in}

\makebox[\textwidth]{Команда:\enspace\hrulefill}


\begin{center}
ТУР ТРЕТИЙ
\end{center}

\begin{questions}


\question На склад было доставлено 30 изделие от первого завода и 70 от второго. Процент бракованных изделий 1 завода равен $4$, а второго --- $5$. Найдите вероятность того, что взятое наугад изделие будет бракованным.

\begin{solution}
 $0.3\cdot0.04+0.7\cdot0.05 = 0.047$
\end{solution}


\question Два друга договорились встретиться в определённом месте между 7 и 8 часами утра. Пришедший первым ждёт второго в течение 10 минут, после чего уходит. Какова вероятность того, что они встретятся, если каждый из друзей может равновероятно прийти в любой момент времени в указанном промежутке?

\begin{solution}
 11/36
\end{solution}

\question Функция плотности случайной величины $X$ имеет вид
\[
f(t)=\begin{cases}
0, \; t< 1 \\
t-a, \; t \in [1;2) \\
0, \; t \geq 2
\end{cases}.
\]
Найдите $a$, $\E(X)$.

\begin{solution}
 $a=\dfrac{1}{2}, \E(X)=\dfrac{19}{12}.$
\end{solution}

\question Случайные величины $X_i, i=1, \ldots, 100500$, распределены равномерно на отрезке $[0;\pi]$. Определите следующую вероятность $P(\min\{X_1,\ldots, X_{100500}\}>e)$

\begin{solution} $\left(\dfrac{\pi-e}{\pi}\right)^{100500}$
\end{solution}

\question
Совместное распределение случайных величина $X$ и $Y$ задано следующим образом:\\
 \begin{tabular}{c|c|c|c|}
 \centering
  & $X=0$ & $X=1$  & $X=2$  \\
\hline $Y=0$ & 0.1 & 0.1 & 0.1 \\
\hline  $Y=100$&  0.2& 0.2 & 0.3 \\
\hline
\end{tabular} \\
Определить $\E(X)$ 

\begin{solution}
 $\E(X)=1.1$
\end{solution}



\question Функция плотности случайной величины имеет следующий вид: $f(x)=\dfrac{1}{\pi(1+x^2)}$.\\ Найдите $\P(X\in[-1;1])$


\begin{solution}
 $\P(X\in[-1;1])=\dfrac12$
\end{solution}

\question Cаша очень любит яблоки, особенно красные, а Маша очень любит приходить к ней в гости и есть ее яблоки. Сегодня в холодильнике у Cаши лежат $3$ красных яблока и $3$ зеленых. С какой вероятностью Саша, знакомая с правилами этикета и основами теории вероятности, вытащит свое любимое яблоко из холодильника не глядя, если она уже вытащила одно, отдала его своей гостье, и оно оказалось зеленым.

\begin{solution}
 $\dfrac35$
\end{solution}

\question Какова вероятность обладать студенческим билетом с 8 не повторяющимися цифрами, если все цифры равновероятно и независимо стоят на 8 позициях?

\begin{solution}
 $\cfrac{10!}{2! \cdot 10^8}$
\end{solution}

\question В турнире по шашкам принимали участие 12 игроков в шашки. При этом каждый из них сыграл одну партию с каждым из остальных. Сколько всего партий было сыграно?

\begin{solution}
 $66$
\end{solution}



\question Иван круглый отличник и его оценка за теорию вероятностей равномерна распределена на отрезке от $8$ до $10$. Каково математическое ожидание его оценки?

\begin{solution}
 $9$ 
\end{solution}

\question Функция распределения случайной величины $X$ имеет вид:
\[
F(x)=\begin{cases}
C_1, \; x< 0 \\
0.5 x^2+ 0.5x+ C_3, \; x \in [0;1] \\
C_2, \; x \geq 1
\end{cases}.
\]
Найдите  $C_3$.

\begin{solution}
 $C_3=0$
\end{solution}


\question Ассистенту Виталию влом проверять контрольные работы,  поэтому он ставит оценки по шкале от 6 до 10 наугад. Какова вероятность что ровно половина группы из 30 человек получит отличную оценку?


\begin{solution}
 $C_{30}^{15}\left(\frac{3}{5}\right)^{15}\left(\frac{2}{5}\right)^{15}$
\end{solution}

\question У Никиты дома есть ковер размером 4 на 4 метра, который очень дорог его папе. Пока родители были на даче, Никита решил устроить тусовку, на которую его друзья принесли кальян. 15 раз случайно упавшие из кальяна угли прожгли ковер в разных местах. Какова вероятность, что по приезде папа застанет целым кусочек ковра размером хотя бы метр на метр?

\begin{solution}
 1
\end{solution}


\question Найдите матожидания числа очков, которые выпадут на 2 одновременно брошенных кубиках, если известно, что выпали разные грани.

\begin{solution}
 $7$
\end{solution}

\question Случайная точка бросается в круг, определить вероятность попадания этой точки в квадрат, вписанный в круг.

\begin{solution}
 $\dfrac{2}{\pi}$
\end{solution}
\question Двоечница Рита очень не любит ходить на лекции и появляется там с вероятностью $\dfrac13$. Найдите наивероятнейшее число лекций, на которых Рита была, если всего было семь лекций.
\begin{solution}
$2$
\end{solution}

\end{questions}




\end{document}
