\documentclass[10pt,a4paper]{article}
\usepackage[utf8]{inputenc}
\usepackage[russian]{babel}

\usepackage{enumitem}
\makeatletter
\AddEnumerateCounter{\asbuk}{\russian@alph}{щ}
\makeatother




\usepackage{amsmath}
\usepackage{amsfonts}
\usepackage{amssymb}
\usepackage{fancyhdr}
\usepackage{booktabs}
\usepackage[left=2cm,right=2cm,top=2cm,bottom=2cm]{geometry}

\DeclareMathOperator*\plim{plim}
\newcommand{\E}{\mathbb{E}}
\renewcommand{\P}{\mathbb{P}}
\newcommand{\cN}{\mathbb{N}}
\newcommand{\Corr}{\mathbb{C}\mathrm{orr}}
\newcommand{\Cov}{\mathbb{C}\mathrm{ov}}
\newcommand{\Var}{\mathbb{V}\mathrm{ar}}

\begin{document}

\setlist[enumerate,2]{label=\asbuk*),ref=\asbuk*}

\pagestyle{fancy}

\fancyhf{}
\fancyhead[L]{ВШЭ, Теория вероятностей}
\fancyhead[C]{Контрольная 2, 12.12.15}
\fancyhead[R]{Поток Риччи}


\begin{enumerate}
\item Функция плотности случайного вектора $\xi=(\xi_1, \xi_2)^T$ имеет вид
\[
f(x,y)=\begin{cases}
0.5x + 1.5y, \text{ если } 0<x<1, \; 0<y<1 \\
0, \text{ иначе }
\end{cases}
\]
Найдите:
\begin{enumerate}
\item Математическое ожидание $\E(\xi_1 \cdot \xi_2)$
\item Условную плотность распределения $f_{\xi_1|\xi_2} (x|y)$
\item Условное математическое ожидание $\E(\xi_1| \xi_2=y)$
\item Константу $k$, такую, что функция $h(x,y)=kx\cdot f(x,y)$ будет являться совместной функцией плотности некоторой пары случайных величин
\end{enumerate}

\item На курсе учится очень много студентов. Вероятность того, что случайно выбранный студент получит «отлично» за контрольную равна $0.2$, «хорошо» "--- $0.3$. Вероятности остальных результатов неизвестны. Пусть $\xi$ и $\eta$ "--- число отличников и хорошистов в случайной группе из $10$ студентов. Найдите $\Cov(\xi,\eta)$, $\Corr(\xi,\eta)$, $\Cov(\xi-\eta,\xi)$. Являются ли случайные величины $\xi-\eta$ и $\xi$ независимыми?

\item Доходности акций компаний А и В "--- случайные величины $\xi$ и $\eta$. Известно, что $\E(\xi)=1$, $E(\eta)=1$, $\Var(\xi)=4$, $\Var(\eta)=9$, $\Corr(\xi,\eta)=-0.5$. Петя принимает решение потратить свой рубль на акции компании А, Вася "--- 50 копеек на акции компании А и 50 копеек на акции компании В, а Маша  принимает решение вложить свой рубль в портфель $R=\alpha\xi+(1-\alpha)\eta$, $(0 \leq \alpha \leq 1)$, обладающий минимальным риском. Найдите $\alpha$, ожидаемые доходности и риски портфелей Пети, Васи и Маши.

\item Будем считать, что рождение мальчика и девочки равновероятны.
\begin{enumerate}
\item С помощью закона больших чисел определите в каком городе, большом или маленьком, больше случается таких дней, когда рождается более 75\% мальчиков.
\item Оцените с помощью неравенства Маркова вероятность того, что среди тысячи новорожденных младенцев мальчиков будет более 75\%.
\item Оцените с помощью неравенства Чебышёва вероятность того, что доля мальчиков среди тысячи новорожденных младенцев будет отличаться от 0.5 более, чем на 0.25
\item С помощью теоремы Муавра-Лапласа вычислите вероятность из предыдущего пункта.
\end{enumerate}

\item Сейчас валютный курс племени «Мумба» составляет 100 оболов за один рубль. Процентное изменение курса за один день "--- случайная величина $\delta_i$ с законом распределения:

\begin{center}
\begin{tabular}{lrr}
\toprule
$\delta_i$ & $-1\%$  & $1\%$ \\
$\P(\cdot)$ & $0.25$  & $0.75$ \\
\bottomrule
\end{tabular}
\end{center}

Найдите вероятность того, что через полгода (171 день) рубль будет стоить более 271 обола, если ежедневные изменения курса происходят независимо друг от друга.

\item Величины $X_1$, $X_2$, \ldots независимы и равновероятно принимают значения $-1$ и $3$.
\begin{enumerate}
\item Найдите $\plim_{n\to\infty} \frac{\sum_{i=1}^n(X_i-\bar X)^2}{n}$
\item С помощью дельта-метода найдите примерный закон распределения $\frac{\sum_{i=1}^{100}(X_i-\bar X)^2}{100}$
\end{enumerate}

\end{enumerate}




\end{document}
