\documentclass[10pt,a4paper]{article}
\usepackage[utf8]{inputenc}
\usepackage[russian]{babel}

\usepackage{enumitem}
\makeatletter
\AddEnumerateCounter{\asbuk}{\russian@alph}{щ}
\makeatother




\usepackage{amsmath}
\usepackage{amsfonts}
\usepackage{amssymb}
\usepackage{fancyhdr}
\usepackage{booktabs}
\usepackage[left=2cm,right=2cm,top=2cm,bottom=2cm]{geometry}

\DeclareMathOperator*\plim{plim}
\newcommand{\E}{\mathbb{E}}
\renewcommand{\P}{\mathbb{P}}
\newcommand{\cN}{\mathbb{N}}
\newcommand{\Corr}{\mathbb{C}\mathrm{orr}}
\newcommand{\Cov}{\mathbb{C}\mathrm{ov}}
\newcommand{\Var}{\mathbb{V}\mathrm{ar}}

\begin{document}

\setlist[enumerate,2]{label=\asbuk*),ref=\asbuk*}

\pagestyle{fancy}

\fancyhf{}
\fancyhead[L]{ВШЭ, Теория вероятностей}
\fancyhead[C]{Контрольная 2, 12.12.15}
\fancyhead[R]{Поток Риччи}


\begin{enumerate}
\item Функция .............. ................................ имеет вид .............. ..............

...................................... .............................. ................................... .................................

................................................. .............................................  .............................. ............

Найдите:
\begin{enumerate}
\item Математическое ожидание ............. ........................  ........................
\item Условную .....................................  ........................
\item .......  математическое ожидание $\E(.....)$  ........................
\item ......., такую, что ......................... ........................ ......... ..................................
\end{enumerate}

\item На курсе учится очень много студентов. Вероятность того, что случайно   ........................

....................................................................  ........................ ........................

...............в случайной группе ..................................  ........................ ........................

....................................................................  ........................ ........................

Найдите $\Cov(\xi,\eta)$, $\Corr(\xi,\eta)$, ........................  ........................ ........................

.............. случайные величины .............. ........................

\item ....................... "--- случайные величины $\xi$ и $\eta$. Известно, что

$\E(...)=...$, ........................................................ Петя ........................ ........................

.............. ....... ......................... А, Вася ......................., а Маша ........................ ........................

.................................................................... ............... ........................ ........................

............................................ Найдите ............................ ........................ ........................

....................


\item Будем считать, что рождение мальчика и девочки равновероятны.
\begin{enumerate}
\item .......................................................... ........................ ........................
 .......... ............... больше .......... .............. ..... ..... мальчиков ........................
\item Оцените с помощью ............ ........... ............. ................. ........................ ..................... ...........
\item ................ .............. .......................... доля ....... ......... ......... младенцев .......... .......... ..........  0.25 ........................
\item ........... теоремы ......... ........ ......... .......... ........................ ........................
\end{enumerate}

\item .......... .............. ...... составляет ........ ......... ...... ......... ........................

.........  "--- случайная величина $\delta_i$ ......... ...... ........................ ........................

.................................................................... ............... ........................

.................................................................... ............... ........................


Найдите вероятность ....................................   ........ ........................ ........................

........................ ............... ........................ ........................ ........................

..... если ....... ........ .......... ....... ......... ........

\item Величины $X_1$, $X_2$, \ldots независимы и ........... ........ ...... ........................
\begin{enumerate}
\item Найдите $\plim_{n\to\infty} .........................$ 
\item С помощью дельта-метода найдите .......... ......... ......... ......... .............. ........... ........................ ........................
\end{enumerate}

\end{enumerate}




\end{document}
