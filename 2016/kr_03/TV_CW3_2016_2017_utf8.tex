\documentclass[a4paper, 12pt]{article}
\usepackage[T1,T2A]{fontenc}
\usepackage[cp1251]{inputenc}
\usepackage[russian]{babel}
\usepackage{amsmath}
\usepackage{mathrsfs}
\usepackage{embedfile} % Чтобы код LaTeXа включился как приложение в PDF-файл
\usepackage{amsfonts}
\usepackage{amscd}
\usepackage[paper=a4paper,top=20.0mm, bottom=20.0mm,left=20.0mm,right=20.0mm,includefoot]{geometry} % размер листа бумаги
\embedfile[desc={Main tex file}]{\jobname.tex} % Прикрепить код LaTeXа в PDF-файл
\usepackage[matrix,arrow,curve]{xy}
\usepackage{soul}
\usepackage{color}
\usepackage{graphics}
\usepackage{multicol}
\usepackage{graphicx}

\begin{document}
\begin{center}
    \textbf{\textbf{\large Контрольная работа №\,3 по ТВ\,и\,МС [2016--2017]}}
\end{center}


\textbf{Ф.\,И.\,О.:}

\textbf{Группа:}

\medskip
\textbf{Задача 1.} Дана реализация случайной выборки: $1$, $10$, $7$, $4$, $-2$. Выпишите определения и найдите значения следующих характеристик:
\begin{itemize}
  \item[а)] вариационного ряда,
  \item[б)] выборочного среднего,
  \item[в)] выборочной дисперсии,
  \item[г)] несмещенной оценки дисперсии,
  \item[д)] выборочного второго начального момента.
  \item[е)] Постройте выборочную функцию распределения.
\end{itemize}


\medskip

\textbf{Задача 2.}
Мама дяди Фёдора каждое лето приезжает в Простоквашино с тремя вечерними платьями. Средняя стоимость и дисперсия цены случайно выбранного платья (из трех) составляет 11 тысяч и 3 тысячи рублей соответственно. Рачительный кот Матроскин случайным образом выбирает одно из платьев и продаёт его как ненужное. Вычислите математическое ожидание и дисперсию стоимости двух оставшихся платьев.

\medskip

\textbf{Задача 3.}
Ресторанный критик ходит по трём типам ресторанов (дешевых, бюджетных и дорогих) города N для того, чтобы оценить среднюю стоимость бизнес-ланча. В городе 40\%
дешевых ресторанов, 50\% — бюджетных и 10\% — дорогих. Стандартное отклонение цены бизнес-ланча составляет $10$, $30$ и $60$ рублей соответственно. В ресторане критик заказывает только кофе. Стоимость кофе в дешевых/бюджетных/дорогих ресторанах составляет 150, 300 и 600 рублей соответственно, а бюджет исследования --- 30\,000 рублей.
\begin{itemize}
  \item[а)] Какое количество ресторанов каждого типа нужно посетить критику, чтобы как можно точнее оценить среднюю стоимость бизнес-ланча при заданном бюджетном ограничении (округлите полученные значения до ближайших целых)?
  \item[б)] Вычислите дисперсию соответствующего стратифицированного среднего.
\end{itemize}

\medskip

\textbf{Задача 4.}
В <<акции протеста против коррупции>> в Москве 26.03.2017 по данным МВД приняло участие 8\,000 человек. Считая, что население Москвы составляет 12\,300\,000 человек, постройте 95\% доверительный интервал для истинной доли желающих участвовать в подобных акциях жителей Москвы. Можно ли утверждать, что эта доля статистически не отличается от нуля?

\medskip

\textbf{Задача 5.}
Для некоторой отрасли проведено исследование об оплате труда мужчин и женщин. Их зарплаты (тыс. руб. в месяц) приведены ниже:
\[
\begin{tabular}{c|c|c|c|c|c}
  \text{мужчины}         &$50$    &$40$    &$45$   &$45$   &$35$   \\ \cline{1-6}
  \text{женщины}         &$60$    &$30$    &$30$   &$35$   &$30$   \\
\end{tabular}
\]
\begin{itemize}
  \item[а)] Считая, что распределение заработных плат мужчин хорошо описывается нормальным распределением, постройте
  \begin{itemize}
    \item 99\%-ый доверительный интервал для математического ожидания заработной платы мужчин,
    \item 90\%-ый доверительный интервал для стандартного отклонения заработной платы мужчин.
  \end{itemize}
  \item[б)]
  \begin{itemize}
    \item Сформулируйте предпосылки, необходимые для построения доверительно интервала для разности математических ожиданий заработных плат мужчин и женщин.
    \item Считая предпосылки выполненными, постройте 90\%-ый доверительный интервал для разности математических ожиданий заработных плат мужчин и женщин.
    \item Можно ли считать зарплаты мужчин и женщин одинаковыми?
  \end{itemize}
\end{itemize}

\medskip

\textbf{Задача 6.}
Пусть $X = (X_1, \, \ldots, \, X_n)$ --- случайная выборка из нормального распределения с нулевым математическим ожиданием и дисперсией $\theta$.
\begin{itemize}
  \item[а)] Используя второй начальный момент, найдите оценку параметра $\theta$ методом моментов.
  \item[б)] Сформулируйте определение несмещённости оценки и проверьте выполнение данного свойства для оценки, найденной в пункте а).
  \item[в)] Сформулируйте определение состоятельности оценки и проверьте выполнение данного свойства для оценки, найденной в пункте а).
  \item[г)] Найдите оценку параметра $\theta$ методом максимального правдоподобия.
  \item[д)] Вычислите информацию Фишера о параметре $\theta$, заключенную в $n$ наблюдениях случайной выборки.
  \item[е)] Сформулируйте неравенство Рао--Крамера--Фреше.
  \item[ё)] Сформулируйте определение эффективности оценки и проверьте выполнение данного свойства для оценки, найденной в пункте г).
\end{itemize}


\medskip

\textbf{Задача 7.}
Аэрофлот утверждает, что 10\% пассажиров, купивших билет, не являются на рейс. В случайной выборке из шести рейсов аэробуса А320, имеющего 180 посадочных мест, число не явившихся оказалось: $5$, $10$, $25$, $0$, $17$, $30$. Пусть число пассажиров $X$, не явившихся на рейс, хорошо описывается распределением Пуассона $\mathbb{P}(\{X = k\}) = \tfrac{\lambda^{k}}{k!}e^{-\lambda}$, $k \in \{0,\, 1,\, 2,\, \ldots\}$. При помощи метода максимального правдоподобия найдите:
\begin{itemize}
  \item[а)] оценку $\mathbb{E}[X]$ и её числовое значение по выборке,
  \item[б)] оценку дисперсии $X$ и её числовое значение по выборке,
  \item[в)] оценку стандартного отклонения $X$ и её числовое значение по выборке,
  \item[г)] оценку вероятности того, что на рейс явятся все пассажиры, а также найдите её числовое значение по выборке.
  \item[д)] Используя асимптотические свойства оценок максимального правдоподобия, постройте 95\% доверительный интервал для $\mathbb{E}[X]$.
  \item[е)] С помощью дельта-метода найдите 95\% доверительный интервал для вероятности полной загруженности самолёта.
\end{itemize}


\end{document} 