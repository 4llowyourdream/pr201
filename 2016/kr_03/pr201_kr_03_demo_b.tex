\documentclass[a4paper, 12pt]{article}


\usepackage{mathrsfs}

\usepackage{amscd}
\usepackage[paper=a4paper,top=20.0mm, bottom=20.0mm,left=20.0mm,right=20.0mm,includefoot]{geometry} % размер листа бумаги


\usepackage{tikz} % картинки в tikz
\usepackage{microtype} % свешивание пунктуации

\usepackage{floatrow} % для выравнивания рисунка и подписи
\usepackage{caption} % для пустых подписей

\usepackage{array} % для столбцов фиксированной ширины

\usepackage{indentfirst} % отступ в первом параграфе

\usepackage{sectsty} % для центрирования названий частей
\allsectionsfont{\centering}

\usepackage{amsmath, amsfonts} % куча стандартных математических плюшек

\usepackage{comment} % для комментариев

\usepackage{multicol} % текст в несколько колонок

\usepackage{lastpage} % чтобы узнать номер последней страницы

\usepackage{enumitem} % дополнительные плюшки для списков
%  например \begin{enumerate}[resume] позволяет продолжить нумерацию в новом списке

\usepackage{url} % для вставки интернет-ссылок

\usepackage{fontspec}
\usepackage{polyglossia}

\setmainlanguage{russian}
\setotherlanguages{english}

% download "Linux Libertine" fonts:
% http://www.linuxlibertine.org/index.php?id=91&L=1
\setmainfont{Linux Libertine O} % or Helvetica, Arial, Cambria
% why do we need \newfontfamily:
% http://tex.stackexchange.com/questions/91507/
\newfontfamily{\cyrillicfonttt}{Linux Libertine O}

\AddEnumerateCounter{\asbuk}{\russian@alph}{щ} % для списков с русскими буквами
\setlist[enumerate, 2]{label=\asbuk*),ref=\asbuk*}

\DeclareMathOperator{\Var}{Var}
\DeclareMathOperator{\E}{\mathbb{E}}

\let\P\relax
\DeclareMathOperator{\P}{\mathbb{P}}
\def\cN{\mathcal{N}}

\usepackage{fancyhdr} % весёлые колонтитулы
\pagestyle{fancy}
\lhead{Теория вероятностей, тренировочный вариант}
\chead{}
\rhead{01.04.2017}
\lfoot{}
\cfoot{}
\rfoot{\thepage/\pageref{LastPage}}
\renewcommand{\headrulewidth}{0.4pt}
\renewcommand{\footrulewidth}{0.4pt}


\begin{document}


Это тренировочный вариант для контрольной номер 3. Вариант контрольной будет отличаться от данного довольно сильно. Там будут другие задачи. Темы — те же.


\begin{enumerate}
\item  Даны значения случайной выборки $x_1=5$, $x_2=3$, $x_3=4$, $x_4=4$, $x_5=11$.
\begin{enumerate}
\item Выпишите вариационный ряд.
\item	Найдите выборочные моду, медиану, среднее.
\item Найдите несмещённую оценку дисперсии $X_i$.
\item Выпишите и нарисуйте выборочную функцию распределения.
\end{enumerate}

\item В обозримой части Вселенной водится всего пять Лиловых кальмароандроидов. Храбрый исследователь глубокого космоса Юрий поймал трёх из них. После поимки Юрий измерил их гипнопотенциал (в рунах): $x_1 = 2.5$, $x_2 = 9.5$, $x_3 = 6$.
\begin{enumerate}
\item Помогите Юрию построить несмещённую оценку для неизвестных науке $\E(X_i)$ и $\Var(X_i)$.
\item Как изменились бы ответы на предыдущий вопрос, если бы Юрий после отлова очередного кальмароандроида отпускал бы его обратно на просторы Вселенной?
\end{enumerate}

\item Исследователь Юрий начал собирать данные о довольно распространённых Саблезубых хомозоидах. Их гипнопотенциал имеет нормальное распределение. По выборке из 10 хомозоидов оказалось, что средний выборочный гипнопотенциал равен 5 рун с выборочным стандартным отклонением в 2 руны.
\begin{enumerate}
  \item Постройте 90\%-ый доверительный интервал для математического ожидания гипнопотенциала хомозоида.
  \item Постройте 90\%-ый доверительный интервал для дисперсии гипнопотенциала хомозоида.
\end{enumerate}


\item   Величины $X_1$, $X_2$, \ldots~независимы и имеют экпоненциальное распределение с параметром $\lambda$. По выборке из 100 наблюдений оказалось, что $\sum x_i = 150$ и $\sum x_i^2 = 1500$. Исследователь Афанасий хочет оценить параметр $\lambda$.
\begin{enumerate}
  \item Найдите оценку $\lambda$ методом максимального правдоподобия.
  \item Оцените теоретическую информацию Фишера $I$.
  \item Постройте 95\%-ый доверительный интервал для $\lambda$ с помощью $\lambda_{ML}$.
  \item Также исследователь Афанасий хочет оценить параметр $\theta = \P(X_i > 1)$. Найдите $\theta_{ML}$ и постройте 95\%-ый доверительный интервал для $\theta$.
\end{enumerate}


\item  Согласно опросу ВЦИОМ 2011 года\footnote{более свежий я на скорую руку не нашёл, \url{https://wciom.ru/index.php?id=236&uid=111345}.} из 1600 опрошенных 32\% согласны с утверждением «Солнце вращается вокруг Земли».

\begin{enumerate}
\item Постройте 90\%-ый доверительный интервал для доли россиян, согласных с данным утверждением.
\item В 2007 году при том же размере выборке согласных с утверждением о Солнце было 28\%. Постройте 95\%-ый доверительный интервал для изменения доли согласных с утверждением.
\item (*) Что подразумевает ВЦИОМ под фразой «Статистическая погрешность не превышает 3,4\%»?
\end{enumerate}

\item Величины $X_1$, $X_2$, \ldots~независимы и имеют биномиальное распределение $Bin(n=20, p)$.
\begin{enumerate}
\item Найдите оценку $p$ методом моментов.
\item Найдите дисперсию оценки $\hat p_{MM}$.
\item Является ли данная оценка несмещённой? состоятельной?
\item Найдите информацию Фишера для отдельного наблюдения $i(p)$.
\item Сформулируйте неравенство Рао-Крамера для данного случая.
\item Является ли оценка $\hat p_{MM}$ эффективной среди несмещённых?
\item Постройте 95\%-ый доверительный интервал для $p$.
\end{enumerate}

\item Величины $X_1$, $X_2$, \ldots~независимы и распределены нормально $\cN(0; 4)$.
\begin{enumerate}
  \item Как распределены величины $Y = (X_1^2 + X_2^2 + X_3^2)/4$, $Z = (X_1^2 + X_2^2)/(X_3^2 + X_4^2)$ и $W = X_1 / \sqrt{X_2^2 + X_3^2}$?
  \item Для каждой из величин $Y$, $Z$ и $W$ найдите с помощью таблиц такое пороговое число $a$, которое величина превышает с вероятностью $0.05$.
\end{enumerate}

\item Ресторанный критик ходит по трём типам ресторанов (дешевых, бюджетных и дорогих) города N для того, чтобы оценить среднюю стоимость бизнес-ланча. В городе 30\% дешевых ресторанов, 60\% — бюджетных и 10\% — дорогих. Стандартное отклонение цены бизнес-ланча составляет 10, 30 и 60 рублей соответственно. В ресторане критик заказывает только кофе. Стоимость кофе в дешевых/бюджетных/дорогих ресторанах составляет 150, 300 и 600 рублей соответственно, а бюджет исследования — 30 000 рублей.
\begin{enumerate}
\item Какое количество ресторанов каждого типа нужно посетить критику, чтобы как можно точнее оценить среднюю стоимость бизнес-ланча при заданном бюджетном ограничении (округлите полученные значения до ближайших целых)?
\item  Вычислите дисперсию соответствующего стратифицированного среднего.
\end{enumerate}



\item Величины $X_i$ независимы и одинаково распределены. Предлагается три оценки математического ожидания:
\[
\hat \mu_A = \frac{2X_1 + X_2 + X_3 + X_4 +\ldots + X_n}{n+1}
\]
\[
\hat \mu_B = \frac{2X_1 - X_2 + X_3 + X_4 + \ldots + X_n}{n}
\]
\[
\hat \mu_C = \frac{X_1 + 2X_2 + 3X_3 + 4X_4 + \ldots + nX_n}{n(n+1)}
\]

\begin{enumerate}
  \item Какие оценки являются несмещёнными? состоятельными?
  \item Какая из несмещённых оценок является эффективной?
\end{enumerate}




\end{enumerate}

\end{document}
