\documentclass[a4paper, 12pt]{article}
\usepackage[T1,T2A]{fontenc}
\usepackage[cp1251]{inputenc}
\usepackage[russian]{babel}
\usepackage{amsmath}
\usepackage{mathrsfs}
\usepackage{embedfile} % Чтобы код LaTeXа включился как приложение в PDF-файл
\usepackage{amsfonts}
\usepackage{amscd}
\usepackage[paper=a4paper,top=20.0mm, bottom=20.0mm,left=20.0mm,right=20.0mm,includefoot]{geometry} % размер листа бумаги
\embedfile[desc={Main tex file}]{\jobname.tex} % Прикрепить код LaTeXа в PDF-файл
\usepackage[matrix,arrow,curve]{xy}
\usepackage{soul}
\usepackage{color}
\usepackage{graphics}
\usepackage{multicol}
\usepackage{graphicx}

\begin{document}
\begin{center}
    \textbf{\textbf{\large Контрольная работа №\,4 по ТВ\,и\,МС [2016--2017]}}
\end{center}


\textbf{Ф.\,И.\,О.:}

\textbf{Группа:}

\bigskip

\textbf{I. Теоретический минимум}

\medskip

В пунктах 1, 3, 11 и 12 предполагается, что $X = (X_1, \, \ldots, \, X_n)$ и $Y = (Y_1, \, \ldots, \, Y_m)$ --- две независимые случайные выборки из нормальных распределений $N(\mu_X, \sigma_X^2)$ и $N(\mu_Y, \sigma_Y^2)$ соответственно.

\begin{itemize}
  \item[1.] Приведите формулу статистики, при помощи которой можно проверить гипотезу $H_0 \colon \sigma_X^2 = \sigma_Y^2$. Укажите распределение этой статистики при верной гипотезе $H_0$.
  \item[2.] Приведите формулу информации Фишера о параметре $\theta$, содержащейся в одном наблюдении случайной выборки.
  \item[3.] Приведите формулу статистики, при помощи которой можно проверить гипотезу $H_0 \colon \mu_X - \mu_Y = \Delta_0$ при условии, что дисперсии $\sigma_X^2$ и $\sigma_Y^2$ неизвестны, но равны между собой. Укажите распределение этой статистики при верной гипотезе $H_0$.
  \item[4.] Дайте определение критической области.
  \item[5.] Приведите формулу плотности нормального распределения $N(\mu, \sigma^2)$.
  \item[6.] Приведите формулы границ доверительного интервала с уровнем доверия $(1 - \alpha)$, $\alpha \in (0;\,1)$, для вероятности появления успеха в случайной выборке $X = (X_1, \, \ldots, \, X_n)$ из распределения Бернулли с параметром $p \in (0;\,1)$.
  \item[7.] Дайте определение несмещенной оценки $\widehat{\theta}$ для неизвестного параметра $\theta \in \Theta$.
  \item[8.] Дайте определение эффективной оценки $\widehat{\theta}$ для неизвестного параметра $\theta \in \Theta$.
  \item[9.] Приведите формулу выборочной дисперсии.
  \item[10.] Приведите формулу выборочной функции распределения.
  \item[11.] Приведите формулы границ доверительного интервала с уровнем доверия $(1 - \alpha)$, $\alpha \in (0;\,1)$, для $\mu_X$ при условии, что дисперсия $\sigma_X^2$ известна.
  \item[12.] Укажите распределение статистики $\frac{\overline{X} - \mu_X}{\sigma / \sqrt{n}}$.
\end{itemize}
\bigskip

\textbf{II. Задачи}

\medskip

\textbf{Задача 1.} В ходе анкетирования ста сотрудников банка <<Альфа>> были получены ответы на вопрос о том, сколько времени они проводят на работе ежедневно. Среднее выборочное оказалось равным 9.5 часам, а выборочное стандартное отклонение 0.5 часа.
\begin{itemize}
  \item[(a)] На уровне значимости 5\,\% проверьте гипотезу о том, что сотрудники банка <<Альфа>> в среднем проводят на работе 10 часов, против альтернативной гипотезы о том, что сотрудники банка <<Альфа>> в среднем проводят на работе менее 10 часов.
  \item[(b)] Найдите точное $P$-значение для наблюдаемой статистики из пункта (a).
  \item[(c)] Сформулируйте предпосылки, которые были использованы вами для выполнения пункта (a).
  \item[(d)] На уровне значимости 5\,\% проверьте гипотезу о $H_0 \colon \sigma^2 = 0.3$.
\end{itemize}


\medskip

\textbf{Задача 2.}
Проверка сорока случайно выбранных лекций показала, что студент Халявин присутствовал только на 16 из них. На уровне значимости 5\,\% проверьте гипотезу о том, что Халявин посещает в среднем половину лекций.

\medskip

\textbf{Задача 3.}
В ходе анкетирования двадцати сотрудников банка <<Альфа>> были получены ответы на вопрос о том, сколько времени они проводят на работе ежедневно. Среднее выборочное оказалось равным 9.5 часам, а выборочное стандартное отклонение 0.5 часа. Аналогичные показатели для сотрудников банка <<Бета>> составили 9.8 и 0.6 часа соответственно.
\begin{itemize}
  \item[(a)] На уровне значимости 5\,\% проверьте гипотезу о равенстве математических ожиданий времени, проводимого на работе сотрудниками банков <<Альфа>> и <<Бета>>.
  \item[(b)] Сформулируйте предпосылки, которые были использованы вами для выполнения пункта (a).
  \item[(с)] На уровне значимости 5\,\% проверьте гипотезу о равенстве дисперсий времени, проводимого на работе сотрудниками банков <<Альфа>> и <<Бета>>.
\end{itemize}

\medskip

\textbf{Задача 4.}
Вася решил проверить известное утверждение о том, что бутерброд падает маслом вниз. Для этого он провел серию из 200 испытаний. Ниже приведена таблица с результатами:
\[
\begin{tabular}{c|c|c}
  \text{Бутерброд}                &\text{Маслом вниз}    &\text{Маслом вверх}       \\ \cline{1-3}
  \text{Число наблюдений}         &$105$    &$95$       \\
\end{tabular}
\]
Можно ли утверждать, что бутерброд падает маслом вниз так же часто, как и маслом вверх. При ответе на вопрос используйте уровень значимости 5\,\%.

\medskip

\textbf{Задача 5.}
Пусть $X = (X_1, \, \ldots, \, X_{100})$ --- случайная выборка из нормального распределения с математическим ожиданием $\mu$ и дисперсией $\nu$. Оба параметра $\mu$ и $\nu$ неизвестны. Используя следующие данные $\sum_{i=1}^{100}x_i = 30$, $\sum_{i=1}^{100}x_i^2 = 146$ и $\sum_{i=1}^{100}x_i^3 = 122$ с помощью теста отношения правдоподобия проверьте гипотезу $H_0 \colon \nu = 1$ на уровне значимости 5\,\%.




\end{document} 