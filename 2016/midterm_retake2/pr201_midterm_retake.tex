\documentclass{article}

\usepackage[top=1cm, bottom=1cm, left=1cm, right=1cm]{geometry} % размер текста на странице

\usepackage[box, % запрет на перенос вопросов
%nopage,
insidebox, % ставим буквы в квадратики
separateanswersheet, % добавляем бланк ответов
nowatermark, % отсутствие надписи "Черновик"
 indivanswers,  % показываем верные ответы
% answers,
lang=RU,
completemulti]{automultiplechoice}


\usepackage{tikz} % картинки в tikz
\usepackage{microtype} % свешивание пунктуации

\usepackage{multicol} % текст в несколько колонок

\usepackage{array} % для столбцов фиксированной ширины

\usepackage{indentfirst} % отступ в первом параграфе

\usepackage{sectsty} % для центрирования названий частей
\allsectionsfont{\centering}

\usepackage{amsmath} % куча стандартных математических плюшек
\usepackage{amsfonts} % шрифты, в чатности \mathbb

\usepackage{lastpage} % чтобы узнать номер последней страницы

\usepackage{enumitem} % дополнительные плюшки для списков
%  например \begin{enumerate}[resume] позволяет продолжить нумерацию в новом списке






\usepackage{todonotes} % для вставки в документ заметок о том, что осталось сделать
% \todo{Здесь надо коэффициенты исправить}
% \missingfigure{Здесь будет Последний день Помпеи}
% \listoftodos --- печатает все поставленные \todo'шки


% более красивые таблицы
\usepackage{booktabs}
% заповеди из докупентации:
% 1. Не используйте вертикальные линни
% 2. Не используйте двойные линии
% 3. Единицы измерения - в шапку таблицы
% 4. Не сокращайте .1 вместо 0.1
% 5. Повторяющееся значение повторяйте, а не говорите "то же"



\usepackage{fontspec}
\usepackage{polyglossia}

\setmainlanguage{russian}
\setotherlanguages{english}

% download "Linux Libertine" fonts:
% http://www.linuxlibertine.org/index.php?id=91&L=1
\setmainfont{Linux Libertine O} % or Helvetica, Arial, Cambria
% why do we need \newfontfamily:
% http://tex.stackexchange.com/questions/91507/
\newfontfamily{\cyrillicfonttt}{Linux Libertine O}

%% эконометрические сокращения
\DeclareMathOperator{\plim}{plim}
\DeclareMathOperator{\Cov}{Cov}
\DeclareMathOperator{\Corr}{Corr}
\DeclareMathOperator{\Var}{Var}
\DeclareMathOperator{\E}{E}
\def \hb{\hat{\beta}}
\def \hs{\hat{\sigma}}
\def \htheta{\hat{\theta}}
\def \s{\sigma}
\def \hy{\hat{y}}
\def \hY{\hat{Y}}
\def \v1{\vec{1}}
\def \e{\varepsilon}
\def \he{\hat{\e}}
\def \z{z}
\def \hVar{\widehat{\Var}}
\def \hCorr{\widehat{\Corr}}
\def \hCov{\widehat{\Cov}}
\def \cN{\mathcal{N}}

\renewcommand{\P}{\mathbb{P}}


\AddEnumerateCounter{\asbuk}{\russian@alph}{щ} % для списков с русскими буквами
\setlist[enumerate, 2]{label=\asbuk*),ref=\asbuk*}



\begin{document}



\element{midterm_retake_2016}{ % в фигурных скобках название группы вопросов
 \AMCcompleteMulti
  \begin{questionmult}{1} % тип вопроса (questionmult — множественный выбор) и в фигурных — номер вопроса
  Пусть случайная величина $\xi \sim U[0;\,1]$. Вероятность $\P(0.5<\xi<0.9)$ равна
 \begin{multicols}{3} % располагаем ответы в 3 колонки
   \begin{choices} % опция [o] не рандомизирует порядок ответов
      \correctchoice{$0.4$}
      \wrongchoice{$\int_{0.5}^{0.9}\frac{1}{\sqrt{2\pi}}\,e^{-t^2/2}\,dt$}
      \wrongchoice{$\int_{0}^{1}\frac{1}{\sqrt{2\pi}}\,e^{-t^2/2}\,dt$}
      \wrongchoice{$1/4$}
      \wrongchoice{$0.5$}
      \end{choices}
  \end{multicols}
  \end{questionmult}
}

\element{midterm_retake_2016}{ % в фигурных скобках название группы вопросов
 \AMCcompleteMulti
  \begin{questionmult}{2} % тип вопроса (questionmult — множественный выбор) и в фигурных — номер вопроса
 Случайные величины $\xi_1, \, \ldots, \, \xi_n, \, \ldots$ независимы и имеют таблицы распределения
\[
\begin{tabular}{c|c|c|c}
  $\xi_i$                     & $-1$    &  $0$     & $1$   \\ \cline{1-4}
  $\mathbb{P}_{\xi_i}$        & $1/4$   &  $1/2$    & $1/4$   \\
\end{tabular}
\]
Рассмотрим их сумму $S_n = \xi_1 + \ldots + \xi_n$. Предел $\lim\limits_{n \rightarrow \infty}\P\Bigl(\frac{S_n - \E[S_n]}{\sqrt{\Var(S_n)}} \in [-1;\,1]\Bigr)$, равен
 \begin{multicols}{3} % располагаем ответы в 3 колонки
   \begin{choices} % опция [o] не рандомизирует порядок ответов
      \correctchoice{$\int_{-1}^{1}\frac{1}{\sqrt{2\pi}}\,e^{-t^2/2}\,dt$}
      \wrongchoice{$0.5$}
      \wrongchoice{$\int_{0}^{1}\frac{1}{\sqrt{2\pi}}\,e^{-t^2/2}\,dt$}
      \wrongchoice{$\int_{-\infty}^{1}\frac{1}{\sqrt{2\pi}}\,e^{-t^2/2}\,dt$}
      \wrongchoice{$\int_{-1}^{1}\frac{1}{2}\,e^{-t/2}\,dt$}
      \end{choices}
  \end{multicols}
  \end{questionmult}
}


\element{midterm_retake_2016}{ % в фигурных скобках название группы вопросов
 \AMCcompleteMulti
  \begin{questionmult}{3} % тип вопроса (questionmult — множественный выбор) и в фигурных — номер вопроса
  Вероятность поражения мишени при одном выстреле равна $0.7$. Вероятность того, что при $200$ выстрелах мишень будет поражена от $130$  до $150$ раз, подсчитанная с помощью центральной предельной теоремы, равна
 \begin{multicols}{3} % располагаем ответы в 3 колонки
   \begin{choices} % опция [o] не рандомизирует порядок ответов
      \correctchoice{$0.87$}
      \wrongchoice{$0.57$}
      \wrongchoice{$0.67$}
      \wrongchoice{$0.77$}
      \wrongchoice{$0.97$}
      \end{choices}
  \end{multicols}
  \end{questionmult}
}

\element{midterm_retake_2016}{ % в фигурных скобках название группы вопросов
 \AMCcompleteMulti
  \begin{questionmult}{4} % тип вопроса (questionmult — множественный выбор) и в фигурных — номер вопроса
  Размер выплаты страховой компанией является неотрицательной случайной величиной с математическим ожиданием 20\,000 рублей. При помощи неравенства Маркова оцените сверху вероятность того, что величина очередной выплаты превысит 50\,000 рублей. Искомая оценка сверху равна
 \begin{multicols}{3} % располагаем ответы в 3 колонки
   \begin{choices} % опция [o] не рандомизирует порядок ответов
      \correctchoice{$0.4$}
      \wrongchoice{$0.2$}
      \wrongchoice{$0.3$}
      \wrongchoice{$0.5$}
      \wrongchoice{$0.6$}
      \end{choices}
  \end{multicols}
  \end{questionmult}
}

\element{midterm_retake_2016}{ % в фигурных скобках название группы вопросов
 \AMCcompleteMulti
  \begin{questionmult}{5} % тип вопроса (questionmult — множественный выбор) и в фигурных — номер вопроса
  Размер выплаты страховой компанией является неотрицательной случайной величиной с математическим ожиданием 50\,000 рублей и стандартным отклонением 10\,000 рублей. При помощи неравенства Чебышева оцените снизу вероятность того, что величина очередной выплаты по данному виду полисов будет отличаться от своего математического ожидания не более чем на 30\,000 рублей. Искомая оценка снизу равна
 \begin{multicols}{3} % располагаем ответы в 3 колонки
   \begin{choices} % опция [o] не рандомизирует порядок ответов
      \correctchoice{$8/9$}
      \wrongchoice{$1/9$}
      \wrongchoice{$2/3$}
      \wrongchoice{$7/9$}
      \wrongchoice{$3/5$}
      \end{choices}
  \end{multicols}
  \end{questionmult}
}

\element{midterm_retake_2016}{ % в фигурных скобках название группы вопросов
 \AMCcompleteMulti
  \begin{questionmult}{6} % тип вопроса (questionmult — множественный выбор) и в фигурных — номер вопроса
  Вероятность поражения мишени при одном выстреле равна $0.5$. Случайная величина $\xi_i$  равна $1$, если при $i$-ом выстреле было попадание, и равна $0$ в противном случае. Предел по вероятности последовательности $\frac{e^{\xi_1} + \ldots + e^{\xi_n}}{n}$ при $n \rightarrow \infty$ равен
 \begin{multicols}{3} % располагаем ответы в 3 колонки
   \begin{choices} % опция [o] не рандомизирует порядок ответов
      \correctchoice{$\frac{1+e}{2}$}
      \wrongchoice{$\frac{1}{2}$}
      \wrongchoice{$\frac{e}{2}$}
      \wrongchoice{$\frac{1}{e}$}
      \wrongchoice{$\frac{1}{1+e}$}
      \end{choices}
  \end{multicols}
  \end{questionmult}
}

\element{midterm_retake_2016}{ % в фигурных скобках название группы вопросов
 \AMCcompleteMulti
  \begin{questionmult}{7} % тип вопроса (questionmult — множественный выбор) и в фигурных — номер вопроса
    Функция совместной плотности случайных величин $X$ и $Y$ имеет вид
    \[
        f_{X,\,Y}(x,\,y) =
        \left\{
          \begin{array}{ll}
            c \cdot (2x + y), & \text{при $(x,\,y) \in [0;\,1] \times [0;\,1]$,} \\
            0,                & \text{при $(x,\,y) \not\in [0;\,1] \times [0;\,1]$.}
          \end{array}
        \right.
    \]
    Константа $c$ равна
 \begin{multicols}{3} % располагаем ответы в 3 колонки
   \begin{choices} % опция [o] не рандомизирует порядок ответов
      \correctchoice{$\frac{2}{3}$}
      \wrongchoice{$\frac{3}{2}$}
      \wrongchoice{$\frac{3}{4}$}
      \wrongchoice{$\frac{4}{3}$}
      \wrongchoice{$1$}
      \end{choices}
  \end{multicols}
  \end{questionmult}
}

\element{midterm_retake_2016}{ % в фигурных скобках название группы вопросов
 \AMCcompleteMulti
  \begin{questionmult}{8} % тип вопроса (questionmult — множественный выбор) и в фигурных — номер вопроса
    Функция совместной плотности случайных величин $X$ и $Y$ имеет вид
    \[
        f_{X,\,Y}(x,\,y) =
        \left\{
          \begin{array}{ll}
            x + y, & \text{при $(x,\,y) \in [0;\,1] \times [0;\,1]$,} \\
            0,                & \text{при $(x,\,y) \not\in [0;\,1] \times [0;\,1]$.}
          \end{array}
        \right.
    \]
    Частная функция плотности $f_X(x)$ равна
 \begin{multicols}{2} % располагаем ответы в 3 колонки
   \begin{choices} % опция [o] не рандомизирует порядок ответов
      \correctchoice{\[
          f_{X}(x) =
          \begin{cases}
              x + 1/2,                     & \text{при }x \in [0;\,1] \\
              0,                           & \text{при }x \not\in [0;\,1]
            \end{cases}
      \]}
      \wrongchoice{\[
          f_{X}(x) =
          \begin{cases}
              1,                     & \text{при }x \in [0;\,1] \\
              0,                           & \text{при }x \not\in [0;\,1]
            \end{cases}
      \]}
      \wrongchoice{\[
          f_{X}(x) =
          \begin{cases}
              2x,                     & \text{при }x \in [0;\,1] \\
              0,                           & \text{при }x \not\in [0;\,1]
            \end{cases}
      \]}
      \wrongchoice{\[
          f_{X}(x) =
          \begin{cases}
              x/2 + 1/2,                     & \text{при }x \in [0;\,1] \\
              0,                           & \text{при }x \not\in [0;\,1]
            \end{cases}
      \]}
      \wrongchoice{\[
          f_{X}(x) =
          \begin{cases}
              x,                      & \text{при }x \in [0;\,1] \\
              0,                           & \text{при }x \not\in [0;\,1]
            \end{cases}
      \]}
      \end{choices}
  \end{multicols}
  \end{questionmult}
}

\element{midterm_retake_2016}{ % в фигурных скобках название группы вопросов
 \AMCcompleteMulti
  \begin{questionmult}{9} % тип вопроса (questionmult — множественный выбор) и в фигурных — номер вопроса
    Совместная функция плотности случайных величин $X$ и $Y$ имеет вид
    \[
        f_{X,\,Y}(x,\,y) =
        \begin{cases}
            x + y, & \text{при }(x,\,y) \in [0;\,1] \times [0;\,1] \\
            0,                & \text{при }(x,\,y) \not\in [0;\,1] \times [0;\,1]
          \end{cases}
    \]
    Математическое ожидание $\E(X)$ равно
 \begin{multicols}{3} % располагаем ответы в 3 колонки
   \begin{choices} % опция [o] не рандомизирует порядок ответов
      \correctchoice{$\frac{7}{12}$}
      \wrongchoice{$\frac{5}{12}$}
      \wrongchoice{$\frac{1}{2}$}
      \wrongchoice{$\frac{2}{3}$}
      \wrongchoice{$\frac{3}{4}$}
      \end{choices}
  \end{multicols}
  \end{questionmult}
}

\element{midterm_retake_2016}{ % в фигурных скобках название группы вопросов
 \AMCcompleteMulti
  \begin{questionmult}{10} % тип вопроса (questionmult — множественный выбор) и в фигурных — номер вопроса
    Случайный вектор $(X, \, Y)$ имеет двумерное нормальное распределение с математическим ожиданием $(-1, \,1)$ и ковариационной матрицей
    $
    \begin{pmatrix}
        2 & 1 \\
        1 & 3 \\
    \end{pmatrix}
    $.
    Вероятность события $\{X + Y < 0\}$ равна
 \begin{multicols}{3} % располагаем ответы в 3 колонки
   \begin{choices} % опция [o] не рандомизирует порядок ответов
      \correctchoice{$0.5$}
      \wrongchoice{$0.33$}
      \wrongchoice{$0.45$}
      \wrongchoice{$0.67$}
      \wrongchoice{$0.74$}
      \end{choices}
  \end{multicols}
  \end{questionmult}
}

\element{midterm_retake_2016}{ % в фигурных скобках название группы вопросов
 \AMCcompleteMulti
  \begin{questionmult}{11} % тип вопроса (questionmult — множественный выбор) и в фигурных — номер вопроса
  Случайная величина $X$ равновероятно принимает значения $1$ и $2$. Её дисперсия $\Var(X)$ равна
 \begin{multicols}{3} % располагаем ответы в 3 колонки
   \begin{choices} % опция [o] не рандомизирует порядок ответов
      \correctchoice{$1/4$}
      \wrongchoice{$1/2$}
      \wrongchoice{$1/8$}
      \wrongchoice{$1/3$}
      \wrongchoice{$2/3$}
      \wrongchoice{$3/2$}
      \end{choices}
  \end{multicols}
  \end{questionmult}
}

\element{midterm_retake_2016}{ % в фигурных скобках название группы вопросов
 \AMCcompleteMulti
  \begin{questionmult}{12} % тип вопроса (questionmult — множественный выбор) и в фигурных — номер вопроса
  В школе три девятых класса: 9А, 9Б и 9В. В 9А классе — 70\% отличники, в 9Б — 30\%, в 9В — 50\%. Если сначала равновероятно выбрать один из трёх классов, а затем внутри класса равновероятно выбрать школьника, то вероятность выбрать отличника равна
 \begin{multicols}{3} % располагаем ответы в 3 колонки
   \begin{choices} % опция [o] не рандомизирует порядок ответов
     \correctchoice{$0.5$}
     \wrongchoice{$0.3$}
     \wrongchoice{$0.4$}
     \wrongchoice{$0.27$}
     \wrongchoice{$1/3^3$}
     \wrongchoice{$0.7\cdot 0.3\cdot 0.5$}
      \end{choices}
  \end{multicols}
  \end{questionmult}
}

\element{midterm_retake_2016}{ % в фигурных скобках название группы вопросов
 \AMCcompleteMulti
  \begin{questionmult}{13} % тип вопроса (questionmult — множественный выбор) и в фигурных — номер вопроса
  Если $\P(A)=0.4$, $\P(B)=0.5$, $\P(A\cup B)=0.8$, то вероятность $\P(A\cap B)$ равна
 \begin{multicols}{3} % располагаем ответы в 3 колонки
   \begin{choices} % опция [o] не рандомизирует порядок ответов
      \correctchoice{$0.1$}
      \wrongchoice{$0.3$}
      \wrongchoice{$0.5$}
      \wrongchoice{$0.2$}
      \wrongchoice{$0.14$}
      \end{choices}
  \end{multicols}
  \end{questionmult}
}

\element{midterm_retake_2016}{ % в фигурных скобках название группы вопросов
 \AMCcompleteMulti
  \begin{questionmult}{14} % тип вопроса (questionmult — множественный выбор) и в фигурных — номер вопроса
  Маша равновероятно бывает в хорошем и плохом настроении. Если она в хорошем настроении, то она надевает шарф с вероятностью $0.7$, а если в плохом, то с вероятностью $0.2$. Сейчас Маша с шарфом. Условная вероятность того, что Маша — в хорошем настроении, равна
 \begin{multicols}{3} % располагаем ответы в 3 колонки
   \begin{choices} % опция [o] не рандомизирует порядок ответов
      \correctchoice{$7/9$}
      \wrongchoice{$6/9$}
      \wrongchoice{$5/9$}
      \wrongchoice{$2/7$}
      \wrongchoice{$5/7$}
      \end{choices}
  \end{multicols}
  \end{questionmult}
}

\element{midterm_retake_2016}{ % в фигурных скобках название группы вопросов
 \AMCcompleteMulti
  \begin{questionmult}{15} % тип вопроса (questionmult — множественный выбор) и в фигурных — номер вопроса
  Если события $A$ и $B$ несовместны, то
 \begin{multicols}{3} % располагаем ответы в 3 колонки
   \begin{choices} % опция [o] не рандомизирует порядок ответов
      \correctchoice{$\P(A\cup B)=\P(A)+\P(B)$}
      \wrongchoice{$\P(A\cup B)=1$}
      \wrongchoice{$\P(A)=\P(B)$}
      \wrongchoice{$\P(A\cap B)=1$}
      \wrongchoice{$\P(A) + \P(B)=1$}
      \wrongchoice{$\P(A) \cdot \P(B)=1$}
      \end{choices}
  \end{multicols}
  \end{questionmult}
}

\element{midterm_retake_2016}{ % в фигурных скобках название группы вопросов
 \AMCcompleteMulti
  \begin{questionmult}{16} % тип вопроса (questionmult — множественный выбор) и в фигурных — номер вопроса
  Если события $A$ и $B$ независимы, то
 \begin{multicols}{3} % располагаем ответы в 3 колонки
   \begin{choices} % опция [o] не рандомизирует порядок ответов
      \correctchoice{$\P(A | B)+\P(\bar A| \bar B)=1$}
      \wrongchoice{$\P(A) + \P(B)=1$}
      \wrongchoice{$\P(A\cap B)=0$}
      \wrongchoice{$\P(A|B)=\P(B|A)$}
      \wrongchoice{$\P(A) + \P(\bar B)=1$}
      \end{choices}
  \end{multicols}
  \end{questionmult}
}

\element{midterm_retake_2016}{ % в фигурных скобках название группы вопросов
 \AMCcompleteMulti
  \begin{questionmult}{17} % тип вопроса (questionmult — множественный выбор) и в фигурных — номер вопроса
  Монетка выпадает орлом с вероятностью $0.3$. Вероятность того, что при трёх подбрасываниях монетка выпадет орлом хотя бы один раз, равна
 \begin{multicols}{3} % располагаем ответы в 3 колонки
   \begin{choices} % опция [o] не рандомизирует порядок ответов
      \correctchoice{$0.657$}
      \wrongchoice{$0.027$}
      \wrongchoice{$0.9$}
      \wrongchoice{$0.1$}
      \wrongchoice{$0.17$}
      \end{choices}
  \end{multicols}
  \end{questionmult}
}

\element{midterm_retake_2016}{ % в фигурных скобках название группы вопросов
 \AMCcompleteMulti
  \begin{questionmult}{18} % тип вопроса (questionmult — множественный выбор) и в фигурных — номер вопроса
  Известно, что $\E(X)=3$, $\E(Y)=2$, $\Var(X)=16$, $\Var(Y)=1$, $\Cov(X,Y)=2$. Ожидание $\E(XY+3X)$ равно
 \begin{multicols}{3} % располагаем ответы в 3 колонки
   \begin{choices} % опция [o] не рандомизирует порядок ответов
      \correctchoice{$17$}
      \wrongchoice{$18$}
      \wrongchoice{$20$}
      \wrongchoice{$19$}
      \wrongchoice{$21$}
      \end{choices}
  \end{multicols}
  \end{questionmult}
}

\element{midterm_retake_2016}{ % в фигурных скобках название группы вопросов
 \AMCcompleteMulti
  \begin{questionmult}{19} % тип вопроса (questionmult — множественный выбор) и в фигурных — номер вопроса
  Известно, что $\E(X)=3$,  $\E(X^2)=10$, $\E(X^3)=0$, $\E(X^4)=200$. Дисперсия $\Var(X+X^2)$ равна
 \begin{multicols}{3} % располагаем ответы в 3 колонки
   \begin{choices} % опция [o] не рандомизирует порядок ответов
      \correctchoice{$41$}
      \wrongchoice{$14$}
      \wrongchoice{$25$}
      \wrongchoice{$101$}
      \wrongchoice{$15$}
      \end{choices}
  \end{multicols}
  \end{questionmult}
}

\element{midterm_retake_2016}{ % в фигурных скобках название группы вопросов
 \AMCcompleteMulti
  \begin{questionmult}{20} % тип вопроса (questionmult — множественный выбор) и в фигурных — номер вопроса
  Если $\Corr(X, Y)= -0.5$ и $\Var(X)=\Var(Y)$, то $\Corr(X + Y, 2Y - 7)$ равна
 \begin{multicols}{3} % располагаем ответы в 3 колонки
   \begin{choices} % опция [o] не рандомизирует порядок ответов
     \correctchoice{$1/2$}
     \wrongchoice{$\sqrt{2}/3$}
     \wrongchoice{$1$}
     \wrongchoice{$0$}
     \wrongchoice{$\sqrt{3}/2$}
     \wrongchoice{$\sqrt{3}/3$}
      \end{choices}
  \end{multicols}
  \end{questionmult}
}

\element{midterm_retake_2016}{ % в фигурных скобках название группы вопросов
 \AMCcompleteMulti
  \begin{questionmult}{21} % тип вопроса (questionmult — множественный выбор) и в фигурных — номер вопроса
  Правильный кубик подбрасывается 7 раз. Вероятность того, что ровно два раза выпадет шестерка, равна
 \begin{multicols}{3} % располагаем ответы в 3 колонки
   \begin{choices} % опция [o] не рандомизирует порядок ответов
      \correctchoice{$21\cdot 5^5/6^7$}
      \wrongchoice{$2\cdot 5^6/6^7$}
      \wrongchoice{$5^5/6^7$}
      \wrongchoice{$1/6^7$}
      \wrongchoice{$42\cdot 5^5/6^7$}
      \end{choices}
  \end{multicols}
  \end{questionmult}
}

\element{midterm_retake_2016}{ % в фигурных скобках название группы вопросов
 \AMCcompleteMulti
  \begin{questionmult}{22} % тип вопроса (questionmult — множественный выбор) и в фигурных — номер вопроса
  Правильная монетка подбрасывается 16 раз. Математическое ожидание и дисперсия числа выпавших орлов равны соответственно
 \begin{multicols}{3} % располагаем ответы в 3 колонки
   \begin{choices} % опция [o] не рандомизирует порядок ответов
      \correctchoice{$8$ и $4$}
      \wrongchoice{$8$ и $2$}
      \wrongchoice{$4$ и $8$}
      \wrongchoice{$4$ и $16$}
      \wrongchoice{$8$ и $16$}
      \end{choices}
  \end{multicols}
  \end{questionmult}
}



\element{midterm_retake_2016}{ % в фигурных скобках название группы вопросов
 \AMCcompleteMulti
  \begin{questionmult}{23} % тип вопроса (questionmult — множественный выбор) и в фигурных — номер вопроса
  Правильный кубик подбрасывается до первой шестёрки. Наиболее вероятное общее количество бросков равняется
 \begin{multicols}{3} % располагаем ответы в 3 колонки
   \begin{choices} % опция [o] не рандомизирует порядок ответов
      \correctchoice{$1$}
      \wrongchoice{$6$}
      \wrongchoice{$36$}
      \wrongchoice{$1/6$}
      \wrongchoice{$1/36$}
      \end{choices}
  \end{multicols}
  \end{questionmult}
}



\element{midterm_retake_2016}{ % в фигурных скобках название группы вопросов
 \AMCcompleteMulti
  \begin{questionmult}{24} % тип вопроса (questionmult — множественный выбор) и в фигурных — номер вопроса
  Подбрасывается 10 правильных игральных кубиков. Математическое ожидание суммы выпавших очков равно
 \begin{multicols}{3} % располагаем ответы в 3 колонки
   \begin{choices} % опция [o] не рандомизирует порядок ответов
      \correctchoice{$35$}
      \wrongchoice{$17.5$}
      \wrongchoice{$36$}
      \wrongchoice{$6$}
      \wrongchoice{$18$}
      \end{choices}
  \end{multicols}
  \end{questionmult}
}


\element{midterm_retake_2016}{ % в фигурных скобках название группы вопросов
 \AMCcompleteMulti
  \begin{questionmult}{25} % тип вопроса (questionmult — множественный выбор) и в фигурных — номер вопроса
  Правильный кубик подбрасывается два раза, величина $X$ — сумма выпавших очков, величина $Y$ равна единице, если в первый раз выпало $2$ и нулю иначе. Ожидание $\E(X|Y=0)$ равно
 \begin{multicols}{3} % располагаем ответы в 3 колонки
   \begin{choices} % опция [o] не рандомизирует порядок ответов
      \correctchoice{$7.3$}
      \wrongchoice{$3.6$}
      \wrongchoice{$7$}
      \wrongchoice{$6.8$}
      \wrongchoice{$5$}
      \wrongchoice{$6$}
      \end{choices}
  \end{multicols}
  \end{questionmult}
}


\element{midterm_retake_2016}{ % в фигурных скобках название группы вопросов
 \AMCcompleteMulti
  \begin{questionmult}{26} % тип вопроса (questionmult — множественный выбор) и в фигурных — номер вопроса
  Правильный кубик подбрасывается два раза, величина $X$ — сумма выпавших очков, величина $Y$ равна единице, если в первый раз выпало $2$ и нулю иначе. Вероятность $\P(X=3|Y=1)$ равна
 \begin{multicols}{3} % располагаем ответы в 3 колонки
   \begin{choices} % опция [o] не рандомизирует порядок ответов
      \correctchoice{$1/6$}
      \wrongchoice{$2/6$}
      \wrongchoice{$3/6$}
      \wrongchoice{$4/6$}
      \wrongchoice{$5/6$}
      \wrongchoice{$6/6$}
      \end{choices}
  \end{multicols}
  \end{questionmult}
}



\element{midterm_retake_2016}{ % в фигурных скобках название группы вопросов
 \AMCcompleteMulti
  \begin{questionmult}{27} % тип вопроса (questionmult — множественный выбор) и в фигурных — номер вопроса
    Совместное распределение величин $X$ и $Y$ задано таблицей

    \begin{tabular}{c|ccc}
     & $Y=-1$ & $Y=0$ & $Y=1$ \\
    \hline
    $X=0$ & $1/6$ & $0$  &  $1/6$\\
    $X=2$ & $1/3$ & $1/6$ &  $1/6$ \\
    \end{tabular}

    Ковариация случайных величин $X$ и $Y$ равна
 \begin{multicols}{3} % располагаем ответы в 3 колонки
   \begin{choices} % опция [o] не рандомизирует порядок ответов
      \correctchoice{$-1/9$}
      \wrongchoice{$-1/3$}
      \wrongchoice{$0$}
      \wrongchoice{$1/3$}
      \wrongchoice{$2/3$}
      \wrongchoice{$-2/3$}
      \end{choices}
  \end{multicols}
  \end{questionmult}
}

\element{midterm_retake_2016}{ % в фигурных скобках название группы вопросов
 \AMCcompleteMulti
  \begin{questionmult}{28} % тип вопроса (questionmult — множественный выбор) и в фигурных — номер вопроса
    Совместное распределение величин $X$ и $Y$ задано таблицей

    \begin{tabular}{c|ccc}
     & $Y=-1$ & $Y=0$ & $Y=1$ \\
    \hline
    $X=0$ & $1/6$ & $0$  &  $1/6$\\
    $X=2$ & $1/3$ & $1/6$ &  $1/6$ \\
    \end{tabular}

    Вероятность $\P(X>0)$ равна
 \begin{multicols}{3} % располагаем ответы в 3 колонки
   \begin{choices} % опция [o] не рандомизирует порядок ответов
      \correctchoice{$4/6$}
      \wrongchoice{$1/6$}
      \wrongchoice{$2/6$}
      \wrongchoice{$3/6$}
      \wrongchoice{$5/6$}
      \wrongchoice{$6/6$}
      \end{choices}
  \end{multicols}
  \end{questionmult}
}


\element{midterm_retake_2016}{ % в фигурных скобках название группы вопросов
 \AMCcompleteMulti
  \begin{questionmult}{29} % тип вопроса (questionmult — множественный выбор) и в фигурных — номер вопроса
    Вася выбирает случайную точку внутри единичного круга с центром в начале координат. Пусть $X$ и $Y$ — абсцисса и ордината этой точки. Значение совместной функции плотности $X$ и $Y$ в точках $A=(1,1)$ и $B=(0.1, 0.1)$ равны
 \begin{multicols}{2} % располагаем ответы в 3 колонки
   \begin{choices} % опция [o] не рандомизирует порядок ответов
      \correctchoice{$f(A)=0$, $f(B)=1/\pi$}
      \wrongchoice{$f(A)=1/\pi$, $f(B)=\pi$}
      \wrongchoice{$f(A)=\pi$, $f(B)=\pi$}
      \wrongchoice{$f(A)=\sqrt{2\pi}$, $f(B)=\sqrt{0.2\pi}$}
      \wrongchoice{$f(A)=2$, $f(B)=0.2$}
      \wrongchoice{$f(A)=2\pi$, $f(B)=0.2\pi$}
      \end{choices}
  \end{multicols}
  \end{questionmult}
}

\element{midterm_retake_2016}{ % в фигурных скобках название группы вопросов
 \AMCcompleteMulti
  \begin{questionmult}{30} % тип вопроса (questionmult — множественный выбор) и в фигурных — номер вопроса
    Вася выбирает случайную точку внутри единичного круга с центром в начале координат. Пусть $X$ и $Y$ — абсцисса и ордината этой точки. При известном и фиксированном $Y$ величина $X$ имеет распределение
 \begin{multicols}{3} % располагаем ответы в 3 колонки
   \begin{choices} % опция [o] не рандомизирует порядок ответов
      \correctchoice{равномерное}
      \wrongchoice{нормальное $\cN(0;1)$}
      \wrongchoice{экспоненциальное}
      \wrongchoice{Пуассона}
      \wrongchoice{биномиальное}
      \wrongchoice{геометрическое}
      \end{choices}
  \end{multicols}
  \end{questionmult}
}


\element{midterm_retake_2016}{ % в фигурных скобках название группы вопросов
 \AMCcompleteMulti
  \begin{questionmult}{31} % тип вопроса (questionmult — множественный выбор) и в фигурных — номер вопроса
    Совместное распределение пары величин $X$ и $Y$ задается функцией плотности
    \[
    f(x) = \begin{cases}
         				9 x^2 y^2, x \in [0,1], y \in [0,1] \\
         				0,\text{ иначе}
     				\end{cases}
    \]
    Вероятность $\P(X<0.5)$ равняется
 \begin{multicols}{3} % располагаем ответы в 3 колонки
   \begin{choices} % опция [o] не рандомизирует порядок ответов
      \correctchoice{$1/8$}
      \wrongchoice{$1/64$}
      \wrongchoice{$1/4$}
      \wrongchoice{$1/96$}
      \wrongchoice{$1/128$}
      \wrongchoice{$1/16$}
      \end{choices}
  \end{multicols}
  \end{questionmult}
}


\element{midterm_retake_2016}{ % в фигурных скобках название группы вопросов
 \AMCcompleteMulti
  \begin{questionmult}{32} % тип вопроса (questionmult — множественный выбор) и в фигурных — номер вопроса
    Случайный вектор $(\xi, \eta)^T$ имеет нормальное распределение
    $\cN \left(
    \begin{pmatrix}
      1 \\
      0
    \end{pmatrix};
    \begin{pmatrix}
      1 & 1/2 \\
      1/2 & 1
    \end{pmatrix}
  \right)$. Условное математическое ожидание и условная дисперсия равны
 \begin{multicols}{2} % располагаем ответы в 3 колонки
   \begin{choices} % опция [o] не рандомизирует порядок ответов
      \correctchoice{$\E(\xi | \eta=1)=3/2$, $\Var(\xi | \eta=1)=3/4$}
      \wrongchoice{$\E(\xi | \eta=1)=1/2$, $\Var(\xi | \eta=1)=3/4$}
      \wrongchoice{$\E(\xi | \eta=1)=0$, $\Var(\xi | \eta=1)=1$}
      \wrongchoice{$\E(\xi | \eta=1)=1$, $\Var(\xi | \eta=1)=1/2$}
      \wrongchoice{$\E(\xi | \eta=1)=1$, $\Var(\xi | \eta=1)=1$}
      \wrongchoice{$\E(\xi | \eta=1)=1/2$, $\Var(\xi | \eta=1)=1/4$}
      \end{choices}
  \end{multicols}
  \end{questionmult}
}


\onecopy{1}{

\noindent{\bf Теория вероятностей и математическая статистика  \hfill Пересдача, 30.01.2017}

\vspace{3ex}

\namefield{\fbox{
  \begin{minipage}{42em}
    Имя, фамилия:\vspace*{3ex}\par
    \noindent\dotfill\vspace{2mm}

    Номер группы:\vspace*{3ex}\par
    \noindent\dotfill\vspace{2mm}
  \end{minipage}
}}

\vspace{3ex}

Можно пользоваться простым калькулятором.  В каждом из 32 вопросов один верный ответ.

\vspace{3ex}

Ни пуха, ни пера!

\vspace{3ex}

\cleargroup{all}

\shufflegroup{midterm_retake_2016}
\copygroup[32]{midterm_retake_2016}{all}


%\shufflegroup{all}

\insertgroup{all}




%\AMCcleardoublepage
\clearpage

\AMCformBegin

% добавляем/убираем коммент
Ура! На этой страничке вопросов уже нет :)
%Это листок для ответов. Учитываются только ответы, перенесённые на этот листок.

\namefield{\fbox{
  \begin{minipage}{42em}
    Имя, фамилия и номер группы:\vspace*{3ex}\par
    \noindent\dotfill\vspace{2mm}
  \end{minipage}
}}


\vspace{2ex}

\AMCform

}
\end{document}
