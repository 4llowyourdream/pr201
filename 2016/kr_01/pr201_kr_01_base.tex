\documentclass[a4paper, 12pt]{article}


\usepackage{mathrsfs}

\usepackage{amscd}
\usepackage[paper=a4paper,top=20.0mm, bottom=20.0mm,left=20.0mm,right=20.0mm,includefoot]{geometry} % размер листа бумаги


\usepackage{tikz} % картинки в tikz
\usepackage{microtype} % свешивание пунктуации

\usepackage{floatrow} % для выравнивания рисунка и подписи
\usepackage{caption} % для пустых подписей

\usepackage{array} % для столбцов фиксированной ширины

\usepackage{indentfirst} % отступ в первом параграфе

\usepackage{sectsty} % для центрирования названий частей
\allsectionsfont{\centering}

\usepackage{amsmath, amsfonts} % куча стандартных математических плюшек

\usepackage{comment} % для комментариев

\usepackage{multicol} % текст в несколько колонок

\usepackage{lastpage} % чтобы узнать номер последней страницы

\usepackage{enumitem} % дополнительные плюшки для списков
%  например \begin{enumerate}[resume] позволяет продолжить нумерацию в новом списке



\usepackage{fontspec}
\usepackage{polyglossia}

\setmainlanguage{russian}
\setotherlanguages{english}

% download "Linux Libertine" fonts:
% http://www.linuxlibertine.org/index.php?id=91&L=1
\setmainfont{Linux Libertine O} % or Helvetica, Arial, Cambria
% why do we need \newfontfamily:
% http://tex.stackexchange.com/questions/91507/
\newfontfamily{\cyrillicfonttt}{Linux Libertine O}

\AddEnumerateCounter{\asbuk}{\russian@alph}{щ} % для списков с русскими буквами


\begin{document}
\begin{center}
    \textbf{\textbf{\large Контрольная работа №\,1 по ТВ\,и\,МС [2016--2017]}}
\end{center}


\textbf{Ф.\,И.\,О.}

\bigskip

\begin{enumerate}
\item  Из семей, имеющих двоих разновозрастных детей, случайным образом выбирается одна семья. Известно, что в семье есть девочка (событие $A$).

\begin{enumerate}
\item	Какова вероятность того, что в семье есть мальчик (событие $B$)?

\item	Сформулируйте определение независимости событий и проверьте, являются ли события $A$ и $B$ независимыми?
\end{enumerate}

\item  Система состоит из $N$ независимых узлов. При выходе из строя хотя бы одного узла, система дает сбой. Вероятность выхода из строя любого из узлов равна 0.000001. Вычислите максимально возможное число узлов системы, при котором вероятность её сбоя не превышает 0.01. 	

\item  Исследование состояния здоровья населения в шахтерском регионе «Велико-кротовск» за пятилетний период показало, что из всех людей с диагностированным заболеванием легких, 22\% работало на шахтах. Из тех, у кого не было диагностировано заболевание легких, только 14\% работало на шахтах. Заболевание легких было диагностировано у 4\% населения региона.

\begin{enumerate}
\item	Какой процент людей среди тех, кто работал в шахте, составляют люди с диагностированным заболеванием легких?

\item	Какой процент людей среди тех, кто НЕ работал в шахте, составляют люди с диагностированным заболеванием легких?
\end{enumerate} 

\item  Студент Петя выполняет тест (множественного выбора) проставлением ответов наугад. В тесте 17 вопросов, в каждом из которых пять вариантов ответов и только один из них правильный. Оценка по десятибалльной шкале формируется следующим образом:
\[
    \text{Оценка} = \left\{
                      \begin{array}{ll}
                        \text{ЧПО} - 7, & \text{если $\text{ЧПО}\in [8;\,17]$,} \\
                        1,              & \text{если $\text{ЧПО}\in [0;\,7]$,}
                      \end{array}
                    \right.
\]
где ЧПО означает число правильных ответов.

\begin{enumerate}
\item	Найдите наиболее вероятное число правильных ответов.

\item	Найдите математическое ожидание и дисперсию числа правильных ответов.

\item	Найдите вероятность того, что Петя получит «отлично» (по десятибалльной шкале получит 8, 9 или 10 баллов).

Студент Вася также выполняет тест проставлением ответов наугад.

\item	Найдите вероятность того, что все ответы Пети и Васи совпадут.
\end{enumerate}

\newpage
\item  Продавец высокотехнологичного оборудования контактирует с одним или двумя потенциальными покупателями в день с вероятностями 1/3 и 2/3 соответственно. Каждый контакт заканчивается «ничем» с вероятностью 0.9 и покупкой оборудования на сумму в 50\,000 у.\,е. с вероятностью 0.1. Пусть $\xi$ — случайная величина, означающая объем дневных продаж в у.\,е.

\begin{enumerate}
\item	Вычислите  $\mathbb{P}(\{\xi = 0\})$.

\item	Сформулируйте определение функции распределения и постройте функцию распределения случайной величины $\xi$.

\item	Вычислите математическое ожидание и дисперсию случайной величины $\xi$.
\end{enumerate}


\item  Интервал движения поездов метро фиксирован и равен $b$ минут, т.\,е. каждый следующий поезд появляется после предыдущего ровно через $b$ минут. Пассажир приходит на станцию в случайный момент времени. Пусть случайная величина $\xi$, означающая время ожидания поезда, имеет равномерное распределение на отрезке $[0; \, b]$.

\begin{enumerate}
\item Запишите плотность распределения случайной величины $\xi$.

\item	Найдите константу $b$, если известно, что в среднем пассажиру приходится ждать поезда одну минуту, т.\,е. $\mathbb{E}[\xi] = 1$.

\item	Вычислите дисперсию случайной величины $\xi$.

\item	Найдите вероятность того, что пассажир будет ждать поезд менее одной минуты.

\item	Найдите квантиль порядка $0.25$ распределения случайной величины $\xi$.

\item	Найдите центральный момент порядка 2017 случайной величины $\xi$.

\item	Постройте функцию распределения случайной величины $\xi$.

Марья Ивановна из суеверия всегда пропускает два поезда и садится в третий.

\item	Найдите математическое ожидание и дисперсию времени, затрачиваемого Марьей Ивановной на ожидание «своего» поезда.

Глафира Петровна не садится в поезд, если видит в нем подозрительного человека. Подозрительные люди встречаются в каждом поезде с вероятностью $3/4$.

\item	Найдите вероятность того, что Глафире Петровне придется ждать не менее пяти минут, чтобы уехать со станции.

\item	Найдите математическое ожидание времени ожидания «своего» поезда для Глафиры Петровны.
\end{enumerate}

\item  (Бонусная задача) На первом этаже десятиэтажного дома в лифт заходят 9 человек. Найдите математическое ожидание числа остановок лифта, если люди выходят из лифта независимо друг от друга.
\end{enumerate}

\end{document} 