\documentclass[pdftex,12pt,a4paper]{article}

%%%%%%%%%%%%%%%%%%%%%%%  Загрузка пакетов  %%%%%%%%%%%%%%%%%%%%%%%%%%%%%%%%%%

%\usepackage{showkeys} % показывать метки в готовом pdf 

\usepackage{etex} % расширение классического tex
% в частности позволяет подгружать гораздо больше пакетов, чем мы и займёмся далее

\usepackage{cmap} % для поиска русских слов в pdf
\usepackage{verbatim} % для многострочных комментариев
\usepackage{makeidx} % для создания предметных указателей
\usepackage[X2,T2A]{fontenc}
\usepackage[utf8]{inputenc} % задание utf8 кодировки исходного tex файла
\usepackage{setspace}
\usepackage{amsmath,amsfonts,amssymb,amsthm}
\usepackage{mathrsfs} % sudo yum install texlive-rsfs
\usepackage{dsfont} % sudo yum install texlive-doublestroke
\usepackage{array,multicol,multirow,bigstrut} % sudo yum install texlive-multirow
\usepackage{indentfirst} % установка отступа в первом абзаце главы
\usepackage[british,russian]{babel} % выбор языка для документа
\usepackage{bm}
\usepackage{bbm} % шрифт с двойными буквами
\usepackage[perpage]{footmisc}

% создание гиперссылок в pdf
\usepackage[pdftex,unicode,colorlinks=true,urlcolor=blue,hyperindex,breaklinks]{hyperref} 

% свешиваем пунктуацию 
% теперь знаки пунктуации могут вылезать за правую границу текста, при этом текст выглядит ровнее
\usepackage{microtype}

\usepackage{textcomp}  % Чтобы в формулах можно было русские буквы писать через \text{}

% размер листа бумаги
\usepackage[paper=a4paper,top=13.5mm, bottom=13.5mm,left=16.5mm,right=13.5mm,includefoot]{geometry}

\usepackage{xcolor}

\usepackage[pdftex]{graphicx} % для вставки графики 

\usepackage{float,longtable}
\usepackage{soulutf8}

\usepackage{enumitem} % дополнительные плюшки для списков
%  например \begin{enumerate}[resume] позволяет продолжить нумерацию в новом списке

\usepackage{mathtools}
\usepackage{cancel,xspace} % sudo yum install texlive-cancel

\usepackage{minted} % display program code with syntax highlighting
% требует установки pygments и python 

\usepackage{numprint} % sudo yum install texlive-numprint
\npthousandsep{,}\npthousandthpartsep{}\npdecimalsign{.}

\usepackage{embedfile} % Чтобы код LaTeXа включился как приложение в PDF-файл

\usepackage{subfigure} % для создания нескольких рисунков внутри одного

\usepackage{tikz,pgfplots} % язык для рисования графики из latex'a
\usetikzlibrary{trees} % tikz-прибамбас для рисовки деревьев
\usepackage{tikz-qtree} % альтернативный tikz-прибамбас для рисовки деревьев
\usetikzlibrary{arrows} % tikz-прибамбас для рисовки стрелочек подлиннее

\usepackage{todonotes} % для вставки в документ заметок о том, что осталось сделать
% \todo{Здесь надо коэффициенты исправить}
% \missingfigure{Здесь будет Последний день Помпеи}
% \listoftodos --- печатает все поставленные \todo'шки


% более красивые таблицы
\usepackage{booktabs}
% заповеди из докупентации: 
% 1. Не используйте вертикальные линни
% 2. Не используйте двойные линии
% 3. Единицы измерения - в шапку таблицы
% 4. Не сокращайте .1 вместо 0.1
% 5. Повторяющееся значение повторяйте, а не говорите "то же"



%\usepackage{asymptote} % пакет для рисовки графики, должен идти после graphics
% но мы переходим на tikz :)

%\usepackage{sagetex} % для интеграции с Sage (вероятно тоже должен идти после graphics)

% metapost создает упрощенные eps файлы, которые можно напрямую включать в pdf 
% эта группа команд декларирует, что файлы будут этого упрощенного формата
% если metapost не используется, то этот блок не нужен
\usepackage{ifpdf} % для определения, запускается ли pdflatex или просто латех
\ifpdf
	\DeclareGraphicsRule{*}{mps}{*}{}
\fi
%%%%%%%%%%%%%%%%%%%%%%%%%%%%%%%%%%%%%%%%%%%%%%%%%%%%%%%%%%%%%%%%%%%%%%


%%%%%%%%%%%%%%%%%%%%%%%  Внедрение tex исходников в pdf файл  %%%%%%%%%%%%%%%%%%%%%%%%%%%%%%%%%%
\embedfile[desc={Main tex file}]{\jobname.tex} % Включение кода в выходной файл
\embedfile[desc={title_bor}]{/home/boris/science/tex_general/title_bor_utf8.tex}

%%%%%%%%%%%%%%%%%%%%%%%%%%%%%%%%%%%%%%%%%%%%%%%%%%%%%%%%%%%%%%%%%%%%%%



%%%%%%%%%%%%%%%%%%%%%%%  ПАРАМЕТРЫ  %%%%%%%%%%%%%%%%%%%%%%%%%%%%%%%%%%
\setstretch{1}                          % Межстрочный интервал
\flushbottom                            % Эта команда заставляет LaTeX чуть растягивать строки, чтобы получить идеально прямоугольную страницу
\righthyphenmin=2                       % Разрешение переноса двух и более символов
\pagestyle{plain}                       % Нумерация страниц снизу по центру.
\widowpenalty=300                     % Небольшое наказание за вдовствующую строку (одна строка абзаца на этой странице, остальное --- на следующей)
\clubpenalty=3000                     % Приличное наказание за сиротствующую строку (омерзительно висящая одинокая строка в начале страницы)
\setlength{\parindent}{1.5em}           % Красная строка.
%\captiondelim{. }
\setlength{\topsep}{0pt}
%%%%%%%%%%%%%%%%%%%%%%%%%%%%%%%%%%%%%%%%%%%%%%%%%%%%%%%%%%%%%%%%%%%%%%



%%%%%%%% Это окружение, которое выравнивает по центру без отступа, как у простого center
\newenvironment{center*}{%
  \setlength\topsep{0pt}
  \setlength\parskip{0pt}
  \begin{center}
}{%
  \end{center}
}
%%%%%%%%%%%%%%%%%%%%%%%%%%%%%%%%%%%%%%%%%%%%%%%%%%%%%%%%%%%%%%%%%%%%%%


%%%%%%%%%%%%%%%%%%%%%%%%%%% Правила переноса  слов
\hyphenation{ }
%%%%%%%%%%%%%%%%%%%%%%%%%%%%%%%%%%%%%%%%%%%%%%%%%%%%%%%%%%%%%%%%%%%%%%

\emergencystretch=2em


% DEFS
\def \mbf{\mathbf}
\def \msf{\mathsf}
\def \mbb{\mathbb}
\def \tbf{\textbf}
\def \tsf{\textsf}
\def \ttt{\texttt}
\def \tbb{\textbb}

\def \wh{\widehat}
\def \wt{\widetilde}
\def \ni{\noindent}
\def \ol{\overline}
\def \cd{\cdot}
\def \fr{\frac}
\def \bs{\backslash}
\def \lims{\limits}
\DeclareMathOperator{\dist}{dist}
\DeclareMathOperator{\VC}{VCdim}
\DeclareMathOperator{\card}{card}
\DeclareMathOperator{\sign}{sign}
\DeclareMathOperator{\sgn}{sign}
\DeclareMathOperator{\Tr}{\mbf{Tr}}
\def \xfs{(x_1,\ldots,x_{n-1})}
\DeclareMathOperator*{\argmin}{arg\,min}
\DeclareMathOperator*{\amn}{arg\,min}
\DeclareMathOperator*{\amx}{arg\,max}

\DeclareMathOperator{\Corr}{Corr}
\DeclareMathOperator{\Cov}{Cov}
\DeclareMathOperator{\Var}{Var}
\DeclareMathOperator{\corr}{Corr}
\DeclareMathOperator{\cov}{Cov}
\DeclareMathOperator{\var}{Var}
\DeclareMathOperator{\bin}{Bin}
\DeclareMathOperator{\Bin}{Bin}
\DeclareMathOperator{\rang}{rang}
\DeclareMathOperator*{\plim}{plim}
\DeclareMathOperator{\med}{med}


\providecommand{\iff}{\Leftrightarrow}
\providecommand{\hence}{\Rightarrow}

\def \ti{\tilde}
\def \wti{\widetilde}

\def \mA{\mathcal{A}}
\def \mB{\mathcal{B}}
\def \mC{\mathcal{C}}
\def \mE{\mathcal{E}}
\def \mF{\mathcal{F}}
\def \mH{\mathcal{H}}
\def \mL{\mathcal{L}}
\def \mN{\mathcal{N}}
\def \mU{\mathcal{U}}
\def \mV{\mathcal{V}}
\def \mW{\mathcal{W}}


\def \R{\mbb R}
\def \N{\mbb N}
\def \Z{\mbb Z}
\def \P{\mbb{P}}
\def \p{\mbb{P}}
\newcommand{\E}{\mathbb{E}}
\def \D{\msf{D}}
\def \I{\mbf{I}}

\def \a{\alpha}
\def \b{\beta}
\def \t{\tau}
\def \dt{\delta}
\newcommand{\e}{\varepsilon}
\def \ga{\gamma}
\def \kp{\varkappa}
\def \la{\lambda}
\def \sg{\sigma}
\def \sgm{\sigma}
\def \tt{\theta}
\def \ve{\varepsilon}
\def \Dt{\Delta}
\def \La{\Lambda}
\def \Sgm{\Sigma}
\def \Sg{\Sigma}
\def \Tt{\Theta}
\def \Om{\Omega}
\def \om{\omega}

%\newcommand{\p}{\partial}
\newcommand{\PP}{\mathbb{P}}

\def \ni{\noindent}
\def \lq{\glqq}
\def \rq{\grqq}
\def \lbr{\linebreak}
\def \vsi{\vspace{0.1cm}}
\def \vsii{\vspace{0.2cm}}
\def \vsiii{\vspace{0.3cm}}
\def \vsiv{\vspace{0.4cm}}
\def \vsv{\vspace{0.5cm}}
\def \vsvi{\vspace{0.6cm}}
\def \vsvii{\vspace{0.7cm}}
\def \vsviii{\vspace{0.8cm}}
\def \vsix{\vspace{0.9cm}}
\def \VSI{\vspace{1cm}}
\def \VSII{\vspace{2cm}}
\def \VSIII{\vspace{3cm}}

\newcommand{\bls}[1]{\boldsymbol{#1}}
\newcommand{\bsA}{\boldsymbol{A}}
\newcommand{\bsH}{\boldsymbol{H}}
\newcommand{\bsI}{\boldsymbol{I}}
\newcommand{\bsP}{\boldsymbol{P}}
\newcommand{\bsR}{\boldsymbol{R}}
\newcommand{\bsS}{\boldsymbol{S}}
\newcommand{\bsX}{\boldsymbol{X}}
\newcommand{\bsY}{\boldsymbol{Y}}
\newcommand{\bsZ}{\boldsymbol{Z}}
\newcommand{\bse}{\boldsymbol{e}}
\newcommand{\bsq}{\boldsymbol{q}}
\newcommand{\bsy}{\boldsymbol{y}}
\newcommand{\bsbeta}{\boldsymbol{\beta}}
\newcommand{\fish}{\mathrm{F}}
\newcommand{\Fish}{\mathrm{F}}
\renewcommand{\phi}{\varphi}
\newcommand{\ind}{\mathds{1}}
\newcommand{\inds}[1]{\mathds{1}_{\{#1\}}}
\renewcommand{\to}{\rightarrow}
\newcommand{\sumin}{\sum\limits_{i=1}^n}
\newcommand{\ofbr}[1]{\bigl( \{ #1 \} \bigr)}     % Например, вероятность события. Большие круглые, нормальные фигурные скобки вокруг аргумента
\newcommand{\Ofbr}[1]{\Bigl( \bigl\{ #1 \bigr\} \Bigr)} % Например, вероятность события. Больше больших круглые, большие фигурные скобки вокруг аргумента
\newcommand{\oeq}{{}\textcircled{\raisebox{-0.4pt}{{}={}}}{}} % Равно в кружке
\newcommand{\og}{\textcircled{\raisebox{-0.4pt}{>}}}  % Знак больше в кружке

% вместо горизонтальной делаем косую черточку в нестрогих неравенствах
\renewcommand{\le}{\leqslant}
\renewcommand{\ge}{\geqslant}
\renewcommand{\leq}{\leqslant}
\renewcommand{\geq}{\geqslant}


\newcommand{\figb}[1]{\bigl\{ #1  \bigr\}} % большие фигурные скобки вокруг аргумента
\newcommand{\figB}[1]{\Bigl\{ #1  \Bigr\}} % Больше больших фигурные скобки вокруг аргумента
\newcommand{\parb}[1]{\bigl( #1  \bigr)}   % большие скобки вокруг аргумента
\newcommand{\parB}[1]{\Bigl( #1  \Bigr)}   % Больше больших круглые скобки вокруг аргумента
\newcommand{\parbb}[1]{\biggl( #1  \biggr)} % большие-большие круглые скобки вокруг аргумента
\newcommand{\br}[1]{\left( #1  \right)}    % круглые скобки, подгоняемые по размеру аргумента
\newcommand{\fbr}[1]{\left\{ #1  \right\}} % фигурные скобки, подгоняемые по размеру аргумента
\newcommand{\eqdef}{\mathrel{\stackrel{\rm def}=}} % знак равно по определению
\newcommand{\const}{\mathrm{const}}        % const прямым начертанием
\newcommand{\zdt}[1]{\textit{#1}}
\newcommand{\ENG}[1]{\foreignlanguage{british}{#1}}
\newcommand{\ENGs}{\selectlanguage{british}}
\newcommand{\RUSs}{\selectlanguage{russian}}
\newcommand{\iid}{\text{i.\hspace{1pt}i.\hspace{1pt}d.}}

\newdimen\theoremskip
\theoremskip=0pt
\newenvironment{note}{\par\vskip\theoremskip\textbf{Замечание.\xspace}}{\par\vskip\theoremskip}
\newenvironment{hint}{\par\vskip\theoremskip\textbf{Подсказка.\xspace}}{\par\vskip\theoremskip}
\newenvironment{ist}{\par\vskip\theoremskip Источник:\xspace}{\par\vskip\theoremskip}

\newcommand*{\tabvrulel}[1]{\multicolumn{1}{|c}{#1}}
\newcommand*{\tabvruler}[1]{\multicolumn{1}{c|}{#1}}

\newcommand{\II}{{\fontencoding{X2}\selectfont\CYRII}}   % I десятеричное (английская i неуместна)
\newcommand{\ii}{{\fontencoding{X2}\selectfont\cyrii}}   % i десятеричное
\newcommand{\EE}{{\fontencoding{X2}\selectfont\CYRYAT}}  % ЯТЬ
\newcommand{\ee}{{\fontencoding{X2}\selectfont\cyryat}}  % ять
\newcommand{\FF}{{\fontencoding{X2}\selectfont\CYROTLD}} % ФИТА
\newcommand{\ff}{{\fontencoding{X2}\selectfont\cyrotld}} % фита
\newcommand{\YY}{{\fontencoding{X2}\selectfont\CYRIZH}}  % ИЖИЦА
\newcommand{\yy}{{\fontencoding{X2}\selectfont\cyrizh}}  % ижица

%%%%%%%%%%%%%%%%%%%%% Определение разрядки разреженного текста и задание красивых многоточий
\sodef\so{}{.15em}{1em plus1em}{.3em plus.05em minus.05em}
\newcommand{\ldotst}{\so{...}}
\newcommand{\ldotsq}{\so{?\hbox{\hspace{-0.61ex}}..}}
\newcommand{\ldotse}{\so{!..}}
%%%%%%%%%%%%%%%%%%%%%%%%%%%%%%%%%%%%%%%%%%%%%%%%%%%%%%%%%%%%%%%%%%%%%%

%%%%%%%%%%%%%%%%%%%%%%%%%%%%% Команда для переноса символов бинарных операций
\def\hm#1{#1\nobreak\discretionary{}{\hbox{$#1$}}{}}
%%%%%%%%%%%%%%%%%%%%%%%%%%%%%%%%%%%%%%%%%%%%%%%%%%%%%%%%%%%%%%%%%%%%%%

\setlist[enumerate,1]{label=\arabic*., ref=\arabic*, partopsep=0pt plus 2pt, topsep=0pt plus 1.5pt,itemsep=0pt plus .5pt,parsep=0pt plus .5pt}
\setlist[itemize,1]{partopsep=0pt plus 2pt, topsep=0pt plus 1.5pt,itemsep=0pt plus .5pt,parsep=0pt plus .5pt}

% Эти парни затем, если вдруг не захочется управлять списками из-под уютненького enumitem
% или если будет жизненно важно, чтобы в списках были именно русские буквы.
%\setlength{\partopsep}{0pt plus 3pt}
%\setlength{\topsep}{0pt plus 2pt}
%\setlength{\itemsep}{0 plus 1pt}
%\setlength{\parsep}{0 plus 1pt}

%на всякий случай пока есть
%теоремы без нумерации и имени
%\newtheorem*{theor}{Теорема}

%"Определения","Замечания"
%и "Гипотезы" не нумеруются
%\newtheorem*{defin}{Определение}
%\newtheorem*{rem}{Замечание}
%\newtheorem*{conj}{Гипотеза}

%"Теоремы" и "Леммы" нумеруются
%по главам и согласованно м/у собой
%\newtheorem{theorem}{Теорема}
%\newtheorem{lemma}[theorem]{Лемма}

% Утверждения нумеруются по главам
% независимо от Лемм и Теорем
%\newtheorem{prop}{Утверждение}
%\newtheorem{cor}{Следствие} 

%\usepackage{showkeys} % показывать метки

% специальная штука под задачник
% создает команды:
% \problem{ текст задачи }
% \solution{ текст решения }
% \problemonly  - после этой команды будут печататься только \problem{} и \problemtext{}
% \solutiononly - после этой команды будут печататься только \solution{} и \solutiontext{}
% \problemandsolution - после этой команды печатается все
% \secsolution - задает новую (виртуальную) секцию для решений

% может потребоваться %\addtocounter{secsolution}{число глав без задач решений, не прогнанных через problemonly}

% как работать
% файл с решениями отдельной главы должен выглядеть так:
% \problem{ dddd} \solution{ddddddd}
% \problem{df sldk} \solution{ dfssd}

% главный файл может выглядеть двумя способами:

% Способ 1. (для решений контрольной, рядом задачи и ответы)
% \problemandsolution
% \input{file with problems}

% Способ 2. (для задачника, сначала все задачи, затем все ответы)
% \problemonly
% \input{file with problems}

% \solutiononly
% \input{file with problems}

% ААААААААААААААААААААААА надо делать!!!!
% Способ 3. - основной (вариант способа 2)

% \problemonly2
% \input{file with problems}

%\solutiononly - эта команда сама сделает все!




% файл с задачами:
% \section{Первая}
% \problem{ dddd} \solution{ddddddd}
% \problem{df sldk} \solution{ dfssd}
% \problemtext{Этот текст не будет напечатан после solutiononly}
% \section{Вторая}
% \problem{ ааа} \solution{dыва}
% \problem{ыавыв} \solution{ ыва}
% \solutiontext{Этот текст не будет напечатан после problemonly}



% начало кода:

\let\oldsection\section % сохраняем команду \section, т.к. мы ее переопределим
\let\oldsubsection\subsection % сохраняем команду \subsection, т.к. мы ее переопределим

\newcommand{\restoresection}{ % команда для восстановления \section \subsection
\renewcommand{\section}[1]{\oldsection{##1}}
\renewcommand{\subsection}[1]{\oldsubsection{##1}}
}

\newcounter{problem}[section]
%создаем новый счетчик "problem",
% будет автоматом сбрасываться на 0 при старте нового раздела
% при создании счетчик сам встанет на 0

\newcounter{secsolution}
% - это номер секции решаемой задачи (поскольку решения идут в одной секции, то номер секции надо менять в ручную)
\newcounter{solution}[secsolution]
% - это номер решаемой задачи, сам сбрасывается при увеличении secsolution на 1



\renewcommand{\thesecsolution}{\arabic{secsolution}}
% команда \thesecsolution просто выводит номер secsolution

\newcommand{\newsecsolution}{
\stepcounter{secsolution} % без создания ссылки увеличит secsolution на 1 со сбросом подчиненного счетчика
}
% команда \newsecsolution увеличит номер секции на 1 и установит номер решения внутри секции равным 0

\renewcommand{\theproblem}{\thesection.\arabic{problem}.}
\renewcommand{\thesolution}{\thesecsolution.\arabic{solution}.}
% обновляем команду \theproblem - она должна выводить номер секции и номер задачи внутри секции
% почему обновляем? - потому, что она создалась при создании счетчика problem

\newcommand{\problem}[1]{}
\newcommand{\solution}[1]{}
% создаем команды \problem, \solution с одним аргументом, которые ничего не делает
% ниже они будут переопределены


\newcommand{\problemtext}[1]{}
\newcommand{\solutiontext}[1]{}
% эти две команды будут выводить текст, заложенный внутри них только внутри соответствующей секции, в другой - ничего не будет делать
% в отличие от этой команды \problem \solution делают ссылки, слово "задача" и пр.


\newcommand{\problemonly}{
% эта команда переопределяет команду \problem

\setcounter{problem}{0}
\setcounter{solution}{0}
\setcounter{secsolution}{0}

\renewcommand{\problemtext}[1]{##1}
\renewcommand{\solutiontext}[1]{}


\renewcommand{\section}[1]{\oldsection{##1}\newsecsolution}
\renewcommand{\subsection}[1]{\oldsubsection{##1}}


\renewcommand{\problem}[1]{
\refstepcounter{problem}
% \phantomsection % создаем точку привязки для команды \label % не нужна, т.к. есть refstepcounter
\vspace{0.5ex plus 0.2ex minus 0.2ex}
Задача
\hyperref[s\theproblem]{\theproblem} % гиперссылка на метку "s1.1."
\label{p\theproblem} % метка "p1.1."
\par ##1}
\renewcommand{\solution}[1]{}
}

\newcommand{\solutiononly}{
% эта команда переопределяет команду \solution

\setcounter{problem}{0}
\setcounter{solution}{0}
\setcounter{secsolution}{0}

\renewcommand{\problemtext}[1]{}
\renewcommand{\solutiontext}[1]{##1}

\renewcommand{\section}[1]{\newsecsolution} % можно сюда чего-то добавить, чтобы решения отделялись как-то по секциям
\renewcommand{\subsection}[1]{}

\renewcommand{\problem}[1]{}
\renewcommand{\solution}[1]{
\refstepcounter{solution}
% \phantomsection
\hyperref[p\thesolution]{\thesolution} \label{s\thesolution}
##1}}



\newcommand{\problemandsolution}{
% эта команда переопределяет команды \solution, \problem

\setcounter{problem}{0}
\setcounter{solution}{0}
\setcounter{secsolution}{0}

\renewcommand{\problemtext}[1]{##1}
\renewcommand{\solutiontext}[1]{##1}

\renewcommand{\section}[1]{\oldsection{##1}\newsecsolution}
\renewcommand{\subsection}[1]{\oldsubsection{##1}}

\renewcommand{\problem}[1]{
\refstepcounter{problem}
% \phantomsection % создаем точку привязки для команды \label
Задача
\hyperref[s\theproblem]{\theproblem} % гиперссылка на метку "s1.1."
\label{p\theproblem} % метка "p1.1."
\par ##1}
\renewcommand{\solution}[1]{
\refstepcounter{solution}
% \phantomsection
\hyperref[p\thesolution]{\thesolution} \label{s\thesolution}
##1}
}




\title{Разные тексты по теории вероятностей}
\author{Борис Демешев, \url{boris.demeshev@gmail.com}}
\date{\today}


\newcommand{\indef}[1]{\textbf{#1}}


\numberwithin{equation}{page} % уравнения нумеруются на каждой стр. отдельно


\newtheorem{myth}[equation]{Теорема} % нумерация сквозная с уравнениями

\theoremstyle{definition} % убирает курсив и что-то еще наверное делает ;)
\newtheorem{mydef}[equation]{Определение}

\theoremstyle{definition}
\newtheorem{myex}[equation]{Пример}

%\newtheorem{assertion}{Утверждение}
%\newtheorem{lemma}{Лемма}

\theoremstyle{definition}
\newtheorem*{myproof}{Доказательство}

%\newtheorem{problem}{Задача}

\makeindex % команда для создания предметного указателя


\bibliographystyle{plain} % стиль оформления ссылок



\begin{document}

%\maketitle
%\tableofcontents{}


%\pagestyle{myheadings} \markboth{ТВИМС-задачник. Демешев Борис. roah@yandex.ru }{ТВИМС-задачник. Демешев Борис. roah@yandex.ru }
%\maketitle
%\tableofcontents{}

%\parindent=0 pt % отступ равен 0

%\part{Многомерное нормальное распределение}
%\subsubsection*{Векторные случайные величины}

Векторная случайная величина - это просто вектор-столбец из случайных величин:

\begin{equation}
\vec{X}=\left(\begin{array}{c} X_{1} \\ X_{2} \\ ... \\ X_{n} \\  \end{array} \right)
\end{equation}

В принципе можно говорить о векторе-строке или о случайной матрице, но нам понадобятся только вектор-столбцы. Две самых распространенных характеристики одномерной случайной величины $X$ - это среднее $\E(X)$ и дисперсия $\Var(X)$.
У векторных случайных величин также есть среднее:

\begin{equation}
\E(\vec{X})=\left(\begin{array}{c} \E(X_{1}) \\ \E(X_{2}) \\ ... \\ \E(X_{n}) \\  \end{array} \right)
\end{equation}



И ковариационная матрица:

\begin{equation}
\Var(\vec{X})=\left(
\begin{array}{cccc} 
\Var(X_{1}) & \Cov(X_{1},X_{2}) & ... & \Cov(X_{1},X_{n}) \\ 
\Cov(X_{2},X_{1}) & \Var(X_{2}) & ... & \Cov(X_{2},X_{n}) \\ 
... &&&\\ 
\Cov(X_{n},X_{1}) & \Cov(X_{n},X_{1})  & ... & \Var(X_{n})\\ 
\end{array} 
\right)
\end{equation}

В ковариационной матрицы в $i$-ой строке в $j$-ом столбце находится $\Cov(X_{i},X_{j})$. На диагонали $i$-ый элемент, это $\Cov(X_{i},X_{i})=\Var(X_{i})$. Из-за этого сразу следует, что ковариационная матрица симметрична.

Для простоты мы будем обозначать транспонирование с помощью штриха, вот так: $A'$. Путаницы с производной не возникнет.

С помощью транспонирования легко сказать, что ковариационная матрица симметрична: $\Var(X)'=\Var(X)$.

Конечно же, дисперсию можно записать через математическое ожидание: $\Var(X)=E[(X-\E(X))^{2}]=\E(X^{2})-\E(X)^{2}$

Для ковариационной матрицы это выглядит так:
\begin{equation}
\Var(\vec{X})=E[(X-\E(X))\cdot (X-\E(X))']=\E(XX')-\E(X)\cdot \E(X)'
\end{equation}

Упр. Убедитесь в этом.


Мы быстренько пройдемся по свойствам этих вещей:

Если $A$ - неслучайная матрица и $b$ - неслучайный вектор подходящих размеров, то:
\begin{enumerate}
\item $\E(A\vec{X}+b)=A\E(\vec{X})+b$
\item $\E(\vec{X}A+b)=\E(\vec{X})A+b$
\item $\Var(AX+b)=A\Var(X)A'$.
\item $\Var(X)$ - неотрицательно определена, имеет $n$ неотрицательных собственных чисел (с учетом кратности), где $n$ размерность вектора $X$.
\end{enumerate}

Проверка этих свойств - еще одно упражнение по линейной алгебре.


\subsubsection*{Краткая схема разных определений}

Есть три с половиной подхода определить многомерное нормальное распределение.

Вектор $\vec{X}$ имеет многомерное нормальное распределение, если выполнено одно из трех равносильных условий:
\begin{enumerate}
\item Вектор $\vec{X}$ представим в виде $\vec{X}=A\vec{Z}+\mu$, где $\vec{Z}$ --- это вектор независимых стандартных нормальных случайных величин
\item Любая линейная комбинация $c_{1}X_{1}+c_{2}X_{2}+...+c_{n}X_{n}$ имеет одномерное нормальное распределение
\item Характеристическая функция вектора $\vec{X}$ имеет вид:
\begin{equation}
\phi(\vec{u})=\exp\left(i\cdot \mu'\cdot u-\frac{1}{2}u'Vu \right)
\end{equation}
\end{enumerate}

В случае $det(V)\neq 0$ первые три формулировки становятся эквивалентны четвертой:

Вектор $\vec{X}$ имеет невырожденное многомерное нормальное распределение, если
\begin{enumerate}
\item Функция плотности вектора $\vec{X}$ имеет вид:
\begin{equation}
p(\vec{x})=(2\pi)^{-n/2}(\det(V))^{-1/2}\cdot \exp\left(-\frac{1}{2}(\vec{x}-\mu) V^{-1} (\vec{x}-\mu)' \right) 
\end{equation}
\end{enumerate}

Мы выбираем первое свойство как определение, а остальные будет доказывать как теоремы.


\subsubsection*{Определение и функция плотности}


Очень часто бывает удобно говорить о константе как о нормально распределенной случайной величине с нулевой дисперсией. Это просто соглашение. Никакой функции плотности у константы конечно же нет! Просто когда мы говорим $N(\mu,\sigma^{2})$, $\sigma>0$, мы имеем ввиду <<честное>> нормальное распределение c функцией плотности
\begin{equation}
p(x)=\frac{1}{\sqrt{2\pi\sigma^{2}}}\exp\left(-\frac{(x-\mu)^{2}}{2\sigma^{2}}\right)
\end{equation}
, а когда говорим $N(\mu,0)$ имеем ввиду просто константу $\mu$.

Т.е. говоря <<нормальное распределение>> мы теперь можем иметь ввиду и константу. Если же нам надо подчеркнуть, что речь идет о <<честном>> нормальном распределении с функцией плотности, то мы будем говорить <<невырожденное нормальное распределение>>.


Теперь мы можем дать несколько необычное определение одномерной нормальной величины... Пусть $a$ и $b$ - это константы.

\begin{mydef}
Если случайная величина $X$ представима в виде $X=aZ+b$, где $Z\sim N(0,1)$, то мы говорим, что $X$ имеет \indef{нормальное распределение} $N(b,a^{2})$. Если, кроме того, $a\neq 0$, то мы говорим, что $X$ имеет \indef{невырожденное нормальное распределение}.
\end{mydef}


Теперь мы готовы сказать, что такое многомерное нормальное распределение. Пусть матрица $A$ - неслучайная, вектор $b$ - неслучайный. Размерности подходящие.

\begin{mydef} \label{def:mult_norm}
Если $X=AZ+b$, где вектор $Z$ состоит из независимых $Z_{i}\sim N(0;1)$, то мы говорим, что вектор $X$ имеет \indef{нормальное распределение} $N(b,AA')$. Если, кроме того, $A$ обратимая матрица, $det(A)\neq 0$, то мы говорим, что $X$ имеет \indef{невырожденное нормальное распределение}.
\end{mydef}

В этой главе и в книге в целом мы \indef{постараемся} уточнять, но обычно из контекста понятно, включает ли автор в понятие нормального распределения вырожденные случаи. Как правило, да, включает.


\begin{myex}
Вектор $X$ имеет вырожденное нормальное распределение, у него нет двумерной функции плотности:

\begin{equation}
\left(\begin{array}{c} X_{1} \\ X_{2}  \end{array} \right)=
\left(\begin{array}{cc} 0 & 1 \\ 0 & 1  \end{array} \right)\cdot
\left(\begin{array}{c} Z_{1} \\ Z_{2}  \end{array} \right)=
\left(\begin{array}{c} Z_{1} \\ Z_{1}  \end{array} \right)
\end{equation}

\end{myex}



Невырожденный нормальный вектор имеет функцию плотности...
\begin{myth}
Вектор $X$ имеет невырожденное нормальное распределение $N(b,AA')$, если и только если его функция плотности имеет вид 
\begin{equation}
p(\vec{x})=(2\pi)^{-n/2}(\det(AA'))^{-1/2}\cdot \exp\left(-\frac{1}{2}(\vec{x}-b) (AA')^{-1} (\vec{x}-b)' \right) 
\end{equation}
Если не обращать внимания на константу, которая нужна чтобы интеграл под функцией плотности равнялся единице, то:
\begin{equation}
p(\vec{x})\sim \exp\left(-\frac{1}{2}(\vec{x}-b) (AA')^{-1} (\vec{x}-b)' \right) 
\end{equation}
\end{myth}

\begin{proof}


\end{proof}



\subsubsection*{Напоминалка про характеристические функции}


Опытный охотник увидев уши, торчащие из кустов, скажет: <<Ага, это заяц!>> Для этих же целей нужна и характеристическая функция. Она позволяет опознать закон распределения случайной величины. 

Узнав например, что характеристическая функция случайной величины $W$ равна $\phi(u)=\frac{\exp(iu)-1}{iu}$, опытный охотник скажет: <<Ага, это равномерная на $[0;1]$!>>. Увидев характеристическую функцию $\phi(u)=(p\cos(u)+ip\sin(u)+1-p)^{n}$, опытный охотник узнает биномиальное распределение $Bin(n,p)$.

Настала пора для формального определения:

\begin{mydef}
Характеристической функцией случайной величины $W$ называется функция $\phi:\mathbb{R}\to \mathbb{C}$:
\begin{equation}
\phi_{W}(u):=\E(\cos(uW))+i\E(\sin(uW))
\end{equation}
Или более кратко:
\begin{equation}
\phi_{W}(u):=\E(\exp(iuW))
\end{equation}
\end{mydef}

Косинус и синус - ограниченные функции, поэтому для любой случаной величины $W$ существуют $\E(\sin(uW))$ и $\E(\cos(uW))$. А значит и характеристическая функция всегда существует.
Тот факт, что по характеристической функции можно узнать закон распределения строго формулируется в виде теоремы:


\begin{myth}
Пусть $X_{1}$ и $X_{2}$ --- две случайной величины. Функции распределения $F_{1}(t)$ и $F_{2}(t)$ совпадают если и только если совпадают характеристические функции $\phi_{1}(u)$ и $\phi_{2}(u)$
\end{myth}









Для описания многомерных нормальных величин характеристическая функция оказывается более удобна, чем функция плотности. Дело в том, что у вырожденных нормальных векторов функции плотности нет, а характеристическая функция существует всегда.



\subsubsection*{Еще два эквивалентных определения}


\begin{myth}
Вектор $X$ имеет нормальное распределение $N(b,AA')$, возможно вырожденное, если и только если его характеристическая функция имеет вид 
\begin{equation}
\phi(\vec{u})=\exp\left(i\cdot \mu'\cdot u-\frac{1}{2}u'Vu \right)
\end{equation}

\end{myth}



Важная особенность многомерного нормального распределения.

\begin{myth} \label{th:normal_one2mult}
Вектор $X=(X_{1},...,X_{n})$ имеет многомерное нормальное распределение (возможно вырожденное) если и только если любая линейная комбинация $Y=c_{1}X_{1}+c_{2}X_{2}+...+c_{n}X_{n}$ имеет одномерное нормальное распределение (возможно вырожденное).
\end{myth}

\begin{proof}
Если $X$ - многомерное нормальное, то $X=AZ+b$. Следовательно, $Y=c_{1}X_{1}+c_{2}X_{2}+...+c_{n}X_{n}=\vec{c}X=\vec{c}AZ+\vec{c}b$. Значит $Y$ по определению \ref{def:mult_norm} имеет нормальное распределение.

Если любая линейная комбинация нормальна...

\end{proof}






\subsubsection*{Стандартизация}

В одномерном случае, если $\Var(X)\neq 0$, то случайную величину $X$ можно <<стандартизировать>>, т.е. превратить в случайную величину с нулевым средним и единичной дисперсией: 

\begin{equation}
Z:=\frac{X-\E(X)}{\sqrt{\Var(X)}}
\end{equation}

У случайной величины $Z$: $\E(Z)=0$, $\Var(Z)=1$.

В многомерном случае, если матрица $\Var(X)$ обратима, то случайный вектор $\vec{X}$ можно <<стандартизовать>>, т.е. превратить в вектор некоррелированных случайных величин с нулевым средним и единичной дисперсией.

\begin{equation}
\vec{Z}:=(\Var(X))^{-1/2}(\vec{X}-\E(\vec{X}))
\end{equation}

Для этой операции нужно понимать, что такое $A^{\frac{1}{2}}$. Для положительно определенной матрицы эта операция корректна. Если $A$ положительно определена, то у нее есть разложение $A=PDP^{-1}$ причем $D$ - диагональная матрица, где на главной диагонали стоят положительные собственные значения $A$. И, стало быть, $A^{\frac{1}{2}}=PD^{\frac{1}{2}}P^{-1}$.

Если нужно решить какую-то задачу, связанную с многомерным нормальным распределением, то стандартизация может здорово облегчить вычисления. Даже в вырожденном случае имеет смысл перейти к рассмотрению независимых нормальных $N(0;1)$ случайных величин.
\begin{myex}
Пусть рост любой женщины имеет распределение $N(165,25)$. А корреляция между ростом матери и ростом дочери равна $\rho$. Какова вероятность того, что дочь высокая (выше среднего роста), если мама - высокая?

Начинаем решать:

Обозначим $X_{1}$ - рост матери, $X_{2}$ - рост дочери. $\P(X_{2}>165|X_{1}>165)=\frac{\P(X_{1}>165\cap X_{2}>165)}{\P(X_{1}>165)}=2\P(X_{1}>165\cap X_{2}>165)$.

Если решать без стандартизации, <<в лоб>>. Вылезает интеграл, который берется усердным трудом...

Если перейти к стандартным $Z_{1}$ и $Z_{2}$...


%\left(\begin{array}{c} X_{1} \\ X_{2} \end{array} \right)\sim 

\end{myex}


\subsubsection*{Многомерное --- это больше чем несколько одномерных}

Из теоремы \ref{th:normal_one2mult} следует в частности, что:

\begin{myth}
Если вектор $\vec{X}$ имеет многомерное нормальное распределение, то и любая его компонента $X_{i}$ имеет нормальное распределение. Кроме того, если распределение $\vec{X}$ --- невырожденное, то и распределение каждой компоненты $X_{i}$ невырожденное.
\end{myth}

Обратное утверждение неверно. Мы приведем пример, в котором $X_{1}$ и $X_{2}$ имеют нормальное распределение по отдельности, но не имеют совместного нормального распределения.

\begin{myex}
Пусть $X_{1}\sim N(0;1)$, а $K$ --- это случайная величина равновероятно принимающая значения $1$ и $-1$, причем $K$ и $X_{1}$ независимы. Определим
\begin{equation}
X_{2}=K\cdot |X_{1}|
\end{equation}
Проверим, что $X_{2}$ имеет нормальное распределение. Для положительных $x$:
\begin{multline}
\P(X_{2}\leq x)=\P(K=-1)+\P(K=1)\cdot \P(|X_{1}|\leq x)=\\
=0.5+0.5\P(|X_{1}|\leq x)=0.5+\P(X_{1}\in [0;x])=\P(X_{1}\leq x)=F(x)
\end{multline}
Для отрицательных $x$ проверьте сами!

Можно заметить, что $X_{1}+X_{2}$ не является нормально распределенной случайной величиной. Более того, распределение $X_{1}+X_{2}$ не является непрерывным. Действительно,
\begin{equation}
\P(X_{1}+X_{2}=0)=\P(X_{1}+K|X_{1}|=0)=\P(K\neq \sgn(X_{1}))=0.5
\end{equation}
Если бы вектор $(X_{1}, X_{2})$ имел совместное нормальное распределение, то сумма $X_{1}+X_{2}$ была бы нормально распределенной.
Кстати, найдем ковариацию между $X_{1}$ и $X_{2}$:
\begin{multline}
\Cov(X_{1},X_{2})=\E(X_{1}X_{2})-\E(X_{1})\E(X_{2})=\E(X_{1}X_{2})=\E(X_{1}\cdot K\cdot|X_{1}|)=\\
=\E(K)\E(X_{1}\cdot |X_{1}|)=0
\end{multline}
Значит наши $X_{1}$ и $X_{2}$ были некоррелированны, нормальны по отдельности. Вместе с тем они не были нормальны в совокупности. И конечно, они зависимы, т.к. $|X_{1}|=|X_{2}|$.

\end{myex}


\subsubsection*{Некоррелированность и независимость}
% включена в абзац про связь одномерного и многомерного

Для многомерного нормального распределения некоррелированность равносильна независимости.

Доказательство для двумерного...





\subsubsection*{Условное распределение}

Предположим, что вектор $(X,Y)$ имеет совместное нормальное распределение. Часто возникает задача прогнозирования случайной величины $Y$, если значение случайной величины $X$ известно.

%Для начала мы отметим, что при известном $X$ условное распределение $Y$ также будет нормальным.

Для удобства вычислений всегда используем стандартизацию! Вместо исходных $X$ и $Y$ рассмотрим:
\begin{equation}
Z_{1}=\frac{X-\E(X)}{\sigma_{X}}, \quad Z_{2}=\frac{Y-\E(Y)}{\sigma_{Y}}
\end{equation}
Естественно, $Z_{1}\sim N(0;1)$ и $Z_{2}\sim N(0;1)$ и $\Corr(Z_{1},Z_{2})=\Corr(X,Y)$.

Найдем условную функцию плотности $Z_{2}$ при известном $Z_{1}$:
\begin{equation}
p(z_{2}|z_{1})=\frac{p(z_{1},z_{2})}{p(z_{1})}
\end{equation}

Поскольку нас интересует только зависимость от $z_{2}$, то на $p(z_{1})$ можно не обращать внимания. Значком $\sim$ мы будем обозначать равенство с точностью до константы не зависящей от $z_{2}$:
\begin{multline}
p(z_{2}|z_{1})\sim p(z_{1},z_{2})\sim \exp\left(-\frac{1}{2}(z_{1},z_{2})\left(\begin{matrix}
1 & \rho \\ 
\rho & 1
\end{matrix}\right)^{-1}\left( \begin{matrix}
z_{1} \\ 
z_{2}
\end{matrix}\right)   \right)=\\
=\exp\left(-\frac{1}{2(1-\rho^{2})}(z_{1},z_{2})\left(\begin{matrix}
1 & -\rho \\ 
-\rho & 1
\end{matrix}\right)\left( \begin{matrix}
z_{1} \\ 
z_{2}
\end{matrix}\right)   \right)= \\
=\exp\left(-\frac{1}{2(1-\rho^{2})}\left(z_{2}^{2}-2\rho z_{1}z_{2}+z_{1}^{2}\right) \right)\sim 
\exp\left(-\frac{1}{2(1-\rho^{2})}\left(z_{2}-\rho z_{1}\right)^{2}\right)
\end{multline}

Сравним полученный результат с функцией плотности одномерного нормального распределения:
\begin{equation}
p(x)\sim\exp\left(-\frac{(x-\mu)^{2}}{2\sigma^{2}}\right)
\end{equation}

Замечаем, что $\sigma^{2}=1-\rho^{2}$ и $\mu=\rho z_{2}$. И оформляем результат вычислений в виде теоремы:
\begin{myth}
Если $Z_{1}$ и $Z_{2}$ имеют совместное нормальное распределение, $\E(Z_{i})=0$, $\Var(Z_{i})=1$ и $\Corr(Z_{1},Z_{2})=\rho$, то условное распределение $Z_{1}$ при известном $Z_{2}$ является нормальным и:
\begin{equation}
\E(Z_{2}|Z_{1}=z)=\rho\cdot z
\end{equation}
\begin{equation}
\Var(Z_{2}|Z_{1}=z)=1-\rho^{2}
\end{equation}
\end{myth}

Формулы дают нам смысл:
\begin{enumerate}
\item Корреляция стандартизированных нормальных величин показывает на сколько в среднем растет одна случайная величина при росте другой на единицу.
\item Условная дисперсия $Z_{2}$ не зависит от конкретного значения $Z_{1}$. Это несколько неожиданно, но интуитивного объяснения я не знаю.
\end{enumerate}
 

Для возврата к исходным $X$ и $Y$ замечаем, что условие $X=x$ равносильно тому, что $Z_{1}=\frac{x-\mu_{x}}{\sigma_{x}}$ и, кроме того, $Y=\mu_{y}+\sigma_{y}Z_{2}$.

Сделав обратную замену, получаем:
\begin{myth}
Если $X$ и $Y$ имеют совместное нормальное распределение, $X\sim N(\mu_{x},\sigma^{2}_{x})$, $Y\sim N(\mu_{y},\sigma^{2}_{y})$ и $\Corr(X,Y)=\rho$, то условное распределение $Y$ при известном $X$ является нормальным и:
\begin{equation}
\E(Y|X=x)=\mu_{y}+\rho\sigma_{y}\frac{x-\mu_{x}}{\sigma_{x}}
\end{equation}
\begin{equation}
\Var(Y|X=x)=(1-\rho^{2})\sigma_{y}^{2}
\end{equation}
\end{myth}

Получаем трактовку корреляции: Если $X$ и $Y$ имеют совместное нормальное распределение, то корреляция  показывает на сколько своих стандартных отклонений в среднем растет $Y$  при росте $X$ на одно свое стандартное отклонение. 


\subsubsection*{Геометрический смысл ковариационной матрицы}


\begin{mydef}
Матрица $R$ называется \indef{матрицей поворота} если одновременно выполнены два условия:
$R'R=I$ и $\det(R)=1$
\end{mydef}

Почему определение поворота именно такое?

Условие $RR'=I$ означает две вещи:
\begin{enumerate}
\item Вектор $R\vec{x}$ имеет такую же длину, как и вектор $\vec{x}$:

\begin{equation}
|\vec{x}|^{2}=\vec{x}'\vec{x}=\vec{x}'R'R\vec{x}=(R\vec{x})'(R\vec{x})=|R\vec{x}|^{2}
\end{equation}

\item Угол между $\vec{x}$ и $\vec{y}$ равен углу между $R\vec{x}$ и $R\vec{y}$
\begin{equation}
\cos(\vec{x},\vec{y})=\frac{\vec{x}'\vec{y}}{|\vec{x}||\vec{y}|}=
\frac{\vec{x}'R'R\vec{y}}{|R\vec{x}||R\vec{y}|}=
\frac{(R\vec{x})'(R\vec{y})}{|R\vec{x}||R\vec{y}|}=\cos(R\vec{x},R\vec{y})
\end{equation}
\end{enumerate}

Условию $R'R=I$ подходят матрицы с определителем $\det(A)=\pm 1$. Дополнительное  условие $\det(R)=1$ исключает <<отражения>>.


Упражнение. Мы доказали, что матрицы вида $R'R=I$ сохраняют углы и длины. Докажите, что никакие другие матрицы не сохраняют одновременно углы и длины.

Решение. Рассмотрим вектор $e_{k}=(0,0,\ldots,0,1,0,\ldots,0)'$. 

Из линейной алгебры:
\begin{myth}
Если $A_{n\times n}$ действительная симметричная положительно полу-определенная матрица, то
\begin{enumerate}
\item У $A$ имеется ровно $n$ действительных собственных чисел
\item $A$ представима в виде 
\begin{equation}
A=RDR^{-1}=RDR'
\end{equation}, 
где $D$ --- диагональная матрица из собственных чисел матрицы $A$, а $R$ --- матрица поворота из собственных векторов матрицы $A$.
\end{enumerate}
\end{myth}


Для начала сформулируем геометрические факты:
\begin{mydef}
Множество точек называется \indef{эллипсоидом}, если оно задается системой уравнений
\begin{equation}
(\vec{x}-\vec{x}_{0})' A (\vec{x}-\vec{x}_{0})=1
\end{equation}
, где $A$ --- положительно определенная матрица. Точка $\vec{x}_{0}$ --- центр эллипсоида. В двумерном случае эллипсоид называют \indef{эллипсом}.
\end{mydef}


Почему определение эллипса именно такое?

Матрицу $A$ можно представить в виде $A=R'DR$, где $R$ --- матрица поворота, а $D$ --- диагональная матрица собственных чисел матрицы $A$. Отсюда получаем, что уравнение эллипса можно записать в виде:
\begin{equation}
(R(\vec{x}-\vec{x}_{0}))' D (R(\vec{x}-\vec{x}_{0}))=1
\end{equation}
Если ввести обозначения $\vec{z}=R(\vec{x}-\vec{x}_{0})$, то уравнение примет вид:
\begin{equation}
\sum_{i=1}^{n} d_{ii}z_{i}^{2}=1
\end{equation}

Именно в силу этого представления:
\begin{mydef}
Для эллипса $(\vec{x}-\vec{x}_{0})' A (\vec{x}-\vec{x}_{0})=1$ собственные векторы матрицы $A$ называют \indef{направлениями полуосей}. Если $\lambda_{i}$ --- это собственное число матрицы $A$, то величины $1/\sqrt{\lambda_{i}}$ называют \indef{длинами полуосей}.
\end{mydef}


Теперь мы готовы нарисовать линии уровня многомерного нормального распределения!
\begin{myth}
Для невырожденного нормального распределения линии уровня функции плотности являются эллипсоидами. Направления главных осей задаются собственными векторами ковариационной матрицы. Соотношение длин полуосей равно соотношению корней из собственных чисел ковариационной матрицы.
\end{myth}


\begin{proof}

Для доказательства нам потребуется технический факт из линейной алгебры:
\begin{myth}
Eсли $\vec{a}$ собственный вектор для матрицы $V$ с собственным числом $\lambda$, то $\vec{a}$ собственный вектор для матрицы $V^{-1}$ с собственным числом $1/\lambda$.
\end{myth}
\begin{proof}
Если $\vec{a}$ собственный вектор матрицы $V$, то с одной стороны:
\begin{equation}
V^{-1}\cdot (V\cdot \vec{a})=V^{-1} (\lambda \vec{a})=\lambda V^{-1}\vec{a}
\end{equation}

С другой стороны:
\begin{equation}
(V^{-1}\cdot V)\cdot \vec{a}=\vec{a}
\end{equation}

Т.е. $\lambda V^{-1}\vec{a}=\vec{a}$. Или:
\begin{equation}
V^{-1}\vec{a}=\frac{1}{\lambda}\vec{a}
\end{equation}

\end{proof}


Итак, пусть $X\sim N(\vec{\mu};V)$.

Тогда условие
\begin{equation}
p(\vec{x})=const
\end{equation}
после преобразований равносильно тому, что:
\begin{equation}
(\vec{x}-\vec{\mu})\cdot V^{-1}\cdot (\vec{x}-\vec{\mu})'=const
\end{equation}

Т.е. мы получили наше определение эллипсоида с $A=\frac{1}{const}V^{-1}$.

Направления полуосей задаются собственными векторами $A$, значит они совпадают с собственными векторами $V$. 

Длины полуосей обратно пропорциональны корням из собственных чисел $A$, значит они прямо пропорциональны корням из собственных чисел $V$.
\end{proof}


\begin{myex}
Пусть $X\sim N(0;V)$ и 
\begin{equation}
V=
\left(
\begin{array}{cc}
5 & 1 \\ 
1 & 9
\end{array} 
\right)
\end{equation}

Нарисуйте линии уровня функции плотности $p(x_{1},x_{2})$


Решение.

Тут обязательно картинки. Эллипс. Главные оси. С кодом R и Sage!

\end{myex}


В случае независимых нормальных случайных величин с одинаковой дисперсией линиями уровня будут окружности (сферы при более высоких размерностях). Поскольку поворот никак не влияет на линию уровня мы бесплатно получаем следующую теорему:

\begin{myth} \label{th:rotate_normal}
Если матрица $R$ --- это матрица поворота и вектор $\vec{Z}\sim N(\vec{0},I)$ то вектор $R\vec{Z}\sim N(\vec{0},I)$
\end{myth}



\subsubsection*{Хи-квадрат распределение}

Как известно,

\begin{mydef} Если случайная величина $W$ представима в виде $W=\sum_{i=1}^{k}Z_{i}^{2}$, где $Z_{i}$ --- независимые стандартные нормальные случайные величины, то говорят, что $W$ имеет \indef{хи-квадрат распределение c $k$ степенями свободы}.
\end{mydef}


К сожалению, проверять, что что-то имеет хи-квадрат распределение напрямую очень неудобно.

\begin{myex}
Давайте попробуем по определению показать, что если $X_{i}\sim N(\mu;\sigma^{2})$ и независимы, то 
\begin{equation}
\frac{1}{\sigma^{2}}\sum_{i=1}^{n}(X_{i}-\bar{X}_{n})^{2}\sim \chi_{n-1}^{2}
\end{equation}

Как всегда, сначала стандартизируем наши $X_{i}$. 



...


\end{myex}


Гораздо более удобным оказывается следующий способ:
\begin{myth}
Если вектор $\vec{Z}$ состоит из независимых стандартных нормальных случайных величин и все собственные числа симметричной матрицы $A$ равны либо нулю, либо единице, то 
\begin{equation}
\vec{Z}'A\vec{Z}\sim \chi_{r}^{2}
\end{equation}
где $r$ --- количество собственных чисел матрицы $A$ равных единице.
\end{myth}

\begin{proof}
Матрица $A$ представима в виде $A=R'DR$. Поэтому:
\begin{equation}
\vec{Z}'A\vec{Z}=\vec{Z}'R'DR\vec{Z}=(R\vec{Z})'D(R\vec{Z})
\end{equation}

Остается заметить, что $R\vec{Z}$ --- это вектор независимых стандартных нормальных случайных величин в силу теоремы \ref{th:rotate_normal}. Если обозначить этот вектор буквой $\vec{W}=R\vec{Z}$, то
\begin{equation}
\vec{Z}'A\vec{Z}=\vec{W}'D\vec{W}=\sum_{i=1}^{n}d_{ii}W_{i}^{2}
\end{equation}

Матрица $D$ --- это матрица собственных чисел матрицы $A$, т.е. $d_{ii}$ равны либо нулю, либо единице. Получается, что $\vec{Z}'A\vec{Z}$ --- это сумма $r$ стандартных независимых нормальных случайных величин.
\end{proof}


\begin{myth}
Собственные числа симметричной матрицы $A$ равны либо нулю, либо единице, если и только если  $A^{2}=A$
\end{myth}

\begin{proof}
С одной стороны:
\begin{equation}
A^{2}=(R'DR)(R'DR)=R'D(RR')DR=R'DDR=R'D^{2}R
\end{equation}
С другой стороны $A=R'DR$. Значит $A^{2}=A$ если и только если $D^{2}=D$. Но $D$ --- диагональная матрица, поэтому условие $D^{2}=D$ равносильно тому, что на диагонали стоят либо нолики, либо единички.
\end{proof}


\begin{myex}
Тот же пример только быстрее...
...

\end{myex}


\subsubsection*{Границы на хвостовые вероятности}
При изучении броуновского движения полезны два неравенства.

\begin{myth} Если $Z\sim N(0;1)$, то:
\begin{equation}
\frac{1}{\sqrt{2\pi}}\frac{\exp(-x^{2}/2)}{x+x^{-1}}\leq \P(Z\geq x)\leq \frac{1}{\sqrt{2\pi}}\frac{\exp(-x^{2}/2)}{x}, \quad x>0
\end{equation}
\end{myth}

\begin{proof} Заметим, что $\exp(-t^{2}/2)'=-t\exp(-t^{2}/2)$. Получаем верхнюю границу:
\begin{multline}
\P(Z\geq x)=\int_{x}^{\infty} \frac{1}{\sqrt{2\pi}}\exp(-t^{2}/2)dt\leq \\
\leq \int_{x}^{\infty} \left(\frac{t}{x} \right)\frac{1}{\sqrt{2\pi}}\exp(-t^{2}/2)dt=
\frac{1}{\sqrt{2\pi}}\frac{\exp(-x^{2}/2)}{x}
\end{multline}
Заметим, что $(t^{-1}\exp(-t^{2}/2))'=(1+t^{-2})\exp(-t^{2}/2)$. Получаем нижнюю границу:
\begin{multline}
\P(Z\geq x)=\int_{x}^{\infty} \frac{1}{\sqrt{2\pi}}\exp(-t^{2}/2)dt
\geq \int_{x}^{\infty} \left(\frac{1+t^{-2}}{1+x^{-2}} \right)\frac{1}{\sqrt{2\pi}}\exp(-t^{2}/2)dt= \\
=\frac{1}{\sqrt{2\pi}}\frac{1}{1+x^{-2}}\frac{\exp(-x^{2}/2)}{x}=
\frac{1}{\sqrt{2\pi}}\frac{\exp(-x^{2}/2)}{x+x^{-1}}
\end{multline}

\end{proof}

\begin{myth} Если $Z\sim N(0;1)$, то:
\begin{equation}
\frac{2x}{\sqrt{2\pi e}}\leq \P(|Z|\leq x)\leq \frac{2x}{\sqrt{2\pi}}, \quad 0<x\leq 1
\end{equation}
\end{myth}

\begin{proof}
Если $|t|\leq 1$, то
\begin{equation}
\frac{1}{\sqrt{2\pi e}}\leq \frac{1}{\sqrt{2\pi}}\exp(-t^{2}/2)\leq \frac{1}{\sqrt{2\pi}}
\end{equation}
Интегрируя это неравенство от $-x$ до $x$ получаем требуемое.
\end{proof}


Еще кусок из блога... (ссылка?) ...

\subsubsection*{Откуда взялось $\pi$ в формуле?}

Есть много способов объяснить, откуда берется $\pi$ в формуле функции плотности. Вот то, которое нравится мне.

Рассмотрим пару независимых стандартных нормальных случайных величин $X$ и $Y$. Их функция плотности имеет вид:
\begin{equation}
p(x,y)=c\cdot \exp\left( -\frac{1}{2} (x^{2}+y^{2})\right)
\end{equation}

Рассмотрим более подробно функцию
\begin{equation}
f(x,y)=\exp\left( -\frac{1}{2} (x^{2}+y^{2})\right)
\end{equation}

Значение $f(x,y)$ зависит только от расстояния от точки $(x,y)$ до начала координат. Значит объем под <<шляпой>> является фигурой вращения.

Картинка (слева и справа повернутая):

...

Уже понятно, что $\pi$ --- в деле. Остается вспомнить, что объем фигуры вращения определяется по формуле $Vol=\int_{a}^{b}\pi r^{2}(t)dt$.

В нашем случае: $a=0$, $b=f(0,0)=1$, а $r(t)$ находится из условия:
\begin{equation}
\exp\left( -\frac{1}{2} r^{2}(t)\right)=t
\end{equation}

Находим $r^{2}(t)$ и получаем:
\begin{equation}
r^{2}(t)=-2\ln(t)
\end{equation}

Находим объем фигуры вращения:
\begin{equation}
Vol=\int_{0}^{1} \pi r^{2}(t)dt=\int_{0}^{1} \pi (-2\ln(t))dt=-2\pi\int_{0}^{1}\ln(t)dt=2\pi 
\end{equation}

Интеграл от $\ln(t)$ можно взять либо по частям, либо заметив, что:
\begin{equation}
\int_{0}^{1}\ln(t)dt=-\int_{0}^{\infty}\exp(-t)dt
\end{equation}
Картинка:


Настоящие знатоки берут интеграл $\int_{0}^{\infty}\exp(-t)dt$ по методу Мамикона Мнацаканяна \cite{apostol:visual_calculus} в уме глядя на картинку:



Пояснение к картинке:

Производная функции $\exp(-x)$ равна ей самой умноженной на минус один. Поэтому тень от касательной всегда имеет длину один. Интересующая нас площадь равна площади треугольника плюс оставшаяся площадь. Касательные заметающие оставшуюся площадь можно перенести так, чтобы они замели треугольник. Значит интеграл равен удвоенной площади треугольника.

 
Мы доказали, что объем под функцией $f(x,y)$ равен $2\pi$. Следовательно, функция плотности $p(x,y)$ должна иметь вид:
\begin{equation}
p(x,y)=\frac{1}{2\pi}\cdot \exp\left( -\frac{1}{2} (x^{2}+y^{2})\right)
\end{equation}

Для независимых величин $p(x,y)=p(x)p(y)$, следовательно:
\begin{equation}
p(x)=\frac{1}{\sqrt{2\pi}}\cdot \exp\left( -\frac{1}{2} x^{2}\right)
\end{equation}



\subsubsection*{ЦПТ --- доказательство через характеристические функции}

\begin{myth}
Пусть $X_{n}$ --- последовательность случайных величин с характеристическими функциями $\phi_{n}(u)$. Пусть кроме того, $X$ --- случайная величина с характеристической функцией $\phi(u)$. Последовательность $X_{n}$ сходится по распределению к случайное величине $X$ если и только если последовательность функций $\phi_{n}(u)$ сходится поточечно к функции $\phi(u)$.
\end{myth}




%\part{Теорема Берри-Эссена}
%На бытовом языке центральная предельная теорема формулируется так:

При большом $n$ распределение $\bar{X}_{n}$ похоже на нормальное.

Возникает естественный вопрос - {}``А большое $n$ - это сколько?''.
Кто-то говорит 30, кто-то 60... Пора покончить с этим безобразием!

Ответ даёт теорема Берри-Эссена (Berry-Essen):

Если $X_{i}$ независимы и одинаково распределены, $Z_{n}=\frac{\bar{X}_{n}-\mu}{\sqrt{\frac{\sigma^{2}}{n}}}$,
$Z\sim N(0;1)$, то: 

\begin{equation}
|\P(Z_{n}\leq t)-\P(Z\leq t)|\leq c\cdot\frac{\E(|X_{i}-\mu|^{3})}{\sigma^{3}\sqrt{n}}
\end{equation}


На момент создания этих заметок (март 2011) про константу $c$ известно
, что она лежит где-то в интервале $[0,4097;0,4784]$. 

Сама центральная предельная теорема утверждает только то, что:

\[
\P(Z_{n}\leq t)\underset{n\to\infty}{\longrightarrow}\P(Z\leq t)
\]


Поскольку 

\begin{multline}
|\P(Z_{n}\in[a;b])-\P(Z\in[a;b])|=|\P(Z_{n}\leq b)-\P(Z_{n}\leq a)-(\P(Z\leq b)-\P(Z\leq a))|=\\
=|\P(Z_{n}\leq b)-\P(Z\leq b)+\P(Z\leq a)-\P(Z_{n}\leq a)|\leq\\
\leq|\P(Z_{n}\leq b)-\P(Z\leq b)|+|\P(Z\leq a)-\P(Z_{n}\leq a)|\leq2c\cdot\frac{\E(|X_{i}-\mu|^{3})}{\sigma^{3}\sqrt{n}}
\end{multline}


Для простоты можно завысить $c$ и считать его равным $0,5$. Тогда
мы получаем:

\begin{equation}
|\P(Z_{n}\in[a;b])-\P(Z\in[a;b])|\leq\frac{\E(|X_{i}-\mu|^{3})}{\sigma^{3}\sqrt{n}}\label{eq:ab_error}
\end{equation}


Давайте применим эту теорему к биномиальному распределению:

\begin{tabular}{|c|c|c|}
\hline 
$X_{i}$ & 0 & 1\tabularnewline
\hline 
\hline 
Prob & $1-p$ & $p$\tabularnewline
\hline 
\end{tabular}

В этом случае:$\E(X_{i})=p$, $\Var(X_{i})=p(1-p)$, $\sigma=\sqrt{p(1-p)}$
и $\E(|X_{i}-p|^{3})=p(1-p)(p^{2}+(1-p)^{2})$:

\begin{equation}
\frac{\E(|X_{i}-\mu|^{3})}{\sigma^{3}\sqrt{n}}=\frac{p(1-p)(p^{2}+(1-p)^{2})}{(p(1-p))^{3/2}\sqrt{n}}=\frac{p^{2}+(1-p)^{2}}{\sqrt{p(1-p)n}}
\end{equation}


Для наглядности несколько цифр%
\footnote{Погрешность посчитана по формуле \ref{eq:ab_error}, т.е. при завышенном
$c$. Фактическая погрешность может быть гораздо меньше.%
}:

\begin{tabular}{|c|c|c|}
\hline 
$n$ & $p$ & Погрешность при оцеке $\P(\bar{X}_{n}\in[a;b])$\tabularnewline
\hline 
\hline 
50 & 0.5 & \tabularnewline
\hline 
100 & 0.5 & \tabularnewline
\hline 
500 & 0.1 & \tabularnewline
\hline 
1000 & 0.1 & \tabularnewline
\hline 
\end{tabular}

Аналогичный вопрос возникает при замене биномиального распределения
на распределение Пуассона. В этом случае аналогичная теорема имеет
вид:

Если $X\sim Bin(n,p)$ и $Y\sim Poisson(\lambda=np)$, то:

$|\P(X\in A)-\P(Y\in A)|\leq$

Кстати говоря, на пуассоновское можно заменять не только биномиальное
распределение, но и другие похожие

...

Доказательства для любопытных...

Упражнения



%\part{Две геометрии случайных величин}

\section{Два способа задать геометрию}

Множество случайных величин - это векторное (линейное) пространство. Если сложить две случайных величины, то получится случайная величина, если умножить случайную величину на число получится случайная величина. Поэтому случайные величины - это векторы.

Чтобы задать геометрию достаточно определить скалярное произведение, то есть действие $<X,Y>$, которое любой паре случайных величин ставит в соответствие число. При этом должны выполняться свойства:

1. $<X,Y>=<Y,X>$

2. $<X+Y,Z>=<X,Z>+<Y,Z>$

3. $<X,X>\geq 0$

4. $<X,X>=0$ только если $X=0$.

Почему скалярное произведение определяет геометрию? Геометрия - это же про длины, углы и расстояния! Все это восстанавливается из скалярного произведения по формулам 9-го класса. Длину любого вектора теперь можно найти по формуле $||X||=\sqrt{<X,X>}$.
Для того, чтобы найти угол между векторами достаточно знать косинус этого угла. А косинус определяется как: $cos(X,Y)=\frac{<X,Y>}{||X||||Y||}$. Расстояние определяется как длина разности: $d(X,Y)=||X-Y||$.

Мы используем двойные палочки для длины чтобы отличать ее от модуля: модуль случайной величины - это случайная величина, а длина - это константа.

Можно предложить много разных геометрий (или, что то же самое, скалярных произведений) в пространстве случайных величин, но интересными, пожалуй, являются две: $<X,Y>=\E(XY)$ и $<X,Y>=\Cov(X,Y)$.

Геометрия позволит <<увидеть>> некоторые понятия и теоремы. Чаще всего мы будет сталкиваться с теоремой Пифагора, настолько часто, что можно быть уверенным: если что-то неотрицательное равно сумме двух неотрицательных частей, то это теорема Пифагора и можно ее проиллюстрировать.

\section{Геометрия ожидаемого произведения}
Пусть скалярное произведение задано $<X,Y>=\E(XY)$.

Легко убедиться, что первые три требования к скалярному произведению выполнены:

1. $\E(XY)=\E(YX)$

2. $\E((X+Y)Z)=\E(XZ)+\E(YZ)$

3. $\E(X^2)\geq 0$

Четвертое требование выполнено не совсем полностью. Оказывается $\E(X^2)$ может быть равно нулю, даже если $X$ не всегда ноль. Например, пусть $Y$ равномерно на $[0;1]$, а $X$ равен 1, если $Y=0.5$ и 0 иначе. В данном примере $X$ может равняться одному, но  $\E(X^{2})=0$. Вызвано это тем, что $P(X=0)=1$. То есть в этой геометрии нулевой считается случайная величина, которая равна нулю с вероятностью один. 

Если вероятность события $A$ равна 1, то говорят, что $A$ происходит почти наверное. В геометрии ожидаемого произведения случайные величины почти наверное равные нулю не отличимы от настоящего нуля. И, следовательно, если $X=Y$ почти наверное, то в этой геометрии $X$ не отличим от $Y$. Действительно, в этом случае $X-Y=0$ почти наверное, то есть $X-Y$ не отличима от нуля.

Что нам дает введение геометрии?

Теперь вполне серьезно можно говорить о длине случайное величины $||X||=\sqrt{\E(X^{2})}$ или об угле между двумя случайными величинами $\angle(X,Y)=arccos(\frac{\E(XY)}{\sqrt{\E(X^{2})\E(Y^{2})}})$. А что из этого?

Автоматически возникает понятие перпендикулярных (ортогональных) векторов: случайные величины перпендикулярны, если угол между ними равен $\frac{\pi}{2}$. Или, если косинус угла между ними равен нулю. Или, $\E(XY)=0$.

Теорему Пифагора никто не отменял: если $X\bot Y$, то $||X-Y||^{2}=||X||^{2}+||Y||^{2}$.

Доказательство: $\E((X-Y)^2)=\E(X^{2})+\E(Y^{2})-2\E(XY)=\E(X^{2})+\E(Y^{2})$.

Заметим пару интересных фактов в нашей геометрии: 

Длина любой константы равна ее модулю: $||c||=\sqrt{\E(c^{2})}=\sqrt{c^{2}}=|c|$, или $||c||^{2}=c^{2}$.

Если $\E(Y)=0$, то случайная величина $Y$ перпендикулярна любой константе: $<Y,c>=\E(Yc)=c\E(Y)=0$.

Рисунок 1. числовая прямая, ей перпендикулярная величина $Y$ и неперпендикулярная $X$

Применим теорему Пифагора чтобы увидеть дисперсию...

Пусть $X$ - произвольная случайная величина. У случайной величины $Y=X-\E(X)$ матожидание равно нулю, $\E(Y)=\E(X)-\E(X)=0$, поэтому $Y$ перпендикулярна любой константе, в частности, $Y\bot \E(X)$. Заметим кстати, что $||X-\E(X)||^{2}=\E((X-\E(X))^{2})=\Var(X)$.

Применяя теорему Пифагора к $X-\E(X)\bot \E(X)$ получаем:
$||X||^{2}=||X-\E(X)||^{2}+||\E(X)||^{2}$

Переходя к ожиданиям, получаем $\E(X^{2})=\Var(X)+(\E(X))^{2}$.

Рисунок 2. числовая прямая, ей неперпендикулярная $X$, подписи $\E(X)^{2}$...

Из рисунка видно, что $\E(X)$ это проекция случайной величины $X$ на множество констант! Это действительно так в нашей геометрии: 

Во-первых, $X-\E(X)\bot \E(X)$.

Во-вторых, если взять любую другую константу $c\neq \E(X)$, то расстояние от $X$ до этой константы $c$ будет больше, чем до константы $\E(X)$: $||X-c||>||X-\E(X)||$. Доказательство: Рассмотрим функцию $\E((X-c)^2)=\E(X^{2})+c^{2}-2c\E(X)$. Относительно $c$ это парабола с ветвями вверх и вершиной при $c=\E(X)$. Значит наименьшое значение функции равно $||X-\E(X)||$.



\section{Связь со школьной геометрией}

\section{Условное ожидание - это проекция!}


\section{Геометрия ковариации}
Пусть скалярное произведение задано $<X,Y>=\Cov(X,Y)$.

Все требования к скалярному произведению кроме ... выполнены.

Требование ... выполнено с оговорками.

Эта геометрия наглядна тем, что некоррелированные случайные величины в ней перпендикулярны.
Если $Corr(X,Y)=0$, то $\Cov(X,Y)=0$ и, следовательно, $X$ и $Y$ ортогональны.

Напомним, что независимость означает некоррелированность любых\footnote{Любых борелевских функций (для знакомых с теорией меры)} функций $f(X)$ и $g(Y)$.



\section{Частная корреляция}



В анализе временных рядов при изучении процессов ARMA используется понятие частной корреляции. При этом указывается некий шаманский способ ее подсчета (как правило это система уравнений Юла-Воркера) и изредка - ее интуитивная интерпретация. Мы же беремся рассказать что это такое на самом деле в рамках геометрии ковариации! Станет ясна связь между формулой расчета и интуивной интерпретацией!

Рассмотрим три случайных величины $X$, $Y$ и $Z$. Обычная корреляция $Corr(X,Y)$ - это косинус угла между $X$ и $Y$, $cos(X,Y)$. Что же такое частная корреляция $X$ $Y$ при фиксированном (????) $Z$, $Corr(X,Y|Z)$ (????).

Очень просто! Изначально $X$ коррелировано с $Z$ и $Y$ коррелировано с $Z$. Возьмем и <<очистим>> $X$ и $Y$ от воздействия $Z$, то есть найдем самые похожие на них величины $\hat{X}$ и $\hat{Y}$ не коррелированные с $Z$. От случайной величины $\hat{X}$ требуется чтобы она была некоррелирована с $Z$ и максимально похожа на $X$, то есть чтобы квадрат расстояния $||X-\hat{X}||^{2}=\Var(X-\hat{X})$ был минимальным. Геометрически это означает следующее: есть $Z^{\bot}$ - множество случайных величин, некоррелированных с $Z$ (ортогональных $Z$). Просто спроецируем $X$ и $Y$ на это множество $Z^{\bot}$. Так частная корреляция между $X$ и $Y$ при фиксированном $Z$ это просто обычная корреляция между $\hat{X}$ и $\hat{Y}$, $Corr(X,Y|Z)=Corr(\hat{X},\hat{Y})$ или косинус угла между проекциями $X$ и $Y$ на плоскость $Z^{\bot}$. 

Давайте попробуем посчитать на простом примере, не связанном с временными рядами.
Пусть $X$, $Y$, $Z$




Из этого определения следует формула расчета, предлагаемая процедурой Юла-Воркера.








Чтобы изложения было законченным - проиллюстрируем интуитивную интерпретацию 
% стрелочки для AR и MA процессов - нужно ли, все таки уже прямо не связано?



 то есть спроецируем величины $X$ и $Y$ на множество случайных величин
% продумать обозначение и перевод








%\part{Теорема Дуба}
%\input{../stop_theorem/stop_theorem.tex}

%\part{Метод моментов}
%
\section{Метод моментов}


Пусть случайные величины $X_{1}$, ..., $X_{n}$ независимы и одинаково распределены. Закон больших чисел говорит нам, что среднее выборочное $ \bar{X} $ является хорошей оценкой для математического ожидания $ \E(X_{i}) $:

\[ \bar{X}_{n} \stackrel{P}{\longrightarrow} \E(X_{i}) \]

На практике это означает, что при больших $ n $ эти величины равны:

\[ \bar{X}_{n}\approx \E(X_{i}) \]

На этой нехитрой идее и построен метод моментов. Как конкретно используется идея понятно из следующих двух примеров:

Пример 1. Допустим, что случайные величины $ X_{1} $, ..., $ X_{n} $ независимы и равномерны на $ [\theta;\theta+1] $. Постройте оценку неизвестного параметра $ \theta $ с помощью метода моментов.

В данном случае $ \E(X_{i})=\theta+0.5 $ и, следовательно:

\[ \bar{X}_{n}\approx \theta+0.5 \]
Выражаем $ \theta $:
\[ \theta\approx \bar{X}_{n}-0.5 \]
Это и есть нужная нам оценка:
\[ \hat{\theta}_{MM}:=\bar{X}_{n}-0.5 \]


Пример 2. Неправильная монетка выпадает орлом с неизвестной вероятностью $ p $. Провели несколько экспериментов и каждый раз записывали, сколько раз ее потребовалось подкинуть до появления первого орла. Обозначим эти величины $X_{1}$, ..., $ X_{n} $. Постройте оценку неизвестного параметра $ p $ с помощью метода моментов.

Величины $ X_{i} $ имеют геометрическое распределение, поэтому $ \E(X_{i})=\frac{1}{p} $. Принцип метода моментов гласит:

\[ \bar{X}_{n}\approx \frac{1}{p}\]
Выражаем неизвестный параметр $ p $:
\[ p\approx \frac{1}{\bar{X}_{n}} \]
Это и есть нужная нам оценка:
\[ \hat{p}_{MM}:= \frac{1}{\bar{X}_{n}} \]


Если говорить более формально...

Определение. Пусть $ X_{i} $ одинаково распределены и независимы, а $ \E(X_{i}) $ зависит от неизвестного параметра $ \theta $, скажем $ \E(X_{i})=f(\theta) $. Тогда оценкой метода моментов называется случайная величина:

\[ \hat{\theta}_{MM}:=f^{-1}(\bar{X}_{n}) \]

Конечно, иногда бывают ситуации, когда математическое ожидание $ \E(X_{i}) $ не зависит от $ \theta $. Например, если $ X_{i} $ равномерны на $ [-\theta;\theta] $, то математическое ожидание $ \E(X_{i})=0 $. Что делать в такой ситуации? 

Неспроста же наш метод называется методом моментов... Напомним, что $ k $-ым моментом случайной величины $ X_{i} $ называется математическое ожидание $ \E(X_{i}^{k}) $...


Итак, если условия $\bar{X}_{n}\approx \E(X_{i})$ связанного с первым моментом не хватило, то на помощь придет второй момент случайной величины. В силу того же закона больших чисел:

\[ \frac{\sum X_{i}^{2}}{n} \approx \E(X_{i}^{2})\]


Пример 3. Величины $ X_{i} $ независимы и равномерны на $ [-\theta;\theta] $. Постройте оценку неизвестного параметра $ \theta $ с помощью метода моментов.

Убеждаемся, что $\E(X_{i})=0$. Находим $ \E(X_{i}^{2}) $:

\[  \E(X_{i}^{2}) = \int _{-\theta}^{\theta} x^{2} \frac{1}{2\theta}dx=...=\frac{\theta^{2}}{3} \]

Согласно принципу метода моментов:

\[ \frac{\sum X_{i}^{2}}{n} \approx \frac{\theta^{2}}{3} \]

Выражаем $ \theta $:

\[ \theta\approx \sqrt{3\frac{\sum X_{i}^{2}}{n} }\]

Это и есть нужная нам оценка:

\[ \hat{\theta}_{MM}= \sqrt{3\frac{\sum X_{i}^{2}}{n} }\]

Если не хватит и второго момента, тогда воспользуемся третьим и т.д. Для произвольного $k$ мы имеем:

\[ \frac{\sum X_{i}^{k}}{n} \approx \E(X_{i}^{k})\]

В большинстве случаев хватает именного первого момента. Последующие моменты нужны чаще всего при оценке нескольких параметров.









%\part{Проверка гипотез}
%

\begin{enumerate}
\item Исходный набор данных, например, $X_1$, $X_2$, ... $X_n$ ужимается до одной случайной величины, статистики $W$
\item Мудрые теоремы говорят нам: при верной $H_0$ статистика $W$ имеет такое-то распределение с таким-то $E(W)$
\item Далеко ли $W$ легла от $E(W)$? 
\item Слишком далекие значения статистики $W$ от своего среднего $E(W)$ могут говорить о том, что среднее $E(W)$ было рассчитано неверно.
\item P-value -- вероятность получить более далекое от $E(W)$ значение статистики $W$, чем то значение, что было получено нами
\item Если P-value$>\alpha$, то наше наблюдение достаточно близко к $\E(W)$ и $H_0$ не отвергается. Если P-value$<\alpha$, то наше наблюдение достаточно далеко от $\E(W)$ и $H_0$ отвергается.
\item Почему мы говорим <<не отвергается>>? Почему бы не сказать просто <<принимается>>? Дело в том, что не отвергаться могут сразу несколько противоречивых гипотез. Например при $\bar{X}=1.46$ могут не отвергаться гипотезы $\mu_X=1.45$ и $\mu_X=1.47$, однако было бы не хорошо говорить, что мы их обе приняли.
\end{enumerate}




%\part{Трюк с окружностью}
%If we divide the rope into n segments then the kth largest segment will have expected length
$ \frac{1}{n} \left( \frac{1}{n} + \cdots + \frac{1}{k} \right) $
In particular, the smallest segment has expected length $ 1/n^2 $.

Note that equivalently we could throw n random points on a circle of unit circumference and consider lengths of segments on the circle.

Derivation of the general formula from the formula for the smallest segment
Len $n$ points be thrown uniformly on the circle. Suppose the smallest segment has length x. We can remove an initial prefix of length x from each segment to get a uniformly generated ensemble of $n-1$ segments on a circle of length $1-nx$. In this new arrangement, the smallest segment has expected size $ (1-nx)/(n-1)^2 $. Therefore the expected size of the second smallest segment is
$Ex + \frac{1-nEx}{(n-1)^2} = \frac{1}{n^2} + \frac{1-1/n}{(n-1)^2} = \frac{1}{n^2} + \frac{1}{n(n-1)}$
We can get the general formula in the same way.

Derivation of the formula for the smallest segment
????






\bibliography{/home/boris/science/tex_general/opit}


\printindex % печать предметного указателя здесь


\end{document}