

\begin{enumerate}
\item Исходный набор данных, например, $X_1$, $X_2$, ... $X_n$ ужимается до одной случайной величины, статистики $W$
\item Мудрые теоремы говорят нам: при верной $H_0$ статистика $W$ имеет такое-то распределение с таким-то $E(W)$
\item Далеко ли $W$ легла от $E(W)$? 
\item Слишком далекие значения статистики $W$ от своего среднего $E(W)$ могут говорить о том, что среднее $E(W)$ было рассчитано неверно.
\item P-value -- вероятность получить более далекое от $E(W)$ значение статистики $W$, чем то значение, что было получено нами
\item Если P-value$>\alpha$, то наше наблюдение достаточно близко к $\E(W)$ и $H_0$ не отвергается. Если P-value$<\alpha$, то наше наблюдение достаточно далеко от $\E(W)$ и $H_0$ отвергается.
\item Почему мы говорим <<не отвергается>>? Почему бы не сказать просто <<принимается>>? Дело в том, что не отвергаться могут сразу несколько противоречивых гипотез. Например при $\bar{X}=1.46$ могут не отвергаться гипотезы $\mu_X=1.45$ и $\mu_X=1.47$, однако было бы не хорошо говорить, что мы их обе приняли.
\end{enumerate}


