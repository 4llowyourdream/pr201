\element{prob_one_sample}{ % в фигурных скобках название группы вопросов
 \AMCcompleteMulti
  \begin{questionmult}{1} % тип вопроса (questionmult --- множественный выбор) и в фигурных --- номер вопроса
  Случайные величины $X$ и $Y$ распределены нормально. Для тестирования гипотезы о равенстве дисперсий выбирается $m$ наблюдений случайной величины $X$ и $n$ наблюдений случайной величины $Y$. Какое распределение может иметь статистика, используемая в данном случае?
 \begin{multicols}{3} % располагаем ответы в 3 колонки
   \begin{choices} % опция [o] не рандомизирует порядок ответов
      \correctchoice{$F_{m-1,n-1}$}
      \wrongchoice{$F_{m+1,n+1}$}
      \wrongchoice{$F_{m,n}$}
      \wrongchoice{$t_{m+n-2}$}
      \wrongchoice{$t_{m+n-1}$}
      \end{choices}
  \end{multicols}
  \end{questionmult}
}

\element{prob_one_sample}{ % в фигурных скобках название группы вопросов
 \AMCcompleteMulti
  \begin{questionmult}{2} % тип вопроса (questionmult --- множественный выбор) и в фигурных --- номер вопроса
  Требуется проверить гипотезу о равенстве математических ожиданий в двух нормальных выборках размером $m$ и $n$. Если дисперсии неизвестны, но равны, то тестовая статистика имеет распределение
 \begin{multicols}{3} % располагаем ответы в 3 колонки
   \begin{choices} % опция [o] не рандомизирует порядок ответов
      \correctchoice{$t_{m+n-2}$}
      \wrongchoice{$F_{m+1,n+1}$}
      \wrongchoice{$F_{m,n}$}
      \wrongchoice{$F_{m-1,n-1}$}
      \wrongchoice{$t_{m+n-1}$}
      \end{choices}
  \end{multicols}
  \end{questionmult}
}


\element{prob_one_sample}{ % в фигурных скобках название группы вопросов
 \AMCcompleteMulti
  \begin{questionmult}{3} % тип вопроса (questionmult --- множественный выбор) и в фигурных --- номер вопроса
  Требуется проверить гипотезу о равенстве дисперсий по двум нормальным выборкам размером $20$ и $16$ наблюдений. Несмещённая оценка дисперсии по первой выборке составила $60$, по второй --- $90$. Тестовая статистика может быть равна
 \begin{multicols}{3} % располагаем ответы в 3 колонки
   \begin{choices} % опция [o] не рандомизирует порядок ответов
      \correctchoice{$1.5$}
      \wrongchoice{$1$}
      \wrongchoice{$1.224$}
      \wrongchoice{$2$}
      \wrongchoice{$4$}
      \end{choices}
  \end{multicols}
  \end{questionmult}
}


\element{prob_one_sample}{ % в фигурных скобках название группы вопросов
 \AMCcompleteMulti
  \begin{questionmult}{4} % тип вопроса (questionmult --- множественный выбор) и в фигурных --- номер вопроса
  Требуется проверить гипотезу о равенстве математических ожиданий по двум нормальным выборкам размером $20$ и $16$ наблюдений. Истинные дисперсии по обеим выборкам известны, совпадают и равны $16$. Разница выборочных средних равна $1$. Тестовая статистика может быть равна
 \begin{multicols}{3} % располагаем ответы в 3 колонки
   \begin{choices} % опция [o] не рандомизирует порядок ответов
      \correctchoice{$1.5$}
      \wrongchoice{$1$}
      \wrongchoice{$1.224$}
      \wrongchoice{$2$}
      \wrongchoice{$4$}
      \end{choices}
  \end{multicols}
  \end{questionmult}
}


\element{prob_one_sample}{ % в фигурных скобках название группы вопросов
 \AMCcompleteMulti
  \begin{questionmult}{5} % тип вопроса (questionmult --- множественный выбор) и в фигурных --- номер вопроса
  При проверке гипотезе о равенстве математических ожиданий в двух нормальных выборках размеров $m$ и $n$ при известных, но не равных дисперсиях, тестовая статистика имеет распределение
 \begin{multicols}{3} % располагаем ответы в 3 колонки
   \begin{choices} % опция [o] не рандомизирует порядок ответов
      \correctchoice{$N(0;1)$}
      \wrongchoice{$F_{m-1,n-1}$}
      \wrongchoice{$F_{m}$}
      \wrongchoice{$t_{m+n-2}$}
      \wrongchoice{$t_{m+n-1}$}
      \end{choices}
  \end{multicols}
  \end{questionmult}
}


\element{prob_one_sample}{ % в фигурных скобках название группы вопросов
 \AMCcompleteMulti
  \begin{questionmult}{6} % тип вопроса (questionmult --- множественный выбор) и в фигурных --- номер вопроса
  При проверке гипотезы о равенстве долей используется следующее распределение
 \begin{multicols}{3} % располагаем ответы в 3 колонки
   \begin{choices} % опция [o] не рандомизирует порядок ответов
      \correctchoice{$N(0;1)$}
      \wrongchoice{$F_{m-1,n-1}$}
      \wrongchoice{$F_{m, n}$}
      \wrongchoice{$t_{m+n-2}$}
      \wrongchoice{$t_{m+n-1}$}
      \end{choices}
  \end{multicols}
  \end{questionmult}
}


\element{prob_one_sample_rejected}{ % в фигурных скобках название группы вопросов
 \AMCcompleteMulti
  \begin{questionmult}{7} % тип вопроса (questionmult --- множественный выбор) и в фигурных --- номер вопроса
   Есть две нормально распределённых выборки размером $20$ и $16$ наблюдений. Истинные дисперсии по обеим выборкам неизвестны и равны. Выборочные средние по обеим выборкам совпадают. Гипотеза о равенстве математических ожиданий
 %\begin{multicols}{3} % располагаем ответы в 3 колонки
   \begin{choices} % опция [o] не рандомизирует порядок ответов
      \correctchoice{не отвергается на любом разумном уровне значимости}
      \wrongchoice{отвергается на любом разумном уровне значимости}
      \wrongchoice{не отвергается на 5\%-ом и отвергается на 1\%-ом уровне значимости}
      \wrongchoice{не отвергается на 1\%-ом и отвергается на 5\%-ом уровне значимости}
      \wrongchoice{Гипотезу невозможно проверить}
      \end{choices}
  %\end{multicols}
  \end{questionmult}
}


\element{prob_one_sample}{ % в фигурных скобках название группы вопросов
 \AMCcompleteMulti
  \begin{questionmult}{8} % тип вопроса (questionmult --- множественный выбор) и в фигурных --- номер вопроса
  Для проверки гипотезы о равенстве долей в двух выборках  могут использоваться следующие распределения
 \begin{multicols}{3} % располагаем ответы в 3 колонки
   \begin{choices} % опция [o] не рандомизирует порядок ответов
      \wrongchoice{только $N(0;1)$}
      \correctchoice{$N(0;1)$ и $\chi^2_1$}
      \wrongchoice{только $\chi^2_1$}
      \wrongchoice{$N(0;1)$ и $F_{m,n}$}
      \wrongchoice{только $F_{m,n}$}
      \end{choices}
  \end{multicols}
  \end{questionmult}
}

\element{prob_one_sample}{ % в фигурных скобках название группы вопросов
 \AMCcompleteMulti
  \begin{questionmult}{9} % тип вопроса (questionmult --- множественный выбор) и в фигурных --- номер вопроса
  Доля успехов в первой выборке равна $0.55$, доля успехов во второй выборке --- $0.4$. Количество наблюдений в выборках равно $40$ и $20$ соответственно. Тестовая статистика для проверки гипотезы о равенстве долей может быть равна
 \begin{multicols}{3} % располагаем ответы в 3 колонки
   \begin{choices} % опция [o] не рандомизирует порядок ответов
      \correctchoice{$1.1$}
      \wrongchoice{$2.2$}
      \wrongchoice{$1.2$}
      \wrongchoice{$2.4$}
      \wrongchoice{$0.9$}
      \end{choices}
  \end{multicols}
  \end{questionmult}
}

\element{prob_one_sample}{ % в фигурных скобках название группы вопросов
 \AMCcompleteMulti
  \begin{questionmult}{10} % тип вопроса (questionmult --- множественный выбор) и в фигурных --- номер вопроса
  Доля успехов в первой выборке равна $0.8$, доля успехов во второй выборке --- $0.3$. Количество наблюдений в выборках $40$ и $20$ соответственно. Гипотеза о равенстве долей
 %\begin{multicols}{3} % располагаем ответы в 3 колонки
   \begin{choices} % опция [o] не рандомизирует порядок ответов
     \correctchoice{отвергается на любом разумном уровне значимости}
     \wrongchoice{не отвергается на любом разумном уровне значимости}
     \wrongchoice{не отвергается на 5\%-ом и отвергается на 1\%-ом уровне значимости}
     \wrongchoice{не отвергается на 1\%-ом и отвергается на 5\%-ом уровне значимости}
     \wrongchoice{Гипотезу невозможно проверить}
      \end{choices}
  %\end{multicols}
  \end{questionmult}
}























\element{prob_two_samples}{ % в фигурных скобках название группы вопросов
 \AMCcompleteMulti
  \begin{questionmult}{2} % тип вопроса (questionmult --- множественный выбор) и в фигурных --- номер вопроса
  Для выборки $X_1,\ldots,X_n$, имеющей нормальное распределение, проверяется гипотеза $H_0: \sigma^2=\sigma_0^2$ против $H_a: \sigma^2 > \sigma_0^2$. Критическая область имеет вид
 %\begin{multicols}{3} % располагаем ответы в 3 колонки
   \begin{choices} % опция [o] не рандомизирует порядок ответов
      \correctchoice{$(A,+\infty)$, где $A$ таково, что $\P(\chi^2_{n-1} < A) =1-\alpha$}
      \wrongchoice{$(A,+\infty)$, где $A$ таково, что $\P(\chi^2_{n-1} < A)  =\alpha$}
      \wrongchoice{$(0,A)$, где $A$ таково, что $\P(\chi^2_{n-1} < A)  =1-\alpha$}
      \wrongchoice{$(-\infty,A)$, где $A$ таково, что $\P(\chi^2_{n-1} < A)  =1-\alpha$}
      \wrongchoice{ $(0,A)$, где $A$ таково, что $\P(\chi^2_{n-1} < A)  =\alpha$}
      \end{choices}
  %\end{multicols}
  \end{questionmult}
}


% \element{prob_two_samples_rejected}{ % в фигурных скобках название группы вопросов
%  \AMCcompleteMulti
%   \begin{questionmult}{3} % тип вопроса (questionmult --- множественный выбор) и в фигурных --- номер вопроса
%   Для выборки $X_1,\ldots,X_n$, имеющей нормальное распределение, проверяется гипотеза $H_0: \sigma^2=\sigma_0^2$ против $H_a: \sigma^2 < \sigma_0^2$. Критическая область имеет вид
%  %\begin{multicols}{3} % располагаем ответы в 3 колонки
%    \begin{choices} % опция [o] не рандомизирует порядок ответов
%       \correctchoice{$(0,A)$, где $A$ таково, что $F_{\chi^2_{n-1}} (A) =\alpha$}
%       \wrongchoice{$(0,A)$, где $A$ таково, что $F_{\chi^2_{n-1}} (A) =1-\alpha$}
%       \wrongchoice{$(-\infty,A)$, где $A$ таково, что $F_{\chi^2_{n-1}} (A) =\alpha$}
%       \wrongchoice{ $(A,+\infty)$, где $A$ таково, что $F_{\chi^2_{n-1}} (A) =1-\alpha$}
%       \end{choices}
%   %\end{multicols}
%   \end{questionmult}
% }


\element{prob_two_samples}{ % в фигурных скобках название группы вопросов
 \AMCcompleteMulti
  \begin{questionmult}{4} % тип вопроса (questionmult --- множественный выбор) и в фигурных --- номер вопроса
  При подбрасывании игральной кости 600 раз шестерка выпала 105 раз. Гипотеза о том, что кость правильная
 %\begin{multicols}{3} % располагаем ответы в 3 колонки
   \begin{choices} % опция [o] не рандомизирует порядок ответов
      \correctchoice{не отвергается при любом разумном значении $\alpha$}
      \wrongchoice{отвергается при любом разумном значении $\alpha$}
      \wrongchoice{отвергается при $\alpha = 0.05$, не отвергается при $\alpha = 0.01$}
      \wrongchoice{отвергается при $\alpha = 0.01$, не отвергается при $\alpha = 0.05$}
      \wrongchoice{Гипотезу невозможно проверить}
      \end{choices}
  %\end{multicols}
  \end{questionmult}
}


\element{prob_two_samples}{ % в фигурных скобках название группы вопросов
 \AMCcompleteMulti
  \begin{questionmult}{5} % тип вопроса (questionmult --- множественный выбор) и в фигурных --- номер вопроса
  Величины $X_1,\ldots,X_n$ --- выборка из нормально распределенной случайной величины с неизвестным математическим ожиданием и известной дисперсией. На уровне значимости $\alpha$ проверяется гипотеза $H_0: \mu = \mu_0$ против $H_a: \mu \neq \mu_0$. Обозначим $\varphi_1$ и $\varphi_2$ вероятности ошибок первого и второго рода соответственно. Между параметрами задачи всегда выполнено соотношение
 \begin{multicols}{3} % располагаем ответы в 3 колонки
   \begin{choices} % опция [o] не рандомизирует порядок ответов
      \correctchoice{$\varphi_1 = \alpha$}
      \wrongchoice{$\varphi_1 = 1 - \alpha$}
      \wrongchoice{$\varphi_2 = \alpha$}
      \wrongchoice{$\varphi_2 = 1 - \alpha$}
      \wrongchoice{$\varphi_1 + \varphi_2 = \alpha$}
      \end{choices}
  \end{multicols}
  \end{questionmult}
}


\element{prob_two_samples}{ % в фигурных скобках название группы вопросов
 \AMCcompleteMulti
  \begin{questionmult}{6} % тип вопроса (questionmult --- множественный выбор) и в фигурных --- номер вопроса
  По случайной выборке из 200 наблюдений было оценено выборочное среднее $\bar{X} = 25$  и несмещённая оценка дисперсии $\hat{\sigma}^2 = 25$. В рамках проверки гипотезы $H_0: \mu = 20$ против $H_a: \mu > 20$ можно сделать вывод, что гипотеза $H_0$
 %\begin{multicols}{3} % располагаем ответы в 3 колонки
   \begin{choices} % опция [o] не рандомизирует порядок ответов
      \correctchoice{отвергается при любом разумном значении $\alpha$}
      \wrongchoice{отвергается при $\alpha = 0.05$, не отвергается при $\alpha = 0.01$}
      \wrongchoice{отвергается при $\alpha = 0.01$, не отвергается при $\alpha = 0.05$}
      \wrongchoice{не отвергается при любом разумном значении $\alpha$}
      \wrongchoice{Гипотезу невозможно проверить}
      \end{choices}
  %\end{multicols}
  \end{questionmult}
}


\element{prob_two_samples}{ % в фигурных скобках название группы вопросов
 \AMCcompleteMulti
  \begin{questionmult}{7} % тип вопроса (questionmult --- множественный выбор) и в фигурных --- номер вопроса
   По выборке $X_1,\ldots, X_n$ из нормального распределения строятся по стандартным формулам доверительные интервалы для математического ожидания. Получен интервал $(a_1,a_2)$ при известной дисперсии и интервал $(b_1,b_2)$ при неизвестной дисперсии. Всегда справедливы следующие соотношения:
 \begin{multicols}{2} % располагаем ответы в 3 колонки
   \begin{choices} % опция [o] не рандомизирует порядок ответов
      \correctchoice{$|a_1-b_1| = |a_2-b_2|$}
      \wrongchoice{$a_2 - a_1 < b_2 - b_1$}
      \wrongchoice{$a_2 - a_1 > b_2 - b_1$}
      \wrongchoice{$a_1<0,b_1<0,a_2>0,b_2>0$}
      \wrongchoice{$a_1>0,b_1>0,a_2>0,b_2>0$}
      \end{choices}
  \end{multicols}
  \end{questionmult}
}


\element{prob_two_samples}{ % в фигурных скобках название группы вопросов
 \AMCcompleteMulti
  \begin{questionmult}{8} % тип вопроса (questionmult --- множественный выбор) и в фигурных --- номер вопроса
  Величины $X_1,\ldots,X_n$ --- выборка из нормального распределения.  Статистика $U=\frac{5-\bar{X}}{5/\sqrt{n}}$ применима для проверки
 %\begin{multicols}{3} % располагаем ответы в 3 колонки
   \begin{choices} % опция [o] не рандомизирует порядок ответов
      \wrongchoice{гипотезы $H_0: \mu = 5$ при известной дисперсии, равной 5, при любых $n$}
      \correctchoice{гипотезы $H_0: \mu = 5$ при известной дисперсии, равной 25, при любых $n$}
      \wrongchoice{гипотезы $H_0: \mu = 5$ при известной дисперсии, равной 5, при больших $n$}
      \wrongchoice{гипотезы $H_0: \mu = 5$ при известной дисперсии, равной 25, только при больших $n$}
      \wrongchoice{гипотезы $H_0: \sigma = 5$}
      \end{choices}
  %\end{multicols}
  \end{questionmult}
}

\element{prob_two_samples}{ % в фигурных скобках название группы вопросов
 \AMCcompleteMulti
  \begin{questionmult}{9} % тип вопроса (questionmult --- множественный выбор) и в фигурных --- номер вопроса
Выборочная доля успехов в некотором испытании составляет $0.3$. Исследователь Ромео хочет, чтобы длина двустороннего 95\%-го доверительного интервала для истинной доли не превышала $0.1$. Количество наблюдений, необходимых для этого, примерно равно
 \begin{multicols}{3} % располагаем ответы в 3 колонки
   \begin{choices} % опция [o] не рандомизирует порядок ответов
      \correctchoice{$322$}
      \wrongchoice{$161$}
      \wrongchoice{$113$}
      \wrongchoice{$225$}
      \wrongchoice{$81$}
      \end{choices}
  \end{multicols}
  \end{questionmult}
}

\element{prob_two_samples}{ % в фигурных скобках название группы вопросов
 \AMCcompleteMulti
  \begin{questionmult}{10} % тип вопроса (questionmult --- множественный выбор) и в фигурных --- номер вопроса
  Пусть $X_1,\ldots,X_n$ --- выборка из нормального распределения с известной дисперсией $\sigma^2$. Пусть $U = \frac{\bar{X}-\mu_0}{\sigma/\sqrt{n}}$. Величина $U^2$ имеет распределение
 \begin{multicols}{3} % располагаем ответы в 3 колонки
   \begin{choices} % опция [o] не рандомизирует порядок ответов
     \correctchoice{$\chi^2_1$}
     \wrongchoice{$\chi^2_{n-1}$}
     \wrongchoice{$t_1$}
     \wrongchoice{$t_{n-1}$}
     \wrongchoice{$F_{1,n-1}$}
      \end{choices}
  \end{multicols}
  \end{questionmult}
}
























\element{prob_sample_char}{ % в фигурных скобках название группы вопросов
 \AMCcompleteMulti
  \begin{questionmult}{1} % тип вопроса (questionmult --- множественный выбор) и в фигурных --- номер вопроса
  Дана реализация выборки: 3, 1, 2. Выборочный начальный момент первого порядка равен
 \begin{multicols}{3} % располагаем ответы в 3 колонки
   \begin{choices} % опция [o] не рандомизирует порядок ответов
      \correctchoice{2}
      \wrongchoice{1}
      \wrongchoice{0}
      \wrongchoice{3}
      \wrongchoice{14/3}
      \end{choices}
  \end{multicols}
  \end{questionmult}
}

\element{prob_sample_char}{ % в фигурных скобках название группы вопросов
 \AMCcompleteMulti
  \begin{questionmult}{2} % тип вопроса (questionmult --- множественный выбор) и в фигурных --- номер вопроса
  Дана реализация выборки: 3, 1, 2. Несмещённая оценка дисперсии равна
 \begin{multicols}{3} % располагаем ответы в 3 колонки
   \begin{choices} % опция [o] не рандомизирует порядок ответов
      \correctchoice{1}
      \wrongchoice{1/2}
      \wrongchoice{1/3}
      \wrongchoice{2/3}
      \wrongchoice{2}
      \end{choices}
  \end{multicols}
  \end{questionmult}
}


\element{prob_sample_char}{ % в фигурных скобках название группы вопросов
 \AMCnoCompleteMulti
  \begin{questionmult}{3} % тип вопроса (questionmult --- множественный выбор) и в фигурных --- номер вопроса
  Выберите НЕВЕРНОЕ утверждение про эмпирическую функцию распределения $F_n(x)$
 %\begin{multicols}{3} % располагаем ответы в 3 колонки
   \begin{choices} % опция [o] не рандомизирует порядок ответов
      \correctchoice{$F_n(x)$ является невозрастающей функцией}
      %\wrongchoice{$F_n(x)$ является состоятельной оценкой функции распределения $F(x)$}
      \wrongchoice{$F_n(x)$ имеет разрыв в каждой точке вариационного ряда}
      \wrongchoice{$F_n(x)$ асимптотически нормальна}
      \wrongchoice{$\E(F_n(x))=F(x)$}
      \wrongchoice{$\Var(F_n(x))=F(x)(1-F(x))$}
      \end{choices}
  %\end{multicols}
  \end{questionmult}
}


\element{prob_sample_char}{ % в фигурных скобках название группы вопросов
 \AMCcompleteMulti
  \begin{questionmult}{4} % тип вопроса (questionmult --- множественный выбор) и в фигурных --- номер вопроса
 Юрий Петров утверждает, что обычно посещает половину занятий по Статистике. За последние полгода из 36 занятий он не посетил ни одного. Вычислите значение критерия хи-квадрат Пирсона для гипотезы, что утверждение Юрия Петрова истинно и укажите число степеней свободы
 \begin{multicols}{3} % располагаем ответы в 3 колонки
   \begin{choices} % опция [o] не рандомизирует порядок ответов
      \correctchoice{$\chi^2 = 36$, $df=1$}
      \wrongchoice{$\chi^2 = 2$, $df=2$}
      \wrongchoice{$\chi^2 = 14$, $df=1$}
      \wrongchoice{$\chi^2 = 20$, $df=2$}
      \wrongchoice{$\chi^2 = 24$, $df=1$}
      \end{choices}
  \end{multicols}
  \end{questionmult}
}


\element{prob_sample_char}{ % в фигурных скобках название группы вопросов
 \AMCcompleteMulti
  \begin{questionmult}{5} % тип вопроса (questionmult --- множественный выбор) и в фигурных --- номер вопроса
  Производитель фломастеров попросил трёх человек оценить два вида фломастеров: «Лесенка» и «Erich Krause» по 10-балльной шкале:

\begin{center}
\begin{tabular}{lrrr} \toprule
 & Пафнутий & Андрей & Карл \\
\midrule
Лесенка & 9 & 7 & 6 \\
Erich Krause & 8 & 9 & 7 \\
\bottomrule
\end{tabular}
\end{center}

\textbf{Точное} $P$-значение ($P$-value) статистики теста знаков равно

 \begin{multicols}{3} % располагаем ответы в 3 колонки
   \begin{choices} % опция [o] не рандомизирует порядок ответов
      \correctchoice{3/8}
      \wrongchoice{1/8}
      \wrongchoice{1/3}
      \wrongchoice{1/2}
      \wrongchoice{2/3}
      \end{choices}
  \end{multicols}
  \end{questionmult}
}


\element{prob_sample_char}{ % в фигурных скобках название группы вопросов
 \AMCcompleteMulti
  \begin{questionmult}{6} % тип вопроса (questionmult --- множественный выбор) и в фигурных --- номер вопроса
  Производитель фломастеров попросил трёх человек оценить два вида фломастеров: «Лесенка» и «Erich Krause» по 10-балльной шкале:

\begin{center}
\begin{tabular}{lrrr} \toprule
 & Пафнутий  & Андрей & Карл \\
\midrule
Лесенка & 9 & 7 & 6 \\
Erich Krause & 8 & 9 & 7 \\
\bottomrule
\end{tabular}
\end{center}

Вычислите модуль значения статистики теста знаков. \textbf{Используя нормальную аппроксимацию}, проверьте на уровне значимости 0.1 гипотезу о том, что фломастеры имеют одинаковое качество.

 \begin{multicols}{3} % располагаем ответы в 3 колонки
   \begin{choices} % опция [o] не рандомизирует порядок ответов
      \correctchoice{0.58, $H_0$ не отвергается}
      \wrongchoice{0.58, $H_0$ отвергается}
      \wrongchoice{0.43, $H_0$ не отвергается}
      \wrongchoice{1.65, $H_0$ отвергается}
      \wrongchoice{1.96, $H_0$ отвергается}
      \end{choices}
  \end{multicols}
  \end{questionmult}
}


\element{prob_sample_char}{ % в фигурных скобках название группы вопросов
 \AMCcompleteMulti
  \begin{questionmult}{7} % тип вопроса (questionmult --- множественный выбор) и в фигурных --- номер вопроса
   Кузнец Вакула в течение 100 лет ведет статистику о прилете аистов и рождении младенцев на хуторе близ Диканьки. У него получилась следующая таблица сопряженности

\begin{center}
\begin{tabular}{lrr} \toprule
& Аисты прилетали  & Аисты не прилетали \\
\midrule
Появлялся младенец & 30 & 10 \\
Не появлялся младенец & 30 & 30 \\
\bottomrule
\end{tabular}
\end{center}

Укажите число степеней свободы статистики Пирсона и на уровне значимости 5\% определите, зависит ли появление младенца от прилета аистов

 \begin{multicols}{3} % располагаем ответы в 3 колонки
   \begin{choices} % опция [o] не рандомизирует порядок ответов
      \correctchoice{$df=1$, зависит}
      \wrongchoice{$df=1$, не зависит}
      \wrongchoice{$df=3$, зависит}
      \wrongchoice{$df=4$, зависит}
      \wrongchoice{$df=2$, зависит}
      \end{choices}
  \end{multicols}
  \end{questionmult}
}



\element{prob_sample_char}{ % в фигурных скобках название группы вопросов
 \AMCcompleteMulti
  \begin{questionmult}{8} % тип вопроса (questionmult --- множественный выбор) и в фигурных --- номер вопроса
  В коробке 50 купюр пяти различных номиналов. Случайным образом достаются две купюры. Номиналы вынимаемых купюр
 \begin{multicols}{2} % располагаем ответы в 3 колонки
   \begin{choices} % опция [o] не рандомизирует порядок ответов
   \correctchoice{отрицательно коррелированы}
   \wrongchoice{положительно коррелированы}
    \wrongchoice{не коррелированы и не зависимы}
    \wrongchoice{не коррелированы, но зависимы}
    \wrongchoice{положительно коррелированы, но не зависимы}
      \end{choices}
  \end{multicols}
  \end{questionmult}
}



\element{prob_sample_char}{ % в фигурных скобках название группы вопросов
 \AMCcompleteMulti
  \begin{questionmult}{9} % тип вопроса (questionmult --- множественный выбор) и в фигурных --- номер вопроса
Экзамен принимают два преподавателя: Злой и Добрый. Они поставили следующие оценки:

\begin{center}
\begin{tabular}{lrrrrr} \toprule
Злой   & 2 & 3 & 10 & 8 & 3 \\
Добрый & 6 & 4 & 7  & 8 & \\
\bottomrule
\end{tabular}
\end{center}

Значение статистики критерия Вилкоксона о совпадении распределений оценок равно

 \begin{multicols}{3} % располагаем ответы в 3 колонки
   \begin{choices} % опция [o] не рандомизирует порядок ответов
     \correctchoice{22.5}
     \wrongchoice{7.5}
     \wrongchoice{19}
     \wrongchoice{20}
     \wrongchoice{20.5}
      \end{choices}
  \end{multicols}
  \end{questionmult}
}



\element{prob_sample_char}{ % в фигурных скобках название группы вопросов
 \AMCcompleteMulti
  \begin{questionmult}{10} % тип вопроса (questionmult --- множественный выбор) и в фигурных --- номер вопроса
Датчик случайных чисел выдал два значения псевдослучайных чисел: $0.5$ и $0.9$. Вычислите значение критерия Колмогорова и проверьте гипотезу о соответствии распределения равномерному на уровне значимости $0.1$. Критическое значение статистики Колмогорова для уровня значимости $0.1$ и двух наблюдений равно $0.776$.
 \begin{multicols}{3} % располагаем ответы в 3 колонки
   \begin{choices} % опция [o] не рандомизирует порядок ответов
      \correctchoice{$0.5$, $H_0$ не отвергается}
      \wrongchoice{$1.4$, $H_0$ отвергается}
      \wrongchoice{$0.9$, $H_0$ отвергается}
      \wrongchoice{$0.9$, $H_0$ не отвергается}
      \wrongchoice{$0.4$, $H_0$ не отвергается}
      \end{choices}
  \end{multicols}
  \end{questionmult}
}


























\element{prob_ml_mm}{ % в фигурных скобках название группы вопросов
 \AMCnoCompleteMulti
  \begin{questionmult}{1} % тип вопроса (questionmult --- множественный выбор) и в фигурных --- номер вопроса
  Выберите НЕВЕРНОЕ утверждение про метод максимального правдоподобия (ММП):
 %\begin{multicols}{3} % располагаем ответы в 3 колонки
   \begin{choices} % опция [o] не рандомизирует порядок ответов
      \correctchoice{Оценки ММП асимтотически нормальны $\cN(0;1)$}
      \wrongchoice{ММП применим для зависимых случайных величин}
      \wrongchoice{ММП применим для оценивания двух и более параметров}
      \wrongchoice{При выполнении технических предпосылок оценки ММП состоятельны}
      \wrongchoice{ММП оценки не всегда совпадают с оценками метода моментов}
      \end{choices}
  %\end{multicols}
  \end{questionmult}
}

\element{prob_ml_mm}{ % в фигурных скобках название группы вопросов
 \AMCcompleteMulti
  \begin{questionmult}{2} % тип вопроса (questionmult --- множественный выбор) и в фигурных --- номер вопроса
  Если величина $\hat\theta$ имеет нормальное распределение $\cN(2;0.01^2)$, то, согласно дельта-методу, $\hat\theta^2$ имеет примерно нормальное распределение
 \begin{multicols}{3} % располагаем ответы в 3 колонки
   \begin{choices} % опция [o] не рандомизирует порядок ответов
      \correctchoice{$\cN(4;16\cdot 0.01^2)$}
      \wrongchoice{$\cN(4;4\cdot 0.01^2)$}
      \wrongchoice{$\cN(2;4\cdot 0.01^2)$}
      \wrongchoice{$\cN(4;8\cdot 0.01^2)$}
      \wrongchoice{$\cN(4;2\cdot 0.01^2)$}
      \end{choices}
  \end{multicols}
  \end{questionmult}
}


\element{prob_ml_mm}{ % в фигурных скобках название группы вопросов
 \AMCcompleteMulti
  \begin{questionmult}{3} % тип вопроса (questionmult --- множественный выбор) и в фигурных --- номер вопроса
  Случайные величины $X_1$, $X_2$ и $X_3$ независимы и одинаково распределены,

\begin{center}
  \begin{tabular}{lrr} \toprule
  $X_i$ & 3 & 5 \\
  \midrule
  $\P(\cdot)$ & $p$ & $1-p$ \\
  \bottomrule
  \end{tabular}
\end{center}

  Имеется выборка из трёх наблюдений: $X_1=5$, $X_2=3$, $X_3=5$. Оценка неизвестного $p$, полученная методом максимального правдоподобия, равна:


 \begin{multicols}{3} % располагаем ответы в 3 колонки
   \begin{choices} % опция [o] не рандомизирует порядок ответов
      \correctchoice{$1/3$}
      \wrongchoice{$1/2$}
      \wrongchoice{$1/4$}
      \wrongchoice{$2/3$}
      \wrongchoice{Метод неприменим}
      \end{choices}
  \end{multicols}
  \end{questionmult}
}


\element{prob_ml_mm}{ % в фигурных скобках название группы вопросов
 \AMCcompleteMulti
  \begin{questionmult}{4} % тип вопроса (questionmult --- множественный выбор) и в фигурных --- номер вопроса
    Случайные величины $X_1$, $X_2$ и $X_3$ независимы и одинаково распределены,

\begin{center}
    \begin{tabular}{lrr} \toprule
    $X_i$ & 3 & 5 \\
    \midrule
    $\P(\cdot)$ & $p$ & $1-p$ \\
    \bottomrule
    \end{tabular}
\end{center}

    По выборке оказалось, что $\bar X = 4.5$. Оценка неизвестного $p$, полученная методом моментов, равна:


   \begin{multicols}{3} % располагаем ответы в 3 колонки
     \begin{choices} % опция [o] не рандомизирует порядок ответов
        \correctchoice{$1/4$}
        \wrongchoice{$1/2$}
        \wrongchoice{$1/3$}
        \wrongchoice{$2/3$}
        \wrongchoice{Метод неприменим}
      \end{choices}
  \end{multicols}
  \end{questionmult}
}

\element{prob_ml_mm}{ % в фигурных скобках название группы вопросов
 \AMCcompleteMulti
  \begin{questionmult}{5} % тип вопроса (questionmult --- множественный выбор) и в фигурных --- номер вопроса
  Величины $X_1$, $X_2$, \ldots, $X_{2016}$ независимы и одинаково распределены, $\cN(\mu ; 42)$. Оказалось, что $\bar X =  -23$. Про оценки метода моментов, $\hat \mu_{MM}$, и метода максимального правдоподобия, $\hat \mu_{ML}$, можно утверждать, что

 \begin{multicols}{3} % располагаем ответы в 3 колонки
   \begin{choices} % опция [o] не рандомизирует порядок ответов
      \correctchoice{$\hat \mu_{ML} = -23$, $\hat\mu_{MM} = -23$}
      \wrongchoice{$\hat \mu_{ML} > -23$, $\hat\mu_{MM} = -23$}
      \wrongchoice{$\hat \mu_{ML} = -23$, $\hat\mu_{MM} > -23$}
      \wrongchoice{$\hat \mu_{ML} < -23$, $\hat\mu_{MM} = -23$}
      \wrongchoice{$\hat \mu_{ML} = -23$, $\hat\mu_{MM} < -23$}
      \end{choices}
  \end{multicols}
  \end{questionmult}
}


\element{prob_ml_mm}{ % в фигурных скобках название группы вопросов
 \AMCnoCompleteMulti
  \begin{questionmult}{6} % тип вопроса (questionmult --- множественный выбор) и в фигурных --- номер вопроса
 Выберите НЕВЕРНОЕ утверждение про логарифмическую функцию правдоподобия $\ell(\theta)$

 %\begin{multicols}{3} % располагаем ответы в 3 колонки
   \begin{choices} % опция [o] не рандомизирует порядок ответов
      \correctchoice{Функция $\ell(\theta)$ имеет максимум при $\theta=0$}
      \wrongchoice{Функция $\ell(\theta)$ может принимать отрицательные значения}
      \wrongchoice{Функция $\ell(\theta)$ может принимать положительные значения}
      \wrongchoice{Функция $\ell(\theta)$ может принимать значения больше единицы}
      \wrongchoice{Функция $\ell(\theta)$ может иметь несколько экстремумов}
      \end{choices}
  %\end{multicols}
  \end{questionmult}
}


\element{prob_ml_mm}{ % в фигурных скобках название группы вопросов
 \AMCcompleteMulti
  \begin{questionmult}{7} % тип вопроса (questionmult --- множественный выбор) и в фигурных --- номер вопроса
Величины $X_1$, \ldots, $X_n$ независимы и одинаково распределены, $\E(X_1^2)=2\theta + 4$. По выборке из 100 наблюдений оказалось, что $\sum_{i=1}^{100} X_i^2 = 200$. Оценка метода момента, $\hat\theta_{MM}$, равна

 \begin{multicols}{3} % располагаем ответы в 3 колонки
   \begin{choices} % опция [o] не рандомизирует порядок ответов
      \correctchoice{-1}
      \wrongchoice{1}
      \wrongchoice{0}
      \wrongchoice{2}
      \wrongchoice{Метод неприменим}
      \end{choices}
  \end{multicols}
  \end{questionmult}
}



\element{prob_ml_mm}{ % в фигурных скобках название группы вопросов
 \AMCcompleteMulti
  \begin{questionmult}{8} % тип вопроса (questionmult --- множественный выбор) и в фигурных --- номер вопроса
По выборке из 100 наблюдений построена оценка метода максимального правдоподобия, $\hat \theta_{ML} = 42$. Вторая производная лог-функции правдоподобия равна $\ell''(\hat\theta) = -1$. Ширина 95\%-го доверительного интервала для неизвестного параметра $\theta$ примерно равна
 \begin{multicols}{3} % располагаем ответы в 3 колонки
   \begin{choices} % опция [o] не рандомизирует порядок ответов
   \correctchoice{4}
   \wrongchoice{1}
    \wrongchoice{2}
    \wrongchoice{8}
    \wrongchoice{1/2}
  \end{choices}
  \end{multicols}
  \end{questionmult}
}



\element{prob_ml_mm}{ % в фигурных скобках название группы вопросов
 \AMCcompleteMulti
  \begin{questionmult}{9} % тип вопроса (questionmult --- множественный выбор) и в фигурных --- номер вопроса
  Проверяется гипотеза $H_0$: $\theta = \gamma$ против альтернативной гипотезы $H_a$: $\theta \neq \gamma$, где $\theta$ и $\gamma$ --- два неизвестных параметра. Выберите верное утверждение о распределении статистики отношения правдоподобия, $LR$:

 \begin{multicols}{2} % располагаем ответы в 3 колонки
   \begin{choices} % опция [o] не рандомизирует порядок ответов
     \correctchoice{Если верна $H_0$, то $LR \sim \chi_1^2$}
     \wrongchoice{Если верна $H_a$, то $LR \sim \chi_1^2$}
     \wrongchoice{Если верна $H_a$, то $LR \sim \chi_2^2$}
     \wrongchoice{И при $H_0$, и при $H_a$, $LR \sim \chi_1^2$}
     \wrongchoice{И при $H_0$, и при $H_a$, $LR \sim \chi_2^2$}
      \end{choices}
  \end{multicols}
  \end{questionmult}
}



\element{prob_ml_mm}{ % в фигурных скобках название группы вопросов
 \AMCcompleteMulti
  \begin{questionmult}{10} % тип вопроса (questionmult --- множественный выбор) и в фигурных --- номер вопроса
  По 100 наблюдениям получена оценка метода максимального правдоподобия, $\hat\theta = 20$, также известны значения лог-функции правдоподобия $\ell(20) = -10$ и $\ell(0)= - 50$. С помощью критерия отношения правдоподобия, $LR$, проверьте гипотезу $H_0$: $\theta = 0$ против $H_0$: $\theta \neq 0$ на уровне значимости 5\%.
 \begin{multicols}{2} % располагаем ответы в 3 колонки
   \begin{choices} % опция [o] не рандомизирует порядок ответов
      \correctchoice{$LR = 80$, $H_0$ отвергается }
      \wrongchoice{$LR = 60$, $H_0$ не отвергается}
      \wrongchoice{$LR = 40$, $H_0$ не отвергается}
      \wrongchoice{$LR = 40$, $H_0$  отвергается}
      \wrongchoice{Критерий неприменим}
      \end{choices}
  \end{multicols}
  \end{questionmult}
}



























\element{prob_estimators}{ % в фигурных скобках название группы вопросов
 \AMCcompleteMulti
  \begin{questionmult}{1} % тип вопроса (questionmult --- множественный выбор) и в фигурных --- номер вопроса
Пусть $X = (X_1, \ldots , X_n)$ --- случайная выборка из биномиального распределения $Bi(5, p)$. Известно, что $\P(X = x) =C_n^x p^x(1-p)^{n-x} $. Информация Фишера $I_n(p)$ равна:
 \begin{multicols}{3} % располагаем ответы в 3 колонки
   \begin{choices} % опция [o] не рандомизирует порядок ответов
      \correctchoice{$\frac{5n}{p(1-p)}$}
      \wrongchoice{$\frac{p(1-p)}{5n}$}
      \wrongchoice{$\frac{5p(1-p)}{n}$}
      \wrongchoice{$\frac{n}{5p(1-p)}$}
      \wrongchoice{$\frac{n}{p(1-p)}$}
      \end{choices}
  \end{multicols}
  \end{questionmult}
}

\element{prob_estimators}{ % в фигурных скобках название группы вопросов
 \AMCcompleteMulti
  \begin{questionmult}{2} % тип вопроса (questionmult --- множественный выбор) и в фигурных --- номер вопроса
Пусть $X = (X_1, \ldots , X_n)$ --- случайная выборка из экспоненциального распределения с плотностью
\[
f(x; \theta) =
\begin{cases}
\frac{1}{\theta}\exp(-\frac{x}{\theta}) \text{ при } x \geq 0,  \\
0 \text{ при } x < 0.
\end{cases}
\]
Информация Фишера $I_n(p)$ равна:
 \begin{multicols}{3} % располагаем ответы в 3 колонки
   \begin{choices} % опция [o] не рандомизирует порядок ответов
      \correctchoice{$\frac{n}{\theta^2}$}
      \wrongchoice{$n \theta^2$}
      \wrongchoice{$\frac{\theta^2}{n}$}
      \wrongchoice{$\frac{\theta}{n}$}
      \wrongchoice{$\frac{n}{\theta}$}
      \end{choices}
  \end{multicols}
  \end{questionmult}
}


\element{prob_estimators}{ % в фигурных скобках название группы вопросов
 \AMCcompleteMulti
  \begin{questionmult}{3} % тип вопроса (questionmult --- множественный выбор) и в фигурных --- номер вопроса
Пусть $X = (X_1, \ldots , X_n)$ --- случайная выборка из равномерного на $(0, \theta)$ распределения. При каком значении константы $c$ оценка  $\hat{\theta} = c \bar{X}$ является несмещённой?
 \begin{multicols}{3} % располагаем ответы в 3 колонки
   \begin{choices} % опция [o] не рандомизирует порядок ответов
      \correctchoice{$2$}
      \wrongchoice{$1$}
      \wrongchoice{$\frac{1}{2}$}
      \wrongchoice{$\frac{1}{n}$}
      \wrongchoice{$n$}
      \end{choices}
  \end{multicols}
  \end{questionmult}
}

\element{prob_estimators_rejected}{ % в фигурных скобках название группы вопросов
 \AMCcompleteMulti
  \begin{questionmult}{3} % тип вопроса (questionmult --- множественный выбор) и в фигурных --- номер вопроса
Пусть $X = (X_1, \ldots , X_n)$ --- случайная выборка из биномиального распределения $Bi(5, p)$. При каком значении константы $c$ оценка  $\hat{p} = c \bar{X}$ является несмещённой?
 \begin{multicols}{3} % располагаем ответы в 3 колонки
   \begin{choices} % опция [o] не рандомизирует порядок ответов
      \correctchoice{$\frac{1}{5}$}
      \wrongchoice{$5$}
      \wrongchoice{$1$}
      \wrongchoice{$\frac{1}{n}$}
      \wrongchoice{$n$}
      \end{choices}
  \end{multicols}
  \end{questionmult}
}

\element{prob_estimators}{ % в фигурных скобках название группы вопросов
 \AMCcompleteMulti
  \begin{questionmult}{4} % тип вопроса (questionmult --- множественный выбор) и в фигурных --- номер вопроса
Последовательность оценок $\hat{\theta}_1, \hat{\theta}_2, ...$ называется состоятельной, если
 \begin{multicols}{2} % располагаем ответы в 3 колонки
   \begin{choices} % опция [o] не рандомизирует порядок ответов
      \correctchoice{$P(|\hat{\theta}_n - \theta | > t) \to 0$ для всех $t > 0$}
      \wrongchoice{$\Var(\hat{\theta}_n) \geq Var(\hat{\theta}_{n + 1})$}
      \wrongchoice{$\Var(\hat{\theta}_n) \to 0$}
      \wrongchoice{$\E(\hat{\theta}_n) = \theta$}
      \wrongchoice{$\E(\hat{\theta}_n) \to \theta$}
      \end{choices}
  \end{multicols}
  \end{questionmult}
  }

\element{prob_estimators_rejected}{ % в фигурных скобках название группы вопросов
 \AMCcompleteMulti
  \begin{questionmult}{4} % тип вопроса (questionmult --- множественный выбор) и в фигурных --- номер вопроса
Пусть $X = (X_1, \ldots , X_n)$ --- случайная выборка из распределения с плотностью
\[
f(x; \theta) =
\begin{cases}
\frac{1}{\theta}\exp(-\frac{x}{\theta}) \text{ при } x \geq 0,  \\
0 \text{ при } x < 0.
\end{cases}
\]
При каком значении константы $c$ оценка  $\hat{\theta} = c \bar{X}$ является несмещённой?
 \begin{multicols}{3} % располагаем ответы в 3 колонки
   \begin{choices} % опция [o] не рандомизирует порядок ответов
      \correctchoice{$1$}
      \wrongchoice{$n$}
      \wrongchoice{$\frac{1}{n}$}
      \wrongchoice{$\frac{n}{n + 1}$}
      \wrongchoice{$\frac{n + 1}{n}$}
      \end{choices}
  \end{multicols}
  \end{questionmult}
}


\element{prob_estimators}{ % в фигурных скобках название группы вопросов
 \AMCcompleteMulti
  \begin{questionmult}{5} % тип вопроса (questionmult --- множественный выбор) и в фигурных --- номер вопроса
Пусть $X = (X_1, \ldots , X_n)$ --- случайная выборка из равномерного на $(0, 2\theta)$ распределения. Оценка $\hat{\theta} = X_1$
 \begin{multicols}{2} % располагаем ответы в 3 колонки
   \begin{choices} % опция [o] не рандомизирует порядок ответов
      \correctchoice{Несмещённая}
      \wrongchoice{Состоятельная}
      \wrongchoice{Эффективная}
      \wrongchoice{Асимптотически нормальная}
      \wrongchoice{Нелинейная}
      \end{choices}
  \end{multicols}
  \end{questionmult}
}

\element{prob_estimators_rejected}{ % в фигурных скобках название группы вопросов
 \AMCcompleteMulti
  \begin{questionmult}{5} % тип вопроса (questionmult --- множественный выбор) и в фигурных --- номер вопроса
Пусть $X = (X_1, \ldots , X_n)$ --- случайная выборка. Случайные величины $X_1, ... , X_n$ имеют дискретное распределение, которое задано при помощи таблицы

\begin{center}
\begin{tabular}{lrrr} \toprule
$X_i$  & -3 & 0 & 2 \\
\midrule
$\P_{X_i}$ & $\frac{2}{3} - \theta$ & $\frac{1}{3}$ & $\theta$\\
\bottomrule
\end{tabular}
\end{center}

При каком значении константы $c$ оценка  $\hat{\theta}_n = c (\bar{X} + 2)$ является несмещённой?
 \begin{multicols}{3} % располагаем ответы в 3 колонки
   \begin{choices} % опция [o] не рандомизирует порядок ответов
      \correctchoice{$\frac{1}{5}$}
      \wrongchoice{$3$}
      \wrongchoice{$\frac{1}{3}$}
      \wrongchoice{$5$}
      \wrongchoice{$1$}
      \end{choices}
  \end{multicols}
  \end{questionmult}
}


\element{prob_estimators_rejected}{ % в фигурных скобках название группы вопросов
 \AMCcompleteMulti
  \begin{questionmult}{6} % тип вопроса (questionmult --- множественный выбор) и в фигурных --- номер вопроса
Пусть $X = (X_1, \ldots , X_n)$ --- случайная выборка. Случайные величины $X_1, ... , X_n$ имеют дискретное распределение, которое задано при помощи таблицы

\begin{center}
\begin{tabular}{lrrr} \toprule
$X_i$  & -4 & 0 & 3 \\
\midrule
$\P_{X_i}$ & $\frac{3}{4} - \theta$ & $\frac{1}{4}$ & $\theta$\\
\bottomrule
\end{tabular}
\end{center}

При каком значении константы $c$ оценка  $\hat{\theta}_n = c (\bar{X} + 3)$ является несмещённой?
 \begin{multicols}{3} % располагаем ответы в 3 колонки
   \begin{choices} % опция [o] не рандомизирует порядок ответов
      \correctchoice{$\frac{1}{6}$}
      \wrongchoice{$4$}
      \wrongchoice{$\frac{1}{4}$}
      \wrongchoice{$6$}
      \wrongchoice{$1$}
      \end{choices}
  \end{multicols}
  \end{questionmult}
}


\element{prob_estimators}{ % в фигурных скобках название группы вопросов
 \AMCcompleteMulti
  \begin{questionmult}{7} % тип вопроса (questionmult --- множественный выбор) и в фигурных --- номер вопроса
Пусть $X = (X_1, \ldots , X_n)$ --- случайная выборка и $I_n(\theta)$ --- информация Фишера. Тогда несмещённая оценка $\hat{\theta}$ называется эффективной, если
 \begin{multicols}{3} % располагаем ответы в 3 колонки
   \begin{choices} % опция [o] не рандомизирует порядок ответов
      \correctchoice{$\Var(\hat{\theta}) \cdot I_n (\theta) = 1$}
      \wrongchoice{$I^{-1}_n (\theta) \leq \Var(\hat{\theta})$}
      \wrongchoice{$I^{-1}_n (\theta) \geq \Var(\hat{\theta})$}
      \wrongchoice{$\Var(\hat{\theta}) = I_n (\theta)$}
      \wrongchoice{$\Var(\hat{\theta}) \leq I_n (\theta)$}
      \end{choices}
  \end{multicols}
  \end{questionmult}
}

\element{prob_estimators}{ % в фигурных скобках название группы вопросов
 \AMCcompleteMulti
  \begin{questionmult}{8} % тип вопроса (questionmult --- множественный выбор) и в фигурных --- номер вопроса
Пусть $X = (X_1, \ldots , X_n)$ --- случайная выборка и $\ell(\theta) = \ell(X_1, ... , X_n; \theta)$ --- логарифмическая функция правдоподобия. Тогда информация Фишера $I_n(\theta)$ равна
 \begin{multicols}{3} % располагаем ответы в 3 колонки
   \begin{choices} % опция [o] не рандомизирует порядок ответов
      \correctchoice{$-\E \left( \frac{\partial^2 \ell (\theta)}{\partial \theta^2} \right)$}
      \wrongchoice{$-\E \left( \frac{\partial \ell (\theta)}{\partial \theta} \right)$}
      \wrongchoice{$\E \left( \frac{\partial \ell (\theta)}{\partial \theta} \right)$}
      \wrongchoice{$\E \left( \frac{\partial^2 \ell (\theta)}{\partial \theta^2} \right)$}
      \wrongchoice{$-\E \left( \left( \frac{\partial \ell (\theta)}{\partial \theta} \right) ^2 \right)$}
      \end{choices}
  \end{multicols}
  \end{questionmult}
}

\element{prob_estimators}{ % в фигурных скобках название группы вопросов
 \AMCcompleteMulti
  \begin{questionmult}{9} % тип вопроса (questionmult --- множественный выбор) и в фигурных --- номер вопроса
Пусть $X = (X_1, \ldots , X_n)$ --- случайная выборка и $\ell(\theta) = \ell(X_1, ... , X_n; \theta)$ --- логарифмическая функция правдоподобия. Тогда информация Фишера $I_n(\theta)$ равна
 \begin{multicols}{3} % располагаем ответы в 3 колонки
   \begin{choices} % опция [o] не рандомизирует порядок ответов
      \correctchoice{$\E \left( \left( \frac{\partial \ell (\theta)}{\partial \theta} \right) ^2 \right)$}
      \wrongchoice{$\E \left( \frac{\partial \ell (\theta)}{\partial \theta} \right)$}
      \wrongchoice{$- \E \left( \frac{\partial \ell (\theta)}{\partial \theta} \right)$}
      \wrongchoice{$ \E \left( \frac{\partial^2 \ell (\theta)}{\partial \theta^2} \right)$}
      \wrongchoice{$- \E \left( \frac{\partial \ell (\theta)}{\partial \theta} \cdot \frac{\partial \ell (\theta)}{\partial \theta} \right)$}
      \end{choices}
  \end{multicols}
  \end{questionmult}
}
