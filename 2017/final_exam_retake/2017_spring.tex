\element{prob_kolya}{ % в фигурных скобках название группы вопросов
 \AMCcompleteMulti
  \begin{questionmult}{1} % тип вопроса (questionmult — множественный выбор) и в фигурных — номер вопроса
Ковариационная матрица вектора $X=(X_1,X_2)$ имеет вид
$
\begin{pmatrix}
10 & 3 \\
3 & 10
\end{pmatrix}
$.
Дисперсия разности элементов вектора, $\Var(X_1-X_2)$, равняется
\begin{multicols}{3} % располагаем ответы в {k} колонки
   \begin{choices} % опция [o] не рандомизирует порядок ответов
      \wrongchoice{18}
      \wrongchoice{15}
      \wrongchoice{2}
       \wrongchoice{12}
       \correctchoice{14}
      \end{choices}
 \end{multicols}
  \end{questionmult}
}

\element{prob_kolya}{ % в фигурных скобках название группы вопросов
 \AMCcompleteMulti
  \begin{questionmult}{2} % тип вопроса (questionmult — множественный выбор) и в фигурных — номер вопроса
  Для построения доверительного интервала для разности математических ожиданий по двум независимым нормальным выборкам размера $m$ и $n$ в случае известных и неравных дисперсий используется распределение
 \begin{multicols}{3} % располагаем ответы в 3 колонки
   \begin{choices} % опция [o] не рандомизирует порядок ответов
      \correctchoice{$t_{m+n-2}$}
      \wrongchoice{$t_{m-1,n-1}$}
      \wrongchoice{$t_{m+n}$}
      \wrongchoice{$\chi^2_{m+n-2}$}
      \wrongchoice{$\cN(0;1)$}
  \end{choices}
  \end{multicols}
  \end{questionmult}
}


\element{prob_kolya}{ % в фигурных скобках название группы вопросов
 \AMCcompleteMulti
  \begin{questionmult}{3} % тип вопроса (questionmult — множественный выбор) и в фигурных — номер вопроса
Выборочная доля успехов в некотором испытании составляет $0.3$. Исследователь Ромео хочет, чтобы длина двустороннего 95\%-го доверительного интервала для истинной доли не превышала $0.1$. Количество наблюдений, необходимых для этого, примерно равно
 \begin{multicols}{3} % располагаем ответы в 3 колонки
   \begin{choices} % опция [o] не рандомизирует порядок ответов
      \correctchoice{$322$}
      \wrongchoice{$161$}
      \wrongchoice{$113$}
      \wrongchoice{$225$}
      \wrongchoice{$81$}
      \end{choices}
  \end{multicols}
  \end{questionmult}
}


\element{prob_kolya_rejected}{ % в фигурных скобках название группы вопросов
 \AMCcompleteMulti
  \begin{questionmult}{4} % тип вопроса (questionmult — множественный выбор) и в фигурных — номер вопроса
  Для проверки гипотезы о равенстве математических ожиданий используются две нормальные выборки размером 25 и 16 наблюдений. Разница выборочных средних равна 1. Тестовая статистика НЕ может быть равна
 \begin{multicols}{3} % располагаем ответы в 3 колонки
   \begin{choices} % опция [o] не рандомизирует порядок ответов
     \wrongchoice{$1.36$}
     \wrongchoice{$2.13$}
     \wrongchoice{$1.17$}
     \wrongchoice{$1.85$}
     \wrongchoice{$1.56$}
      \end{choices}
  \end{multicols}
  \end{questionmult}
}


\element{prob_kolya_rejected}{ % в фигурных скобках название группы вопросов
 \AMCcompleteMulti
  \begin{questionmult}{5} % тип вопроса (questionmult — множественный выбор) и в фигурных — номер вопроса
  Для построения доверительного интервала для разности математических ожиданий в двух нормальных выборках размеров $m$ и $n$ при известных и не равных дисперсиях тестовая статистика имеет распределение
 \begin{multicols}{3} % располагаем ответы в 3 колонки
   \begin{choices} % опция [o] не рандомизирует порядок ответов
      \correctchoice{$\cN(0;1)$}
      \wrongchoice{$t_{m-1,n-1}$}
      \wrongchoice{$t_{m+n-2}$}
      \wrongchoice{$\chi^2_{m+n-2}$}
      \wrongchoice{$t_{m+n}$}
      \end{choices}
  \end{multicols}
  \end{questionmult}
}


\element{prob_kolya}{ % в фигурных скобках название группы вопросов
 \AMCcompleteMulti
  \begin{questionmult}{6} % тип вопроса (questionmult — множественный выбор) и в фигурных — номер вопроса
  При проверке гипотезы о равенстве долей можно использовать распределение
 \begin{multicols}{3} % располагаем ответы в 3 колонки
   \begin{choices} % опция [o] не рандомизирует порядок ответов
      \correctchoice{$\cN(0;1)$}
      \wrongchoice{$t_{m-1,n-1}$}
      \wrongchoice{$\chi^2_{m+n-2}$}
      \wrongchoice{$t_{m+n-2}$}
      \wrongchoice{$t_{m+n}$}
      \end{choices}
  \end{multicols}
  \end{questionmult}
}


\element{prob_kolya_rejected}{ % в фигурных скобках название группы вопросов
 \AMCcompleteMulti
  \begin{questionmult}{7} % тип вопроса (questionmult — множественный выбор) и в фигурных — номер вопроса
   При проверке гипотезы о равенстве дисперсий в двух выборках размером в 3 и 5 наблюдений было получено значение тестовой статистики 10. Если оценка дисперсии по одной из выборок равна 8, то другая оценка дисперсии может быть равна
 \begin{multicols}{3} % располагаем ответы в 3 колонки
   \begin{choices} % опция [o] не рандомизирует порядок ответов
      \correctchoice{$4/5$}
      \wrongchoice{$4$}
      \wrongchoice{$4/3$}
      \wrongchoice{$25$}
      \wrongchoice{$1/5$}
      \end{choices}
  \end{multicols}
  \end{questionmult}
}


\element{prob_kolya}{ % в фигурных скобках название группы вопросов
 \AMCcompleteMulti
  \begin{questionmult}{8} % тип вопроса (questionmult — множественный выбор) и в фигурных — номер вопроса
  Пусть $\hat\sigma^2_1$ и  $\hat\sigma^2_2$ — несмещённые оценки дисперсий, полученные по независимым нормальным выборкам размером $m$ и $n$ соответственно. Тогда статистика $\hat\sigma^2_1/\hat\sigma^2_2$ имеет распределение
 \begin{multicols}{3} % располагаем ответы в 3 колонки
   \begin{choices} % опция [o] не рандомизирует порядок ответов
      \wrongchoice{$F_{m+1, n+1}$}
      \wrongchoice{$t_{m+n-2}$}
      \wrongchoice{$\chi^2_{m+n-2}$}
      \wrongchoice{$F_{m,n}$}
      \wrongchoice{$F_{m,n-2}$}
      \end{choices}
  \end{multicols}
  \end{questionmult}
}

\element{prob_kolya}{ % в фигурных скобках название группы вопросов
 \AMCcompleteMulti
  \begin{questionmult}{9} % тип вопроса (questionmult — множественный выбор) и в фигурных — номер вопроса
  Требуется проверить гипотезу о равенстве математических ожиданий по независимым нормальным выборкам размером 33 и 16 наблюдений. Истинные дисперсии по обеим выборкам известны, совпадают и равны 196. Разница выборочных средних равна 1. Тестовая статистика может быть равна
 \begin{multicols}{3} % располагаем ответы в 3 колонки
   \begin{choices} % опция [o] не рандомизирует порядок ответов
      \wrongchoice{$1/2$}
      \wrongchoice{$1/7$}
      \correctchoice{$1/4$}
      \wrongchoice{$1/49$}
      \wrongchoice{$1/14$}
      \end{choices}
  \end{multicols}
  \end{questionmult}
}

\element{prob_kolya_rejected}{ % в фигурных скобках название группы вопросов
 \AMCcompleteMulti
  \begin{questionmult}{10} % тип вопроса (questionmult — множественный выбор) и в фигурных — номер вопроса
  Требуется проверить гипотезу о равенстве математических ожиданий по двум нормальным выборкам размером 33 и 16 наблюдений. Истинные дисперсии по обеим выборкам известны, совпадают и равны 196. Разница выборочных средних равна 1. Тестовая статистика может быть равна
 \begin{multicols}{3} % располагаем ответы в 3 колонки
   \begin{choices} % опция [o] не рандомизирует порядок ответов
     \correctchoice{$-1/2$}
     \wrongchoice{$-1/7$}
     \wrongchoice{$-1/4$}
     \wrongchoice{$-1/49$}
     \wrongchoice{$-1/14$}
   \end{choices}
  \end{multicols}
  \end{questionmult}
}











\element{prob_borya}{ % в фигурных скобках название группы вопросов
 \AMCcompleteMulti
  \begin{questionmult}{1} % тип вопроса (questionmult — множественный выбор) и в фигурных — номер вопроса
  В методе главных компонент
 %\begin{multicols}{3} % располагаем ответы в 3 колонки
   \begin{choices} % опция [o] не рандомизирует порядок ответов
      %\correctchoice{$F_{m-1,n-1}$}
      \correctchoice{выборочная корреляция первой и второй главных компонент равна нулю}
      \wrongchoice{выборочная дисперсия первой главной компоненты минимальна}
      \wrongchoice{первая главная компонента сильнее всего коррелирована с первой переменной}
      \wrongchoice{выборочная корреляция первой и второй главных компонент равна единице}
      \wrongchoice{выборочная дисперсия первой главной компоненты равна единице }
      \end{choices}
  %\end{multicols}
  \end{questionmult}
}

\element{prob_borya}{ % в фигурных скобках название группы вопросов
 \AMCcompleteMulti
   \begin{questionmult}{3} % тип вопроса (questionmult — множественный выбор) и в фигурных — номер вопроса
  Перед Васей случайная выборка размера $n$ с истинной функцией распределения $F(x)$. Выберите верное утверждение про эмпирическую функцию распределения $F_n(x)$
 %\begin{multicols}{3} % располагаем ответы в 3 колонки
   \begin{choices} % опция [o] не рандомизирует порядок ответов
      \wrongchoice{$F_n(x)$ является невозрастающей функцией}
      %\wrongchoice{$F_n(x)$ является состоятельной оценкой функции распределения $F(x)$}
      \wrongchoice{$F_n(x)$ непрерывна в каждой точке вариационного ряда}
      \wrongchoice{$F_n(x)$ асимптотически имеет $F$-распределение}
      \correctchoice{$\E(F_n(x))=F(x)$}
      \wrongchoice{$\Var(F_n(x))=F(x)(1-F(x))/n$}
      \end{choices}
  %\end{multicols}
  \end{questionmult}
}


\element{prob_borya}{ % в фигурных скобках название группы вопросов
 \AMCcompleteMulti
  \begin{questionmult}{3} % тип вопроса (questionmult — множественный выбор) и в фигурных — номер вопроса
  Величины $X_1$, $X_2$, \ldots, $X_{10}$ представляют собой случайную выборку с $\E(X_i) = 2\theta - 1$. Оказалось, что $\bar X_{10}=3$. Оценка $\hat\theta_{MM}$ метода моментов равна
 \begin{multicols}{3} % располагаем ответы в 3 колонки
   \begin{choices} % опция [o] не рандомизирует порядок ответов
      \correctchoice{$2$}
      \wrongchoice{$1$}
      \wrongchoice{$3$}
      \wrongchoice{$15.5$}
      \wrongchoice{Недостаточно данных}
      \end{choices}
  \end{multicols}
  \end{questionmult}
}


\element{prob_borya}{ % в фигурных скобках название группы вопросов
 \AMCcompleteMulti
  \begin{questionmult}{4} % тип вопроса (questionmult — множественный выбор) и в фигурных — номер вопроса
    Величины $X_1$, $X_2$, \ldots, $X_{10}$ представляют собой случайную выборку с $\E(X_i) = 2\theta - 1$. Оказалось, что $\bar X_{10}=3$. Оценка $\hat\theta_{ML}$ метода максимального правдоподобия равна
   \begin{multicols}{3} % располагаем ответы в 3 колонки
     \begin{choices} % опция [o] не рандомизирует порядок ответов
        \correctchoice{Недостаточно данных}
        \wrongchoice{$1$}
        \wrongchoice{$3$}
        \wrongchoice{$15.5$}
        \wrongchoice{$2$}
      \end{choices}
  \end{multicols}
  \end{questionmult}
}


\element{prob_borya}{ % в фигурных скобках название группы вопросов
 \AMCcompleteMulti
  \begin{questionmult}{5} % тип вопроса (questionmult — множественный выбор) и в фигурных — номер вопроса
  Нелогарифмированная функция правдоподобия
 \begin{multicols}{2} % располагаем ответы в 3 колонки
   \begin{choices} % опция [o] не рандомизирует порядок ответов
      \correctchoice{может принимать значения больше единицы}
      \wrongchoice{может принимать отрицательные значения}
      \wrongchoice{возрастает по оцениваемому параметру $\theta$}
      \wrongchoice{убывает по оцениваемому параметру $\theta$}
      \wrongchoice{асимпотитически распределена $\cN(0;1)$}
      \end{choices}
  \end{multicols}
  \end{questionmult}
}


\element{prob_borya}{ % в фигурных скобках название группы вопросов
 \AMCcompleteMulti
  \begin{questionmult}{6} % тип вопроса (questionmult — множественный выбор) и в фигурных — номер вопроса
  Оценка метода моментов
 \begin{multicols}{2} % располагаем ответы в 3 колонки
   \begin{choices} % опция [o] не рандомизирует порядок ответов
      \correctchoice{не требует знания точного закона распределения}
      \wrongchoice{всегда несмещённая}
      \wrongchoice{эффективнее оценки максимального правдоподобия}
      \wrongchoice{не может быть получена в малой выборке}
      \wrongchoice{не применима для дискретных случайных величин}
      \end{choices}
  \end{multicols}
  \end{questionmult}
}


\element{prob_borya}{ % в фигурных скобках название группы вопросов
 \AMCcompleteMulti
  \begin{questionmult}{7} % тип вопроса (questionmult — множественный выбор) и в фигурных — номер вопроса
   По большой выборке была построена оценка  максимального правдоподобия $\hat a$. Оказалось, что $\ell''(\hat a) = -16$. Ширина 95\%-го доверительного интервала для параметра $a$ примерно равна
 \begin{multicols}{3} % располагаем ответы в 3 колонки
   \begin{choices} % опция [o] не рандомизирует порядок ответов
      \correctchoice{$1$}
      \wrongchoice{$2$}
      \wrongchoice{$3$}
      \wrongchoice{$4$}
      \wrongchoice{$5$}
      \end{choices}
  \end{multicols}
  \end{questionmult}
}


\element{prob_borya_rejected}{ % в фигурных скобках название группы вопросов
 \AMCcompleteMulti
  \begin{questionmult}{8} % тип вопроса (questionmult — множественный выбор) и в фигурных — номер вопроса
  Величины $X_1$, $X_2$, \ldots, $X_n$ представляют собой случайную выборку из $\cN(\mu; \sigma^2)$. Вася оценивает оба параметра с помощью максимального правдоподобия. При этом
 \begin{multicols}{3} % располагаем ответы в 3 колонки
   \begin{choices} % опция [o] не рандомизирует порядок ответов
      \correctchoice{$\E(\hat \mu)=\mu$, $\E(\hat\sigma^2) < \sigma^2$}
      \wrongchoice{$\E(\hat \mu)=\mu$, $\E(\hat\sigma^2) = \sigma^2$}
      \wrongchoice{$\E(\hat \mu)=\mu$, $\E(\hat\sigma^2) > \sigma^2$}
      \wrongchoice{$\E(\hat \mu)>\mu$, $\E(\hat\sigma^2) = \sigma^2$}
      \wrongchoice{$\E(\hat \mu)<\mu$, $\E(\hat\sigma^2) = \sigma^2$}
      \end{choices}
  \end{multicols}
  \end{questionmult}
}

\element{prob_borya_rejected}{ % в фигурных скобках название группы вопросов
 \AMCcompleteMulti
  \begin{questionmult}{9} % тип вопроса (questionmult — множественный выбор) и в фигурных — номер вопроса
    Если величина $\hat\theta$ имеет нормальное распределение $\cN(3;0.01^2)$, то, согласно дельта-методу, $\hat\theta^3$ имеет примерно нормальное распределение
   \begin{multicols}{3} % располагаем ответы в 3 колонки
     \begin{choices} % опция [o] не рандомизирует порядок ответов
        \correctchoice{$\cN(27;27^2\cdot 0.01^2)$}
        \wrongchoice{$\cN(4;16\cdot 0.01^2)$}
        \wrongchoice{$\cN(27;27\cdot 0.01^2)$}
        \wrongchoice{$\cN(3;3\cdot 0.01^2)$}
        \wrongchoice{$\cN(27;3\cdot 0.01^2)$}
      \end{choices}
  \end{multicols}
  \end{questionmult}
}

\element{prob_borya}{ % в фигурных скобках название группы вопросов
 \AMCcompleteMulti
  \begin{questionmult}{10} % тип вопроса (questionmult — множественный выбор) и в фигурных — номер вопроса
    Есть два неизвестных параметра, $\theta$ и $\gamma$. Вася проверяет гипотезу $H_0$: $\theta = 1$ и $\gamma = 2$ против альтернативной гипотезы о том, что хотя бы одно из равенств нарушено. Выберите верное утверждение об асимптотическом распределении статистики отношения правдоподобия, $LR$:

   \begin{multicols}{2} % располагаем ответы в 3 колонки
     \begin{choices} % опция [o] не рандомизирует порядок ответов
       \correctchoice{Если верна $H_0$, то $LR \sim \chi_2^2$}
       \wrongchoice{Если верна $H_0$, то $LR \sim \chi_1^2$}
       \wrongchoice{Если верна $H_a$, то $LR \sim \chi_2^2$}
       \wrongchoice{И при $H_0$, и при $H_a$, $LR \sim \chi_1^2$}
       \wrongchoice{И при $H_0$, и при $H_a$, $LR \sim \chi_2^2$}
   \end{choices}
  \end{multicols}
  \end{questionmult}
}


















\element{prob_dima}{ % в фигурных скобках название группы вопросов
 \AMCcompleteMulti
  \begin{questionmult}{1} % тип вопроса (questionmult — множественный выбор) и в фигурных — номер вопроса
  Пусть $X = (X_1, \, \ldots, \, X_n)$ — случайная выборка из распределения Пуассона с параметром $\lambda > 0$. Информация Фишера о параметре $\lambda$, заключенная во всех наблюдениях случайной выборки, равна
 \begin{multicols}{3} % располагаем ответы в 3 колонки
   \begin{choices} % опция [o] не рандомизирует порядок ответов
      %\correctchoice{$F_{m-1,n-1}$}
      \wrongchoice{$1 / \lambda$}
      \wrongchoice{$e^{-\lambda}$}
      \correctchoice{ $n / \lambda$}
      \wrongchoice{$\lambda / n$}
      \wrongchoice{$\lambda$}
      \end{choices}
  \end{multicols}
  \end{questionmult}
}

\element{prob_dima_rejected}{ % в фигурных скобках название группы вопросов
 \AMCcompleteMulti
  \begin{questionmult}{2} % тип вопроса (questionmult — множественный выбор) и в фигурных — номер вопроса
  Пусть $X = (X_1, \, \ldots, \, X_n)$ — случайная выборка из распределения Бернулли с параметром $p \in (0;\,1)$. Информация Фишера о параметре $p$, заключенная в~\textsc{одном} наблюдении случайной выборки, равна
 \begin{multicols}{3} % располагаем ответы в 3 колонки
   \begin{choices} % опция [o] не рандомизирует порядок ответов
      \correctchoice{$\frac{1}{p(1-p)}$}
      \wrongchoice{$p$}
      \wrongchoice{$1/p$}
      \wrongchoice{$p/n$}
      \wrongchoice{$n/p$}
  \end{choices}
  \end{multicols}
  \end{questionmult}
}


\element{prob_dima_rejected}{ % в фигурных скобках название группы вопросов
 \AMCcompleteMulti
  \begin{questionmult}{3} % тип вопроса (questionmult — множественный выбор) и в фигурных — номер вопроса
  Пусть $X = (X_1, \, \ldots, \, X_n)$ — случайная выборка из нормального распределения с математическим ожиданием $\mu$ и дисперсией $\sigma^2 = 3$. Информация Фишера о параметре $\mu$, заключенная в \textsc{двух} наблюдениях случайной выборки, равна
 \begin{multicols}{3} % располагаем ответы в 3 колонки
   \begin{choices} % опция [o] не рандомизирует порядок ответов
      \correctchoice{$2 / 3$}
      \wrongchoice{$\mu / 2$}
      \wrongchoice{$2 / \mu$}
      \wrongchoice{$3 / 2$}
      \wrongchoice{$2 \mu^2$}
      \end{choices}
  \end{multicols}
  \end{questionmult}
}


\element{prob_dima_rejected}{ % в фигурных скобках название группы вопросов
 \AMCcompleteMulti
  \begin{questionmult}{4} % тип вопроса (questionmult — множественный выбор) и в фигурных — номер вопроса
    Пусть $X = (X_1, \, \ldots, \, X_n)$ — случайная выборка из распределения с плотностью распределения
  \[
      f(x; \theta) =
      \begin{cases}
          \frac{1}{\theta} e^{-\frac{x}{\theta}}, \text{ при } x \geq 0, \\
          0, \text{ при } x < 0
      \end{cases},
  \]
  где $\theta > 0$ — неизвестный параметр распределения. Информация Фишера о параметре $\theta$, заключенная в \textsc{трёх} наблюдениях случайной выборки, равна
 \begin{multicols}{3} % располагаем ответы в 3 колонки
   \begin{choices} % опция [o] не рандомизирует порядок ответов
     \correctchoice{$3 / \theta^2$}
     \wrongchoice{$\theta$}
     \wrongchoice{$1 / \theta$}
     \wrongchoice{$\theta^2 / 3$}
     \wrongchoice{$\theta^2$}
      \end{choices}
  \end{multicols}
  \end{questionmult}
}


\element{prob_dima}{ % в фигурных скобках название группы вопросов
 \AMCcompleteMulti
  \begin{questionmult}{5} % тип вопроса (questionmult — множественный выбор) и в фигурных — номер вопроса
  Пусть $\hat{\theta}$ — несмещенная оценка для неизвестного параметра $\theta$, а также выполнены условия регулярности. Неравенство Крамера-Рао представимо в виде
 \begin{multicols}{3} % располагаем ответы в 3 колонки
   \begin{choices} % опция [o] не рандомизирует порядок ответов
      \correctchoice{$I_n^{-1}(\theta) \leq \Var(\hat{\theta})$}
      \wrongchoice{$\Var(\hat{\theta}) \cdot I_n(\theta) > 1$}
      \wrongchoice{$\Var(\hat{\theta}) \cdot I_n(\theta) \leq 1$}
      \wrongchoice{$\Var(\hat{\theta}) \leq I_n(\theta)$}
      \wrongchoice{$I_n(\theta) \leq \Var(\hat{\theta})$}
      \end{choices}
  \end{multicols}
  \end{questionmult}
}


\element{prob_dima_rejected}{ % в фигурных скобках название группы вопросов
 \AMCcompleteMulti
  \begin{questionmult}{6} % тип вопроса (questionmult — множественный выбор) и в фигурных — номер вопроса
    Пусть $X = (X_1, \, \ldots, \, X_n)$ — случайная выборка из дискретного распределения с таблицей распределения

    \vspace{5mm}
    \begin{tabular}{cccc}
    \toprule
      $X_i$  & $-2$    & $0$      & $1$  \\
      \midrule
      $\P(\cdot)$        & $1/2 - \theta$      & $1/2$    & $\theta$  \\
    \bottomrule
    \end{tabular}
    \vspace{5mm}


    Несмещённой является оценка
 \begin{multicols}{3} % располагаем ответы в 3 колонки
   \begin{choices} % опция [o] не рандомизирует порядок ответов
      \correctchoice{$(\bar{X} + 1) / 3$}
      \wrongchoice{$\bar{X}$}
      \wrongchoice{$\bar{X} + 1$}
      \wrongchoice{$\bar{X} - 1$}
      \wrongchoice{$(\bar{X} - 1) / 3$}
      \end{choices}
  \end{multicols}
  \end{questionmult}
}


\element{prob_dima}{ % в фигурных скобках название группы вопросов
 \AMCcompleteMulti
  \begin{questionmult}{7} % тип вопроса (questionmult — множественный выбор) и в фигурных — номер вопроса
   Пусть $X = (X_1, \, \ldots, \, X_n)$ — случайная выборка из равномерного распределения на отрезке $[0; \, \theta]$, где $\theta > 0$ — неизвестный параметр. Несмещённой является оценка
 \begin{multicols}{3} % располагаем ответы в 3 колонки
   \begin{choices} % опция [o] не рандомизирует порядок ответов
      \correctchoice{$2 \bar{X}$}
      \wrongchoice{ $\bar{X}$}
      \wrongchoice{$\bar{X} / 2$}
      \wrongchoice{$X_{(1)}$}
      \wrongchoice{ $X_{1}$}
      \end{choices}
  \end{multicols}
  \end{questionmult}
}


\element{prob_dima_rejected}{ % в фигурных скобках название группы вопросов
 \AMCcompleteMulti
  \begin{questionmult}{8} % тип вопроса (questionmult — множественный выбор) и в фигурных — номер вопроса
    Пусть $X = (X_1, \, \ldots, \, X_n)$ — случайная выборка из дискретного распределения с таблицей распределения

  \vspace{5mm}
  \begin{tabular}{cccc}
    \toprule
    $X_i$    & $-2$     & $0$   & $1$  \\
    \midrule
    $\P(\cdot)$        & $1/2 - \theta$      & $1/2$    & $\theta$  \\
    \bottomrule
  \end{tabular}
  \vspace{5mm}


  Состоятельной является оценка
 \begin{multicols}{3} % располагаем ответы в 3 колонки
   \begin{choices} % опция [o] не рандомизирует порядок ответов
      \correctchoice{$(\bar{X} + 1) / 3$}
      \wrongchoice{$\bar{X}$}
      \wrongchoice{ $\bar{X} + 1$}
      \wrongchoice{ $\bar{X} - 1$}
      \wrongchoice{$(\bar{X} - 1) / 3$}
      \end{choices}
  \end{multicols}
  \end{questionmult}
}

\element{prob_dima}{ % в фигурных скобках название группы вопросов
 \AMCcompleteMulti
  \begin{questionmult}{9} % тип вопроса (questionmult — множественный выбор) и в фигурных — номер вопроса
  Пусть $X = (X_1, \, \ldots, \, X_n)$ — случайная выборка из равномерного распределения на отрезке $[0; \, \theta]$, где $\theta > 0$ — неизвестный параметр.
  Состоятельной является оценка
 \begin{multicols}{3} % располагаем ответы в 3 колонки
   \begin{choices} % опция [o] не рандомизирует порядок ответов
      \correctchoice{$2 \bar{X}$}
      \wrongchoice{$\bar{X}$}
      \wrongchoice{$\bar{X} / 2$}
      \wrongchoice{$X_{(1)}$}
      \wrongchoice{$X_{1}$}
      \end{choices}
  \end{multicols}
  \end{questionmult}
}

\element{prob_dima_rejected}{ % в фигурных скобках название группы вопросов
 \AMCcompleteMulti
  \begin{questionmult}{10} % тип вопроса (questionmult — множественный выбор) и в фигурных — номер вопроса
  Пусть $X = (X_1, \, \ldots, \, X_n)$ — случайная выборка из нормального распределения с математическим ожиданием $\mu = 3$ и дисперсией $\sigma^2$. Несмещённой оценкой параметра $\sigma^2$ является
 \begin{multicols}{3} % располагаем ответы в 3 колонки
   \begin{choices} % опция [o] не рандомизирует порядок ответов
     \correctchoice{$\frac{1}{n} \sum_{i=1}^{n}(X_i - 3)^2$}
     \wrongchoice{$\frac{1}{n} \sum_{i=1}^{n}(X_i - \bar{X})^2$}
     \wrongchoice{$\frac{1}{n+1} \sum_{i=1}^{n}(X_i - \bar{X})^2$}
     \wrongchoice{$\frac{1}{n-1} \sum_{i=1}^{n}(X_i - 3)^2$}
     \wrongchoice{$\frac{1}{n+1} \sum_{i=1}^{n}(X_i - 3)^2$}
   \end{choices}
  \end{multicols}
  \end{questionmult}
}


\element{prob_dima}{ % в фигурных скобках название группы вопросов
 \AMCcompleteMulti
  \begin{questionmult}{11} % тип вопроса (questionmult — множественный выбор) и в фигурных — номер вопроса
  Оценка  $\hat\theta_n$ называется состоятельной оценкой параметра $\theta$, если
 \begin{multicols}{2} % располагаем ответы в 3 колонки
   \begin{choices} % опция [o] не рандомизирует порядок ответов
     \correctchoice{$\hat\theta_n \xrightarrow{P}\theta$}
     \wrongchoice{$\E(\hat\theta_n)=\theta$}
     \wrongchoice{$\Var(\hat\theta_n)=\frac{\sigma^2}{n}$}
     \wrongchoice{$\Var(\hat\theta_n) \to 0$}
     \wrongchoice{Для любой оценки $T$ из класса $\mathcal{K}$ и любого $\theta$ выполнено $\E((\hat\theta_n-\theta)^2)\leq \E((T-\theta)^2)$}
   \end{choices}
  \end{multicols}
  \end{questionmult}
}


\element{prob_dima}{ % в фигурных скобках название группы вопросов
 \AMCcompleteMulti
  \begin{questionmult}{12} % тип вопроса (questionmult — множественный выбор) и в фигурных — номер вопроса
    Оценка  $\hat\theta_n$ параметра $\theta$ называется эффективной в некотором классе оценок $\mathcal{K}$, если
   \begin{multicols}{2} % располагаем ответы в 3 колонки
     \begin{choices} % опция [o] не рандомизирует порядок ответов
       \wrongchoice{$\hat\theta_n \xrightarrow{P}\theta$}
       \wrongchoice{$\E(\hat\theta_n)=\theta$}
       \wrongchoice{$\Var(\hat\theta_n)=\frac{\sigma^2}{n}$}
       \wrongchoice{$\Var(\hat\theta_n) \to 0$}
       \correctchoice{Для любой оценки $T$ из класса $\mathcal{K}$ и любого $\theta$ выполнено $\E((\hat\theta_n-\theta)^2)\leq \E((T-\theta)^2)$}
   \end{choices}
  \end{multicols}
  \end{questionmult}
}




















\element{prob_ivan}{ % в фигурных скобках название группы вопросов
 \AMCcompleteMulti
  \begin{questionmult}{1} % тип вопроса (questionmult — множественный выбор) и в фигурных — номер вопроса
Если P-значение (P-value) больше уровня значимости  $\alpha$, то гипотеза  $H_0: \; \sigma=1$
\begin{multicols}{2} % располагаем ответы в {k} колонки
   \begin{choices} % опция [o] не рандомизирует порядок ответов
      \wrongchoice{Отвергается}
      \wrongchoice{Отвергается, только если  $H_a: \; \sigma>1$}
      \wrongchoice{Отвергается, только если  $H_a: \; \sigma\neq 1$}
       \wrongchoice{ Отвергается, только если  $H_a: \; \sigma<1$}
       \correctchoice{Не отвергается}
      \end{choices}
  \end{multicols}
  \end{questionmult}
}



\element{prob_ivan_rejected}{ % в фигурных скобках название группы вопросов
 \AMCcompleteMulti
  \begin{questionmult}{2} % тип вопроса (questionmult — множественный выбор) и в фигурных — номер вопроса

По случайной выборке из 49 наблюдений было оценено выборочное среднее $\bar{X} = 8$  и несмещённая оценка дисперсии $\hat{\sigma}^2 = 4$ проверяется гипотеза $H_0: \mu = 7$ против $H_a: \mu \ne 7$. Тогда значение тестовой статистики

 \begin{multicols}{3} % располагаем ответы в 3 колонки
   \begin{choices} % опция [o] не рандомизирует порядок ответов
      \correctchoice{3.5}
      \wrongchoice{1.75}
      \wrongchoice{3}
      \wrongchoice{1.5}
      \wrongchoice{-1.75}
      \end{choices}
  \end{multicols}
  \end{questionmult}
}



\element{prob_ivan_rejected}{ % в фигурных скобках название группы вопросов
 \AMCcompleteMulti
  \begin{questionmult}{3} % тип вопроса (questionmult — множественный выбор) и в фигурных — номер вопроса

По выборке из 100 наблюдений $X_1,\ldots,X_{n}$, имеющей нормальное распределение с неизвестной дисперсией, был получен 95\% доверительный интервал для математического ожидания $[16,24]$. Значит, $\bar{X}$ был равен

 \begin{multicols}{3} % располагаем ответы в 3 колонки
   \begin{choices} % опция [o] не рандомизирует порядок ответов
      \correctchoice{20}
      \wrongchoice{18}
      \wrongchoice{20.5}
      \wrongchoice{19}
      \wrongchoice{21}
      \end{choices}
  \end{multicols}
  \end{questionmult}
}




\element{prob_ivan_rejected}{ % в фигурных скобках название группы вопросов
 \AMCcompleteMulti
  \begin{questionmult}{4} % тип вопроса (questionmult — множественный выбор) и в фигурных — номер вопроса

По выборке из 100 наблюдений $X_1,\ldots,X_{n}$, имеющей нормальное распределение с неизвестной дисперсией, был получен 95\% доверительный интервал для математического ожидания $[16,24]$. Считая критическое значение t-статистики равным 2, несмещенная оценка дисперсии была равна

 \begin{multicols}{3} % располагаем ответы в 3 колонки
   \begin{choices} % опция [o] не рандомизирует порядок ответов
      \correctchoice{400}
      \wrongchoice{18}
      \wrongchoice{3}
      \wrongchoice{1.5}
      \wrongchoice{-1.75}
      \end{choices}
  \end{multicols}
  \end{questionmult}
}




\element{prob_ivan}{ % в фигурных скобках название группы вопросов
 \AMCcompleteMulti
  \begin{questionmult}{5} % тип вопроса (questionmult — множественный выбор) и в фигурных — номер вопроса

По выборке из 5 наблюдений $X_1,\ldots,X_{5}$, имеющей экспоненциальное распределение, для проверки гипотезы о математическом ожидании $H_0: \mu = \mu_0$ против $H_a: \mu \ne \mu_0$, можно считать, что величина $\frac{\bar{X} - \mu_0}{\sqrt{\hat{\sigma}^2 / n}}$ имеет распределение

 \begin{multicols}{3} % располагаем ответы в 3 колонки
   \begin{choices} % опция [o] не рандомизирует порядок ответов
      \wrongchoice{$\cN(0,1)$}
      \wrongchoice{$t_4$}
      \wrongchoice{$t_5$}
      \wrongchoice{$\chi^2_5$}
      \wrongchoice{$\chi^2_4$}
   \end{choices}
  \end{multicols}
  \end{questionmult}
}





\element{prob_ivan_rejected}{ % в фигурных скобках название группы вопросов
 \AMCcompleteMulti
  \begin{questionmult}{6} % тип вопроса (questionmult — множественный выбор) и в фигурных — номер вопроса

Вася 50 раз подбросил монетку, 23 раза она выпала «орлом», 27 раз — «решкой». При проверке гипотезы о том, что монетка — «честная», Вася будет пользоваться статистикой, имеющей распределение

 \begin{multicols}{3} % располагаем ответы в 3 колонки
   \begin{choices} % опция [o] не рандомизирует порядок ответов
      \correctchoice{$\cN(0,1)$}
      \wrongchoice{$t_{50}$}
      \wrongchoice{$t_{49}$}
      \wrongchoice{$t_{51}$}
      \wrongchoice{$\chi^2_{49}$}
      \end{choices}
  \end{multicols}
  \end{questionmult}
}




\element{prob_ivan}{ % в фигурных скобках название группы вопросов
 \AMCcompleteMulti
  \begin{questionmult}{7} % тип вопроса (questionmult — множественный выбор) и в фигурных — номер вопроса

Вася 25 раз подбросил монетку, 10 раз она выпала «орлом», 15 раз — «решкой». При проверке гипотезы о том, что монетка — «честная», Вася может получить следующее значение тестовой статистики

 \begin{multicols}{3} % располагаем ответы в 3 колонки
   \begin{choices} % опция [o] не рандомизирует порядок ответов
      \correctchoice{-1}
      \wrongchoice{0.4}
      \wrongchoice{1.02}
      \wrongchoice{-1.02}
      \wrongchoice{2}
      \end{choices}
  \end{multicols}
  \end{questionmult}
}




\element{prob_ivan_rejected}{ % в фигурных скобках название группы вопросов
 \AMCcompleteMulti
  \begin{questionmult}{8} % тип вопроса (questionmult — множественный выбор) и в фигурных — номер вопроса

По выборке $X_1,\ldots,X_{n}$, имеющей нормальное распределение с неизвестным математическим ожиданием, строится доверительный интервал для дисперсии. Он НЕ может иметь вид

 \begin{multicols}{3} % располагаем ответы в 3 колонки
   \begin{choices} % опция [o] не рандомизирует порядок ответов
      \correctchoice{$(-\infty, a)$}
      \wrongchoice{$(0, a)$}
      \wrongchoice{$(a, b)$}
      \wrongchoice{$(b, +\infty)$}
      \wrongchoice{$(0, +\infty)$}
      \end{choices}
  \end{multicols}
  \end{questionmult}
}



\element{prob_ivan}{ % в фигурных скобках название группы вопросов
 \AMCcompleteMulti
  \begin{questionmult}{9} % тип вопроса (questionmult — множественный выбор) и в фигурных — номер вопроса

По выборке $X_1,\ldots,X_{n}$, имеющей нормальное распределение с неизвестным математическим ожиданием, проверяется гипотеза о дисперсии $H_0: \sigma^2 = 30$ против $H_a: \sigma^2 \ne 30$. Известно, что $\sum_{i=1}^{n} (X_i - \bar{X})^2 = 270$. Тестовая статистика может быть равна

 \begin{multicols}{3} % располагаем ответы в 3 колонки
   \begin{choices} % опция [o] не рандомизирует порядок ответов
      \correctchoice{9}
      \wrongchoice{3}
      \wrongchoice{Не хватает данных}
      \wrongchoice{27}
      \wrongchoice{6}
      \end{choices}
  \end{multicols}
  \end{questionmult}
}






\element{prob_ivan}{ % в фигурных скобках название группы вопросов
 \AMCcompleteMulti
  \begin{questionmult}{10} % тип вопроса (questionmult — множественный выбор) и в фигурных — номер вопроса

По выборке $X_1,\ldots,X_{n}$, имеющей нормальное распределение с неизвестным математическим ожиданием, проверяется гипотеза о дисперсии $H_0: \sigma^2 = 30$ против $H_a: \sigma^2 \ne 30$. Тестовая статистика будет иметь распределение

 \begin{multicols}{3} % располагаем ответы в 3 колонки
   \begin{choices} % опция [o] не рандомизирует порядок ответов
      \correctchoice{$\chi^2_{n-1}$}
      \wrongchoice{$\chi^2_{n}$}
      \wrongchoice{$\cN(0,1)$}
      \wrongchoice{$t_n$}
      \wrongchoice{$t_{n-1}$}
      \end{choices}
  \end{multicols}
  \end{questionmult}
}















\element{prob_ev_rejected}{ % в фигурных скобках название группы вопросов
 \AMCcompleteMulti
  \begin{questionmult}{1} % тип вопроса (questionmult — множественный выбор) и в фигурных — номер вопроса
Дана реализация выборки: 7, -1, 3, 0. Выборочный начальный момент второго порядка равен

 \begin{multicols}{3} % располагаем ответы в 3 колонки
   \begin{choices} % опция [o] не рандомизирует порядок ответов
      \correctchoice{$19.75$}
      \wrongchoice{$-1$}
      \wrongchoice{$0.75$}
      \wrongchoice{$2.25$}
      \wrongchoice{$59$}
      \end{choices}
  \end{multicols}
  \end{questionmult}
}



\element{prob_ev}{ % в фигурных скобках название группы вопросов
 \AMCcompleteMulti
  \begin{questionmult}{2} % тип вопроса (questionmult — множественный выбор) и в фигурных — номер вопроса

Дана реализация выборки: 7, -2, 3, 0. Первая порядковая статистика принимает значение

 \begin{multicols}{3} % располагаем ответы в 3 колонки
   \begin{choices} % опция [o] не рандомизирует порядок ответов
      \correctchoice{$-2$}
      \wrongchoice{$-1$}
      \wrongchoice{$7$}
      \wrongchoice{$0$}
      \wrongchoice{$2.25$}
      \end{choices}
  \end{multicols}
  \end{questionmult}
}



\element{prob_ev}{ % в фигурных скобках название группы вопросов
 \AMCcompleteMulti
  \begin{questionmult}{3} % тип вопроса (questionmult — множественный выбор) и в фигурных — номер вопроса

Дана реализация выборки: 7, -2, 3, 0.  Выборочная функция распределения в точке $(-2)$ принимает значение

 \begin{multicols}{3} % располагаем ответы в 3 колонки
   \begin{choices} % опция [o] не рандомизирует порядок ответов
      \correctchoice{$0.25$}
      \wrongchoice{$0$}
      \wrongchoice{$0.5$}
      \wrongchoice{$0.75$}
      \wrongchoice{$1$}
      \end{choices}
  \end{multicols}
  \end{questionmult}
}




\element{prob_ev}{ % в фигурных скобках название группы вопросов
 \AMCcompleteMulti
  \begin{questionmult}{4} % тип вопроса (questionmult — множественный выбор) и в фигурных — номер вопроса

Трёх случайно выбранных студентов 2-го курса попросили оценить сложность Теории вероятностей и Статистики по 10 бальной шкале

\vspace{5mm}
\begin{tabular}{lrrr}
  \toprule
   & Аким & Ариадна & Темуужин \\
   \midrule
   Теория вероятностей & 6 & 7 & 8 \\
   Статистика & 5 & 6 & 10 \\
  \bottomrule
\end{tabular}
\vspace{5mm}


Тест знаков отвергает гипотезу о том, что Статистика и Теории вероятностей одинаково сложны в пользу альтернативы, что Статистика проще при уровне значимости

 \begin{multicols}{3} % располагаем ответы в 3 колонки
   \begin{choices} % опция [o] не рандомизирует порядок ответов
      \correctchoice{$1/2$}
      \wrongchoice{$3/8$}
      \wrongchoice{$1/3$}
      \wrongchoice{$0.05$}
      \wrongchoice{$0.1$}
      \end{choices}
  \end{multicols}
  \end{questionmult}
}




\element{prob_ev}{ % в фигурных скобках название группы вопросов
 \AMCcompleteMulti
  \begin{questionmult}{5} % тип вопроса (questionmult — множественный выбор) и в фигурных — номер вопроса

Преподаватель в течение 10 лет ведет статистику о посещаемости лекций. Он заметил, что перед контрольной работой посещаемость улучшается. Преподаватель составил следующую таблицу сопряженности

\vspace{5mm}
\begin{tabular}{lrr}
  \toprule
   & Контрольная будет & Контрольной не будет \\
   \midrule
   Пришло больше половины курса & 35 & 80 \\
   Пришло меньше половины курса & 5 & 200 \\
  \bottomrule
\end{tabular}
\vspace{5mm}


Если $T$ — статистика Пирсона, а $k$ — число степеней свободы её распределения, то
 \begin{multicols}{3} % располагаем ответы в 3 колонки
   \begin{choices} % опция [o] не рандомизирует порядок ответов
      \correctchoice{$T>52$, $k=1$}
      \wrongchoice{$T<52$, $k=1$}
      \wrongchoice{$T<52$, $k=4$}
      \wrongchoice{$T>52$, $k=2$}
      \wrongchoice{$T>52$, $k=3$}
      \end{choices}
  \end{multicols}
  \end{questionmult}
}





\element{prob_ev}{ % в фигурных скобках название группы вопросов
 \AMCcompleteMulti
  \begin{questionmult}{6} % тип вопроса (questionmult — множественный выбор) и в фигурных — номер вопроса

Экзамен принимают два преподавателя: Б.Б.~Злой и Е.В.~Добрая. Они поставили следующие оценки:


\vspace{5mm}
\begin{tabular}{lccccc}
  \toprule
   Е.В. Добрая & 7 & 6 & 6  & 8 &   \\
   Б.Б. Злой   & 6 & 3 & 10 & 9 & 3 \\
  \bottomrule
\end{tabular}
\vspace{5mm}


Значение статистики Вилкоксона для гипотезы о совпадении распределений оценок равно

 \begin{multicols}{3} % располагаем ответы в 3 колонки
   \begin{choices} % опция [o] не рандомизирует порядок ответов
      \wrongchoice{$22.5$}
      \wrongchoice{$7.5$}
      \wrongchoice{$19$}
      \wrongchoice{$20$}
      \wrongchoice{$20.5$}
      \end{choices}
  \end{multicols}
  \end{questionmult}
}
