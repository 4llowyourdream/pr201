\documentclass[pdftex,12pt,a4paper]{article}

%%%%%%%%%%%%%%%%%%%%%%%  Загрузка пакетов  %%%%%%%%%%%%%%%%%%%%%%%%%%%%%%%%%%

%\usepackage{showkeys} % показывать метки в готовом pdf 

\usepackage{etex} % расширение классического tex
% в частности позволяет подгружать гораздо больше пакетов, чем мы и займёмся далее

\usepackage{cmap} % для поиска русских слов в pdf
\usepackage{verbatim} % для многострочных комментариев
\usepackage{makeidx} % для создания предметных указателей
\usepackage[X2,T2A]{fontenc}
\usepackage[utf8]{inputenc} % задание utf8 кодировки исходного tex файла
\usepackage{setspace}
\usepackage{amsmath,amsfonts,amssymb,amsthm}
\usepackage{mathrsfs} % sudo yum install texlive-rsfs
\usepackage{dsfont} % sudo yum install texlive-doublestroke
\usepackage{array,multicol,multirow,bigstrut} % sudo yum install texlive-multirow
\usepackage{indentfirst} % установка отступа в первом абзаце главы
\usepackage[british,russian]{babel} % выбор языка для документа
\usepackage{bm}
\usepackage{bbm} % шрифт с двойными буквами
\usepackage[perpage]{footmisc}

% создание гиперссылок в pdf
\usepackage[pdftex,unicode,colorlinks=true,urlcolor=blue,hyperindex,breaklinks]{hyperref} 

% свешиваем пунктуацию 
% теперь знаки пунктуации могут вылезать за правую границу текста, при этом текст выглядит ровнее
\usepackage{microtype}

\usepackage{textcomp}  % Чтобы в формулах можно было русские буквы писать через \text{}

% размер листа бумаги
\usepackage[paper=a4paper,top=13.5mm, bottom=13.5mm,left=16.5mm,right=13.5mm,includefoot]{geometry}

\usepackage{xcolor}

\usepackage[pdftex]{graphicx} % для вставки графики 

\usepackage{float,longtable}
\usepackage{soulutf8}

\usepackage{enumitem} % дополнительные плюшки для списков
%  например \begin{enumerate}[resume] позволяет продолжить нумерацию в новом списке

\usepackage{mathtools}
\usepackage{cancel,xspace} % sudo yum install texlive-cancel

\usepackage{minted} % display program code with syntax highlighting
% требует установки pygments и python 

\usepackage{numprint} % sudo yum install texlive-numprint
\npthousandsep{,}\npthousandthpartsep{}\npdecimalsign{.}

\usepackage{embedfile} % Чтобы код LaTeXа включился как приложение в PDF-файл

\usepackage{subfigure} % для создания нескольких рисунков внутри одного

\usepackage{tikz,pgfplots} % язык для рисования графики из latex'a
\usetikzlibrary{trees} % tikz-прибамбас для рисовки деревьев
\usepackage{tikz-qtree} % альтернативный tikz-прибамбас для рисовки деревьев
\usetikzlibrary{arrows} % tikz-прибамбас для рисовки стрелочек подлиннее

\usepackage{todonotes} % для вставки в документ заметок о том, что осталось сделать
% \todo{Здесь надо коэффициенты исправить}
% \missingfigure{Здесь будет Последний день Помпеи}
% \listoftodos --- печатает все поставленные \todo'шки


% более красивые таблицы
\usepackage{booktabs}
% заповеди из докупентации: 
% 1. Не используйте вертикальные линни
% 2. Не используйте двойные линии
% 3. Единицы измерения - в шапку таблицы
% 4. Не сокращайте .1 вместо 0.1
% 5. Повторяющееся значение повторяйте, а не говорите "то же"



%\usepackage{asymptote} % пакет для рисовки графики, должен идти после graphics
% но мы переходим на tikz :)

%\usepackage{sagetex} % для интеграции с Sage (вероятно тоже должен идти после graphics)

% metapost создает упрощенные eps файлы, которые можно напрямую включать в pdf 
% эта группа команд декларирует, что файлы будут этого упрощенного формата
% если metapost не используется, то этот блок не нужен
\usepackage{ifpdf} % для определения, запускается ли pdflatex или просто латех
\ifpdf
	\DeclareGraphicsRule{*}{mps}{*}{}
\fi
%%%%%%%%%%%%%%%%%%%%%%%%%%%%%%%%%%%%%%%%%%%%%%%%%%%%%%%%%%%%%%%%%%%%%%


%%%%%%%%%%%%%%%%%%%%%%%  Внедрение tex исходников в pdf файл  %%%%%%%%%%%%%%%%%%%%%%%%%%%%%%%%%%
\embedfile[desc={Main tex file}]{\jobname.tex} % Включение кода в выходной файл
\embedfile[desc={title_bor}]{/home/boris/science/tex_general/title_bor_utf8.tex}

%%%%%%%%%%%%%%%%%%%%%%%%%%%%%%%%%%%%%%%%%%%%%%%%%%%%%%%%%%%%%%%%%%%%%%



%%%%%%%%%%%%%%%%%%%%%%%  ПАРАМЕТРЫ  %%%%%%%%%%%%%%%%%%%%%%%%%%%%%%%%%%
\setstretch{1}                          % Межстрочный интервал
\flushbottom                            % Эта команда заставляет LaTeX чуть растягивать строки, чтобы получить идеально прямоугольную страницу
\righthyphenmin=2                       % Разрешение переноса двух и более символов
\pagestyle{plain}                       % Нумерация страниц снизу по центру.
\widowpenalty=300                     % Небольшое наказание за вдовствующую строку (одна строка абзаца на этой странице, остальное --- на следующей)
\clubpenalty=3000                     % Приличное наказание за сиротствующую строку (омерзительно висящая одинокая строка в начале страницы)
\setlength{\parindent}{1.5em}           % Красная строка.
%\captiondelim{. }
\setlength{\topsep}{0pt}
%%%%%%%%%%%%%%%%%%%%%%%%%%%%%%%%%%%%%%%%%%%%%%%%%%%%%%%%%%%%%%%%%%%%%%



%%%%%%%% Это окружение, которое выравнивает по центру без отступа, как у простого center
\newenvironment{center*}{%
  \setlength\topsep{0pt}
  \setlength\parskip{0pt}
  \begin{center}
}{%
  \end{center}
}
%%%%%%%%%%%%%%%%%%%%%%%%%%%%%%%%%%%%%%%%%%%%%%%%%%%%%%%%%%%%%%%%%%%%%%


%%%%%%%%%%%%%%%%%%%%%%%%%%% Правила переноса  слов
\hyphenation{ }
%%%%%%%%%%%%%%%%%%%%%%%%%%%%%%%%%%%%%%%%%%%%%%%%%%%%%%%%%%%%%%%%%%%%%%

\emergencystretch=2em


% DEFS
\def \mbf{\mathbf}
\def \msf{\mathsf}
\def \mbb{\mathbb}
\def \tbf{\textbf}
\def \tsf{\textsf}
\def \ttt{\texttt}
\def \tbb{\textbb}

\def \wh{\widehat}
\def \wt{\widetilde}
\def \ni{\noindent}
\def \ol{\overline}
\def \cd{\cdot}
\def \fr{\frac}
\def \bs{\backslash}
\def \lims{\limits}
\DeclareMathOperator{\dist}{dist}
\DeclareMathOperator{\VC}{VCdim}
\DeclareMathOperator{\card}{card}
\DeclareMathOperator{\sign}{sign}
\DeclareMathOperator{\sgn}{sign}
\DeclareMathOperator{\Tr}{\mbf{Tr}}
\def \xfs{(x_1,\ldots,x_{n-1})}
\DeclareMathOperator*{\argmin}{arg\,min}
\DeclareMathOperator*{\amn}{arg\,min}
\DeclareMathOperator*{\amx}{arg\,max}

\DeclareMathOperator{\Corr}{Corr}
\DeclareMathOperator{\Cov}{Cov}
\DeclareMathOperator{\Var}{Var}
\DeclareMathOperator{\corr}{Corr}
\DeclareMathOperator{\cov}{Cov}
\DeclareMathOperator{\var}{Var}
\DeclareMathOperator{\bin}{Bin}
\DeclareMathOperator{\Bin}{Bin}
\DeclareMathOperator{\rang}{rang}
\DeclareMathOperator*{\plim}{plim}
\DeclareMathOperator{\med}{med}


\providecommand{\iff}{\Leftrightarrow}
\providecommand{\hence}{\Rightarrow}

\def \ti{\tilde}
\def \wti{\widetilde}

\def \mA{\mathcal{A}}
\def \mB{\mathcal{B}}
\def \mC{\mathcal{C}}
\def \mE{\mathcal{E}}
\def \mF{\mathcal{F}}
\def \mH{\mathcal{H}}
\def \mL{\mathcal{L}}
\def \mN{\mathcal{N}}
\def \mU{\mathcal{U}}
\def \mV{\mathcal{V}}
\def \mW{\mathcal{W}}


\def \R{\mbb R}
\def \N{\mbb N}
\def \Z{\mbb Z}
\def \P{\mbb{P}}
\def \p{\mbb{P}}
\newcommand{\E}{\mathbb{E}}
\def \D{\msf{D}}
\def \I{\mbf{I}}

\def \a{\alpha}
\def \b{\beta}
\def \t{\tau}
\def \dt{\delta}
\newcommand{\e}{\varepsilon}
\def \ga{\gamma}
\def \kp{\varkappa}
\def \la{\lambda}
\def \sg{\sigma}
\def \sgm{\sigma}
\def \tt{\theta}
\def \ve{\varepsilon}
\def \Dt{\Delta}
\def \La{\Lambda}
\def \Sgm{\Sigma}
\def \Sg{\Sigma}
\def \Tt{\Theta}
\def \Om{\Omega}
\def \om{\omega}

%\newcommand{\p}{\partial}
\newcommand{\PP}{\mathbb{P}}

\def \ni{\noindent}
\def \lq{\glqq}
\def \rq{\grqq}
\def \lbr{\linebreak}
\def \vsi{\vspace{0.1cm}}
\def \vsii{\vspace{0.2cm}}
\def \vsiii{\vspace{0.3cm}}
\def \vsiv{\vspace{0.4cm}}
\def \vsv{\vspace{0.5cm}}
\def \vsvi{\vspace{0.6cm}}
\def \vsvii{\vspace{0.7cm}}
\def \vsviii{\vspace{0.8cm}}
\def \vsix{\vspace{0.9cm}}
\def \VSI{\vspace{1cm}}
\def \VSII{\vspace{2cm}}
\def \VSIII{\vspace{3cm}}

\newcommand{\bls}[1]{\boldsymbol{#1}}
\newcommand{\bsA}{\boldsymbol{A}}
\newcommand{\bsH}{\boldsymbol{H}}
\newcommand{\bsI}{\boldsymbol{I}}
\newcommand{\bsP}{\boldsymbol{P}}
\newcommand{\bsR}{\boldsymbol{R}}
\newcommand{\bsS}{\boldsymbol{S}}
\newcommand{\bsX}{\boldsymbol{X}}
\newcommand{\bsY}{\boldsymbol{Y}}
\newcommand{\bsZ}{\boldsymbol{Z}}
\newcommand{\bse}{\boldsymbol{e}}
\newcommand{\bsq}{\boldsymbol{q}}
\newcommand{\bsy}{\boldsymbol{y}}
\newcommand{\bsbeta}{\boldsymbol{\beta}}
\newcommand{\fish}{\mathrm{F}}
\newcommand{\Fish}{\mathrm{F}}
\renewcommand{\phi}{\varphi}
\newcommand{\ind}{\mathds{1}}
\newcommand{\inds}[1]{\mathds{1}_{\{#1\}}}
\renewcommand{\to}{\rightarrow}
\newcommand{\sumin}{\sum\limits_{i=1}^n}
\newcommand{\ofbr}[1]{\bigl( \{ #1 \} \bigr)}     % Например, вероятность события. Большие круглые, нормальные фигурные скобки вокруг аргумента
\newcommand{\Ofbr}[1]{\Bigl( \bigl\{ #1 \bigr\} \Bigr)} % Например, вероятность события. Больше больших круглые, большие фигурные скобки вокруг аргумента
\newcommand{\oeq}{{}\textcircled{\raisebox{-0.4pt}{{}={}}}{}} % Равно в кружке
\newcommand{\og}{\textcircled{\raisebox{-0.4pt}{>}}}  % Знак больше в кружке

% вместо горизонтальной делаем косую черточку в нестрогих неравенствах
\renewcommand{\le}{\leqslant}
\renewcommand{\ge}{\geqslant}
\renewcommand{\leq}{\leqslant}
\renewcommand{\geq}{\geqslant}


\newcommand{\figb}[1]{\bigl\{ #1  \bigr\}} % большие фигурные скобки вокруг аргумента
\newcommand{\figB}[1]{\Bigl\{ #1  \Bigr\}} % Больше больших фигурные скобки вокруг аргумента
\newcommand{\parb}[1]{\bigl( #1  \bigr)}   % большие скобки вокруг аргумента
\newcommand{\parB}[1]{\Bigl( #1  \Bigr)}   % Больше больших круглые скобки вокруг аргумента
\newcommand{\parbb}[1]{\biggl( #1  \biggr)} % большие-большие круглые скобки вокруг аргумента
\newcommand{\br}[1]{\left( #1  \right)}    % круглые скобки, подгоняемые по размеру аргумента
\newcommand{\fbr}[1]{\left\{ #1  \right\}} % фигурные скобки, подгоняемые по размеру аргумента
\newcommand{\eqdef}{\mathrel{\stackrel{\rm def}=}} % знак равно по определению
\newcommand{\const}{\mathrm{const}}        % const прямым начертанием
\newcommand{\zdt}[1]{\textit{#1}}
\newcommand{\ENG}[1]{\foreignlanguage{british}{#1}}
\newcommand{\ENGs}{\selectlanguage{british}}
\newcommand{\RUSs}{\selectlanguage{russian}}
\newcommand{\iid}{\text{i.\hspace{1pt}i.\hspace{1pt}d.}}

\newdimen\theoremskip
\theoremskip=0pt
\newenvironment{note}{\par\vskip\theoremskip\textbf{Замечание.\xspace}}{\par\vskip\theoremskip}
\newenvironment{hint}{\par\vskip\theoremskip\textbf{Подсказка.\xspace}}{\par\vskip\theoremskip}
\newenvironment{ist}{\par\vskip\theoremskip Источник:\xspace}{\par\vskip\theoremskip}

\newcommand*{\tabvrulel}[1]{\multicolumn{1}{|c}{#1}}
\newcommand*{\tabvruler}[1]{\multicolumn{1}{c|}{#1}}

\newcommand{\II}{{\fontencoding{X2}\selectfont\CYRII}}   % I десятеричное (английская i неуместна)
\newcommand{\ii}{{\fontencoding{X2}\selectfont\cyrii}}   % i десятеричное
\newcommand{\EE}{{\fontencoding{X2}\selectfont\CYRYAT}}  % ЯТЬ
\newcommand{\ee}{{\fontencoding{X2}\selectfont\cyryat}}  % ять
\newcommand{\FF}{{\fontencoding{X2}\selectfont\CYROTLD}} % ФИТА
\newcommand{\ff}{{\fontencoding{X2}\selectfont\cyrotld}} % фита
\newcommand{\YY}{{\fontencoding{X2}\selectfont\CYRIZH}}  % ИЖИЦА
\newcommand{\yy}{{\fontencoding{X2}\selectfont\cyrizh}}  % ижица

%%%%%%%%%%%%%%%%%%%%% Определение разрядки разреженного текста и задание красивых многоточий
\sodef\so{}{.15em}{1em plus1em}{.3em plus.05em minus.05em}
\newcommand{\ldotst}{\so{...}}
\newcommand{\ldotsq}{\so{?\hbox{\hspace{-0.61ex}}..}}
\newcommand{\ldotse}{\so{!..}}
%%%%%%%%%%%%%%%%%%%%%%%%%%%%%%%%%%%%%%%%%%%%%%%%%%%%%%%%%%%%%%%%%%%%%%

%%%%%%%%%%%%%%%%%%%%%%%%%%%%% Команда для переноса символов бинарных операций
\def\hm#1{#1\nobreak\discretionary{}{\hbox{$#1$}}{}}
%%%%%%%%%%%%%%%%%%%%%%%%%%%%%%%%%%%%%%%%%%%%%%%%%%%%%%%%%%%%%%%%%%%%%%

\setlist[enumerate,1]{label=\arabic*., ref=\arabic*, partopsep=0pt plus 2pt, topsep=0pt plus 1.5pt,itemsep=0pt plus .5pt,parsep=0pt plus .5pt}
\setlist[itemize,1]{partopsep=0pt plus 2pt, topsep=0pt plus 1.5pt,itemsep=0pt plus .5pt,parsep=0pt plus .5pt}

% Эти парни затем, если вдруг не захочется управлять списками из-под уютненького enumitem
% или если будет жизненно важно, чтобы в списках были именно русские буквы.
%\setlength{\partopsep}{0pt plus 3pt}
%\setlength{\topsep}{0pt plus 2pt}
%\setlength{\itemsep}{0 plus 1pt}
%\setlength{\parsep}{0 plus 1pt}

%на всякий случай пока есть
%теоремы без нумерации и имени
%\newtheorem*{theor}{Теорема}

%"Определения","Замечания"
%и "Гипотезы" не нумеруются
%\newtheorem*{defin}{Определение}
%\newtheorem*{rem}{Замечание}
%\newtheorem*{conj}{Гипотеза}

%"Теоремы" и "Леммы" нумеруются
%по главам и согласованно м/у собой
%\newtheorem{theorem}{Теорема}
%\newtheorem{lemma}[theorem]{Лемма}

% Утверждения нумеруются по главам
% независимо от Лемм и Теорем
%\newtheorem{prop}{Утверждение}
%\newtheorem{cor}{Следствие} 

\emergencystretch=2em \voffset=-2cm \hoffset=-2cm
\unitlength=0.6mm \textwidth=18cm \textheight=26cm


\title{Преподавание теории вероятностей для <<нематематиков>>}


\begin{document}
%\maketitle

\parindent=0 pt % no indent

За 10 лет преподавания теории вероятностей у меня накопилось несколько идей. Ими и хочу поделиться.

\begin{enumerate}
\item Не бывает событий без случайных величин!

Обычная программа по теории вероятностей начинается примерно так:
\begin{enumerate}
\item События
\item Условная вероятность, независимость событий
\item Случайные величины
\end{enumerate}
Традиция вводить случайные величины после событий ошибочна! Во-первых, преподавателю, следующему этой традиции на первых этапах приходится сознательно избегать понятия <<случайной величины>> и говорить

Найдите $\P( \mbox{количество орлов при 10 подбрасываниях монетки равно 6})$

Вместо
Величина $X$ --- количество орлов при 10 подбрасываниях монетки. Найдите $\P(X=6)$, $\P(X=7)$.

Во-вторых, именно сравнивая понятия <<случайной величины>> и <<события>> проще объяснить, чем они отличаются. В-третьих, большинство вероятностей, которые реально приходится считать, формулируются именно с помощью случайных величин. В-четвертых, введение случайных величин на первом занятии позволяет не терять время и раньше приступить к другим сюжетам! 

\item <<Дерево>> событий заменяет формулу Байеса

Общепринято давать студентам три формулы, связанные с условной вероятности: саму формулу условной вероятности, формулу полной вероятности и формулу Байеса. На мой взгляд, <<дерево>> событий и формула условной вероятности прекрасно заменяют собой все три формулы. 

\item Английский алфавит привычнее греческого

Есть разные традиции обозначений в теории вероятностей: российская --- строчные греческие буквы для случайных величин и $M$ и $D$ для мат. ожидания и дисперсии и европейская (американская?) --- заглавные английские буквы для случайных величин и $E$ и $Var$ для мат. ожидания и дисперсии. Следует предпочесть английский вариант. Во-первых, для большинства студентов английские буквы привычнее греческих. Больше половины студентов второго курса не смогут с первой попытки правильно назвать буквы $\phi$, $\zeta$, $\psi$, и $\eta$! Во-вторых, можно открыть arXiv и посмотреть свежии статьи по теории вероятностей. В моей случайной выборке из 25 статей 24 статьи оказалось в традиции $E$, $Var$.

\item Обозначения должны быть строгими, но без занудства

Граница между строгостью и занудством проходит по моему так:

Вместо строгого обозначения с двойными скобками $P(\{X=1\})$ вполне можно использовать обозначение $P(X=1)$. Для строго изложения в учебнике можно один раз оговорить это в где-нибудь в начале.
Вместе с тем недопустимы $P(A+B+C)$ вместо $P(A\cup B\cup C)$


\item Очень важны примеры с количествами.

С количествами студенту-нематематику работать гораздо проще, чем с дробями. Если задача на условную вероятность с началом  «Вероятность выиграть лотерею равна 0.001...» вызывает трудности, то она станет горазда понятее при простой переделке «Из 10 тысяч лотерейных билетов 10 выигрышных...». Есть статистические исследования, которые говорят, что этот простой метод существенно лучше всех остальных подходов к объяснению условной вероятности, например, \cite{sedlmeier:bayesian_reasoning}.

\item Индикатор события и разложение случайной величины в сумму.

Самая простая случайная величина, индикатор события A, могла бы появляться уже на первой лекции. А вместе с тем во многих курсах ее вообще нет... 

Например, индикатор крайне полезен для  разложения случайной величины в сумму более простых случайных величин. Практика показывает, что этот прием можно использовать на втором занятии. Используя его студенты легко решают задачи вида: «В 20 уток одновременно выстрелили 30 охотников. Каждый выбирал цель наугад и каждый стреляет без промаха. Сколько в среднем уток выжило?»

\item Сюжеты, которые можно рассказать...

Если есть время расширить курс, то имеет смысл рассказывать про Цепи Маркова, броуновское движение, подробнее про метод максимального правдоподобия. Традиционно включают производящие функции, которые студентам-нематематикам практически полностью не нужны. 

\item Проблема нестрогости курса теории вероятностей без теории меры

В курсе теории вероятностей без теории меры можно говорить <<любая функция>> и <<любое числовое подмножество>> вместо <<функция, измеримая по Борелю>>, <<борелевское подмножество>>. В учебнике в определении можно дать сноску. 

\item Независимость и некоррелированность

Большинство учебников так или иначе говорит: из независимости следует некоррелированность. Но мало кто указывает на два простых факта. Независимость величины $X$ и $Y$ --- это то же самое, что некоррелированность $f(X)$ и $g(Y)$ для произвольных  функций $f$ и $g$. Равенство $E(Y|X)=E(X)$ --- это то же самое, что некоррелированность $f(X)$ и $Y$ для произвольной  функции $f$.


\item Разбор популярных интуитивных ошибок.

Есть вопросы, на которые человек не знающий теории вероятностей, скорее всего интуитивно даст неверный ответ. Их обязательно нужно разбирать. Для самого слабого студента это может стать единственной пользой от курса. Самая распространенная бытовая ошибка – это перепутывание условных вероятностей $\P(A|B)$ и $\P(B|A)$. Список других распространенных ошибок можно найти в \cite{fischbein:misconceptions}.

\item Упрощенные таблицы для нормального распределения. 

Традиционные таблицы для нормального распределения могут быть сложным барьером для слабого студента. В ней отсутствуют отрицательные числа, а десятые и сотые доли разнесены по строкам и столбцам. Чтобы посчитать вероятность того, что стандартная нормальная величина лежит от -1.7 до 0.5 нужно выполнить несколько преобразований. При текущей доступности компьютеров можно за 10 минут сделать таблицу нормального распределения, где число целиком, с десятыми и сотыми долями, указано в одном столбике, а значение функции распределения – в соседнем. Причем диапазон аргументов идет от -3 до 3. Может быть имеет смысл попытаться возродить номограммы?

\item Программа-минимум

Бывает, что на курс теории вероятностей выделено ужасающе малое количество времени. Например, для пяти занятий я бы попробовал такую программу-минимум:
\begin{enumerate}
\item События и случайные величины. 
\item Вероятность и среднее.
\item Дисперсия и корреляция.
\item Условная вероятность и условное среднее.
\item Нормальное распределение и центральная предельная теорема.
\end{enumerate}


\item Публикуйте варианты контрольных!

Во-первых, студентам будет на что ориентироваться. Во-вторых, исчезает неравенство: Вася достал прошлогодний вариант, а Петя --- нет. У Васи --- выше балл. Такого не будет! В-третьих, если прошлогодний вариант опубликован поневоле придется сочинить новый и от этого повысится качество преподавания!


\item Не стоит организовывать свой форум в интернете

Лучше сказать студентам форумы, где всегда много народа
\url{math.stackexchange.com}, \url{stats.stackexchange.com} или на русском...
Во-первых, в известных форумах участвует больше людей и студенты быстрее получат ответ на свой вопрос. Во-вторых, студенты увидят, что спрашивают другие. В-третьих, можно сэкономить свое время.


\item Полезные компьютерные программы и технологии.

В моей работе я часто использую:
\begin{enumerate}
\item \LaTeX --- язык программирования для создания документов. 
\item mercurial, tortoisehg, bitbucket --- система управления версиями: дает удобный доступ ко всем версиям текстового документа. 
\item \url{docs.google.com} --- дает возможность одновременного редактирования документов, полезна если несколько семинаристов проверяют контрольные работы
\item \url{www.dropbox.com} --- позволяет создать на своем компьютере папку, которая автоматически синхронизируется с хранилищем в Интернете, удобно для передачи студентам свежей версии лекций или домашних работ
\end{enumerate}

\item Софт связанный с теорией вероятностей.

Современная статистика немыслима без использования компьютеров. При этом для научной работы желательно использовать открытые бесплатные и кроссплатформенные программы. Их использование позволяет сделать любой эксперимент или исследование легко повторимыми. Вот мой выбор:
\begin{enumerate}
\item OpenOffice Calc или Gnumeric – в качестве электронной таблицы
\item Gretl – в качестве программы для эконометрических рассчетов
\item R – язык программирования для статистических рассчетов
\item Geogebra – для рисования графиков и простых анимаций
\item Python --- язык программирования общего назначения, вполне пригодный для экспериментов по теории вероятностей.
\end{enumerate}

\item Для проверки гипотезы о соответствии закона распределения данному или гипотезы о независимости признаков вместо статистики Пирсона лучше использовать статистику отношения правдоподобия.
То есть вместо 
\[ \sum_i \frac{(X_i-np_i)^2}{np_i} \]
использовать
\[ \sum_i X_i(\ln(X_i)-\ln(np_i)) \]
У метода максимального правдоподобия очень много достоинств. Поэтому имеет смысл использовать его, а не узкую формулу подходящую только к этому случаю. Например, сами студенты могут вывести формулу из общей формулы $LR=2(\ln(\hat{l}_{UR})-\ln(\hat{l}_{R}))$. Например, можно легко сравнить между собой три вложенных друг в друга модели...


Кстати, опасно, на мой взгляд, распространенное обозначение 
\[ \sum_i \frac{(O_i-E_i)^2}{E_i} \]
Буква $O$ похожа на ноль, а буква $E$ уже используется для математического ожидания. 


Для гипотезы о независимости признаков я выбираю обозначения:
\[ \sum_{ij} X_{ij}\left(\ln(X_{ij})-\ln\left(\frac{X_{i.}X_{.j}}{n} \right)\right) \]

\item В ряде учебников предлагается ошибочный подход для проверки гипотезы о равенстве дисперсий против альтернативной гипотезе о неравных дисперсиях: <<Следует взять оценки дисперсий в двух выборках и делить большую на меньшую>>. Этот подход является верным только в том случае, если количества наблюдений в выборках совпадают. 


Если размеры выборок отличаются, то верный подход заключается в правильном расчете критических значений $F$-статистики. Правая граница обычно указана в таблице. А левую границу $F_{n,k}^{1-\alpha}$ можно посчитать зная правую границу $F_{k,n}$-распределения по формуле $1/F_{k,n}^{\alpha}$.


\item $H_0$ по английски читается как <<h naught>>.


\end{enumerate}


\bibliographystyle{plain}
\bibliography{opit}


\end{document}
