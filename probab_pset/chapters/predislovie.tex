% !Mode:: "TeX:UTF-8"
%Идеи:
%1. Междисциплинарное взаимодействие \\
%1.1. включить вопросы на тему простых преобразований в известных моделях \\
%CAPM, rational expectation
%1.2. включить что-то на тему инструментальных переменных
%Задана (табличка-сделано!, ф. плотности-?? как бы по проще-то...???) для $X,u$ %придумайте переменную $Z$, которая была бы коррелирована с $X$, но не с $u$
%1.3. calculate partial correlation


%ver 23.09.10, добавлены задачи от Алексея Суздальцева
%ver 11.05.08, задачи про бассейн, оценки для геометрического распределения
%ver 12.03.08, корректоры очепяток, конь на шахматной доске \\
%ver 26.02.08, задача по статистике
%ver 26.04.07, about 500 problems \\
%ver 12.05.07, Расстояние между минимумом и максимумом \\
%Спелестолог и батарейки \\
%ver 18.05.07, minor correction, more small problems \\
%ver 20.05.07, Ковбои \\
%ver 29.05.07, Дополнительные патроны, Четыре шкатулки \\
%ver 10.06.07, Новый шаман \\
%ver 14.07.07, добавляются задачи из wilmott bt, стр 13 \\
%ver 16.10.07, Китайский ресторан

\subsection{Обращение к читателю}
Задачник находится в стадии разработки. Смелее спрашивайте и высказывайте
своё мнение Борису Демешеву, \href{mailto:boris.demeshev@gmail.com}{boris.demeshev@gmail.com}

Предлагайте свои задачи!


\subsection{Цель}
Есть много сборников задач. Зачем этот:

"--* Открытость и доступность. \url{http://demeshev.wordpress.com/materials/}

"--* Красивые задачи.

"--* Задачи под курс НИУ ВШЭ.

Всё то, что можно рассказать без теории меры.


\subsection{Об ответах и упрощениях}
Большинство ответов имеет совсем простой вид в духе
$\frac{a}{a+b}$, и их, очевидно, нельзя упростить. Некоторые ответы
простым образом выражаются через биномиальные коэффициенты. Не
упрощаются, но встречаются в ответах: $\sum_{i=1}^{n}\frac{1}{i}$,
$\sum_{i=1}^{n} i^{k}$. Ответы в виде громоздких сумм биномиальных
коэффициентов не используются, если это не оговорено в условии.
Используется сумма геометрической прогрессии, разложение в ряд
Тейлора для $e^{x}$.


%\textbf{Комменты прочие} \\
%Как пишется Еська-Иська?\\
%Забавные комментарии и рисунки \\
%Отработать классификацию \\
%Как быть если разные пункты задачи относятся к разным разделам? \\
%Открытые вопросы \\
%Ответы \\
%Как называются функции в английском excel? \\
%Разобраться с дефисом и тире\\
%Разобраться с Рыцари близнецы \\
%Отделенные точки и запятые от предыдущего текста \\
%Спросить про копирайт чужих задач \\
%Разбить в задачах с табличками распределения пункты а-б-в на
%отдельные задачи \\
%Задачи на корреляцию \\
% метки (тэги) - у одной задачи может быть несколько меток
% например: нормальное распределение - двойной интеграл

