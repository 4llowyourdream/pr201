% !Mode:: "TeX:UTF-8"
% MM, ML
\section{Метод максимального правдоподобия, метод моментов}

\problem{Падал первый снег и 30 школьников ловили снежинки. В среднем каждый поймал $3{,}3$ снежинки. Количества снежинок пойманные каждым --- независимые Пуассоновские случайные величины с общим параметром $\lambda$. 
\begin{enumerate}
\item Оцените $\lambda$ используя метод максимального правдоподобия
\item Оцените дисперсию полученной оценки
\item На уровне значимости 5\% проверьте гипотезу о том, что $\lambda=3$ используя тест множителей Лагранжа
\item ... используя тест отношения правдоподобия
\item ... используя тест Вальда 
\item Постройте 95\% доверительный интервал Вальда для $\lambda$
\end{enumerate}}
\solution{ Логарифмическая функция правдоподобия
\begin{equation}
l(\lambda)=-n\lambda+\sum x_i \ln\lambda-\sum \ln(x_i !)
\end{equation}
Первая производная:
\begin{equation}
l'(\lambda)=-n+\frac{\sum x_i}{\lambda}
\end{equation}
Оценка метода максимального правдоподобия имеет вид $\hat{\lambda}=\bar{X}_n$
Ожидаемае информация Фишера $I(\lambda)=n/\lambda$, наблюдаемая информация Фишера $I(\hat{\lambda})=n/\hat{\lambda}$.
Тест Вальда $W=(\hat{\lambda}-3)^2\cdot I(\hat{\lambda})$ \\
Тест множителей Лагранжа $LM=l'(3)\cdot I^{-1}(3)$ \\
Тест отношения правдоподобия $LR=2(l(\hat{\lambda})-l(3))$ 
}



\problem{В коробке 10 внешне не отличимых шоколадных конфет. Внутри $k$ штук из них есть орех. Мы выбирали конфеты наугад по одной и ели. Первый орех оказался в третьей по счету конфете. 

Оцените неизвестный параметр $k$ методом моментов, методом максимального правдоподобия.}
\solution{}



\problem{
Интервал времени в минутах между спам-письмами по электронной почте --- случайная величина с функцией плотности 
\begin{equation}
p(t)=
\begin{cases}
a^2\cdot t\cdot \exp(-at), \quad t\geq 0 \\
0, \quad t<0
\end{cases},
\end{equation}
где $a$ --- неизвестный параметр. По выборке из 20 наблюдений известно, что $\sum_{i=1}^{n}X_{i}=625$, $\sum_{i=1}^{n}\ln(X_i)=25$.


\begin{enumerate}
\item Оцените $a$ методом моментов
\item Оцените $a$ методом максимального правдоподобия
\item Постройте 95\% доверительный интервал для $a$ с помощью максимального правдоподобия
\end{enumerate}}


\solution{
\begin{equation}
E(\bar{X})=E(X_{i})=\int_{0}^{\infty} a^2\cdot t^2 \exp(-at)dt
\end{equation}

Для удобства заметим, что
\begin{equation}
\int_{0}^{\infty}t^n\exp(-at)dt=0+\int_{0}^{\infty}t^{n-1}\frac{n}{a}\exp(-at)dt
\end{equation}

Получаем мат. ожидание:
\begin{multline}
E(\bar{X})=\int_{0}^{\infty} a^2\cdot t\frac{2}{a}\cdot \exp(-at)dt=\\
=\int_{0}^{\infty} 2a\cdot t\cdot \exp(-at)dt=\int_{0}^{\infty} 2a\cdot \frac{1}{a} \exp(-at)dt=\\
=\int_{0}^{\infty} 2\cdot \exp(-at)dt=\frac{2}{a}
\end{multline}

Метод моментов
\begin{equation}
\bar{X}=\frac{2}{\hat{a}_{MM}}
\end{equation}

Получаем $\hat{a}_{MM}$:
\begin{equation}
\hat{a}_{MM}=\frac{2}{\bar{X}}=\frac{40}{625}=\frac{8}{125}=0.064
\end{equation}

Метод максимального правдоподобия
\begin{equation}
l=\sum_{i=1}^{n}\ln(p(x_{i}))=\sum_{i=1}^{n}(2\ln(a)+ln(x_{i})-ax_{i})=2n\ln(a)+\sum \ln(x_{i})-a\sum x_{i}
\end{equation}

\begin{equation}
l'(a)=\frac{2n}{a}-\sum x_{i}
\end{equation}

Получаем $\hat{a}_{ML}$:
\begin{equation}
\hat{a}_{ML}=\frac{2n}{\sum X_{i}}=\frac{2}{\bar{X}}=\hat{a}_{MM}
\end{equation}

Наблюдаемая информация Фишера
\begin{equation}
J_{n}(\hat{a})=-l''(\hat{a})=\frac{2n}{\hat{a}^{2}}=2n\cdot \left(\frac{\sum X_{i}}{2n}\right)^2=\frac{(\sum X_{i})^2}{2n}=\frac{625^2}{40}
\end{equation}

Доверительный интервал
\begin{equation}
\hat{a}\pm 1.96\cdot J_{n}^{-1/2}=0.064\pm 0.0198=[0.0442;0.0838]
\end{equation} }

\problem{В банке 10 независимых клиентских <<окошек>>. В момент открытия в банк вошло 10 человек. Каждый клиент встал к отдельному окошку. Других клиентов банке в этот день не было. Предположим, что время обслуживания одного клиента распределено экспоненциально с параметром $\lambda$. Оцените параметр $\lambda$  и оцените дисперсию оценки методом максимального правдоподобия в каждой из ситуаций 

\begin{enumerate}
\item Менеджер записал время обслуживания клиента в каждом окошке. Окошко \No 1 обслужило своего клиента за 10 минут, окошко \No 2 обслужило своего клиента за 20 минут; оставшуюся часть записей менеджер благополучно затерял. 
\item Менеджер наблюдал за окошками в течение получаса и записывал время обслуживания клиента. Окошко \No 1 обслужило своего клиента за 10 минут, окошко \No 2 обслужило своего клиента за 20 минут; остальные окошки еще обслуживали своих первых клиентов в тот момент, когда менеджер удалился. 
\item  Менеджер наблюдал за окошками в течение получаса. За эти полчаса два окошка успели обслужить своих клиентов. Остальные окошки еще обслуживали своих первых клиентов в тот момент, когда менеджер удалился. 
\item Менеджер наблюдал за окошками и решил записать время обслуживания первых двух клиентов. Через  10 минут от начала работы был обслужен первый клиент в одном из окошек, через 20 минут от начала работы был обслужен второй клиент. Сразу после того, как был обслужен второй клиент менеджер прекратил наблюдение. 
\item Одновременно с открытием банка началась деловая встреча директора банка с инспектором по охране труда. Время проведения таких встреч --- случайная величина, имеющая экспоненциальное распределение со средним временем 30 минут. За время проведения встречи было обслужено два клиента. 
\item Одновременно с открытием банка началась деловая встреча директора банка с инспектором по охране труда. Время проведения таких встреч - случайная величина, имеющая экспоненциальное распределение со средним временем 30 минут. За время проведения встречи было обслужено два клиента, один за 10 минут, второй --- за 20 минут. 
\item Изменим условие: в банке 11 окошек, при открытии банка вошло 11 клиентов. Других клиентов в этот день не было. Клиент попавший в окошко \No 11 смотрел за остальными. Раньше клиента из окошка  \No 11 освободилось двое клиентов: за 10 минут и за 20 минут. 
\end{enumerate} }
\solution{}


\problem{Предположим, что доход жителей страны распределен экспоненциально. 
Имеется выборка из 1000 наблюдений по жителям столицы. Если возможно, постройте 90\% доверительный интервал для $\lambda$ в следующих случаях
\begin{enumerate}
\item Столицу можно считать случайной выборкой из жителей страны
\item В столице селятся только люди с доходом больше 100 тыс. рублей. 
\item В столице селятся только люди с доходом больше $m$ тыс. рублей, где $m$ - неизвестная константа. При этом постройте также и 90\% доверительный интервал для $m$ 
\item В столице живут 10\% самых богатых жителей страны
\item В столице живут $p$\% самых богатых жителей страны, где $p$ - неизвестная константа. Постройте также 90\% доверительный интервал для $p$
\end{enumerate} }

\solution{}


