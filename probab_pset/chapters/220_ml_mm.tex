% !Mode:: "TeX:UTF-8"
% MM, ML

\problem{Падал первый снег и 30 школьников ловили снежинки. В среднем каждый поймал $3{,}3$ снежинки. Количества снежинок пойманные каждым --- независимые Пуассоновские случайные величины с общим параметром $\lambda$. 
\begin{enumerate}
\item Оцените $\lambda$ используя метод максимального правдоподобия
\item Оцените дисперсию полученной оценки
\item На уровне значимости 5\% проверьте гипотезу о том, что $\lambda=3$ используя тест множителей Лагранжа
\item ... используя тест отношения правдоподобия
\item ... используя тест Вальда 
\item Постройте 95\% доверительный интервал Вальда для $\lambda$
\end{enumerate}}
\solution{ Логарифмическая функция правдоподобия
\begin{equation}
l(\lambda)=-n\lambda+\sum x_i \ln\lambda-\sum \ln(x_i !)
\end{equation}
Первая производная:
\begin{equation}
l'(\lambda)=-n+\frac{\sum x_i}{\lambda}
\end{equation}
Оценка метода максимального правдоподобия имеет вид $\hat{\lambda}=\bar{X}_n$
Ожидаемае информация Фишера $I(\lambda)=n/\lambda$, наблюдаемая информация Фишера $I(\hat{\lambda})=n/\hat{\lambda}$.
Тест Вальда $W=(\hat{\lambda}-3)^2\cdot I(\hat{\lambda})$ \\
Тест множителей Лагранжа $LM=l'(3)\cdot I^{-1}(3)$ \\
Тест отношения правдоподобия $LR=2(l(\hat{\lambda})-l(3))$ 
}


