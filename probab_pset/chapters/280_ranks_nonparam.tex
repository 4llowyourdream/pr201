% !Mode:: "TeX:UTF-8"

\problem{Сёрен Кьеркегор случайным образом выбирает $k$ чисел от 1 до $(n+k)$ без повторений и складывает их. Чему равны математическое ожидание и дисперсия полученной суммы?}
\solution{$\E(R)=k\cdot \frac{1+(n+k)}{2}$, $\Var(R)=\frac{nk(n+k+1)}{12}$}



% untyp
\problem{
Из 10 опрошенных студентов часть предпочитала готовиться по
синему учебнику, а часть - по зеленому. В таблице представлены их
итоговые баллы.  \\
\begin{tabular}{|c|c|c|c|c|c|c|}
  \hline
  Синий & 76 & 45 & 57 & 65 &  &  \\
  \hline
  Зеленый & 49 & 59 & 66 & 81 & 38 & 88 \\
  \hline
\end{tabular} \\
а) С помощью теста Манна-Уитни (Mann-Whitney) проверьте гипотезу о
том, что выбор учебника не меняет закона распределения оценки. \par
\emph{Разрешается использование нормальной аппроксимации} \par
б) Возможно ли в этой задаче использовать (Wilcoxon Signed Rank Test)? } 
\solution{} 

% untyp
\problem{
Имеются результаты экзамена в двух группах. Группа 1: 45,
67, 87, 71, 34, 12, 54, 57; группа 2: 46, 66, 81, 72, 11, 47, 55,
51, 9, 99. С помощью теста Манна-Уитни на уровне значимости $5\%$ проверьте гипотезу о том, что результаты двух групп не отличаются. } 
\solution{} 

% untyp
\problem{
Имеются результаты нескольких студентов до и после
апелляции (в скобках указан результат до апелляции):  48(47),
54(52), 67(60), 56(60), 55(58), 55(60), 90(70), 71(81), 72(87),
69(60). Предполагая, что изменение оценки на апелляции симметрично распределено, на уровне значимости $5\%$ проверьте гипотезу о том, что
апелляция в среднем не сказывается на результатах. } 
\solution{} 

% untyp
\problem{
Имеются наблюдения за говорливостью 30 попугаев
(слов/день): 34, 56, 32, 45, 34, 45, 67, 1, 34, 12, 123, ... , 37
(всего 13 наблюдений меньше 40). Проверить гипотезу о том, что
медиана равна 40 (слов/день). } 
\solution{} 

% untyp
\problem{
Вашему вниманию представлены результаты прыжков в длину
Васи Сидорова на двух соревнованиях. На первых среди болельщиц
присутствовала Аня Иванова (его первая любовь): 1,83; 1,64; 2,27;
1,78; 1,89; 2,33; 1,61; 2,31. На вторых Аня среди болельщиц не
присутствовала: 1,26; 1,41; 2,05; 1,07; 1,59; 1,96; 1,29; 1,52;
1,18; 1,47. С помощью теста (Mann-Whitney) проверьте гипотезу о
том, что присутствие Ани Ивановой положительно влияет на
результаты Васи Сидорова. Уровень значимости $\alpha=0.05$. } 
\solution{} 

% untyp
\problem{
Некоторые результаты 2-х контрольных по теории вероятностей
выглядят следующим образом (указан результат за вторую контрольную
и в скобках результат за первую): 43(55), 113(108), 97(53),
68(42), 94(67), 90.5(97), 35(91), 126(127), 102(78), 89(83). Можно
ли считать (при $\alpha=0.05$), что вторую контрольную написали
лучше? Предположим, что разница в баллах распределена симметрично.} 
\solution{} 

% untyp
\problem{
Садовник осматривал по очереди розовые кусты вдоль ограды. Всего вдоль ограды растет 30 розовых кустов. Из них оказалось 20 здоровых и 10 больных. \par
Вот заметки садовника: $+++\ominus++\ominus\ominus\ominus++++\ominus\ominus ++\ominus+++++\ominus\ominus\ominus++++$ \par
(+ - здоровый куст, $\ominus$ - больной куст) \par
а) С помощью теста серий проверьте гипотезу о независимости испытаний \par
б) Какой естественный смысл имеет эта гипотеза? \par
Подсказка: можно использовать нормальное распределение }
\solution{}
