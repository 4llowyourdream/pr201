% !Mode:: "TeX:UTF-8"
\section{Пуассоновский поток и экспоненциальное распределение}

\subsection{Пуассоновский поток}


\problem{ Предположим, что кузнечики на большой поляне распределены
по пуассоновскому закону с  $\lambda=3$  на квадратный метр. Какой
следует взять сторону квадрата, чтобы вероятность найти в нем хотя
бы одного кузнечика была равна  $0,8$? }
\solution{ }


\problem{Саша красит стены в своей комнате, а Алёша "--- в своей. У каждой комнаты четыре стены. Предположим, что время покраски одной стены и для Саши, и для Алёши "--- экспоненциальная случайная величина с параметром $\lambda$. Какова вероятность того, что Саша успеет покрасить 3 стены раньше, чем Алёша "--- две?}

\solution{Каждая следующая стена равновероятно покрашена Сашей и Алёшей. Значит, нам нужны $\PP(SSS)+\PP(SSAS)+\PP(SASS)+\PP(ASSS)=\frac{5}{16}$. По другому: для простоты положим $\lambda=1$. Пусть $T$ "--- время, когда Саша закончит 3 стены. Функция плотности гамма-распределения (сумма трёх экспоненциальных) $f(t)=0{,}5t^{2}e^{-t}$. Нам нужна вероятность того, что к тому времени Алёша успеет меньше двух стен: $\int_{0}^{\infty} \PP(N_{t}<2 \mid T=t)\,dt =\ldots=\frac{5}{16}$.}

\problem{Машины подъезжают к светофору пуассоновским потоком с интенсивностью $\lambda $. Для простоты будем считать, что первая машина подъезжает в $ t=0 $. Светофор горит зелёным только в целые моменты времени, и этого достаточно чтобы пропустить одну машину, т.\,е. светофор горит красным при $ t\in(0;1) $, $ t\in(1;2) $, $ t\in(2;3) $ и т.\,д. Какой будет средняя длина очереди через продолжительное время? Чему будет равна вероятность, что очередь пуста?}
\solution{Производящая функция удовлетворяет соотношению:
\[ g(t)=\exp(\lambda (t-1))\frac{g(t)+tg(0)-g(0)}{t} \]
\[ g(t)=g(0)\frac{(t-1)\exp(\lambda (t-1))}{t-\exp(\lambda (t-1))} \]
Из условия $ g(1)\to 1 $ находим $ g(0)=1-\lambda $ и, помучившись, $\E(X_{\infty})=g'(1)=\frac{\lambda(2-\lambda)}{2\cdot(1-\lambda)} $.}

\cat{Poisson} \cat{gen_fun}

\problem{В офисе два телефона: зелёный и красный. Входящие звонки на красный "--- Пуассоновский поток событий с интенсивностью $\lambda_{1}=4$ звонка в час, входящие на зелёный "--- с интенсивностью $\lambda_2=5$ звонков в час. Секретарша Василиса Премудрая одна в офисе. Время разговора "--- случайная величина, имеющая экспоненциальное распределение со средним временем $5$ минут. Если Василиса занята разговором, то на второй телефон она не отвечает. Сколько звонков в час в среднем пропустит Василиса, потому что будет занята разговором по другому телефону? Являются ли пропущенные звонки Пуассоновским потоком? }
\solution{}

\problem{В офисе два телефона: зелёный и красный. Входящие звонки на красный "--- Пуассоновский поток событий с интенсивностью $\lambda_{1}=4$ звонка в час, входящие на зелёный "--- с интенсивностью $\lambda_2=5$ звонков в час. Секретарша Василиса Премудрая одна в офисе. Перед началом рабочего дня она подбрасывает монетку и отключает один из телефонов: зелёный "--- если выпала решка, красный "--- если выпал орёл. Обозначим за $Y$ время от начала дня до первого звонка. Найдите функцию плотности $Y$. }
\solution{}

\problem{Случайная величина $X$ имеет экспоненциальное распределение с параметром $\lambda$. Найдите медиану $X$. }
\solution{}

\subsection{Пуассоновское приближение}
% при замене на Poisson(\lambda=np) ошибка не превосходит
% min(1,1/\lambda)\sum p_{i}^{2}

\problem{ Используя пуассоновское предупреждение найдите вероятности
\begin{enumerate}
\item В гирлянде 25 лампочек. Вероятность брака для отдельной
лампочки равна 0,01. Какова вероятность того, что гирлянда
полностью исправна? 
\item По некоему предмету незачет получило всего 2\% студентов.
Какова вероятность того, что в группе из 50 студентов будет ровно
1 человек с незачетом? 
\item Вася испек 40 булочек. В каждую из них он кладет изюминку с
$p=0,02$. Какова вероятность того, что всего окажется 3 булочки с
изюмом? 
\end{enumerate} }
\solution{ }

\problem{ Вася каждый день подбрасывает монетку 10 раз. Монетка с
вероятностью 0,005 встает на ребро. Используя пуассоновскую
аппроксимацию, оцените вероятность того, что за 100 дней монетка
встанет на ребро ровно 3 раза. }
\solution{ }

\problem{ Страховая компания <<Ой>> заключает договор страхования от
<<невыезда>> (не выдачи визы) с туристами, покупающими туры в
Европу. Из предыдущей практики известно, что в среднем отказывают
в визе одному из 130 человек. Найдите вероятность того, что из 200
застраховавшихся в <<Ой>> туристов, четверым потребуется страховое
возмещение. }
\solution{ }

\problem{
Вася, владелец крупного Интернет-портала, вывесил на главной
странице рекламный баннер. Ежедневно его страницу посещают 1000
человек. Вероятность того, что посетитель портала кликнет по
баннеру равна 0,003. С помощью пуассоноского приближения оцените
вероятность того, что за один день не будет ни одного клика по
баннеру.}
\solution{ }



