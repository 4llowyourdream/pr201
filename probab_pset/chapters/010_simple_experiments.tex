% !Mode:: "TeX:UTF-8"
\section{Простые эксперименты}
% simple_experiments

\subsection{Дискретные простые эксперименты}
%Эксперимент состоит из одного "этапа"

%Правило сложения вероятностей.
%Если события несовместны, то
%P(хотя бы одно)= сумма
%Р(все сразу)=0

%1.1. дискретные случайные величины (P, E)

\problem{
Подбрасываются два кубика. Какова вероятность выпадения хотя бы
одной шестёрки? Какова вероятность того, что шестёрка не выпадет
ни разу? }
\solution{
$\PP(N\geq 1)=1-\frac{5}{6}^2$; $\PP(N=0)=\frac{5}{6}^2$. }
\cat{die}

\problem{
$\Omega =\{a, b, c\}$, $\PP\ofbr{a, b}=0{,}8$, $\PP\ofbr{b,c}=0{,}7$. Найдите
$\PP\ofbr{a}$, $\PP\ofbr{b}$, $\PP\ofbr{c}$.}
\solution{$\PP\ofbr{b}=0{,}2$, $\PP\ofbr a=0{,}6$, $\PP\ofbr{c}=0{,}5$.  }

\problem{
$A$  и  $B$  несовместны,  $\PP(A)=0{,}3$, $\PP(B)=0{,}4$. Найдите
$\PP(A^{c} \cap B^{c} )$.}
\solution{ $\PP(A^{c} \cap B^{c} )=1-0{,}3-0{,}4$.}

\problem{
$\PP(A)=0{,}3$,  $\PP(B)=0{,}8$. В каких пределах
может лежать  $\PP(A\cap B)$? }
\solution{ $\PP(A\cap B)\in[0{,}1;0{,}3]$.}


\problem{
Кубик подбрасывается два раза. Какова вероятность того, что результат
второго броска будет строго больше, чем результат первого?
Какова вероятность того, что в сумме будет 6? Что в сумме будет 9? Что максимум равен
5? Что минимум равен 3? Что разница будет равна 1 или 0? }
\solution{ $\PP\ofbr{N_{2}>N_{1}}=\frac{15}{36}$; \par
$\PP\ofbr{N_{1}+N_{2}=6}=\frac{5}{36}$; \par
$\PP\ofbr{N_{1}+N_{2}=9}=\frac{4}{36}$; \par
$\PP\ofbr{\max\{N_{1},N_{2}\}=5}=\frac{9}{36}$; \par
$\PP\ofbr{\min\{N_{1},N_{2}\}=3}=\frac{7}{36}$; \par
$\PP\ofbr{|N_{1}-N_{2}|\leq 1}=\frac{16}{36}$. }
\cat{die}

\problem{ \label{shokoladnie konfeti}
На подносе лежит 20 шоколадных конфет, одинаковых с виду. В
четырёх из них есть орех внутри. Маша съела 5 конфет. Какова
вероятность того, что в наугад выбранной оставшейся конфете будет
орех? }
\solution{$\frac{4}{20}$. }

%%%%% пошло применение ожидания

\problem{ \label{ojidanie ot bernulli}
Пусть  $X$  принимает два значения, причём $\PP\ofbr{X=1}=p$ и
$\PP\ofbr{X=0}=1-p$. Найдите $\E(X)$.}
\solution{ $\E(X)=p$.}



\problem{
Пусть существует всего два момента времени, $t = 0$ и $t =
1$. Cтоимости облигаций и акций в момент времени $t$ обозначим соответственно
$B_{t}$ (bond) и $S_{t}$ (share). Известно, что $B_{0}=1$,
$B_{1}=1{,}1$, $S_{0}=5$, $S_{1}=
\begin{cases}
10, & p_{\text{high}}=0{,}7; \\
2, & p_{\text{low}}=0{,}3.
\end{cases}$ \\
Индивид может покупать акции и облигации по указанным ценам без
ограничений. Например, можно купить минус одну акцию: это
означает, что в момент времени $t=0$ индивид получает 5 рублей, а в момент $t = 1$ в
зависимости от состояния природы должен заплатить 10 рублей или 2
рубля.
\begin{enumerate}
\item Чему равна безрисковая процентная ставка за период?
\item Найдите дисконтированные математические ожидания будущих цен
акций и облигаций. Совпадают ли они с ценами нулевого периода?
\item Найдите такие вероятности $q_{\text{high}}$ и $q_{\text{low}}$, чтобы
дисконтированное математическое ожидание будущих цен
совпало с ценами нулевого периода.
\item Индивиду предлагают купить некий актив, который приносит 8
рублей в состоянии мира $\omega_{\text{high}}$ и 11 рублей в состоянии
мира $\omega_{\text{low}}$. Посчитайте ожидание стоимости этого актива с
помощью вероятностей $p$ и с помощью вероятностей $q$. Придумайте
такую комбинацию акций и облигаций, которая в будущем приносит 8 и
11 рублей соответственно, и найдите её стоимость.\end{enumerate} }
\solution{ }



\problem{
Игральный кубик подбрасывается два раза. Пусть  $X_{1}$ и $X_{2} $
"--- результаты подбрасывания. Найдите вероятности $\PP(\min
\left\{X_{1},X_{2} \right\}=4)$  и $\PP(\min
\left\{X_{1},X_{2} \right\}=2)$. }
\solution{ }


\problem{  \label{simple third}
На десяти карточках написаны числа от 1 до 9. Число 8 фигурирует
два раза, остальные числа "--- по одному разу. Карточки извлекают в
случайном порядке. Какова вероятность того, что девятка появится позже обеих
восьмёрок? }
\solution{ Устно: $\frac{1}{3}$.}

\problem{
17 заключённых, 5 камер. Заключённых распределяют по камерам по очереди, равновероятно в каждую. Какова вероятность, что Петя и Вася сидят в одной камере? }
\solution{ $0{,}2$. }
% решабельна ли более сложная задача, где конфигурации рассадок равновероятны?


\problem{
Кость подбрасывается два раза. Пусть  $X$  и  $Y$  "---
результаты
подбрасываний. Найдите  $\E\parb{|X\hm-Y|}$. }
\solution{ }

\problem{
\foreignlanguage{british}{We throw 3 dices one by one. What is the probability that we obtain 3 points in strictly increasing order?} }
\solution{ $\frac{C_{6}^{3}}{6^{3}}$. }

\problem{ \label{tri chisla}
Из 10 цифр (от 0 до 9) выбираются 3 наугад (возможны повторения).
Обозначим числа (в порядке появления): $X_{1}$, $X_{2}$, $X_{3}$.
Какова вероятность того, что $X_{1}>X_{2}>X_{3}$? }
\solution{ $\frac{C_{10}^{3}}{10^{3}}$, т.\,к. каждый способ выбрать три разных числа соответствует благоприятной
комбинации. }



\problem{
Кубик подбрасывается 3 раза. Какова вероятность того, что сумма первых двух подбрасываний будет больше третьего? }
\solution{ }

\problem{ \zdt{<<Масть>> при игре в бридж }

Часто приходится слышать, что некто при игре в бридж получил на
руки 13 пик. Какова вероятность (при условии, что карты хорошо
перетасованы) получить 13 карт одной масти?
\begin{note}
Каждый из четырёх игроков в бридж получает 13 карт из колоды в 52 карты.
\end{note}
\begin{ist}
Mosteller.
\end{ist}
}
\solution{ }


\problem{ \label{maksimum iz kartochek}
На карточках написаны числа от
1 до 100. В левую руку Маша берёт одну карточку, в правую "--- $k$~карточек.
Какова вероятность того, что число на карточке в левой руке
окажется больше числа на любой карточке из
правой руки? }
\solution{$\frac{1}{k+1}$, т.\,к. одна из $k+1$ карточек должна быть наибольшей.  }



\problem{ \label{sleeping beauty} \zdt{Спящая красавица}

Спящая красавица согласилась принять участие в научном
эксперименте. В воскресенье её специально уколют веретеном. Как
только она заснёт, будет подброшена правильная монетка. Если
монетка выпадет орлом, то спящую красавицу разбудят в понедельник
и спросят о том, как выпала монетка. Если монетка выпадет решкой,
то спящую царевну разбудят в понедельник, спросят о монетке, снова
уколют веретеном, разбудят во вторник и снова спросят о монетке.
Укол веретена вызывает легкую амнезию, и красавица не сможет
определить, просыпается ли она в первый раз или во второй.
Красавица только что проснулась.
\begin{enumerate}
\item Какова вероятность того, что сегодня понедельник?
\item Как следует отвечать красавице, если за каждый верный ответ ей
дарят молодильное яблоко?
\item Как следует отвечать красавице, если за неверный ответ её тут
же превращают в тыкву?
\end{enumerate}
\begin{note}
Осторожно! Некорректные вопросы!
\end{note}
 }
\solution{<<Сегодня понедельник>> "--- это \textbf{не} событие. Вероятность не
определена. Это функция от времени.

Вероятность того, что монетка выпала орлом, равна $0{,}5$. Поэтому ей
всё равно, как отвечать, если наказанием является превращение в
тыкву, и нужно отвечать: <<Решка!>> "--- если наградой является
молодильное яблоко. Предполагается, что красавица максимизирует
ожидаемое количество молодильных яблок.  }



\problem{
Пусть события $A_{0}$, $A_{1}$ и $A_{2}$ несовместны и вместе
покрывают всё $\Omega$. Обозначим $p_{0}=\P(A_{1}\cup A_{2})$, $p_{1}=\P(A_{0}\cup A_{2})$,
$p_{2}=\P(A_{0}\cup A_{1})$. Перечислите все условия, которым удовлетворяют $p_{0}$, $p_{1}$,
$p_{2}$. }
\solution{ }

\problem{
Найдите вероятность того, что произойдёт ровно одно событие из $A$ и $B$, если $\P(A)=0{,}3$, $\P(B)=0{,}2$, $\P(A\cap B)=0{,}1$.    }
\solution{ }

\problem{
Вася наугад выбирает два разных натуральных числа от 1 до 4.
\begin{enumerate}
\item Какова вероятность того, что будет выбрано число 3?
\item Какова вероятность того, что сумма выбранных чисел будет чётная?
\item Каково математическое ожидание суммы выбранных чисел?
\end{enumerate}
 }
\solution{ $\PP\ofbr{3}=\frac{1}{2}$, $\PP\ofbr{\Sigma\text{ чёт.}}=\frac{1}{3}$, $\E(\Sigma)=5$. }


\problem{
Известно, что когда соревнуются А и Б, то А побеждает с вероятностью $x$ (Б, соотвественно, с вероятностью $(1-x)$). Когда соревнуются А и В, то А побеждает с вероятностью $y$ (В, соответственно, с вероятностью $(1-y)$).
\begin{enumerate}
\item  Придумайте модель, которая бы позволяла узнать вероятность победы Б над В.
\item  Покажите, что можно придумать другую модель и получить другую вероятность.
\end{enumerate} }
\solution{
Если предположить, что у каждого игрока есть своя сила (константа), а вероятности победить в схватке для двух игроков относятся так же, как их силы, то $x=\frac{a}{a+b}$, $y=\frac{a}{a+c}$. Легко находим, что $\frac{b}{b+c}=\frac{y-xy}{x+y-2xy}$. }



\problem{  В клубе 25 человек.
\begin{enumerate}
\item  Сколькими способами можно выбрать комитет
из четырёх человек?
\item  Сколькими способами можно выбрать руководство, состоящее из
директора, зама и кассира?
\end{enumerate}
 }
\solution{ Комитет можно выбрать $C_{25}^{4}$ способами, руководство "--- $C_{25}^{3}3!$.}

\problem{ Сколькими способами можно расставить 5 человек в очередь?}
\solution{$5!$. }

\problem{ Сколькими способами можно покрасить 12 комнат, если требуется 4
покрасить жёлтым цветом, 5 "--- голубым и 3 "--- зелёным?}
\solution{ $C_{12}^{4}C_{8}^{5}$. }

\problem{ Шесть студентов (три юноши и три девушки), стоят в очереди за
пирожками в случайном порядке. Какова вероятность того, что юноши
и девушки чередуются?}
\solution{$2\cdot\frac{3!3!}{6!}$. }


\problem{
Где-то в начале 17 века Галилея попросили объяснить следующее:
количество троек натуральных чисел, дающих в сумме 9, такое же,
как количество троек, дающих в сумме 10; но при трёхкратном
подбрасывании кубика 9 в сумме выпадает реже, чем 10. Дайте корректное объяснение. }
\solution{ }


\problem{В классе 30 человек, и все разного роста. Учитель физкультуры хочет отобрать и поставить в порядке возрастания роста 5 человек. Сколькими способами это можно сделать?}
\solution{$C_{30}^{5}$. Расположить по росту можно только в одном порядке. }

\problem{Случайная величина $X$ равновероятно принимает одно из пяти значений: 1, 2, 3, 8 и 9. 
\begin{enumerate}
\item Найдите математическое ожидание и медиану $X$
\item Найдите значение $u$ при котором функция $f(u)=\E(|X-u|)$ достигает минимума
\item Найдите значение $u$ при котором функция $g(u)=\E((X-u)^2)$ достигает минимума
\item Сделайте выводы
\end{enumerate}
}
\solution{$\min f(u)=\med(X)$, $\min g(u)=\E(X)$}

\problem{Контрольную писали 32 человека, каждый из четырех вариант писали 8 человек. Какова вероятность того, что Вася и Петя писали один и тот же вариант?}
\solution{Семь человек пишут тот же вариант, что Вася, 24 человека --- другой. Значит вероятность равна $7/31$. }



\subsection{Непрерывные простые эксперименты}
%1.2. непрерывные случайные величины (P, E для равномерной)
\problem{
Поезда метро идут регулярно с интервалом 3 минуты. Пассажир
приходит на платформу в случайный момент времени. Пусть $X$
"--- время ожидания поезда в минутах.

Найдите $\P(X<1)$, $\E(X)$. }
\solution{$\frac{1}{3}$, $1{,}5$. }


\problem{
Светофор 40 секунд горит зелёным светом, 3 секунды "--- жёлтым, 30
секунд "--- красным, затем цикл повторяется. Петя подъезжает к светофору. На жёлтый свет Петя предпочитает остановиться.
\begin{enumerate}
\item  Какова вероятность, что Петя сможет проехать сразу?
\item  Какова средняя задержка Пети на светофоре?
\item  Вася, стоящий рядом со светофором, смотрит на него в течение 3
секунд. Какова вероятность того, что он увидит смену цвета?
\end{enumerate}
 }
\solution{ }

\problem{
Случайные величины $X$, $Y$, и $Z$ независимы и равномерны на $[0;1]$. Какова вероятность того, что $X+Y>Z$? }
\solution{ }


\problem{
\foreignlanguage{british}{At a bus stop you can take bus \#1 and bus \#2. Bus \#1 passes 10 minutes after bus \#2 has passed whereas bus \#2 passes 20 mins after bus \#1 has passed. What is the average waiting time to get on a bus at that bus stop?}

\begin{ist}
Wilmott forum, \texttt{catid=26\&threadid=55617}.
\end{ist}
 }
\solution{ $\frac{25}{3}$. }


\problem{
На множестве $A:=\{x\geq 0,\ 0\leq y\leq e^{-x}\}$ случайно (равномерно) выбирается точка. Пусть $X$ "--- абсцисса этой точки. Найдите следующие вероятности: $\PP(X>1)$, $\PP(X\in (1;5))$, $\PP(X \in [1;5])$. }
\solution{ $\int_{1}^{\infty}e^{-x}\,dx$; $\PP(X\in (1;5))=\PP(X \in [1;5])=\int_{1}^{5}e^{-x}\,dx$. }


\problem{На плоскости нарисован треугольник с вершинами $(0,0)$, $(2,0)$ и $(1,1)$. Случайным образом, равномерно, выбирается точка внутри этого треугольника. Случайная величина $X$ --- абсцисса полученной точки. Найдите
\begin{enumerate}
\item $\P(X>1)$, $\P(X<0.5)$, $\P(X=0.2)$
\item $\E(X)$
\end{enumerate}
}
\solution{$\E(X)=1$}
\todo[inline]{Тут спросить про минимизацию $f(u)=\E(|X-u|)$? }

\problem{Предположим, что завтрашний курс тугриков к луидорам --- случайная величина $X$, равномерная на отрезке $[0;1]$. Финансовый аналитик Вовочка строит прогноз $a$. За неправильный прогноз Вовочка заплатит штраф. Какой прогноз следует сделать Вовочке чтобы минимизировать ожидаемое значение штрафа, если
\begin{enumerate}
\item штраф считается по формуле $|X-a|$
\item штраф считается по формуле $0.75(X-a)$ при $X>a$ и $0.25(a-X)$ при $X<a$
\end{enumerate} 
}
\solution{Квантили распределения}

\subsection{Смешанные простые эксперименты, или содержание эксперимента неясно}
%1.3. смешанные случайные величины (P, E для смеси с равномерной)
\problem{
Как связаны между собой $\PP(A)$ и $\E(\inds{A})$? }
\solution{Равны.}

