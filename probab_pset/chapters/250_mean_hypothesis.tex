% !Mode:: "TeX:UTF-8"
\section{Гипотезы о среднем и сравнении среднего при большом $n$}


\problem{Предположим, что исходные наблюдения $X_{i}$ нормальны $N(\mu,\sigma^{2})$ и независимы. Константы $\mu$ и $\sigma$ неизвестны. Вовочка строит доверительный интервал для $\mu$ по первой половине доступных наблюдений. Петя --- по всем наблюдениям. Может ли получится у Вовочки интервал меньшей ширины, чем у Пети?}
\solution{Да. Например, если первая половина наблюдений попала рядом с $\mu$, а вторая --- далеко. Ну не повезло Пете.}

% untyp
\problem{Вася и Петя выясняют, кто лучше умеет знакомиться с девушками. Вася попытался познакомиться с 100 девушек, из них 54 девушки дали ему номер своего телефона. Петя  попытался познакомиться с 900 девушек, из них 495 дали ему номер своего телефона. 

Вася и Петя изучили курс матстата и начали спорить. Петя утверждает, что ему в среднем чаще девушки дают свой номер телефона и аргументирует это так: давай проверим гипотезу, что в среднем ровно половина девушек даёт номер своего телефона, против альтернативной гипотезы, что больше половины. По твоим данным эта гипотеза не отвергается, а по моим --- отвергается.

Вася утверждает, что он лучше убеждает девушек. Аргументирует это так: давай проверим гипотезу, что в среднем 60\% девушек даёт номер своего телефона, против альтернативной гипотезы, что меньше 60\:. По твоим данным эта гипотеза отвергается, а по моим --- не отвергается. 

Кто из них прав?}
\solution{Оба они делают одну ошибку: если $H_0$ не отвергается, это не значит, что она --- верна. Корректнее было бы провести тест на сравнение средних. }
Идея: Кирилл Фурманов


