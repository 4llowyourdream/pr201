% !Mode:: "TeX:UTF-8"
\section{Сложные эксперименты}

% test comment
\subsection{<<Продвинутая>> комбинаторика }
%(использование "голого" биномиального коэффициента без каких бы то ни было заморочек можно делать в главе 1. Здесь всё, где надо применять комбинаторику с умом.
%Эту главу будут читать не все.
% возможно, что комбинаторная формула проста, но нужно догадаться до неё

\problem{
\ENG{Jenny and Alex flip $n$ fair coins each.}
\begin{enumerate}
\item \ENG{What is the probability that they get the same number of tails?}
\item  Пусть $a_{n}=\sqrt{n}\cdot p_{n}$, где $p_{n}$ "--- найденная вероятность. Найдите $\lim a_{n}$.
\end{enumerate}
 }
\solution{ $C_{2n}^{n}/2^{2n}$. Из $2n$ подбрасываний выберем $n$. Выбранные в зоне Jenny соответствуют невыбранным в зоне Alex. б) Через формулу Стирлинга: $\frac{1}{\sqrt{\pi}}$. }

\problem{
\ENG{2 couples and a single person are to be randomly placed in 5 seats in a row. What is the probability that no person that belongs to one of the couples sits next to his/her pair?} }
\solution{ }


\problem{
Встретились 6 друзей. Каждый дарит подарок одному из других 5 человек. Какова вероятность того, что найдется хотя бы одна пара человек, которая вручит подарки друг другу? }
\solution{ }


\problem{\zdt{Хулиган и случайная система}

На плоскости нарисовано $n$ прямых. Среди них нет параллельных, и никакие три не пересекаются в одной точке. Хулиган берёт первую прямую. Эта прямая делит плоскость на две полуплоскости. Хулиган случайным образом закрашивает одну из этих двух полуплоскостей. Затем хулиган поступает аналогично с каждой прямой. Какова вероятность того, что на плоскости останется хотя бы один незакрашенный кусочек?


\begin{ist}
Алексей Суздальцев.
\end{ist}
}

\solution{$\frac{n^2+n+2}{2^{n+1}}$
В оригинале у Алексея: Пусть $A$ "--- матрица $n\times2$, $\vec x$ и $\vec b$ "--- векторы подходящих размерностей. Известно, что как в матрице $A$, так и в расширенной матрице $\begin{pmatrix}
A & b  \end{pmatrix}$ все миноры максимального порядка ненулевые. Вася берёт систему $Ax=b$ и в каждой строке независимо от других ставит вместо знака равенства равновероятно знак <<больше>> или знак <<меньше>>. Какова вероятность того, что полученная система неравенств имеет решение?

Решение: остаться может либо один кусочек, либо ни одного. Всего есть $2^{n}$ способов выбирать полуплоскости. Некоторые из этих способов приводят к тому, что закрашена вся плоскость. Кусочков всего $\frac{n^{2}+n+2}{2}$. Каждый кусочек однозначно определяет способ выбора полуплоскостей. }

\problem{
На столе есть следующие предметы:
\begin{itemize}
\item 4 отличающихся друг от друга чашки;
\item 4 одинаковых гранёных стакана;
\item 10 одинаковых кусков сахара;
\item 7 соломинок разных цветов.
\end{itemize}
Сколькими способами можно разложить:
\begin{enumerate}
\item Сахар по чашкам;
\item Сахар по стаканам;
\item Соломинки по чашкам;
\item Соломинки по стаканам?
\end{enumerate}
Как изменятся ответы, если требуется, чтобы пустых ёмкостей не
оставалось? }
\solution{ }

\problem{
Сколькими способами можно разложить $k$ кусков сахара по
$n$ различающимся чашкам?

Подсказка: ответ "--- всего лишь биномиальный коэффициент. }
\solution{ $C_{k+n-1}^{n-1}$. }

% test comment 2

\problem{ \zdt{Генуэзская лотерея (задача Леонарда Эйлера)}

Из 90 чисел выбираются 5 наугад. Назовем серией последовательность
из нескольких чисел, идущих подряд. Например, если выпали числа
23, 24, 77, 78 и 79 (неважно в каком порядке), то есть две серии
(23-24, 77-78-79). Определите вероятность того, что будет ровно $k$ серий.

\begin{note}
Сама лотерея возникла в 17 веке.
\end{note}
 }
\solution{ }

\problem{ \label{sudba-don-juan-1} \zdt{Судьба Дон-Жуана} (см. тж. с.~\pageref{sudba-don-juan-2})

У Васи $n$  знакомых девушек (их всех зовут по-разному). Он пишет
им $n$  писем, но по рассеянности раскладывает их в конверты
наугад. Случайная величина $X$ обозначает количество девушек, получивших письма,
написанные лично для них.
\begin{enumerate}
\item Какова вероятность того, что Маша получит письмо, адресованное ей?
\item Какова вероятность того, что Маша и Лена получат письма, адресованные им?
\item Какова вероятность того, что хотя бы одна девушка получит
письмо, адресованное именно ей? Каков предел этой вероятности при
$n\rightarrow +\infty$?
\item Какова вероятность того, что произойдет ровно $k$ совпадений?
\end{enumerate}
}
\solution{ $1-\frac{1}{2!}+\frac{1}{3!}-\ldots$; $\frac{1}{e}$.  }


\problem{  \zdt{Покер}

Выбирается 5 карт из колоды (52 карты без джокеров, достоинством
от 2 до туза, всего 13 достоинств). Рассчитайте вероятности
комбинаций:\label{combo}
\begin{enumerate}
\item Pair (пара) "--- две карты одного достоинства, три остальные "--- разного
\item Two pairs (две пары) "--- две карты одного достоинства и две другого;
\item Three of Kind (тройка) "--- три карты одного достоинства, две остальные "--- разного;
\item Straight (стрит) "--- пять последовательных карт, есть карты разных мастей
\item Flush (масть) "--- все карты одной масти, но не последовательных достоинств
\item Full House (фул-хаус) "--- три карты одного достоинства и две другого;
\item Four of Kind  (каре) "--- четыре карты одного достоинства;
\item Straight Flush (стрит-флэш) "--- пять последовательных карт одной масти;
\item Royal Flush (роял-флэш) "--- старшие пять последовательных карт одной масти?
\end{enumerate}

}
\solution{ }

\problem{ \label{vilkodir} \zdt{<<Вилкодыр>>}

Есть $n$ дырочек, расположенных в линию, на расстоянии в 1~см друг от друга. У каждой вилки два штырька на расстоянии в 1 см.
\begin{enumerate}
\item Сколькими способами можно воткнуть $k$ одинаковых вилок?
\item Как изменится ответ, если дырочки расположены по окружности?
\end{enumerate}
}
\solution{Представим себе мифический объект <<вилкодыр>>. Он может
превращаться либо в вилку, либо в дырку. Вилкодыров у нас $n-k$.
Из них нужно выбрать $k$ вилок. Ответ: $C_{n-k}^{k}$ для линейного и
$C_{n-k}^{k}\frac{n}{n-k}$ для кругового расположения дырочек. }

\problem{ \label{lampochki v riad}
В ряду $n$ лампочек. Из них надо
зажечь 8, причём так, чтобы было три серии (по
2, 3 и 3 горящих лампочки). Сколькими способами это можно сделать? }
\solution{$3\cdot C_{n+1-d}^{k}$, где $k$ "--- число серий, $d$ "--- суммарная
длина серий. Домножение на 3 взялось из трёх вариантов (2-3-3,
3-2-3, 3-3-2). Мифический объект "--- серия из горящих лампочек
плюс негорящая справа.}

\problem{
Вася играет в преферанс. Он взял прикуп, снёс две карты и
выбрал козыря. У Васи на руках четыре козыря. Какова вероятность,
что
оставшиеся четыре козыря разделились как 4:0, 3:1, 2:2?

\begin{note}
Для тех, кто не знает, как играть в преферанс: 32 карты, из
которых 8 "--- будущие козыри, раздаются по 10 между 3 игроками, ещё
две кладутся в прикуп.
\end{note}
 }
\solution{ }

\problem{
Перетасована колода в 52 карты.
\begin{enumerate}
\item Какова вероятность того, что какие-нибудь туз и король будут лежать рядом?
\item Какова вероятность того, что какой-нибудь туз будет лежать за
каким-нибудь королем?
\end{enumerate}
 }
\solution{ }

\problem{
Чему равна сумма $C_{n}^{0}-C_{n}^{1}+C_{n}^{2}-...$?

Её применение к matching problem. }
\solution{ }

\problem{
В линию выложено $n$ предметов друг за другом. Сколькими способами
можно выбрать $k$ предметов из линии так, чтобы не были выбраны
соседние предметы? }
\solution{ $C_{n-k+1}^{k}$.

Решение 1. Отдельно рассмотрим два случая: самый правый предмет
выбран и самый правый предмет не выбран. В каждом случае
склеиваем предмет и примыкающий к нему справа предмет <<разделитель>>.

Решение 2. Удалим $k-1$ предмет из линии. Из оставшихся предметов
выберем $k$. Вернём удалённые как <<разделители>>. }

\problem{
\ENGs Given eight distinguishable rings, let $n$ be the number of
possible five-ring arrangements on the four fingers (not the
thumb) of one hand. The order of rings on each finger is
significant, but it is not required that each finger have a ring. Find the leftmost three nonzero digits of $n$. \RUSs
\begin{ist}
AIME 2000, 5.
\end{ist}
 }
\solution{Всего расположений $\binom{8}{5}\binom{8}{3}5! = 376\,320$, и три левые цифры "--- это \boxed{376}. }

\problem{ \ENGs
5 numbers are randomly picked from 90. In your bet cards, you get to choose 5 numbers.  How many cards have you got to fill in,
to guarantee that at least one of them has 4 right numbers? \RUSs
\begin{ist}
Wilmott forum.
\end{ist}
 }
\solution{ \ENGs
The answer to the original problem (\numprint{43948843} bet cards) was quoted already several times assuming that positioning of right numbers is irrelevant:
\begin{itemize}
\item there is eactly one bet card choice with 5 right numbers and
\item there are $5 \times (90-5) = 425$ bet card choices with exactly 4 right numbers.
\end{itemize}

Since the total number of different bet card choices is $\binom{90}{5}$, we have to fill out $\binom{90}{5} - 425 - 1 +1 = \numprint{43948843}$ bet cards to have at least 4 right numbers with probability 1.
\RUSs
}


\problem{
В контрольной 20 вопросов. Все ответы разные. Вася успел переписать у друга все верные ответы, но не знает, в каком порядке они идут. Отлично ставят ответившим верно на не менее чем 15 вопросов. Какова вероятность того, что Вася получит отлично? }
\solution{
$p=\frac{1}{20!}\cdot \bigl(C^{0}_{20}+C^{1}_{20}\cdot 0+C^{2}_{20}+C^{3}_{20}\cdot 2+C^{4}_{20}\cdot(C^{2}_{4}+3)+C^{5}_{20}\cdot(2C^{2}_{5}+4)\bigr)$. }

\problem{
There are $k$ books of mine among $n$ books. We put them in a shelf randomly. Which is the possibility that there are $p$ books of my who are placed continuously? (At least? Exactly?)
\begin{ist}
AoPS, \texttt{f=498\&t=192257}.
\end{ist}
 }
\solution{ Ugly sum? }


\problem{
На каждой карточке вы можете отметить любые 5 чисел из 100. Сколько карточек нужно купить, чтобы гарантированно угадать 3 числа из выпадающих в лотереи 7 чисел?

\begin{note}
Могут быть громоздкие вычисления.
\end{note}
 }
\solution{ }

\problem{
Сколькими способами можно поставить в очередь $a$ мужчин и $b$ женщин так, чтобы нигде двое мужчин не стояли рядом? }
\solution{ }

\problem{
Известно, что функция $f(n,k)$ удовлетворяет условиям:
\begin{itemize}
\item $f(n,k) = f(n-1,k) + f(n, k-1)$;
\item $f(n,0)=f(n,n)=1$.
\end{itemize}

Что это за функция такая?}
\solution{$C_{n+m}^{n}$ }

\problem{ \zdt{Усталые влюблённые}

В вагоне метро на длинную скамейку в $n$ мест садятся случайным образом $k>\frac{n}{2}$ пассажиров. Какова вероятность того, что после этого на скамейку сможет сесть влюблённая пара (влюблённым обязательно надо сидеть рядом)?

\begin{ist}
Алексей Суздальцев.
\end{ist}
}

\solution{Любую рассадку пассажиров можно представить в виде последовательности из нулей и единиц длины $n$, в которой единиц ровно $k$. Всего таких последовательностей $C_n^k$. Найдем количество последовательностей, \emph{не} удовлетворяющих условию, то есть не содержащих сдвоенных нулей (назовем такие последовательности \emph{плохими}).

Припишем к каждой плохой последовательности фиктивную единицу справа. Тогда в новой последовательности после любого нуля стоит единица, а значит, вся последовательность состоит из паттернов <<1>> и <<01>>. Паттернов <<01>> ровно $n-k$, (столько же, сколько и нулей), всего же паттернов столько же, сколько и единиц, то есть $k+1$. Таким образом, мы имеем $k+1$ позиций, из которых надо выбрать $n-k$, куда встанет паттерн <<01>>. Способов сделать это всего $C_{k+1}^{n-k}$, что и равно числу плохих последовательностей.

Значит, искомая вероятность равна $1-\frac{C_{k+1}^{n-k}}{C_n^k}$.}


\problem{
В классе 28 человек, среди них 18 девочек. Класс построили в 4 ряда по 7 человек. Какова вероятность того, что рядом с Вовочкой будет стоять хотя бы одна девочка?
\begin{note}
Для Вовочки любая девушка "--- рядом :).
\end{note}
 }
\solution{ }



\subsection{Геометрическое распределение (и близкие по духу)}
% первое упоминание о методе первого шага


\problem{
Равной силы команды играют до трёх побед. Какова вероятность того,
что будет ровно 3 партии? Ровно 4? Ровно 5? }
\solution{ $\PP(N=3)=2\frac{1}{2}^{3}$; $\PP(N=4)=2C_{3}^{1}\frac{1}{2}^{4}$; $\PP(N=5)=2C_{4}^{2}\frac{1}{2}^{5}$. }
\cat{geom_d}

\problem{Вася стреляет по мишени бесконечное количество раз. Он попадает по мишени с очень маленькой вероятностью. Какова вероятность того, что до первого попадания по мишени Васе потребуется больше времени, чем в среднем уходит на одно попадание?}
\solution{Путь $p$ "--- вероятность. Тогда $\E(X)=\frac{1}{p}$. Нас интересует $\PP\left(X>\frac{1}{p}\right)=\left(1-\frac{1}{p}\right)^{\frac{1}{p}}\approx \frac{1}{e}$.}
\cat{geom_d}


\problem{  \zdt{Геометрическое распределение}
Кубик подбрасывают до первого выпадения шестерки. Случайная величина  $N$ "---
число подбрасываний.
\begin{itemize}
\item Найдите $\PP(N=6)$, $\PP(N=k)$, $\PP(N>10)$ и $\PP(N>30\mid N>20)$, $\E(N)$.
\item Найдите $\E\left(\frac{1}{N}\right)$.
\end{itemize}
 }
\solution{ }
\cat{geom_d}


\problem{ \label{s chego vse nachinalos}\zdt{С чего всё начиналось\ldotst{}}

Париж, Людовик XIV, 1654 год, высшее общество говорит о рождении
новой науки "--- теории вероятностей. Ах, кавалер де Мере, <<fort
honn\^{e}te homme sans \^{e}tre math\'{e}maticien>>\ldotst{} (<<благородный
человек, хотя и не математик>>). Старая задача, неправильные
решения которой предлагались тысячелетиями (например, одно из
неправильных решений предлагал изобретатель двойной записи, кумир
бухгалтеров, Лука Пачоли), наконец решена правильно! Два игрока
играют в честную игру до шести побед. Игрок, первым выигравший
шесть партий (не обязательно подряд), получает 800 рублей. К
текущему моменту первый игрок выиграл 5 партий, а второй "--- 3
партии. Они вынуждены прервать игру в данной ситуации. Как им поделить приз по справедливости? }
\solution{ $700:100$.}


\problem{ \zdt{Von Neumann. Что делать, если монетка неправильная?}

Имеется <<несправедливая>> монетка, выпадающая гербом с некоторой
вероятностью. Под раундом будем подразумевать двукратное
подбрасывание монеты. Проводим первый раунд. Если результат раунда
"--- ГР (сначала герб, затем решка), то считаем, что выиграл первый
игрок. Если результат раунда "--- РГ, то считаем, что выиграл второй
игрок. Если результат раунда "--- ГГ или РР, то проводим ещё один
раунд. И так далее, пока либо не определится победитель, либо
количество раундов не достигнет числа $n$.
\begin{enumerate}
\item Найдите вероятности <<ничьей>>, выигрыша первого игрока, выигрыша
второго игрока в зависимости от $n$. Найдите пределы этих
вероятностей при $n\rightarrow +\infty$.
\item Как с помощью неправильной монетки сымитировать правильную?
\end{enumerate}
 }
\solution{ }

\problem{ \label{Monty Hell problem} \textit{Monty Hell problem} (не путать с Monty Hall)

\textbf{Сказка.} Ежедневно Кощей Бессмертный получает пенсию в размере 10~золотых монет. Затрат у Кощея нет никаких. Поэтому с начала пенсионного возраста он аккуратно нумерует
каждую полученную монету и кладет её в сундук. Ночью Мышка-норушка крадёт одну золотую монету из сундука.
\begin{enumerate}
\item Какова вероятность того, что $i$-я монета когда-либо исчезнет из Сундука?
\item Какова вероятность того, что хотя бы одна монета пролежит в сундуке бесконечно долго?
\item Дни сокращаются в продолжительности (каждый последующий "--- в два раза короче, чем предыдущий). Сколько монет будет в сундуке в конце времени?
\end{enumerate}
\begin{hint}
$(1-x) \le e^{-x}$.
\end{hint}
\begin{note}
А где надсказка?
\end{note}
 }
\solution{ Вероятность, что $i$-я монета когда-либо исчезнет, равна $1$, а того, что пролежит бесконечно долго, "--- 0. }

\problem{
Случайным образом выбирается натуральное число $X$. Вероятность выбора числа $n$ такова: $\PP(X=n)=2^{-n}$.
\begin{enumerate}
\item  Какова вероятность того, что будет выбрано чётное число? Нечётное число? Число, большее пяти? Число от 3 до 11?
\item Пусть независимо друг от друга выбираются $c$ чисел. Пусть $K_{c}$ "--- количество невыбранных чисел на отрезке от одного до наибольшего выбранного числа. Найдите $\PP(K_{c}=k).$
\end{enumerate}
\begin{ist}
AMM E3061, T.~Ferguson and C.~Melolidakis.
\end{ist}
 }
\solution{ $P(K_{c}=k)=2^{-(k+1)}$ вне зависимости от $c$. Для начала обнулим значение $K_{c}$ и возьмём в руку $c$ монеток. Подкинем монетки. Если все выпали орлом, мы прибавляем единичку к $K_{c}$. Если все выпали решкой, то мы объявляем значение $K_{c}$. Если часть выпала орлом, часть решкой, то выкинем те, что выпали решкой, и снова перейдем к подкидыванию монеток. В результате имее: рост $K_{c}$ на единицу или глобальная остановка процесса происходит равновероятно. Значит, $K_{c}$ распределено геометрически. }
\cat{hard}

\problem{Величины $X_1$, $X_2$, \ldots независимы и одинаково распределены с некоторой функцией плотности $f$. Величина $X_i$ --- это количество осадков в $i$-ый год. Пусть $Y$ --- номер года, когда впервые будет превышено количество осадков, выпавших в первом году.
Найдите закон распределения $Y$ и $\E(Y)$}
\solution{Найдем $\P(Y>k)$. Это вероятность того, что в первые $k$ лет не будет достигнут уровень первого года. Значит это вероятность того, что первый год дал наибольшее количество осадков за первые $k$ лет. В силу симметрии $\P(Y>k)=1/k$. Отсюда $\P(Y=k)=\P(Y>k-1)-\P(Y>k)=1/k(k-1)$ и $\E(Y)=+\infty$}

% untyp
\problem{Преподаватель по теории вероятностей пообещал своим студентам, что включит задачу на геометрическое распределение в экзамен с вероятностью 1/3. Чтобы исполнить своё обещание он подбросил одну монетку два раза и посчитал количество орлов. Оказалось, что орлов было ровно два. На основании этого количества он принял решение. Какое решение он принял и почему?}
\solution{Подбрасывая монетки детерминированное количество раз нельзя получить вероятность 1/3, значит преподаватель заранее не знал, сколько раз он будет подбрасывать монетку.  Простое правило принятия решения может иметь примерно такой вид: если выпало А, то давать задачу, если выпало Б, то не давать задачу, если не выпало ни А, ни Б, то повторить эксперимент. Орлов может быть либо два, либо один, либо 0. На два орла преподаватель повторять эксперимент не стал. Разумно предположить, что он использовал простую стратегию. Значит два орла означают <<включать>>, один орел --- <<не включать>>, ноль орлов --- подбрасывать монетку еще два раза. }
Идея: Николай Арефьев

\problem{Вася прыгает в длину несколько раз подряд. Результаты васиных прыжков --- независимые одинаково распределенные непрерывные случайные величины. Прыгнув в первый раз он записывает результат. И прыгает до тех пор, пока не перепрыгнет свой первый результат. Величина $X$ --- сколько прыжков Васе потребуется сделать дополнительно, чтобы перепрыгнуть первый результат. Найдите $\P(X=k)$ и $\E(X)$.}
\solution{$\P(X=k)=\frac{1}{k(k+1)}$, т.к. последний прыжок должен быть самым длинным из $k+1$ прыжка, а первый --- самым длинным из $k$ оставшихся. $\E(X)=\infty$.}




\subsection{Из \textit{n} предметов выбирается \textit{k}}
\problem{ \label{id008}
Из 50 деталей 4 бракованные. Выбирается наугад 10 деталей на проверку.
Какова вероятность не заметить брак? }
\solution{$\frac{C_{46}^{10}}{C_{50}^{10}}$. }

\problem{
Есть 4 карты одного достоинства. Наугад выбираются две.
Какова вероятность того, что они будут разного цвета? }
\solution{$\frac{1}{3}$. }

\problem{ \label{5 iz 36}
Какова вероятность полностью угадать комбинацию в лотерее 5 из
36?}
\solution{$\frac{1}{C_{36}^{5}}$. }

\problem{
В мешке 50 орехов, из них 5 пустые. Вы выбираете наугад 10
орехов. Какова вероятность того, что ровно один из них будет
пустой?}
\solution{ $\frac{C_{45}^{9}C_{5}^{1}}{C_{50}^{10}}$.}


\problem{ \label{tri shara iz korobki}
Из коробки с 4 синими и 5 зелёными шарами достают 3 шара. Пусть
$B$  и  $G$  "--- количество извлечённых синих и зелёных шаров.
Найдите  $\E(B)$,  $\E(G)$,  $\E(B\cdot G)$,  $\E(B-G)$. }
\solution{$\E(B)=3\cdot\frac{4}{9}=\frac{4}{3}$; $\E(G)=3-\E(B)=\frac{8}{3}$; $\E(B-G)=-\frac{4}{3}$; $\E(B\cdot G)=2\cdot\frac{5}{6}$.  }




\problem{
На факультете $n$ студентов. Из них наугад выбирают $a$ человек. Через год $b_{-}$ студентов покидают факультет, $b_{+}$ студентов приходят на факультет. Из них снова наугад выбирают $a$. Какова вероятность того, что хотя бы одного выберут два раза? }
\solution{ }


\problem{ \label{cube-cut-1}(см. тж. с.~\pageref{cube-cut-2})

 \ENGs A wooden cube that measures 3 cm along each edge is painted red. The painted cube is then cut into 27 pieces of 1-cm cubes.
\begin{enumerate}
\item If I choose one of the small cubes at random and toss it in the air, what is the probability that it will land red-painted side up?
\item If I put all the small cubes in a bag and randomly draw out 3, what is the probability that at least 3 faces on the cubes I choose are painted red?
\item If I put the small cubes in a bag and randomly draw out 3, what is the probability that exactly 3 of the faces are painted red?
\item Invent a new question!
\end{enumerate} \RUSs
\begin{ist}
\url{http://letsplaymath.wordpress.com/2007/07/25/puzzle-random-blocks/}
\end{ist}
 }
\solution{ Вероятность выпадения красной стороны сверху равна $\frac{1}{3}$. }

\problem{
Контрольную пишут 40 человек. Половина пишет первый вариант, половина "--- второй. Время написания работы каждым студентом "--- независимые непрерывные случайные величины. Какова вероятность того, что в тот момент, когда будет сдана последняя работа первого варианта, останется ещё 5 человек, пишущих второй вариант? }
\solution{ $\frac{C_{20}^{1}C_{20}^{5}}{C_{40}^{6}}\frac{1}{6}$ (вероятность заданной шестёрки финалистов помножить на вероятность выбора одного человека из шести). }


\problem{В бридж играют четыре игрока: Юг, Восток, Север, Запад. Перемешанная колода в 52 карты раздаётся игрокам по очереди по одной карте. Юг и Север получили 11~пик. Какова вероятность того, что две оставшиеся пики оказались у одного игрока? Разделились между остальными игроками? Каковы вероятности различных раскладов пик между остальными игроками, если Юг и Север получили 8~пик?}
\solution{}

\problem{ \label{korrektori ochepiatok} \zdt{Корректоры очепяток}

Вася замечает очепятку с вероятностью $0{,}7$; Петя независимо от Васи замечает очепятку с вероятностью $0{,}8$. В книге содержится 100 опечаток. Какова вероятность того, что Вася заметит 30 опечаток, Петя "--- 50, причём 14 опечаток будут замечены обоими корректорами? }
\solution{
$\frac{100!}{14!16!36!34!}0{,}7^{30}0{,}3^{70}0{,}8^{50}0{,}2^{50}$. }



\subsection{Биномиальное распределение (до дисперсии)}
%1.4. Эксперимент состоит из множества одинаковых этапов
%(сюда можно отнести простые задачи на биномиальное распределение и совсем простую комбинаторику)

% спорные случаи - эксперимент повторяется два раза - можно отнести в одношаговые (если обозримо в явном виде выписать все исходы)

\problem{
Монетка подбрасывается 5 раз. Какова вероятность того, что будет
выпадет ровно один орёл? Ровно два? Ни одного? }
\solution{$\PP(N=1)=C_{5}^{1}(\frac{1}{2})^{5}$; $\PP(N=2)=C_{5}^{2}(\frac{1}{2})^{5}$; $\PP(N=0)=C_{5}^{0}(\frac{1}{2})^{5}$. }
\cat{die} \cat{binomial}

\problem{
Какова вероятность при шести подбрасываниях кубика получить ровно
две шестёрки? }
\solution{$\PP(N=2)=C_{6}^{2}(\frac{1}{6})^{2}(\frac{5}{6})^{4}$. }
\cat{binomial}


\problem{
Какова вероятность того, что у десяти человек не будет ни одного совпадения дней рождений? Каков минимальный размер компании, чтобы вероятность одинакового дня рождения была больше половины? }
\solution{$\frac{365\cdot 364\cdot 363\cdot \ldots \cdot 356}{365^{n}}$; минимальная компания состоит из $23$ человек. }


\problem{
Маша подбрасывает монетку три раза, а Саша "--- два раза. Какова
вероятность того, что у Маши герб выпадет больше раз, чем у
Саши?}
\solution{ $\PP=\frac{1}{4}(1-\frac{1}{8})+\frac{1}{2}(\frac{1}{8}+3\frac{1}{8})+\frac{1}{4}\frac{1}{8}=\frac{1}{2}$.}


\problem{ \label{deti raznih polov}
Сколько детей должно быть в семье, чтобы вероятность того,
что имеется по крайней мере один ребенок каждого пола, была больше
0,95? }
\solution{$(\frac{1}{2})^{(n-1)}\le 0{,}15$.  }


\problem{ \zdt{Осторожный фальшивомонетчик}

Дворцовый чеканщик кладёт в каждый ящик вместимостью в сто монет
одну фальшивую. Король подозревает чеканщика и подвергает проверке
монеты, взятые наудачу по одной в каждом из 100 ящиков.
\begin{enumerate}
\item Какова вероятность того, что чеканщик не будет разоблачён?
\item Каков ответ в предыдущей задаче, если 100 заменить на $n$?
\end{enumerate}
\begin{ist}
Mosteller.
\end{ist}
}
\solution{ $0{,}99^{n}$. }


\problem{ \label{strategia udvoenia} \zdt{Стратегия удвоения}

В казино имеется рулетка, которая с вероятностью $0{,}5$ выпадает
или на чёрное, или на красное. Игрок, поставивший сумму $n$ и угадавший
цвет, получает обратно сумму $2n$. Вася решил играть по следующей
схеме. Сначала он ставит доллар. Если он выигрывает, то покидает
казино, если проигрывает, то удваивает ставку и ставит два
доллара. Если выигрывает, то покидает казино, если проигрывает, то
снова удваивает ставку и ставит четыре доллара и т.\,д., пока не
выиграет в первый раз или впервые не хватит денег на новую
удвоенную ставку. У Васи имеется 1\,050 долларов.
\begin{enumerate}
\item Какова вероятность того, что Вася покинет казино после выигрыша?
\item Каков ожидаемый выигрыш Васи?
\end{enumerate}
\begin{note}
В реальности вероятность меньше $0{,}5$, т.\,к. на
рулетке есть 0 и (иногда) 00. Их наличие, естественно, уменьшает и
вероятность, и ожидаемый выигрыш.
\end{note}
 }
\solution{ $1-\frac{1}{1024}$; $0$. }


\problem{ \ENGs
When the $n$'s dice are thrown at the one time, find the probability such that the sum of the cast is $n+3$? \RUSs }
\solution{ }


\problem{
Пусть $X_{1}$, $X_{2}$,..., $X_{n}$ "--- НОРСВ, такие, что $X_{i}= \begin{cases}  1, & p; \\ 0, & (1-p).\end{cases}$ Пусть $k$ "--- такая константа, что $2k\ge n$. Найдите вероятность того, что самая длинная серия из единиц имеет длину $k$. Что делать при $2k<n$? }
\solution{ }
\cat{wrong_class}


\problem{ \ENGs
Suppose you are given a random number generator, which draws samples from an uniform distribution between $0$ and $1$.
The question is: how many samples you have to draw, so that you are 95\% sure that at least 1 sample lies between $0.70$ and $0.72$? \RUSs }
\solution{ }


\problem{  \zdt{Биномиальное распределение}

Кубик подбрасывают 5 раз. Пусть $N$ "--- количество выпадений шестёрки. Найдите $\PP(N=3)$, $\PP(N=k)$  и
$\PP(N>4 \mid N>3)$, $\E(N)$.}
\solution{ }

\problem{ \zdt{Максимальная вероятность для биномиального распределения}

Пусть $X$ распределена биномиально. Общее число экспериментов
равно $n$, вероятность успеха в отдельном испытании равна $p$.
\begin{enumerate}
\item Найдите $\frac{\PP(X=k)}{\PP(X=k-1)}$.
\item При каких $k$ дробь $\frac{\PP(X=k)}{\PP(X=k-1)}$ будет не меньше 1?
\item Каким должно быть $k$, чтобы $\PP(X=k)$ была максимальной?
\end{enumerate}
 }
\solution{ }

\problem{ Известно, что предварительно зарезервированный билет на автобус
дальнего следования выкупается с вероятностью 0{,}9. В обычном
автобусе 18~мест, в микроавтобусе 9~мест. Компания <<Микро>>,
перевозящая людей в микроавтобусах, допускает резервирование 10~билетов на один микроавтобус. Компания <<Макро>>, перевозящая
людей в обычных автобусах допускает резервирование 20~мест на один автобус. У какой компании больше вероятность оказаться в ситуации нехватки
мест? }
\solution{ }

\problem{ \ENGs
The psychologist Tversky and his colleagues say that about four
out of five people will answer (a) to the following question:
\begin{quote}
A certain town is served by two hospitals. In the larger
hospital about 45 babies are born each day, and in the smaller
hospital 15 babies are born each day. Although the overall
proportion of boys is about 50 percent, the actual proportion at
either hospital may be more or less than 50 percent on any day. At
the end of a year, which hospital will have the greater number of
days on which more than 60 percent of the babies born were boys?
\end{quote}
\begin{center}
\begin{tabular}{ccc}
(a) the large hospital & (b) the small hospital & (c) neither (about the same) \\
\end{tabular}
\end{center}\RUSs

Дайте верный ответ и попытайтесь объяснить, почему большинство
людей ошибается при ответе на этот вопрос. }
\solution{В маленьком роддоме <<мальчиковых>> дней больше. В силу закона больших чисел, чем больше число наблюдений, тем сильнее выборочная доля мальчиков должна быть похожа на вероятность рождения мальчика.}

\problem{
В забеге участвуют 12 лошадей. Каждый из 10 зрителей пытается составить свой прогноз для трёх призовых мест. Какова вероятность того, что хотя бы один из них окажется прав? }
\solution{ $ 1-(\frac{1319}{1320})^{10}\approx 0{,}008 $. }


\problem{
Есть $N$ монеток. Каждая из них может быть фальшивой с
вероятностью $p$. Известно, сколько весят настоящие. Известно, что
фальшивые весят меньше, чем настоящие. Каждая фальшивая может иметь
своё отклонение от правильного веса. Задача "---
определить, является ли фальшивой каждая монета. Предлагается следующий способ:
\begin{quote}
Разбить монеты на группы по $n$ монет в каждой группе. Взвесить
каждую группу. Если вес группы совпадает с эталонным, то вся
группа признается настоящей. Если вес группы меньше эталонного, то
каждая монеты из
группы взвешивается отдельно.
\end{quote}

Предположим, что $N$ делится на $n$. Пусть $X$ "--- требуемое число взвешиваний.
\begin{enumerate}
\item Найдите $\E(X)$;
\item При каком условии на $p$ и $n$ предложенный способ более
эффективен чем взвешивание каждой монеты?
\item Исследуйте поведение функции $\frac{\E(X)}{N}$ от $n$ (есть ли минимум, максимум и т.\,д.).
\end{enumerate}
 }
\solution{ }


\problem{ \zdt{Задача Банаха (Banach's matchbox problem)}

У Маши есть две коробки, в каждой из которых осталось по $n$~конфет. Когда Маша хочет конфету, она выбирает наугад одну из
коробок и берёт конфету оттуда. Рано или поздно Маша впервые
откроет пустую коробку. В этот момент другая коробка содержит
некоторое количество конфет. Обозначим за $u_r$ вероятность того, что
в другой коробке ровно $r$ конфет.
\begin{enumerate}
\item  Найдите $u_r$.
\item  Найдите вероятности $v_r$ того, что в тот момент, когда из
одной коробки возьмут последнюю конфету (она только станет
пустой!), в другой будет находится ровно $r$ конфет.
\item  Найдите вероятность того, что коробка, которая была опустошена
раньше, не будет первой коробкой, открытой пустой.
\end{enumerate}
 }
\solution{
Пункт 1. Последняя попытка взять конфету "--- из пустой коробки. Назовём
эту коробку $A$. Из предыдущих $n+(n-r)$ конфет $n$ приходятся на
коробку $A$. Вероятности равны $\frac{1}{2}$. Получаем:
$u_{r}=\frac{C_{2n-r}^{n}}{2^{n+(n-r)}}$ }


\problem{
В уездном городе $N$ два родильных дома, в одном ежедневно рождается 50 человек, в другом "--- 100 человек. В каком роддоме чаще рождается одинаковое количество мальчиков и девочек?}
\solution{В маленьком. }

\problem{
\ENGs Let you choose an infinite sequence of integers between 1 and 10, what is the possibility that your sequence doesn't have any ``1''? \RUSs }
\solution{ 0. }


\problem{ \ENGs
There are three coins in a box.  These coins when flipped, will
come up heads with respective probabilities $0.3$, $0.5$, $0.7$.  A
coin is randomly selected (meaning uniform distribution!) from among
these three and then flipped $10$ times.  Let $N$ be the number of
heads obtained on the first ten flips. \RUSs
\begin{enumerate}
\item Найдите $\PP(N=0)$.
\item If you win \$1 each time a head appears and you lose \$1 each time a tail appears, is this a fair game?  Explain.
\end{enumerate}
 }
\solution{ }


% серия задач, которые могут казаться интуитивно противоречивыми...
\problem{Как почувствовать разницу в 0{,}01\cite{sekei:paradox}? Пусть вероятность того, что Маша находится целый день в хорошем настроении, равна 0{,}99, а вероятность того, что Саша находится в хорошем настроении, равна 0{,}999\,9. Какова вероятность того, что Маша будет целый год непрерывно в хорошем настроении? Саша?}
\solution{0{,}025\,5 и 0{,}964\,2.}


\problem{Петя подбрасывает 10 монеток. Если из этих 10~подбрасываний будет как минимум 8~одинаковых, то мы назовём это чудом. Какова вероятность чуда? Какова вероятность хотя бы одного чуда, если, кроме Пети, ещё 9~человек подбрасывает по 10~монеток?}
\solution{$\frac{7}{64}$; около $\frac{2}{3}$.}

\problem{Вероятность того, что прошлогодний грецкий орех будет червивым, равна 0{,}25. Сколько минимум нужно взять грецких орехов, чтобы среди них был хотя бы один нормальный с вероятностью 99,9\,\%?}
\solution{5.}

\problem{ \zdt{Биномиальные числа Фибоначчи}

Пусть $\{F_k\}$ "--- последовательность Фибоначчи, а $X$ "--- число выпавших орлов при $n$ подбрасываниях правильной монетки. Вычислить $\E(F_{1+X})$.
\begin{ist}
Алексей Суздальцев.
\end{ist}
}

\solution{$\frac{F_{2n+1}}{2^n}$. У чисел Фибоначчи есть свойство: $F_{n}=F_{n-1}+F_{n-2}\hm =L(1+L)F_{n} \hm=\ldots \hm =L^{k}(1+l)^{k}F_{n} \hm=(1+L)^{k}F_{n-k}$.}

\problem{Семеро друзей выбирают, пойти им в кино на фильм ужасов или на комедию. Каждый из них предпочтёт комедию независимо от других с вероятностью $0{,}6$. Есть два способа голосования, А и Б. Способ~А "--- все голосуют одновременно, выбирается альтернатива, набравшая больше голосов. Способ~Б "--- голосование в два тура. Первый тур: трое самых старших друзей голосуют между собой и большинством решают, за что они втроём будут голосовать единогласно во втором туре: за комедию или за ужасы. Второй тур: голосуют все семеро, но трое старших голосуют так, как согласованно договорились на первом туре. При каком способе голосования выше шансы пойти на комедию? }
\solution{}


\problem{Маша и Саша учатся в одном классе. Маша и Саша учатся по одним и тем же $n$ учебникам. В один день Маша и Саша независимо друг от друга приносят случайное подмножество своих учебников в школу.
\begin{enumerate}
\item Какова вероятность того, что у Маши не будет ни одного учебника, которого бы не было у Саши?
\item Какова вероятность того, что вместе у Саши и Маши будут все $n$ учебников хотя бы в одном экземпляре?
\end{enumerate}}
\solution{Можно считать, что каждый учебник Саша и Маша берут с вероятностью $0{,}5$. Ответ в обоих пунктах: $0{,}75^n$.}


\subsection{Деревья и прочее без условных вероятностей}

\problem{
Подбрасывается кубик, а затем монетка подбрасывается столько раз,
сколько очков на выпавшей грани. Какова вероятность того, что
орёл выпадет ровно 4 раза?}
\solution{ $\PP=\frac{1}{6}\left(\left(\frac{1}{2}\right)^{4}+C_{5}^{4}\left(\frac{1}{2}\right)^{5}+C_{6}^{4}\left(\frac{1}{2}\right)^{6}\right).$
}


\problem{ \label{ritsari-bliznetsi} \zdt{Рыцари-близнецы }

Король Артур проводит рыцарский турнир, в котором, так же как и в
теннисе, порядок состязания определяется жребием. Среди восьми рыцарей, одинаково искусных в
ратном деле, два близнеца.
\begin{enumerate}
\item Какова вероятность того, что они встретятся в поединке?
\item Каков ответ в случае $2^n$ рыцарей?
\end{enumerate}
 }
\solution{$\PP_1=\frac{1}{7}+\frac{1}{14}+\frac{1}{28}$; $\PP_2=\frac{1}{2^{n}-1}\cdot 2\cdot\left(1-0{,}5^{n}\right)$.  }
\begin{ist}
Mosteller.
\end{ist}

\problem{ \label{Vasia i Petia na lektsii}
Вася посещает 60\,\% лекций по теории вероятностей, Петя "--- 70\,\%. Они
из разных групп и посещают лекции независимо друг от друга. Какова
вероятность, что на следующую лекцию придут оба? Хотя бы один из
них?}
\solution{ $\PP(N=2)=0{,}7\cdot 0{,}6=0{,}42$. $\PP(N\geq 1)=1-(1-0{,}7)\cdot (1-0{,}6)=0{,}88$. }

\problem{ \label{vtoroi v finale} \zdt{Выйдет ли второй в финал?} \par
В теннисном турнире участвуют 8~игроков. Есть три тура
(четвертьфинал "--- полуфинал "--- финал). Противники в первом туре
определяются случайным образом. Предположим, что лучший игрок
всегда побеждает второго по мастерству, а тот, в свою очередь
побеждает всех остальных. Проигрывающий в финале занимает второе
место. Какова вероятность
того, что это место займет второй по мастерству игрок?
\begin{ist}
обработка Mosteller.
\end{ist}
}
\solution{ $\PP=\frac{4}{7}$. }

\problem{ \ENGs
The Wimbledon Men's Singles Tournament has 128 players. The first round pairings are completely random, subject to the constraint that none of the top 32 players can be paired against each other. Two competitors, Olivier Rochus, and his brother Christophe are competing, and neither are in the elite group of 32 players. What is the probability that these brothers play in the first round (as actually occurred)? \RUSs }
\solution{ }

\problem{
Первый автобус отходит от остановки в 5:00. Далее интервалы между
автобусами равновероятно составляют 10 или 15 минут, независимо от
прошлых интервалов. Вася приходит на остановку в 5:42.
\begin{enumerate}
\item Какова ожидаемая длина интервала, в который он попадает?
\item  Какова ожидаемая длина следующего интервала?
\item  Пусть $t\to\infty$ (???)
\end{enumerate}
 }
\solution{ Ожидаемая длина следующего интервала "--- 12{,}5 минут. }

\problem{ \ENGs
There are two ants on opposite corners of a cube. On each move, they can travel along an edge to an adjacent vertex. What is the probability that they both return to their starting point after 4 moves? \RUSs }
\solution{$(\frac{7}{27})^{2}$. }


\problem{ \label{legkomislennii chlen juri} \zdt{Легкомысленный член жюри} \par
В жюри из трёх человек два члена независимо друг от друга
принимают правильное решение с вероятностью $p$, а  третий для
вынесения    решения бросает монету (окончательное решение
выносится большинством голосов). Жюри из одного человека выносит
справедливое решение с вероятностью $p$. Какое из этих жюри
выносит справедливое решение с большей вероятностью?
\begin{ist}
Mosteller.
\end{ist}
 }
\solution{ $p-\frac{p^{2}}{2}<p$, т.\,е. жюри из одного человека лучше. }


\problem{ \label{Simpson's paradox} \zdt{Simpson's paradox} \par
Тренер хочет отправить на соревнование самого сильного из своих
спортсменов. Самым сильным игроком тренер считает того, у кого
больше всех шансов получить первое место, если бы соревнование
проводилось среди своих. У тренера два спортсмена: А, постоянно
набирающий 3~штрафных очка при выполнении упражнения, и Б,
набирающий 1~штрафное очко с вероятностью 0{,}54 и 5~штрафных очков
с вероятностью 0{,}46.
\begin{enumerate}
\item Кого отправит тренер на соревнования?
\item Кого отправит тренер на соревнования, если, помимо А и Б, у него
тренируется спортсмен В, набирающий 2 штрафных очка с вероятностью
0,56, 4 штрафных очка с вероятностью 0,22 и 6 штрафных очков с
вероятностью 0,22.
\item Мораль?
\end{enumerate}
 }
\solution{ Спортсмена Б, если нет спортсмена В; спортсмена А, если есть спортсмен В. Мораль "--- зависимость от третьей альтернативы. }


\problem{
В турнире участвуют 8 человек, разных по силе. Более сильный побеждает более слабого. Проигравший выбывает, победитель выходит в следующий тур.
Какова вероятность того, что $i$-й по силе игрок дойдет до финала? }
\solution{ }


\problem{ \ENGs
A bag contains a total of $N$ balls either blue or red. If $5$ balls are chosen from the bag the probability all of them being blue is 0.5. What are the values of $N$ for which this is possible? \RUSs}
\solution{ }

\problem{ \ENGs
Each of two boxes contains both black and white marbles, and the total number of marbles in the two boxes is 25. One marble is taken out of each box randomly. The probability that both marbles are black is $\frac{27}{50}$. What is the probability that both marbles are white? \RUSs }
\solution{ }


\problem{ \label{gadanie v pole}
Маша с подружкой гуляют в поле. Подружка предлагает погадать на
суженого. Она зажимает в руке шесть травинок так, чтобы концы
травинок торчали сверху и снизу. Маша сначала связывает эти
травинки попарно между собой сверху, а затем и снизу (получается
три завязывания сверху и три завязывания снизу). Если при этом все
шесть травинок окажутся связанными в одно кольцо, то это означает,
что Маша в текущем году выйдет замуж.
Какие шансы у Маши?
\begin{note}
Будем считать, что завязывание травинок в <<трилистник>>, <<восьмерку>> и прочие нетривиальные узлы также
означает замужество.
\end{note}
\begin{ist}
Баврин, Фрибус, <<Старинные задачи>>.
\end{ist}
 }
\solution{$\frac{8}{15}$. }


\problem{
Две урны содержат одно и то же количество
шаров, несколько чёрных и несколько белых каждая. Из них
извлекаются $n$ ($n>3$) шаров с возвращением. Найти число $n$ и
содержимое обеих урн, если вероятность того, что все белые шары
извлечены из первой урны, равна вероятности того, что из второй
извлечены либо все белые, либо все чёрные шары.
\begin{ist}
Mosteller.
\end{ist}
}
\solution{ }
\cat{wrong_class}

\problem{ \label{po rublu za 6}
Кость подбрасывается 3 раза. Размер ставки "--- 1 рубль. Если
шестёрка не выпадает ни разу, то ставка проиграна, если шестёрка
выпадает хотя бы один раз, то ставка возвращается, плюс
выплачивается выигрыш по 1 рублю за каждую шестёрку. Найдите
стоимость этой лотереи. }
\solution{$\E(X)=3\cdot\frac{1}{6}+1+(-2)\cdot\left(\frac{5}{6}\right)^{3}$ }

\problem{
\label{Parrondo's game} \zdt{Parrondo's game}

Назовем <<рублёвой игрой с вероятностью $p$>> игру, в которой
игрок выигрывает 1~рубль с вероятностью $p$ и проигрывает один
рубль с вероятностью $(1-p)$. Игра $A$ "--- это рублёвая игра с вероятностью 0,45.
Игра $B$ состоит в следующем: если сумма в твоём кошельке делится
на три, то ты играешь в рублёвую игру с вероятностью 0,05; если же
сумма в твоем кошельке не делится на три, то ты играешь в рублёвую
игру с вероятностью 0{,}7. Что будет происходить с ожидаемым благосостоянием игрока, если он
\begin{enumerate}
\item Будет постоянно играть в игру $A$?
\item  Будет постоянно играть в игру $B$?
\item  Будет постоянно играть $A$ или $B$ с вероятностью по 0,5?
\item  Придумайте <<лохотрон>> для интеллектуалов.
\end{enumerate}
 }
% d - идея Ромы Мартусевича
\solution{ При игре только в $A$ "--- убывать; только в $B$ убывать; в вероятностную комбинацию "--- возрастать. }

\problem{ \zdt{Parrondo's game --- альтернатива} \ENGs

A much simpler example is dealing cards from a well-shuffled deck. Suppose I get \$14 if two cards in a row match in rank (two 2's or two Kings for examples), and pay \$1 if they don't. The chance of two cards in a row matching is $\frac{1}{17}$, so I pay \$16 for each \$14 I win.

Now I play the same game, alternating the deal between two decks. Now the chance of two successive cards matching is $\frac{1}{13}$, so I pay \$12 for every \$14 I win.

Each game individually loses money, but alternate them and you win money. Eureka! We're all rich. \RUSs
\begin{ist}
Wilmott forum.
\end{ist}
 }
\solution{ }

\problem{ \zdt{Триэль }

Три гусара "--- $A$, $B$ и $C$ "--- стреляются за прекрасную даму. Стреляют
они по очереди ($A$, $B$, $C$, $A$, $B$, $C$, \ldotst), каждый стреляет в
противника по своему выбору. $A$ попадает с вероятностью 0.1, $B$
"--- 0.5, $C$ "--- 0.9. Триэль продолжается до тех пор, пока в живых не
останется только один. Предположим, что стрелять в воздух нельзя.
\begin{enumerate}
\item Какой должна быть стратегия $A$?
\item У кого какие шансы на победу?
\end{enumerate}
 }
\solution{ }

\problem{ \zdt{Триэль-2}

Три гусара "--- $A$, $B$ и $C$ "--- стреляются за прекрасную даму. Стреляют
они одновременно, каждый стреляет в противника по своему выбору.
$A$ попадает с вероятностью 0,1, $B$ "--- 0,5, $C$ "--- 0,9. Триэль
продолжается до тех пор, пока в живых не
останется только один или никого.
\begin{enumerate}
\item Какой должна быть стратегия $A$?
\item У кого какие шансы на прекрасную даму?
\end{enumerate} }
\solution{ }


\problem{
% переписать (у Менделя - не горошины вроде бы?)
У диплоидных организмов наследственные характеристики определяются
парой генов. Вспомним знакомые нам с 9-го класса горошины чешского
монаха Менделя. Ген, определяющий форму горошины, имеет две
аллели:  <<А>> (гладкая) и <<а>> (морщинистая). <<А>> доминирует над
<<а>>. В популяции бесконечное количество организмов. Родители
каждого потомка определяются случайным образом. Одна аллель
потомка выбирается наугад из аллелей матери, другая "--- из аллелей
отца. Начальное распределение
генотипов имеет вид: <<АА>> "--- 30\,\%, <<Аа>> "--- 60\,\%, <<аа>> "--- 10\,\%.
\begin{enumerate}
\item  Каким будет распределение генотипов в $n$-м поколении?
\item  Заметив закономерность, сформулируйте и докажите теорему
Харди"--~Вайнберга для произвольного начального распределения
генотипов.
\end{enumerate}
 }
\solution{ }

\problem{
У диплоидных организмов наследственные характеристики определяются
парой генов. Некий ген, сцепленный с полом, имеет две аллели:
<<А>> и <<а>>, т.\,е. девочка может иметь один из трёх генотипов
(<<АА>>, <<Аа>>, <<аа>>), а мальчик "--- только два (<<А>> и <<а>>; хромосома,
определяющая мужской пол, короче и не содержит нужного участка).
От мамы ребёнок наследует одну из двух аллелей (равновероятно), а
от отца либо наследует (тогда получается девочка), либо нет (тогда
получается мальчик). <<А>> доминирует <<а>>. В популяции
бесконечное количество организмов. Родители каждого
потомка определяются случайным образом.
\begin{enumerate}
\item Верно ли, что численность генотипов стабилизируется со временем?
\item Известно, что дальтонизм является признаком, сцепленным с
полом. Догадавшись, является ли он рецессивным или доминантным,
определите, среди кого (мужчин или женщин) дальтоников больше.
\end{enumerate}

/проверить биологию/ }
\solution{ }



\problem{
В коробке находится четыре внешне одинаковые лампочки. Две
лампочки исправны, две "--- нет. Лампочки извлекают из коробки по
одной до тех пор, пока не будут извлечены обе исправные.
\begin{enumerate}
\item Какова вероятность того, что опыт закончится извлечением трёх
лампочек?
\item  Каково ожидаемое количество извлеченных лампочек?
\end{enumerate}
 }
\solution{ }


\problem{ \label{spelestolog} \zdt{Спелестолог и батарейки}

У спелестолога в каменоломнях сели батарейки в налобном фонаре, и он оказался в абсолютной темноте. В рюкзаке у него 8~батареек: 5 новых и 3 старых. Для работы фонаря требуется две новые батарейки. Спелестолог вытаскивает из рюкзака две батарейки наугад и вставляет их в фонарь. Если фонарь не начинает работать, то спелестолог откладывает эти две батарейки и пробует следующие и т.\,д.
\begin{enumerate}
\item Сколько попыток в среднем потребуется?
\item Какая попытка вероятнее всего будет первой удачной?
\item Творческая часть. Поиграйтесь с задачей. Случайна ли прогрессия в ответе? Сравните с вариантом «6 новых $+$ 4 старых» и т.\,д.
\end{enumerate}
}

\solution{ \begin{tabular}{|c|c|c|c|c|}\hline
$N$ & 1 & 2 & 3 & 4 \bigstrut \\ \hline
$\PP$ & $\frac{5}{14}$ & $\frac{4}{14}$ & $\frac{3}{14}$ & $\frac{2}{14}$ \bigstrut \\ \hline
\end{tabular}

Решение для $6=4+2$: $\PP(N=1)=\frac{C_{4}^{2}}{C_{6}^{2}}=\frac{6}{15}$; $\PP(N=3)=\frac{4\cdot 2}{C_{6}^{2}}\frac{3\cdot 1}{C_{5}^{2}}=\frac{4}{15}$; $\PP(N=2)=\frac{5}{15}$; $\E(N)=\frac{28}{15}$. }



\problem{ \label{dva ferzia}
Два ферзя (чёрный и белый) ставятся наугад на шахматную доску.
\begin{enumerate}
\item Какова вероятность того, что они будут <<бить>> друг друга?
\item К чему стремится эта вероятность для шахматной доски со
стороной, стремящейся к бесконечности?
\end{enumerate}
 }
\solution{Вероятность того, что ферзи будут угрожать друг другу, равна $\frac{14}{63}+\frac{1}{64}\frac{1}{63}4(7\cdot 7+5\cdot 9+3\cdot 11+ 1\cdot 13)$.
Шахматная доска делится на четыре квадратных зоны с одинаковым числом клеток, покрываемых ферзём. Если длина стороны будет стремиться в бесконечность, то эта вероятность будет стремиться к
0, так как она равна отношению длин нескольких линий ко всей площади.  }


\problem{
На день рождения к Васе пришли две Маши, два Саши, Петя и Коля. Все вместе с Васей сели за круглый стол. Какова вероятность, что Вася окажется между двумя тёзками? }
\solution{ Слева должен сесть тот, у кого есть тёзка. $p_{1}=\frac{4}{6}$. Справа должен сесть его парный. $p_{2}=\frac{1}{5}$, итого $p=p_{1}\cdot p_{2}=\frac{2}{15}$. }



\problem{
Равновероятно независимо друг от друга выбираются три числа от 1 до 20. Какова вероятность того, что третье попадет между двух первых? }
\solution{ $\frac{57}{200}=0{,}285$. }

\problem{ \ENGs Five distinct numbers are randomly distributed to players numbered 1 through 5. Whenever two players compare their numbers, the one with the higher one is declared the winner. Initially, players 1 and 2 compare their numbers; the winner then compares with player 3. Let $X$ denote the number of times player 1 is a winner. Find the distribution of $X$. \RUSs }
\solution{ }



\problem{ \label{simple optimization}
Подбрасывается правильный кубик. Узнав результат, игрок выбирает,
подкидывать ли кубик второй раз. Игрок получает сумму денег, равную
количеству очков при последнем подбрасывании.
\begin{enumerate}
\item Каков ожидаемый выигрыш игрока при оптимальной стратегии?
\item Каков ожидаемый выигрыш игрока, если максимальное количество подбрасываний равно трём?
\end{enumerate}
 }
\solution{$\frac{1}{2}\cdot5+\frac{1}{2}\frac{7}{2}=4{,}25$. }

\problem{На столе стоят 42~коробки, они занумерованы от 0 до 41. В каждой коробке 41~шар, в коробке с номером $i$ лежат $i$ белых шаров, а остальные чёрные. Мы наугад выбираем коробку, а затем из неё достаём три шара. Какова вероятность того, что они будут одного цвета?
\begin{ist}
\url{http://math.stackexchange.com/questions/70760/}
\end{ist}
} % 42 --- это потому что это число является ответом на вопрос Жизни, Вселенной и Всего Такого?
\solution{Можно представить себе другое условие: в коробке 42 занумерованных шара, мы выбираем один наугад. Красим шары с меньшим номером в белый, остальные "--- в чёрный. Затем берём три шара. Это равносильно тому, что мы возьмём 4~шара с номерами 1, 2, 3, 4 и выберем из них разделитель цветов случайно. Значит, вероятность равна $\frac{1}{2}$.}


