% !Mode:: "TeX:UTF-8"
\section{Случайное блуждание}
\subsection{Дискретное случайное блуждание}


\problem{Какова вероятность того, что трёхмерное случайное блуждание бесконечное количество раз пересечёт вертикальную ось?}
\solution{1. Про вертикальные шаги можно забыть, а вероятность бесконечное количество раз посетить 0 для двумерного случайного блуждания равна 1.}


\problem{Пусть $X_{n}$ "--- симметричное случайное блуждание. Сколько времени в среднем придётся ждать, пока остаток от деления $X_{n}$ на 183 окажется равным 39?}
\solution{$39^{2}$.}



\subsection{Принцип отражения}
\problem{  \zdt{Выборы} \par
После выборов, в которых участвуют два кандидата, A и B, за них
поступило $a$ и $b$ ($a>b$) бюллетеней соответственно, скажем, 3 и
2. Если подсчёт голосов производится последовательным извлечением
бюллетеней из урны, то какова вероятность того, что хотя бы один
раз число вынутых бюллетеней, поданных за A и B, было одинаково? Какова вероятность того, что A всё время лидировал?
\begin{ist}
Mosteller.
\end{ist}
}
\solution{ }

\problem{ \zdt{Ничьи при бросании монеты} \par
Игроки A и B в орлянку играют $n$ раз. После первого бросания
каковы шансы на то, что в течение всей игры их выигрыши не
совпадут?
\begin{ist}
Mosteller.
\end{ist}}
\solution{ }

\problem{Доходность акции следует симметричному дискретному случайному блужданию. Какова вероятность того, что в момент времени $2k+1$ доходность будет выше, чем когда-либо в прошлом?

Источник: Алексей Суздальцев}
\solution{$\frac{C_{2k}^{k}}{2^{2k+1}}$. Совсем простого решения не знаю, хотя ответ простой.}



\subsection{Броуновское движение}
