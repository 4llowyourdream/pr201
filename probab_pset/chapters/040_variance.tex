% !Mode:: "TeX:UTF-8"
\section{$\Var$, $\sigma$, conditional $\Var$, conditional $\sigma$}
\subsection{Дискретный случай}

\problem{ \label{do 2-h v summe}
Кубик подбрасывают до тех пор, пока накопленная сумма очков на
гранях не превысит 2. Пусть  $X$  "--- число подбрасываний кубика.
Найдите  $\E(X)$, $\Var(X)$,
$\Var(36X-5)$, $\E(36X-17)$. }
\solution{ \begin{tabular}{|c|c|c|c|} \hline
  % after \\: \hline or \cline{col1-col2} \cline{col3-col4} ...
  $X$ & 1 & 2 & 3 \\ \hline
  $\PP$ & $\frac{24}{36}$ & $\frac{11}{36}$ & $\frac{1}{36}$ \bigstrut \\  \hline
\end{tabular}

$\E(X)=\frac{49}{36}$, $\E(36X-17)=32$. $\Var(X)=\frac{371}{1296}$, $\Var(36X-5)=371$. }


\problem{ \label{iz 5 detalei 2 brakovannih}
Из 5-ти деталей 3 бракованных. Сколько потребуется в среднем
попыток, прежде чем обнаружится первая дефектная деталь? Какова
дисперсия числа попыток?}
\solution{\begin{tabular}{|c|c|c|c|} \hline
  % after \par: \hline or \cline{col1-col2} \cline{col3-col4} ...
  $X$ & 1 & 2 & 3 \\  \hline
 $\PP$ & $\frac{12}{20}$ & $\frac{6}{20}$ & $\frac{2}{20}$ \bigstrut \\
  \hline
\end{tabular} \par
$\E(X)=\frac{3}{2}$, $\Var(X)=\frac{9}{20}$.  }


\problem{
Бросают два правильных игральных кубика. Пусть  $X$  "--- наименьшая
из выпавших граней, а  $Y$  "--- наибольшая.
\begin{enumerate}
\item  Рассчитайте  $\PP(X=3\cap Y=5)$;
\item  Найдите  $\E(X)$,  $\Var(X)$, $\E(3X-2Y)$.
\end{enumerate}
\begin{ist}
Cut the knot.
\end{ist}
 }
\solution{ }

\problem{
Из колоды в 52 карты извлекаются две. Пусть  $X$  "--- количество
тузов. Найдите закон распределения  $X$, $\E(X)$, $\Var(X)$.}
\solution{ }

\problem{  \label{Iska priglasil 3 druzei}
Иська пригласил трёх друзей навестить его. Каждый из них появится
независимо от другого с вероятностью $0{,}9$, $0{,}7$ и $0{,}5$
соответственно. Пусть $N$ "--- количество пришедших гостей.
\begin{enumerate}
\item Рассчитайте вероятности $\PP(N=0)$, $\PP(N=1)$, $\PP(N=2)$ и $\PP(N=3)$;
\item Найдите $\E(N)$ и $\Var(N)$.
\end{enumerate}
 }
\solution{$\PP(N=0)=0{,}1\cdot 0{,}3\cdot 0{,}5$; $\PP(N=3)=0{,}9\cdot 0{,}7\cdot 0{,}5$;
$\PP(N=1)=0{,}9\cdot 0{,}3\cdot 0{,}5+0{,}1\cdot 0{,}7\cdot 0{,}5+0{,}1\cdot0{,}3\cdot 0{,}5$. $\E(N)=0{,}9+0{,}7+0{,}5$; $\Var(N)=0{,}9\cdot 0{,}1+0{,}7\cdot 0{,}3+0{,}5\cdot
0{,}5$.  }


\problem{
В коробке лежат три монеты: их достоинство "--- 1, 5 и 10 копеек соответственно. Они извлекаются в случайном порядке. Пусть $X_{1}$,  $X_{2}$  и  $X_{3}$  "--- достоинства монет в порядке их
появления из коробки.
\begin{enumerate}
\item Верно ли, что  $X_{1}$  и  $X_{3}$ одинаково распределены?
\item  Верно ли, что  $X_{1}$  и  $X_{3}$ независимы?
\item  Найдите $\E(X_{2})$
\item  Найдите дисперсию $\bar{X}_{2} =\frac{X_{1} +X_{2} }{2} $.
\end{enumerate}
 }
\solution{ }

\problem{  \zdt{Easy}

Пусть $X$ "--- сумма очков, выпавших в результате двукратного подбрасывания кубика. Найдите $\E(X)$, $\Var(X)$. }
\solution{ }

\problem{
Охотник, имеющий 4 патрона, стреляет по дичи до первого
попадания или до израсходования всех патронов. Вероятность
попадания при первом выстреле равна 0{,}6, а при каждом последующем
уменьшается на 0{,}1. Найдите
\begin{enumerate}
\item  Закон распределения числа патронов, израсходованных охотником;
\item  Математическое ожидание и дисперсию этой случайной величины.
\end{enumerate}
 }
\solution{ \small

\begin{tabular}{|c|c|c|c|c|}  \hline
  % after \\: \hline or \cline{col1-col2} \cline{col3-col4} ...
  $x_{i}$ & 1 & 2 & 3 & 4 \\  \hline
  $p_i=\PP(X=x_{i})$ & $0{,}6$& $(1-0{,}6)\cdot 0{,}5 = 0{,}2$ & $(1-0{,}6)\cdot(1-0{,}5)\cdot 0{,}4 = 0{,}08$ & $1-p_{1}-p_{2}-p_{3} = 0{,}12$ \\ \hline
\end{tabular}

\normalsize $\E(X)=1{,}7$, $\Var(X)\approx 1{,}08$. }

 
\problem{У Васи есть две неправильные монетки, выпадающие орлом с вероятностями $p_1$ и $p_2$.
\begin{enumerate}
  \item Вася $n$ раз наугад берет одну из монеток и подбрасывает ее. Найдите матожидание и дисперсию числа выпавших орлов.
  \item Вася наугад берет одну из монеток и подбрасывает ее $n$ раз. Найдите матожидание и дисперсию числа выпавших орлов.
  \item Сравните матожидания и дисперсии, полученные выше. Поясните результат интуитивно.
\end{enumerate}

Источник: Алексей Суздальцев}


\solution{В первом случае мы имеем просто <<составную>> монетку. Матожидание и дисперсия равны $n\frac{p_1+p_2}{2}$ и $n\frac{p_1+p_2}{2}\left(1-\frac{p_1+p_2}{2}\right)$ соответственно.\\
Во втором случае все немного сложнее. Матожидание будет тем же, а дисперсия будет равна $\frac{n(p_1(1-p_1)+p_2(1-p_2))}{2}+\frac{n^2(p_1-p_2)^2}{4}$, что, как нетрудно проверить, всегда не меньше дисперсии в первом случае (равенство достигается, естественно, только при $p_1=p_2$).\\
Во втором эксперименте по сравнению с первым больше вероятности <<крайних>> значений числа выпавших орлов, но меньше вероятности <<срединных>> значений. Это и является интуитивным обоснованием того, что матожидания в двух случаях одинаковые, а дисперсия во втором случае больше.}






\subsection{Непрерывный случай}
%Здесь появляется:
%Е когда задана функция плотности

\problem{  \ENGs
A large quantity of pebbles lies scattered uniformly over a
circular field; compare the labour of collecting them on by one
\begin{enumerate}
\item At the center $O$ of the field;
\item At a point $A$ on the circumference.
\end{enumerate}\RUSs
\begin{ist}
Grimmett, экзамен 1858 года в St John's College, Кембридж.
\end{ist}
 }
\solution{ }

\subsection{? Способ расчёта ожидания и дисперсии через условную}

\problem{
Число $x$ выбирается равномерно на отрезке $[0; 1]$. Затем случайно выбираются числа из отрезка $[0; 1]$ до тех пор, пока не появится число больше $x$.
\begin{enumerate}
\item  Сколько в среднем потребуется попыток?
\item  Сколько в среднем потребуется попыток, если $x$ выбирается равномерно на отрезке $[0;r]$, $r<1$?
\item  Сколько в среднем потребуется попыток, $x$ не выбирается равномерно, а имеет функцию плотности $p(t)=2(1-t)$ для $t\in[0;1]$?
\end{enumerate}
 }
\solution{ }



\problem{
Петя сообщает Васе значение величины $X\sim U[0;1]$. Вася изготавливает неправильную монетку, которая выпадает <<орлом>> с вероятностью  $X$ и подкидывает ее 20 раз. 
\begin{enumerate}
\item  Какова вероятность, что выпадет ровно 5 орлов? 
\item Каково среднее количество выпавших орлов? Дисперсия? 
\end{enumerate} }
\solution{ }



