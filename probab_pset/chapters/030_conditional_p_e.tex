% !Mode:: "TeX:UTF-8"
\section{Условные вероятности и ожидания. Дополнительная информация}
%Правило умножения вероятностей:
%Если A B независимы, то

\subsection{Условная вероятность}

\problem{ \ENGs
A bag contains a counter, known to be either white or black. A white counter is put in, the bag is shaken, and a counter is drawn out, which proves to be white. What is now the chance of drawing a white counter? \RUSs}
\solution{ }

\problem{ \ENGs
You have a hat in which there are three pancakes: one is golden on both sides, one is brown on both sides, and one is golden on one side and brown on the other. You withdraw one pancake, look at one side, and see that it is brown. What is the probability that the other side is brown? \RUSs}
\solution{ }

\problem{ \ENGs
The inhabitants of an island tell truth one third of the time. They lie with the probability of $\frac{2}{3}$. On an occasion, after one of them made a statement, another fellow stepped forward and declared the statement true. What is the probability that it was indeed true? \RUSs }
\solution{ }


\problem{
На кубиках написаны числа от 1 до 100. Кубики свалены в кучу. Вася выбирает наугад из кучи по очереди три кубика.
\begin{enumerate}
\item Какова вероятность, что полученные три числа будут идти в возрастающем порядке?
\item Какова вероятность, что полученные три числа будут идти в возрастающем порядке, если известно, что первое меньше последнего?
\end{enumerate}
 }
\solution{ $\frac{1}{6}$; $\frac{1}{3}$. }


\problem{ (дописать)

Наследование группы крови контролируется аутосомным геном. Три его аллеля обозначаются буквами А, В и 0. Аллели А и В доминантны в одинаковой степени, а аллель 0 рецессивен по отношению к ним обоим. Поэтому существует четыре группы крови. Им соответствуют следующие генотипы:
\begin{itemize}
\item Первая (I) "--- 00;
\item Вторая (II) "--- АА, А0;
\item Третья (III) "--- ВВ, В0;
\item Четвёртая (IV) "--- АВ.
\end{itemize}

Наследование резус-фактора кодируется тремя парами генов и происходит независимо от наследования группы крови. Наиболее значимый ген имеет два аллеля, аллель D доминантный, аллель d рецессивный. Таким образом, получаем следующие генотипы:
\begin{itemize}
\item Резус-положительный "--- DD, Dd;
\item Резус-отрицательный "--- dd.
\end{itemize}

Если у беременной женщины резус"=отрицательная кровь, а у плода резус"=положительная, то есть риск возникновения гемолитической болезни (у матери образуются антитела к резус фактору, безвредные для неё, но вызывающие разрушение эритроцитов плода).

Перед нами два семейства: Монтекки и Капулетти. \\
...}
\solution{ }


\problem{ \label{rekordnaia volna}
Пусть $X_{i}$ "--- НОРСВ, такие, что $\PP(X_{i}=X_{j})=0$. Обозначим за
$E_{k}$ событие, состоящее в том, что $X_{k}$ оказалась
<<рекордом>>, т.\,е. больше, чем все предыдущие $X_{i}$ ($i<k$). Для
определённости будем считать, что $E_{1}=\Omega$.
\begin{enumerate}
\item Найдите $\PP(E_{k})$.
\item  Верно ли, что $E_{k}$ независимы?
\item  Какова вероятность того, что второй рекорд будет зафиксирован в $n$-й момент времени?
\item  Сколько в среднем времени пройдёт до второго рекорда?
\end{enumerate}

\begin{ist}
Williams, 4.3.
\end{ist}
 }
\solution{ Какая-то из первых $k$ величин будет наибольшей. В силу \iid{}
получаем, что $\PP(E_{k})=\frac{1}{k}$. $E_k$ независимы: например, если известно,
что 10-е наблюдение было рекордом, это ничего не говорит о рекордах в первых 9-ти
наблюдениях. Вероятность второго рекорда в $n$-й момент равна $\frac{1}{n(n-1)}$,
а в среднем времени до второго рекорда пройдёт $\infty$. }

\problem{Известно, что $\PP(A \mid B)=\PP(A \mid B^{c})$. Верно ли, что $A$ и $B$ независимы?}
\solution{Да.}


\problem{ \zdt{Randomized response technique}

В анкету для чиновников включён скользкий вопрос: <<Берёте ли Вы
взятки?>>. Чтобы стимулировать чиновников отвечать правдиво,
используется следующий прием. Перед ответом на вопрос чиновник втайне от анкетирующего подкидывает специальную монетку, на гранях
которой написано <<правда>>, <<ложь>>. Если монетка выпадает
<<правдой>>, то предлагается отвечать на вопрос правдиво, если
монетка выпадает на <<ложь>>, то предлагается солгать. Таким
образом, ответ <<да>> не обязательно означает, что чиновник берёт
взятки.

Допустим, что треть чиновников берёт взятки, а монетка
неправильная и выпадает <<правдой>> с вероятностью 0{,}2.
\begin{enumerate}
\item Какова вероятность того, что чиновник ответит <<да>>?
\item  Какова вероятность того, что чиновник берёт взятки, если он
ответил <<да>>? Если ответил <<нет>>?
\end{enumerate}
\todo[inline]{Вставить построение несмещённой оценки?}
}
\solution{ }

\problem{
Пусть события  $A$  и  $B$  независимы и $\PP(B)>0$.
Чему равна  $\PP(A \mid B)$? }
\solution{ $ \PP(A \mid B)=\PP(A)$. }

\problem{
Из колоды в 52 карты извлекается одна карта наугад. Верно ли, что
события <<извлечён туз>> и <<извлечена пика>> являются
независимыми? }
\solution{ Да. }

\problem{
Из колоды в 52 карты извлекаются по очереди две карты наугад.
Верно ли, что события <<первая карта "--- туз>> и <<вторая карта "---
туз>> являются независимыми? }
\solution{ Нет. }

\problem{
Известно, что $\PP(A)=0{,}3$, $\PP(B)=0,{4}$, $\PP(C)=0{,}5$. События
$A$ и $B$ несовместны, события $A$ и $C$ независимы и
$\PP(B\mid C)=0{,}1$.
Найдите $\PP(A\cup B\cup C)$. }
\solution{ }

\problem{
Имеется три монетки. Две <<правильных>> и одна "--- с орлами по
обеим сторонам. Петя выбирает одну монетку наугад и подкидывает её
два раза. Оба раза выпадает орёл. Какова вероятность того, что
монетка <<неправильная>>? }
\solution{ }

\problem{
Самолёт упал либо в горах, либо на равнине. Вероятность того, что самолёт упал в горах, равна 0{,}75. Для поиска пропавшего самолёта выделено 10 вертолётов. Каждый вертолёт можно использовать только в одном месте. Как распределить имеющиеся вертолёты, если вероятность обнаружения пропавшего самолёта отдельно взятым вертолётом равна: $0{,}95$? $0,6$ (пасмурно)? $0{,}1$ (туман)? }
\begin{ist}
Айвазян, экзамен РЭШ.
\end{ist}
\solution{ }

\problem{
Социологическим опросам доверяют 70\,\% жителей. Те, кто доверяет опросам, всегда отвечают искренне; те, кто не доверяет, отвечают наугад, равновероятно выбирая <<да>> или <<нет>>. Социолог Петя  в анкету очередного опроса включил вопрос: <<Доверяете ли Вы социологическим опросам?>>
\begin{enumerate}
\item Какова вероятность, что случайно выбранный респондент ответит <<Да>>?
\item  Какова вероятность того, что он действительно доверяет, если известно, что он ответил <<Да>>?
\end{enumerate}
 }
\solution{ }

\problem{
Два охотника выстрелили в одну утку. Первый попадает с
вероятностью $0{,}4$, второй "--- с вероятностью $0{,}6$. Какова вероятность того, что утка была убита первым
охотником, если в утку попала ровно
одна пуля? }
\solution{
$p=\frac{0{,}4\cdot 0{,}4}{0{,}4\cdot 0{,}4+0{,}6\cdot 0{,}6}=\frac{4}{13}$.}

\problem{
С вероятностью $0{,}3$ Вася оставил конспект в одной из 10
посещённых им сегодня аудиторий. Вася осмотрел 7 из 10 аудиторий и
конспекта в них не нашёл.
\begin{enumerate}
\item  Какова вероятность того, что конспект будет найден в следующей
осматриваемой им аудитории?
\item  Какова (условная) вероятность того, что конспект оставлен
где-то в другом месте?
\end{enumerate}
 }
\solution{ }

\problem{
Вася гоняет на мотоцикле по единичной окружности с центром в
начале координат. В случайный момент времени он останавливается.
Пусть случайные величины  $X$  и  $Y$  "--- это Васины абсцисса и
ордината в момент остановки. Найдите  $\PP\left(X>\frac{1}{2} \right)$,
$\PP\left(X>\frac{1}{2} \bigm| Y<\frac{1}{2} \right)$. Являются ли события
$A=\left\{X>\frac{1}{2} \right\}$  и
$B=\left\{Y<\frac{1}{2} \right\}$  независимыми?
\begin{hint}
$\cos\left(\frac{\pi }{3} \right)=\frac{1}{2}$, длина окружности $l=2\pi r$.
\end{hint}
}
\solution{ }

\problem{
Пусть  $\PP(A)=1/4$,  $\PP(A \mid B)=\frac{1}{2}$  и $\PP(B\mid A)=\frac{1}{3}$. Найдите $\PP(A\cap
B)$, $\PP(B)$  и  $\PP(A\cup B)$.}
\solution{ }

\problem{
Примерно\footnote{Цифры условные. Celui qui ne mange pas de
bifsteak au cause de la vache folle --- il est fou! Jolivaldt.} 4\,\%
коров заражены <<коровьим бешенством>>. Имеется тест, позволяющий
с определённой степенью достоверности установить, заражено ли мясо
прионом или нет. С вероятностью $0{,}9$ заражённое мясо будет признано
заражённым. <<Чистое>> мясо будет признано заражённым с
вероятностью 0{,}1. Судя по тесту, эта партия мяса заражена. Какова
вероятность того, что она действительно заражена?}
\solution{ }

\problem{
\emph{Роме Протасевичу, искавшему со мной у Мутновского
вулкана в
тумане серую палатку...}

Есть две тёмные комнаты, $A$ и $B$. В одной из них сидит чёрная кошка.
Первоначально предполагается, что вероятность нахождения кошки в
комнате $A$ равна $\alpha$. Вероятность найти чёрную кошку в темной
комнате (если она там есть) с одной попытки равна $p$.  Допустим,
что вы сделали $a$ неудачных попыток поиска кошки в комнате $A$ и
$b$ неудачных попыток в комнате $B$.
\begin{enumerate}
\item Чему равна условная вероятность нахождения кошки в комнате $A$?
\item  При каком условии на $(a-b)$ эта вероятность будет больше
$0{,}5$?
\end{enumerate}
 }
\solution{ }

\problem{
Кубик подбрасывается два раза. Найдите вероятность
получить сумму, равную 8, если на первом кубике выпало 3.}
\solution{ $\frac{1}{6}$. }

\problem{
В коробке 10 пронумерованных монеток, $i$-я монетка выпадает
орлом с вероятностью $\frac{i}{10}$. Из коробки была вытащена одна
монетка наугад. Она выпала орлом. Какова вероятность того, что это
была пятая монетка? }
\solution{
$\frac{1}{11}$.  }

\problem{ Вы играете две партии в шахматы против незнакомца. Равновероятно
незнакомец может оказаться новичком, любителем или профессионалом.
Вероятности вашего выигрыша в отдельной партии, соответственно,
будут равны 0{,}9; 0{,}5; 0{,}3.
\begin{enumerate}
\item Какова вероятность выиграть первую партию?
\item Какова вероятность выиграть вторую партию, если вы выиграли
первую?
\end{enumerate}
 }

\solution{ $p_{a}=\frac{1}{3}(0{,}9+0{,}5+0{,}3)=\frac{17}{30}$, $p_{b}=\frac{1}{3}(0{,}9^{2}+0{,}5^{2}+0{,}3^{2})/p_{a}=\frac{115}{170}$. }

\problem{
В каких из перечисленных случаев вероятность наличия флэша (см. \hyperref[combo]{карточные комбинации} на стр.~\pageref{combo}) больше, чем при полном отсутствии информации:
\begin{enumerate}
\item Первая карта из имеющихся "--- это туз;
\item Первая карта из имеющихся "--- это туз бубей;
\item На руках имеется хотя бы один туз;
\item На руках имеется туз бубей.
\end{enumerate}
 }

\solution{ \ENGs Unverified, but no calculation:

An arbitrary hand can have two aces but a flush hand can't.  The
average number of aces that appear in flush hands is the same as the
average number of aces in arbitrary hands, but the aces are spread out
more evenly for the flush hands, so set (3) contains a higher fraction
of flushes.

Aces of spades, on the other hand, are spread out the same way over
possible hands as they are over flush hands, since there is only one of
them in the deck.  Whether or not a hand is flush is based solely on a
comparison between different cards in the hand, so looking at just one
card is necessarily uninformative.  So the other sets contain the same
fraction of flushes as the set of all possible hands. \RUSs }


\problem{\ENGs A man has 3 equally favorite seats to fish at. The probability with which the man can succeed at catching at each seat is 0.6, 0.7, 0.8 respectively. It is known that the man dropped the hint at one seat three times and just caught one fish. Find the probability that the fish was caught at the first seat. \RUSs}
\solution{$\approx 0{,}503$. }


\problem{
Есть 101 мешок конфет. В каждом мешке 100 шоколадных конфет, неотличимых с виду. В $i$-ом мешке $i-1$ конфета с орехом, остальные без ореха. \\
Мы выбираем мешок наугад и съедаем из него две конфеты. 
\begin{enumerate}
\item Какова вероятность, что первая будет с орехом? 
\item  Какова вероятность, что вторая с орехом, если первая с орехом? 
\end{enumerate}
}
\solution{
а) 1/2 б) 2/3 (есть ли красивое решение? можно пробиться через сумму квадратов) }


\problem{ В классе было 2 мальчика и сколько-то девочек. Заходит еще кто-то. Ребята решили, что народу стало слишком много, выбрали одного человека жеребьевкой и выгнали. Какова вероятность того, что вошел мальчик, если выгнали мальчика?}
\solution{$3/5$ при любом количестве девочек}

\problem{ Вася подкидывает монетку четыре раза. Если монетка выпадает орлом, то он кладет в мешок черный шар, если решкой --- белый. Петя не знает, как выпадала монетка, и достает два шара из мешка наугад. Первый шар черного цвета. Какова вероятность того, что второй будет белым? }
\solution{ }


\problem{ Внутри квадрата с вершинами $(0;0)$, $(0;1)$, $(1;0)$ и $(1;1)$ равновероятно выбирается одна точка. Пусть  $X$  и  $Y$  --- абсцисса и ордината этой точки. Найдите $\P(X<0,75)$,  $P(X\le a)$  для произвольного $a$, $\P(X>0,5|X+Y>0,5)$, $\P(X+Y>0,5|X>0,5)$, $\P(X\cdot Y>1/3|X+Y<2/3)$,  $\P(X>0,3|Y<0,7)$}
\solution{ }





\subsection{Условное среднее}


\problem{ \label{dve shkatulki} \zdt{Две шкатулки}

Васе предлагают две шкатулки и обещают, что в одной из них денег
в два раз больше, чем в другой. Вася открывает наугад одну из них
"--- в ней $a$ рублей. Вася может взять либо деньги, либо
оставшуюся шкатулку.
\begin{enumerate}
\item Правильно ли Вася считает, что ожидаемое количество денег в
неоткрытой шкатулке равно $\frac{1}{2}\left( {\frac{1} {2}a}
\right)+\frac{1}{2}( {2a} ) = 1\frac{1} {4}a$ и что
поэтому нужно изменить свой выбор?
\item Пусть известно, что в пару шкатулок кладут $3^k$ и $3^{k+1}$
рублей с вероятностью $p_k  = ( {\frac{1} {2}} )^k $. Стоит ли
Васе изменить свой выбор после того, как он открыл
первую шкатулку? Почему?
\end{enumerate}
 }
\solution{Вася считает неправильно: условное распределение суммы можно определить, только зная безусловное.

Концепция условного ожидания неприменима? Вставить это в иллюстрацию условного ожидания? При заданном безусловном распределении Васе следует сменить
свой выбор вне зависимости от того, что он увидел в первой шкатулке. Вторая открытая лучше первой открытой. Это возможно
из-за того, что безусловная ожидаемая сумма равна бесконечности
для обеих шкатулок.  }


\problem{В кабинет бюрократа скопилась очередь ещё до его открытия. Пусть время обслуживания страждущих "--- независимые экспоненциальные случайные величины. Посетитель, пришедший через $t$ минут после открытия, узнал, что первый посетитель уже ушёл, а второй ещё сидит в кабинете. Найдите ожидаемое время обслуживания первого посетителя, $\E(X_{1} \mid X_{1}\le t < X_{1}+X_{2}) $.}
\solution{$\frac{t}{2}$.}
\cat{poisson} \cat{exp} % может перекинуть в Пуассоновский процесс?


\problem{
Пете и Васе предложили одну и ту же задачу. Они могут правильно решить её с вероятностями 0{,}7 и 0{,}8 соответственно. К задаче предлагается 5 ответов на выбор, поэтому будем считать, что выбор каждого из пяти ответов равновероятен, если задача решена неправильно.
\begin{enumerate}
\item Какова вероятность несовпадения ответов Пети и Васи?
\item  Какова вероятность того, что Петя ошибся, если ответы совпали?
\item  Каково ожидаемое количество правильных решений, если ответы совпали?
\end{enumerate}
 }
\solution{ }

\problem{Автобусы ходят регулярно с интервалом в 10~минут. Вася приходит на остановку в случайный момент времени и ждёт автобуса не больше $a$ минут. Величина $a$ "--- константа из интервала $(0;10)$. Если автобус приходит меньше чем за $a$ минут, то Вася уезжает на нём. Если автобуса нет в течение $a$ минут, то Вася заходит в ближайшую кафешку перекусить и через случайное время возвращается на остановку. На второй раз он ждёт до прихода автобуса.
\begin{enumerate}
\item Какое время Вася в среднем проводит в ожидании автобуса?
\item  Постройте график получившейся функции от $a$.
\end{enumerate}
}
\solution{$f(a)=5+\frac{a(10-a)}{20}$.}

\problem{
Игрок получает 13 карт из колоды в 52 карты. \\
\begin{enumerate}
\item Какова вероятность того, что у него как минимум два туза, если
известно, что у него есть хотя бы один туз? 
\item Какова вероятность того, что у него как минимум два туза, если
известно, что у него есть туз пик? 
\item Объясните, почему эти две вероятности отличаются. 
\end{enumerate}
}
\solution{ }

\problem{ В уездном городе  $N$  проживают  $10^{7}$  человек. Каждый из них
может обладать редким даром ясновидения с вероятностью
$p=10^{-7}$ независимо от других. 
\begin{enumerate}
\item Каково ожидаемое количество ясновидящих? 
\item Известно, что Петя --- ясновидящий. Какова вероятность
найти еще одного ясновидящего в городе $N$?
\end{enumerate}
}
\solution{ 1, почти 1. }

\problem{  
Цвет глаз кодируется несколькими генами. В целом более темный цвет доминирует более светлый. Ген карих глаз доминирует ген синих. Т.е. у носителя пары bb глаза
синие, а у носителя пар BB и Bb --- карие. У диплоидных организмов
(а мы такие :)) одна аллель наследуется от папы, а одна --- от мамы.
В семье у кареглазых родителей два сына --- кареглазый и синеглазый.
Кареглазый женился на синеглазой девушке. Какова вероятность
рождения у них синеглазого ребенка?}
\solution{ }

\problem{
У тети Маши --- двое детей, один старше другого. Предположим, что вероятности рождения мальчика и девочки равны и не зависят от дня недели, а пол первого и второго ребенка независимы. 
\begin{enumerate}
\item Известно, что хотя бы один ребенок --- мальчик. Какова
вероятность того, что другой ребенок --- девочка?
\item Тетя Маша наугад выбирает одного своего
ребенка и посылает к тете Оле, вернуть учебник по теории
вероятностей. Это оказывается мальчик. Какова вероятность того,
что другой ребенок --- девочка? 
\item Известно, что старший ребенок --- мальчик. Какова вероятность того, что другой ребенок --- девочка? 
\item На вопрос: <<А правда ли тетя Маша, что у вас есть сын, родившийся в пятницу?>>. Она ответила: <<Да>>. Какова вероятность того, что другой ребенок --- девочка?
\end{enumerate}
}
\solution{$ 2/3 $, $1/2$, $ 1/2 $, $ 14/27 $ }

\problem{ В урне 5 белых и 11 черных шаров. Два шара извлекаются по
очереди. Какова вероятность того, что второй шар будет черным?
Какова вероятность того, что первый шар --- белый, если известно,
что второй шар --- черный?}
\solution{ }

\problem{ Monty-hall \\
Вы играете в <<Поле Чудес>> и Вам предлагают <<3 шкатулки>>.
Назовем их a, b и c. В одной из трех шкатулок лежит 1000 рублей.
(Введем соответственно события A, B и C, где A означает <<деньги
лежат в
шкатулке a>>). Вы выбираете наугад одну из трех шкатулок. \\
Ведущий, который знает, где лежат деньги, убирает одну пустую
шкатулку, не выбранную Вами (среди двух не выбранных Вами
обязательно есть пустая, если таковых две, то ведущий убирает
любую наугад). Допустим, Вы выбрали шкатулку b, а ведущий после
этого убрал шкатулку c. \\
Найдите условную вероятность того, что приз лежит в выбранной Вами
шкатулке. Имеет ли Вам смысл изменить Ваш выбор?

\emph{Альтернативный вариант условия-1} \\
После того, как Вы выбрали шкатулку, ведущий открывает наугад одну
из пустых шкатулок (при этом он может открыть Вашу и разочаровать
Вас). Допустим, Вы выбрали шкатулку b, а ведущий после этого
открыл шкатулку c. Найдите условную вероятность того, что приз
лежит в выбранной Вами шкатулке. Имеет ли Вам смысл изменить Ваш
выбор? \\
\emph{Альтернативный вариант условия-2} \\
После того, как Вы выбрали шкатулку, ведущий открывает наугад одну
из оставшихся шкатулок (при этом он может оказаться открытой
шкатулка с деньгами). Допустим, Вы выбрали шкатулку b, а ведущий
после этого открыл шкатулку c и она оказалась пустой. Найдите
условную вероятность того, что приз лежит в выбранной Вами
шкатулке. Имеет ли Вам смысл изменить Ваш выбор? }

\solution{solution 1: \\
Задача эквивалентна следующей: игрок выбирает шкатулку. Затем (она не открывается) игрок выбирает оставить ее или взять обе другие. Очевидно, во втором случае шансы в два раза выше. \\
Solution 2: \\
Игрок не получает информации --- вероятность не меняется. Лучше сменить выбор.  }

\problem{ multi-stage monty hall \\
Suppose there are four doors, one of which is a winner. The host says:
<<You point to one of the doors, and then I will open one of the other non-winners. Then you decide whether to stick with your original pick or switch to one of the remaining doors. Then I will open another (other than the current pick) non-winner. You will then make your final decision by sticking with the door picked on the previous decision or by switching to the only other remaining door>>
Optimal strategy? \\
source: cut-the-knot -- Bhaskara Rao }
\solution{ stick-switch }

\problem{ В школе три девятых класса, <<А>>, <<Б>> и <<В>>, одинаковые по
численности. В <<А>> классе 30\% обожают учителя географии, в
<<Б>> классе --- 40\% и в <<В>> классе --- 70\%. Девятиклассник Петя
обожает учителя географии. Какова вероятность того, что он из
<<Б>>
класса?}
\solution{ }

\problem{ В урне 7 красных, 5 желтых и 11 белых шаров. Два шара
выбирают наугад. Какова вероятность, что это красный и белый, если
известно, что они разного цвета.}
\solution{ }

\problem{ Саша едет на день рождения к Маше и ищет её дом. Её дом находится
южнее по улице. Одна треть встречных прохожих --- местные. Местные всегда
лгут, неместные говорят правду с вероятностью $\frac{3}{4}$.
Изначально Саша оценивает вероятность того, что дом находится
южнее, как $a$. Саша спросил первого встречного прохожего и
получил ответ <<севернее>>. Как Саша изменит свою
субъективную вероятность? }
\solution{ }

\problem{ Самолет упал в горах, в степи или в море. Вероятности,
соответственно, равны $0,5$, $0,3$ и $0,2$. Если он упал в горах,
то при поиске его найдут с вероятностью $0,7$. В степи --- $0,8$, на
море --- $0,2$. Самолет искали в горах, в степи и не нашли. Какова
вероятность того, что он упал в море? }
\solution{ }

\problem{ Русская рулетка. \\
Давайте сыграем в русскую рулетку\ldots Вы привязаны к стулу и не
можете встать. Вот пистолет. Вот его барабан --- в нем шесть гнезд
для патронов, и они все пусты. Смотрите: у меня два патрона. Вы
обратили внимание, что я их вставил в соседние гнезда барабана?
Теперь я ставлю барабан на место и вращаю его. Я подношу пистолет
к вашему виску и нажимаю на спусковой крючок. Щелк! Вы еще живы.
Вам повезло! Сейчас я собираюсь еще раз нажать на крючок. Что вы
предпочитаете: чтобы я снова провернул барабан или чтобы просто
нажал на спусковой крючок? \\
\url{http://forum.eldaniz.ru/index.php?topic=293.60} }
\solution{ }

\problem{ Четыре свидетеля, A, B, C и D, говорят правду независимо
друг от друга с вероятностью $\frac{1}{3}$. A утверждает, что B
отрицает, что C заявил, что D солгал. Какова условная
вероятность того, что D сказал правду? }
\solution{ }

\problem{
Подробности о пожаре (Ах, а правда ли, что тетя Соня забыла
выключить утюг?) передаются по цепочке из четырех человек
(А-B-C-D), каждый из которых говорит следующему имеющуюся у него
информацию с вероятностью $p$, а с вероятностью $1-p$ говорит
совершенно
противоположное. D говорит, что тетя Соня утюг выключила. \\
Как зависит от $p$ условная вероятность того, что тетя Соня
действительно выключила утюг? }
\solution{ }

\problem{
Есть четыре населенных пункта $A$, $B$, $C$ и $D$. Прямая
дорога между каждыми двумя существует с вероятностью $p$. 
\begin{enumerate}
\item Какова вероятность того, что можно добраться из $A$ в $D$?
\item Какова вероятность того, что можно добраться из $A$ в $D$, если
между $B$ и $C$ нет прямой дороги? 
\end{enumerate}
}
\solution{ }



\problem{ В урне лежат 5 пронумерованных от одного до пяти шаров. По
очереди вытаскиваются два шара. Какова вероятность того, что
разница в номерах будет больше двух? Какова вероятность того, что
первым был вытащен шар с номером 2, если разница в номерах была
больше двух?}
\solution{ }

\problem{ A regular $n$-polygon has vertices numbered 0, 1, 2,\ldots, $n-1$ in clockwise. Let the vertex  0 be a starting point. When you roll a dice, you will move the coin clockwise by the number on the dice. Denote the number of the arriving vertice by $X$. Again roll a dice, you will move from the vertex $X$ to the vertex $Y$. 
\begin{enumerate}
\item Are  $X$ and $Y$ independent?
\item Find the value of $n$ such that $X$ and $Y$ are independent  
\end{enumerate}
Source: Kyoto University entrance exam/Science , Problem 6, 1st Round, 1990 }
\solution{ }

\problem{
Будем говорить, что событие $A$ благоприятствует событию $B$, если $\P(B\mid A)>\P(B)$. \\
Известно, что $A$ благоприятствует $B$, $B$ благоприятствует $C$. \\
Верно ли, что $A$ благоприятствует $C$? }
\solution{ не обязательно }

\problem{ Два неравенства
\begin{enumerate}
\item Известно, что $\P(A\mid B)>\P(A)$. Верно ли, что $\P(B\mid A)>\P(B)$?  
\item Известно, что $\P(A\mid B)>\P(B)$. Верно ли, что $\P(B\mid A)>\P(A)$?   
\end{enumerate}
}
\solution{ а) да; б) нет }

\problem{ На Древе познания Добра и Зла растет 6 плодов познания Добра и 5 плодов познания Зла. Адам и Ева съели по 2 плода. Какова вероятность того, что Ева познала Зло, если Адам познал Добро? }
\solution{ }

\problem{ A sniper has 0.8 chance to hit the target if he hit his last shot and 0.7 chance to hit the target if he missed his last shot. It is known he missed on the 1st shot and hit on the 3rd shot. \\
What is the probability he hit the second shot?}
\solution{ $8/11$ }

\problem{
Снайпер попадает в <<яблочко>> с вероятностью 0.8, если в предыдущий раз он попал в <<яблочко>>; и с вероятностью 0.7, если в предыдущий раз он не попал в <<яблочко>> или если это был первый выстрел. Снайпер стрелял по мишени 3 раза.
\begin{enumerate}
\item Какова вероятность попадания в <<яблочко>> при втором выстреле? \\
\item Какова вероятность попадания в <<яблочко>> при втором выстреле, если при первом снайпер попал, а при третьем --- промазал?
\end{enumerate}
}
\solution{
a) $p=0.7\cdot 0.8+ 0.3\cdot 0.7=0.77$ \\
b) $p=\frac{0.7\cdot0.8\cdot0.2}{0.7\cdot 0.8\cdot 0.2 + 0.7\cdot 0.2 \cdot 0.3}=\frac{8}{11}$ }

\problem{
Есть две неправильные монетки. Первая выпадает орлом с вероятностью 0.1, вторая выпадает орлом с вероятностью 0.9. Из этих двух монеток равновероятно выбирают одну и подбрасывают ее 2 раза. 
\begin{enumerate}
\item Верно ли, что результат первого и второго подбрасывания независимы? \\
\item Известно, что выбрали первую монетку. Верно ли, что результат первого и второго подбрасывания независимы? 
\end{enumerate}
}
\solution{нет, да}

\problem{
Вы равновероятно могли получить письмо из Москвы или из Игарки. Все буквы в названии города в обратном адресе кроме одной нечитаемы из-за загрязнения на конверте. Единственная различимая буква --- это буква <<а>>. Какова условная вероятность того, что письмо пришло из Москвы? }
\solution{
Из названия города случайным образом оставляем одну букву. \\ $p=\frac{0.5\frac{1}{6}}{0.5\frac{1}{6}+0.5\frac{2}{6}}=1/3$}

\problem{
Вася кидает дротик в мишень три раза. Его броски независимы друг от друга. Известно, что во второй раз он попал дальше от центра, чем в первый раз. Какова условная вероятность того, что в третий раз он попадет ближе к центру, чем в первый раз? }
\solution{
$\frac{1}{3}$ }

\problem{
В одном мешке лежат только спелые яблоки, в другом --- одинаковое количество спелых и зеленых. Вы случайным образом вытаскиваете яблоко из мешка, оно --- спелое, вы кладете его обратно. Какова вероятность, что следующее яблоко из того же мешка будет зеленым? \\
Какова вероятность, что следующее яблоко из того же мешка будет зеленым, если было $n$ попыток достать яблоко, и каждый раз вытаскивалось и клалось обратно спелое яблоко? }
\solution{
 $\frac{1}{2+2^{n}}$ }

\problem{
Three dice are rolled.  If no two show the same face, what is the probability
that one is an ``ace'' (one spot showing.)? }
\solution{ }

\problem{ Given that a throw with ten dice produced at least one ace, what is
the probability of two or more aces? }
\solution{ }

\problem{
Определение. События $A_{1}$ и $A_{2}$ называются \emph{условно
независимыми} относительно события $B$, если $\P(A_{1}\cap
A_{2}\mid B)=\P(A_{1}\mid B)\cdot \P(A_{2}\mid B)$. 
\begin{enumerate}
\item Приведите пример таких $A_{1}$, $A_{2}$ и $B$, что $A_{1}$ и
$A_{2}$, независимы, но не являются условно независимыми
относительно $B$. \\
\item Приведите пример таких $A_{1}$, $A_{2}$ и $B$, что $A_{1}$ и
$A_{2}$, зависимы, но являются условно независимыми
относительно $B$. 
\end{enumerate}
}
\solution{ }

\problem{
В урне 99 белых и один черный шар. Один шар извлекается из урны наугад. Петя сказал, что шар --- белый. Вася сказал, что шар --- белый. Какова вероятность того, что шар действительно белый, если Петя говорит правду с вероятностью 0.8, а Вася --- с вероятностью 0.9, независимо от Пети? }
\solution{ }

\problem{
У нас ходят два автобуса --- 10-ый и 20-ый. Десятый приходит через десять минут после 20-го; 20-ый --- через 20 минут после десятого. Я прихожу на остановку в случайный момент времени. 
\begin{enumerate}
\item Сколько мне в среднем ждать автобуса? 
\item Сколько мне еще в среднем ждать автобуса, если я уже прождал $m$ минут? 
\end{enumerate}
}
\solution{ а) $frac{25}{3}$ }

\problem{ Suppose ten balls are inserted into a bag based on the tosses of an unbiased coin using the following rules: insert white ball when the coin turns up heads and insert black ball when the coin turns up tails. \\
Suppose someone who knows how the balls were selected but not what their colors are selects ten balls from the bag one at a time at random, returning each ball and mixing the balls thoroughly before making another selection. If all ten examined balls turn out to be white, what is the probability to the nearest percent that all ten balls in the bag are white?}
\solution{ about 7\% }

\problem{
Неподписанную работу мог написать один из трех человек: Аня --- отличница, Петечка и Вовочка --- двоешники. Аня всегда отвечает на вопросы теста правильно, Петечка и Вовочка --- наугад. Тест --- данетка. 
\begin{enumerate}
\item Какова вероятность того, что на 4-ый вопрос теста будет дан верный ответ? 
\item Какова вероятность того, что на 4-ый вопросы теста будет дан верный ответ, если на первые три вопроса даны верные ответы?  
\end{enumerate}
source: used at NYC interview (wilmott, bt)}
\solution{ а) $2/3$ б) $0.9$ }

\problem{ Нестандартный кубик \\
Нестандартный кубик изготавливают следующим образом: на каждой грани равновероятно независимо от других граней пишут одно из чисел от одного до шести. Т.е. на кубике могут оказаться даже только шестерки. Затем этот кубик подбрасывается два раза. 
\begin{enumerate}
\item Верно ли, что результаты подбрасываний независимы? 
\item Какова вероятность того, что в первый раз выпадет шесть? 
\item Какова вероятность того, что во второй раз выпадет шесть, если в первый раз выпало шесть? 
\item Какова вероятность того, что шесть выпадет два раза подряд? 
\item Чему равна корреляция результатов подбрасываний? 
\item Чему равно ожидаемое количество шестерок на кубике, если из $n$ подбрасываний оказалось $k$ шестерок? 
\end{enumerate}
}
\solution{
а) нет, т.к. если выпало шесть, то это увеличивает ожидаемое количество шестерок на кубике \\
б) $1/6$ (можно считать, что кубик сначала подбросили, а потом подписали стороны) \\
в) $11/36$ \\
г) $11/216$ }


