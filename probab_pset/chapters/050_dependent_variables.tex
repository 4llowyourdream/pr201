% !Mode:: "TeX:UTF-8"
\section{Связь между случайными величинами, Cov, Corr}
\subsection{Дискретный случай}
%(частная/маржинальная/... ф распределения)


\problem{ \label{vtoroie podkidivanie}
 У Васи есть $n$ монеток,
каждая из которых выпадает орлом с вероятностью $p$. В первом
раунде Вася подкидывает все монетки, во втором раунде Вася
подкидывает только те монетки, которые выпали орлом в первом
раунде. Пусть $R_{i}$ "--- количество монеток,
подкидывавшихся и выпавших орлом во $i$-м раунде.
\begin{enumerate}
\item Каков закон распределения величины $R_{2}$?
\item  Найдите $\Corr(R_{1},R_{2})$.
\item  Как ведет себя корреляция при $p\to 0$ и $p\to 1$? Почему?
\end{enumerate}
 }
\solution{ Закон распределения по сути эксперимента: $\bin(n; p^{2}$); $\Corr(R_{1},R_{2}) =\sqrt{\frac{p}{1+p}}$. При $p\to 0$ корреляция равна 0; при  $p\to 1$ составляет 0{,}5. Почему "--- чез = чёрт его знает.
}



\problem{<<Корреляция --- это мера линейной связи>>

Найдите все случайные величины $X$ такие, что $corr(X,X^2)=1$.


Источник: Алексей Суздальцев}

\solution{Случайные величины, принимающие два значения $x_1<x_2$, такие, что $|x_1|<|x_2|$.}



\problem{ Вася может получить за экзамен равновероятно либо 8 баллов, либо 7 баллов. Петя может получить за экзамен либо 7 баллов --- с вероятностью 1/3; либо 6 баллов --- с вероятностью 2/3. Известно, что корреляция их результатов равна 0.7. 


Какова вероятность того, что Петя и Вася покажут одинаковый результат? }
\solution{ }




\subsection{Непрерывный случай (Cov, Corr)}


\problem{Пусть $ X $ равномерно на $ [0;1] $. Если известно, что $ X=x $, то случайная величина $ Y $ равномерна на $ [x;x+1] $. Найдите $ P(X+Y<1) $ и $ f_{X|Y}(x|y) $. Как распределено $ Y-X $?}
\solution{$ P(X+Y<1)=1/4 $, $ f_{X|Y}(x|y)=1/f(y) $ при $ y\in [x;x+1], x\in[0;1] $. Равномерно.}


\problem{ Внутри круга единичного радиуса с центром в начале координат случайно равномерно выбирается точка. Пусть $X$ и $Y$ - ее абсцисса и ордината. Найдите совместную функцию плотности $p(x,y)$, частную функцию плотности $p(x)$, условную функцию плотности $p(x|y)$, $E(X|Y)$, $E(X^{2}|Y)$, $Cov(X,Y)$. \\
Являются ли $X$ и $Y$ независимыми? }
\solution{ }



\problem{ Рассмотрим кольцо, задаваемое системой неравенств: $x^{2}+y^{2}\geq 1$ и $x^{2}+y^{2}\le 4$. Случайным образом, равномерно на этом кольце, выбирается точка. 
Пусть $X$ и $Y$ --- ее координаты. 
\begin{enumerate}
\item Чему равна корреляция $X$ и $Y$?
\item Зависимы ли $X$ и $Y$?  
\end{enumerate} }

\solution{ $\Cov(X,Y)=0$, т.к. и рост, и падение $X$ несут в себе одинаковую информацию об $Y$, но они Зависимы, т.к. $X$ содержит информацию об $Y$ }



\subsection{Общие свойства Cov и Var}

\problem{Верно ли, что $X$ и $Y$  независимы, если известно, что
\begin{enumerate}
\item Величина $X$ и любая функция $g(Y)$ некоррелированы?
\item Любая функция $f(X)$ и любая функция $g(Y)$ некоррелированы?
\end{enumerate}
}
\solution{В первом случае "--- нет, например, $X \sim \mN(0;1)$ и $Y=X^2$. Во втором "--- да: возьмём индикаторы и получим стандартное определение независимости}


\subsection{Преобразование случайных величин (преобразования)}


\problem{ \label{simmetria razbitia otrezka}
На отрезке равномерно и независимо выбираются две точки. Верно ли,
что длины получающихся трёх отрезков распределены одинаково? }
\solution{Да. Возьмём окружность. Наугад отметим три точки. Одну будем
трактовать как разрезающую окружность на отрезок. Две других "---  как
разрезающие отрезок на три части. }
\cat{uniform} \cat{circle_trick}


\problem{ \zdt{Птички на проводе-1}

На провод, отрезок $[0; 1]$, равномерно и независимо друг от друга
садятся $n$ птичек. Пусть $Y_{1}$,\ldots, $Y_{n+1}$ "--- расстояния
от левого столба до первой птички, от первой птички от второй и т.\,д.
\begin{enumerate}
\item Найдите функцию плотности $Y_{1}$;
\item Верно ли, что все $Y_{i}$ одинаково распределены?
\item  Верно ли, что все $Y_{i}$ независимы?
\item Найдите $\Cov(Y_{i},Y_{j})$ (вроде бы ковариации равны?);
\item  Как распределена величина $n\cdot Y_{1}$ при больших $n$? Почему?
\end{enumerate}
 }
\solution{ Пусть $X_{i}$ "--- координата $i$-ой птички. $\PP(Y_{1}\le t)=1-\PP(\min\{X_{i}\}>t)=1-(1-t)^{n}$. Далее, $\lim \PP(nY_{1}\le t)=\lim 1-\left(1-\frac{t}{n}\right)^{n}=1-e^{-t}$. Распределение экспоненциальное с параметром $\lambda=1$. }
\cat{uniform} \cat{circle_trick} \cat{exponential}


\problem{ \zdt{Птички на проводе-2}

На провод, отрезок $[0;1]$, равномерно и независимо друг от друга
садятся $n$ птичек. Мы берем ведро жёлтой краски и для каждой
птички красим участок провода от неё до ближайшей к ней соседки.

Какая часть провода будет окрашена при больших $n$? }
\solution{ Пусть $n$ велико, тогда $Y_{i}$ можно считать независимыми и
$nY_{i}$ "--- экспоненциально распределёнными. Не красятся только <<большие>> интервалы, т.\,е. интервалы, чья
длина больше, чем каждого из двух соседних интервалов. <<Больших>>
интервалов примерно треть. Находим $\E(B)=\E(\max\{Y_{1},Y_{2},Y_{3}\})$. $\delta=1-\E(B)\frac{n}{3}=\frac{7}{18}$. }
\begin{ist}
Marcin Kuczma.
\end{ist}

\problem{Длительность разговора клиента в минутах $X$ "--- экспоненциальная случайная величина со средним две минуты. Стоимость разговора, $Y$, составляет $5$ рублей за весь разговор, если разговор короче двух минут, и $2{,}5$ рубля за минуту, если разговор длиннее двух минут. Неполные минуты оплачиваются пропорционально, например, за 3{,}5 минуты нужно заплатить $2{,}5\cdot 3{,}5$ рублей. Найдите $\E(Y)$, $\Var(Y)$, постройте функцию распределения $Y$. }
\solution{}

\problem{Вася пришёл на остановку. Ему нужен 42-й или 21-й автобус. Время до прихода 42-го равномерно на отрезке $[0;10]$ минут, время до прихода 21-го равномерно на отрезке $[0;20]$ минут. Время прихода 42-го и время прихода 21-го "--- независимые величины. Обозначим за $Y$  время, которое Вася проведёт в ожидании на остановке. Найдите функцию плотности $Y$, $\E(Y)$, $\Var(Y)$. }
\solution{}


\subsection{Прочее про несколько случайных величин}

\problem{\zdt{Парадокс голосования}
\par
Пусть $X$, $Y$, $Z$ "--- дискретные
случайные величины, их значения попарно различны с вероятностью 1.
Докажите, что $\min\left\{\PP(X>Y),\PP(Y>Z),\PP(Z>X)\right\}\le \frac{2}{3}$.
Приведите пример, при котором эта граница точно достигается. }
\solution{}

