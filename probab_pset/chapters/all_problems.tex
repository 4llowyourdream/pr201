% !Mode:: "TeX:UTF-8"
% to do:
% "Some probability paradoxes in choice from among rando malternatives"
% Szekely, Paradoxes, waiting time




\section{Простые эксперименты}
% simple_experiments

\subsection{Дискретные простые эксперименты}
%Эксперимент состоит из одного "этапа"

%Правило сложения вероятностей.
%Если события несовместны, то
%P(хотя бы одно)= сумма
%Р(все сразу)=0

%1.1. дискретный случайные величины (P, E)

%1.2. непрерывные случайные величины (P, E для равномерной)
%1.3. смешанные случайные величины (P, E для смеси с равномерной)
%1.4. Эксперимент состоит из множества одинаковых этапов, но независимость в явном виде не используется
%(сюда можно отнести простые задачи на биномиальное распределение и совсем простую комбинаторику)
%Деревья появляются здесь!




\problem{ \label{simple optimization}
Подбрасывается правильный кубик. Узнав результат, игрок выбирает,
подкидывать ли кубик второй раз. Игрок получает сумму денег равную
количеству очков при последнем подбрасывании. 
\begin{enumerate}
\item Каков ожидаемый выигрыш игрока при оптимальной стратегии? \\
\item Каков ожидаемый выигрыш игрока, если максимальное количество подбрасываний равно трем? }
\end{enumerate}
\solution{$\frac{1}{2}5+\frac{1}{2}\frac{7}{2}=4.25$ }


\problem{
What is the probability that a random 2 digit base b number will be relativly prime with its digit reversal? \\
Comment: maybe many cases... }
\solution{ }

\problem{
 Неправильную монетку, у которой <<орел>> выпадает с
вероятностью $p$, подбрасывали 50 раз. При этом оказалось, что она
выпала на <<орла>> 41 раз. При каком  $p$  вероятность этого
события будет максимальной?}
\solution{ }

\problem{
 Игральный кубик подбрасывается 100 раз. Найдите ожидаемую
сумму очков, дисперсию суммы, стандартное отклонение суммы.}
\solution{ }


\problem{ \label{mosti} Мосты \\
Картинка. На картинке: два берега, посреди мелкие острова,
расположенные прямоугольником размера $n(n-1)$. В результате
паводка каждый мост был размыт с вероятностью $1/2$ независимо от
других. Какова вероятность того, что с одного берега можно добраться на другой? }
\solution{$1/2$. Если рассмотреть <<проплывы>> под мостами (по научному,
двойственный граф), то он выглядит точно также. Вероятность того,
что есть проплыв под разрушенными мостами равна вероятности
добраться с одного берега на другой. А в сумме они равны единице. }

\problem{ Русская рулетка \\
Шестизарядный револьвер. В нем три пули занимают три соседних места. Барабан крутят один раз. Задем первый игрок стреляет себе в голову. Если он остается жив, то барабан не перекручивается, и второй игрок стреляет себе в голову. Затем револьвер возвращается первому игроку и т.д., до тех пор, пока кто-то не погибнет. \\
Кем лучше быть в этой игре, первым или вторым? }
\solution{
у второго шансы проиграть равны $\frac{1}{3}$ }

\problem{
In a tournament, there are no ties. There are 7 teams and each
team plays each other exactly once. Each team has a 50\% chance of
winning each game. The winner is awarded 1 point and loser no
points. Total points are accumulated to decide the ranking of the
teams. In the first game, Team A beats team B. What is the
probability that A finishes with more points than B? \\
source: aops, t=110957}
\solution{ }

\problem{ Нестандартный кубик \\
Нестандартный кубик изготавливают следующим образом: на каждой грани равновероятно независимо от других граней пишут одно из чисел от одного до шести. Т.е. на кубике могут оказаться даже только шестерки. Затем этот кубик подбрасывается два раза. \\
а) Верно ли, что результаты подбрасываний независимы? \\
б) Какова вероятность того, что в первый раз выпадет шесть? \\
в) Какова вероятность того, что во второй раз выпадет шесть, если в первый раз выпало шесть? \\
г) Какова вероятность того, что шесть выпадет два раза подряд? \\
д) Чему равна корреляция результатов подбрасываний? \\
е) Чему равно ожидаемое количество шестерок на кубике, если из $n$ подбрасываний оказалось $k$ шестерок? }
\solution{
а) нет, т.к. если выпало шесть, то это увеличивает ожидаемое количество шестерок на кубике \\
б) $1/6$ (можно считать, что кубик сначала подбросили, а потом подписали стороны) \\
в) $11/36$ \\
г) $11/216$ }

\problem{ Suppose ten balls are inserted into a bag based on the tosses of an unbiased coin using the following rules: insert white ball when the coin turns up heads and insert black ball when the coin turns up tails. \\
Suppose someone who knows how the balls were selected but not what their colors are selects ten balls from the bag one at a time at random, returning each ball and mixing the balls thoroughly before making another selection. If all ten examined balls turn out to be white, what is the probability to the nearest percent that all ten balls in the bag are white?}
\solution{
 about 7\% }

\problem{
Неподписанную работу мог написать один из трех человек: Аня - отличница, Петечка и Вовочка - двоешники. Аня всегда отвечает на вопросы теста правильно, Петечка и Вовочка - наугад. Тест - данетка. \\
а) Какова вероятность того, что на 4-ый вопрос теста будет дан верный ответ? \\
б) Какова вероятность того, что на 4-ый вопросы теста будет дан верный ответ, если на первые три вопроса даны верные ответы?
source: used at NYC interview (wilmott, bt)}
\solution{ а) $2/3$ б) $0.9$ }

\problem{
Есть 101 мешок конфет. В каждом мешке 100 шоколадных конфет, неотличимых с виду. В $i$-ом мешке $i-1$ конфета с орехом, остальные без ореха. \\
Мы выбираем мешок наугад и съедаем из него две конфеты. \\
а) Какова вероятность, что первая будет с орехом? \\
б) Какова вероятность, что вторая с орехом, если первая с орехом? }
\solution{
а) 1/2 \\
б) 2/3 (есть ли красивое решение? можно пробиться через сумму квадратов) }


\problem{
В классе было 2 мальчика и сколько-то девочек. Заходит еще кто-то. Ребята решили, что народу стало слишком много, выбрали одного человека жеребьевкой и выгнали. Какова вероятность того, что вошел мальчик, если выгнали мальчика?}
\solution{$3/5$ при любом количестве девочек}

\problem{ Вася подкидывает монетку четыре раза. Если монетка выпадает орлом, то он кладет в мешок черный шар, если решкой - белый. Петя не знает, как выпадала монетка, и достает два шара из мешка наугад. Первый шар черного цвета. Какова вероятность того, что второй будет белым? }
\solution{ }


\section{Характеристики одномерных случайных величин}
\subsection{Var в дискретном случае}
\subsection{E и Var в непрерывном случае}
\subsection{Условное ожидание и условная дисперсия}
\subsection{Способ подсчета - через условные}

 
\problem{У Васи есть две неправильные монетки, выпадающие орлом с вероятностями $p_1$ и $p_2$.
\begin{enumerate}
  \item Вася $n$ раз наугад берет одну из монеток и подбрасывает ее. Найдите матожидание и дисперсию числа выпавших орлов.
  \item Вася наугад берет одну из монеток и подбрасывает ее $n$ раз. Найдите матожидание и дисперсию числа выпавших орлов.
  \item Сравните матожидания и дисперсии, полученные выше. Поясните результат интуитивно.
\end{enumerate}

Источник: Алексей Суздальцев}


\solution{В первом случае мы имеем просто <<составную>> монетку. Матожидание и дисперсия равны $n\frac{p_1+p_2}{2}$ и $n\frac{p_1+p_2}{2}\left(1-\frac{p_1+p_2}{2}\right)$ соответственно.\\
Во втором случае все немного сложнее. Матожидание будет тем же, а дисперсия будет равна $\frac{n(p_1(1-p_1)+p_2(1-p_2))}{2}+\frac{n^2(p_1-p_2)^2}{4}$, что, как нетрудно проверить, всегда не меньше дисперсии в первом случае (равенство достигается, естественно, только при $p_1=p_2$).\\
Во втором эксперименте по сравнению с первым больше вероятности <<крайних>> значений числа выпавших орлов, но меньше вероятности <<срединных>> значений. Это и является интуитивным обоснованием того, что матожидания в двух случаях одинаковые, а дисперсия во втором случае больше.}










\section{Связь между случайными величинами}
% 2dimensions



\subsection{Дискретное двумерное распределение}
% поиск Cov/Corr/частных законов распределения

\subsection{Непрерывное двумерное распределение}
% поиск Cov/Corr/частных законов распределения

\subsection{Арифметические свойства Cov/Var/Corr}

\subsection{Преобразования случайных величин}

\problem{Еще трюк

$X$,$Y$ и $Z$ --- положительные независимые одинаково распределенные случайные величины. Известно, что с.в. $M=\min\left\{\frac{X}{Y};\frac{Y}{Z};\frac{X}{Z}\right\}$ имеет плотность $f(m)=a^{-m}\ln{a}$. Найти значение параметра $a$.

Источник: Алексей Суздальцев}

\solution{6, т.к. $P(M>1)=P(X>Y>Z)=\frac{1}{6}$. Опасность: нужно проверить задачу на существование!}

   
\problem{Пусть $X_{1}$, $X_{2}$, ... $X_{n}$ независимы, одинаково распределены с $E(X_{i})=0$ и $Var(X_{i})=1$. Какое максимальное значение может достигать $E(\max\{X_{1},X_{2},...,X_{n}\})$? }
\cat{Вариационное исчисление}
\solution{Наша задача максимизировать $\int x n F^{n-1}dF$ при ограничениях: $\int xdF=0$, $\int x^{2}dF=1$.
Ответ: $F(x)=\left(\frac{a(n)x+1}{n}\right)^{1/(n-1)}$, support зависит от $n$, $E=\frac{n-1}{\sqrt{2n-1}}$ }



\subsection{Прочее...}
\problem{Придумайте такое совместное распределение величин $X$ и $Y$, чтобы каждая из них имела равномерное распределение на [0;1], а коэффициент корреляции между ними был по модулю строго больше нуля и строго меньше единицы.

Источник: Алексей Суздальцев}

\solution{Поэкспериментируйте с носителем пары икс и игрек, например можно взять $Y=f(X)$, где $f(x)=x$ если $x>a$ и $f(x)=a-x$ если $x<a$}


\problem{Еще полезное упражнение

Под записью $X\teq{d}Y$ будем понимать <<случайные величины $X$ и $Y$ имеют одинаковое распределение>>. Какие из следующих утверждений верны?
\begin{enumerate}
  \item если $X\teq{d}Y$, то $f(X)\teq{d}f(Y)$ для любой функции $f\colon\mathbb{R}\to\mathbb{R}$;
  \item если $X+Y\teq{d}Z$, то $X\teq{d}Z-Y$;
  \item если $X_1\teq{d}Y_1$, $X_2\teq{d}Y_2$, то $X_1+X_2\teq{d}Y_1+Y_2$;
  \item если $X\teq{d}Y$, то случайные вектора $\begin{pmatrix}
                                                  X \\
                                                  Y \\
                                                \end{pmatrix}
$ и $\begin{pmatrix}
                                                  Y \\
                                                  X \\
                                                \end{pmatrix}
$ имеют одинаковое распределение;
\item если $X\teq{d}Y$, то $g(X,Y)\teq{d}g(Y,X)$ для любой функции $g\colon\mathbb{R}^2\to\mathbb{R}$.
\end{enumerate}

Источник: Алексей Суздальцев}

\solution{первое верно, остальные --- нет.}


\problem{Интуиция иногда подводит...
\begin{enumerate}
     \item Случайные величины $X$ и $Y$ зависимы, случайные величины $Y$ и $Z$ зависимы. Верно ли, что случайные величины $X$ и $Z$ зависимы?
     \item Пусть $cov(X,Y)>0$, $cov(Y,Z)>0$. Верно ли, что $cov(X,Z)>0$? $cov(X,Z)\geqslant0$?
   \end{enumerate}

Источник: Алексей Суздальцев}
   
\solution{Нет и нет}


\problem{Инструментальные переменные. Простой для вычисления пример (но не очень удачный)

В эконометрике используется такое понятие, как инструментальная переменная. Это переменная, которая коррелирована с независимой переменной $X$ и некоррелирована с ошибками наблюдения $\varepsilon$. Допустим, что $Z_{1}$ и $Z_{2}$ ) независимые, нормальные $N(0;1)$ случайные величины. Пусть $X=2Z_{1}+Z_{2}$ и $\varepsilon=Z_{1}-Z_{2}$. Придумайте случайную величину вида $aZ_{1}+bZ_{2}$, которая была бы коррелирована с $X$, но была бы некоррелирована с $\varepsilon$ }
\solution{Любая вида $aZ_{1}+aZ_{2}$}





\problem{
Пусть $P(A|B)>P(A)$. Что можно сказать про $Cov(1_{A},1_{B})$? }
\solution{ }

\problem{  
Чему равна  $p_{X|Y} (x|y)$, если случайные величины
$X$  и  $Y$  независимы? }
\solution{ }

\problem{  
$\begin{array}{|c|ccc|}  \hline {Y} & {X=1} & {2} & {3} \\
\hline {1} & {?} & {?} & {?} \\ {2} & {?} & {0} & {?} \\ {3} & {0}
& {?} & {0} \\  \hline  \end{array}$ ;
Известно, что  $P(Y=1|X=k)={1\mathord{\left/ {\vphantom
{1 3}} \right. \kern-\nulldelimiterspace} 3} $,
$P(X=k|Y=1)={k\mathord{\left/ {\vphantom {k 6}} \right.
\kern-\nulldelimiterspace} 6}$  для всех  $k$.  \\
а) Заполните пропуски \\
б) Найдите $E(XY)$. }
\solution{ }

\problem{
$\begin{array}{|cccc|}
\hline {Y} & {X=0} & {3} & {6} \\  \hline {1} & {?} & {?} & {?} \\
{2} & {0,1} & {0,05} & {?} \\  \hline  \end{array}$
Известно, что  $X$  и  $Y$  независимы,
$P(Y=2|X=0)={1\mathord{\left/ {\vphantom {1 4}} \right.
\kern-\nulldelimiterspace} 4} $. \\
а) Заполните пропуски \\
б) Найдите  $E(X/Y)$. }
\solution{ }

\problem{
$\begin{array}{|c|ccc|}  \hline {Y} & {X=1} & {2} & {3} \\  \hline
{1} & {0,1} & {0,2} & {0,3} \\ {2} & {0,15} & {0,15} & {?} \\ {3}
& {0,05} & {0} & {0,05} \\  \hline  \end{array}$\\
а) Заполните пропуски \\
б) Найдите $P(X>2)$,  $P(X=1|Y=1)$, $P(Y=1|X=2)$,
$P(X=1|X=2)$, $P(X=1\cap Y=1)$.}
\solution{ }

\problem{  
Совместный закон распределения $X$ и $Y$ задан таблицей: \\
\begin{tabular}{|c|c|c|c|}
  \hline
  % after \\: \hline or \cline{col1-col2} \cline{col3-col4}...
   & $Y=-2$ & $Y=0$ & $Y=1$ \\
  \hline
  $X=-1$ & 0,1 & 0,1 & 0,2 \\
  $X=0$ & 0,2 & $p$ & 0,1 \\
  \hline
\end{tabular} \\
Найдите $p$, $P(|X|\ge|Y|)$, $E(XY)$, $E(X|Y=0)$, $Cov(X,Y)$, $Corr(X,Y)$ }
\solution{ }

\problem{
Пусть $X$ и $Y$ независимы, одинаково непрерывно распределены. \\
Верно ли, что $E(X|X>Y)=E(\max\{X,Y\})$? }
\solution{ }

\problem{
Совместный закон распределения случайных величин  $X$  и
$Y$ задан таблицей:
$$\begin{array}{|c|ccc|}  \hline {} & {Y=-1} & {Y=0} & {Y=2} \\  \hline {X=0} & {0,2} & {c} & {0,2} \\ {X=1} & {0,1} & {0,1} & {0,1} \\  \hline  \end{array}$$
Найдите  $c$,  $P(Y>-X)$,  $E(X\cdot Y^{2}
)$,  $E(Y|X>0)$, $Cov(X,Y)$, $Corr(X,Y)$ }
\solution{ }

\problem{
Совместная функция плотности имеет вид: \\
$p_{X,Y}(x,y)=\left\{
\begin{array}{ll}
  2-x-y, & $ если $x\in[0;1],y\in[0;1] \\
  0, & $ иначе $ \\
\end{array}
\right.$ \\
Найдите $P(Y>2X)$, $E(Y)$, $E(XY)$ $Cov(X,Y)$, $E(X|Y>0,5)$,
частную (предельную) функцию плотности $p_{Y}(t)$, условную
функцию плотности $p_{X|Y}(x|y)$, $E(X|Y)$. Верно ли, что величины $X$ и $Y$ являются независимыми?
Ответы: \\
$P(Y>2X)=7/24$, $E(XY)=1/6)$, $E(X|Y>0.5)=\frac{7/48}{3/8}=7/18$ \\
$X$ и $Y$ зависимы \\
$E(X)=E(Y)=\frac{5}{12}$ }
\solution{ }

\problem{  
Пусть  $X$  - сумма очков, выпавших в результате двукратного
подбрасывания кубика. Пусть  $Y$  - разность очков (число на
первом минус число на втором). Найдите $E(XY)$,
$Cov(X,Y)$, $Corr(X,Y)$ }
\solution{ }

\problem{  
Найдите ковариацию, корреляцию и дисперсию суммы двух
независимых случайных величин. }
\solution{ }

\problem{
Пусть $X$ и $Y$ равномерны на $[0;1]$ и независимы. Найдите закон распределение числа корней многочлена $t^{3}-X^{2}t+Y=0$. }
\solution{ }

\problem{  
You roll 2 dice. X is the number of 1s shown and Y is the number of 6's. Each of X, Y can take the values 0, 1or 2.
What is the joint distribtuion p(x,y) the covariance cov(X,Y) and E(X|Y)?}
\solution{ The covariance of X and Y is -1/18. The expected value of X, conditional on Y, is (2-Y)/5. }

\problem{
Вероятность дождя в субботу 0.5, вероятность дождя в воскресенье 0.3. Корреляция между наличием дождя в субботу и наличием дождя в воскресенье равна $r$. \\
Какова вероятность того, что в выходные вообще не будет дождя? }
\solution{ $0.5r\cdot\sqrt{0.21}+0.35$ }

\problem{
Пусть $X$ имеет геометрическое распределение с параметром $p_{1}$, Пусть $Y$ имеет геометрическое распределение с параметром $p_{2}$. Как распределена величина $min\{X,Y\}$? Прокомментируйте }
\solution{ геометрически с $p=1-(1-p_{1})\cdot (1-p_{2})$ }

\problem{
 Время обслуживания клиента в окошке А - с.в., имеющая
экспоненциальное распределение с  $\lambda =7$, в окошке В - с.в.
$B$, имеющая экспоненциальное распределение с  $\lambda =12$.
Величины   и  $B$  независимы. Найдите функцию плотности
$(A+B)$,  $P(2A>B)$.}
\solution{ }

\problem{
Вася решает тест путем проставления каждого ответа наугад.
В тесте 5 вопросов. В каждом вопросе 4 варианта ответа. Пусть  $X$
- число правильных ответов,  $Y$  - число неправильных ответов и
$Z=X-Y$. \\
а)  Найдите  $P(X>3)$ \\
б)  Найдите  $Var(X)$  и  $Cov(X,Y)$ \\
в)  Найдите  $Corr(X,Z)$ }
\solution{ }

\problem{
Какова вероятность того, что квадратное уравнение с действительными коэффициентами имеет два различных корня? \\
а) $ax^{2}+bx+c=0$, где $a$, $b$ и $c$ равномерны на $[-1;1]$ \\
b) $ax^{2}+bx+c=0$, где $a$, $b$ и $c$ равномерны на $[0;1]$ \\
с) $x^{2}+bx+c=0$, где $b$ и $c$ равномерны на $[0;1]$ }
\solution{ }

\problem{
 В коробке 6 красных и 2 зеленых пуговицы. Пуговицы
вытаскивают наугад до появления двух одноцветных. Пусть  $X$  -
общее количество извлеченных пуговиц в ходе эксперимента, а  $Y$ -
количество извлеченных в ходе эксперимента зеленых пуговиц. Какова
вероятность того, что эксперимент окончится при извлечении третьей
пуговицы? Найдите  $E(X)$, $Var(X)$,
$Cov(X,Y)$. }
\solution{ }

\problem{  
Из урны с 5 красными и 4 синими шарами достаются 3 шара. Найдите
закон распределения числа красных шаров  $N$. Найдите $E(N)$,
$Var(N)$,  $Cov(N,N^{2})$.}
\solution{ }

\problem{  
Из урны с 5 занумерованными шарами достается 3 шара. Пусть $X$  -
максимальный из полученных трех номеров. Найдите закон
распределения  $X$,  $E(X)$,
$Var(X)$,  $Cov(6X-3,X+4)$ }
\solution{ }

\problem{
Пусть  $\Omega =\{a,b,c\}$,  $X(a)=1$, $X(b)=2$, $X(c)=-2$
$P(a)=P(b)=\frac{1}{4} $, $P(c)=\frac{1}{2} $. Найдите все
случайные величины $Y$, такие, что $E(Y)=0$, $Cov(X,Y)=0$ и
$Var(Y)=1$. }
\solution{ }

\problem{  
$\begin{array}{c|ccc} {} & {Y=-1} & {Y=0} & {Y=2} \\  \hline
{X=0} & {0,2} & {0,1} & {0,1} \\ {X=1} & {0,3} & {0,1} & {0,2}
\end{array}$ ; \\
a) Найдите  $E(X)$,  $E(Y)$, $E(XY)$, $E(X^{2} )$, $Var(X)$,  $Var(Y)$, $Cov(X,Y)$.\\
б) Найдите $P(X=1|Y=-1)$. \\
в) Постройте функции распределения с.в.  $X$  и  $Y$. \\
г) Найдите $P(X>Y)$,  $P(X+2Y<0,5)$ }
\solution{ }

\problem{
$p_{X,Y}
(x,y)=\left\{\begin{array}{l} {\sqrt[{3}]{6} -x-y,\quad }
{if\quad x,y\ge 0,\, x+y\le \sqrt[{3}]{6} } \\ {0,\quad otherwise}
\end{array}\right. ;$  \\
a) Найдите  $E(X)$,  $E(Y)$, $E(XY)$, $E(X^{2} )$, $Var(X)$,  $Var(Y)$, $Cov(X,Y)$.\\
б) Найдите функции плотности с.в.  $X$,  $Y$  и $Z=X+Y$.\\
в) Постройте функции распределения с.в.  $X$  и  $Y$. \\
г) Найдите $P(X>Y)$,  $P(X+2Y<0,5)$ }
\solution{ }

\problem{
$p=\left\{\begin{array}{l} {2,\quad
if\quad } {x,y\ge 0,\, x+y\le 1} \\ {0,\quad otherwise}
\end{array}\right.$ \\
a) Найдите  $E(X)$,  $E(Y)$, $E(XY)$, $E(X^{2} )$, $Var(X)$,  $Var(Y)$, $Cov(X,Y)$.\\
б) Найдите функции плотности с.в.  $X$,  $Y$  и $Z=X+Y$.\\
в) Постройте функции распределения с.в.  $X$  и  $Y$. \\
г) Найдите $P(X>Y)$,  $P(X+2Y<0,5)$ }
\solution{ }

\problem{
Совместная функция плотности имеет вид \\
$p_{X,Y}(x,y)=\left\{
\begin{array}{ll}
  x+y, & $ если $x\in[0;1],y\in[0;1] \\
  0, & $ иначе $ \\
\end{array}
\right.$ \\
Найдите  $P(Y>X)$,  $E(X)$, $E(X|Y>X)$, $Cov(X,Y)$, частную
(предельную) функцию плотности $p_{Y}(t)$, условную функцию
плотности $p_{X|Y}(x|y)$, $E(X|Y)$. Верно ли, что величины $X$ и $Y$ являются независимыми?}
\solution{ }

\problem{  
Пусть  $X$  равновероятно принимает значения -1, 0, +1.
а) Найдите  $Cov(X,Y)$, если  $Y=X^{2} $. \\
б) Верно ли, что $X$ и $Y$ независимы? \\
Ответ: $Cov(X,Y)=0$, $X$ и $Y$ зависимы }
\solution{ }

\problem{  
Пусть  $X$  и  $Y$  независимы и имеют одинаковый закон
распределения. С.в.  $X$  равновероятно принимает натуральные
значения от 1 до 3. Найдите закон распределения суммы
$(X+Y)$. }
\solution{ }

\problem{   
$X$  и  $Y$  независимы и равномерны на отрезке $[0;1]$.
a) Найдите функцию плотности  $Z=X+Y$. \\
б) Найдите функцию плотности $Z=XY$ }
\solution{ 
$p(t)=\left\{\begin{array}{c}
t,$ если $t\in [0;1) \\
2-t, $ если $t \in [1;2) \\
0, $ иначе$ \\
\end{array}\right.$\\
$p(t)=\left\{\begin{array}{c}
-ln(t),$ если $t\in (0;1) \\
0, $ иначе$ \\
\end{array}\right.$}

\problem{  
Найдите  $c$, частные функции плотности  $p_{X} (t)$ и $p_{Y}
(t)$,  $E(X)$, $E(Y)$, $E(X\cdot Y)$, $Cov(X,Y)$ :

а)  $p(x,y)=\left\{\begin{array}{l} {c\cdot (x+y),\; if\; x\in
\left[1;2\right],y\in \left[0;2\right]}
\\ {0,\; otherwise} \end{array}\right.$ ; б)
$p(x,y)=\left\{\begin{array}{l} {c,\; if\; x\ge 0,y\ge 0,x+y\le 1}
\\ {0,\; otherwise} \end{array}\right.$}
\solution{ }

\problem{  
С точностью до константы найдите функцию плотности с.в. $X$. Верно
ли, что  $X$  и  $Y$  независимы? В <<б>> дополнительно найдите
значение  $c$.

а)  $p_{X,Y}(x,y)=c\cdot \exp (-\frac{x^{2} +y^{2} }{2} )$ ; б)
$p_{X,Y} (x,y)=\frac{c}{1+x^{2} +x^{2} y^{2} +y^{2}}$ }
\solution{ }
\problem{  
В треугольнике образованном точками  $A(0,0)$, $B(2,0)$  и
$C(1,1)$  ф. плотности имеет вид:

а)  $p(x,y)=c(x+2y)$ ; б) $p(x,y)=cxy$ ; в) $p(x,y)=c$.

Вне треугольника плотность равна нулю. Найдите  $c$,  $p_{X} (t)$
и  $p_{Y} (t)$,  $E(X)$, $E(Y)$, $E(X\cdot Y)$.}
\solution{ }

\problem{ \label{ravnomernaia summa}
Возможно ли, что $U=X+Y$, $X$ и
$Y$ - iid, а $U$ -
равномерна на $[0;1]$? }
\solution{Нет, на квадрате появляется противоречие  }

\problem{ Внутри круга радиуса 1 равномерно выбирается точка. Пусть $X$ и $Y$ - ее абсцисса и ордината. Найдите совместную функцию плотности $p(x,y)$, частную функцию плотности $p(x)$, условную функцию плотности $p(x|y)$, $E(X|Y)$, $E(X^{2}|Y)$, $Cov(X,Y)$. \\
Являются ли $X$ и $Y$ независимыми? }
\solution{ }

\problem{ Вася может получить за экзамен равновероятно либо 8 баллов, либо 7 баллов. Петя может получить за экзамен либо 7 баллов - с вероятностью 1/3; либо 6 баллов - с вероятностью 2/3. Известно, что корреляция их результатов равна 0.7. \\
Какова вероятность того, что Петя и Вася покажут одинаковый результат? }
\solution{ }

\problem{
Suppose X, Y are random variables with joint density $f(x,y) = a^{3}xe^{-ay}$ for $0<x<y$ and 0 otherwise. \\
a) What is the density of $Y$? What is $E(Y)$? \\
b) What is $E(X|Y=1)$? \\
source: aops, t=177445 }
\solution{ 
$p(y)=\frac{a^{3}}{2}y^{2}e^{-ay}$, $y>0$ \\
$E(Y)=3/a$ \\
$E(X|Y=1)=1$ }


\problem{
Пусть $X_{1}$, $X_{2}$, ..., $X_{n}$ - iid $U[0;1]$ \\
а) Найдите функцию плотности для порядковой статистики $X_{(k)}$ \\
б) Что изменится, если $X_{i}$ - iid с функцией плотности $p(t)$ и функцией распределения $F(t)$? \\
Solution: \\
http://en.wikipedia.org/wiki/Order\_statistic }
\solution{ }
\problem{
Рассмотрим кольцо, задаваемое системой неравенств: $x^{2}+y^{2}\geq 1$ и $x^{2}+y^{2}\le 4$. Случайным образом, равномерно на этом кольце, выбирается точка. \\
Пусть $X$ и $Y$ - ее координаты. \\
а) Чему равна корреляция $X$ и $Y$? \\
б) Зависимы ли $X$ и $Y$? \\
Ответы: \\
а) 0, т.к. и рост, и падение $X$ несут в себе одинаковую информацию об $Y$ \\
б) Зависимы, т.к. $X$ содержит информацию об $Y$ }
\solution{ }
\problem{
Say $X$ has a density $f(x) = 3x^{ - 4}$ if $x > 1$, and 0 otherwise. Now
say $X_1,...,X_{16}$ are independent with density $f$. Let $Y=(X_1X_2...X_{16})^{1/16}$. Find $E(Y)$ and $Var(Y)$.}
\solution{ Найдем функцию плотности $\ln(X_{i})$. Окажется, что это экспоненциальное распределение с параметром $\lambda=3$ \\
Находим закон распределения суммы $\ln(X_{i})$. \\
Находим закон распределения $Y$. \\
Source: aops, t=187872 }

\problem{
Say X and Y are independent random variables with densities f and g, respectively:\\
$f(x) = e^{-x}$ if $x > 0$ \\
$g(y) = 2e^{-2y}$ if $y > 0$ \\
Calculate a density function for $Y/(X+1)$ \\
Solution: \\
We begin by considering the cumulative distribution function of $Z =Y/(X+1)$:\\
$F_Z(z) = \Pr\left[\frac {Y}{X + 1} \le z\right]
   = \Pr[Y \le z(X + 1)]
   = \int_{x = 0}^\infty \Pr[Y \le z(X + 1) | X = x] f_X(x) \, dx
   = \int_{x = 0}^\infty F_Y(z(x + 1)) f_X(x) \, dx$ .
Since Y is exponentially distributed with mean 1/2, it follows that
the CDF of Y is simply $F_Y(y) = 1 - e^{ - 2y}$. Hence $F_Z(z) = \int_{x = 0}^\infty (1 - e^{ - 2z(x + 1)}) e^{ - x} \, dx = 1
   - \frac {e^{ - 2z}}{1 + 2z}$. Therefore the density of Z is the derivative of the CDF, or $f_Z(z) = \frac {4e^{ - 2z}(1 + z)}{(1 + 2z)^2}$. \\
Source: aops, t=187870 }
\solution{ }

\problem{
С.в. $X$ принимает значения 1 и 2 с вероятностями 0,3 и 0,7 соответственно. С.в. $Y$ принимает те же значения с вероятностями 0,5 и 0,5. Корреляция между $X$ и $Y$ равна $r$. \\
a) Найдите закон распределения величины $Z=XY$ если $r=0.5$ \\
б) При каком $r$ значение $E(XY)$ будет максимальным? }
\solution{ }

\problem{
Пусть величины $X$ и $Y$ имеют совместную функцию плотности: \\
$p(x,y)=\frac{1}{4}(1+xy)$, при $|x|<1$, $|y|<1$. \\
а) Верно ли, что $X$ и $Y$ независимы? \\
б) Верно ли, что $X^{2}$ и $Y^{2}$ независимы? }
\solution{ }

\problem{
Пусть $X_{1}$ и $X_{2}$ независимы и имеют стандартное нормальное
распределение; величина $Y$ независимо от $X_{i}$ равновероятно
принимает значения 1 или -1. Пусть также $Z_{i}=YX_{i}$. \\
а) Верно ли, что $Z_{1}$ и $Z_{2}$ независимы? \\
б) Верно ли, что $Z_{1}^{2}$ и $Z_{2}^{2}$ независимы? }
\solution{ }
\problem{
Пусть совместный закон распределения $X$ и $Y$ задан следующим
образом: \\
$p_{1,1}=p_{-1,1}=\frac{1}{32}$,
$p_{-1,-1}=p_{1,-1}=p_{1,0}=p_{0,1}=\frac{3}{32}$,
$p_{-1,0}=p_{0,-1}=\frac{5}{32}$, $p_{0,0}=\frac{8}{32}$. \\
а) Верно ли, что $X$ и $Y$ независимы? \\
б) Верно ли, что $X^{2}$ и $Y^{2}$ независимы? }
\solution{ }

\problem{
Пусть $Z\sim U[0;2\pi]$, $X=\cos(Z)$, $Y=\sin(Z)$. \\
а) Верно ли, что $X$ и $Y$ независимы? \\
б) Найдите $Corr(X,Y)$ }
\solution{ зависимы, но корреляция равна нулю }


\problem{
Пусть $X\sim N(0;1)$ и $Y=X^{2}-1$. \\
а) Верно ли, что $X$ и $Y$ независимы? \\
б) Найдите $Corr(X,Y)$ }
\solution{ }
\problem{
Приведите пример таких $X$ и $Y$, что $P(X=Y)=0$, однако $X$ и $Y$
одинаково распределены. }
\solution{ }
\problem{
Приведите пример таких $X$, $Y$ и $Z$, что $X$ и $Y$ одинаково
распределены, а $XZ$ и $YZ$ имеют разное распределение. }
\solution{ }
\problem{
Пусть $X\sim U[0;1]$, $Y=1-X$, $Z=|2X-1|$, с помощью этих величин
определим также: \\
$X_{1}=tg\left(\frac{\pi X}{2}\right)$, $Y_{1}=tg\left(\frac{\pi
X}{2}\right)$,
$Z_{1}=-2tg\left(\frac{\pi X}{2}\right)$, \\
$X_{2}=tg\left(\frac{\pi X}{2}\right)$, $Y_{2}=tg\left(\frac{\pi
Y}{2}\right)$,
$Z_{1}=-2tg\left(\frac{\pi Z}{2}\right)$, \\
a) Верно ли, что $X$, $Y$ и $Z$ одинаково распределены? \\
б) Верно ли, что $X_{1}$, $X_{2}$, $Y_{1}$, $Y_{2}$ одинаково
распределены? \\
в) Верно ли, что $Z_{1}$ и $Z_{2}$ одинаково распределены? \\
г) Найдите $E(X_{1}+Y_{1}+Z_{1})$ и $E(X_{2}+Y_{2}+Z_{2})$. Почему
они не равны? }
\solution{ }
\problem{
Пусть $\Omega=\{\omega_{1},...,\omega_{9}\}$, где
$\omega_{1}$,...,$\omega_{6}$ - перестановки чисел 1, 2 и 3, а
$\omega_{7}=(1,1,1)$, $\omega_{8}=(2,2,2)$, $\omega_{9}=(3,3,3)$.
Пусть каждый исход $\omega_{9}$ имеет вероятность $frac{1}{9}$.
Проведем одно испытания и определим величину $X_{k}$, равную тому
числу, которое находится на месте с номером $k$, $k=1,2,3$. \\
а) Верно ли, что величины $X_{1}$, $X_{2}$ и $X_{3}$ попарно
независимы? \\
б) Можно ли выразить $X_{3}$ через $X_{1}$ и $X_{2}$? }
\solution{ }

\problem{ Пусть $X$, $Y$, $Z$ - независимы и равномерны на $[0;1]$. \\
Как распределена величина $(XY)^{Z}$? }
\solution{ равномерно :) \\
Here is my intuitive explanation: \\
Let $W = log(XY)^Z=Z(logX+logY)$. \\
We know that the log of a uniform random variable has the distribution of an exponential random variable, therefore $logX$ and $logY$ are 2 independent expo(1).
We can interpret them as the inter-arrival times of a poisson process with intensity 1. Given the second arrival time T2 of a PP(1), it is well known that the first arrival time is uniformly distributed between 0 and T2.
An other way of saying that is $Z*T2$ has the same distribution of the first arrival time of a PP(1), with Z~Unif[0,1].
We can conclude by seeing that $T2 = logX+logY$, so $W = Z*T2$ is an exponential random variable with parameter1 (first arrival of a PP(1))
and $e^W$ is Unif[0,1]  }

\problem{ Пусть $X$, $Y$ - независимы и равномерны на $[0;1]$.
Как распределена величина $Z=\frac{X}{Y}$? }
\solution{ }

\problem{
Suppose that a race is run along a linear track and that the starting point S and
ending point E are chosen randomly where S is uniform on $[0,1]$ and E is uniform
on $[8,9]$. Let L be the length of the race. Find the pdf of L. }
\solution{ }

\problem{
Two independent random variables uniformly distributed in $[0,1]$. How do you transform them, so that they stay uniformly distributed in $[0,1]$, but the correlation
between them becomes $\rho$? \\
source: Morgan Stanley interview, wilmott bteaser forum, carlitos, pavlinair, wb3517 }

\solution{ possible solution: If X,Y are uniform in $[0,1]$, we can define a new variable $Z$ that maps Y into the interval $[0,a]$ if $X<a$, and into the interval $[a,1]$ if $X>a$. The joint distribution for $X,Z$ is uniform in the subregion defined by the squares $((0,0) (a,a))$ and $((a,a) (1,1))$. The marginal distribution for $Z$ is obviouly uniform in $[0,1]$ 

$[$edited: I thought the problem was to get a correlation equal to 0.5$]$ 

$a=0.5$ gives the maximum correlation that can be produced using this method $(\rho=0.75)$. The $R$ function below calculates the correlation for a given a (it's straightforward to solve it analytically, but I didn't do it for the general case). 

If the desired correlation is over 0.75, the method can be generalized to get a joint distribution consisiting of more than two squares along the diagonal $((0,0) (a,a))$ $((a,a) (b,b)) ...... ((z z) (1 1))$. In the limit, the joint distribution collapses to the diagonal (perfect correlation, $X=Z$). \par

Solution2:  

Let $Y_{2}=Y$ if $X \le (1+p)/2$, or $1-Y$ otherwise. $Y_{2}$ is still $U[0,1]$ and $Corr(Y, Y_{2})=p$ }

\problem{
Пусть $X$ и $Y$ равномерны и одинаково распределены. Возможно ли, что их сумма будет равномерной? Аргументируйте ответ. }
\solution{ Ответ: да, только если $X=Y$ тождественно (через дисперсию суммы) }


\problem{
Рассмотрим пару случайных величин $X$ и $Y$. $X$ задается таблицей: \\
$\begin{array}{|c|ccc|}
\hline
$X$ & 0.5 & 1 & 2 \\
\hline
Prob & $p$ & $(1-2p)$ & $p$ \\
\hline
\end{array}$ \\
А $Y=1/X$. \\
а) При каких $p$ $X=Y$? \\
б) При каких $p$ $X$ и $Y$ одинаково распределены? \\
в) Найдите $E(X)$, $Var(X)$, $Cov(X,Y)$, $Corr(X,Y)$ \\
г) Чему равен предел $Corr(X,Y)$ при $p\to 0$? \\
д) Вася считает, что при $p\approx 0$ величины $X$ и $Y$ совпадают, поэтому корреляция должна быть близка к 1. Прокомментируйте Васино утверждение. }
\solution{ а) $p=0$ \\
б) при любых допустимых $p$ \\
в) $E(X)=1+p/2$, $Var(X)=\frac{(5-p)p}{4}$, $Cov(X,Y)=-(4+p)p/4$, $Corr(X,Y)=-(4+p)/(5-p)$ \\
г) $-4/5$ \\
д) Вася не прав. Даже если две величины равны с вероятностью 0.9999, а с вероятностью 0.0001 изменяются в противоположном направлении, то корреляция будет сильно отрицательной. }

\problem{
Пусть $X$ и $Y$ независимы и одинаково геометрически распределены с параметром $p$. \\
а) Найдите $P(X\ge 2Y)$ \\
б) Найдите $P(X\ge nY)$ }
\solution{
Ответ: $\frac{qp^{n-1}}{1-p^{n+1}}$ }

\problem{
Пусть функция плотности $X$ и $Y$ имеет вид: $p(x,y)=ax+by$ при $x\in [0;1]$, $y\in [0;1]$ \\
а) Какому условию должны удовлетворять $a$ и $b$? \\
б) Найдите $E(XY)$ \\
в) Найдите $P(XY<t)$ \\
г) Найдите функцию плотности для $Z=XY$ }
\solution{ }

\problem{
Пусть каждая из случайных величин $X$ и $Y$ принимает только два значения. Известно также, что $Cov(X,Y)=0$. Можно ли утверждать, что случайные величины независимы? }
\solution{Да. Т.к. $Cov(aX+b,cY+d)=acCov(X,Y)$, то можно считать, что $X$ и $Y$ принимают значения только 0 и 1. Тогда $E(XY)-E(X)E(Y)=P(X=1\cap Y=1)-P(X=1)P(Y=1)=0$. Что и означает независимость. }

\problem{
Пусть $X\sim U[-2;10]$, $Y\sim U[-5;3]$. \\
Найдите $P(max\{X,Y\}>0)$ \\
добавка: как-то добавить зависимость (ковариацию)? }
\solution{ }

\problem{
Случайные величины $X$ и $Y$ заданы двумерной функцией плотности $p_{X,Y}(t_{1},t_{2})$. Известно, что $p(4,8)=9$. Примерно найдите вероятность $P(X\in [4;4.003]\cap Y\in [7.999;8])$. }
\solution{ $9\cdot 0.003\cdot 0.001$ }

\problem{
Случайный процесс $E_{t}$ - последовательность iid случайных величин, равновероятно равных 0 или 1. Процесс $X_{t}$ получен из процесса $E_{t}$ путем чередования знаков у ненулевых $E_{t}$. Знак первого ненулевого $E_{t}$ равновероятно заменяется либо на положительный, либо на отрицательный. \\
Для примера: \\
$E_{t}$: 0, 1, 0, 1, 1, 1, 0, 1, 0, 0, 1\\
$X_{t}$: 0, -1, 0, 1, -1, 1, 0, -1, 0, 0, 1 \\
Найдите $Cov(X_{t},X_{t-k})$ }

\solution{
I additionally assume that the first non-zero $X_{t}$ is equally likely to be $1$ or $-1$. 
So: \\
$E(X_{t})=0$ for all $t$.  \\
$Var(X_{t})=E(X_{t}^{2})=E(E_{t}^{2})=E(E_{t})=1/2$.\\
$Cov(X_{t},X_{t-k})=E(X_{t}X_{t-k})=P(X_{t}=1\cap X_{t-k}=1)+P(X_{t}=-1\cap X_{t-k}=-1)-P(X_{t}=-1\cap X_{t-k}=1)-P(X_{t}=1\cap X_{t-k}=-1)$.\\
$P(X_{t}=1\cap X_{t-k}=1)=P(X_{t}=1\cap X_{t-k}=1|E_{t}=1\cap E_{t-k}=1)\cdot P(E_{t}=1)\cdot P(E_{t-k}=1)=1/4\cdot P(X_{t}=1\cap X_{t-k}=1|E_{t}=1\cap E_{t-k}=1)$.\\
The first probability ($P(X_{t}=1\cap X_{t-k}=1|E_{t}=1\cap E_{t-k}=1)$) is the probability that during $k-1$ trial between time $t$ and time $t-k$ there will be an odd number of $E_{n}=1$. This probability is equal to $0.5$ for all $k>1$ and it is equal to $0$ for $k=1$.\\
As a final result we get:\\
$Cov(X_{t},X_{t-k})=0$ for $k>1$\\
$Cov(X_{t},X_{t-1})=-1/4$ for $k=1$\\
And $Corr(X_{t},X_{t-k}$ is equal to $0$ (for $k>1$) and $-1/2$ (for $k=1$) }

\problem{Могут ли существовать 3 случайные величины, такие что корреляция между любыми двумя из них была бы отрицательной? А 100 таких случайных величин?}
\solution{да, могут!}

\problem{
Пусть $X$ принимает значения 1, 2 и 3 равновероятно, а $Y=1_{X<3}$. 

а) Коррелированы ли $X$ и $Y$? 

б) Если возможно придумайте величину $Z$ коррелированную с $X$, но не коррелированную с $Y$ 

Комментарий: в эконометрике в случае, если регрессор коррелирован с ошибкой, одним из методов решения является нахождение инструментальной переменной, коррелированной с регрессором и некоррелированной с ошибкой }
\solution{например, подойдет: $Z=2XY-3Y$ }


\problem{
Пусть $f$ и $g$ - возрастающие функции. Докажите, что $Cov(f(X),g(X))\geq 0$. }
\solution{ Для любых $a$ и $b$: $(f(a)-f(b))(g(a)-g(b))\geq 0$. Что равносильно условию: $f(a)g(a)+f(b)g(b)\geq f(a)g(b)+f(b)g(a)$.
Берем две независимые одинаково распределенные случайные величины $X$ и $Y$. Получаем $E(f(X)g(X))+E(f(Y)g(Y))\geq E(f(X)g(Y))+E(f(Y)g(X))$.
Пользуясь одинаковостью распределения и независимостью: $E(f(X)g(X))\geq E(f(X))E(g(X))$, что и означает положительность ковариации.}






\section{Пуассоновские и экспоненциальные величины}
%  poisson

\subsection{Пуассоновский процесс}
\subsection{Пуассоновское приближение}


\problem{
Число человек, заглянувших в магазин, распределено по
Пуассону с параметром $\lambda$. Каждый из посетителей делает
покупку с
вероятностью $p$. Как распределено число покупок? \\ }
\solution{ }
\problem{ Пуассоновский курятник \\
Число яиц $X$, которые снесет курочка Ряба, распределено по Пуассону с
параметром $\lambda$. Каждое яйцо оказывается золотым с
вероятностью $p$. \\
a) Как распределено количество золотых яиц, $G$? \\
a') Простых яиц, $N$? \\
b) Какова вероятность того, что Ряба снесет хотя бы одно простое
яйцо и ни одного золотого? \\
в) Верно ли, что количество простых и золотых яиц независимы? \\
с) $E(G|X)$? \\
d) $E(G)$ \\
e) $E(X|G)$ }
\solution{ }
\problem{
Предположим, что кузнечики на большой поляне распределены
по пуассоновскому закону с  $\lambda=3$  на квадратный метр. Какой
следует взять сторону квадрата, чтобы вероятность найти в нем хотя
бы одного кузнечика была равна  $0,8$? }
\solution{ }
\problem{
Маршруты А и Б ходят независимо друг от друга. Вася приходит на остановку в случайным момент времени и садится в первый попавшийся. Сколько в среднем ему приходится ждать в случае: \\
а) автобусы ходят регулярно, А через каждые 10 минут, Б - через каждые 15 минут \\
б) время между двумя автобусами одного маршрута имеет экспоненциальное распределение, с ожиданием 10 и 15 минут соответственно. }
\solution{ }
\problem{
Пусть $N$, количество телефонных звонков, поступающих в фирму за
сутки, имеет распределение Пуассона с параметром $\lambda=5$.
Найдите
$P(N=3)$, $P(N>3)$  и  $P(N=0)$,  $E(N)$. }
\solution{ }
\problem{  
Машины, проезжающие мимо поста ГИБДД на юг представляют собой
пуассоновский поток событий с $\lambda_{S}=200$ машин в день.
Машины, проезжающие на север - пуассоновский поток с
$\lambda_{N}=170$ машин в день. Предположим, что эти потоки
независимы. \\
Каков средний интервал времени между машинами (без учета
направления движения)? }
\solution{ }
\problem{
Дождь - пуассоновский поток (дождь - это и вправду поток!) с параметром
$\lambda=10$ капель на квадратный метр в секунду. Масса капли - полграмма. Вася держит
кастрюлю с площадью основания 300 см$^{2}$. Ожидаемое количество воды за
два часа? Вероятность ни одной капли в течении 30 секунд? }
\solution{ }
\problem{  
Маша и Саша пошли в лес по грибы\footnote{Предполагается, что
нахождение грибов, равно как отправка смс и нарушители
представляют собой Пуассоновский поток событий.} Саша собирает все
грибы, а Маша - только подберезовики. Саша в среднем находит один
гриб за одну минуту, Маша - один гриб за десять минут. \\
a) Какова вероятность того, за 10 минут они найдут ровно 10
грибов?
\\
b) Какова вероятность того, что следующий гриб попадется позже,
чем через
минуту, если Маша только что нашла подберезовик?}
\solution{ }
\problem{
Оля и Юля пишут смс Маше. Оля отправляет Маше в среднем 5
смс в час. Юля отправляет Маше в среднем 2 смс в час. Какова
вероятность того, что Маша получит ровно 7 смс за час? Сколько
времени в среднем проходит между смс, получаемыми Машей от
подруг?}
\solution{ }
\problem{
Пост майора ГИБДД Иванова И.И. в среднем ловит одного
нарушителя в час. Найдите вероятности событий. Два нарушителя
появятся с интервалом менее 30 минут. Следующего нарушителя ждать
еще более 40 минут, если уже целых три часа никто не превышал
скорость.}
\solution{ }


\problem{
В магазине две кассирши (ах, да! две хозяйки кассы). Допустим, что время обслуживания клиента распределено экспоненциально. Тетя Зина обслуживает в среднем $a$ клиентов в час, а тетя Маша - $b$. Два клиента подошли к кассам одновременно. \\
а) Какова вероятность того, что тетя Зина обслужит клиента быстрее? \\
б) Как распределено время обслуживания того клиента, который освободится быстрее? \\
в) Каково условное среднее время обслуживания клиента тетей Зиной, если известно, что она обслужила клиента быстрее тети Маши? }
\solution{ $\frac{a}{a+b}$, этот ответ можно получить взяв двойной интеграл или рассмотрев малое приращение $\Delta t$ для двух независимых пуассоновских потоков; экспоненциально с параметром $a+b$, $\frac{1}{a+b}$ }

\problem{  
Times to gather preliminary information from arrivals at
an outpatients clinic follow an exponential distribution with mean
15 minutes. Find the probability, for a randomly chosen arrival,
that more than 18 minutes will be required. }
\solution{ }
\problem{
Пусть случайные величины  $X$  и  $Y$  независимы и распределены
по Пуассону с параметрами  $\lambda _{X} =5$  и $\lambda _{Y} =15$
соответственно. Найдите условное распределение случайной величины
$X$, если известно, что  $X+Y=50$.}
\solution{ }
\problem{  [{\it Пуассоновское приближение}]\\
а) В гирлянде 25 лампочек. Вероятность брака для отдельной
лампочки равна 0,01. Какова вероятность того, что гирлянда
полностью исправна? \\
б) По некоему предмету незачет получило всего 2\% студентов.
Какова вероятность того, что в группе из 50 студентов будет ровно
1 человек с незачетом? \\
в) Вася испек 40 булочек. В каждую из них он кладет изюминку с
$p=0,02$. Какова вероятность того, что всего окажется 3 булочки с
изюмом? }
\solution{ }
\problem{
Вася каждый день подбрасывает монетку 10 раз. Монетка с
вероятностью 0,005 встает на ребро. Используя пуассоновскую
аппроксимацию, оцените вероятность того, что за 100 дней монетка
встанет на ребро ровно 3 раза. }
\solution{ }
\problem{
Страховая компания <<Ой>> заключает договор страхования от
<<невыезда>> (не выдачи визы) с туристами, покупающими туры в
Европу. Из предыдущей практики известно, что в среднем отказывают
в визе одному из 130 человек. Найдите вероятность того, что из 200
застраховавшихся в <<Ой>> туристов, четверым потребуется страховое
возмещение. }
\solution{ }
\problem{
Вася, владелец крупного Интернет-портала, вывесил на главной
странице рекламный баннер. Ежедневно его страницу посещают 1000
человек. Вероятность того, что посетитель портала кликнет по
баннеру равна 0,003. С помощью пуассоноского приближения оцените
вероятность того, что за один день не будет ни одного клика по
баннеру.}
\solution{ }
\problem{
Время устного ответа на экзамене распределено по экспоненциальному закону. В среднем на ответ одного студента уходит 10 минут. \\
а) Какова вероятность того, что Иванов будет отвечать более получаса? \\
б) Какова вероятность того, что Иванов будет отвечать еще более получаса, если он уже отвечает 15 минут? \\
в) Сколько времени в среднем длится ответ одного студента? }
\solution{ 
а) $\int_{30}^{+\infty}p(t)dt=e^{-3}\approx 0.05$ \\
б) такой же результат, как в <<а>> \\
в) $1/\lambda=10$ }

\problem{
Время между приходами студентов в столовую распределено экспоненциально; в среднем за 10 минут приходит 5 студентов. Время обслуживания имеет экспоненциальное распределение; в среднем за 10 минут столовая может обслужить 6 студентов. 

а) Как распредено количество в очереди? 

б) Какова средняя длина очереди? 

Подсказка: если сейчас в очереди $n$ человек, то через малый промежуток времени $dt$ в очереди может оказаться... }
\solution{ 
а) геометрическое распределение \\
б) $E(N)=\frac{\lambda_{in}}{\lambda_{capacity}-\lambda_{in}}$}

\problem{
Годовой договор страховой компании со спортсменом-теннисистом, предусматривает выплату страхового возмещения  в случае травмы специального вида. Из предыдущей практики известно, что вероятность получения теннисистом такой травмы  в любой фиксированный день равна 0,00037. Для периода действия договора вычислите \\
а) Наиболее вероятное число страховых случаев  \\
б) Математическое ожидание числа страховых случаев \\
в) Вероятность того, что не произойдет ни одного страхового случая \\
г) Вероятность того, что произойдет ровно 2 страховых случая \\
P.S. Указанные вероятности вычислите двумя способам: используя биномиальное распределение и распределение Пуассона.}
\solution{ <<б>> $365\cdot 0.00037=0.13505$ \\
Следовательно, <<a>>, ближайшее целое равно 0. \\
Для Пуассоновского распределения: $\lambda=0.13505$ \\
в) $P(N=0)=0.99963^{365}\approx e^{-\lambda}$ \\
г) $P(N=2)=C_{365}^{2}0.99963^{363}0.00037^{2}\approx e^{-\lambda}\lambda^{2}/2$ }


\problem{
Showing independence
Let X and Y be independent random variables with an exponential distribution with parameters $\lambda_{1}$ and $\lambda_{2}$. Let $U = min\{X,Y\}$ and $V = max\{X,Y\}$. Also, let $W = V - U$.  \\
Show that U and W are independent. \\
source: aops, t=174061}
\solution{ }

\problem{ Renyi representation \\
Let $X_{i}$ be iid random variables with exponential distribution of the same parameter $\lambda$. Let $Y_{i}$ be the ordering of $X_{i}$, i.e. $Y_{1}$ is the smallest value among $X_{i}$, $Y_{2}$ is the second smallest and so on.\\
Prove that the  spacings $Y_{1}$, $Y_{2}-Y_{1}$, $Y_{3}-Y_{2}$... are independent random variables with exponential distribution with parameters $n\lambda$, $(n-1)\lambda$, $(n-2)\lambda$ ...\\
source: aops, t=174504 }
\solution{ }

\problem{
The arrival of buses at a given bus stop follows Poisson law with rate 2. The arrival of taxis at the same bus stop is also Poisson, with rate 3. What is the probability that next time I'll go to the bus stop I'll see at least two taxis arriving before a bus? Exactly two taxis? }

\solution{ The probability of observing a taxi before a bus is given by $3/(3+2)=3/5$ since the waiting times are independent and exponentially distributed. By the memoryless property both processes then restart and hence the probability of observing (at least) two taxis before the first bus is $(3/5)^2=9/25$. The probability of observing exactly two taxis before the first bus is $(3/5)^2*(2/5)=18/125$. }


\problem{
Пусть $T_{k}$ - время наступления $k$-го события в Пуассоновском потоке с интенсивностью $\lambda$, а $N_{t}$ - количество событий, наступивших к моменту времени $t$ \\
а) Как связаны $P(T_{k}\le t)$ и $P(N_{t}\ge k)$? \\
б) Найдите вероятность того, что $T_{k}$ будет меньше $t$ \\
в) Найдите функцию плотности $T_{k}$ }
\solution{ }

\problem{
Let $T_1$ and $T_5$ be the time of the first and fifth arrivals in a Poisson process with rate $\lambda$.
a) Find the conditional density of $T_1$ given that there are 10 arrivals in the time interval (0,1). \\
b) Find the conditional density of $T_5$ given that there are 10 arrivals in the time interval (0,1). \\
c) Recognize the answers to a) and b)! \\
Hint A: Используйте то, что $p_{T_{k}}(t|N_{1}=10)=P(N_{1}=10|T_{k}=t)\frac{p_{T_{k}}(t)}{P(N_{1}=10)}$ \\
Hint B: Используйте то, что условное распределение $T_{k}$ - это распределение порядковой статистики }
\solution{ }

\problem{ Большой секрет \\
Большой секрет знают $n$ человек. Беседы между $i$-ым и $j$-ым человеком являются Пуассоновским процессом с интенсивностью $\lambda$. В каждой беседе эти люди, естестенно, говорят о Большом секрете. С вероятностью $p$ каждая беседа может стать достоянием общественности. \\
Сколько в среднем пройдет времени прежде чем Большой секрет перестанет им быть? \\
Source: Steele, Models for management secrets, Management science, vol. 35, no 2, 1989 }
\solution{Общее число бесед - Пуассоновский процесс с интенсивностью $n(n-1)\lambda$ \\
Число разглашенных бесед - Пуассоновский процесс с интенсивностью $pn(n-1)\lambda$ \\
Ожидаемое время до разглашения: $\frac{1}{n(n-1)p\lambda}$ }

\problem{ Волшебный Сундук \\
Если присесть на Волшебный Сундук, то сумма денег, лежащих в нем, увеличится в два раза. Изначально в Сундуке был один рубль. Предположим, что <<посадки>> на Сундук - Пуассоновский процесс с интенсивностью $\lambda$. Каково ожидаемое количество денег в Сундуке к моменту времени $t$? }
\solution{ найти $E(2^{N})$, вроде бы $e^{\lambda t}$}

\problem{ 
Время работы телевизора <<Best>> до первой поломки является
случайной величиной, распределенной по показательному закону. Среднее время безотказной работы телевизора фирмы <<Best>> составляет 10 лет.\\
а) Какова вероятность того, что телевизор проработает более 15 лет? \\
б) Какова вероятность, что телевизор,
проработавший 10 лет, проработает еще не менее 15 лет? }
\solution{ }

\problem{
К продавцу воздушных шариков подходят покупатели: мамы, папы и дети. Предположим, что это независимые Пуассоновские потоки с интесивностями 12, 10 и 6 чел/час. \\
а) Какова вероятность того, что за час будет всего 30 покупателей? \\
б) Какова вероятность того, что подошло одинаковое количество мам, пап детей, если за некий промежуток времени подошло ровно 30 покупателей? \\
подсказка: сумма Пуассоновских потоков - Пуассоновский поток }
\solution{ }

\problem{
Пусть $X$ - распределена экспоненциально с параметром $\lambda$ и $a>0$. 

Как распределена величина $Y=aX$? }
\solution{экспоненциально с $\frac{\lambda}{a}$ }

\problem{ 
Бассейн объемом в 1 условный литр обновляется со скоростью $\lambda$ литров в минуту. Т.е. за минуту $\lambda$ литров воды выливается и $\lambda$ литров чистой воды наливается. Предположим, что вода идеально перемешивается. \\
а) Какая доля бассейна успевает обновиться за время $t$? \\
б) Какова скорость изменения доли в момент времени $t$? \\
в) Объясните связь с экспоненциальным распределением. \\
г) сколько будет жить в бассейне случайно выбранная капля. }
\solution{ в) $dx=(1-x)vdt$, $x(t)=1-e^{-\lambda t}$ }

\problem{ 
В бассейн с 6 дорожками пришло 12 человек. Если какой-нибудь человек замечает, что есть дорожка (не обязательно соседняя), более пустая, чем та, по которой он плавает, то он переплывает на нее. Предположим, что каждый человек замечает более пустую дорожку независимо от других не сразу, а через случайное время, распределенное экспоненциально с параметром $\lambda=1$. Сначала все пришли на дорожку номер 1.\\
Сколько в среднем пройдет времени, прежде чем плавающие равномерно распределяться по бассейну? }
\solution{ минимум экспоненциальных - тоже экспоненциальное, значит время будет равно $E(T)=\frac{1}{12}+\frac{1}{11}+...+\frac{1}{3}$ }
% придумано в бассейне
% можно попробовать вариант с переходом только на соседнюю дорожку
% вариант с изначально заданным распределением по дорожкам? в общем виде?

\problem{
Допустим, что поток посетителей ларька - Пуассоновский процесс с интенсивностью $\lambda$ человек в час. Продавщица каждый день закрывает ларек на обед с 13:00 до 14:00, выходных нет. \\
Сколько (в среднем) посетителей придут к закрытому ларьку за неделю? \\
Link: \\
10336, Expected number of sums in a given set, American Mathematical Monthly, Vol. 104, No 1 (Jan. 1997) \\
http://www.jstor.org/stable/2974832 \\
Авторы: Kotlarsky, Agnew, Schilling, Roters }
% копия задачи на ML
\solution{ $7\lambda$ }

\problem{ 
$X$ - имеет экспоненциальное распределение с параметром $\lambda=1$. Как распределена величина $X^{2}$? }
\solution{ }

\problem{И в воздух чепчики бросали-2. Всречая приезжающих из армии и от двора, $n$ женщин кричат <<ура>> и подбрасывают в воздух $n$ чепчиков (каждая свой). Ловят чепчики в случайном порядке. Как распределено количество женщин поймавших свой личный чепчик при большом $n$?}
\solution{по Пуассону с $\lambda=1$}

\section{Компьютер}

\subsection{Компьютерные эксперименты}
% computer_exper

\problem{
Нарисуйте 3 реализации броуновского движения, используя 5,
10, 100
и 1000 слагаемых. }
\solution{ }
\problem{
В одной партии игрок выигрывает $2^{n}$ рублей с
вероятностью $2^{-n}$, $n\in\mathbb{N}$. Постройте 3 реализации
зависимости
среднего выигрыша от числа партий. }
\solution{ }
\problem{
Постройте график для riffle-shuffle }
\solution{ }
\problem{
Пусть $X_{1}$, $X_{2}$,..., $X_{n}$ - iid последовательность из 0
и 1. $P(X_{i}=1)=p$. Постройте эмпирический закон распределения
длины самой длинной серии из единиц для $n=20$ и $p=0.9$. \\
Т.е. по горизонтали должно быть отложено $i=0..20$, а по
вертикали, частота с которой самая длинная серия из единиц была
длины $i$. }
\solution{ }
\problem{ Мыльная пленка \\
Пьяница падает в канаву, где его штрафует милиционер. \\
Source: Сосинский }
\solution{ }
\problem{
 Сгенерите 1000 равномерно распределенных на отрезке
$[0;1]$ с.в.  $X_{i} $. Рассчитайте значения случайных величин
$\bar{X}$ и  $\hat{\sigma }^{2} $. Сравните их с  $E(X_{i} )$  и
$Var(X_{i} )$. Постройте гистограмму
(эмпирическую функцию плотности) для полученных 1000 чисел.}
\solution{ }
\problem{
 Две независимые равномерно распределенные с.в. делят
отрезок $[0;1]$  на три части. Проведите подобный эксперимент по
разделению отрезка 1000 раз. Постройте эмпирические функции
плотности длин левой, средней и правой частей. Каков
(эмпирический) вывод?}
\solution{ }
\problem{
 Сгенерите 10 равномерно распределенных на отрезке  $[0;1]$
с.в. Рассчитайте сумму  $X=X_{1} +...+X_{10} $. Повторите
эксперимент 1000 раз. Постройте эмпирическую функцию плотности
полученных 1000 значений сумм.}
\solution{ }
\problem{
Проведя 1000 экспериментов на компьютере найдите $5\%$-ое
пороговое значение статистик $U_{1}$ (Mann-Whitney, $n_{1}=5$ и
$n_{2}=4$) и $T^{+}$ (Wilcoxon Signed Rank Test, $n=9$) для
двусторонней альтернативной гипотезы. Сравните его с
асимптотическим. }
\solution{ }
\problem{
Пусть $S_{n}=X_{1}+X_{2}+...+X_{n}$, где $X_{i}$ равномерны и независимы. Отнормируйте $S_{n}$ так, чтобы среднее равнялось нулю, а дисперсия единице. Постройте график функции плотности отномированной $S_{n}$ на фоне нормальной функции плотности для $n=2,3,4,5,6$. }
\solution{ }



\subsection{Компьютерные вычислительные}
\problem{Randomly draw from a dock of 26 red and 26 black cards, stop whenever you like and get paid by the number of red cards minus the number of black. what is the optimal stopping time and corresponding expected profit of the game. source: wilmott, bt }
\solution{ }

\problem{Подбрасывается правильная монетка. В любой момент вы можете сказать <<Хватит>>. Ваш выигрыш равен доле орлов на момент остановки. С помощью компьютера определите, чему равен ожидаемый выигрыш при использовании оптимальной стратегии? При решении на компьютере можно считать, что число подбрасываний ограничено скажем 500.}
\solution{Около 0.7925}

\problem{У Васи есть 100 рублей. Вася открывает карты из колоды (26 красных и 26 черных) одну за одной в случайном порядке. Перед открытием каждой карты Вася может поставить на цвет любую целую сумму рублей в пределах своего капитала. Если он угадал цвет, то его ставка возвращается удвоенной, если нет, то он теряет ставку. Задача Васи - максимизировать ожидаемый финальный выигрыш. С помощью компьютера определите, как выглядит оптимальная стратегия и какую сумму он в среднем выигрывает? }
\solution{Ожидаемая сумму в концу игры - 808 рублей}





\section{Методы решения задач}

\subsection{Первый шаг}
% first_step


\problem{ Уравнение Пелля-0 \par
Найдите все решения в целых числах уравнения Пелля: $a^{2}-3b^{2}=1$. \par
Hint: Рассмотрите коэффициенты последовательности $(2+\sqrt{3})^{n}$ }
\solution{ }

\problem{ Уравнение Пелля \par
В мешке 2 апельсина и 1 яблоко. Если бы Вася сразу извлек два предмета наугад, то, конечно, вероятность взять два апельсина равнялась бы одной третьей. Перед тем, как Вася берет два предмета, в мешок добавили $a$ апельсинов и $b$ яблок, но при этом вероятность взять два апельсина не поменялась. \par
Найдите 9-ое наименьшее возможное $a$ }
\solution{ }

\problem{ Уравнение Пелля-2 \par
На бумаге в клеточку отмечено $2\times n$ точек, (нарисовать какие...). \par
Каждый вертикальный и горизонтальный отрезок Тиша прорисовывает с вероятностью $0.5$. Пусть $P(n)$ - вероятность того, что все точки окажутся связаны, но при этом не будет ни одной петли. \par
Найдите $P(10)$ }
\solution{ }

\problem{ Числа Фибоначчи и монетки \par
A fair coin is tossed 2007 times. Find the probability that at no point during the tossing are two heads flipped consecutively. Выразите ответ с помощью чисел Фибоначчи.}
\solution{ 
Count number of binary strings that do not contain two consecutive 1s:
Let $f_{a}(n)$ denote the number of binary strings of length n, with last digit a (0 or 1).

$f_{0}(1) = 1$; $f_{1}(1) = 1$ (There's one binary string that ends in 0, and one that ends in 1).\par
The number of binary strings of length n ending with 0, is the sum of the number of binary strings of length n-1 ending in 0 or 1.\par
$f_{0}(n) = f_{0}(n-1)+f_{1}(n-1)$ \par
Also the number of binary strings of length n ending with 1, is the sum of the number of binary strings of length n-1 ending in 0.
$f_{1}(n) = f_{0}(n-1)$\par
Adding the previous two equations together;
$f_{0}(n)+f_{1}(n) = 2f_{0}(n-1)+f_{1}(n-1)
= [f_{0}(n-1)+f_{1}(n-1)]+[f_{0}(n-2)+f_{1}(n-2)]$\par
The LHS represents the total number of binary strings of length n without two consecutive heads. This is the fibonacci sequence since $F_{n}= F_{n-1}+F_{n-2} $\par
So the answer is $\frac{F_{1997}}{2^{1997}}$}

\problem{
A player is playing the following game. In each turn he flips a coin and guesses the outcome. If his guess is correct, he gains 1 point; otherwise he loses all his points. Initially the player has no points, and plays the game until he has 2 points. \par
(a) Find the probability $p_{n}$ that the game ends after exactly $n$ flips.\par
(b) What is the expected number of flips needed to finish the game?}
\solution{ }

\problem{ \label{5 ranshe 7}$[$Mosteller$]$ \par
Два кубика подбрасываются неограниченное число раз. Какова
вероятность того, что сумма очков равная пяти, появится раньше
суммы очков, равной семи? }
\solution{Применяя метод первого шага получаем уравнение,
$p=\frac{4}{36}\cdot 1+\frac{6}{36}\cdot 0+\frac{26}{36} \cdot
p$. }

\problem{ Китайский ресторан \par
Каждый момент времени в китайский ресторан приходит новый посетитель.
Если сейчас в ресторане сидит $n$ человек, а за конкретным столиком сидит $b$ человек, то вероятность того, что новый посетитель присоединится к этому столику равна $\frac{b}{n+\theta}$. С вероятностью $\frac{\theta}{n+\theta}$ посетитель сядет за отдельный столик. \par
Каково ожидаемое число занятых столиков к моменту времени $n$? }
\solution{ }

\problem{ $[$Mosteller$]$ На краю утеса \par
Пьяница стоит на расстоянии одного шага от края пропасти. Он
шагает случайным образом либо к краю утеса либо от него. На каждом
шагу вероятность отойти от края равна 2/3, а шаг к краю имеет
вероятность 1/3. Каковы шансы пьяницы избежать падения?}
\solution{ }

\problem{ Amoeba \par
A population starts with a single amoeba. For this one and for the generations thereafter, there is a probability of 3/4 that an individual amoeba will split to create two amoebas, and a 1/4 probability that it will die out without producing offspring. What is the probability that the family tree of the original amoeba will go on for ever? 
Source: cut-the-knot }
\solution{ quadratic equation for p }

\problem{
Let X be the sum of realized tosses of a fair dice ( for exemple if you toss a dice 3 times and you got 4 ,6,1 then X=4+6+1=11).
what's expected number of throwings until you got X>=15?}
\solution{ 
Solve the difference equation
$E_{n} = (7/6)\cdot E_{n-1} - (1/6)\cdot E_{n-7}$ \par
to get E(15) about 4.76 }

\problem{
Suppose the probability to get a head when throwing an unfair coin is p, what's the expected number of throwings in order to get two consecutive heads? }
\solution{ }

\problem{
What is the probability of getting two consecutive heads after exactly n throws? \par
Выведите разностное уравнение (2-го порядка). \par
(?) Решается ли оно в явном виде - непонятно. }
\solution{ }

\problem{ Book Index Range \par
The index of a book lists every page on which certain words appear. To save space these are listed in ranges; for example, if a word occurs on pages 1, 2, 3, 5, 8, and 9, then its index contains ranges: 1-3, 5, 8-9.\par
A certain word appears on each page of an n-page book (n > 0) independently with probability p. Find the expected number of entries in its index entry

Source: cut-the-knot }
\solution{ 
Let rn(p) be the sought expectation. We shall show that rn(p) = p + (n - 1)p(1 - p) \par


by induction on n.

When n = 1, (1) becomes r1(p) = p, which is clearly true.

Suppose n > 1 and assume (1) holds for rn-1(p), which is the expected number of ranges for an (n-1)-page book. The addition of one more page may or may not change the total number of ranges. When page n is added, the number of ranges increases by one if the term occurs on page n and does not occur on page (n - 1); this happens with probability p(1 - p). Otherwise, the number of ranges does not change. Therefore,                 rn(p)   = p(1 - p)·[rn-1(p) + 1] + [1 - p(1 - p)]·rn-1(p)
        = rn-1(p) + p(1 - p)\par
        = p + (n - 1)p(1 - p).}
\problem{ $[$Mosteller$]$ Разорение игрока \par
У игрока М имеется 1 доллар, а у игрока N 2 доллара. После каждого
тура один из игроков выигрывает у другого один доллар. Игрок М
более искусен, чем N, так что он выигрывает 2/3 игр. Игроки
состязаются до банкротства одного из них. Какова вероятность
выигрыша для M? }
\solution{ }

\problem{ \label{sumashedshaia starushka}Сумасшедшая старушка \par
В самолет, имеющий $n$ мест, проданы все билеты. Для посадки в
самолет пассажиры выстроились в очередь (не обязательно по номерам
билетов). Среди пассажиров есть сумасшедшая старушка. Она
растолкала всех локтями, первой ворвалась в салон и села на первое
понравившееся ей место. Нормальный пассажир садится на свое место,
если оно не занято; если оно занято, то пассажир садится
произвольным образом на любое свободное. \par
а) Какова вероятность того,
что последний в очереди пассажир сядет на свое место? \par
б) Какова вероятность того, что 7-ой с конца пассажир сядет на свое место? \par
с) Каково примерно ожидаемое количество пассажиров, севших на свои места? \par
в) <<заразное сумасшествие>>:  в начале очереди стоит $k$
сумасшедших старушек. Как изменятся ответы? }
\solution{ $1/2$, для $n=2$ очевидно, далее по индукции \par
(no-recursion) \par
The xth seat cannot be empty, where x is not equal to 1 or 100, otherwise the xth guy will sit on it.
He can only sit at the 100th or the 1st one. The prob of 1st seat is empty = the prob of the 100th seat is empty by symmetry,
So P =0.5. \par
The prob that the 1st guy/lady choose the 1st seat is equal to the prob that he/she chooses the 100th seat.
This is how the symmetry argument comes in. \par
b) вероятности с конца равны: $1/2$, $1/3$, $1/4$ и т.д. }

\problem{ Сумасшедшие старушки и благоразумный пассажир  \par
На самолет, имеющий $100$ мест, проданы все билеты. Для посадки в
самолет пассажиры выстроились в очередь (не обязательно по номерам
билетов). Первые 99 пассажиров - сумасшедшие старушки. Они садятся
на наугад выбранные места. Последний пассажир садится на то место,
которое указано в его билете. Если это место занято, то он с
помощью стюардессы сгоняет старушку со своего законного места.
Согнанная с чужого места сумасшедшая старушка становится
благоразумной и садится на свое место по билету. Возможно для
этого придется согнать еще одну старушку и т.д. \par
a) Какова вероятность того, что потревожат бабу Дусю? \par
б) Каково ожидаемое количество потревоженных старушек? \par
Source: Колмогоровская студенческая олимпиада\footnote{
Немного по истории возникновения задачи. \par
Родилась она чисто из жизни - а именно когда тогда еще юному автору приходилось часто ездить из Питера во Псков. Чаще всего билет на покупался в общий вагон поезда. На билете всегда было указано место. Но не тут-то было! Приходя в вагон, обнаруживалось, что пришедшие первыми занимали самые лучшие места (утверждая, что номера мест на билетах - это просто для статистики). И вот тут-то и начинались эти цепочные сгоняния пассажиров.
В оригинале задачка звучала так: <<В общем вагоне поезда Ленинград-Псков N мест. ... N-й пассажир, придя последним, обнаруживал, что свободно только только самое дурацкое место - около туалета...>> \par
Но в редакции Кванта решили, то писать про бардак и туалет в общем вагоне поезда Лениград-Псков как-то не очень. И самовольно переделали условие на кинотеатр. \par
Автор: Игорь Алексеев} \par
links: \url{http://kvant.mccme.ru/1985/06/p34.htm} }
\solution{ $\frac{99}{2}$}

\problem{ Ковбои \par
Собрались 100 ковбоев. Каждый из них выбрал себе из остальных своего самого главного врага случайным образом. Далее ковбои по очереди стреляют, каждый в своего главного врага. Естественно, если жив сам, и если жив самый главный враг. Ковбои стреляют без промаха. \par
Сколько ковбоев в среднем останется в живых? \par
Какова вероятность, что случайно выбранный ковбой останется в живых? \par
Допустим, среди этих ковбоев есть ковбои без оружия, т.е. они стрелять не могут, но могут служить мишенью. Пусть доля ковбоев-пацифистов равна $\gamma$. Ответьте на предыдущий вопросы при больших $n$. \par
Источник: лента $math_ru$ }
\solution{ $\frac{1}{2}$, $\frac{1}{2-\gamma}$ }

\problem{ \label{sumashedshaia starushka 2}Сумасшедшая старушка вяжет в дороге. \par
В сумке у старушки $k>0$ красных лоскутков и $z>0$ зеленых.
Старушка вяжет шарфики для внуков. Для этого она достает из сумки
лоскутки наугад по одному и подшивает их друг к другу пока у них
совпадает цвет. Если она извлекает лоскуток, чей цвет отличается
от предыдущего, то она кладет его обратно в сумку, перемешивает
лоскутки в сумке
и начинает вязать новый шарф. Какова вероятность того, что последний шарф будет красным? }
\solution{$1/2$. Доказательство по индукции. Для $k=1$ и $z=1$ ответ
очевиден. При других $k$ и $z$ с некоторой вероятностью $\alpha$
вяжется всего два
шарфа и c вероятностью $1-\alpha$ снижается размерность задачи. Искомая вероятность равна $P=\frac{\alpha}{2}\cdot 1+\frac{\alpha}{2}\cdot 0+(1-\alpha)\cdot \frac{1}{2}$  }

\problem{ Новый шаман \label{new shaman}\par
Прежнего шамана племени забодало носорогом. На его смену пришел молодой и перспективный шаман. В силу долгой засухи вождь обратился к шаману с просьбой вызвать дождь. 

Чтобы вызвать дождь необходимо произнести заклинание <<АБРА>>. Молодой шаман знает, что в заклинании участвуют всего три буквы: <<А>>, <<Б>> и <<Р>>; и что в заклинании нет двух одинаковых букв подряд. Больше ничего о заклинании молодому шаману неизвестно. Поэтому шаман произносит буквы по одной наугад соблюдая эти два правила. 

Сколько времени придется в среднем ждать вождю, прежде чем пойдет дождь? \par
Source: \url{http://www.artofproblemsolving.com/Forum/viewtopic.php?t=151755} }
\solution{Answer: 24 \par
Solution: \par
Решаем систему уравнений: \par
$x_{\emptyset}=1+\frac{1}{3}x_{a}+\frac{2}{3}x_{other}$ \par
$x_{a}=1+\frac{1}{2}x_{ab}+\frac{1}{2}x_{other}$ \par
$x_{ab}=1+\frac{1}{2}x_{abr}+\frac{1}{2}x_{other}$ \par
$x_{abr}=1+\frac{1}{2}0+\frac{1}{2}x_{other}$ \par
$x_{other}=1+\frac{1}{2}x_{a}+\frac{1}{2}x_{other}$ \par
Solution (Non state-space): \par
If the bug is both not on A and has deviated from the ABCA sequence, it will take him an expected two moves to get back to A by a simple geometric sum. \par
Now suppose the bug is on A beginning a possible ABCA. He has a 1/2 chance of moving to C, and that adds 3 expected moves: one to move to C, expected two to get back to A. \par
The alternative is that he moved to B with 1/2 chance. Now he has a 1/2 (1/4 cumulative) chance of moving back to A, adding 2 moves (ABA). Otherwise, he's still going. \par
Now he's on C. There's a 1/2 (1/8 cumulative) chance of moving to B and adding 5 moves (ABCB plus the two moves to return to A). Else, he moves to A, gets squished, and finishes with 3 moves (ABCA). \par
So we have the following possibilities: \par
AC: +3 moves (1/2 = 4/8 chance) \par
ABA: +2 moves (1/4 = 2/8 chance) \par
ABCB: +5 moves (1/8 chance) \par
ABCA: +3 moves and finish (1/8 chance) \par
So the expected value is $3 \cdot 4+2 \cdot 2+5+3 = \fbox{24}$ }

\problem{ \label{On-Off}
Вася бьет мячом по воротам 100 раз. В первый раз вероятность
попасть равна $frac{1}{2}$, в каждый последующий раз вероятность
попасть увеличивается - Вася становится метче; при этом разные
удары независимы. Какова вероятность того, что Вася попадет в
ворота четное число раз? \par
Вариация: \par
Вася нажимает на пульте телевизора кнопку <<On-Off>> 100 раз
подряд. Пульт старый, поэтому в первый раз кнопка срабатывает с
вероятностью $\frac{1}{2}$, затем вероятность срабатывания падает.
Какова вероятность того, что после всех нажатий телевизор будет
включен, если сейчас он выключен? }
\solution{$\frac{1}{2}$ }

\problem{ Прогноз погоды 

Пусть погода может быть сухой или дождливой. На следущий день погода остается такой же как была с вероятностью $p$ и меняется с вероятностью $(1-p)$. \par
Какова вероятность того, что через $n$ дней погода будет такая же как сейчас? \par
Solution: \par
$p_{n}=pp_{n-1}+(1-p)(1-p_{n-1})$ или $p_{n}=(2p-1)p_{n-1}+1-p$ \par
Начальные условия: $p_{0}=1$, $p_{1}=p$\par
Общее решение имеет вид: $p_{n}=\frac{1}{2}(2p-1)^{n}+\frac{1}{2}$ \par
Source: aops, t=179732 \par
As a simplified model for weather forecasting, suppose that the weather (either wet or dry) tomorrow will be the same as the weather today with probability $p$. \par
If the weather is dry on January 1st, what is the probability (in terrms of $p$) that the weather will be dry on February 1st? }
\solution{ }

\problem{ \label{wait for n}
Каждая партия может закончится выигрышем в 1 очко, с вероятностью
$\frac{1}{4}$; выигрышем в 2 очка, с вероятностью $\frac{1}{2}$
или проигрышем. Петр играет до первого проигрыша. Какова
вероятность того, что он накопит $n$ очков? }
\solution{$p_{n}=\frac{1}{3}p_{n-1}+\frac{2}{3}p_{n-2}$,
$p_{0}=\frac{1}{4}$, $p_{1}=\frac{1}{16}$ }

\problem{ \label{mabinogion easy} Мабиногские овцы без пастуха\par
And on one side of the river he saw a flock of white sheep, and on
the other a flock of black sheep. And whenever one of the white
sheep bleated, one of the black sheep would cross over and become
white; and when one of the black sheep bleated, one of the white
sheep would cross over, and become black. \par
Welsh tale <<Peredur the Son of Evrawc>> (english translation by
Lady Charlotte Guest in <<The Mabinogion>>) \par
Когда Передур подъехал к Мабиногским овцам, 40 были черными и 60 -
белыми. Предположим, что блеющая овца выбирается наугад из всего
стада. \par
а) Докажите, что вероятность того, что все овцы рано или поздно
станут черными равна вероятности того, что при 99 подбрасываниях
монетки, она выпадет орлом не более 39 раз. \par
б) Оцените вероятность того, что все овцы рано или поздно станут черными }
\solution{ bad solution: \par
$b_{100}=1$ \par
$b_{0}=0$ \par
$b_{n}=\frac{n}{100}b_{n+1}+\frac{100-n}{100}b_{n-1}$ \par
После замены $\triangle_{n}=b_{n}-b_{n-1}$: \par
$b_{100}=\triangle_{1}+\triangle_{2}+...+\triangle_{100}=1$ \par
$\triangle_{n+1}=\frac{100-n}{n}\triangle_{n}$ \par
Решая, находим: \par
$\triangle_{n}=C_{99}^{n-1}2^{-99}$ \par
$b_{n}=\sum_{0\le k\le (n-1)}C_{99}^{k}2^{-99}$ }

\problem{
A queue of $(n+m)$ people is waiting at a box office; $n$
of them have 5-pound notes and $m$ have 10-pound notes. The
tickets cost 5 pounds each. When the box office opens there is no
money in the till. If each customer buys just one ticket, what is
the
probability that none of them will have to wait for change? \par
Source: drmath forum }
\solution{ }
\problem{ \label{f dlia nepravilnoi moneti}[Ross, Probability models]  \par
Игроки по очереди подбрасывают монетку. Выигрывает тот, кто первым
выбросит <<орла>>. Монетка неправильная, и <<орел>> выпадает с
вероятностью $p$. Пусть $f(p)$ - вероятность выигрыша того, кто
бросает первым. \par
Попробуйте интуитивно (не находя $f(p)$) определить: \par
а) Является ли $f(p)$ монотонной? \par
б) $\lim_{p\to 0}f(p)$ и $\lim_{p\to 1}f(p)$ \par
в) Найдите $f(p)$ и проверьте интуицию }
\solution{$f(p)=\frac{1}{2-p}$  }

\problem{ $[$Problem of the points$]$ \par
Саша и Маша играют в крестики-нолики до $n$ побед (не обязательно
подряд). Отдельную партию Саша выигрывает с вероятностью $p$, Маша
- с вероятностью $1-p$. Пусть Саше осталось победить $s$ раз, а
Маше - $m$. Обозначим $f(s,m)$ ожидаемое количество оставшихся
партий и $P(s,m)$ - вероятность того, что победит Саша \par
а) Как выражается $f(s,m)$ через $f(s-1,m)$ и $f(s,m-1)$ \par
b) Найдите $f(s,1)$ и $f(1,m)$ \par
в) Как выражается $P(s,m)$ через $P(s-1,m)$ и $P(s,m-1)$ \par
c) Найдите $P(s,1)$ и $P(1,m)$ \par
Comment: No Closed form solution! }
\solution{ }
\problem{
Имеется неправильная монетка, выпадающая орлом с
вероятностью $p$. Сколько подбрасываний монетки в среднем
потребуется для того,
чтобы: \par
а) Орел выпал два раза подряд \par
б) Орел выпал два раза не обязательно подряд \par
в) Какая-нибудь грань выпала два раза подряд \par
г) Какая-нибудь грань выпала два раза не обязательно подряд }
\solution{ }
\problem{
A gambler flips a fair coin and wins a dollar for every heads and loses a dollar for every tail. His strategy is to stop playing if --
a) He makes X dollars, or b) he has played N times (whichever occurs first). \par
On an average how many times does he flip the coin. }
\solution{ }
\problem{
Сколько подбрасываний кубика в среднем потребуется, чтобы:
а) Какая-нибудь грань выпала два раза подряд? \par
б) Произведение двух подряд выпавших граней оказалось меньше 6? }
\solution{ }
\problem{ \label{ogranichennie patroni}
Вася дошел до последнего уровня игры. У него осталось $40$
патронов, а перед ним монстр, вероятность убить которого с одного
выстрела равна $0.2$. Сколько выстрелов (в среднем) сделает Вася?}
\solution{$E(X)=\frac{1-q^{n}}{1-q}$ }

\problem{ Дополнительные патроны \label{additional shoot} \par
Вы в тире, и у Вас 100 патронов. С вероятностью $0.01$ Вы попадает в глаз Усамы Бен Ладена, за что получаете 20 дополнительных патронов, с вероятностью $0.05$ Вы попадаете в нос Усамы Бен Ладена, за что получаете 5 дополнительных патронов. Вы стреляете до тех пор, пока патроны не кончатся. Сколько в среднем Вы сделаете выстрелов? \par
Source: www.wilmott.com-forum-brainteasers }
\solution{ Во первых, заметим, что ожидаемое количество выстрелов, если у Вас осталось $n$ патронов имеет вид $E_{n}=k\cdot n$. \par
Во-вторых, получим уравнение на $E_{n}$: \par
$E_{n}=1+E_{n-1}+kE(X)$, где $E(X)$ - ожидаемый выигрыш патронов от одного выстрела. \par
Находим $k$: $k=\frac{1}{1-E(X)}$ \par
Ответ задачи: $\frac{100}{1-0.45}$ \par
Solution2: Интуитивно: $100+100\cdot 20\cdot 0.01+100(\cdot 20\cdot)^{2}+...$  }

\problem{
Consider a random binary tree. Let p be a fixed parameter between 0 and 1. Starting with the complete infinite binary tree retain each edge randomly and independently with probability p. Our random binary tree is the portion connected to the root. So for example the binary tree consisting of the root alone will be selected with probability $(1-p)^2$ (ie when neither edge out of the root is retained). \par
Some of these random binary trees will contain an infinite number of vertices. Throw these out. Then what is the expected number of vertices as a function of p of a random binary tree selected in this way? }
\solution{ Case p<0.5 \par
For p < 0.5 the infinite tree clause is a red herring, since the probability of an infinite tree is 0. \par
Let E be the expected size of the tree connected to the root. E is also therefore the expected size of the left subtree and also of the right subtree. So\par
$E = 1 *(1-p)^2 + 2(1+E)p(1-p) + (1+2E)p^2$\par
Solving for E we get $E = 1/(1-2p)$ \par
Case p>0.5 \par
Пусть $T_{k}$ - событие, состоящее в том, что осталось дерево с $k$ вершинами. $P(T_{k})=p^{k-1}(1-p)^{k+1}$ (просто считаем число оставшихся и число удаленных ребер для такого дерева) \par
Если бы $p<0.5$, то $\sum P(T_{k})=1$. \par
Поэтому при $p>0.5$ (что равносильно перемене мест $p$ и $1-p$) получаем $\sum P(T_{k})=(\frac{1-p}{p})^2$
The conclusion we can draw is that C is in fact the probability of getting a finite tree for p > 1/2 and since we are only interested in finite subtrees as final outcomes, Bayes lets us forget the scaling factor and we get the same probability distribution as for p < 1/2. Therefore the answer in the case of p > 1/2 is $E=\frac{1}{2p-1}$ \par
Case p=0.5 \par
Infinity by continuity? }


\problem{ \label{orli podriad} 
Маша и Саша подбрасывают неправильную
монетку, выпадающую орлом с вероятностью $p$. Маша выигрывает,
если серия из $k$ орлов подряд появится раньше серии из
$r$ решек. Обозначим $A$ - событие <<Маша выиграла>>, $B$ - сначала выпал орел. \par
а) Составьте два уравнения, связывающих $P(A|B)$ и $P(A|B^{c})$ \par
b) Найдите $P(A)$ }
\solution{$P(A|B)=p^{k-1}\cdot 1+(1-p^{k-1})P(A|B^{c})$ \par
$P(A|B^{c})=q^{r-1}\cdot 0+(1-q^{r-1})P(A|B)$ \par
$P(A)=p\cdot P(A|B)+q\cdot P(A|B^{c})$  }


\problem{ \label{chetnost chisla chernih} 
В коробке 123 черных шара и
321 белый шар. Извлекается наугад пара шаров. Если оба шара одного
цвета, то место пары кладется белый шар. Если извлеченная пара
шаров разного цвета, то вместо нее кладется черный шар. Так
повторяется до тех пор, пока в корзине не останется один шар.
Какова вероятность того, что он
будет черным? }
\solution{
Число черных всегда нечетно. Единица.}

\problem{ \label{Spagetti} Спагетти \par
На тарелке запутавшись и вконец устав перед казнью лежат $n$
спагетти. Мы связываем концы спагетти между собой наугад. \par
а) Каково ожидаемое количество получающихся колец? \par
б) Каково ожидаемое количество колец длиной в одну спагетти? }
\solution{Сделав первый узел мы с вероятностью $\frac{1}{2n-1}$ получаем 1
узел и гарантированно снижаем число спагетти на единицу.
Следовательно, $E(X)=\sum_{i=1}^{n}\frac{1}{2n-1}$,
примерно, $\frac{1}{2}\ln{n}$ \par
б) вероятность того, что заданная спагетти завяжется сама с собой равна $\frac{1}{2n-1}$, значит среднее равно $\frac{n}{2n-1}$  }

\problem{
Два игрока по очереди подбрасывают кубик. Выигрывает тот,
кто
первым выбросит шестерку. \par
а) Каковы шансы первого игрока на победу? \par
б) Рассчитайте шансы каждого игрока на победу, если игроков трое }
\solution{ }

\problem{ Испытания до первого успеха \par
Сколько в среднем раз надо бросать кость до появления шестерки? }
\solution{ }

\problem{
В один конверт кладут сумму  $10^{N} $, в другой - $10^{N+1} $,
где  $N$  - грань выпавшая на кубике. Петя и Вася берут конверты
наугад и каждый открывает свой. Далее их по очереди
(Петя-Вася-Петя-Вася-...) спрашивают <<хотел бы ты заплатить 1
рубль за возможность поменяться конвертами?>>. Реального обмена не
происходит. Им просто по очереди несколько раз задают один и тот
же вопрос. Каждый слышит все предыдущие ответы и говорит правду.
Как они будут отвечать? }
\solution{ }

\problem{  \label{kopilki}
На столе стоят $n$ копилок. Достать содержимое копилки можно двумя
способами: либо разбить копилку, либо открыть дно специальным
ключиком. У каждой копилки свой ключик. Мы раскладываем ключи по
копилкам наугад (один ключ в одну копилку). Затем разбиваем две
копилки. Далее разбивать
копилки запрещается. \par
а) Какова вероятность того, что мы сможем достать все ключи? \par
б) Как изменится ответ, если изначально разбить не две, а $k$
копилок? \par
в) Сколько в среднем можно добыть ключей, если разбивать $k$
копилок? \par
г) Какая доля ключей в среднем будет найдена,
если разбивать $k$ копилок? }
\solution{б) $f(k,n)=\frac{k}{n}f(k-1,n-1)+\frac{n-k}{n}f(k,n-1)$. \par
Ключ из первой разбитой копилки с вероятностью $\frac{k}{n}$ не
дает ничего нового, а с вероятностью $\frac{n-k}{n}$ <<разбивает>>
еще одну копилку. \par
Решение $f(k,n)=\frac{k}{n}$. \par
В силу простоты ответа хочется какого-нибудь более красивого
решения. Welcome! \par
в) $\frac{k(n+1)}{k+1}$ \par
г) $\frac{k(n+1)}{(k+1)n}$. Красивое док-во - welcome!  }

\problem{ Копилки-2 \par
На столе стоят $n$ копилок. Достать содержимое копилки можно двумя
способами: либо разбить копилку, либо открыть дно специальным
ключиком. У каждой копилки свой ключик. Мы связываем ключи в
несколько связок. Затем связки ключей раскладываются по копилкам
наугад (часть копилок остается пустыми, в остальных - по одной
связке). Затем разбиваем две копилки. Далее разбивать
копилки запрещается. \par
а) Какова вероятность того, что мы сможем достать все ключи? \par
б) Как изменится ответ, если изначально разбить не две, а $k$
копилок? \par
в) Сколько в среднем можно добыть ключей, если разбивать одну
копилку? Может плохой ответ? }
\solution{ }

\problem{
В вершинах треугольника три ежика. С вероятностью $p$ каждый ежик
независимо от других двигается по часовой стрелке, с вероятностью
$(1-p)$ он двигается против часовой стрелки. Сколько в среднем
пройдет времени прежде,
чем они встретятся в одной вершине? \par
При каком $p$ ожидаемое время встречи минимально? }
\solution{ 
У системы 4 состояния (1-1-1, 1-2-0, 2-1-0, 3-0-0). \par
Пишем три уравнения на ожидаемые времена. \par
Решая находим $E(T)=\frac{3}{p(1-p)}$ \par
Минимум при $p=0.5$ \par
Ответ слишком красивый... красивое решение???? }

\problem{
Монетка подбрасывается 16 раз. Какова вероятность того, что ни разу не выпадет ни три орла подряд, ни две решки подряд? }
\solution{ }

\problem{ Стопка газет \label{newspapers} \par
У Пети стопка из $n$ номеров газеты <<Вышка>> лежащих в случайном
порядке. Петя сортирует газеты следующим образом. Он
последовательно просматривает стопку сверху вниз. Если
просматриваемый выпуск более свежий, чем лежащий сверху стопки, то
Петя перекладывает более свежий выпуск наверх стопки и
начинает просматривать стопку заново. \par
Сколько <<переносов>> более свежих номеров наверх в среднем будет
сделано до того момента, когда наверху окажется первый выпуск
газеты? \par
Source: \par
http://www.artofproblemsolving.com/Forum/viewtopic.php?t=124903 }
\solution{Solution 1: \par
$p_{2}=\frac{1}{2}$ \par
С вероятностью $\frac{n-1}{n}$ сверху стопки лежит номер, меньший
$n$, в этом случае можно считать, что $n$-ый номер вообще
отсутствует в стопке. \par
С вероятностью $\frac{1}{n}$ сверху стопки лежит $n$-ый номер,
тогда обязательно происходит одно перекладывание, после которого
мысленно выкинув $n$-ый номер можно считать, что имеется случайно
упорядоченная стопка из $(n-1)$ выпуска.\par
$p_{n}=\frac{n-1}{n}p_{n-1}+\frac{1}{n}(p_{n-1}+1)$ \par
Итого: $p_{n}=\sum_{i=2}^{n}\frac{1}{i}\approx n\ln(n)$ \par
Solution 2: \par
Пусть $q_{i}$- вероятность того, что число $i$ <<уберут>> с верха стопки.\par
$q_{1}=0$ \par
Вероятность того, что число $i$ <<уберут>> с верха стопки равна
вероятности того, что среди чисел $1$, $2$,... $i$ число $i$ будет
первым, т.е. $\frac{1}{i}$. \par
$E(X)=E(X_{2})+...+E(X_{n})=\sum_{i=2}^{n}\frac{1}{i}\approx
n\ln(n)$  }

\problem{
За Круглым столом сидит Король Артур и еще $(n-1)$ рыцарь. Король Артур подбрасывает игральный кубик. Если кубик выпадает на $1$ или $2$, то он объявляется победителем. Если на $3$ или $4$, то кубик передается соседу слева. Если на $5$ или $6$, то кубик передается соседу справа. Подкидывания и передача кубика повторяются до тех пор, пока не определится победитель. \par
а) Найдите вероятность победы Короля Артура для $n$ рыцарей. \par
б) Найдите $lim_{n\to\infty}$ }
\solution{ }

\problem{
Жестокий тиран издал новый указ. Отныне за каждого новорожденного
мальчика семья получает денежную премию, но если в семье рождается
вторая девочка, то всю семью убивают. \par
Предположим, что бедные жители страны остро нуждаются в деньгах,
но не хотят рисковать своей жизнью. 

а) Какой будет средняя доля мальчиков в стране? 

б) Каким будет среднее число детей в семье? 

в) Какой будет средняя доля мальчиков в семье? 

г) Какой будет средняя доля семей, с одним ребенком? 

д) Каким будет среднее отношение числа мальчиков к числу девочек в стране?

е) Каким будет среднее отношение числа мальчиков к числу девочек в семье?

ж) Каким будет среднее отношение числа девочек к числу мальчиков в стране?

з) Почему некорректен вопрос про среднее отношение числа девочек к числу мальчиков в семье?
}
\solution{ В каждой семье одна девочка!

г) половина: как только рождается девочка семья отказывается от дальнейших детей

б) $e=0.5+0.5(e+1)$, $e=2$

в) $\sum 0.5^{i}\frac{i-1}{i}=\sum 0.5^{i}-\sum 0.5^{i}\frac{1}{i}=1+ln(1/2)=1-ln(2)$

е) в каждой семье одна девочка, поэтому $2-1=1$ (см. пункт б)

з) с вероятностью 0.5 эта величина не определена

а) $E(\frac{M_{1}+...+M_{n}}{M_{1}+...+M_{n}+n})=E(\frac{\bar{M}}{\bar{M}+1})$, по ЗБЧ (плюс DCT, все дроби ограничены сверху) искомое ожидание стремится к 0.5 при  $n\to\infty$ 

д) $E(\frac{M_{1}+...+M_{n}}{n})=E(M_{1})=1$

ж) $E(\frac{n}{M_{1}+...+M_{n}})=E(\frac{1}{\bar{M}})$, по ЗБЧ и (uniform integrable) стремится к 1.
С вероятностью $1/2^{n}$ - неопределена, как обойти?


 }

\problem{
Пусть $L_{t}$ - число точек, которые симметричное случайное
блуждание посетит ровно один раз к моменту времени $t$. \par
Найдите $E(L_{t})$. }
\solution{ }

\problem{
Сколько раз в среднем нужно подбрасывать монетку до появления $n$ орлов подряд? }
\solution{ }

\problem{
Let $X_{i}$ be independent and uniformly distribuited on $(0;1)$ . Let $x\in(0;1)$ and define  $N=min(n\ge 1|X_{1}+X_{2}+\ldots+X_{n}>x)$. \par
Find  $E(X)$, $Var(X)$.}
\solution{ 
$P(X=n)=P(x_{1}+\ldots+x_{n-1}\le x)-P(x_{1}+\ldots+x_{n}\le x)$ \par
По индукции можем показать, что $P(x_{1}+\ldots+x_{n}\le x)=\frac{x^{n}}{n!}$\par
Поэтому $P(X=n)=\frac{x^{n-1}}{(n-1)!}-\frac{x^{n}}{n!}$ \par
Далее - ряд Тейлора: \par
$E(N)=exp(x)$, $Var(x)=exp(x)(1+2x)-exp(2x)$ \par
Solution 2: \par
Пусть сейчас накоплено $a$ денег. Обозначим $N(a)$ - число оставшихся шагов. Введем две функции $f(a)=E(N(a))$ и $g(a)=E(N^{2}(a))$. \par
Методом первого шага можно получить уравнения: \par
$f(a)=\int_{a}^{x}f(t)dt$ \par
$g(a)=1+\int_{a}^{x}g(t)dt+2\int_{a}^{x}f(t)dt$ \par
И начальные условия $f(x)=g(x)=1$ \par
Далее перейти к дифурам, решить их. \par
Уравнения с интегралами можно пояснить дискретным случаем. }


\problem{
Three players A, B and C take turns to roll a die; they do this in the order ABCABC...\par
(a) Show that the probability that, of the three players, A is the first to throw a 6, B the second, and C the third, is 216/1001. \par
(b) Show that the probability that the first 6 to appear is thrown by A, the second 6 to appear is thrown by B, and the third 6 to appear is thrown by C, is 46656/753571.}
\solution{ }

\problem{
A spider moves on the eight vertices of a cube in the following way: at each step the spider is equally likely to move to each of three adjacent vertices, independently of its past motion. Let $I$ be the initial vertex occupied by the spider, and let $O$ be the vertex opposite of $I$. \par
a) Calculate the expected number of steps until the spider returns to $I$ for the first time. \par
b) Calculate the expected number of visits to $O$ until the first return to $I$. \par
c) Calculate the expected number of steps until the first visit to $O$. }
\solution{ 8, 1, 10, используйте симметрию куба }


\problem{
Suppose Bob and Nick are playing a game. Bob rolls a die and continues to roll until he gets a 2. He keeps track of all the outcomes of the rolls. He calls the sum of the outcomes of the rolls (including the last outcome 2) $X$. Nick does the same thing as Bob, only he keeps rolling until he gets a 5 and he calls the sum of the outcomes of his rolls (including the last outcome 5) $Y$. At the end of the game, Bob pays Nick $Y$ dollars and Nick pays Bob $X$ dollars. Does one of the boys have an advantage over the other? Which one? }
\solution{ игра справедливая! Перевес в сумме в точности компенсируется последним слагаемым! }


\problem{
Миша должен перебрать ведро яблок (выкинуть гнилые). В ведре 12 яблок, из них 4 гнилых. Миша выбирает яблоки следующим образом: достает по 3 яблока случающим образом, выкидывает гнилые, а нормальные кладет обратно в корзину. Он действует,  таким образом, до тех пор, пока не достанет 3 не гнилых яблока. После этого он идет к дедушке и говорит, что все перебрал. Найдите вероятность того, что дедушка возьмет на проверку гнилое яблоко (дедушка случайным образом берет всего одно яблоко).
source: банк задач первой олимпиады школьников }
\solution{  (a better one?). Вероятность зависит о числа гнилых яблок в ведре. Получаем четыре неизвестных $p_{1}$, $p_{2}$, $p_{3}$, $p_{4}$. Они легко находятся начиная с $p_{1}$. }


\problem{
Миша предстоит перебрать ведро в котором $n$ грибов. Среди грибов одна бледная поганка, остальные хорошие. Миша осматривает грибы по одному, доставая их наугад, и осмотренные откладывает в сторону. Бледную поганку Миша определяет безошибочно. \par
а) Каково математическое ожидание и дисперсия числа осмотренных грибов? \par
б) Каково математическое ожидание числа осмотренных грибов, если бледных поганок две?}
\solution{ }

\problem{
Правильную монетку подбрасывают $n$ раз. Пусть $X_{1}$,...$X_{K}$ - длины последовательных серий. \par
a) (easy) Чему равно $X_{1}+X_{2}+...+X_{K}$? \par
б) Чему равно $E(X_{1}^2+...+X_{K}^{2})$? \par
в) $E(max(X_{1},...,X_{K}))$, asy=? \par
ru-math }
\solution{ $n$, $3n$ (асимптотика), $cln(n)$ }

\problem{
You pay 1\$ to play a game, a fair coin is tossed and if heads you get 3\$, if tails nothing. You start with 50\$ and play as many times as you like.
What is the probability that you will go bankrupt if you keep playing? }
\solution{ }

\problem{
Imagine two people, each of whom tosses a coin repeatedly.
Person A keeps tossing until the first time three successive tosses come down head, tail, head.
Person B tosses until they first get a head followed by two tails.\par
On average, who will have to wait longer? \par
At what probability $p$ (approximately) of a head are the two expectations equal? }

\solution{ 
$E[T_A]=10$\par
$E[T_B]=8$\par
We wanna equate $(1+p-p^3)/(p-p^3)-1$ and $(p^2-3p+3)/(1-p)^2 + 1/p -1$. \par
It simplifies to $p^3 - 2p^2 - p + 1 = 0$, $p \approx 0.5550$ }


\problem{
Монетку подрасывают до выпадения трех орлов подряд. Найдите ожидание и дисперсию числа подбрасываний. }
\solution{ }

\problem{
A fair coin is tossed repeatedly. What is the average number of tosses required to obtain a head? }
\solution{$2$  }

\problem{
A die is rolled repeatedly. On average, how many times must the die be rolled to get a 6? }
\solution{$6$ }

\problem{
A couple decides to keep having children until they have at least one boy and one girl. What is the average number of children they will have? (Assume no twins.) }
\solution{$3$ }

\problem{
I alternately toss a fair coin and throw a fair die until I either toss a head or throw a 2. If I toss the coin first, what is the probability that I throw a 2 before I toss a head?  }
\solution{$1/7$ }

\problem{
A fair coin is tossed repeatedly. What is the average number of tosses required to obtain two heads in a row? }
\solution{ $6$ }
 
\problem{
An apple is located at vertex $A$ of pentagon $ABCDE$, and a worm is located two vertices away, at $C$. Every day the worm crawls with equal probability to one of the two adjacent vertices. When it reaches vertex $A$, it stops to dine. What is the average number of days until dinner? }
\solution{ }

\problem{
Five people stand at the vertices of a pentagon, throwing frisbees at each other. They have two frisbees, initially at adjacent vertices. At each step, each frisbee is thrown either to the left or to the right with equal probability. This process continues until one person is the target of two frisbees simultaneously; then the game stops. What is the average number of steps taken? }
\solution{ }

\problem{
Three men, $A$, $B$, and $C$, are to fight a pistol duel (truel?). $A$'s chance of hitting a target is $0.3$, $B$'s chance is $0.5$, and $C$ never misses. They are to fire at their choice of target in succession in the order $A$, $B$, $C$, cyclically (but a hit man loses further turns and is no longer shot at) until one man is left. What should $A$'s strategy be? }
\solution{ }

\problem{
Erica flips a fair coin. Nate then flips the coin and wins if it matches Erica's flip. If it doesn't match, then Noah flips the coin and wins if it matches Nate's flip. If Noah doesn't win, then Erica flips and wins if it matches Noah's flip. The game continues until there is a winner. What is the probability that Nate wins the game?}
\solution{ }

\problem{
Two persons, $X$ and $Y$, play with a die an unlimited number of times. $X$ wins a game if the outcome is 1 or 2; $Y$ wins the game in the other cases. A player wins the match if he wins two consecutive games. Determine the probability that $X$ wins the match. }
\solution{ }

\problem{
Daniel and Scott are playing a game where a player wins as soon as he has two points more than his opponent. Both players start at zero, and points are earned one at a time. If Daniel has a $0.6$ chance of winning each point, what is the probability that he will win the game? }
\solution{ }
\problem{
Coupons are each randomly numbered 1 to 5. All five numbers are required to win a prize. On average, how many coupons must be taken to win?  }
\solution{ }
\problem{
Alice and Bob flip a coin until either HHT or HTT occurs. Alice wins if the pattern HHT comes first; Bob wins if HTT comes first. What is the probability that Alice will win? }
\solution{ }
\problem{
A fair coin is tossed repeatedly. What is the probability that the first occurrence of three heads in a row is earlier than the first occurrence of two tails in a row? }
\solution{ }
\problem{
A fair coin is tossed repeatedly. What is the probability that the first occurrence of HH will occur before the first occurrence of THTH? }
\solution{ }
\problem{
A fair coin is tossed repeatedly. What is the average number of tosses required to obtain the sequence THTTH? }
\solution{ }
\problem{
Player $A$ has 1 dollar, and Player $B$ has 2 dollars. Each play gives one of the players 1 dollar from the other. Player $A$ is enough better than Player $B$ that he wins 70\% of the plays. They play until one is bankrupt. What is the chance that Player  $A$ wins? }
\solution{ }
\problem{
Two players, $A$ and $B$, start with $a$ and $b$ dollars, respectively. A fair coin is repeatedly tossed. If the result is heads, then $A$ pays $B$ one dollar. Otherwise, $B$ pays $A$ one dollar. The game ends when one player goes broke. What is the probability that $A$ wins? }
\solution{ }
\problem{
A bug is on a number line, starting at 0. A fair coin is tossed repeatedly. If the coin comes up heads, the bug goes 1 unit left; otherwise, the bug goes 1 unit right. What is the probability that the bug eventually returns to 0? }
\solution{ }

\problem{
From where the drunken man stands, one step toward the cliff would send him over the edge. He takes random steps, either toward or away from the cliff. At any step, his probability of taking a step away is $\frac{2}{3}$, of a step toward the cliff $\frac{1}{3}$. What is his chance of falling over the edge? }
\solution{$0.5$ }

\problem{
A microbe either splits into two perfect copies of itself or dies. If the probability of splitting is $\frac{2}{3}$ (and is independent of the microbe's ancestry), what is the probability that a microbes's descendants die out? }
\solution{$0.5$ }

\problem{
Two players, $A$ and $B$, repeatedly play a game of chance. At each turn, either $A$ wins 1 dollar from B with probability $0.51$, or $A$ loses 1 dollar to $B$ with probability $0.49$. Both $A$ and $B$ have unlimited capital. What is $A$'s expected peak cumulative loss? }
\solution{ }
\problem{
A particle moves at each step 2 units to the right or 1 unit to the left, with probability $0.5$ each. The particle starts at 1. What is the probability that the particle will ever reach the origin? }
\solution{ }
\problem{
A coin is tossed repeatedly. What is the probability that at some point, the number of tails will be greater than twice the number of heads? }
\solution{ }
\problem{
Real numbers are chosen at random from the unit interval $(0;1)$. If after choosing the $N$-th number the sum of the numbers so chosen first exceeds 1, what is the expected value of $N$? }
\solution{ }
\problem{
Real numbers are chosen at random from the unit interval $(0;2)$. If after choosing the $N$-th number the sum of the numbers so chosen first exceeds 2, what is the expected value of $N$? \par

Source: previous 26 problems, aops, t=38284 }
\solution{ }
\problem{
A fair coin is tossed repeatedly. Find the probability of obtaining five consecutive heads before two consecutive tails. }
\solution{ }
\problem{
A and B play a series of games. Each game is independently won by A with probability $p$ and by B with probability $(1-p)$. They stop when the total number of wins of one of the players is two greater than that of the other player. The player with the greater number of total wins is declared the winner. Find the probability that A is the match winner. \par
Source: aops, t=133655 }
\solution{ }
\problem{
A and B are having a shooting contest. Each of them have 50
bullets. For each bullet, both A and B have 50\% probability of hitting the
target. After they finish shooting, the total numbers of bullets
that hit the target are counted. If A hit the target more than B,
A wins. If they hit the target the same number of times, it's a
tie. Otherwise, B wins. Apparently if A and B both get 50 shots, they
have the same probability of winning. The question is: what is
the probability of A winning if A gets 51 shots and B gets 50 shots? }
\solution{ 
1/2. \par
One way of seeing this is as follows: first look at what happens after both have done 50 shots. Either A is ahead (with probability, say $p$), or B is ahead (which by symmetry, is also equal to p), or both are tied (with probability, say, $q$). So obviously, $2p+q=1$. Now what are the ways A can win after 51 shots? He can win if he's already ahead after 50 shots (which happens with probability $p$) and then the outcome of the 51st shot is irrelevant, or if he's tied after 50 shots and has a success on the 51st shot (which occurs with probability $q\cdot 1/2$). (If A is behind B after 50, he either stays behind or at best ties at the 51st, and if A and B are tied at 50, then they still stay tied if A misses his 51st shot.) Thus, the probability of A winning is $p+q/2=1/2$. \par
If you know some advanced probability theory, then a simpler argument, called coupling, says that without loss of generality, you can assume that A and B have exactly the same hits and misses during their first 50 shots. Then A wins if and only if he succeeds on his 51st shot, which occurs with probability $1/2$.\par
Note that 50 isn't special here; this works for any n and n+1.\par
Another simplified version of this problem: If A flips $n+1$ and B flips $n$ fair coins, what that the probability that A gets more heads than B is $1/2$. }

\problem{
Имеется $n$ неправильных монеток. Монетка номер $k$ выпадает орлом с вероятностью $p_{k}=\frac{1}{2k+1}$. Все $n$ монеток подбрасываются одновременно. Какова вероятность, что выпадет нечетное число орлов? }
\solution{ 
$P_n = P_{n-1}\cdot \frac{2n-1}{2n+1} + \frac{1}{2n+1}$ и $P_n =\frac{n}{2n+1}$}

\problem{ Вариация к началу 2008 года \par
Дед Мороз развешивает новогодние гирлянды. Аллея состоит из 2008 елок. Каждой гирляндой Дед Мороз соединяет две елки (не обязательно соседние). В результате Дед Мороз повесил 1004 гирлянды и все елки оказались украшенными. Какова вероятность того, что существует хотя бы одна гирлянда, пересекающаяся с каждой из других? \par
Например, гирлянда 5-123 (гирлянда соединяющая 5-ую и 123-ю елки) пересекает гирлянду 37-78 и гирлянду 110-318.
Например, на рисунке: \par
(picture here) \par
зеленая гирлянда пересекается с красной и с синей, а красная и синяя не пересекаются }
\solution{ }
\problem{
Имеется $n$ монеток. Каждая выпадает орлом с вероятностью $p$. В первом раунде подкидывают все монетки. Во втором раунде подкидывают те монетки, которые не выпали орлом в первом раунде. В третьем раунде подкидывают те монетки, которые не выпали орлом ни в первом, ни во втором раунде. И т.д. до тех пор, пока будут монетки не выпадавшие орлом ни разу. Пусть $X$ - число монет, подброшенных в последнем раунде. \par
а) Как связаны $E(X)$ и $P(X=1)$? \par
б) Существует ли предел $\lim_{n\to\infty}P(X=1)$? }
\solution{ }
\problem{
В озере сейчас живет одна амеба. Каждую минуту каждая амеба равновероятно превращается в 0, 1, 2 или 3 амебы. \par
а) Какова вероятность того, что популяция амеб в озере рано или поздно вымрет? \par
б) Сколько в среднем амеб в озере будет через $n$ минут? }
\solution{ 
а) Уравнение методом первого шага, корень $p=\sqrt{2}-1$. \par
б) $1.5^{n}$ }

\problem{ Убийцы и мирные граждане \par
Вы приехали в уездный город $N$. В городе кроме Вас живут $M$ мирных граждан и $U$ убийц. Каждый день на улице случайным образом встречаются два человека. Если встречаются два мирных гражданина, то они пожимают друг другу руки. Если встречаются мирный гражданин и убийца, то убийца убивает мирного гражданина. Если встречаются двое убийц, то оба погибают. \par
Каковы Ваши шансы выжить в этом городе? Зависят ли они от Вашей стратегии?  }
\solution{ 
Если $U$ нечетно, то шансы выжить равны нулю. Если $U$ четно, то шансы выжить равны $\frac{1}{U+1}$ }

\problem{
Правильная монетка подбрасывается бесконечное количество раз. Какова вероятность того, что $h$ орлов подряд появятся раньше, чем $t$ решек подряд? }
\solution{ 
First step, $\frac{2^{t}-1}{2^{t}+2^{h}-2}$ \par
Может какое устное решение? Ибо красивая формула? }

\problem{
В углу доски стоит шахматная фигура. Мы двигаем ее наугад (согласно правилам хода для этой фигуры). Сколько в среднем пройдет ходов прежде чем она вернется в исходную клетку, если эта фигура: \par
а) король \par
б) конь \par
в) ладья \par
г) слон \par
д) ферьзь \par
Подсказка: находим стационарное распределение, обращаем вероятность }
\solution{ б) 168  }

\problem{ A coin, which lands on heads with probability $p$ is continually flipped. Compute the expected number of flips required to get a string of $r$ heads in a row. }
\solution{ }

\problem{
Пусть $G$ свободная группа, порожденная двумя элементами $a$ и
$b$. Т.е. $G$ состоит из всех 'слов' конечной длины, состоящих из
'букв' $a$, $b$, $a^{-1}$, $b^{-1}$. 'Буквы' $a$ и $a^{-1}$, также
как и 'буквы' $b$ и $b^{-1}$ сокращаются. Имеется особое
'слово'-'единица' $e$, $e\cdot a=a$ etc. В начальный момент
времени написано 'слово' $e$. В каждый следующий момент справа
дописывается одна из четырех 'букв' (производится сокращение, если
это возможно; буквы не переставляются). \par
а) Какова вероятность того, что когда-нибудь снова будет написано
$e$? \par
б) Каков (as) предел средней длины слова? }
\solution{
а) $P=p+(1-p)P^{2}$ \par
б) $\frac{1}{2}$ }

\problem{
Пусть $S_{n}$ - симметричное случайное блуждание. Т.е. $S_{0}=0$,
$S_{n}=S_{n-1}+X_{n}$, где $X_{n}$ - iid,
$P(X_{n}=1)=P(X_{n}=-1)=0.5$. \par
а) Найдите $\lim_{m\to \infty}\frac{P(S_{2m}=2r)}{P(S_{2m}=0)}$ \par
б) Найдите $\lim_{m\to \infty}P(S_{2m}=2r)$, $\lim_{n\to \infty}P(S_{n}\ge k)$ \par
в) Запись $k|m$ означает, что $k$ делит $m$ ($m$ делится на $k$).
Найдите $\lim_{n\to \infty}P(2|S_{n})$ и $\lim_{n\to
\infty}P(3|S_{n})$ }
\solution{ 
а) 1 \par
б) Исходя из п. а, получаем 0 (можно воспользоваться ф.
Стирлинга); $P(S_{n}\ge 1)=P(S_{n}\le -1)
\rightarrow 0.5$ и $P(S_{n}\ge k)\rightarrow 0.5$ \par
в) $\lim_{n\to \infty}P(2|S_{n})$ не существует; \par
Введем величины: \par
$x_{n}=P(3|S_{n})$, $y_{n}=P(3|S_{n}+1)$ и $z_{n}=P(3|S_{n}+2)$ \par
Тогда: $x_{0}=1$, $y_{0}=0$, $z_{0}=0$ \par
$x_{n}=0.5y_{n-1}+0.5z_{n-1}$ \par
$y_{n}=0.5x_{n-1}+0.5z_{n-1}$ \par
$z_{n}=0.5x_{n-1}+0.5y_{n-1}$ \par
Решая систему разностных уравнений ($\lambda_{1}=-0.5$,
$\lambda_{2}=-0.5$, $\lambda_{3}=1$, двум кратным корням
соответствует два линейно независимых собственных вектора)
находим, что $x_{n}\rightarrow 1/3$.}


\problem{
Пусть $S_{n}$ - несимметричное случайное блуждание. Т.е.
$S_{0}=0$, $S_{n}=S_{n-1}+X_{n}$, где $X_{n}$ - iid,
$P(X_{n}=1)=p=1-P(X_{n}=-1)$. \par
а) Найдите вероятность того, что случайное блуждание когда-нибудь
вернется в исходную точку \par
б) Пусть $p>0.5$. Найдите ожидаемое время выигрыша $r$ рублей. \par
}
\solution{ }
\problem{
$[$Steele, 1.1.$]$ \par
Корпорация стремится поддерживать курс своих акций на уровне
20\$. Поэтому вероятности изменения курса выглядят так: \par
$P(Y_{n+1}=21|Y_{n}=20)=0.9$ и $P(Y_{n+1}=19|Y_{n}=20)=0.1$ \par
При $k>20$: $P(Y_{n+1}=k+1|Y_{n}=k)=1/3$ и $P(Y_{n+1}=k-1|Y_{n}=k)=2/3$ \par
При $k<20$: $P(Y_{n+1}=k+1|Y_{n}=k)=2/3$ и $P(Y_{n+1}=k-1|Y_{n}=k)=1/3$ \par
Найдите ожидаемое время падения курса акций с 25\$ до 18\$. }
\solution{ 
$x_{18}=0$ \par
$x_{19}=2/3x_{20}+1/3x_{18}+1$ \par
$x_{20}=0.1x_{19}+0.9x_{21}+1$ \par
$x_{21}=2/3x_{20}+1/3x_{22}+1$ \par
$x_{22}-x_{20}=2(x_{21}-x_{20})$ \par
$x_{25}-x_{20}=5(x_{21}-x_{20})$ \par
Решая, получаем: $x_{19}=77$, $x_{20}=114$, ,
$x_{21}=117$, $x_{22}=120$, $x_{25}=129$ }


\problem{
Для лотерии выпущено 10000 билетов, из них 100 билетов являются призовыми и еще 200 дают право получить еще один билет. Какова вероятность получить приз, если купить один билет? }
\solution{  $\frac{100}{10000-200}$ }

\problem{
I roll 100 standard dice. I get one point for each die for which
the number of dots on the top face is greater than two. What is
the probability that the number of points I get is...\par
a) Even? b) Divisible by three? c) =1 (mod 3)?\par
source: aops, t=111454}
\solution{ }
\problem{ breaking sticks\par
Take a stick and break it at a location selected with uniform density along its length. Throw away the left-hand piece and break the right-hand one at a location selected with uniform density along its length. Continue forever. What is the probability that one of the discarded left-hand pieces is more than half as long as the original stick? \par
Source: aops, t=152328 }
\solution{ 
The answer appears to be $\log 2$. \par
A possible proof could go as follows: Let $X_{1}$,$X_{2}$,$\ldots$ random variables with values
in $[0;1]$ such that $X_{n}$ is distributed uniformly on $[0,X_{n-1}]$. We have to calculate
$p : = \sum_{n}P(X_{n}\ge 1/2)$. \par
Using the recursion
$P(X_{n+1}\ge c) = \int_{0}^{1-c}dy P(X_{n}\ge c/(1-y)) = \int_{c}^{1}dz \frac c{z^{2}}P(X_{n}\ge z)$
it is easy to prove inductively that
$P(X_{n}\ge c) = 1-c \sum_{k < n}\frac{(-\log c)^{k}}{k!}= c \sum_{k \ge n}\frac{(-\log c)^{k}}{k!}$
and thus interchanging the sums over n and k we get
$p = c (-\log c) \sum_{k \ge 0}\frac{(-\log c)^{k}}{k!}=-\log c$
where $c = 1/2$.}


\problem{
Alfred and Bonnie play a game in which they take turns tossing a
fair coin. The winner of a game is the first person to obtain a
head. Alfred and Bonnie play this game several times with the
stipulation that the loser of a game goes first in the next game.
Suppose that Alfred goes first in the first game, and what is the
probability that he wins the sixth game? \par
source: aops, t=80995 }
\solution{ 
Вероятность выиграть, если ходить первым: $p=0.5+0.5\cdot 0.5\cdot p$; $p=2/3$ \par
Пусть $p_{n}$ - вероятность выиграть $n$-ую партию (для А). \par
$p_{n}=2/3\cdot (1-p_{n-1})+1/3\cdot p_{n-1}$ \par
$p_{n}=2/3-1/3\cdot p_{n-1}$ и $p_{1}=1$ }


\problem{ Consider a random walk on a fractal lattice (see attached picture):\par
Картинка: из центра 4 направления. Далее каждое направление делится на три, потом еще на три и т.д. до бесконечности... \par
Particle starts at origin O. \par
At every time-step it moves in one of four directions, north south east or west, with equal probability.\par
The north-south generator does not commute with the east-west generator -- e.g. {east, north} <> {north, east}\par
Clearly the only way the particle can revisit a particular point is by retracing its steps.\par
Two questions: \par
1. What is the probability that the particle ever hits the point A? \par
2. What is the probability of hitting A before B?}
\solution{ }
\problem{ Google page-rank \par
Поисковик google рассчитывает рейтинг страницы (pagerank) по следующему алгоритму: предполагается, что если на странице есть $n$ ссылок, то пользователь с вероятностью 85\% уходит на одну из этих ссылок (выбирая саму ссылку равновероятно) и с вероятностью 15\% уходит на случайно выбираемую страницу. Рейтинг страницы - это вероятность того, что после длительного блуждания пользователь окажется на данной странице. \par
Рассчитайте рейтинг для следующей сети \par
(картинка) \par
ссылка: википедия }
\solution{ }
\problem{ Выборы \par
В выборах участвуют два кандидата. В начале выборов каждый из них голосует сам за себя. Затем каждый из миллиона жителей по очереди голосует за одного из кандидатов. При этом если за кандидата А было подано $n$ голосов, а за кандидата В - $m$ голосов, то вероятность того, что очередной избиратель будет голосовать за А равна $\frac{n}{n+m}$. \par
Какова вероятность того, что за кандидата А проголосует менее 20\% жителей? \par
source: Prof. Vazirani’s Problems }

\solution{ 
Процедуру голосования можно представить себе так: \par
Положим на прямую красный шар. Затем по одному будем класть белые шары (миллион штук). Причем класть их будем равновероятно на любое место между уже положенными шарами, или с любого края полоски. \par
В результате положение красного шара (отделяющего голоса за разных кандидатов) распределено равновероятно. \par
Ответ: $\frac{0.2\cdot 10^{6}}{10^{6}+1}\approx 0.2$ }

\problem{
В коробке лежат 30 зеленых и 50 красных шаров. Мы извлекаем два наугад, один берем в левую руку, другой - в правую. Шар в левой руке красим в цвет шара, находящегося в правой руке. Затем возвращаем шары в коробку. Снова извлекаем два наугад и т.д. до тех пор пока в коробке все шары не окрасятся в один цвет. \par
а) Какова вероятность того, что шары будут окрашены в красный цвет? \par
б) Сколько в среднем пар шаров будет извлечено? }
\solution{ 
а) разностное уравнение с н.у., решение $p(k)=k/n$, где $k$ - число красных шаров, $n$ - общее число шаров, может что по проще? \par
решение 2: $K_{t}$ - число красных шаров - мартингал, \par
если $T$ момент остановки, то $E(K_{T})=k_{0}$, значит $P(K_{T}=n)\cdot n=k_{0}$. }

\problem{
You have a black box with $N$ balls in it - each of a different color.
Suppose you take turns as follows - randomly pick a ball in each hand,
and paint the left hand ball the same color as the right hand ball.
You replace both balls in the box before the next turn. \par
How many turns do you expect before all balls are the same color? \par
Solution: \par

source: Ariel Landau in sci.math in September 1995 \par
solution: Lew Mammel, Jr.,  Robert Israel }
\solution{ Пусть $ A_{i} $ --- это событие, состоящее в том, что в конце <<победил>> цвет $ i $, а $ X_{i} $ --- начальное количество шаров цвета $ i $. Тогда $ E(T|X_{1},...,X_{n})=\sum E(1_{A_{i}}T|X_{1},...,X_{n}) $. Заметим, что $ E(1_{A_{i}}T|X_{1},...,X_{n})=E(1_{A_{i}}T|X_{i}) $. Введем обозначение $ f(k)=E(1_{A_{i}}T|X_{i}=k) $. Тогда получается, что $ E(T|X_{1},...,X_{n})=f(X_{1})+...+f(X_{n}) $.

Методом первого шага получаем уравнение на $ f(k) $:

\[ E(1_{A_{i}}T|X_{i}=k)=dE(1_{A_{i}}(T+1)|X_{i}=k+1)+dE(1_{A_{i}}(T+1)|X_{i}=k-1)+(1-2d)E(1_{A_{i}}(T+1)|X_{i}=k) \]

Отдельно убеждаемся в том, что $ P(A_{i}|X_{i}=k)=k/n $ (например, через теорему Дуба). И получаем:

\[ f(k+1)-2f(k)+f(k-1)=\frac{n-1}{k-1} \]

Начальные условия, $ f(0)=f(n)=0 $.

В частности, $ f(1)=(n-1)^2/n $.

}
\problem{ balls and urns \par
There are $n$ urns. Each urn has $a$ white balls and $b$ black balls. \par
We extract at random a ball from the first urn and put it into the second one. We extract at random a ball from the second urn and put it into the third one. And so on. \par
At the last step we extract a ball from the last urn. what is the probability of the last ball extracted to be white? \par
source: aops, april 08 }

\solution{ вероятность не меняется после одного шага, значит она постоянна $\frac{a}{a+b}$ }


\problem{
У Вас на руках $n$ игральных карт, часть из них черные, часть - красные. Вы выбираете одну из них наугад и заменяете на карту другого цвета. Затем снова выбираете одну наугад и снова заменяете ее на карту другого цвета. И так далее до тех пор, пока все карты на руках не станут одноцветными. Найдите закон распределения числа замен и ожидаемое число замен, если: \par
а) изначально на руках 2 красных и 2 черных карты \par
б) изначально на руках 3 черных и 3 красных карты }
\solution{ 
а) Рисуем граф: $(2:2)\to(3:1)$, $(3:1)\to(4:0)$ с вероятностью $0.25$ и $(3:1)\to(2:2)$ с вероятностью $0.75$. \par
Чтобы получить один цвет за $n$ замен нужно $\frac{n-2}{2}$ раз пройти цикл $(2:2)\to(3:1)\to(2:2)$ и дополнительно сделать шаги $(2:2)\to(3:1)\to(4:0)$. \par
Отсюда: $p_{n}=\left(\frac{3}{4}\right)^{\frac{n}{2}-1}\frac{1}{4}$ \par
(геометрическое распределение для четных $n$) \par
б) Также рисуем граф. \par
Из состояние $(4:2)$ за пару ходов можно либо вернуться в $(4:2)$, либо перейти в $(6:0)$. Далее аналогично. }

\problem{
Картинку в студию: квадрат $ABCD$, центр $E$. Нарисованы все стороны квадрата и все отрезки, соединяющие $E$ с вершинами. Жук начинает в точке $E$. Из $E$ жук равновероятно идет в любую точку ($A$, $B$, $C$ и $D$). Если на $n$-ом ходу жук находится в вершине квадрата, то он с вероятностью $\frac{n-1}{n}$ идет в центр и с вероятностью $\frac{1}{2n}$ в каждую из двух соседних вершин. \par
а) Какова вероятность того, что ровно через $n$ ходов он снова будет в $E$? \par
б) Какова вероятность того, что ровно через $n$ ходов он впервые вернется в $E$? \par
в) Каково математическое ожидание времени возвращения в $E$? \par
source: wilmott, bt, 63190 }
%Consider a square ABCD along with its two diagonals intersecting in point E. A flea %hops from any one of the points A, B, C, D, E to an adjacent point. For example, from %point A, The flea hops to D or B or E, but not to C. Assume that the flea is %originally positioned at point E. The probability of the flea hopping on the n-th hop %from any of the points A, B, C, D to point E is (n-1)/n. For example, if the flea, %originally positioned at E, has hopped from E to A to B to A to D, then the %probability that it will now (ON THE 5-TH HOP) hop to point E is 4/5, and that it will %hop to another adjacent point is 1/5. Starting from point E, compute the expected %number of hops the flea does before it returns to point E.
\solution{ $e$}

\problem{
Игральный кубик с $n$ гранями подбрасывают до тех пор, пока каждый следующий бросок больше предыдущего. Если какой-то результат меньше либо равен предыдущему, то подбрасывания прекращаются. Игра прекращается на первом ходу, если при первом подбрасывании выпало число меньше либо равное $k$. \par
а) Сколько в среднем будет подбрасываний? \par
б) Допустим $k=0$ (т.е. на первом броске игра никогда не заканчивается). Чему равен предел среднего числа подбрасываний при бесконечном $n$? \par
в) Чему равно среднее значение последнего результата? \par
г) Чему равно среднее значение суммы подбрасываний? \par
Коммент: в и г не проверялись на компактность ответа \par
Source: wilmott, bt, threadid=63237 }

%Keep rolling an $n$-sided fair die until you get a number that is less than or equal %to a previously rolled number, at which point in time the game ends. The game ends on %the first roll if the rolled number is less than or equal to $k$. On average, how many
%times do you roll before the game ends? What happens as n gets unboundedly large? \par
\solution{ 
Let E(k,n) denote the expectation of the number of rolls before the game ends. We are, of course, particularly interested in finding E(0,n).

We roll at least once. Suppose number j comes up. If j is in the set {1, 2, …, k} , then the game ends. This happens with a probability of k/n. If j is in the set {k+1, K+2, …, n}, then the game continues. The probability of j being any one of the numbers in the set {k+1, K+2, …, n} is 1/n.

Imagine j is one of the numbers in the set {k+1, K+2, …, n}. Before you roll again, you ask yourself “starting from this point in time, on average, how many times do I roll before the game ends?” This is basically the same question as E(k,n), with the difference that instead of k, we now have j. So, the answer is E(j,n).

We can now express E(k,n) in terms of E(j,n)’s. Here’s how:

E(k,n) = 1 + (k/n)*[The first roll j<=k] + sigma{[(1/n)*E(j,n)] as j runs from k+1 through n}.

Of course [The first roll j<=k]tells us that in this event, there is no more rolls as the game has already ended. So,

(*1): E(k,n) = 1 + (1/n)*sigma{[E(j,n)] as j runs from k+1 through n}.

So,

(*2): sigma{[E(j,n)] as j runs from k+1 through n} = n*E(k,n) – n.

The preceding sum begins with j=k+1. So, writing a new sum in which j begins with k+2, we get:

(*3): sigma{[E(j,n)] as j runs from k+2 through n} = n*E(k+1,n) – n.

Now we will ‘break up’ (*1):

E(k,n) = 1 + (1/n)*(E(k+1,n) + sigma{[E(j,n)] as j runs from k+2 through n}), which in conjunction with (*3) becomes:

E(k,n) = 1 + (1/n)*( E(k+1,n) + (n*E(k+1,n) – n)), which, when simplified, becomes:

E(k,n) = ((n+1)/n)*E(k+1,n).

Or equivalently,

(*4): $E(k,n) = (n/(n+1))*E(k-1,n)$.

Iterating (*4), gives us:

(*5): $E(k,n) = {(n/(n+1))^k}*E(0,n)$.

Now, recall that k can be any number in the set {0, 1, 2, …, n}. So, for k=n, we do know already that E(n,n) =1. So, for k=n, (*5) gives us:

$1 = E(n,n) ={(n/(n+1))^n}*E(0,n)$. From which we get

$E(0,n) = (1+1/n)^n$.}

\problem{
A and B are to play a game. A third player, N constantly throws 2 dices.
Player A wins if N rolls '12'. Player B wins if there are 2 consecutive 7 rolled by N.\par
What is the probability that A wins?}
\solution{ }

\problem{
На даче у мистера А две входных двери. Сейчас у каждой двери стоит две пары ботинок. Перед каждой прогулкой он выбирает наугад одну из дверей для выхода из дома и надевает пару ботинок, стоящую у выбранной двери. Возвращаясь с прогулки мистер А случайным образом выбирает дверь, через которую он попадет в дом и снимает ботинки рядом с этой дверью. Сколько прогулок мистер А в среднем совершит, прежде чем обнаружит, что у выбранной им для выхода из дома двери не осталось ботинок? \par
Источник: American Mathematical Monthly, problem E3043, (1984, p.310; 1987, p.79)}
\solution{ 12}

\problem{
У Мистера Х есть $n$ зонтиков. Зонтики мистер Х хранит дома и на работе. Каждый день утром мистер Х едет на работу, а каждый день вечером - возвращается домой. При этом каждый раз дождь идет с вероятностью 0.8 независимо от прошлого, (т.е. утром дождь идет с вероятностью 0.8 и вечером дождь идет с вероятностью 0.8 вне зависимости от того, что было утром). Если идет дождь и есть доступный зонтик, то мистер Х обязательно возьмет его в дорогу. Если дождя нет, то мистер Х поедет без зонтика. \par
Какой процент поездок окажется для мистера Х неудачными (т.е. будет идти дождь, а зонта не будет) в долгосрочном периоде? }
\solution{ 
Пусть $p_{i}$ - вероятность того, что в очередном месте (на работе или дома) окажется $i$ зонтиков. \par
Получаем соотношения: \par
$p_{0}=0.2p_{n}$ (1) \par
$p_{i}=0.2p_{n-i}+0.8p(n-i+1)$, $0<i<n$ (2) \par
$p_{n}=p_{0}+0.8p_{1}$ (3) \par
Также, конечно, $\sum p_{i}=1$. (4) \par
Рассмотрим систему из $(n-1)$ уравнения типа (2). Если немного пофантизировать, то можно заметить, что они равновильны системе $p_{1}=p_{2}=...=p_{n}$. \par
Пользуясь (3) и (4) получаем $p_{n}(0.2+n)=1$. \par
Значит $p_{0}=\frac{1}{n+0.2}$. Нас интересует $0.8p_{0}$. }

\problem{
You are many squares or units away from your destination. Let n be the number of squares involved. You roll a pair of dice repeatedly until you reach (by exact count) or pass that target square. Let P(n) = the probability of landing on said target square in the process. What is the limit of P(n) as n approaches infinity? 
source: AMATYC I-1 by Dave Matson }
\solution {}

\problem{Two teams, A and B will play a series of games. The probability of A winning each game is p. The overall winner is the first team to have won two more games than the other.

a) Find the propbability that team A is the overall winner

b)Find the expected number of games played.

source: aops, t=287021, Ross, chapter 2, Problem 49, Kent Merryfield}
\solution{
Let $P$ be the conditional probability that $A$ eventually wins, given that they are tied. (Hence the overall probability of $A$ winning starting from the beginning will be $P$.) Let $P_1$ be the probability that $A$ wins, given that $A$ has the advantage, and let $P_2$ be the probability that $A$ wins, given that B has the advantage.

From the tied state, we move to the state of $A$ having the advantage with probability $p$ and to the state of $B$ having the advantage with probability $1 - p$. That gives us
$P = pP_1 + (1 - p)P_2$
Similar consideration of what happens from the advantage-A state and the advantage-B state gives rise to two more equations:
$P_1 = p + (1 - p)P$ 

$P_2 = pP + (1 - p)\cdot 0$

In matrix form, we have this system:
$\left[\begin{array}{ccc|c}1 & - p & p - 1 & 0 \\ p - 1 & 1 & 0 & p \\ - p & 0 & 1 & 0\end{array}\right]$
Solving it (off stage), we get that $P = \frac {p^2}{1 - 2p + 2p^2}$, $P_1 = \frac {p - p^2 + p^3}{1 - 2p+2p^2}$ and $P_2 = \frac {p^3}{1 - 2p + 2p^2}$.

$P$ is the answer we want. We say that A wins the match with probability $\frac {p^2}{1 - 2p + 2p^2}$.

If we put the values $p = 0$,$p = 1$,$p = \frac12$ into this expression, we get the results we should expect. A graph of this function will be sigmoidal, steepest in the middle, and approximately doubling the advantage in a neighborhood of the middle. That is, if p is 51\%, then the probability of A winning the match will be nearly 52\%.

We can solve for the expected time in the same way. Let $T$ be the conditional expectation of the remaining time, given a tied state, and let $T_1$ and $T_2$ be the expected remaining time from the states of advantage-A and advantage-B, respectively. We get the following system of equations:
$T = 1 + pT_1 + (1 - p)T_2$ 

$T_1 = p + (1 - p)(1 + T)$ 

$T_2 = p(1 + T) + (1 - p)$

That gives us $T=\frac{2}{1-2p+2p^2}$ and also $T_1=\frac{3-4p+2p^2}{1-2p+2p^2}$ and $T_2=\frac{1+2p^2}{1-2p+2p^2}$.

$T$ is the answer we want. Starting at the beginning, the expected length of the match is $\frac{2}{1-2p+2p^2}$ games. This has a minimum of 2 games when $p=0$ or $p=1$ (which makes sense) and a maximum value of 4 games when $p=\frac12$}




\problem{Вася подбрасывает монетку до тех пор, пока хотя бы один раз не выпадет орел и хотя бы один раз - решка. Монетка неправильная, орел выпадает с вероятностью $p$. Пусть $X$ - требуемое число подбрасываний. Найдите $P(X=k)$, $E(X)$, $Var(X)$.}
\solution{}


\problem{У Васи есть магический кубик с $n$ гранями. Грани выпадают равновероятно. Магическое свойство состоит в том, что после того как на кубике выпадает любая грань (допустим $k$), то он превращается в магический кубик с $k$ гранями. Сколько подбрасываний будет у кубика прежде чем он превратиться в <<кубик>> с одной гранью?}
\solution{
Let $T(n)$ denote the random variable that counts the number of throws before the originally n-sided die becomes a one-sided die.

Let $E(n)=E[T(n)]$ and let X denote the result of the first throw of the die. So, $P(X=k)=(1/n)$ for $k=1,2,...,n$.

Clearly E(1)=0.

$E(2)=E[T(2)]=E[T(2) | X=1]*P{X=1}+E[T(2) | X=2]*P{X=2}={(1+E(1))+(1+E(2))}*(1/2)$.

So, $2*E(2)=2+E(1)+E(2)$, so $E(2)=2+E(1)=2$ as $E(1)=0$.

That $E(2)=2$ is what we should expect. It's like asking how many times we need to flip a fair coin until we get heads for the first time.

Assume $n>2$.

Now $E(n)=E[T(n)]=SUM{(E[T(n) | X=k]*P{X=k}), [k:1,n]}=(1/n)*SUM{(E[T(n) | X=k]), [k:1,n]}$.

Clearly $E[T(n) | X=k]=1+E[T(k)]=1+E(k)$.

So $E(n)=(1/n)*SUM{(1+E(k)), [k:1,n]}$.

Simplifying we get $n*E(n)=n+SUM{(E(k)), [k:1,(n-1)]}+E(n)$.

So, $(n-1)*E(n)=n+SUM{(E(k)), [k:1,(n-1)]}$. ((EQ1))

Reducing n in ((EQ1)) by 1, we get $(n-2)*E(n-1)=(n-1)+SUM{(E(k)), [k:1,(n-2)]}$. ((EQ2))

Rewrite ((EQ1)) as $(n-1)*E(n)=1+(n-1)+SUM{(E(k)), [k:1,(n-2)]}+E(n-1)$. ((EQ3))

((EQ2)) and ((EQ3)) together imply $(n-1)*E(n)=1+(n-2)*E(n-1)+E(n-1)=1+(n-1)*E(n-1)$.

So, $E(n)=(1/(n-1))+E(n-1)$. Thus, $E(n)=2+SUM{(1/k), [k:2,(n-1)]}$, along with E(1)=0 and E(2)=2.}
\source{wilmott, bt,t=76442 }

\problem{
Я стою в точке $x=0$. Затем подбрасываю кубик и делаю соответствующее количество шагов вправо. Потом снова подбрасываю кубик и снова иду вправо. И т.д. В какой точке должен стоять мой друг, чтобы вероятность нашей встречи была максимальна? Какова вероятность вероятность встречи если друг стоит очень далеко? }
\solution{В точке 6. Простым подсчетом доказывает, что сначала (до 6) вероятность растет. Затем вероятности для 7-12 меньше, чем для $x=6$. А далее по индукции. Заметим, что $p_{k}=\frac{1}{6}(p_{k-6}+p_{k-5}+...+p_{k-1})$. Интуитивно. Если сделать $n$ шагов, то мы посетим $n$ чисел, и в среднем пройдем расстояние $3.5n$. Значит доля посещенных клеток равна $1/3.5=2/7$. }

\problem{Упражнение на разностные уравнения

Cтрана движется по прямой. Впереди в $a$ шагах находится Светлое Будущее, а позади (в $b$ шагах) --- Темное Прошлое. Каждый год в стране происходят изменения, в результате которых страна движется либо на шаг вперед, либо на два шага назад. При достижении страной Светлого Будущего или Темного Прошлого история заканчивается и страна остается там, где она остановилась. Пусть вероятность продвижения страны за год на шаг вперед составляет $p=\frac{3}{7}$.
\begin{enumerate}
\item Какова вероятность того, что страна придет к Светлому Будущему?
\item Сколько, в среднем, лет пройдет до конца истории?
\end{enumerate}

Источник: Алексей Суздальцев}

\solution{$P(future)=\frac{5\cdot2^b-\left(-\frac{2}{3}\right)^b-4}{5\cdot2^{a+b}-\left(-\frac{2}{3}\right)^{a+b}-4};$\\
$E(history)=
\left(
  \begin{array}{ccc}
    1 & 1 & 1 \\
  \end{array}
\right)
\left(
  \begin{array}{lll}
     1 & 2^a & (-\frac{2}{3})^a \\
  1 & 2^{-b} & (-\frac{2}{3})^{-b} \\
  1 & 2^{-b-1} & (-\frac{2}{3})^{-b-1} \\
  \end{array}
\right)^{-1}
\left(
  \begin{array}{r}
    -\frac{7}{5}a \\
  \frac{7}{5}b \\
  \frac{7}{5}(b+1) \\
  \end{array}
\right)
$}


\subsection{Разложение в сумму}
% rvs

\problem{ Гипергеометрическое распределение \\
В задачнике $N$ задач. Из них $V$ - Вася умеет решать, а $N-V$ - не умеет. На экзамене предлагается равновероятно выбираемые $n$ задач. Пусть $X$ - число решенных Васей задач на экзамене. \\
а) Найдите $E(X)$ \\
б) Найдите $Var(X)$  }
\solution{ 
$X=X_{1}+...+X_{n}$, $E(X)=n\frac{V}{N}$ \\
$Var(X_{i})=\frac{V(N-V)}{N^{2}}$ \\
$Cov(X_{i},X_{1}+...+X_{N})=0$ \\
$Cov(X_{i},X_{j})=-\frac{Var(X_{i}}{N-1}$ \\
$Var(X)=nVar(X_{i})\frac{N-n}{N-1}$}

\problem{ \label{korreliatsia dlia kubika} $[$Чернова, пример 46$]$ \\
Кубик подбрасывается $n$ раз. Пусть $X_{1}$ -
число выпадений 1, а $X_{6}$ - число выпадений 6. Найдите $Corr(X_{1},X_{6})$ \\
Подсказка: $Cov(X_{1},X_{1}+...+X_{6})$ вам в помощь... }
\solution{$Cov(X_{1},X_{1}+...+X_{6})=0$, т.к. $X_{1}+...+X_{6}=const$ \\
$Corr(X_{1},X_{6})=-\frac{1}{5}$ \\
\ref{karandashi po korobkam} \\
$10\cdot (1-\frac{9}{10}^{7})$  }

\problem{ \label{karandashi po korobkam}
 По 10 коробкам наугад раскладывают 7 карандашей. Каково
ожидаемое (среднее) количество пустых коробок? {\it Подсказка:
представьте результат в виде суммы 10-и случайных величин.}}

\solution{ }
\problem{ $[$Mosteller$]$ Среднее число совпадений \\
Из хорошо перетасованной колоды на стол последовательно
выкладываются карты лицевой стороной наверх, после чего
Аналогичным образом выкладывается вторая колода, так что каждая
карта первой колоды лежит под картой из второй колоды. Каково
среднее  число совпадений   нижней  и верхней  карт?}
\solution{ }

\problem{ Grimmett, 3.3.3. \\
В группе 20 человек. Каждый из них подбрасывает по кубику. Найдите
ожидаемый выигрыш и дисперсию выигрыша группы, если: \\
а) за каждую пару игроков, выкинувших одинаковое количество очков,
группа получает один тугрик \\
б) за каждую пару игроков, выкинувших одинаковое количество очков,
группа получает эту сумму в тугриках }
\solution{ }

\problem{ Coupon's collector problem \\
Внутри упаковки шоколадки <<Веселые животные>> находится наклейка
с изображением одного из 30 животных. Предположим, что все
наклейки равновероятны. Большой приз получит каждый, кто соберет
наклейки всех животных. Какое количество шоколадок в среднем нужно
купить, чтобы выиграть большой приз? }
\solution{ }

\problem{ $[$Mosteller$]$ \\
В $n$ урн случайным образом бросают (один за одним) $k$ шаров.
Найдите математическое ожидание числа пустых урн. }
\solution{ }

\problem{ \label{razorenie}
У Васи $100$ рублей, у Пети - $150$. Они играют в орлянку
правильной монеткой до тех пор, пока все деньги не перейдут к
одному игроку. Какова вероятность, что победит Вася? }
\solution{Пусть $X_{n}$ - благосостояние Васи после $n$-го хода, тогда
$E(X_{n})=100$. $E(X_{final})=250p+0(1-p)$.  }

\problem{
На карточках написаны числа от 1 до $n$. В игре участвуют
$n$ человек. В первом туре каждый получает случайным образом по
одной карточке. Во втором туре карточки выдаются заново. Призы
раздаются по следующему принципу: Человек не получает приз, только
если найдется кто-то другой, кто получил большие числа в каждом
туре.
Каково среднее количество человек, получивших приз? \\
Взято с www.zaba.ru, какая-то олимпиада. }
\solution{ }

\problem{
А.А. Мамонтов сидит в 424 аудитории. Эконометрику
собираются сдавать несколько человек. На поиски пустых аудиторий
послано 3 студента-разведчика. На втором этаже 9 учебных
аудиторий, 5 из них заняты. Каждый из 3 студентов-разведчиков
независимо друг от друга заглядывает в 3 аудитории. Если студент
обнаруживает пустую аудиторию, то он сообщает ее номер А.А.
Мамонтову. Каково среднее
количество обнаруженных пустых аудиторий? }
\solution{ }

\problem{
У Маши 30 разных пар туфель. И она говорит, что мало! Пес
Шарик утащил (без разбору на левые и правые) 17 туфель. Сколько
полных пар в среднем осталось? Сколько полных пар в среднем
досталось Шарику? }
\solution{ }

\problem{
Из колоды в 52 карты извлекается 5 карт. Сколько в среднем
извлекается мастей? Достоинств? Тузов? }
\solution{ Масть: $4\cdot (1-\frac{C_{39}^{5}}{C_{52}^{5}})$ 

Достоинство: $13\cdot (1-\frac{C_{48}^{5}}{C_{52}^{5}})$ 

Туз: $4\cdot \frac{5}{52}$ }

\problem{
Вася пишет друг за другом наугад 100 букв из латинского алфавита. \\
а) Каково ожидаемое количество букв, встречающихся в написанном <<слове>> ровно один раз? \\
б) Как изменилась бы искомая величина, $A_{k,n}$, если бы в алфавите было $k$ букв, а Вася писал бы <<слово>> из $n$ букв? \\
в) Найдите $\lim_{n\to\infty} A_{k,n}$, $\lim_{k\to\infty} A_{k,n}$ }
\solution{ }

\problem{
За круглым столом сидят в случайном порядке $n$ супружеских пар (всего $2n$ человек). Пусть $X$ - число пар, где супруги оказались напротив друг друга. \\
Найдите $E(X)$ и $Var(X)$ }
\solution{ }

\problem{
Suppose there were $m$ married couples, but that $d$ of these 2m people have died. Regard the $d$ deaths as striking the $2m$ people at random. Let $X$ be the number of surviving couples. Find: $E(X)$ and $Var(X)$ }
\solution{ }

\problem{ throwing pies \\
A group of initially N people play the following game. Each one
picks another person at random as a target, and at the voice of <<now!>> they throw
their pies at their selected targets with perfect aim. Each player hit by a pie must
abandon the game; the ones not hit by a pie are called <<survivors>>. They keep playing
until all of them have been hit or only one survivor remains.  \\
a) If at a given stage of the game there are n survivors, what is the expected number of survivors at the next stage? \\
b) If at a given stage of the game there are n survivors what is the
probability of having exactly k survivors at the next stage? \\
c) What is the asymptotic behavior as N tends to infinity of the probability of ending up with one survivor. \\
Generalize the problem assuming that the players’ aim is not perfect. Assume that the probability p of hitting the selected target is constant and the same for everybody.}
\solution{ }

\problem{
Над озером взлетело 20 уток. Каждый из 10 охотников
стреляет в утку по своему выбору. Каково ожидаемое количество
убитых уток, если охотники стреляют без промаха? Как изменится
ответ, если вероятность попадания равна 0,7? Каким будет ожидаемое
количество охотников, попавших в цель? }
\solution{ }

\problem{ $[$9.57, 9.23 Кочетков$]$ \\
Десять человек садится на первом этаже в лифт шестиэтажного
здания. Каждый выходит на случайно выбираемом им этаже (кроме
первого).\\
а) Сколько остановок (в среднем) сделает лифт? \\
Ответьте на следующие вопросы с помощью
$S_{n}(k)=1^{n}+2^{n}+...+k^{n}$: \\
б) До скольки этажей (в среднем) лифт вообще не доедет? \\
в) На каком этаже (в среднем) будет первая остановка? \\
г) На каком этаже (в среднем) будет последняя остановка? \\
д) Сколько этажей (в среднем) лифт <<проскочит>> при подъеме? }
\solution{ }

\problem{Сломанный лифт

Этажи в доме пронумерованы $0,1,2,\ldots k$. В лифт на нулевом этаже садятся $n$ человек. Считаем, что все распределения людей по этажам равновероятны. Лифт сломан и может довезти пассажиров только до первого из вызванных этажей.
\begin{enumerate}
\item До какого, в среднем, этажа доедет лифт?
\item Скольким, в среднем, людям придется идти пешком? Найдите предел этого матожидания при $k\to\infty$ и удостоверьтесь, что ответ согласуется с интуицией.
\end{enumerate}

Источник: Алексей Суздальцев
}

\solution{Лифт доедет в среднем до этажа номер $\frac{1^n+2^n+3^n+\ldots+k^n}{k^n}$. При этом в среднем до своего этажа не доедут
\\$n\left(1-\frac{1^{n-1}+2^{n-1}+3^{n-1}+\ldots+k^{n-1}}{k^n}\right)$ человек. Предел этого выражения равен $n(1-\int_0^1x^{n-1}dx)=n-1$, что вполне согласуется с интуицией: если этажей очень много (а народ все еще распределен по этажам более-менее равномерно), то на первом из вызванных этажей выйдет, скорее всего, только один --- остальные пойдут пешком.}





\problem{ Судьба Дон Жуана \\
У Васи  $n$  знакомых девушек (их всех зовут по-разному). Он пишет
им  $n$  писем, но, по рассеянности, раскладывает их в конверты
наугад. С.в.  $X$  обозначает количество девушек, получивших
письма, написанные лично для них. Найдите  $E(X)$. }
\solution{ Ответ: $E(X)=1$ }

\problem{
Вам предложена следующая игра. Изначально на кону 0 рублей. Раз за разом подбрасывается правильная монетка. Если она выпадает орлом, то казино добавляет на кон 100 рублей. Если монетка выпадает решкой, то все деньги, лежащие на кону, казино забирает себе, а Вы получаете красную карточку. Игра прекращается либо когда Вы получаете третью красную карточку, либо в любой момент времени до этого по Вашему выбору. Если Вы решили остановить игру до получения трех красных карточек, то Ваш выигрыш равен сумме на кону. При получении третьей красной карточки игра заканчивается и Вы не получаете ничего. \\
а) Как выглядит оптимальная стратегия в этой игре? \\
б) Чему при этом будет равен средний выигрыш? }
\solution{ }

\problem{
В каждой из двух урн находится по 50 белых и 50 черных шаров. Вася одновременно вытаскивает по шару из каждой урны и выбрасывает их. \\
Пусть $X$ количество раз, когда из обеих урн были одновременно вытащены белые шары. \\
Найдите $E(X)$, $Var(X)$ }
\solution{ }

\problem{
На карточках написаны числа от 1 до $n$. Вася достает их одну за другой наугад. Если номер карточки является соседним с номером предыдущей карточки, то Вася получает 1 рубль. Пусть $X$ - Васин выигрыш. \\
Найдите $E(X)$, $Var(X)$ }
\solution{ }

\problem{
Вася называет наугад 50 чисел от 1 до 100 (допускаются повторения), а Петя называет наугад 50 чисел от 1 до 100 (без повторов). \\
Пусть $X$ и $Y$ это суммы этих чисел. \\
a) Сравните $E(X)$ и $E(Y)$ \\
б) Сравните $Var(X)$ и $Var(Y)$  }
\solution{ Solution:\\
$E(X)=E(Y)$, $Var(X)>Var(Y)$}

\problem{
Кубик подбрасывается до тех пор, пока каждая грань не
выпадет по разу. Найдите математическое ожидание и дисперсию числа
подбрасываний. }
\solution{ }

\problem{
Правильная монетка подбрасывается  $n$  раз. Серия - это
последовательность подбрасываний из одинаковых результатов (к
примеру, в последовательности ОООРРРО три серии). \\
а) Каково ожидаемое количество серий? \\
б) Дисперсия числа серий? \\
в) А если монетка неправильная и выпадает гербом с вероятностью  $p$? }
\solution{ }

\problem{ Рулет \\
Длинный рулет разрезан на $n$ частей. Каждый из $k$ гостей по очереди забирает себе один кусочек, выбираемый случайным образом. В результате остается $n-k$ кусочков рулета. Оставшиеся кусочки рулета лежат <<сериями>>, разделенными <<дырками>> от забранных кусочков. Каково ожидаемое число <<серий>> оставшихся кусочков? К чему стремится эта величина при $n\to\infty$?\\
Aвтор: Алексей Суздальцев  }
\solution{ Решение 1: \\
$X$ - число <<серий>>, $X=X_{1}+...+X_{n}$, где $X_{i}$ - индикатор, показывающий, начинается ли новая серия с $i$-го кусочка. \\
Ответ: $\frac{n-k}{n}+(n-1)\frac{k}{n}\frac{n-k}{n-1}=(k+1)\frac{n-k}{n}$ \\
Решение 2: \\
Закольцуем рулет, добавив в него еще один кусочек (для хозяина дома - для Алексея Суздальцева). Получаем $(k+1)$ потенциальную серию. Вероятность того, что некая серия непуста, равна $\frac{n-k}{n}$.\\
Ответ, конечно, $(k+1)\frac{n-k}{n}$ }

\problem{
Петя ищет 6 нужных ему книг в стопке из 30 книг. Книги внешне не отличимы. Сколько книг в среднем ему придется пересмотреть? Просмотренные книги Петя в общую кучу не возвращает.  }
\solution{Можно считать, что Петя берет книги подряд из хорошо перетасованной стопки. Соответственно он берет 6 книг и 6 интервалов (книги до 1-ой нужной, книги от 1-ой нужной до 2-ой нужной, и т.д.). Если считать, что средняя длина всех интервалов одинаковая, то получается такой ответ: $6+6\cdot\frac{24}{7}$. \\
Доказательство того, что средняя длина всех интервалов одинаковая: \\
Расположим 30 книг по кругу. Среди этих 30 книг отметим случайным образом 7 книг и занумеруем их (опять же случайным образом) от 1 до 7. Эти 7 книг разбивают круг из книг на 7 частей. В силу симметрии средняя длина каждой части одинакова и равна $\frac{24}{7}$. Будем трактовать книгу номер 1 как разбивающую круг на стопку. А книги 2-7, как нужные Пети. }

\problem{
В здании 10 этажей, на каждом этаже 30 окон. Вечером в каждом окне независимо от других свет включается с вероятностью $p$. \\
a) Чему равно ожидаемое количество <<ноликов>> на фасаде здания? \\
б) Чему равно ожидаемое количество <<крестиков>> на фасаде здания? \\
в) При каких $p$ эти количества максимальны? Минимальны? \\
Примечание: два разных нолика могут иметь общие точки \\
Вставить рисунок нолика и рисунок крестика, пример подсчета }
\solution{ }
\problem{ 
Если смотреть на корпус Ж здания Вышки с Дурасовского переулка, то видно 40 окон (5 этажей, т.к. первый не видно, и 8 окон на каждом этаже, уточнить по месту). Допустим, что каждое из них освещено вечером независимо от других с вероятностью одна вторая. Назовем <<уголком>> комбинацию из 4-х окон, расположенных квадратом, в которой освещено ровно три окна (не важно, какие). Пусть $X$ - число <<уголков>>, возможно пересекающихся, на всем корпусе Ж. \\
Найдите  $E(X)$ и $Var(X)$ \\
Примечание - для наглядности: \\
\begin{tabular}{|c|c|}
  \hline
  X & X\\
  \hline
    & X \\
  \hline
\end{tabular},
\begin{tabular}{|c|c|}
  \hline
  X & \\
  \hline
  X & X \\
  \hline
\end{tabular},
\begin{tabular}{|c|c|}
  \hline
   & X\\
  \hline
  X & X \\
  \hline
\end{tabular},
\begin{tabular}{|c|c|}
  \hline
  X & X\\
  \hline
  X &  \\
  \hline
\end{tabular} - это <<уголки>>. \\
\begin{tabular}{|c|c|c|}
  \hline
  X & X & X\\
  \hline
    & X & \\
  \hline
  X & X & \\
  \hline
\end{tabular} - в этой конфигурации три <<уголка>>;
\begin{tabular}{|c|c|c|}
  \hline
  X &  & X\\
  \hline
    & X & \\
  \hline
  X &  & X\\
  \hline
\end{tabular} - а здесь - ни одного <<уголка>>. 
}
\solution{ }

\problem{
В урне  $n$  шаров пронумерованных 1,2,... $n$. Наугад
вытаскивают $k$. Найдите ожидание и дисперсию суммы номеров. }
\solution{ }

\problem{ \label{chepchiki}
\emph{Кричали женщины <<Ура!>> и в воздух чепчики бросали!} (А.С. Грибоедов) \par
Приезжающих из армии или от двора встречают $n$ женщин. Они
одновременно подбрасывают вверх $n$ чепчиков. Ловят чепчики
наугад, каждая женщина ловит один чепчик.
Женщины, поймавшие свой чепчик уходят. А женщины,
поймавшие чужой чепчик, снова подбрасывают его вверх.
Подбрасывание чепчиков продолжается до тех пор, пока каждая не
поймает свой чепчик. \par
а) Пусть $N$ - количество женщин, поймавших свой чепчик после
первого подбрасывания. Найдите $E(N)$ и $Var(N)$ \par
б) Пусть $R$ - количество подбрасываний. Найдите $E(R)$. \par
в) Пусть $S$ - количество чепчиков, пойманных мадмуазель NN.
Найдите $E(S)$. \par
г) Какова вероятность того, что ни одна женщина не поймает свой
чепчик после 1-го подбрасывания? \par
$[$Ross, example 3.13$]$ }
\solution{б) 1 г) $1-\frac{1}{2!}+\frac{1}{3!}-\frac{1}{4!}+...$  }

\problem{ \label{mechenii shar}$[$Ross, Pro Models, 2.43$]$ \par
В урне лежат $m$ шаров цвета морской волны и $k$ шаров цвета
<<персик>>. Маша пометила один из шаров цвета <<персик>>. Шары
извлекают в случайном порядке. \par
а) Какова вероятность того, что меченый шар появится раньше, чем
первый шар цвета морской волны? \par
б) Каково ожидаемое количество шаров цвета <<персик>>, извлеченных
до первого шара цвета морской волны? \par
в) Каково ожидаемое количество шаров цвета <<персик>>, извлеченных
между первым и вторым шарами цвета морской волны? }
\solution{a) Персиковые шары кроме меченого на вероятность не влияют.
Следовательно, $P=1/(m+1)$ \par
b) $E=k/(m+1)$ }

\problem{ \label{izvlekaem do odnotsvetnih}
В урне лежат $a$ шаров абрикосового
цвета и $b$ шаров бежевого цвета. Шары извлекаются до тех пор,
пока в урне не останутся шары только
одного цвета. \par
а) Какова вероятность того, что в конце останутся шары цвета
абрикос? \par
б) Каково ожидаемое количество оставшихся шаров? \par
в) Каково ожидаемое количество оставшихся шаров, если известно,
что остались шары цвета беж \par
Hint: номер \ref{mechenii shar} }
\solution{ a) Доизвлекаем все шары. Искомая вероятность равна вероятности
того, что последним будет извлечен абрикос. А эта вероятность
равна $\frac{a}{a+b}$ \par
b) Пусть с.в. $A$ - число шаров-абрикосов в конце, а $B$ - число
шаров беж в конце. Т.е. если в конце извлекается три абрикосовых
шара, то $A=3$, а $B=0$. Искомая величина это
$E(A)+E(B)=\frac{a}{b+1}+\frac{b}{a+1}$ \par
в) Поменяем порядок. Каково ожидаемое количество шаров до первого
абрикосового, если известно, что первый шар - беж.
$1+\frac{b-1}{a+1}$ }


\problem{ \label{izvlekaem do odnotsvetnih2}
В урне лежат $a$ абрикосовых
шаров, $b$ белых шаров и $c$ синих шаров. Шары извлекаются наугад по-очереди без возвращения. \par
а) Какова вероятность того, что первыми полностью будут извлечены абрикосовые шары? \par
б) Сколько шаров в среднем останется после того, как впервые полностью будет извлечен какой-нибудь цвет? Сколько шаров при этом в среднем будет извлечено? \par
в) Какова вероятность того, что абрикосовые будут полностью извлечены раньше белых, если известно, что белые были полностью извлечены позже синих? \par
Hint: номера \ref{mechenii shar} и \ref{izvlekaem do odnotsvetnih} \par
Source: AMM E2724 by Harry Lass }
\solution{
а) $p_{abc}$ - вероятность того, что сначала полностью извлекаются абрикосовые шары, затем полностью извлекаются белые и в конце - синие. \par
Заметим, что: \par
$p_{abc}+p_{bac}=\frac{c}{a+b+c}$ - вероятность того, что синий шар - последний \par
$p_{abc}+p_{acb}+p_{bac}=\frac{c}{a+c}$ - вероятность того, что абрикосовые вышли раньше синих \par
Отсюда находим $p_{acb}$ и все остальные \par
б) Рассмотрим случай, когда невзятым остается какой-нибудь синий шар. Значит, он лежит либо после всех белых, либо после всех абрикосовых. Вероятность этог равна $p_{c}=\frac{1}{a+1}+\frac{1}{b+1}-\frac{1}{a+b+1}$. Значит в среднем остается $ap_{a}+bp_{b}+cp_{c}$ шаров. }

\problem{ \label{abrikos, bej i raznie viborki}
В урне $a$ шаров цвета абрикос, $b$ шаров цвета беж и $c$ шаров
цвета бедра испуганной нимфы. \par
Извлекается наугад $n\le a+b+c$ шаров. Пусть $A$ и $B$ -
количество извлеченных шаров цвета абрикос и цвета беж \par
а) Какова вероятность того, что $i$-ый извлеченный шар будет
абрикосовым? \par
б) Каково среднее количество извлеченных шаров цвета абрикос? \par
в) Найдите $Var(A)$, $Cov(A,B)$, $Corr(A,B)$ \par
г) Как изменятся ответы, если после извлечения шара и записывания
его цвета шар будет возвращаться обратно в урну? }
\solution{a) $\frac{a}{a+b+c}$ \par
b) $\frac{na}{a+b+c}$ \par
в) $Var(A_{i})=p_{a}(1-p_{a})$, $Cov(A_{i},B_{i})=-p_{a}p_{b}$,
$Cov(A_{i},B_{j})=-p_{a}p_{b}\frac{1}{N-1}$ \par
$Var(A)=np_{a}(1-p_{a})\frac{N-n}{N-1}$, $Cov(A,B)=-n
p_{a}p_{b}\frac{N-n}{N-1}$,
$Corr(A,B)=-\sqrt{\frac{p_{a}p_{b}}{(1-p_{a})(1-p_{b})}}$ \par
г) $Var(A_{i})=p_{a}(1-p_{a})$, $Cov(A_{i},B_{i})=-p_{a}p_{b}$,
$Cov(A_{i},B_{j})=0$ \par
$Var(A)=np_{a}(1-p_{a})$, $Cov(A,B)=-n p_{a}p_{b}$,
$Corr(A,B)=-\sqrt{\frac{p_{a}p_{b}}{(1-p_{a})(1-p_{b})}}$ }

\problem{ Grimmett, 3.4.4. \par
В урне $A$ находятся $a$ шаров цвета <<абрикос>>, в урне $B$ - $a$
шаров цвета беж. В каждый момент времени выбирают по одному шару
наугад из каждой урны и меняют местами. \par
Найдите ожидаемое количество шаров цвета <<абрикос>> в урне $A$
после $k$ шагов. }
\solution{ }

\problem{ \label{kart do 1 tuza}
Из хорошо перетасованной колоды в 52
карты, содержащей четыре туза, извлекаются сверху карты до
появления первого туза.
На каком месте в среднем появляется первый туз? }
\solution{Решение (предложила Саша Серова)\par
Сначала сделаем колоду из четырех тузов, лежащих стопкой в
случайном порядке. Затем карты будем по одной класть на случайное
место в формируемой колоде. Всего 5 мест. По индукции видно, что
вероятность для каждой карты попасть на место впереди первого туза
равна $\frac{1}{5}$. Если $X$ - количество карт, попавших раньше
первого туза, то $X=X_{1}+...+X_{48}$ и
$E(X)=48\cdot\frac{1}{5}=\frac{48}{5}$. \par
Решение (идея Mosteller, 50 задач?) \par
Добавим в колоду ложного, пятого, туза. Разложим колоду случайным
образом по окружности. Пятый туз разделит окружность на обычную
колоду (именно за этим он и нужен). До разрезания окружности пять
тузов разбивали оставшиеся 48 карт на пять равнораспределенных
частей. }

\problem{
Из колоды в 52 карты случайным образом извлекаются 26 карт и с сохранением порядка между собой кладутся наверх колоды. \par
Сколько в среднем карт при этом остается на своих местах? }
\solution{ Solution:
Остаются на своих местах карты, выбранные сверху колоды и карты, невыбранные снизу колоды. \par
далее-? }

\problem{ $[$Ross, 2.45$]$ \par
У кота Базилио $n$ пустых копилок и $z$ золотых.
%Кот Базилио выбрал вероятности $p_{1}$, $p_{2}$,...,
%$p_{n}$. Кот Базилио, естественно, в курсе, что
%$p_{1}+...+p_{n}=1$.
Базилио наугад бросает один за одним золотые в копилки. Если
золотой падает в копилку, где уже есть золотые, то слышно приятное позвякивание. \par
Найдите ожидаемое количество позвякиваний. }
\solution{ }

\problem{ Игла Бюффона, Buffon needle \par
Плоскость расчерчена параллельными линиями, находящимися на расстоянии в 1 см. Случайным образом на плоскость бросается веревка длины $a$ см. Пусть $X$ - количество пересечений веревки с начерченными линиями. \par
а) Верно ли, что $E(X)$ пропорционально $a$? \par
б) Если вместо веревки взять жесткое кольцо с диаметром 1 см, то чему будет равно $X$, $E(X)$? \par
в) Чему равно $E(X)$ для веревки длины $a$? \par
г) Если вместо веревки бросается иголка длины $a<0.5$, то чему равна вероятность того, что она пересечет хотя бы одну линию? }
\solution{ Solution (Bill Taylor?): \par
a) да, т.к. разбив веревку на две части получим, что $E(X)=E(X_{1})+E(X_{2})$ \par
б) При $a=\pi$ получаем $E(X)=2$ \par
в, г) Для произвольного $a$ получаем $E(X)=\frac{2a}{\pi}$ }

\problem{ Треугольник Бюффона? \par
Плоскость расчерчена параллельными линиями, находящимися на расстоянии в 1 см. Случайным образом на плоскость бросается треугольник со сторонами $a<1$, $b<1$, $c<1$.\par
Какова вероятность того, что линия будет пересекать треугольник? 

Source: aops, t=179733 }
\solution{Пользуясь предыдущей задачей: \par
Ожидаемое число пересечений равно $\frac{2(a+b+c)}{\pi}$. \par
Вероятность в два раза меньше (одна линия пересекает треугольник в двух местах): \par
$P=\frac{a+b+c}{\pi}$  }

\problem{ Условная вероятность по Бюффону \par
Плоскость расчерчена параллельными линиями, находящимися на расстоянии в 1 см. Случайным образом на плоскость бросается треугольник с периметром $P<2$. Строго внутрь него случайным образом бросается второй треугольник с периматром $p<P$. \par
Какова вероятность того, что прямая малый треугольник, если она пересекает большой?}

\solution{ Ожидаемое число пересечений большого треугольника: $2P/\pi$. \par
Ожидаемое число пересечений малого треугольника: $2p/\pi$. \par
Вероятности пересечений - в 2 раза меньше. \par
Условная вероятность равна отношению периметров. }


\problem{ Условная вероятность по Бюффону-2 (?) \par
Одна в другой в пространстве находятся две сферы, радиусов $R$ и $r$. Случайным образом в пространстве проводится прямая. Какова вероятность того, что она пересечет малую сферу, если она пересекает большую? }
\solution{ 
Переформулируем задачу, как задачу Бюффона (параллельные прямые в пространстве...) \par
Вероятность в два раза меньше ожидаемого числа пересечений. Ожидаемое число пересечений пропорционально площади поверхности. Ответ - отношение площадей поверхностей, т.е. $R^{2}/r^{2}$ \par
Коммент: \par
Как эти все задачи аккуратно оформлять? Ведь нельзя провести <<равновероятным>> образом прямую в пространстве. Нельзя <<равновероятно>> бросить иголку на плоскость... }

\problem{ \label{podpravka i udavi}
В лесу живет $N$ удавов. Каждый из них имеет свою длину.
Обозначим $\sigma^2$ дисперсию длины наугад выбранного удава.
Сегодня $n$ удавов выползли погреться на солнышке на большой
поляне. Обозначим
$\overline{X_{n}}$ среднюю длину выползших удавов. \par
а) Чему равно $Cov(X_{1},\sum_{i=1}^{N} X_{i})$? \par
б) Выразите $Cov(X_{i},X_{j})$ через $\sigma^{2}$ для $i\neq j$. \par
в) Выразите $Var(\bar{X}_{n})$ через $\sigma^{2}$. }
\solution{ a) 0 б) $Cov(X_{i},X_{j})=-\frac{1}{N-1}\sigma^{2}$ в)
$Var(\bar{X}_{n})=\frac{\sigma^{2}}{n}\frac{N-n}{N-1}$ }

\problem{ Do men have more sisters than women? \par
В семье $n$ детей. Предположим, что вероятности рождения мальчика и девочки равны. Дед Мороз спросил каждого мальчика <<Сколько у тебя сестер?>> и сложив эти ответы получил $X$. Затем Дед Мороз спросил каждую девочку <<Сколько у тебя сестер?>> и cложив эти ответы получил $Y$. \par
Найдите $E(X)$ и $E(Y)$? \par
source: cut-the-knot \par
дважды сестра считается дважды  }
\solution{ Если в семье $n$ детей, то ожидаемое количество мужских и женских сестер
одинаково и равно $n(n-1)/2$ (?)}

%повтор - Маша белые мухоморы
\problem{
Допустим, что на острове Независимом погода в разные дни независима. Известно, что день оказывается солнечным с вероятностью $p$, пасмурным без дождя с вероятностью $q$ и дождливым с вероятностью $1-p-q$. Пусть $X$ - число солнечных дней, а $Y$ число пасмурных дней без дождя за год. \par
a) Найдите $Cov(X,Y)$ \par
б) Прокомментируйте знак ковариации  }
\solution{ Ответ: $Cov(X,Y)=-365pq$, чем больше $X$, тем (скорей всего) меньше $Y$ }

\problem{ Девятый вал \par
% [надо бы все допроверить...]\par
Пусть $X_{n}$ - iid, непрерывно распределены,
%$U[0;1]$ - условие на равномерность не обязательно?
это размер $n$-ой волны. Будем
называть волну большой, если она больше предыдущей и больше
последующей. \par
а) Какова вероятность того, что $n$-ая волна - большая? \par
б) Какой по счету в среднем появляется очередная большая волна? \par
в) Попробуйте (не совсем строго) обосновать название <<Девятый вал>> }

\solution{
a) Одна из трех последовательных волн обязательно больше двух
других. Поскольку волны iid, то вероятность того, что большой
будет именно вторая равна $1/3$. \par
a) Если волны iid $U[0;1]$, то равенство вероятности одной третьей
соответствует разрезанию куба на три пирамиды. \par
%Пусть $T_{n}$ - время появления очередной большой волны.
%$E(T_{n})=E(\frac{T_{1}+T_{2}+...+T_{N}}{N})=E$ ???
}

\problem{
Из грота ведут 10 штреков, с длинами 100м, 200м,... 1000м. Самый длинный штрек оканчивается выходом на поверхность. Остальные - тупиком. Вася выбирает штреки наугад (естественно, в тупиковый штрек он два раза не ходит). Какой в среднем путь он нагуляет прежде чем выберется на поверхность. }
\solution{ Вероятность посещения каждого тупикового штрека равна 0.5. \par
Следовательно: $E(X)=1000+0.5(100+...+900)=3250$ }

\problem{
У меня в кармане 3 рубля мелочью. Среди монет всего одна монета достоинством 50 копеек. Я извлекаю монеты по одной наугад до извлечения 50 копеечной монеты. Какую сумму в среднем я извлеку? }
\solution{Каждая монета кроме 50 копеечной выбирается с вероятностью 0.5. \par
Следовательно, $E(S)=0.5+\frac{2.5}{2}=1.75$  }

\problem{ Модница \par
В шкатулке у Маши 100 пар сережек. Каждый день утром она выбирает одну пару наугад, носит ее, а вечером возвращает в шкатулку. Проходит год. \par
а) Сколько в среднем пар окажутся ни разу не надетыми? \par
б) Сколько в среднем пар окажутся одетыми не менее двух раз? \par
в*) Как изменятся ответы, если каждый день Маша покупает себе новую пару сережек и вечером добавляет ее в шкатулку? }
\solution{а) $100\cdot (0.99)^{365}$ \par
б) $100\cdot \left(1-0.99^{365}-365\cdot 0.01\cdot 0.99^{364}\right)$ \par
в) Пусть $g(x,k)$ - среднее количество надетых сережек, если осталось $k$ дней, а к текущему дню надето $x$ сережек. \par
Разностное уравнение: $g(x,k)=\frac{x}{T-k}g(x,k-1)+\frac{T-k-x}{T-k}g(x+1,k-1)$, где $T=365+100$ \par
Начальное условие $g(x,0)=x$. \par
После первой итерации ясно, что $g(x,k)=a_{k}x+b_{k}$ \par
После второй: $a_{k}=a_{k-1}\frac{T-k-1}{T-k}$, $b_{k}=a_{k-1}+b_{k-1}$ \par
Получаем $a_{k}=1-\frac{k}{T-1}$, $b_{k}=\frac{k}{2}\frac{2T-k-1}{T-1}$ \par
Нам нужно найти $g(0,365)=b_{365}$ }

\problem{ Две корзины, две игры \par
В корзине А лежит 50 белых и 50 черных шаров. \par
В корзину Б поместили 100 наугад выбранных шаров из корзины, где изначально лежали 100 белых и 100 черных шаров. \par
Игра 1. Из корзины наугад извлекается один шар. Если он белого цвета, Вы получаете 100 рублей. \par
Игра 2. Из корзины извлекаются все шары. За каждый белый шар Вы получаете 1 рубль. \par
Сравните ожидаемый выигрыш и дисперсию выигрыша для четырах случаев: \par
а) Игра 1 с корзиной А \par
а) Игра 1 с корзиной Б \par
а) Игра 2 с корзиной А \par
а) Игра 2 с корзиной Б }
\solution{ }

\problem{
В магазине продается 50 видов конфет. Каждый из 10 покупателей покупает один вид конфет, выбираемый наугад. Сколько видов конфет будет в среднем куплено? }
\solution{ }

\problem{
На отрезке выбирается равномерно и независимо друг от друга $n$ точек. Они разбивают отрезок на $(n+1)$ часть.  \par
а) Верно ли, что длины полученных частей одинаково распределены? \par
б) Чему равна ожидаемая длина самого левого части? \par
в) После разбития отрезка на $(n+1)$ часть еще одна точка равномерно выбирается на отрезке. Какова средняя длина части, в которую она попадает?}
\solution{ 
а) да, но зависимы - разбиваем окружность на $(n+1)$ часть \par
б) $\frac{1}{n+1}$ \par
в) $\frac{2}{n+2}$ - разбиваем окружность на $(n+2)$ части}

\problem{
На кольцевой автодороге вокруг города $N$ есть единственный сервис-центр. Вася и Петя, гости города $N$, независимо друг от друга сломались
в разных точках кольца и не могут определить свое местоположение на кольце. Поэтому автосервис высылает два транспортировщика, одного по часовой стрелки, другого - против часовой стрелки.
Каждый буксир едет до ближайшей сломавшейся машины. \par
Обозначим $L$ - суммарное расстояние, которое проедут буксиры от сервис центра до Пети и Васи. \par
а) Найдите $E(L)$ \par
б) Найдите функцию плотности $L$ \par
в) Верно ли, что данная стратегия (посылать две машины в разные стороны) минимизирует мат. ожидание и дисперсию расходов сервиса на бензин? \par
(предполагается, что возвращаются машины кратчайшим путем) }
\solution{ 
а) Пусть $Y_{1}$ (против часовой от Сервиса), $Y_{2}$ (по часовой от Сервиса)  и $Y_{3}$ (от Пети до Васи) - длины трех отрезков, на которые делят окружность Петя, Вася и Сервис. \par
Поэтому $E(L)=2/3$ \par
б) $P(Y_{1}+Y_{2}\le t)=P(Y_{3}>1-t)$ \par
Развернем окружность в отрезок, приняв за ноль координату того, кто сломался против часовой от Сервиса.  \par
$P(Y_{3}>1-t)=P(X_{1}>1-t)P(X_{2}>1-t)=t^{2}$ \par
Здесь $X_{1}$ и $X_{2}$ - координаты Сервиса и второго сломавшегося относительно начала отрезка. \par
в) Похоже, что матожидание не зависит от стратегии (за исключением заведомо глупых стратегий), а дисперсия - минимальна.
Solution b2: \par
Разрежем окружность на Сервисе (звучит то как!), $X_{1}$ и $X_{2}$ - координаты Пети и Васи \par
$P(L\le t)=P(X_{1}<X_{2}\cap X_{1}-X{2}\le t-1)+P(X_{2}<X_{1}\cap X_{2}-X_{1}\le t-1)=2P(X_{2}\le X_{1}+t-1)$ \par
Далее ищем площадь в осях $(X_{1},X_{2})$, там треугольник. Получаем $P(L\le t)=2\frac{t^{2}}{2}=t^{2}$ \par
Solution b3: \par
$P(L\le t)=2\int_{0}^{t}(t-a)da=t^{2}$. Пусть Петя попал в точку $a$ против часовой от Сервиса (плотность равна 1). Тогда Васе может попасть в отрезок
длиной $t-a$ по часовой от Сервиса. <<Суммируем>> по всем $a$, на два множим потому, что против часовой мог оказаться Вася. }

\problem{
Пусть $X_{1}$,... $X_{n}$ независимы и имеют функцию распределения $F(t)$. Обозначим $\hat{F}(t)$ - долю величин, оказавшихся не выше $t$. \par
а) Найдите $E(\hat{F}(t))$, $Var(\hat{F}(t))$, $Cov(\hat{F}(t),\hat{F}(s))$ }
% б) Является ли оценка $\hat{F}(t)$ состоятельной? \par
% задача скопирована в статистику, там вопрос про состоятельность
\solution{ }

\problem{
Экстрасенс дед Агнав угадывает результат выпадения правильной монетки с вероятностью $p$. Монетку подбрасывают $n$ раз. За верное угадывание жюри ему выплачивает 1 рубль, причем, если дед Агнав угадывает $k$ раз подряд, то при каждом угадывании выплата растет на 1 рубль. Например, если результат угадываний имеет вид: $+,+,-,-,+,-,+,+,+$, то дед Агнав получит $1+2+1+1+2+3=10$ рублей. \par
а) Чему равен ожидаемый выигрыш? \par
б) Чему равна дисперсия выигрыша? \par
% хорош ли ответ в б? \par
source: aops, t=194628 }
% Suppose Bob can predict the outcome of a binary event with probability $p$.
% Bob’s employer pays him 1 dollar per correct prediction; with a catch:
% If Bob gets $k$ correct predictions in a row; then each $i$-th correct prediction rewards Bob $i$ dollars. \par
% For example, if Bob has the sequence T,F,T,F,T,F,T ; then his payoff is 4 dollars, but if it is T,T,T,T,F,F,F; then it is 1+2+3+4 dollars. \par
% So if there are N events that Bob is to predict; what is his expected yearly salary?
\solution{ }

\problem{ 
Вовочка получает пятерку с вероятностью 0.1, четверку - с вероятностью 0.2, тройку - с вероятностью - 0.3 и двойку с вероятностью 0.4. В этом четверти он писал 20 контрольных. Сколько разных оценок он в среднем получит?  }
\solution{Ответ: $4-0.9^{20}-0.8^{20}-0.7^{20}-0.6^{20}$ }

\problem{
В коробке лежат разноцветные шарики. Всего $c$ цветов, и $h$ шариков каждого цвета. Вовочка достает шарики из коробки по одному в случайном порядке. Если цвет шарика совпадает с цветом предыдущего, то Вовочка кричит <<Ого!!!>>. \par
Сколько раз в среднем Вовочка крикнет <<Ого!!!>> пока достанет все шарики? }
\solution{ $c-1$ }

\problem{Маша собрала 100 грибов. Она собирала все грибы подряд без разбора. В лесу белые грибы встречаются с вероятностью $p>0$, мухоморы - с вероятностью $q>0$, $p+q\leq 1$. Найдите корреляцию между количеством белых и мухоморов у Маши. При каких $p$ и $q$ корреляция наиболее велика (по модулю)?}
\solution{Количества белых и мухоморов - биномиальные с.в. Дело за ковариацией. $B=B_{1}+...+B_{100}$, $M_=M_{1}+...+M_{100}$, $Cov(B,M)=100Cov(B_{1},M_{1})=-100pq$. $Corr(B,M)=-sqrt{\frac{pq}{(1-p)(1-q)}}$}

\problem{В коробке лежит 10 шариков занумерованных от 1 до 10. Маша вытаскивает их по одному в случайном порядке. Если очередной шарик имеет наименьший номер среди оставшихся в коробке, то Маша получает столько рублей, сколько написано на шарике. Каков средний машин Выигрыш?}
\solution{Шарик с надписью 1 всегда приносит рубль, шарик с надписью 2 рубля - с вероятностью 1/2 и т.д. Значит средний выигрыш равен 10}

\problem{К потолку веревочкой привязано металлическое колечко единичной длины. Мы разрезаем его в двух случайно (равномерно) выбираемых местах. Один из кусочков падает на пол, а второй остается привязанным к потолку. Какова вероятность того, что кусок привязанный к потолку больше, чем кусок привязанный к полу? Чему равна средняя длина кусочка упавшего на пол?}
\solution{Точку привязки веревки к колечку можно рассматривать как случайно выбираемую третью точку. Значит средняя длина куска оставшегося привязанным к потолку $2/3$.}

\problem{На бесконечном столе лежит бесконечный ряд карточек, на которых снизу написаны числа --- независимые реализации некого непрерывного (одинакового для всех карточек) распределения. Вася открывает карточки, пока число, написанное на очередной карточке не меньше, чем число, написанное на предыдущей карточке. Сколько, в среднем, карточек откроет Вася?

Источник: Алексей Суздальцев}
\solution{Конечно, $e$.}




\subsection{о-малое}

\subsection{Вероятностный метод}
% probab_method

\problem{ \label{Euler's formula}[Williams, 4.2] \\
Для $s>1$ обозначим $\xi(s)=\sum_{n=1}^{\infty}\frac{1}{n^{s}}$. \\
Рассмотрим случайную величину $X$ с
$P(X=n)=\frac{n^{-s}}{\xi(s)}$.  \\
а) Докажите, что события $\{X$ делится на $p\}$ и $\{X$ делится на
$q\}$ независимы, если $p$ и $q$ - два различных простых числа \\
б) Докажите формулу Эйлера: $\frac{1}{\xi(s)}=\prod_{p\text{ is
prime}}(1-p^{-s})$ \\
в) Найдите $P(\{X$ не делится на квадрат ни одного простого числа$\}$ \\
г) Найдите $P(X=xk|X\mod{k}=0)$ \\
д) Пусть с.в. $Y$ имеет такое же распределение как и $X$, и они
независимы. Пусть с.в. $H$ - наибольший общий делитель $X$ и $Y$.
Найдите $P(H=n)$. }
\solution{a) $P(\{X$ делится на $p\})=p^{-s}$ \\
b) Левая и правая части - это вероятность того, что число не
делится ни на одно простое. Такое число одно - единица. \\
в) $P=\frac{1}{\xi(2s)}$ \\
г) Если $X$ делится на $k$, то условное распределение
$\frac{X}{k}$ совпадает с безусловным распределением $X$: \\
$P(\frac{X}{k}=n|X=0\mod{k})=n^{-s}/\xi(S)=P(X=n)$ \\
д) [better is welcome?] \\
Аналогично, можно заметить, что в случае, когда $X$ и $Y$ делятся
на $k$, то $\frac{X}{k}$ и $\frac{Y}{k}$ будут (условно)
независимы: \\
$P(\frac{X}{k}=x\cap \frac{Y}{k}=y|X=0\mod k\cap Y=0\mod k)=
P(\frac{X}{k}=x|X=0\mod k)P(\frac{Y}{k}=y|Y=0\mod k)$ \\
$P(p$ не является общим делителем $X$ и $Y)=(1-p^{-2s})$ \\
$P(X$ и $Y$ не имеют общих делителей$)=\prod_{p\text{ is
prime}}(1-p^{-2s})=\frac{1}{\xi(2s)}$. \\
Вероятность того, что $X$ и $Y$ делятся на $k$ равна $k^{-2s}$. \\
Искомая вероятность равна $\frac{k^{-2s}}{\xi(2s)}$  }


\problem{
В круге радиуса 16 выбрано 650 точек. Докажите, что существует
кольцо с внутренним радиусом 2 и внешним радиусом 3, которое
содержит не менее 10 точек. }
\solution{ }

\problem{ \label{tsvetnaia sfera} [Grimmett, Stirzaker] \\
На сфере 10\% точек окрашены белым, а остальные 90\% - красным
цветом. Форма окраски неизвестна. Верно ли, что можно так
вписать куб, что все его вершины будут одноцветными? }
\solution{ Да. Впишем куб наугад. $A_{i}=\{i$-ая вершина белая$\}$.
$P(A_{i})=0.1$ и $P(\cup A_{i})\le \sum P(A_{i})=0.8$.
Следовательно, $P(\{$все вершины красные$\})\ge 0.2$, т.е. искомый
куб существует. Множества измеримы по условию! }

\problem{
Пусть $A$ - произвольная матрица. Докажите, что существует вектор $v$ состоящий из единиц или минус единиц, такой что $vAv^{t}\ge tr(A)$. \\
Source: http://www.artofproblemsolving.com/Forum/viewtopic.php?t=153979}
\solution{ 
Just take the mean value of the quadratic function $f=\sum_{i,j}a_{ij}x_{i}x_{j}$ over the points of $\{-1,1\}^{n}$. That is to say, let $x_{i}$ be a random variable taking -1,1 with the same probability. Now we want to find the expected value of f. This expected value is the sum of expected values of $a_{ij}x_{i}x_{j}$ for different i,j's. Now it is easily seen that this expected value for $i\neq j$ is 0 and is $a_{ii}$ otherwise. So the expected value of f is just $\sum_{i}a_{ii}$ which is the trace of the matrix which is equal to the sum of eigenvalues.}


\problem{
Let $S$ be a finite set of points in the plane such that no three of them are on a line. For each convex polygon $P$ whose vertices are in $S$, let $a(P)$ be the number of vertices of $P$, and let $b(P)$ be the number of points of $S$ which are outside $P$.
Prove that for every real number $x$: \\
$\sum_{P}{x^{a(P)}(1-x)^{b(P)}}=1$, where the sum is taken over all convex polygons with vertices in $S$. 

Remark. A line segment, a point, and the empty set are considered as convex polygons of 2, 1, and 0 vertices respectively.}

\solution{ 
Let's colour the points black and white and Let the probability that a point has colour black be $x$. Now look at any convex polygon $P$. The probability that all its vertices are black is $x^{a(P)}$ and that all the points outside it are white is $(1-x)^{b(P)}$. The given summation is the probability that there exists a polygon with vertices black and all points outside it are white. But we know this is true (i.e. 1) as all we have to do is consider the convex hull of the black points. Now this is true for all $x\in(0,1)$. But this is a polynomial with finite degree and the number of points at which it evaluates 1 is infinite. so, this is an identity and holds $\forall x\in \mathbb{R}$\\
Продумать }

\problem{ Erdos-Ko-Rado theorem \\
$[$Alon, Spencer, Probabilistic method$]$ \\
$[$нуждается в обработке.........$]$ \\
Если: $\Omega=\{0,1,...,n-1\}$. И $\mathcal{H}$ - набор
пересекающихся подмножеств $\Omega$, состоящих из $k\le n/2$
элементов, т.е., если $A\in\mathcal{H}$ и $B\in\mathcal{H}$, то $A\cap B\neq \emptyset$. \\
То: число элементов $\#\mathcal{H}\le C_{n-1}^{k-1}$ \\
Доказательство: \\
а) Пусть $A_{s}$ - множество из $k$ чисел подряд, начиная с числа
$s$ (при необходимости, счет продолжается с нуля). Сколько
множеств $A_{s}$ может входить в $\mathcal{H}$? \\
б) Пусть $\sigma$ - случайная перестановка, и $i$ случайный номер.
Найдите $P(A\in\mathcal{H})$ \\
в) Завершите док-во }
\solution{ }

\problem{ Turan's theorem $[$Grimmett, Stirzaker$]$ \\
Пусть $G$ - конечный граф без петель и двойных ребер. Обозначим
$d_{k}$ - число ребер, исходящих из вершины $k$ (степень вершины).
На графе можно отметить несколько вершин так, чтобы ни одна из них
не была напрямую связана с другой. Докажите, что максимальное
количество таких вершин $n\ge\sum_{k}\frac{1}{d_{k}+1}$ }
\solution{ }

\problem{ $[$by idea of Sasha Serova$]$ \\
В копилке 40 золотых и 60 серебряных монет. \\
а) Какова вероятность того, что при выборе 20 монет ровно $i$
окажутся серебряными? \\
б) Чему равна сумма этих вероятностей ($i=0..20$)? \\
в) Предложите вероятностное доказательство формулы
$\sum_{i}C_{a}^{i}C_{b}^{n-i}=C_{a+b}^{n}$ }
\solution{ }

\problem{
Со стола упало 10 чашек. Каждая разбивается с вероятностью 0.5. \\
а) Какова вероятность, что разобьется ровно $k$ чашек? \\
б) Предложите вероятностное доказательство формулы
$\sum_{i}C_{n}^{i}=2^{n}$ }
\solution{ }

\problem{ In a square of side 1, you have been able to capture 51 ants, who are possibly panicking.\\
You have a glass, whose radius is 1/7.\\
Show that you can at any moment position the glass so as to encompass at least 4 of them. }
\solution{
Накроем стаканом случайную (равномерно выбираемую) область, целиком лежащую внутри квадрата. \\
Ожидаемое число муравьев, попадающих в эту область равно $(1-\delta)\cdot 51\cdot \pi\left(\frac{1}{7}\right)^{2}$. Здесь $\delta$ - это площади четырех ненакрываемых кусков в углах квадрата. Получаем, что ожидаемое число больше 3-х ($\delta=\frac{4-\pi}{49}$). Значит, есть точка, где накроется 4 муравья. }


% from 31 problem of RaviB

\problem{ Пусть $p_{n,k}$ - число перестановок из $n$ элементов, оставляющих на месте $k$ элементов. Найдите $\sum_{k} kp_{n,k}$. }
\solution{ Выберем случайную перестановку. Посчитаем среднее число неподвижных точек, получим 1. Значит $\sum_{k} k\frac{p_{n,k}}{n!}=1$. Отсюда $\sum_{k} kp_{n,k}=n!$. }

\problem{Тервер на службе у алгебры
\begin{enumerate}
\item $1\cdot C_n^1+2\cdot C_n^2+3\cdot C_n^3+\cdots+n\cdot C_n^n=?$
\item $1\cdot C_n^1+4\cdot C_n^2+9\cdot C_n^3+\cdots+n^2\cdot C_n^n=?$
\item $\sum\limits_{i = 1}^9 {\frac{{i \cdot C_{223}^i  \cdot C_{1785}^{9 - i} }}{{C_{2008}^9}}}=?$ 
\end{enumerate}

Источник: Алексей Суздальцев}
\solution{(a) $n2^{n-1}$ (b) $(n+n^2)2^{n-2}$. (c) $\frac{2007}{2008}$. В первых двух случаях используем полученные разложением в сумму формулы моментов биномиального распределения, в третьем --- гипергеометрического.}





\section{Нормальное распределение, ЦПТ}
% normal
% сюда также относятся задачи
% на распределения связанные с нормальным (не задачи на проверку гипотез)

\subsection{Одномерное нормальное распределение}
\subsection{ЦПТ}
\subsection{Многомерное нормальное распределение}




\problem{
Пусть $X\sim N(0;1)$; $Z$ равновероятно принимает значения $1$ и
$-1$; $X$ и $Z$ независимы. Рассмотрим $Y=X\cdot Z$.
Найдите:\\
а) закон распределения $Y$; \\
б) $Cov(X,Y)$; \\
в) условное распределение $X$, если известно, что $Y=y$. \\
г) верно ли, что $X+Y$ нормально? \\
д) Что изменится, если $P(Z=1)=1-P(Z=-1)=p$? \\
е) Верно ли, что если $X\sim N$, $Y\sim N$, $Corr(X,Y)=0$, то
$X+Y\sim N$? }
\solution{ }
\problem{
Пусть $X\sim t_{n}$. Как распределена величина $Y=X^{2}$? }
\solution{ }
\problem{
Пусть $X\sim N(0;1)$ \\
Найдите функцию плотности для $Z=\frac{1}{X^{2}}$ }
\solution{ }
\problem{
Пусть $X_{i}$ - iid $N(0;1)$. Найдите
$P(X_{1}^{2}+X_{2}^{2}>6.37\cdot
(X_{3}^{2}+X_{4}^{2}+X_{5}^{2}))$. }
\solution{ }
\problem{
Пусть $X_{i}$ - iid $N(0;1)$. \\
а) Как распределена случайная величина $\frac{X_{1}}{|X_{2}|}$?\\
б) Найдите
$P\left(X_{5}>2.3\sqrt{X_{1}^{2}+X_{2}^{2}+X_{3}^{2}+X_{4}^{2}}\right)$. }
\solution{ }
\problem{
Пусть $Y\sim \chi_{n}^{2}$ и $W\sim t_{n}$. Найдите $E(Y)$,
$E(W)$ и [т?] $Var(Y)$. }
\solution{ }
\problem{
Найдите $P(Y>2)$, если $Y=\sum_{i=1}^{9} X_{i}^{2}$, а
$X_{i}$ - iid $N(0;1)$. }
\solution{ }
\problem{
Пусть $Y_{1}$ имеет $\chi^{2}$ распределение с $5$-ю степенями
свободы, а $Y_{2}$ - $\chi^{2}$ распределение с $15$-ю степенями
свободы, причем $Y_{1}$ и $Y_{2}$ независимы. Как
распределена их сумма? }
\solution{ }
\problem{  На плоскости выбирается точка со случайными координатами. Абсцисса
и ордината независимы и распределены $N(0;1)$. Какова вероятность
того, что расстояние от точки до начала координат будет больше
2,45? }
\solution{ }
\problem{  
Пусть  $X\sim N(0;1)$. Найдите $P(X>0,5)$, $P(-1<X<2)$, $P(X^{2}
>3)$,  $P(X<0,3)$,
$P(|X|<0,8)$}
\solution{ }
\problem{
 Пусть  $X\sim N(4;9)$,  $Y\sim
N(-5;16)$,  $Z\sim N(20;100)$ ;  $X$,  $Y$ и $Z$  - независимы.
Найдите  $P(X>8)$,  $P(X\in \left[1;5\right])$, $P(Y\in
[-10;-3))$, $P(Z>100)$, $P(X+Y>3)$, $P(|Z|>10)$,
$P(4Y+Z>15)$.}
\solution{ }
\problem{  
Монетку подбрасывают 1000 раз. Пусть  $S$  - общее количество
<<орлов>>. Найдите  $P(S>550)$, $P(S<480)$,
$P(S<400)$. }
\solution{ }
\problem{  
Доходность акций компании А представляет собой с.в. $X\sim
N(50;5^{2} )$, а доходность акций компании B — с.в. $Y\sim
N(80;9^{2} )$. Определите вероятность того, что средний доход по
пакету из восьми акций А и двух акций В составит
не менее 75. Известно, что  $Cor(X,Y)=-0,4$. }
\solution{ }
\problem{
 Страховая компания заключила 16000 договоров. В среднем
страховой случай наступает у одного человека из 10. Пусть  $S$  -
количество наступивших страховых случаев. Найдите $P(S>1800)$,
$P(1550<S<1650)$,
$P(S<2000)$.}
\solution{ }
\problem{
 Дневные расходы электроэнергии на предприятии - случайная
величина, с матожиданием 1400 КВт и стандартным отклонением 50
КВт. Какова вероятность того, что за 70 дней средние дневные
расходы будут меньше 1340 КВт? больше 1500 КВт? от 1300 КВт до
1500 КВт?}
\solution{ }
\problem{  
Пусть  $X\sim N((\begin{array}{l} {2} \\ {3}
\end{array});(\begin{array}{cc} {9} & {-1} \\ {-1} &
{16} \end{array}))$. Найдите  $E(X_{1} )$, $E(X_{1} +2X_{2} )$,
$Var(X_{1} -X_{2} )$, $P(X_{1}
>X_{2} )$,  $P(2X_{1} +X_{2} <5)$. Как распределена
случайная величина  $X_{1}$  при
условии, что  $X_{2} =6$?  $X_{2} =-3$?}
\solution{ }
\problem{
 Пусть  $X\sim N(7;16)$. Найдите $E(X|X>11)$, $E(X|X<10)$,
$E(X|X\in[0;10])$. }
\solution{ }
\problem{
 В большом-большом городе  $N$  80\% аудиокиосков торгуют
контрафактной продукцией. Какова вероятность того, что в наугад
выбранных 90 киосках более 60 будут торговать контрафактной
продукцией? менее 50? от 40 до 80? от 70 до 75? }
\solution{ }
\problem{ $[$Айвазян, экзамен РЭШ$]$ \\
В поселке 2500 жителей. Каждый из них в среднем 6 раз в месяц ездит в город, выбирая день поездки независимо от других людей. Поезд ходит в город один раз в сутки. \\
а) Какой наименьшей вместимостью должен обладать поезд, чтобы он переполнялся в среднем не чаще 1 раза в 100 дней? \\
б) Сколько в среднем человек будет ехать в таком поезде, если предположить, что при переполнении часть людей полностью откажется от поездки? }
\solution{ 
$X_{i}$ - индикатор того, едет ли сегодня $i$-ый человек \\
$E(X_{i})=\frac{1}{5}$, $Var(X_{i})=\frac{2}{5}$ \\
$P(S>v)=0.01$ \\
$\frac{v-500}{1000}=2.33$ \\
$v=2830$ ??? }

\problem{
Маша никогда не готовится к зачетам и экзаменам и рассчитывает
только на везение. Предположим, что Машина оценка - случайная
величина, с матожиданием 5 баллов и стандартным отклонением 3
балла. Какова вероятность того, что при сдаче 100 экзаменов Машин
средний балл будет меньше 4? больше 6? от 4,8 до
5,1? от 4,9 до 5? }
\solution{ }

\problem{
 Количество опечаток в газете - с.в. с матожиданием 10 и
дисперсией 25. Какова вероятность того, что по 144 газетам среднее
количество опечаток не превысит 11? будет от 10 до 10,5? Будет
больше 9,5? меньше 20? }
\solution{ }

\problem{  [{\it полезно запомнить}] \\
Пусть  $X\sim N(\mu ;\sigma ^{2} )$. Найдите $P(|X-\mu |>2\sigma
)$,  $P(|X-\mu |>3\sigma )$. }
\solution{ }

\problem{
Случайный вектор  $(\begin{array}{cc} {X_{1} } & {X_{2} }
\end{array})$  имеет нормальное распределение с
математическим ожиданием  $(\begin{array}{cc} {2} & {-1}
\end{array})$  и ковариационной матрицей
$(\begin{array}{cc} {9} & {-4,5} \\ {-4,5} & {25}
\end{array})$. Найдите  $P(X_{1} +3X_{2}
>20)$.}
\solution{ }

\problem{ \label{geometric sense of correlation}
Пусть величины $X$ и $Y$ имеют совместное нормальное распределение с нулевым средним, единичной дисперсией и корреляцией $\rho$. \\
a) Представьте $Y$ в виде $Y=aX+bZ$, так, чтобы $Z\sim N(0;1)$, а $Z$ и $X$ не были бы коррелированы. \\
б) Что представляет собой множество $X>0\cap Y>0$ в осях $(X,Z)$? \\
в) Чему равна вероятность $P(XY>0)$? \\
г) Постройте график $P(XY>0)$ как функции от $\rho$ }
\solution{
a) $Y=\rho X+\sqrt{1-\rho^{2}} Z$ \\
б) Угол с градусной мерой $\theta=\pi/2+\arcsin(\rho)$ \\
в) $P(XY>0)=\theta/\pi$ }

\problem{
Пусть величины $X$, $Y$, $Z$ - имеют совместное нормальное распределение, с математическим ожиданием $0$ и некоей ковариационной матрицей $B$. \\
Как зависит от $B$ вероятность $P(XYZ>0)$?}
\solution{ 
Никак. Если рассмотреть величины $-X$, $-Y$, $-Z$, то у них такое же математическое ожидание и такая же ковариационная матрица. Значит $P(XYZ>0)=P((-X)(-Y)(-Z)>0)$. Но эти вероятности в сумме дают 1, значит они равны по 0.5. }

\problem{
Пусть $X$ и $Y$ независимы и имеют стандартное нормальное распределение. Отметим точку с координатами $(X,Y)$ на плоскости. Пусть $Z$ - квадрат расстояния до начала координат, а $W$ - угол с осью абсции. \\
а) Как распределены $Z$ и $W$? \\
б) Независимы ли они?}
\solution{ $Z\sim\chi_{2}^{2}$, $W\sim U[0;2\pi]$, независимы }

\problem{
Ползи улитка по склону Фудзи, ползи улитка до самых высот... Басе (?)\\
По склону горы Фудзи ползет улитка. Каждое утро она принимает
решение либо ползти вверх (с вероятностью 0,9), либо целый день
спать. Если улитка спит, то она во сне сползает вниз на 2 м.
Какова вероятность того, что за полгода улитка достигнет
вершины Фудзи? \\
Тем, кто случайно забыл, напомним: \\
Высота Фудзи - 3770 м. Скорость виноградной улитки - 7 см в
минуту. \\
Допустим, что приняв решение ползти вверх, улитка ползет вверх 7
часов, а остальное время любуется видами (не сползая при этом
вниз). }
\solution{ }

\problem{  [Старый знакомый]\\
Известна ф. плотности:  $p_{X}(t)=c\cdot \exp (-8t^{2} +5t)$. Найдите $E(X)$,  $\sigma _{X} $. }
\solution{ выделяем полный квадрат, $E(X)=\frac{5}{16}$, $\sigma_{X}=\frac{1}{4}$  }


\problem{
Предположим, что каждый пятый горожанин предпочитает
эскимо. Сколько горожан следует опросить, чтобы вероятность того,
что выборочная доля горожан, предпочитающих эскимо, отличалась от
истинной доли менее чем на 0,05, равнялась 0,8? }
\solution{ }
\problem{  
Пусть  $X\sim N(0;1)$. Выпишите ф. плотности $p(x|X>1)$, $p(x|X<1)$,  $p(x|X\in\left[0;1\right])$. }
\solution{ }
\problem{
Пусть доналоговая прибыль предпринимателя равна  $10X+100$, где  $X\sim N(0;1)$. Если прибыль меньше 110, то налог отсутствует, если прибыль больше 110, то налог равен 10\%. Найдите ожидаемую посленалоговую прибыль. }
\solution{ }

\problem{
Пусть  $X\sim N(10;16)$. Найдите  $a$  в случаях:  $P(X>a+1)=0,5$, $P(|X-10|<a)=0,25$.}
\solution{ }

\problem{
Считая вероятность рождения мальчика равной 0,51, вычислите вероятность того, что среди 10000 новорожденных мальчиков будет больше, чем девочек. }
\solution{ }

\problem{ Пусть вероятность выпадения монетки <<орлом>> равна 0,63.\\
a) Какова вероятность, что в 100 испытаниях выборочная доля
выпадения орлов будет отличаться от истинной менее, чем на 0,07? \\
b) Каким должно быть минимальное количество испытаний, чтобы
вероятность отличия менее чем на 0,02 была больше 0,95? }
\solution{ }

\problem{
Каждый из 160 абонентов шлет в среднем 5 смс в сутки. Какова вероятность того, что за двое суток они пошлют в сумме более 1700 сообщений? }
\solution{ }
\problem{
Пусть $X \sim N( {0,\sigma ^2 } )$. \\
а) Найдите функцию плотности $|X|$ \\
б) Найдите $E(|X|)$ (можно найти без функции плотности). }
\solution{ }
\problem{
Определите математическое ожидание и дисперсию случайной величины,
если ее функция плотности имеет вид $p(t)=c\cdot \exp
(-2\cdot (t+1)^{2} )$. }
\solution{ }
\problem{  
Пусть  $X_{1} $,  $X_{2} $,  $X_{3}$  - iid, $N(0;1)$. Для с.в.
$Y=X_{1} \cdot X_{3} +X_{2} $, $W={X_{1} +X_{2} \mathord{\left/
{\vphantom {X_{1} +X_{2} (|X_{3} |+1)}} \right.
\kern-\nulldelimiterspace} (|X_{3} |+1)} $, $Q={(X_{1} +X_{2}
)\mathord{\left/ {\vphantom {(X_{1} +X_{2} ) (|X_{3} |+1)}}
\right. \kern-\nulldelimiterspace} (|X_{3} |+1)}$ найдите условные
ф. плотности $p(y|x_{3} )$,
$p(w|x_{3} )$, $p(q|x_{3} )$. }
\solution{ }
\problem{
Известно, что  $\ln Y\sim N(\mu ;\sigma ^{2} )$. Найдите  $E(Y)$,  $Var(Y)$. }
\solution{ }
\problem{
 Пусть  $X\sim N(0;1)$. Найдите ф. плотности
для $Y=|X|$,  $E(Y)$. Верно ли, что  $\max
\left\{X,0\right\}=0,5(|X|+X)$? Найдите $E(\max (X,0))$. Как
изменится ответ, если
$X\sim N(0;\sigma ^{2} )$? }
\solution{ }
\problem{  [т] \\
Пусть  $\vec{X}\sim N((\begin{array}{l} {2} \\
{3} \end{array});(\begin{array}{cc} {9} & {-1} \\ {-1} & {16}
\end{array}))$. Как выглядит (с точностью до константы) функция
$p(x_{1} |x_{2} )$? Найдите $E(X_{1} |X_{2} =4)$, $E(X_{1} |X_{2}
=a)$
}
\solution{ }

\problem{
Каждый день цена акции равновероятно поднимается или опускается на
один рубль. Сейчас акция стоит 1000 рублей. Введем случайную
величину  $X_{i} $, обозначающую изменение курса акции за  $i$ -ый
день. Найдите  $E(X_{i} )$  и $Var(X_{i} )$. С помощью центральной
предельной теоремы найдите вероятность того, что через сто дней
акция будет стоить больше
1030 рублей. }
\solution{ }

\problem{
Сейчас акция стоит 100 рублей. Каждый день цена может равновероятно либо возрасти на 8\%, либо упасть на 5\%. \\
a) Какова вероятность того, что через 60 дней цена будет больше 170 рублей? \\
б) Чему равно ожидаемое значение цены через 60 дней? }
\solution{ }
\problem{
В среднем 20\% покупателей супермаркета делают покупку на
сумму свыше 500 рублей. Какова вероятность того, что из 200
покупателей
менее 81\% сделают покупку на сумму не более 500 рублей? }
\solution{ }
\problem{
В самолете пассажирам предлагают на выбор <<мясо>> или <<курицу>>. В самолет 250 мест. Каждый пассажир с вероятностью 0.6 выбирает курицу, и с вероятностью 0.4 - мясо. Сколько порций курицы и мяса нужно взять, чтобы с вероятностью 99\% каждый пассажир получил предпочитаемое блюдо, а стоимость <<мяса>> и <<курицы>> для компании одинаковая? \\
Как изменится ответ, если компания берет на борт одинаковое количество <<мяса>> и <<курицы>>? }
\solution{ 
$K=170$, $M=120$ (симметричный интервал) или $K=M=168$ (площадь с одного края можно принять за 0) \\
Вариант: театр, два входа, два гардероба а) только пары, б) по одному }

\problem{
 The\footnote{Problems are shamelessly
borrowed from Newbold} mean selling price of new homes in a city
over a year was 115000\$. The population standard deviation was
25000\$. A random sample of 100 new homes sales from this city was
taken. What is the probability that the sample mean selling price
was more than \$ 110000? between \$113000 and \$117000?between
\$114000 and \$116000? }
\solution{ }
\problem{  
The number of hours spent studying by students on a large
campus in the week before final exams follows a normal
distribution with standard deviation 8,4 hours. A random sample of
these students is taken to estimate the population mean number of
hours studying. How large a sample is needed to ensure that the
probability that the sample mean differs from the population mean
by more than 2,0 hours is less than 0,05? }
\solution{ }
\problem{
 A corporation receives 120 applications for positions from
recent college graduates in business. Assuming that these
applicants can be viewed as a random sample of all such graduates,
what is the probability that between 35\% and 45\% of them are
women if 40\% of all recent college graduates in business are
women? }
\solution{ }
\problem{
 A video rental chain estimates that annual expenditures of
members on rentals follow a normal distribution with mean \$100.
It was also found that 10\% of all members spend more than \$130
in a year. What percentage of members spend more than \$140 in a
year? }
\solution{ }
\problem{
 It was found that 80\% of seniors at a particular college
had accepted a job offer before graduation. For those accepting
offers, salary distribution was normal with mean \$29000 and
standard deviation \$4000. For a random sample of sixty seniors,
what is the probability that less than 70\% have accepted job
offers? For a random sample of six seniors who have accepted job
offers, what is the probability that the average salary is more
than \$30000? A senior is chosen at random. What is the
probability that he or she has accepted a job offer with a salary
of more than \$30000? }
\solution{ }
\problem{
Портфель страховой компании состоит из 1000 договоров, заключенных
1 января и действующих в течение года. При наступлении страхового
случая по каждому из договоров компания обязуется выплатить 30
тыс. рублей. Вероятность наступления страхового события по каждому
из договоров предполагается равной 0,05 и не зависящей от
наступления страховых событий по другим контрактам. Каков должен
быть  совокупный размер резерва страховой компании для того, чтобы
с вероятностью 0,95 она могла бы удовлетворить
требования, возникающие по указанным договорам? }
\solution{ }

\problem{
В данном регионе кандидата в парламент Обещаева И.И.
поддерживает 60\% населения. Сколько нужно опросить человек, чтобы
с вероятностью 0,99 доля  опрошенных избирателей, поддерживающих
Обещаева И.И.,  отличалась от 0,6 (истинной доли) менее, чем на
0,01? }
\solution{ }

\problem{
Обозначим долю людей, предпочитающих мороженое с шоколадной
крошкой буквой $p$. Чтобы оценить ее, Вася и Петя опросили 100
человек. Затем Вася ушел домой,
а Петя опросил еще 200 человек. \\
Какова вероятность того, что Васин результат будет отличаться
от Петиного более, чем на четыре процентных пункта, если $p=0.6$? \\
При каком $p$ эта вероятность будет максимальной?}
\solution{ }

\problem{
На плоскости случайным образом выбираются 3 точки: $A$, $B$ и $C$. Абсциссы и ординаты независимы и нормально стандартно распределены. Какова вероятность того, что $C$ лежит внутри круга с диаметром $AB$? }
\solution{ }

\problem{
Найдите $E(max\{X,Y\})$, где $X$ и $Y$ нормально стандартно распределены и независимы. }
\solution{ }

\problem{
Data from a large population indicate that the heights of mothers and daughters in this population follow the bivariate normal distribution with correlation 0.5. Both variables have mean 5 feet 4 inches, and standard deviation 2 inches. \\
a) Among the daughters of above average height, what percent were shorter than their mothers? \\
b) Amont the daughters of above average height, what percent have an above average mother? \\
Comment: may \ref{geometric sense of correlation} be of use? }
\solution{ 
Можно проигнорировать среднее и дисперсию, тогда задача примет вид: \\
$D$ и $M$ распределены $N(0;1)$ и имеют корреляцию $0.5$ \\
Вопрос: $P(D<M|D>0)$ или $P(M>0|D>0)$ \\
Можно взять требуемый двойной интеграл перейдя к полярным координатам (что долго) или перейдя к двум независимым нормальным}

\problem{
Пусть $X_{1}$ и $X_{2}$ имеют совместное нормальное распределение, причем каждая $X_{i}\sim N(0;1)$, а корреляция равна $\rho$. \\
а) Выпишите в явном виде (без матриц) совместную функцию плотности \\
б) Пусть $\rho=0.5$. Какое условное распределение имеет $X_{1}$ при условии, что $X_{2}=-1$? }
\solution{ }

\problem{
Известно, что $X$ и $Y$ нормальны в совокупности (вектор $(X,Y)$
имеет двумерное нормальное распределение). Также известно, что
$E(X+Y)=10$, $E(X-Y)=30$, $Var(X)=Var(Y)=4$ и $Var(X+Y)=6$. \\
а) Найдите $E(X)$, $E(Y)$, $Cov(X,Y)$, $Cov(3X,-6Y)$ и $Corr(X,Y)$\\
б) Найдите $P(X-2Y>48)$ }
\solution{ 
a) $E(X)=20$, $E(Y)=-10$, $Cov(X,Y)=-1$, $Cov(3X,-6Y)=18$ и $Corr(X,Y)=-\frac{1}{4}$ \\
b) $E(X-2Y)=40$, $Var(X-2Y)=24$}

\problem{
Пусть $Z\sim N(0,1)$. \\
Докажите, что $E\left(e^{yZ-\frac{1}{2} y^{2}} 1_{\{Z+z\geq 0
\}}\right)=F_{Z}(y+z)$ \\ }
\solution{ }

\problem{
Складывают $n$ чисел. Перед сложением каждое число округляют до ближайшего целого. Появляющуюся при этом ошибку можно считать равномерно распределенной на отрезке $[-0.5;0.5]$. Определите, сколько чисел складывают, если с вероятностью $\frac{1}{2}$ получаемая сумма отличается от настоящей больше чем на 3 (в любую сторону). }
\solution{ }

\problem{ Let X follow a N(0,1), and let N be the cdf of X. Compute A = E(N(X+1)).\\
Compute E(N(X)) (easy) }
\solution{ 
Let $n(x)$ be the pdf of $X$, namely $1/\sqrt(2\pi) exp(-x^{2}/2)$.\\
$A = \int_{x=-\infty}^{+\infty} N(x+1) n(x) dx$ \\
= int int n(y) n(x) dx dy, x = -infinity...infinity, y = -infinity .. x+1\\
= int F(x,y), (x,y) in D(x,y)\\
where, F(x,y) = n(x)n(y), and D(x,y) is the domain of integration, i.e. {(x,y) / y < x+1 }\\
Now, moving to polar coordinates, we see that...\\
A = int 1/2pi exp(-r?/2) r dr dtheta, over a domain D'\\
... the function being integrated does not depend on theta, i.e. is invariant by rotation.\\

Coming back to cartesian coordinate, and rotating D yields:\\
A = int F(x,y) over D'', where D'' = {(x,y), x > -sqrt(2)/2 }\\
and thus, A = 1- N(sqrt(2)/2).}


\problem{
Пусть $U_{1}$ и $U_{2}$ независимы и равномерны на $[0;1]$. \\
$X_{1}=cos(2\pi U_{1})\sqrt{-2ln(U_{2})}$,
$X_{2}=sin(2\pi U_{1})\sqrt{-2ln(U_{2})}$ \\
Как распределена пара $X_{1}$ и $X_{2}$? }
\solution{Ответ: независимы стандартные нормальные }
% две равномерных в две нормальных 

\problem{
Пусть $X$ и $Y$ стандартные нормальные независимые величины. \\
Найдите вероятность того, что точка с координатами $(X,Y)$ лежит внутри окружности радиуса $t$ с центром в начале координат.}
\solution{ 
 $p=1-e^{-\frac{t^{2}}{2}}$ }

\problem{ Эллипс рассеивания \\
Пусть $X$ и $Y$ - двумерно нормально распределены, $E(X)=E(Y)=0$, $Var(X)=Var(Y)=1$, $Cov(X,Y)=r$ \\
а) Зависимы ли $X+Y$ и $X-Y$? \\
б) Придумайте $c_{1}$ и $c_{2}$ так, чтобы $X'=c_{1}(X+Y)$ и $Y'=c_{2}(X-Y)$ были стандартными нормальными \\
в) Найдите $P(X^{2}-2rXY+Y^{2}le t^{2}(1-r^{2})$ (да помогут вам $X'$, $Y'$ и предыдущая задача) \\
г) Изобразите множество $X^{2}-2rXY+Y^{2}le a^{2}(1-r^{2}$ на плоскости }
\solution{ 
а) нет \\
б) $\frac{1}{\sqrt{2(1+r)}}$, $\frac{1}{\sqrt{2(1-r)}}$ \\
в) $p=1-e^{-\frac{t^{2}}{2}}$ }


\problem{
Вася скачивает новый фильм объемом 1 гб с пиратского сайта. Скорость скачивания каждые 5 секунд меняется случайным образом. Средняя скорость скачивания равна 300 кб в секунду, стандартное отклонение скорости равно 20 кб. \\
Какова вероятность того, что Вася будет скачивать фильм больше часа? \\
(выверить цифры) \\
добавка (идея)-?: разрывы связи - пуассоновский процесс (допустим, независимый от скорости) с параметром $\lambda$ \\
При разрыве связи надо начинать с начала.\\
Сколько (в среднем) будет скачивать? \\
Сколько (в среднем) будет разрывов связи? }
\solution{ }

\problem{
Дано: Трехмерное нормальное (ков. матрица, вектор средних). Найти условное распределение $X_{1}$ при известных $X_{2}$ и $X_{3}$. }
\solution{ }


\problem{Упражнение на ЦПТ

Пусть $X_1,X_2,\ldots,X_{100}$ --- независимые и одинаково распределенные величины, имеющие равномерное распределение на [0;1].

$Y=X_1\cdot X_2\cdots X_{100}$. Найдите (приближенно) медиану распределения $Y$.

Источник: Алексей Суздальцев}

\solution{$e^{-100}$}

\problem{Еще упражнение на ЦПТ

Вычислить предел
$$\lim_{n\to\infty}\sum_{k=0}^{\left[\frac{n+\sqrt{n}}{2}\right]}\frac{C_n^k}{2^n}.$$

Источник: Алексей Суздальцев}
\solution{$\Phi(1)\approx0.8413$.}





\section{Случайное блуждание и броуновское движение}


\subsection{Cлучайное блуждание}
% random_walk
% Задача на принцип отражения и другие свойства сл. блуждания
% кроме задач на первый шаг и разложение в сумму

\problem{
За кандидата $A$ подано $a$ голосов, за кандидата $B$ - $b$
голосов, $a>b$. Во время подсчета голосов бюллетени достают из
урны по одному в случайном порядке. Какова вероятность того, что
на протяжении всего подсчета голосов кандидат $A$ будет впереди
кандидата $B$? }
\solution{ }

\problem{
Рассмотрим симметричное случайное блуждание. $N_{k}$ - количество
посещений точки $k$ до первого возвращения к $0$.
Заметим, что $N_{k}$ может принимать значение $+\infty$. \par
a) Докажите, что $P(N_{k}>0)=\frac{1}{2}\frac{1}{k}$ \par
b) Докажите, что
$P(N_{k}>j+1|N_{k}>j)=\frac{1}{2}+\frac{1}{2}\frac{k-1}{k}$ \par
c) Найдите $E(N_{k})$ \par
d) Найдите $P(N_{k}=+\infty)$ \par
e) Как изменятся ответы, если случайное блуждание будет
несимметричным? (вероятности $p$ и $q=1-p$) \par
Source: Steele, 1.3. }
\solution{ 
$P(N_{k}>j+1)=P(N_{k}>j+1\cap N_{k}>j)=P(N_{k}>j+1|
N_{k}>j)P(N_{k}>j)$ \par
$E(N_{k})=P(N_{k}>0)+P(N_{k}>1)+P(N_{k}>2)+...$\par
$E(N_{k})=\frac{1}{2k}+\frac{1}{2k}\left(1-\frac{1}{2k}\right)+\frac{1}{2k}\left(1-\frac{1}{2k}\right)^{2}+...=1$
\par
e) $E(N_{k})=(p/q)^k$ }


\problem{
Частица движется по правильному многоугольнику из $m+1$ вершин,
занумерованных от $0$ до $m$. Стартует она в вершине $0$.
Следующая вершина выбирается равновероятно из двух соседних.
Частица останавливается, когда обойдет все вершины. \par
а) Какова вероятность того, что частица никогда не посетит вершину
$k$? \par
б) Какова вероятность того, что частица закончит свой маршрут в вершине $k$? \par
Source: Ross, example 2.52 }
\solution{ 
$P(\{\text{частица никогда не посетит вершину }k\})=0$ \par
$A_{k}=\{\text{частица закончит свой маршрут в }k\}$ \par
$B_{k}=\{\text{частица когда-нибудь посетит одно из чисел } $k-1$,$k$,$k+1$\}$\par
$P(A_{k})=P(A_{k}\cap B_{k})=P(B_{k})\cdot P(A_{k}|B_{k})=1\cdot
P(A_{k}|B_{k})$ \par
А вероятность $P(A_{k}|B_{k})$ - это вероятность того, что
симметричное случайное блуждание поднимется на $m-1$ раньше, чем
опустится до $-1$. Значит $P(A_{k}|B_{k})=\frac{1}{m}$ \par }

\problem{
Пусть $S_{n}$ - симметричное случайное блуждание, а $\tau$ - время
первого посещения точки 1. С помощью принципа отражения найдите
$P(\tau=2k-1)$. }
\solution{
$P(\tau=2k-1)=P(S_{2k-1}=1 \cap \{S_{t}<1|t<2k-1\})$ \par
Рассмотрим случай $2k-1\ge 3$. Тогда предыдущие два шага частица
сделала вверх, что происходит с вероятностью $2^{-2}$. \par
$P(S_{2k-1}=1 \cap \{S_{t}<1|t<2k-1\})=2^{-2}\cdot P(S_{2k-3}=-1
\cap \{S_{t}<1|t<2k-3\})$ \\
Требуется вероятность закончить в точке $-1$, не касаясь точки
$1$. Траекторий, касающихся $1$, и заканчивающихся в $-1$ ровно
столько сколько траекторий, заканчивающихся в $3$. Следовательно,
\par
$P(\tau=2k-1)=2^{-2}\left(P(S_{2k-3}=-1)-P(S_{2k-3}=3)
\right)=2^{-(2k-1)}\left(C_{2k-3}^{k-1}-C_{2k-3}^{k}\right)$ \par
После арифметических преобразований получаем: \par
$P(\tau=2k-1)=2^{-2k}\frac{1}{2k-1}C_{2k}^{k}$ \par
Эта формула верна и для $2k-1<3$ }


\problem{Доходность акции следует симметричному дискретному случайному блужданию. Какова вероятность того, что в момент времени $2k+1$ доходность будет выше, чем когда-либо в прошлом?

Источник: Алексей Суздальцев}
\solution{$\frac{C_{2k}^{k}}{2^{2k+1}}$. Совсем простого решения не знаю, хотя ответ простой.}



\subsection{Броуновское движение}



\section{Оптимизация, Теория игр}
% optimization

\problem{ \label{optimalnoe raskladivanie}
Есть две пустые урны, 50
белых и 50 черных шаров. Сначала наугад выбирается одна из урн,
затем из нее выбирается один шар наугад. Как следует разложить
шары по урнам (в каждой урне должен лежать хотя бы один шар),
чтобы вероятность вытащить белый шар была
максимальной? }
\solution{ Один белый отдельно, остальные 99 шаров вместе. }

\problem{ $[$Mosteller$]$\\
Чтобы подбодрить сына, делающего успехи в игре в теннис, отец
обещает ему приз, если он выиграет подряд по крайней мере две
теннисные партии против своего отца и клубного чемпиона по одной
из схем: отец - чемпион - отец или чемпион - отец - чемпион по
выбору сына. Чемпион играет лучше отца. Какую схему следует
выбрать сыну? }
\solution{ }
\problem{ $[$Mosteller$]$ Выборка с возврашением или без возвращения? \\
Две урны содержат красные и черные шары, не различимые на ощупь.
Урна А содержит 2 красных и 1 черный шар, урна В-101 красный и 100
черных шаров. Наудачу выбирается одна из урн, и вы получаете
награду, если правильно называете урну после вытаскивания двух
шаров из нее. После вытаскивания первого шара и определения его
цвета вы решаете, вернуть ли в урну этот шар перед вторым
вытаскиванием. Какой должна быть ваша стратегия? }
\solution{ }
\problem{
There are n different pairs of sox in a drawer. One sock is taken
out at a time until a matching pair has been found. \\
a) If n is known what is the most likely value of r, the number of
sox drawn out? \\
b) If r is known what value of n gives the greatest chance of this
happening? \\
Ugly answers? }
\solution{ }
\problem{ Четыре шкатулки \\
Имеется 4 внешне одинаковых шкатулки. Внутри каждой шкатулки написан ее номер. Внутри шкатулки номер 4 также лежит 1 млн. рублей. Вы играете в следующую игру: \\
Вы выбираете шкатулки одну за одной. После выбора шкатулки, Вы можете либо забрать ее, либо продолжить игру. Игра заканчивается в тот момент, когда Вы решаете забрать шкатулку или когда оказывается, что номер внутри выбранной только что шкатулки меньше, чем номер внутри предыдущей шкатулки. Номера внутри шкатулки определяет ведущий, Вы же не знаете содержимого шкатулок.\\
Вашем выигрышем является содержимое последней выбранной шкатулки. \\
Какова оптимальная стратегия? \\
Каков ожидаемый выигрыш при оптимальной стратегии?\\
Source: www.wilmott.com-forum-brainteasers}
\solution{ }
\problem{
Пусть $A$ - корреляционная матрица. Чему равен наибольший возможный  определитель? 

Source: wilmott }

\solution{ $Trace(M)=\sum(eigenvalues)$, Det(M)=product(eigenvalues)\\
For correlation matrix, since trace=N, where N dimension average(eigenvalues)=1, Then,
log(Det)=sum(log(eiegenvalues))=N*average(log(eigenvalues))<=N*log(average(eigenvalues))=0, since log has negative convexity

Hence Det<=1}

\problem{
$[$rather hard$]$
I am trying to find someone whom I know is in a 5-story building, say a bookstore, and based on what I know about her, the probabilities of finding her on each floor are:\\

5th: 15\% \\
4th: 40\%\\
3rd: 10\%\\
2nd:30\%\\
1st: 5\%\\

I start from the 1st floor, and it takes me 1 minute to take an escalator up or down to an adjacent floor, and 5 minutes to completely search a floor, after which I will either have found her and we can leave, or I will know she is not on that floor. Of course, I can stop and search the floors in any order, so the time I spend searching does not have to follow in adjacent floors, but I can only travel between adjacent floors.\\

Assuming she stays where she is, in which order should I search the floors to minimize my expected time until finding her.}

\solution{ 2, 4, 5, 3, 1 has an expected time of 13.8 minutes, this is the minimum.

My algorithm for solving requires some guesswork. Take everything one decision at a time:

(1) Search the first floor or go up? If I search the first floor, 95\% of the time I add five minutes to my time. 5\% of the time save the amount of time it takes to search floors 2 to 5. I know I can search these floors in 24 minutes, 5\% of 24 is less than 95\% of 5, so I skip the first floor.

(2) Search the second floor or go up? If I skip this floor, I clearly have to go all the way to four, since it wouldn't make sense to skip 2 but search 3. If I do that and don't find her, I have to come back to 2. Doing 2 first saves four minutes of travel time and 10\% of the time costs five minutes. Clearly, I search the second floor.

(3) Search the third floor or go right up to four? This is a little more complicated. If I skip the third floor, I clearly have to go four, five then three. If I search the third floor, then I do four and five. Doing four and five takes 14 minutes if she is on three. Searching the third floor costs me five minutes 6/7 of the time and saves me 14 minutes 1/7 of the time; clearly I skip.

(4) Now of course I search the fourth floor. After that I clearly prefer the fifth floor to the third, because it has a higher probability and the cost of skipping it is higher. On my way back down, of course I search the third floor, because it has a higher probability than one and a shorter travel time. After that, only one is left. }

\problem{ Grimmett, 2.7.16. Transitive coins\\
Есть три неправильные монетки ($A$, $B$, $C$), выпадающие с
вероятностью $3/5$ на <<орла>>. Игроки по очереди выбирают себе
монетку. Затем подкидывают. У кого больше очков, тот и выиграл.
Очки считаются так: \\
монетка $A$ - 10 очков за <<решку>> и 2 очка за <<орла>> \\
монетка $B$ - 4 очка за <<решку>> и 4 очка за <<орла>> \\
монетка $C$ - 3 очков за <<решку>> и 20 очков за <<орла>> \\
У какого игрока больше шансов выиграть в этой игре, если каждый
игрок максимизирует вероятность выигрыша? }
\solution{ }

\problem{
If X, Y, and Z are 3 random variables such that X and Y are 90\% correlated, and Y and Z are 80\% correlated, what is the minimum correlation that X and Z can have? }
\solution{ }

\problem{
Имеется дерево (граф) с 2006 ребрами. А и В - это две вершины этого дерева. Мы движемся случайным образом от А к В не поворачивая назад. На каждой развилке мы выбираем равновероятно каждое из возможных ребер. \\
а) При какой форме графа и при каком выборе точек А и В вероятность попасть из А в В будет минимальной? \\
б) Чему будет равна эта вероятность? }
\solution{ }

\problem{
Есть три рулетки: на первой равновероятно выпадают числа 2, 4 и 9;
на второй - 1, 6 и 8; на третьей - 3, 5 и 7. Сначала первый игрок
выбирает рулетку себе, затем второй игрок выбирает рулетку себе.
После этого рулетки запускаются, и выигрывает тот, чья рулетка
покажет большее число. \\
Каким игроком лучше быть в этой игре? Почему? }
\solution{вторым }

\problem{
Вы хотите приобрести некую фирму. Стоимость фирмы для ее
нынешних владельцев - случайная величина, равномерно
распределенная на отрезке [0;1]. Вы предлагаете владельцам продать
ее за называемую Вами сумму. Владельцы либо соглашаются, либо нет.
Если владельцы согласны, то Вы платите обещанную сумму и получаете
фирму. Когда фирма переходит в Ваши руки, ее стоимость сразу
возрастает на
20\%. \\
а)  Чему равен Ваш ожидаемый выигрыш, если Вы предлагаете цену
0,5? \\
б)  Какова оптимальная предлагаемая цена? }
\solution{ }

\problem{ Дама пик\\
Устав от вереницы женихов, разборчивая невеста уединилась. Перед
ней колода из 36 карт, хорошо перемешанная. На этот раз принцесса
должна предсказать появление дамы пик. То есть принцесса открывает
одну за одной карты из колоды и в любой момент может остановиться
и сделать заявление <<Следующая карта будет дама пик>>. Если
это окажется правдой, то
принцесса выиграла.
Оптимальная стратегия? }
\solution{ 
игра не меняется, если делать ставку на последнюю карту. \\
все стратегии оптимальны и дают выигрыш в $1/36$ 
}

\problem{ Пьяный водитель \\
На шоссе подряд идут два поворота. Внешне повороты не отличимы
друг от друга. Будучи слегка <<навеселе>> Вася возвращается домой.
Васе нужен второй по счету поворот, но из-за своего состояния он
не может определить, какой по счету поворот он проезжает. Поэтому
он поворачивает с вероятностью $p$, а с вероятностью $q=1-p$ едет
прямо на каждом повороте. Если Вася повернет на первом повороте,
то заедет в соседнюю деревню и получит полезность 2. Если он
свернет на втором повороте, то получит полезность 5. Если Вася
проедет оба поворота, то получит полезность 1. \\
Найдите оптимальное $p$. }
\solution{ }

\problem{ Ящик с носками \\
В ящике лежат красные и черные носки. Если из ящика наудачу
вытягиваются два носка, то вероятность того, что оба они красные,
равна 1/2. \\
а) Каково минимальное возможное число носков в ящике? \\
б) Каково минимально возможное число носков  в ящике,  если число
черных носков четно? }
\solution{ }

\problem{ \label{pop i balda}Mechanism design \\
Поп нанял Балду, чтобы тот предсказывал ему погоду. Дождь идет с вероятностью $p$. Благодаря народным приметам
Балда точно знает $p$ накануне вечером. Балда выдает Попу
свою оценку $\hat{p}$ вероятности дождя завтра. \\
а) Как будет вести себя Балда, если Поп выплачивает ему
награждение по принципу: если дождь действительно был, то
выплачивается $\hat{p}$, если дождя не было, то
$1-\hat{p}$? \\
б) Поп решил заставить Балду выдавать точные оценки. В случае
дождя Поп платит $f(\hat{p})$, и $f(1-\hat{p})$ в случае
сухой погоды. Какую $f$ следует выбрать Попу? \\
в)??? \\
Подсказка: при решении задачи Поп столкнется со странным
дифференциальным уравнением, у которого много решений, но 9-ти
классов церковно-приходской школы ему вполне должно хватить... \\
Коммент: в условии задачи опечатка - в ЦПХ было 4 класса. Хотя кто
знает, когда там проходили диф. уры? }
\solution{ Чтобы максимум $p_{i}f(\hat{p}_{i})+(1-p_{i})f(1-\hat{p}_{i})$ был
в точке $\hat{p}_{i}=p_{i}$ необходимым условием будет:
$pf'(p)=(1-p)f'(1-p)$. \\
Если левую часть обозначить $q(p)$, то получаем уравнение
$q(p)=q(1-p)$. Берем любую функцию, симметричную относительно
$1/2$ (останется потом только проверить, что $\hat{p}_{i}=p_{i}$ -
это максимум, а не минимум). Например, подойдет $q(x)=1$, тогда
получаем $f(x)=\ln(x)$, или $q(x)=2x(1-x)$, тогда получаем
$f(x)=-x^{2}+2x$ }

\problem{ \label{optimalnaia f plotnosti} Поиск функции плотности \\
Два спортсмена готовятся к соревнованиям. Каждый из них выбирает
свой уровень усилий $e_{i}\in[0;1]$. Побеждает тот, кто приложил
больше усилий при подготовке. Если оба приложили одинаковое
количество усилий, то не побеждает никто. Победитель получает
платеж равный 1. Издержки по приложению усилий равны
$C_{i}=2e_{i}^{2}$ для каждого игрока. \\
а) Существует ли равновесие по Нэшу в чистых стратегиях? \\
б) Найдите равновесие по Нэшу, в котором уровень усилий каждого
игрока имеет функцию плотности $p(t)$, отличную от нуля на $[a;b]$. }
\solution{ а) равновесия в чистых нет. \\
б) Первый игрок должен быть безразличен между усилиями:
$e_{1}\in [a;b]$. Т.е. $U(e_{1})=U(a)=U(b)$. \\
Если первый игрок выбирает уровень усилий $e_{1}$, то он
выигрывает с вероятностью $\int_{0}^{e_{1}}p(t)dt$.
Следовательно: \\
$\int_{0}^{e_{1}}p(t)dt-2e_{1}^{2}=0-2a^{2}=1-2b^{2}$ \\
Поскольку есть стратегия $e_{1}=0$, приносящая полезность 0, любая
играемая стратегия должна приносить платеж не меньше 0. \\
Отсюда $a=0$ и $b=1/\sqrt{2}$.
Взяв производную по $e_{1}$ получаем: \\
$p(t)=4t$ на отрезке $[0;1/\sqrt{2}]$. }

\problem{
Энтропией с.в. $X$  называется число $E\left( {\log _2 \frac{1}
{{p( X )}}} \right)$, где $p( t ) = P( {X = t} )$ для дискретных
с.в. или
 $p( t )$ - функция плотности для непрерывных с.в. \\
 Энтропия
измеряет количество информации (в битах), получаемое при
наблюдении с.в. $X$. Пусть имеется монетка, выпадающая гербом с
вероятностью $\theta $. С.в. $X$ равна единице, если монетка
выпадает гербом, и нулю в противном случае. \\
При каком $\theta$  энтропия будет максимальной? }
\solution{ }

\problem{  \label{raznie pravila}
Маша и Саша играют в быстрые
шахматы. У них одинаковый класс игры и оба предпочитают играть
белыми, т.е. вероятность выигрыша того, кто играет белыми равна
$p>0,5$. Партии играются до 10 побед. Первую партию Маша играет
белыми. Она считает, что в каждой последующей партии белыми должен
играть тот, кто выиграл предыдущую партию. Саша считает, что
ходить белыми нужно по очереди. При каком варианте правил у Маши
больше шансы выиграть?}
\solution{(C.L. Anderson) \\
При любом варианте правил, Маша будет ходить белыми не больше 10
раз, а Саша - не больше 9 раз. Победитель гарантированно
определяется за 19 партий. Если победитель определился раньше, то
дополним турнир недостающими фиктивными партиями (чтобы Саша ходил
белыми ровно 9 раз, а Маша - ровно 10 раз). В турнире из 19 партий
победителем при любом варианте правил оказывается тот, кто выиграл
больше партий. Следовательно, Машины шансы не зависят от
выбираемого варианта правил.  }

\problem{ О пользе гадания на кофейной гуще замолвите слово… [Winkler] \\
Маша пишет на бумажках два любых различных натуральных числа по
своему выбору. Одну бумажку она прячет в левую руку, а другую - в
правую. Саша выбирает любую Машину руку. Маша показывает число,
написанное на выбранной бумажке. Саша высказывает свою догадку о
том, открыл ли он большее из двух чисел или меньшее. Если Саша не
угадал, то Маша выиграла. Перед выбором руки и высказыванием
догадки Саша может обратиться к потомственной гадалке в пятом
поколении Глафире Лукитичне (500\% гарантия, снятие порчи и сглаза
без греха и ущерба для здоровья, исправляет некачественную работу
шарлатанов). Глафира Лукитична называет наугад (ничего не зная о
Маше!) одно из чисел $\left\{ {1\frac{1} {2};2\frac{1}
{2};3\frac{1} {2};...} \right\}$
 с вероятностями соответственно равными $\left\{ {\frac{1}
{2};\frac{1} {4};\frac{1} {8};...} \right\}$. Другими словами,
действия Саши (выбор левой или правой руки и высказываемая версия)
могут зависеть от слов гадалки. Какую стратегию Саше следует
выбрать, чтобы гарантировать себе (вне зависимости от действий
Маши!) вероятность выигрыша строго более 50\%?}
\solution{ }

\problem{ Наш суд - самый гуманный суд в мире! \\
Как известно, истина распределена равномерно на отрезке [0;1].
Истец называет число [0;1] - желаемую компенсацию за моральный
ущерб. Одновременно с истцом ответчик называет свою оценку ущерба
(на том же отрезке). Суд обязывает ответчика выплатить истцу
моральный ущерб в том объеме, который оказался ближе всего к
истине. \\
а) Как следует играть истцу и ответчику? \\
б) А если истина распределена нормально $N(0;1)$? }
\solution{ }

\problem{ \label{ugadivanie ravnomernih} Маша и Саша came back! \\
Маша
наблюдает реализацию двух независимых случайных величин $X$ и $Y$,
распределенных равномерно на $[0;1]$. Она выбирает, значение какой
из них рассказать Саше. Саша выигрывает, если угадает, какая из
величин наибольшая (та, значение которой он узнал от Маши, или
другая). Маша выигрывает,
если Саша ошибется. Найдите оптимальные стратегии. }
\solution{Решение (Winkler) \\
Выбирая свой ответ наугад, Саша может гарантировать себе
вероятность выигрыша 50\%. Передавая Саше ту величину, значение
которой легло от $\frac{1}{2}$ далее, Маша лишает Сашу релевантной
для отгадывания информации. }

\problem{  
В мешке лежат бочонки. На каждом из них написана цифра. На одном
бочонке написана цифра 1, на двух бочонках - цифра 2,..., на
девяти бочонках - цифра 9. Маша вытаскивает один бочонок наугад.
Саша не знает, какой бочонок достала Маша. Саша может задавать
Маше вопросы, на которые можно отвечать только <<да>> или
<<нет>>. \\
а) Как выглядит стратегия, минимизирующая ожидаемое число
вопросов, необходимых чтобы угадать цифру? \\
б) Как выглядит стратегия, минимизирующая число вопросов,
достаточных, чтобы угадать цифру в самом неблагоприятном случае? }
\solution{ }

\problem{
Перед Машей колода в 52 карты. Маша открывает карты одну за одной.
Изначально у Маши 1 доллар. Маша может делать любую ставку в
пределах имеющейся у нее суммы на цвет следующей карты. Найдите
оптимальную стратегию и ожидаемый выигрыш, который приносит эта
стратегия. \\
Предположим бесконечную делимость денег. }
\solution{ }
\problem{ $[$Mosteller$]$ Выигрыш в небезобидной игре \\
Игра состоит из последовательности партий, в каждой из которых вы
или ваш партнер выигрывает очко, вы - с вероятностью $p$ (меньшей,
чем 1/2), он - с вероятностью $1-p$. Число игр должно быть четным.
Для выигрыша надо набрать больше половины очков. Предположим, что
вам известно, что $p=0,45$, и в случае выигрыша вы получаете приз.
Если число партий в игре выбирается заранее, то каков будет ваш
выбор? }
\solution{ }
\problem{ Разборчиая принцесса-1 \\
К принцессе случайным образом выстроилась очередь из $n$ женихов.
Цель принцессы - выбрать самого богатого (даже второй по богатству
жених королевства не устраивает принцессу). Женихи заходят по
очереди. Когда заходит очередной претендент, принцесса узнает его
годовой доход и должна либо выбрать его, либо перейти к
следующему. Вернуться к предыдущему нельзя. Предполагается, что
все варианты расположения женихов по доходу равновероятны. \\
а) Найдите  $g_{n}$  - вероятность того, что принцесса выбрала самого
богатого жениха, если известно, что она остановилась на $n$ -ом
претенденте, доход которого был больше, чем у предыдущих
претендентов. \\
б) Обозначим  $h_{n}$  - вероятность выигрыша принцессы в случае,
если она пропускает  $n$  первых женихов, а далее играет наилучшую
стратегию. Докажите, что  $h_{n}$  не возрастает. \\
в)  Как выглядит  $h_{n}$ левее точки $g_{n}=h_{n}$? \\
г) Составьте разностное уравнение на $h_{n}$ правее точки
$g_{n}=h_{n}$. \\
д) Найдите  $\lim_{n\to \infty } h_{0}$. \\
Тигр: Причем здесь $e$? Причем здесь Березовский? }
\solution{ }

\problem{ Разборчиая принцесса-2 \\
Имеется 100 женихов. Доход $i$-го жениха равен $X_{i}$. Величины
$X_{i}$ - iid, равномерны на отрезке $[0;1]$. Женихи заходят по
очереди. Когда заходит очередной претендент, принцесса узнает его
годовой доход и должна либо выбрать его, либо перейти к
следующему. Вернуться к предыдущему нельзя. Принцесса
максимизирует ожидаемую полезность. \\
Как выглядит оптимальная стратегия? }
\solution{ }

\problem{
Подбрасываются два различных неправильных кубика. Возможно ли, что вероятность того, что сумма равна $i$, $p_{i}\in(2/33;4/33)$? \\
Возможно ли получение всех чисел от 2 до 12 сделать равновероятным? \\
Source: IMS2007 }
\solution{Заметим, что $a_{1}b_{6}+a_{6}b_{1}<2a_{1}b_{1}$ и $a_{1}b_{6}+a_{6}b_{1}<2a_{6}b_{6}$. \\
Перемножаем два неравенства, получаем противоречие. }

\problem{
Усама бен Ладен хочет перенести 1000 тротиловых шашек из одной пещеры в другую. При транспортировке каждая шашка взрывается с вероятностью $p$. Если взрывается одна шашка, то взрываются и все остальные, перевозимые вместе с ней. Сам Усама при взрыве всегда чудом остается жив. Какими партиями нужно переносить шашки, чтобы минимизировать среднее число переносок? 

Подсказка: для простоты можно считать, что размер партии не меняется во времени.

Вариация?:\\
source: чудо-сканер у Жени Надоршина, по мотивам истории со сканированием Mikosch }
\solution{ $-\frac{1}{ln(1-p)}\approx \frac{1}{p}$  }

\problem{
Известно, что случайная величина $X$ принимает значения из отрезка $[0;1]$. \\
а) Найдите наибольшую возможную дисперсию. \\
б) Приведите пример случайной величины, имеющей такую дисперсию. \\
Source: aops, 122082}
\solution{Дискретная, 0 и 1 с вероятностью по 0.5 }

\problem{
Петя с Васей играют в крестики-нолики на поле $3\times 3$. Вася начинает игру крестиками. Петя играет в первый раз и поэтому ставит нолики случайным образом в свободные клетки, и Васе известно об этом. Какова должна быть стратегия Васи, чтобы вероятность его выигрыша была наибольшей? Чему будет равна эта вероятность? }
\solution{ центр-угол-далее по смыслу. (?). $p=95/96$. }

\problem{
Человек хочет купить чудо-пылесос, и конечно, подешевле. Посещение каждого магазина связано с издержками, равными $c>0$. Цена чудо-пылесоса в каждом магазине имеет равномерное распределение на отрезке $[0;M]$. Предположим также, что если человек решит вернуться в уже посещенный магазин, то цена там уже могла измениться, и поэтому уже посещенный магазин можно рассматривать как новый. Магазинов, где продается чудо-пылесос, бесконечно много. \\
a) Как выглядит оптимальная стратегия? \\
б) Сколько в среднем магазинов будет посещено при использовании оптимальной стратегии?\\
в) Чему равны ожидаемые издержки покупателя (цена+издержки посещения магазинов)? \\
г) Чему равна ожидаемая цена покупки? \\
д) Верно ли, что зависимости от $c$ и $M$ имеют ожидаемый знак? \\
е) Что изменится, если предположить, что цена в посещенных магазинах не меняется и покупатель имеет возможность вернуться в уже посещенный магазин? }

\solution{ а) Если текущее $p>\bar{p}$, то ищем дальше, если $p<\bar{p}$, то берем \\
Ожидаемые общие издержки удовлетворяют уравнению: \\
$TC=c+P(p<\bar{p})E(p|p<\bar{p})+(1-P(p<\bar{p}))TC$ \\
Отсюда $TC=\frac{\bar{p}}{2}+\frac{cM}{\bar{p}}$ \\
Оптимальное $\hat{p}=\sqrt{2cM}$ \\
б) $\frac{1}{P(p<\bar{p})}$ \\
в) В точке оптимума $TC=\sqrt{2cM}$ \\
е) Ничего. Возвращаться в уже посещенный магазин не имеет смысла при использовании оптимальной стратегии, т.к. желание вернуться означает то, что из него не надо было уходить }


\problem{ Автомат с кофе и горячим шоколадом \\
Горячий шоколад в автомате стоит 15 рублей. Автомат принимает монеты достоинством 1 рубль, 2 рубля и 5 рублей. У меня на руках имеется одна пятирублевая, три двухрублевых и четыре рублевых монеты. Каждая монета может застрять в автомате с вероятностью $p$. Застрявшая монета не засчитывается и обратно не выдается. Я хочу купить горячий шоколад или по крайней мере минимизировать средние потери вызванные застреванием монеты. \\
а) В каком порядке нужно бросать монеты? \\
б) Какими должны быть вероятности застревания монет, чтобы мне было безразлично в каком порядке их кидать? }
\solution{ а) 1, 2, 5 \\
б) вероятности должны быть пропорциональны достоинству монет: \\
Рассмотрим монеты достоинством $1$ и $n$. Пусть они застревают с вероятностями $p$ и $x$. \\
Ожидаемые потери от последовательности $1$, $n$: $p+(1-p)x(1+n)$ \\
Ожидаемые потери от последовательности $n$, $1$: $x+(1-x)p(1+n)$ \\
Они равны при $x=pn$ \\
Если монет больше двух, а вероятности пропорциональны, то две соседние можно переставить местами, что не скажется на ожидаемых потерях}


\problem{
Кинотеатр обещает вручить приз первому человеку в очереди, чей день рожденья совпадает с днем рождения кого-нибудь из впереди стоящих. Допустим, что Вы можете выбрать любое место в очереди. Какое Вы бы выбрали?  }
\solution{оптимальное место находится руками, 20, хотя саму вероятность выигрыша без компьютера посчитать очень трудоемко. }

\problem{ Mahler's theorem and $\pi$ \\
Теорема Малера (1953) утверждает, что для любых натуральных $p>1$ и $q>1$ выполняется неравенство: $|\pi-\frac{p}{q}|>q^{-42}$. У Вас 1 рубль. Теперь представим, что вы делаете ставки на очень отдаленные цифры числа $\pi$. Т.е. казино выбирает, с какого знака числа $\pi$ начинается игра. Вы знаете выбор казино, но это число настолько велико, что Ваших вычислительных возможностей не хватает на вычисление этих знаков. Вы ставите любое количество денег из имеющихся у Вас на любую цифру. Можно поставить на несколько цифр сразу, можно делать сколь угодно мелкую ставку. Ставка сделанная на верную цифру возвращается в 10 кратном размере. Ставка сделанная на неверную цифру уходит к казино. \\
Можете ли Вы гарантировать себе выигрыш в долгосрочном периоде? }
\solution{ Например, из теоремы Малера следует, что начиная с $n$-го знака в $\pi$ не может идти $41n$ нулей подряд. Следовательно делим доллар на $10^{41n}-1$ часть и ставим их на все последовательности, кроме последовательности из нулей.}

\problem{
Известно, что $P(X\in(-1;3))=0$ и $E(X)=0$. Чему равна минимально возможная дисперсия $X$? }
\solution{ }

\problem{
В киосках продается <<открытка-подарок>>. На открытке есть
прямоугольник размером 2 на 7. В каждом столбце в случайном
порядке находятся очередная буква слова <<подарок>> и звездочка.
Например, вот так: \\
\begin{tabular}{|c|c|c|c|c|c|c|}
  \hline
  П & * & * & А & * & О & К \\
  \hline
  * & О & Д & * & Р & * & * \\
  \hline
\end{tabular} \\
Прямоугольник закрыт защитным слоем, и покупатель не видит, где
буква, а где - звездочка. Следует стереть защитный слой в одном
квадратике в каждом столбце. Можно попытаться угадать любое число
букв. Если открыто $n>0$ букв слова 'подарок' и не открыто ни
одной звездочки, то открытку можно обменять на $50\cdot 2^{n-1}$
рублей. Если открыта хотя бы одна звездочка, то открытка
остается просто открыткой. \\
а) Какой стратегии следует придерживаться покупателю, чтобы
максимизировать ожидаемый выигрыш? \\
б) Чему равен максимальный ожидаемый выигрыш? }
\solution{ }

\problem{
В забеге участвуют две лошади, Метель и Пурга. Вася верит, что Метель побеждает с вероятностью 1/2. Петя верит, что Пурга побеждает с вероятностью 1/3. Каждый из них согласен участвовать в споре, если получают положительную ожидаемую прибыль. \\
Можете ли вы гарантировать себе безрисковое получение 1 рубля?  }
\solution{ да, два линейных уравнения }

\problem{
At a horse race, a horse named <<Winner>> has 25\% chance to win and is posted at 4 to 1 odds. (For every dollar gambler bets, he receives \$5 if horse wins and nothing if it loses). If gambler has square root utility function:\\
What fraction of his money should he bet on <<Winner>>?}
\solution{ }

\problem{ Kelly criterion \\
У Васи имеется стартовый капитал в 100 рублей. Вася может сделать любую ставку в рамках своего капитала. Вероятность выигрыша $p$. В случае выигрыша Вася получает свою ставку обратно в удвоенном размере, в случае проигрыша - теряет ставку. \\
а) Какую часть капитала нужно поставить чтобы максимизировать ожидаемый выигрыш? \\
б) Какую часть капитала нужно поставить, чтобы максимизировать ожидаемый логарифм своего итогового капитала? }
\solution{ }

\problem{ Say we have 4 multiple choice questions with 4 choices each. The answers are marked a,b,c,d. We are told that the right answer to each question has a different mark. I mean, one of the answer is a, one is b, one is c, and one is d. \\
The subject is marine biology and we have no idea what the right answers are. \\
a) Is there a way to get a better expected score than by just random guessing?\\
b) What strategy minimize variance of score? Maximize? }
\solution{ }

\problem{ Phone Call \\
Alice tries to call Bob, who is not at home now. For every second, Bob has a probability p of comming back home, and he won't leave once he is back. Alice will lose c dollars per second until she finally reaches Bob through his home-phone. Each phone call costs Alice D dollars, even if nobody answers it. How should she arrange her calls? (Assume Bob immediately picks up the phone when his home-phone rings if he is at home. Neglect the time of bell ringing.) \\
Source: Chenyang's Favorate Problems }
\solution{ }

\problem{ Unfair Coins \\
There are two unfair coins. One coin has .7 probability head-up; the other has .3 probability head-up. To begin with, you have no information on which is which. Now, you will toss the coin 10 times. Each time, if the coin is head-up, you will receive \$1; otherwise you will receive \$0. You can select one of the two coins before each toss. What is your best strategy to earn more money?
Source: Chenyang's Favorate Problems }
\solution{ }

\problem{
Вам сообщают $n$ чисел, вы хотите запомнить из них 100 <<типичных>>. Т.е. ваша задача отобрать 100 чисел, так, чтобы у всех чисел шансы попасть в 100 отобранных были равны. Трудность состоит в том, что числа вам называют по одному и заранее неизвестно, сколько их окажется. А запомнить более 100 чисел вы не можете. Как отобрать числа?}
\solution{ 
Первые 100 запоминаем. Число номер $k$ берем с вероятностью $1/k$, при этом равновероятно забываем одно из уже набранных. То, что процедура годится доказывается по индукции. }

\problem{
Начинающая певица дает концерты каждый день. Каждый ее концерт приносит продюсеру 0.75 тысяч евро. После каждого концерта певица может впасть в депрессию с вероятностью 0.5. Самостоятельно выйти из депрессии певица не может. В депрессии она не в состоянии проводить концерты. Помочь ей могут только цветы от продюсера. Если подарить цветы на сумму $0\le x\le 1$ тысяч евро, то она выйдет из депрессии с вероятностью $\sqrt{x}$. Дисконт фактор равен $0.8$. \\
Какова оптимальная стратегия продюсера? }
\solution{ 
Рассмотрим совершенно конкурентный невольничий рынок начинающих певиц. Певицы в хорошем настроении продаются по $V_{1}$, в депрессии - по $V_{2}$. \\
$V_{1}=0.75+\delta(0.5V_{1}+0.5V_{2})$ \\
$V_{2}=max_{x}{-x+\delta(\sqrt{x}V_{1}+(1-\sqrt{x})V_{2})}$ }

\problem{
Будучи незамужней Маша испытывает отрицательную полезность $-c$ каждый день. Каждый день она знакомится с новым ухажером и может тут же выскочить за него замуж. Каждый ухажер характеризуется параметром $X$, полезностью, которую Маша получит в день свадьбы с ним (а вы о чем подумали?), $X$ распределено равномерно на $[0;1]$. Ежедневная полезность Маши от замужнего состояния после дня свадьбы равна 0. Дисконт фактор (с которым дисконтируется Машина полезность) равен $\delta$. \\
а) Как выглядит оптимальная стратегия Маши, если она выбирает мужа на всю жизнь? \\
б) Как выглядит оптимальная стратегия Маши, если она легко может развестись? }
\solution{ }
\problem{
На заводе никто не работает, если хотя бы у одного работника день рождения. Сколько нужно нанять работников, чтобы максимизировать ожидаемое количество рабочих человеко-дней? }
\solution{ 
$\E(X)=365\cdot n\cdot \left(\frac{364}{365}\right)^{n}$ \\
$ln(\E(X)=c+ln(n)+nln(364/365)$ \\
FOC: $1/n+ln(364/365)=0$ \\
Approx. FOC: $1/n-1/365=0$, $n=365$ }

\problem{
A gambler has no money, but the host of the casino generously allows him to play 100 games of the following type. He may either 

(1) choose to accept one dollar with no risk or

(2) choose an integer $n>1$, whereupon he wins $n$ dollars with probability 
$\frac{2}{n+1}$ or loses one dollar with probability $\frac{n-1}{n+1}$. He must have at least one dollar to choose option (2).  What is an optimal strategy for the gambler if he wishes to leave with 200 dollars or more, and what is his probability of success using that strategy? 

У Пети нет денег, но он может сыграть 100 игр следующего типа. В каждой игре Петя может по своему желанию: либо без риска получить $1$ рубль, либо назвать натуральное число $n>1$ и выиграть $n$ рублей с вероятностью $\frac{2}{n+1}$ или проиграть $1$ рубль с вероятностью $\frac{n-1}{n+1}$. Чтобы выбирать вторую альтернативу Петя должен иметь как минимум рубль. Пете позарез нужно 200 рублей. 

Как выглядит Петина оптимальная стратегия? 

Source: AMM E3219 by Daniel Rawsthorne }

\solution{
Поиск оптимальной стратегии: 

1. Если сейчас 0 долларов, то брать 1 доллар. 

Назовем ситуацию, <<шоколадной>> если можно выиграть без риска. Т.е. игр осталось больше, чем недостающее количество денег. 

2. Если игрок не в шоколаде, то оптимальным будет рисковать на первом ходе. 

Почему? Получение одного доллара можно перенести на попозже. 

3. В любой оптимальной стратегии достаточно одного успеха для выигрыша. 

Почему? Допустим стратегии необходимо два успеха в двух рискованых играх. Заменим их  на одну рискованную игру. Получим большую вероятность. 

Оптимальная стратегия:  

Если сейчас 0 долларов, то брать доллар. 

Пусть $d$ - дефицит в долларах, а $k$ - число оставшихся попыток. 

Если $d\le k$, то брать по доллару. 

Если $d>k$, то с риском попробовать захапнуть $1+d-k$ долларов. }


\problem{Пусть $X_{i}$ независимы и равномерны на $[0;1]$. Пете узнает значения $X_{i}$ последовательно. В любой момент Петя может сказать <<Стоп!>>. Как только Петя сказал стоп он получает на руки сумму объявленных $X_{i}$ при условии, что эта сумма не превосходит единицу. В противном случае он получает ноль.
а) Оптимальная стратегия Пети?
б) Оптимальная стратегия Пети, если игра продолжается $n$ шагов.}
\solution{$n=2$, Порог $T$, \[ G_{T}=\int_{x = 0}^{T}\left(\int_{y = 0}^{1-x}(x+y)\, dy\right)\, dx+\int_{x = T}^{1}x\, dx \], $ G_{T}=\frac{1+T-T^{2}-\frac{T^{3}}{3}}{2} $}


% такая же в теории игр
\problem{Инвесторы.

Два инвестора соревнуются. Организатор выдает каждому один рубль. Инвесторы одновременно выбирают способ вложения своего рубля. Каждый инвестор может выбрать себе любую случайную величину $X$ с $E(X)=1$ в качестве результата инвестирования. Победитель (тот, у кого окажется больше денег после инвестирования) получает от проигравшего 1 рубль, результаты инвестирования достаются организаторам. 

Как выглядит оптимальная стратегия, если:

а) Результат инвестирования может принимать отрицательные значения?

б) Результат инвестирования не может принимать отрицательные значения? 

в) Что изменится, если результат инвестирования может принимать значения на $[0;b]$, $b>1$, а победитель в качестве выигрыша получает результат инвестиций проигравшего?}
\source{Ferguson?}
\solution{а) оптимальной стратегии нет б) Равновмерно на $[0;2]$, гарантирует выигрыш с вероятностью не менее 0.5.}


\problem{
Цель игры - получить число, стоящее как можно ближе к единице, но не больше единицы. У Пети две попытки. Сначала он узнает число, равномерно распределенное на $[0;1]$. Далее он выбирает ограничиться ли этим числом, или прибавить к нему еще одно имеющее такое же распределение. Затем Вася, зная петин результат тоже получает две попытки, но он не складывает получаемые числа, а довольствуется последним заказанным числом. \\
а) У кого какие шансы на выигрыш? \\
б) Как выглядит оптимальная стратегия? \\
коммент: изложить поаккуратнее }
\solution{ }






\section{Геометрическая вероятность}
% geom_probability

\problem{
На бумаге проведена прямая. На бумагу бросают иголку.
Какова вероятность, что острый угол между прямой и иголкой будет
меньше
10 градусов? }
\solution{ }
\problem{
 Вася бегает по кругу длиной 400 метров. В случайный момент
времени он останавливается. Какова вероятность того, что он будет
ближе, чем в 50 м от точки старта? Дальше, чем в 100 м? }
\solution{ }

\problem{
Внутри квадрата с вершинами $(0;0)$, $(0;1)$, $(1;0)$ и $(1;1)$ равновероятно выбирается одна
точка. Пусть  $X$  и  $Y$  - абсцисса и ордината этой точки.
Найдите $P(X<0,75)$,  $P(X\le a)$  для произвольного $a$,
$P(X>0,5|X+Y>0,5)$, $P(X+Y>0,5|X>0,5)$, $P(X\cdot
Y>1/3|X+Y<2/3)$,  $P(X>0,3|Y<0,7)$}
\solution{ }


\problem{
Внутри квадрата выбирается точка наугад. Какова вероятность того, что она будет ближе к центру, чем к любой из вершин? }
\solution{ }
\problem{
A circle has 2006 points chosen so that the arcs between any two adjacent points are equal. Three of these points are chosen at random.\par
a) Find the probability that the triangle is right \par
b) Find the probability that the triangle formed is isosceles }
\solution{ 
$a=b=\frac{2004\cdot 1003}{C_{2006}^{3}}$ }

\problem{ \label{dve tochki na otrezke}
На отрезке [0;1] (равномерно и независимо друг от друга)
выбираются две точки. \par
а) Какова вероятность того, что расстояние между ними не более
0,25? \par
б) Какова вероятность, что из трех частей, на которые они разбили
отрезок, можно сложить треугольник? }
\solution{б) $1/4$ }

\problem{ \label{korabli na planete} 
На планету (окружность с центром
$O$, НЕ круг) сели три корабля, координаты их посадки независимы и
равномерно распределены по окружности. Два корабля $A$
и $B$ могут связаться друг с другом, если $\angle AOB<\pi/2$. \par
а) Какова вероятность того, что между кораблями будет связь
(возможно непрямая)? \par
б) Какова вероятность того, что каждый корабль сможет добраться до
базы? Запаса горючего хватает, чтобы проехать расстояние равное
радиусу. \par
в) Какова вероятность того, что наименьший суммарный расход
горючего, необходимый для сбора всех кораблей в одной точке, будет
меньше 0,25? Единицы горючего хватает, чтобы один корабль объехал
всю планету. \par
в) Решите аналогичные а-б задачи в трех измерениях. }
\solution{a) $3/8$ \par
в-а) Зафиксируем координату посадки первого корабля. Обозначим
центральный угол между первым и вторым кораблем $\alpha$. Функция
плотности имеет вид $p(\alpha)=\frac{2\pi\sin(\alpha)}{4\pi}$. \par
Итог: $\int_{0}^{\pi/2}p(\alpha)\frac{\alpha+\pi}{2\pi}d\alpha+
\int_{\pi/2}^{\pi}p(\alpha)\frac{\pi-\alpha}{2\pi}d\alpha=\frac{\pi+2}{4\pi}$.  }

\problem{
Two points on the surface of a sphere are drawn uniformly at random.

a) Expected distance? 

b) Maximum pdf distance? }
\solution{ The p.d.f of distance is ~ $sin(x/R)$. take the mode of this function you get $\pi/2$ 

The second point chosen lies on a circle with the first point. The maximum distance between the two is one-half the circumference of this circle, i.e. $\pi\cdot R$, and assuming unit radius, the maximum distance is $\pi$. Half of the second points chosen will lie at a distance longer than $\pi/2$ and half will lie at a distance less than $\pi/2$. Thus, the expected distance is $\pi/2$.}

\problem{
На окружности наугад выбираются точки $A$, $B$, $C$, $D$, $E$ и
$F$. Какова вероятность того, что треугольники $ABC$ и $DEF$ не
пересекаются? }
\solution{ Точки $ABC$ должны идти подряд. Есть 6 способов выбрать
три точки подряд и $C_{6}^{3}$ выбрать три точки
без ограничений. Итого, $\frac{6}{C_{6}^{3}}=0.3$ }

\problem{
В круге радиуса $r$ случайным образом (равномерно)
выбирается
точка. Пусть $X$ - расстояние от точки до центра круга. \par
а) Найдите функцию плотности $X$ \par
b) Найдите $E(X)$ \par
c) Найдите $E(X^{n})$ }
\solution{ $E(X^{n})=\frac{2}{2+n}$ }

\problem{
В окружность радиуса $r$ вписан правильный $n$-угольник.
Внутри него случайным образом (равномерно) выбирается точка. Пусть
$X$ -
расстояние от точки до ближайшей стороны $n$-угольника. \par
а) Найдите функцию плотности $X$ \par
b) Найдите $E(X)$ \par
c) Найдите $\lim_{n}E(X)$ }
\solution{ }
\problem{
На окружности случайным образом выбираются 3 точки. Эти три точки
являются вершинами треугольника. 

а) Какова вероятность того, что
центр окружности лежит внутри построенного треугольника? 

б) Какова вероятность, что треугольник остроугольный? }

\solution{ Одну точку зафиксируем. Вместо двух других точек выберем две оси.
На двух осях есть 4 варианта выбора точек. \par
$\frac{1}{4}$ \par
Solution2: \par
Рисуем множество на плоскости, ищем площадь. \par
Solution3: \par
Пишем интегралы. \par
Solution of b: \par
Это один и тот же вопрос. }

\problem{
На окружности случайным образом выбираются $n$ точек. \par
a) Какова вероятность того, что центр окружности лежит внутри многоугольника с вершинами в этих точках? \par
б) Какова вероятность того, что эти точки можно накрыть дугой с углом $\alpha=2\pi\cdot t$, где $t\in [0;0.5]$? \par
в) Изменится ли ответ задачи, если точки выбираются не на окружности, а на плоскости, так что все углы равновероятны и вероятность попадания в начало координат равна 0?\par
c) In the border of a perfectly circular piece of wood we choose N points at random to place legs and make a table. What is the probability that the table will stand without falling? }

\solution{
Solution of a: это b для случая $t=1/2$ \par
Solution b1: \par
$p(n,t)=n\cdot t^{n-1}$, Для левой точки имеется $n$ вариантов, при заданной левой точке, остальные точки (их $(n-1)$) должны попасть в отрезок длины $2\pi\cdot t$ при длине окружности $2\pi$. \par
Solution b2: \par
Имеется $n(n-1)$ способов выбрать левую и правую границу. \par
Оставшиеся $(n-2)$ точки должны попасть между ними. \par
Получаем интеграл (перебираем все расстояния между границами): \par
$\int_{0}^{t}n(n-1)a^{n-2}da=n\cdot t^{n-1}$
Solution v: no }

\problem{
На окружности случайным образом выбираются точки, до тех пор пока многоугольник, образуемый точками не будет содержать центр окружности. \par
Каково ожидаемое количество сторон у такого многоугольника? }
\solution{ }

\problem{
На окружности случайным образом выбираются точки, до тех пор пока длина минимальной дуги их накрывающей не станет больше $2\pi\cdot t$, где $t\in [0;0.5]$ 

Каково ожидаемое количество точек? }

\solution{ Исходя из задачи про вероятность для $n$ точек лежать внутри дуги, получаем сумму: 

$E(X)=P(X\ge 1)+P(X\ge 2)+P(X\ge 3)+...=1+t^{0}+2t^{1}+3t^{2}+...=1+\frac{1}{(1-t)^{2}}$ }

\problem{
На отрезке $[0;1]$ случайным образом выбираются $n$ точек. Какова вероятность, что их можно накрыть отрезком длины $t$? }
\solution{ }

\problem{ \label{n-gon} 
Имеется правильный $(2n+1)$-угольник. Наугад выбираются три различные точки. \par
a) Какова вероятность того, что центр многоугольника лежит внутри треугольника с вершинами в выбранных точках? \par
b) Чему равен предел этой вероятности? }
\solution{$\frac{n+1}{4n-2}$ }

\problem{ \label{center inside}
На сфере случайным образом выбираются 4 точки. Эти четыре точки
являются вершинами пирамиды. Какова вероятность того, что
центр сферы лежит внутри построенной пирамиды? }
\solution{
Одну точку зафиксируем. Вместо трех других точек выберем три оси.
На трех осях есть 8 вариантов выбора точек. 
$\frac{1}{8}$  }

\problem{ \label{cut of apple} 
Вася отмечает на яблоке $n$ точек случайным образом. Затем Петя
пытается разрезать яблоко на две половинки так, чтобы все точки
лежали в одной половинке. \par
Какова вероятность того, что Пете удастся это сделать? \par
Source: Bay Area Math Meet, Test of Ingenuity 1999 - Problem 20 }
\solution{ Solution \par
Во-первых. Вместо выбора $n$ точек наугад будем выбирать $n$ осей
($n$ пар точек) наугад, а затем на каждой оси выбирать наугад одну
из двух точек.
Каждой оси соответствует один экватор. \par
Во-вторых. Имеющиеся $n$ экваторов разрезают яблоко на $n^{2}-n+2$
части. Доказываем по индукции: каждый новый экватор пересекает
имеющиеся $k$ экваторов $2k$ раз (каждый по два раза), каждое
пересечение дает одну новую часть. \par
В-третьих. На $n$ осях имеется $2^{n}$ разных комбинаций выбора
конкретных $n$ точек. \par
В-четвертых. Рассмотрим произвольную часть из $n^{2}-n+2$
имеющихся. На этой части выберем произвольную точку. Проведем ось
через эту точку. Разрежем яблоко по соответствующему экватору.
Выберем ту половину, в которой лежит выбранная точка. В этой
половине лежат $n$ точек из $n$ заданных пар. Любая точка из той
же части приведет к отрезанию тех же $n$ точек. \par
Итого: $p=\frac{n^{2}-n+2}{2^{n}}$ }

\problem{
На окружности (не на круге!) выбирается равномерным образом две точки. Каково ожидаемое расстояние между ними? }
\solution{ $\frac{8}{2\pi}$, не перепроверял }

\problem{
Внутри единичного круга выбирается равномерным образом две точки. Каково ожидаемое расстояние между ними? \par
Страшный интеграл или...[?] }
\solution{ }

\problem{
Вася хочет случайным образом равномерно выбрать точку внутри круга единичного радиуса.
Для этого он использует две равномерных независимых случайных величины $r$ и $\theta$.
Он берет в качестве координат точки $x=rcos(\theta)$ и $y=rsin(\theta)$. 

Прав ли он? Если нет, то как исправить алгоритм? }
\solution{ нет, например, $x=\sqrt{r}cos(\theta)$, $y=\sqrt{r}sin(\theta)$ }

\problem{ IBM, Ponder this \par
На плоскости взят произвольный треугольник. Внутри него равномерно
и независимо друг от друга выбираются 3 точки. Эти три точки
образуют новый треугольник. Каково ожидаемое отношение площади
нового треугольника к площади исходного? }
\solution{ }

\problem{ IBM, Ponder this \par
На плоскости взят произвольный квадрат. Внутри него равномерно и
независимо друг от друга выбираются 3 точки. Эти три точки
образуют треугольник. Каково ожидаемое отношение площади
треугольника к площади исходного квадрата? }
\solution{ }

\problem{
Пусть $X$ и $Y$ независимы и равномерны на отрезке $[0;1]$. Найдите вероятность того, что $\frac{X}{Y}$ будет ближе к четному числу, чем к нечетному. \par
Подсказка: Да поможет вам святой арктангенс! \par
source: Putnam 1993 }
\solution{ Графически легко получить сумму бесконечного ряда. Она равна $\frac{5-\pi}{4}$. }

\problem{
Св. $b$ распределена равномерно на отрезке $[0;1]$. На плоскости проводится прямая $y=bx$. С какой вероятностью она отрезок с концами в точках $(2;1)$ и $(4;1)$? }
\solution{ }

\problem{
Пусть $X$ и $Y$ независимы и равномерны на отрезке $[0;1]$. Найдите вероятность того, что в десятичной записи $\frac{X}{Y}$ первой ненулевой цифрой будет единица? \par
Link: \url{http://en.wikipedia.org/wiki/Benford's\_law} \par
Source: \url{http://polymathematics.typepad.com/polymath/2007/10/initial-1s.html} }
\solution{Графически на квадрате получаем две суммы геометрической прогрессии. $p=\frac{1}{3}$. \par
Т.е. единица более вероятна чем другие числа. Это как-то связано с распределением Бенфорда. }

\problem{ Benford's law \par
Рассмотрим поподробнее ценники в супермаркетах. Предположим, что вероятность того, что цена начинается на цифру $n$ ($n\in\{1,2,3,...,9\}$) одинакова во всех странах и зависит только от $n$. Иными словами, если перевести цены из одной валюты в другую, вероятность того, что на первом месте стоит цифра $n$ не должна измениться. \par
а) Найдите вероятность того, что цена начинается с цифры 1;
б) Найдите вероятность того, что цена начинается с цифры 9;
Comment: доработать... lim? hints? }
\solution{ }

\problem{
Three sides of a regular hexagon are chosen at random, and their midpoints are connected. Find the probability of the resulting triangle being right. \par
Source: aops, t=172334 }
\solution{0.6 }

\problem{
A stick is broken into $n$ pieces. If three of these pieces are chosen at random, what is the probability that they form a triangle? }
\solution{ ($n=3$): \par
suppose the length of two points is x from one end and y from anohther end. The length of middle portion would be 1-(x+y), You can plot these x, and y on cartesian axes with x and y < 1 and following set of constraints:
x+y>1-(x+y)\par
x+1-(x+y)>y\par
y+1-(x+y).>x \par
when you plot these three constraints on the cartesian system then you will see that the feasible area for traingle formation is 1/8 while the feasible area to break a unit length into three parts is 1/2. Hence Probablity is =1/4}

\problem{
Given a stick of unit length. Break this in two pieces at random. Then break the longest piece in two pieces at random. What is the probability that you can build a triangle from the three resulting pieces? }
\solution{ }

\problem{
У Васи листочек в клеточку размером 5 на 4. Вася наугад отмечает четыре различных узла (может отметить и крайние). Далее Вася закрашивает выпуклый четырехугольник, образованный этими вершинами. \par
Какова вероятность того, получится прямоугольник со сторонами параллельными линиям сетки? }
\solution{ }

\problem{
На отрезке $[0;1]$ равномерно, независимо друг от друга выбираются 2 точки. Они разбивают отрезок на три части. Какова средняя длина наименьшей части? }
\solution{
Можно через интеграл, можно геометрически посчитать объем, получаем $\frac{1}{9}$ }

\problem{
На окружности единичной длины случайным образом равномерно и независимо друг от друга выбирают два отрезка: длины 0,3 и длины 0,4. \par
Найдите функцию распределения длины пересечения этих отрезков. }
\solution{ }

\problem{
К сфере единичного радиуса проведена касательная плоскость. На сфере случайным образом равномерно выбирается точка. \par
а) Как распределено расстояние от точки до заданной плоскости? \par
б) Возможно ли, что сумма нескольких равномерных величин тождественно равна константе? }
\solution{
Ответы: а) равномерно; б) да, рассмотрим $n$-угольник, над вписанной окружностью построим полусферу, на ней равномерно выбираем точку, в качестве случайных величин берем высоты }


\problem{
Find the probability that a point randomly selected (with uniform probability) within a regular n-sided polygon is closer to the center than any side of the polygon. \par
source: wilmott, bt, 63427 }
\solution{$\frac{1}{3}\frac{cos^{2}(\alpha)+2cos(\alpha)}{(1+cos(\alpha))^{2}}$, $\alpha=\frac{\pi}{n}$}

\problem{
Three points are taken at random in a given circle, and a circle is passed through them. Find the probability that the circle through random points will be wholly in
the given circle? \par
Source: Walker }

\solution{2/5 }


\problem{
Сосулька падает и разбивается на $n$ частей. Предположим, что $(n-1)$ точки разбития - это независимые случайные величины равномерно распределенные по всей длине сосульки. Какова вероятность, что из полученных кусков можно сложить многоугольник?

source: The broken spaghetti noodle, Carlos D'Andrea, Emiliano Gomez
}
\solution{ Многоугольник нельзя сложить, когда существует обломок длиннее суммы всех остальных обломков. Или, равносильно, если существует обломок длины больше чем в половину сосульки. Обозначим буквой $\mu$ (гипер)-объем множества $A=\{x_{1}+x_{2}+...+x_{n}=1\}$. Тогда объем множества $A\cap x_{i}>0.5$ равен $0.5^{n-1}\mu$ в силу того, что множества $A$ и $A\cap x_{i}>0.5$ подобны с коэффициентом $0.5$. Искомая вероятность равна $p=1-n\frac{0.5^{n-1}\mu}{\mu}=1-\frac{n}{2^{n-1}}$. }




\section{Без эксперимента, свойства ожидания, дисперсии}
% no_exper

\problem{  
Случайная величина задана таблицей: \\
$\begin{array}{|c|cccc|} \hline {\omega _{i} } & {\omega
_{1} } & {\omega _{2} } & {\omega _{3} } & {\omega _{4} } \\
\hline {P(\omega _{i} )} & {\frac{1}{8} } & {\frac{3}{8} } &
{\frac{2}{8} } & {\frac{2}{8} } \\  \hline {X(\omega _{i} )} & {5}
& {-3} & {2} & {0} \\  \hline  \end{array}$ \\

Определите  $P(X>1)$,  $P(X=5)$, $P(X=5|X>0)$, $P(X>0|X=5)$,
$P(X^{2}>7)$,  $P(X^{3} <0)$, $P(\frac{1}{X-3}<0)$.
Верно ли, что события $A=\left\{|X|>2\right\}$ и
$B=\left\{X>0\right\}$ являются независимыми? }
\solution{ }
\problem{
Случайная величина задана таблицей: \\
$\begin{array}{|c|ccccccc|} \hline {\omega _{i} } & {\omega _{1} }
& {\omega _{2} } & {\omega
_{3} } & {\omega _{4} } & {...} & {\omega _{n} } & {...} \\
\hline {P(\omega _{i} )} & {\frac{1}{2} } & {\frac{1}{4} } &
{\frac{1}{8} } & {\frac{1}{16} } & {...} & {\frac{1}{2^{n} } } &
{...} \\  \hline {Y(\omega
_{i} )} & {2} & {4} & {6} & {8} & {...} & {2n} & {...} \\
\hline  \end{array}$ \\

Найдите  $P(Y>14)$,  $P(Y^{2} >100|Y<12)$, $P(Y<5|Y<9)$, $P(2^{Y}
<16)$, $P(Y^{2}>Y+6)$. }
\solution{ }
\problem{
 Как распределена сумма  $X+Y$? Если: \\
а)  $X$  и  $Y$  независимы, распределены по Пуассону с ожиданием
$\lambda _{X}$  и  $\lambda _{Y} $. \\
б)  $X$  и  $Y$  независимы, распределены биномиально с
параметрами  $(n_{X},p)$  и  $(n_{Y},p)$.
\\
в)  $X$  и  $Y$  независимы, распределены нормально с параметрами
$(\mu _{X},\sigma _{X}^{2} )$  и  $(\mu _{Y}
,\sigma _{Y}^{2} )$. }
\solution{ }
\problem{
Пусть $X$ и $Y$ одинаково распределены и независимы. Какой смысл
имеет величина $\frac{1}{2}E[(X-Y)^{2}]$? \\
Ответ: $Var(X)$ }
\solution{ }
\problem{
Придумайте случайную величину со средним значением 5 и дисперсией 16. \\
Подсказка: можно ограничиться экспериментом с подбрасыванием правильной монетки. }
\solution{ }
\problem{
Пусть $Var(X)=\sigma_{x}^{2}$, $Var(Y)=\sigma_{y}^{2}$. В каких пределах может лежать $Var(X+Y)$? \\
Ответ: $[\sigma_{x}^{2}+\sigma_{y}^{2}-2\sigma_{x}\sigma_{y};\sigma_{x}^{2}+\sigma_{y}^{2}+2\sigma_{x}\sigma_{y}]$ }
\solution{ }
\problem{
а) С помощью неравенства Чебышева укажите границы в
которых находится
$P(|X-E(X)|>2\sigma)$. \\
б) Чему равна указанная вероятность, если $X$ нормально распределена? \\
Предполагается, что $E(X)$ и $Var(X)$ существуют. }
\solution{ }
\problem{
С помощью неравенства Чебышева, укажите границы, в которых
находятся величины; рассчитайте также их точное значение \\
а) $P(-2\sigma<X-\mu<2\sigma)$, $X\sim N(\mu;\sigma^{2})$ \\
b) $P(8<X<12)$, $X\sim U[0;20]$ \\
c) $P(-2<X-E(X)<2)$, $X$ - имеет экспоненциальное распределение с
$\lambda=1$}
\solution{ }
\problem{
Известно, что случайная величина  $X$  принимает три значения.
Также известно, что  $P(X=1)=0,3$ ; $P(X=2)=0,1$ и $E(X)=-0,7$.
Определите чему равно третье значение случайной величины  $X$  и
найдите $Var(X)$.}
\solution{ }
\problem{  
Сравните  $E(X)$  и  $E(Y)$, если известно, что функции
распределения удовлетворяют соотношению $F_{X}
(t)\ge F_{Y} (t)$  для всех  $t$.}
\solution{ }
\problem{
Пусть $X$ - случайная величина. Рассмотрим функцию
$y(t)=E((X-t)^{2})$. Найдите $E(y(X))$. }
\solution{ }
\problem{  
Используя свойства математического ожидания, докажите, что
$Var(X)=E(X^{2} )-(E(X))^{2} $. Выведите аналогичную
формулу для ковариации.}
\solution{ }
\problem{
Докажите неравенство треугольника: $\sigma_{X+Y}\le
\sigma_{X}+\sigma_{Y}$ }
\solution{ }
\problem{
Пусть $Var(X)=aVar(Y)$ и $Corr(X,Y)=0.7$. При каком значении $a$
между $X$ и $Z=X-2Y$ будет отсутствовать линейная взаимосвязь? }
\solution{ }
\problem{  
С.в.  $X_{1}$,  $X_{2}$... независимы,
$P(X_{i}=0)=P(X_{i}=1)=1/2$, $Z=\sum _{i=1}^{+\infty}\frac{1}{2^{i}} X_{i} $. Найдите $E(Z)$,  $\sigma _{Z}$. }
\solution{ }
\problem{  
Случайная величина задана таблицей: \\
$\begin{array}{|c|cccc|}  \hline {\omega _{i} } & {\omega _{1} } &
{\omega _{2} } & {\omega _{3} } & {\omega _{4} } \\  \hline
{P(\omega _{i} )} & {\frac{1}{8} } & {\frac{3}{8} } & {\frac{2}{8}} & {\frac{2}{8} } \\  \hline {X(\omega _{i} )} & {5} & {-3} & {2}
& {0} \\  \hline  \end{array}$ \\
Найдите  $E(X)$, $E(X^{2} )$, $E(\frac{1}{X+10} )$, $E(X^{2}
|X>0)$, $E(X|X^{2} <10)$,  и $P(X\le t)$. }
\solution{ }
\problem{
$\begin{array}{|c|ccccc|} \hline {\omega _{i} } & {\omega
_{1} } & {\omega _{2} } & {...} & {\omega _{n} } & {...} \\
\hline {P(\omega _{i} )} & {\frac{1}{2}} & {\frac{1}{4} } & {...} & {\frac{1}{2^{n} } } & {...} \\
\hline {Y(\omega _{i} )} & {2} & {4} & {...} & {2n} & {...} \\
\hline
\end{array}$ \\

Найдите $E(Y)$. }
\solution{ }
\problem{
Функция плотности с.в. $Z$ имеет вид: \\
$p(t)=\left\{\begin{array}{l}
{1-t,\quad t\in [0;1]} \\ {t+1,\quad t\in \left[-1;0\right]} \\
{0,\quad t\notin [-1;1]} \end{array}\right.$ \\
Найдите $E(Z)$, $E(Z^{2} )$, $E(Z|Z>0)$, $E(Z^{2} ||Z|<1/2)$,
$P(Z\le t)$ для произвольного
$t$, постройте график $P(Z\le t)$ }
\solution{ }
\problem{
Экономика описывается системой уравнений\\
$ \left\{
\begin{array}{l}
  y=c-ai+\epsilon_{IS} \\
  m-p=hy-ki+\epsilon_{LM} \\
\end{array}
\right.$
\\
Эндогенными переменными являются: $y$ и $i$. \\
$y$ - логарифм выпуска \\
$i$ - процентная ставка \\
$m$ - логарифм денежной массы \\
$p$ - логарифм уровня цен \\
$a>0$, $h>0$, $k>0$. \\
$E(\epsilon_{IS})=E(\epsilon_{LM})=0$, Шоки $\epsilon_{IS}$ и
$\epsilon_{LM}$ - независимы \\
а) Чему равна $Var(y)$, если для достижения нужного $E(y)$,
Центробанк использует в качестве инструмента $m$? \\
б) Чему равна $Var(y)$, если для достижения нужного $E(y)$,
Центробанк использует в качестве инструмента $i$ (в этом случае
кривая LM принимает вид $i=const$)? }
\solution{ }
\problem{  
При каком условии на  $E(X)$  будет выполнено $E(X^{2} )=Var(X)$?
При каком условии на $E(X)$  и $E(Y)$ будет выполнено
$E(XY)=Cov(X,Y)$?}
\solution{ }
\problem{
 При каком значении числа  $a$ функция
$f(a)=E((Y-a)^{2} )$  будет наименьшей? Чему будет равно наименьшее значение функции? \\
 }
\solution{ $a=E(Y)$, $f_{min}=Var(Y)$}
\problem{
При каком значении числа $a$ функция $f(a)=E|Y-a|$ будет наименьшей? }


\solution{ При $a$ равном медиане. \\
Возьмем мат. ожидание от $|Y-a|\ge |Y-m|+(m-a)(P(Y<m)-P(Y>m))$ }
\problem{
Пусть с.в. $X$ принимает только натуральные значения. \\
а) Верно ли, что $X=\sum_{i=1}^{+\infty} 1_{X\leq i}$? \\
б) Возьмите математическое ожидание от левой и правой частей}
\solution{ }

\problem{  
Автор книги получает 50 тыс. рублей сразу после заключения
контракта и 5 рублей за каждую проданную книгу. Автор
предполагает, что количество книг, которые будут проданы - это
случайная величина с ожиданием в 10 тыс. книг и стандартным
отклонением в 1 тыс. книг. Чему равен ожидаемый доход автора? Чему
равна дисперсия дохода автора?}
\solution{ }

\problem{
При каких условиях верно, что: \\
а) $E(X^{2})=(E(X))^{2}$ \\
b) $E(XY)=E(X)E(Y)$ }
\solution{а) Если величина - константа с вероятностью 1 б) нехороший вопрос, имелось ввиду $Cov(X,Y)=0$ }

\problem{ $[$Задача о божественной регрессии$]$ \\
Вася знает абсолютно все характеристики $X$, а про $Y$ ему
известны только $E(Y)$ и $Cov(X,Y)$. Вася наблюдает только
случайную величину $X$, а хочет спрогнозировать случайную величину
$Y$. Строго говоря, Васина цель - построить с.в.
$\hat{Y}=\alpha+\beta X$, так чтобы $E(\hat{Y}) = E(Y)$, и
дисперсия ошибки прогноза $Var(Y-\hat{Y})$
 была бы минимальной. Найдите оптимальные коэффициенты $\alpha$  и $\beta$ }
\solution{ }

\problem{ \label{Vasya, Petya i figuristka}
Эксперты Вася и Петя сидят
у телевизора и делают прогноз, сколько баллов получит выступающая
фигуристка. Результат фигуристки - случайная величина $Y$, Васин
прогноз - $X_{1}$, Петин - $X_{2}$.
\\
Известно, что $E(Y-X_{1})=E(Y-X_{2})=0$. Также известно, что
ошибки прогнозов некоррелированы, т.е. $Cov(Y-X_{1},Y-X_{2})=0$.
Дисперсии ошибок прогнозов различны и равны
$Var(Y-X_{1})=\sigma_{1}^{2}$,
$Var(Y-X_{2})=\sigma_{2}^{2}$. \\
а) Поясните смысл условий $E(Y-X_{i})=0$ и
$Cov(Y-X_{1},Y-X_{2})=0$. \\
b) Рассмотрим прогноз $X$ вида $X=a_{1}X_{1}+a_{2}X_{2}$ и
соответствующую ошибку прогноза $e=Y-X$. Найдите числа $a_{1}$ и
$a_{2}$, так чтобы математическое ожидание ошибки равнялось нулю,
а дисперсия была бы минимальной. \\
с) Верно ли, что из условия $\sigma_{1}^{2}>\sigma_{2}^{2}$
следует то, что $a_{1}<a_{2}$? }
\solution{ вроде бы
$a_{1}=\frac{\sigma_{2}^{2}}{\sigma_{1}^{2}+\sigma_{2}^{2}}$ и
$a_{2}=1-a_{1}$ }

\problem{
Случайным процессом с дискретным временем называется
последовательность случайных величин...$X_{-2}$, $X_{-1}$,
$X_{0}$, $X_{1}$, $X_{2}$... Случайный процесс называется
стационарным (weak stationary), если $E(X_{t})$
 существует и не зависит от $t$; $Cov(X_{t},X_{t - k})$
 существует и не зависит от $t$. Допускается, что $Cov(X_{t},X_{t - k})$
 зависит от $k$. \\
а) Верно ли, что сумма двух стационарных процессов стационарна? \\
б) Верно ли, что сумма двух независимых стационарных процессов
стационарна? \\
}
\solution{ а) - нет, б) - да }
\problem{
а)  Известно, что  $E(Z)=-3$  и  $E(Z^{2} )=15$. Найдите $Var(Z)$,
$Var(4-3Z)$  и  $E(5+3Z-Z^{2} )$. \\
б) Известно, что  $Var(X+Y)=20$  и $Var(X-Y)=10$. Найдите
$Cov(X,Y)$  и
$Cov(6-X,3Y)$. }
\solution{ }

\problem{
Пусть $X\sim U[0;1]$, $f(a)=E(X\wedge a)$, $g(a)=E(X\vee a)$ \\
Постройте $f(a)$, $f(f(a))$, $g(a)$, $g(g(a))$ }
\solution{ }

\problem{
При каких условиях $E(\frac{1}{X})=\frac{1}{E(X)}$? }
\solution{Только если $X$ с вероятностью 1 константа. }

\problem{ \label{moloko i jensen} Бабушки и молоко \\
Баба Маша и баба Катя покупают молоко у бабы Нюры. Баба Маша
каждый день покупает ровно 1 литр, а баба Катя - ровно на 10
рублей. В этом месяце баба Нюра продавала молоко в среднем по 10
рублей за литр. Цена каждый день могла быть разная. Кто больше
купил молока? Кто больше заплатил? \\
Трактовка 1. Средняя цена считается как среднее ежедневных цен. \\
Трактовка 2. Баба Маша и баба Катя - единственные клиентки бабы
Нюры, а средняя цена считается как выручка делить на количество
проданного молока. }
\solution{
1. $E(P)=10$. $E(\frac{10}{P})\ge 1$. Заплатили
поровну, купила больше баба Катя. \\
2. $E(P+10)=10E(1+\frac{10}{P})$. $E(\frac{10}{P})\ge 1$. $E(P)\ge
10$. Заплатила больше баба Маша, купила больше баба Катя.}

\problem{ Йенсен

Вася забрасывает удочку $100$ раз.
Вероятность поймать рыбку при одном забрасывании равна $p$. Петя забрасывает удочку случайное количество раз, $N$ (под настроение). Известно, что $E(N)=100$. У кого шансы поймать хотя бы одну рыбку выше?  }
\solution{ У Васи }

\problem{ Снова Йенсен

Тысяча зайцев требует спасения. Дед Мазай выбирает между двумя стратегиями: \\
А. Перевозить зайцев равными партиями по 10 за заход. \\
Б. Перевозить зайцев случайными партиями от 1 до 19 зайцев за заход. \\
В каком случае ожидаемое количество заходов будет меньшим? }
\solution{ }

\problem{
Допустим, что закон распределения $X$ имеет вид:
\begin{tabular}{|c|c|c|c|}
  \hline
  X & 1 & 2 & 3 \\
  \hline
  Prob & $\theta$ & $2\theta$ & $1-3\theta$ \\
  \hline
\end{tabular} \\
а) Найдите $E(X)$ , $Var(X)$ \\
б) При каких $\theta$ среднее будет наибольшим? При каких - наименьшим? \\
в) При каких $\theta$ дисперсия будет наибольшей? При каких - наименьшей? }
\solution{ 
$E(X)=3-4\theta$, $\theta\in[0;1/3]$, $\theta_{max}=0$, $\theta_{min}=1/3$ }

\problem{
В городе Туме случайным образом выбрали семейную пару. Стандартное отклонение возраста мужа оказалось равным 5 годам, а стандартное отклонение возраста жены - 4 годам. Найдите корреляцию возраста жены и возраста мужа, если стандартное отклонение разности возрастов оказалось равным 2 годам. }
\solution{ }

\problem{
Известно, что функция $f$ обладает следующим свойством: $E(f(X))=f(E(X))$ для любой случайной величины $X$. Какой может быть $f$? }
\solution{ только линейной. Рассмотрите случайную величину, которая принимает значение $x$ с вероятностью $p$ и значение $y$ с вероятностью $(1-p)$ }


\problem{
Хорошо известно, что $Var(X)=E(X\cdot X)-E(X)\cdot E(X)$. А что получится если посчитать $E(X\cdot (X-2009))-E(X)E(X-2009)$? }
\solution{ $Var(X)$ }

\problem{Трюк

О независимых случайных величинах $X$ и $Y$ известна следующая информация: $$Var(X)=Var(Y)=\frac{1}{4},$$ $$Var(\min\{X,Y\})=Var(\max\{X,Y\})=\frac{3}{16}.$$ 
\begin{enumerate}
\item Найти $corr\left(\min\{X,Y\};\max\{X,Y\}\right)$.
\item А существуют ли такие величины? 
\end{enumerate} }

\solution{a) $\frac{1}{3}$. $X+Y=\min\{X,Y\}+\max\{X,Y\}$ б) Да, индикаторы}

\problem{<<Корреляция --- это мера линейной связи>>

Найдите все случайные величины $X$ такие, что $corr(X,X^2)=1$.


Источник: Алексей Суздальцев}

\solution{Случайные величины, принимающие (с положительными вероятностями) два значения $x_1$ и $x_2$, такие, что $|x_1|<|x_2|$.}





\section{Функция плотности и условная функция плотности}
% density

\problem{  
Пусть у случайной величины  $X$  функция плотности имеет
вид
$p(t)=\left\{\begin{array}{l} {1/5,\quad t\in [2;7]} \\
{0,\quad t\notin [2;7]} \end{array}\right. $. Определите $P(4\le
X\le 6)$,  $P(X\ge 5)$, $P(X=5)$, $P(X>5)$,
$P(X>5|X>3)$,  $P(-\infty <X<+\infty )$}
\solution{ }

\problem{
Известно, что функция плотности случайной величины  $X$
имеет вид:

$$p(x)=\left\{\begin{array}{l} {cx^{2},\quad 5A;8\; x\in [-2;2]} \\ {0,\quad 8=0G5} \end{array}\right. $$

Найдите значение константы  $A$,  $P(X>1)$, $E(X)$,
$E(\frac{1}{X^{3} +10} )$  и
постройте график функции распределения величины  $X$.}
\solution{ }

\problem{
Найдите  $P(X\in \left[16;23\right])$, если \\
а)  $X$ нормально распределена,  $E(X)=20$,
$Var(X)=25$ \\
б)  $X$  равномерно распределена на отрезке  $\left[0;30\right]$ \\
в)  $X$  распределена экспоненциально и  $E(X)=20$ }
\solution{ }

\problem{ \label{sin} 
Пусть величина $X$ распределена равномерно на отрезке $[0;\pi]$.
Найдите функцию плотности величины $Y=\sin(X)$ }
\solution{$p_{Y}(t)=\frac{2}{\pi\sqrt{1-t^{2}}}$ при $t\in[0;1]$  }

\problem{ $[$Mosteller, доп. вопросы$]$ Короткий кусок стержня \\
а) Если стержень ломается случайным образом на две части, то
какова средняя длина меньшего куска? \\
б) Каково среднее отношение длины короткого куска к длине длинного
куска? \\
в) Каково среднее отношение длины длинного куска к длине короткого
куска? \\
д) Какой вид имеют функции плотности для случаев b, c? \\
е) Мода отношения длины короткого куска к длине длинного? \\
ж) Мода отношения длины длинного куска к длине короткого? }
\solution{ }

\problem{ $[$Mosteller$]$ Сломанный стержень \\
Стержень ломается случайным образом на три части. Найти средние
длины короткого, среднего и длинного кусков. }
\solution{ }

\problem{
Пусть $p_{X}(t)$ - функция плотности с.в. $X$. \\
a) Найдите функцию плотности с.в. $Y=aX+b$, если $a>0$. \\
b) Как изменится ответ, если $a<0$? }
\solution{ }



\problem{  
Пусть  $X$  распределена равномерно на  $\left[0;1\right]$.
Найдите плотность распределения случайных величин  $Y=\ln
\frac{X}{1-X} $,  $Z=-\frac{1}{\lambda } \ln X$  ( $\lambda >0$
),  $W=X^{3} $,  $Q=X-1/X$. }
\solution{ }
\problem{
Пусть $F(t)$ - функция распределения случайной величины. \\
Найдите $\int_{-\infty}^{+\infty} (F(t+a)-F(t))dt$ \\
Answer: $\int_{-\infty}^{+\infty} (F(t+a)-F(t))dt=a$ }
\solution{ }
\problem{  
Пусть вероятность того, что случайная величина  $X$  лежит в
промежутке от  $\left[a;b\right]$, где  $b\ge a\ge 0$,
определяется формулой  $P(a\le X\le b)=\int _{a}^{b}e^{-t} dt $.
Чему равны  $P(X=17)$, $P(0\le X\le 1)$, $P(1\le X\le 2|X>1)$,
$P(X\ge 0)$,  $P(X<0)$?}
\solution{ }
\problem{
 С.в.  $X$  распределена равномерно на отрезке  $[2;8]$.
Найдите  $E(X)$,  $E(X|X>4)$,
$E(X^{2} )$. }
\solution{ }
\problem{
Пусть $X\sim U[0;1]$, $Z=X+Y$, а $Y$ задана табличкой: \\
$\begin{array}{|c|cc|}
\hline
Y & 0 & 3 \\
\hline
Prob & 0.4 & 0.6 \\
\hline
\end{array}$ \\
а) Постройте функцию распределения $Z$, если $X$ и $Y$ независимы \\
б) Постройте функцию распределения $Z$, если $Y=\left\{
\begin{array}{c}
0, X\in [0;0.4] \\
3, X\notin [0;0.4] \\
\end{array}
\right.$\\
в) Найдите $E(Z)$ в обоих случаях }
\solution{ }
\problem{
Пусть $X$ - неотрицательная с.в. с функцией плотности
$p(t)$ и $E(X)<\infty$. При каком $c$ функция $g(t)=c\cdot tp(t)$
также
будет функцией плотности? }
\solution{ }
\problem{  
Распределение с.в.  $X$  называется \underbar{экспоненциальным},
если ее функция плотности имеет вид $p_{X} (x)=\lambda e^{-\lambda
x}$  для  $x>0$.
Найдите  $E(X)$,  $Var(X)$. }
\solution{ }
\problem{  
Пусть цена акции  $A$  имеет ф. плотности  $p=\frac{3}{4} \max
\left\{1-(x-1)^{2},0\right\}$. У Васи есть опцион-пут, дающий ему
право продать акции по цене 1,2 рубля (опцион пут). Какова
вероятность исполнения опциона? Каков ожидаемый Васин
доход? }
\solution{ }
\problem{  
Убыток от пожара - равномерно распределенная с.в. на
$\left[0;1\right]$. Если убыток оказывается больше 0,7, то
страховая компания выплачивает компенсацию 0,7. Чему равны средние
потери? }
\solution{ }
\problem{  
Пусть цена акции  $A$  имеет ф. плотности  $p=\frac{3}{4} \max
\left\{1-(x-1)^{2},0\right\}$. У Васи есть опцион-колл, дающий ему
право купить акции по цене 1 рубль. Каков
ожидаемый Васин доход? }
\solution{ }
\problem{  
Пусть  $X$  - с.в. с  $p_{X} (t)>0$  для всех $t$. Как
распределены  $Y=F_{X} (X)$  и  $Z=-\ln Y$?
\\
$[${\it Ну хоть когда-то должна же функция распределения зависеть
от
случайной величины!}$]$ }
\solution{ }
\problem{
Число $X$ выбирается равномерно на отрезке $[0;1]$. Затем
число
$Y$ выбирается равномерно на отрезке $[0;X]$. \\
a) Найдите условную функцию плотности $p_{Y|X}(x,y)$ \\
b) Найдите $p_{X,Y}(x,y)$ и $p_{Y}(y)$ \\
c) Найдите $E(Y)$, $Var(Y)$, $P(X+Y>1)$ }
\solution{ }
\problem{  
Величина  $Z$  равномерно принимает любое значение из отрезка
$\left[-3;5\right]$. Как выглядит ее функция плотности? Пусть
$X=Z^{2} $. Найдите  $P(X\le 3)$  и $P(X\le 7)$. Найдите
$P(Z<2|X<4)$, $P(Z>X)$,  $P(X>3|Z>1)$.}
\solution{ }
\problem{  
Пусть у случайной величины  $X$  функция плотности имеет
вид
$p(t)=\left\{\begin{array}{l} {1-t,\quad t\in [0;1]} \\
{t+1,\quad t\in \left[-1;0\right]} \\ {0,\quad t\notin [-1;1]}
\end{array}\right. $. Определите  $P(0,4\le X\le 0,6)$,  $P(X\ge 0,5)$,  $P(X>-0,5|X<0,3)$, $P(-\infty <X<+\infty )$.
Верно ли, что события $A=\left\{|X|>0,5\right\}$  и
$B=\left\{X>0\right\}$
являются независимыми?}
\solution{ }
\problem{
Пусть случайные величины $X_{1}$ и $X_{2}$ независимы, $X_{1}$
распределена равномерно на $[-2;1]$, $X_{2}$ - равномерно на
$[0;1]$. \\
a) Найдите $P(max\{X_{1},X_{2}\}>0,5)$ \\
б) Функцию плотности $f_{max\{X_{1},X_{2}\}}(t)$ }
\solution{ }
\problem{
Пусть $Z = \max ( {X_1,X_2 } )$ и случайные величины $X_1$ и $X_2
$ независимы. Найдите функцию распределения $F_Z ( t )$, если
известны функции распределения $F_{X_1 } ( t )$ и $F_{X_2 } ( t
)$. Для случая $X_1 \sim U\left[ {0;5} \right]$, $X_2  \sim
U\left[ {0;3} \right]$
 найдите $E( Z )$.}
\solution{ }
\problem{  
Пусть  $X_{1} $,  $X_{2}$  и  $X_{3}$  - независимы и равномерны
на отрезке  $[0;1]$. Найдите функцию плотности $Y=\max
\left\{X_{1},X_{2},X_{3} \right\}$. }
\solution{ }
\problem{
Пусть $X_{1}$, $X_{2}$, ..., $X_{n}$ независимы и экспоненциально
распределены с параметром $\lambda$. \\
Найдите $E(\min\{X_{1}, ..., X_{n}\})$, $E(\max\{X_{1}, ...,
X_{n}\})$ \\
Solution: \\
$\max\sim X_{1}+\frac{X_{2}}{2}+...+\frac{X_{n}}{n}$ }
\solution{ }
\problem{
Пусть $X$, $Y$ и $Z$ независимы и равномерны на $[0;1]$. Какова вероятность того, что можно построить треугольник со сторонами таких длин? }
\solution{ }
\problem{  
С.в.  $X$  распределена равномерно на отрезке $\left[a;b\right]$.
a) Найдите  $E(X)$  и
$Var(X)$. \\
b) Пусть $c\in [a;b]$. Найдите $E(X|X>c)$, $Var(X|X>c)$ }
\solution{ }
\problem{ Про расстояние между минимумом и максимумом \\
Вася называет три числа на интервале $[0;1]$ наугад. \\
а) Какова вероятность того, что разница между наибольшим и наименьшим будет меньше 1/2? \\
б) Пусть $R$ - разница между наибольшим и наименьшим числом. Найдите функцию плотности случайной величины $R$ \\
Решение: \\
а) Рассмотрим случай $x<y<z$. В этом случае вероятность равна $P=\int_{0}^{1/2}\int{x}^{x+1/2}\int_{y}^{x+1/2}\cdot dz\cdot dy\cdot dx+ \int_{1/2}^{1}\int{x}^{1}\int_{y}^{1}\cdot dz\cdot dy\cdot dx=\frac{1}{12}$\\
Так как возможно 6 вариантов расположения трех чисел, то искомая вероятность равна $P=6\cdot \frac{1}{12}=\frac{1}{2}$ \\
b) Взяв $t$ вместо $\frac{1}{2}$ получаем функцию распределения, а затем и функцию плотности $p(r)=6r(1-r)$ \\
Geometric (intuitive) solution is welcome! }
\solution{ }
\problem{
Случайные величины $X_{1}$..., $X_{n}$ независимы и равномерно распределены на отрезке $[0;1]$ \\
Найдите: \\
%... (?)... \\
%E(X_{min}|X_{max})
%E(X_{i}|X_{min})
%E(X_{max}|X_{i})
%etc., неравенства... 

а) Функцию плотности $X_{min}$, $E(X_{min})$ \\
b) Функцию плотности $X_{max}$, $E(X_{max})$ \\
c) $E(X_{max}|X_{min}>c)$ }

\solution{
a) $\frac{1}{n+1}$ \\
b) $\frac{n}{n+1}$ \\
c) intuition $c\frac{1}{n+1}+1\frac{n}{n+1}$ }


\problem{ Функция СЛЧИС() в русской версии Excel генерирует случайные числа равномерно распределенные на $[0;1]$. Как с помощью нее получить экспоненциальные величины со средним значением $m$? }
\solution{ }
\problem{
Пусть $X$ равномерна на $[-2;1]$, а $Y$ - расстояние от числа $X$ до числа $(-1)$. \\
Найдите фукнцию плотности $Y$ }
\solution{ }
\problem{
Пусть $X$ распределена экспоненциально с параметром $\lambda=1$.  Какую функцию плотности имеет величина $X/2$? Как называется такое распределение? \\
Answer: экспоненциальное, $\lambda=2$. }
\solution{ }
\problem{
Пусть $X$ распределена экспоненциально с параметром $\lambda=1$. Найдите функцию плотности дробной части $X$. Найдите среднее значение дробной части. \\
 }
\solution{  $p(t)=\frac{e^{1-t}}{e-1}$, $\frac{e-2}{e-1}$}

\problem{
Известно, что $X$ равномерно распределена на $[-1;1]$ и $Y=tg(\frac{X\pi}{2})$. \\
Найдите функцию плотности $Y$ }
\solution{ }
\problem{ Максимум равномерных в степени $k$ \label{max uniform} \\
Пусть величины $X_{1}$, ..., $X_{n}$ равномерны на $[0;1]$ и независимы. Пусть $k$ - натуральное число и $M=max\{X_{i}\}$. Найдите $E(M^{k})$ \\
Answer: $\frac{n}{n+k}$ \\
Challenge: find intuitive, without integral, explanation (for k=1 we may break the interval into (n+1) equidistributed parts)... }
\solution{ }

\problem{
Пусть $U$ - распределена равномерно на отрезке $[0;1]$. \\
Как распредлена величина $Y=1+\lfloor ln(U/ln(q)\rfloor$? \\
 }
\solution{ $Y$ имеет геометрическое распределение.}

\problem{
Пусть св. $X$ принимает значения из промежутка $[a;b]$ (возможно не все), $Var(X)=\frac{(b-a)^{2}}{4}$. Найдите закон распределения $X$. \\
 }
\solution{ это максимально возможная дисперсия для такой величины, $X$ равновероятно принимает значения $a$ и $b$.}

\problem{
Случайная величина $X$ задана функцией плотности $p(t)$. Известно, что $p(4)=9$. Примерно найдите вероятность $P(X\in [4;4.003])$. }
\solution{ }

\problem{ Пусть $g(t)$ - возрастающая функция и $Y=g(X)$. Докажите, что функция плотности случайной величины $Y$ имеет вид: $p_{Y}(t)=p_{X}(g^{-1}(t))\frac{dg^{-1}}{dt}$. }
\solution{ }

\problem{Случайная величина $X$ распределена равномерно на $[0;2\pi]$. Найдите $Corr(\cos(X),\cos(a+X)$.}{$\cos{a}$}








\section{Арифметика}
% ariphmetics
\problem{  
Докажите, что  $\frac{S_{n} -E(S_{n} )}{\sqrt{Var(S_{n} )} }
=\frac{\bar{X}_{n} -E(\bar{X}_{n} )}{\sqrt{Var(\bar{X}_{n} )} }
=\frac{\bar{X}_{n} -\mu }{\sqrt{{\sigma ^{2} \mathord{\left/
{\vphantom {\sigma ^{2}  n}} \right. \kern-\nulldelimiterspace} n}} } $, где  $X_{i}$  - iid,  $E(X_{i} )=\mu $,
$Var(X_{i} )=\sigma ^{2}$}
\solution{ }
\problem{
Пусть $p\in[0;1)$, $G=p+p^2+p^3+p^4+...$ и $S=1p+2p^2+3p^3+4p^4+...$ \\
а) Чему равняется сумма $G-pG$? Чему равняется $G$? \\
б) Чему равняется $S-pS$? \\
в) Чему равняется $p\frac{dG}{dp}$? \\
г) Чему равняется $S$? }
\solution{ }

\problem{ \label{dlia ravnomernoi monetki}
Пусть $f(a,b)=\int_{0}^{1}x^{a}(1-x)^{b}dx$ \\
a) Проинтегрировав по частям, докажите, что
$f(a,b)=f(a+1,b-1)\frac{b}{a+1}$ \\
б) Как $f(a,b)$ выражается через $f(a+b,0)$? \\
с) Найдите $f(a,b)$ }
\solution{$f(a,b)=\frac{a!b!}{(a+b+1)!}$ }

\problem{ \label{dlia chi2} 
Пусть $\sum p_{i}=1$ и $\sum \hat{p}_{i}=1$ \\
Докажите, что $\sum \frac{(n\hat{p}_{i}-np_{i})^{2}}{np_{i}}=n \left(\sum{\frac{\hat{p}_{i}^{2}}{p_{i}}} - 1 \right)$ }
\solution{ }
\problem{  
Может ли ковариационная матрица иметь вид $(\begin{array}{cc} {4}
& {2} \\ {-1} & {9}
\end{array})$,  $(\begin{array}{cc} {4} & {7} \\ {7} &
{9} \end{array})$,  $(\begin{array}{cc} {9} & {0} \\
{0} & {-1} \end{array})$,  $(\begin{array}{cc} {9} &
{2} \\ {2} & {1} \end{array})$?}
\solution{ }
\problem{
 Пусть  $X_{i} \sim N(0;1)$  и независимы. Пусть
$Z_{1} =a_{1} (X_{1} -X_{2} )$,  $Z_{2} =a_{2} (X_{1} +X_{2}
-2X_{3} )$,  $Z_{3} =a_{3} (X_{1} +X_{2} +X_{3} -3X_{4} )$ и т.д.
Найдите такие  $a_{i} >0$, чтобы $Var(Z_{i} )=1$. Для этих $a_{i}
$ : Найдите $E(Z_{i} )$, $Cov(Z_{i},Z_{j} )$. Докажите, что $\sum
_{i=1}^{3}(X_{i} -\bar{X})^{2} =Z_{1}^{2} +Z_{2}^{2} $, и что [т]
$\sum _{i=1}^{n}(X_{i}
-\bar{X})^{2} =\sum _{i=1}^{n-1}Z_{i}^{2}$. }
\solution{ }
\problem{
 Пусть  $X_{i}$  независимы, причем  $\forall i$
$E(X_{i} )=\mu $, а  $Var(X_{i} )=\sigma ^{2} $. Найдите
$Var(X_{1} +X_{2} )$,  $E(X_{1} +X_{2} +...+X_{n} )$, $Var(X_{1}
+X_{2} +...+X_{n} )$, $E(\bar{X})$, $Var(\bar{X})$, $E(X_{1}^{2}
)$, $Cov(X_{1} +X_{2},X_{2} +X_{3} )$,  $E(X_{1} \cdot X_{2} )$,
$E((X_{1} -\bar{X})^{2} )$,
$E(\hat{\sigma }^{2} )$. }
\solution{ }
\problem{ [Почему степеней свободы $( {n - 1} )$?]\\
Пусть $\left\{ {X_i } \right\}$ - iid  $N( {\mu ;\sigma ^2 } )$
 и $\widehat\sigma ^2  = \frac{{\sum {( {X_i  - \bar X} )} ^2 }}
{{n - 1}}$. Для $n = 2$ докажите, что $\widehat\sigma ^2$ можно
представить в виде $\widehat\sigma ^2  = \frac{{Y_1^2 }}{1}$, где
$Y_1  \sim N( {0;\sigma ^2 } )$. Как выражается $Y_1$ через $X_1 $
и $X_2 $? a)  Для $n = 3$ докажите, что $\widehat\sigma ^2$ можно
представить в виде $\widehat\sigma ^2 = \frac{{Y_1^2 +Y_2^2}}{2}$, где $\left\{ {Y_i } \right\}$- iid $N( {0;\sigma ^2 } )$.
Как выражаются $Y_i$ через $\left\{ {X_j } \right\}$? b) Как
выглядит данное представление для произвольного $n$? Подсказки:
$Y_3  = const \cdot ( {X_1+X_2+X_3  - 3X_4 } )$, для
доказательства независимости двух нормальных случайных величин
достаточно доказать, что их
ковариация равна нулю. }
\solution{ }
\problem{
Докажите, что $\lim_{n \to \infty }
\frac{{\sum_{i = 1}^{n} {\frac{1} {i}} }} {{\ln n}} = 1$ }
\solution{ }
\problem{  
Верно ли, что  $Var(a_{1} X_{1} +a_{2} X_{2} )=(\begin{array}{cc}
{a_{1} } & {a_{2} }
\end{array})\cdot C\cdot (\begin{array}{cc} {a_{1} } &
{a_{2} } \end{array})^{t} $, где  $C$  - ковариационная
матрица? }
\solution{ }
\problem{  
Упростите  $E(6X-2)$,  $Var(5-3X)$, $Cov(7X-2,4-5Y)$, $\sigma
_{(2-3X)} $, $cor(5+2X,6-7Y)$, $cor(5-3X,6X+8)$. Представьте в
виде суммы $Var(X+Y)$,
$Var(2X+3Y)$  и  $Var(X-Y)$. }
\solution{ }




\section{Смешанные распределения}
% mixed_distr

\problem{ \label{simple bayesian updating}
Пусть $Q$ - случайная величина, принимающая значения на
$[0;1]$ и
имеющая функцию плотности $p_{Q}(t)=...$. \\
Вася узнает значение $Q$ и изготавливает монетку, выпадающую орлом
с данной вероятностью. Затем он передает монетку Пете. Петя знает
$p(t)$, но не знает, какое конкретно значение приняло $Q$. Петя
подкидывает монетку и она выпадает орлом. \\
Как Пете следует подправить свое мнение $p_{Q}(t)$? Т.е. найдите
условную функцию плотности $p_{Q}(t|$монетка выпала
орлом$)$.}
\solution{$p_{Q}(t|$ монетка выпала орлом $)=c\cdot t\cdot p_{Q}(t)$ \\
Находим $P(Q\le t|$ монетка выпала орлом $)$, причем на
знаменатель не обращаем внимания, т.к. он - константа. Затем берем
производную. Или интуитивно.  }

\problem{
Петя сообщает Васе значение величины $X\sim U[0;1]$. Вася изготавливает неправильную монетку, которая выпадает <<орлом>> с вероятностью  $X$ и подкидывает ее 20 раз. \\
а) Какова вероятность, что выпадет ровно 5 орлов? \\
б) Каково среднее количество выпавших орлов? Дисперсия? }
\solution{ }


\problem{ \label{ravnomernaia monetka} 
C.в.  $X\sim U[0;1]$. Вася изготавливает неправильную монетку,
которая выпадает <<орлом>> с вероятностью  $x$ и передает ее Пете.
Петя не знает $x$. Он подкинул монетку один раз. Она выпала
<<орлом>>. Какова вероятность того, что она снова выпадет
<<орлом>>? Как выглядит ответ, если Пете известно, что монетка при
$n$ подбрасываниях  $k$  раз выпала орлом? \\
Подсказка: задача \ref{dlia ravnomernoi monetki}}
\solution{
$f(a,b)=\frac{a!b!}{(a+b+1)!}$ \\
$Prob=f(k+1,n-k)/f(k,n-k)=\frac{k+1}{n+2}$ }

\problem{
Петя сообщает Васе значение случайной величины, равномерно
распределенной на отрезке  $[0;4]$. С вероятностью  $\frac{1}{4}$
Вася возводит Петино число в квадрат, а с вероятностью
$\frac{3}{4}$  прибавляет к Петиному числу 4. Обозначим результат
буквой  $Y$. \\
Найдите  $P(Y<4)$  и функцию плотности случайной
величины  $Y$. \\
Вася выбирает свое действие независимо от Петиного числа. }
\solution{ }

\problem{
Допустим, что оценка  $X$  за экзамен распределена равномерно на
отрезке  $\left[0;100\right]$. Итоговая оценка  $Y$ рассчитывается
по формуле  $Y=\left\{\begin{array}{l} {0,\quad
if\quad X<30} \\ {X,\quad if\quad X\in \left[30;80\right]} \\
{100,\quad if\quad X>80} \end{array}\right. $. \\
Найдите  $E(Y)$,  $E(X\cdot Y)$,
$E(Y^{2})$,  $E(Y|Y>0)$. }
\solution{ }


\problem{ Рассмотрим треугольник с вершинами $(0;0)$, $(1;0)$ и $(1;1)$. Точка $A$ выбирается равномерно на границе треугольника (не внутри треугольника!). Пусть $X$ и $Y$ - абсцисса и ордината получившейся точки. \\
а) Найдите $E(XY)$, $Cov(X,Y)$ \\
б) Тот же вопрос, если точка $A$ выбирается равномерно внутри треугольника }

\solution{ source: aops, t=173773, imho: $\frac{5(\sqrt{2}-1)}{6}$ }

\section{Данетки}
% yesno

\problem{  Известно, что $Corr(X,Y)>0$. Верно ли, что $E(XY)>E(X)E(Y)$? }
\solution{да}

\problem{  Сумма двух нормальных независимых случайных
величин
нормальна? }
\solution{да}

\problem{  \label{dn003} Сумма любых двух непрерывных случайных величин непрерывна? }
\solution{нет}

\problem{  \label{dn004} Нормальная случайная величина может принимать
отрицательные значения? }
\solution{да}

\problem{  \label{dn005} Пуассоновская случайная величина является непрерывной? }
\solution{нет}

\problem{  \label{dn006} Сумма двух независимых равномерно
распределенных величин
равномерна? }
\solution{нет}

\problem{  \label{dn007} Дисперсия суммы зависимых величин всегда
больше суммы
дисперсий? }
\solution{нет}

\problem{  \label{dn008} Дисперсия пуассоновской с.в. равна ее
математическому
ожиданию? }
\solution{да}

\problem{  \label{dn009} Если  $X$  - непрерывная с.в.,  $E(X)=6$  и
$Var(X)=9$, то $Y=\frac{X-6}{3} \sim
N(0;1)$? }
\solution{нет}

\problem{  \label{dn010} Теорема Муавра-Лапласа является частным
случаем центральной
предельной? }
\solution{да}

\problem{  \label{dn011} Для любой случайной величины  $E(X|X>0)\ge
E(X)$? }
\solution{да}

\problem{  \label{dn012} Если  $X$  - случайная величина, то
$Var(X)=Var(16-X)$? }
\solution{да}

\problem{  \label{dn013} Функция распределения случайной величины
является
неубывающей? }
\solution{да}

\problem{  \label{dn014} Дисперсия случайной величины не меньше, чем
ее стандартное
отклонение? }
\solution{нет}

\problem{  \label{dn015} Для любой случайной величины  $E(X^{2} )\ge
(E(X))^{2} $? }
\solution{да}

\problem{  \label{dn016} Если ковариация равна нулю, то случайные
величины
независимы? }
\solution{нет}

\problem{  \label{dn017} Значение функции плотности может превышать единицу? }
\solution{да}

\problem{  \label{dn018} Если события  $A$  и  $B$  не могут произойти
одновременно,
то они независимы? }
\solution{нет}

\problem{  \label{dn019} Для любых событий  $A$  и  $B$  верно, что
$P(A|B)\ge P(A\cap B)$? }
\solution{да}

\problem{  \label{dn020} Функция плотности может быть периодической? }
\solution{да}

\problem{  \label{dn021} Для неотрицательной случайной величины
$E(X)\ge
E(-X)$? }
\solution{да}

\problem{  \label{dn022} Если $X\sim \chi_{n}^{2}$ и $Y\sim
\chi_{n+1}^{2}$, $X$ и $Y$ -
независимы, то  $X$ не превосходит $Y$?}
\solution{нет}

\problem{  \label{dn023} В тесте Манна-Уитни предполагается
нормальность хотя бы одной
из сравниваемых выборок? }
\solution{нет}

\problem{  \label{dn024} График функции плотности случайной величины,
имеющей
$t$-распределение симметричен относительно 0? }
\solution{да}

\problem{  \label{dn025} Мощность больше у того теста, у которого
вероятность ошибки
2-го рода меньше?}
\solution{да}

\problem{  \label{dn026} Если $X\sim t_{n}$, то $X^{2}\sim F_{1,n}$?}
\solution{да}

\problem{  \label{dn027} При прочих равных 90\% доверительный интервал шире 95\%-го? }
\solution{нет}

\problem{  \label{dn028} Несмещенная выборочная оценка дисперсии не
превосходит
квадрата выборочного среднего? }
\solution{}

\problem{  \label{dn029} Если гипотеза отвергает при 5\%-ом уровне
значимости, то
она будет отвергаться и при 1\%-ом уровне значимости? }
\solution{нет}

\problem{  \label{dn030} У t-распределения более толстые <<хвосты>>,
чем у
стандартного нормального? }
\solution{да}

\problem{  \label{dn031} P-значение показывает вероятность отвергнуть
альтернативную
гипотезу, когда она верна? }
\solution{нет}

\problem{  \label{dn032} Если t-статистика равна нулю, то P-значение
также равно
нулю? }
\solution{нет}

\problem{  \label{dn033} Если $X\sim N(0;1)$, то $X^{2}\sim \chi_{1}^{2}$? }
\solution{да}

\problem{  \label{dn034} Пусть $X_{i}$ - длина $i$-го удава в
сантиметрах, а $Y_{i}$ - в дециметрах. Выборочный коэффициент
корреляции между этими
наборами данных равен $\frac{1}{10}$? }
\solution{нет}

\problem{  \label{dn035} Математическое ожидание выборочного среднего
не зависит от
объема выборки, если $X_{i}$ одинаково распределены?}
\solution{верно}

\problem{  \label{dn036} Зная закон распределения $X$ и закон
распределения $Y$
можно восстановить совместный закон распределения пары $(X,Y)$? }
\solution{нет}

\problem{  \label{dn037} Если ты отвечаешь на 10 данеток наугад, то
число правильных ответов - случайная величина, имеющая
биномиальное
распределение с дисперсией $4$? }
\solution{}

\problem{  \label{dn038} 
Если $P(A)>0$ и $P(A^{c})>0$, то  $E(X)=P(A)\cdot E(X|A)+P(A^{c}
)\cdot E(X|A^{c})$?}
\solution{}

\problem{  \label{dn039} 
Если закон распределения величины $X$ задан табличкой

\begin{tabular}{|c|c|c|}
  \hline
  $x_{i}$ & 0 & 1 \\
  \hline
  Вероятность & 0.5 & 0.5 \\
  \hline
\end{tabular}, то $X$ - нормально распределена. 
}
\solution{нет}

\problem{  \label{dn040} 
Если события $A$ и $B$ независимы, события $B$ и $C$ независимы, то события $A$ и $C$ независимы }
\solution{нет}

\problem{  \label{dn041} 
Если события $A$ и $B$ несовместны, события $B$ и $C$ несовместны, то события $A$ и $C$ несовместны \\
}
\solution{нет}

\problem{  \label{dn042} 
Если $P(A)>P(B)$ то $P(A|B)>P(B|A)$ \\
}
\solution{}

\problem{  \label{dn043} 
Если величины $X$ и $Y$ одинаково распределены и $P(X=Y)=0.9999$, то корреляция $X$ и $Y$ близка к единице. }
\solution{нет}

\section{Несортировано}
% unsorted

\problem{  Гладиаторы\\
На арене две команды гладиаторов, $A$ и $B$. Каждый гладиатор
обладает определенной силой, неизменной по ходу игры. В команде
$A$  всего $N_A$ гладиаторов с силами $\left\{ {a_i } \right\}_{i
= 1}^{N_A } $, в команде $B$  - $N_B$ гладиаторов с силами
$\left\{ {b_i } \right\}_{i = 1}^{N_B } $. Игра проходит в виде
последовательных турниров, в каждом из которых участвует по одному
гладиатору от каждой стороны.\\
Если в очередном турнире встречаются гладиаторы с силами $a$ и $b$
, то вероятность победы первого определяется величиной $\frac{a}
{{a + b}}$. Гладиатор, проигравший турнир, выбывает из игры,
выигравший - возвращается в команду. Исходы турниров независимы.
Это означает, что гладиаторы не устают, но и не приобретают опыта.
Стратегия команды предписывает, какого гладиатора выдвигать на
очередной турнир в зависимости текущего состава команды. Игра
ведется до полного выбывания из игры одной из команд.\\
а)  Зависит ли вероятность победы команды $A$ от используемой
стратегии? Если Вы считаете, что да, то укажите оптимальную
стратегию; если нет, то докажите.\\
 б)  Допустим, что при выборе
гладиатора для очередного турнира команда может учитывать не
только свой собственный текущий состав, но и состав
команды-соперника. Соответственно изменяется и понятие стратегии.
Будет ли зависеть вероятность победы команды от
стратегии в этом случае?}
\solution{}


\problem{  Парадокс гладиаторов-вампиров. [Winkler]\\
В отличие от обычного гладиатора, у победившего гладиатора-вампира
сила увеличивается на силу побежденного им гладиатора-вампира. В
остальном правила поединка такие же, как в предыдущей задаче.
Зависит ли вероятность победы команды $A$ от используемой
стратегии?}
\solution{}


\problem{ Двумерное случайное блуждание\\
Выходя из начала координат 0, частица с равной вероятностью
сдвигается на один  шаг либо  на юг, либо на север, и одновременно
(и тоже с равной вероятностью) на один шаг либо на восток, либо на
запад. После того как шаг сделан, движение продолжается
аналогичным образом из нового положения и так далее до
бесконечности. Какова вероятность того, что частица когда-нибудь
вернется в начало координат?}
\solution{}


\problem{ Трехмерное случайное блуждание\\
Как и в предыдущей задаче, частица выходит из начала координат 0 в
трехмерном пространстве. Представим себе точку 0 как центр куба со
стороною длины 2. За один шаг частица попадает в один из восьми
углов куба. Поэтому при каждом шаге частица с равной вероятностью
сдвигается на единицу длины вверх или вниз, на восток или на
запад, на север или на юг. Какова доля частиц, возвращающихся в
начало, при неограниченном времени блуждания?}
\solution{}

\problem{
В киосках продается 'открытка-подарок'. На открытке есть
прямоугольник размером 2 на 7. В каждом столбце в случайном
порядке находятся очередная буква слова 'подарок' и звездочка.
Например, вот так: \\
\begin{tabular}{|c|c|c|c|c|c|c|}
  \hline
  П & * & * & А & * & О & К \\
  \hline
  * & О & Д & * & Р & * & * \\
  \hline
\end{tabular} \par
Прямоугольник закрыт защитным слоем, и покупатель не видит, где
буква, а где - звездочка. Следует стереть защитный слой в одном
квадратике в каждом столбце. Можно попытаться угадать любое число
букв. Если открыто $n>0$ букв слова 'подарок' и не открыто ни
одной звездочки, то открытку можно обменять на $50\cdot 2^{n-1}$
рублей. Если открыта хотя бы одна звездочка, то открытка
остается просто открыткой. \par
а) Какой стратегии следует придерживаться покупателю, чтобы
максимизировать ожидаемый выигрыш? \par
б) Чему равен максимальный ожидаемый выигрыш? \par \par
\emph{Подсказка}: Думайте! }
\solution{}

\problem{
Задача, дороги из А в Б через С1, С2, С3 \par
Вопросы про вероятности}
\solution{}


\problem{
In the game of Racko you have 10 slots for cards to go in. At the
start, cards are dealt to each player and put in order that they
were dealt starting from the <<highest slot>>. The cards are
numbered 1-60. The object is to get all of your cards in order, by
means of drawing and switching in a discard pile. When all your
cards are in order, you call <<Racko>> and the hand is over.\par
What is the probabilty that you will be dealt a <<Racko>>?\par
Source: aops }
\solution{}


\problem{ $X_1, X_2,\dots,Xn$ are independent random variables, uniformly distributed on $[0,1]$. 
What is the probability that $X_1+X_2+...X_n<1$?\par
Solution:\par
We can compose a recurrence relation for the densities of sums in $[0,1].$
A simple convilution formula gives the result $p_n(x)=x^n/n!$ for $x\subset [0,1].$}
\solution{}


\problem{ You break a bar of length 1 unit into 2 pieces, choosing the break point uniformly along the length. What's the mean length of the largest piece? }
\solution{ The largest piece is uniform on $[1/2, 1]$, hence the mean is $3/4$.}

More general problem.
Let $Z_n$ be the smallest piece arising from breaking up the unit stick into n pieces using $(n-1)$ uniform 
$[0,1]$ random variables, say, $X_1,X_2,...,X_{n-1}$. We prove that
$$
  P(Z_n > c) = (1-cn)^{n-1}.
$$
Ror any $0 < c < 1/n$, because then a simple integration by parts calculation shows that $E(Z_n) = 1/n^2$.
To prove this equality, it's enough to show that
$$
  P(X_1 < X_2 < \cdots < X_{n-1}, Z_n > c) = {\frac{(1-cn)^{n-1}}{(n-1)!}}.
$$
Because then I can multiply by $(n-1)!$, i.e., adding this last probability over all possible orderings of the 
$X_i$'s, to cover all possibilities for $\{Z_n > c\}$. Now finally, how do I prove this last equality?
Saying that $X_1 < ... < X_n$ and $Z_n > c$ is equivalent to saying that
$$
  X_1 > c, X_2 - X_1 > c, X_3 - X_2 > c, \ldots, X_{n-1} - X_{n-2} > c, 1 - X_{n-1} > c.
$$
which is equivalent to the following range for the $X_i$'s:
$$
  X_1 \in (c, 1-(n-1)c) \\
  X_2 \in (X_1 + c, 1-(n-2)c) \\
  \ldots \\
  X_k \in (X_{k-1} + c, 1-(n-k)c) \\
  \ldots \\
  X_{n-1} \in (X_{n-2} + c, 1 -c)
$$
So now, to calculate the probability of this event, we just have to integrate 1 over this region, 
i.e., we have to calculate the integral
$$
  \int_c^{1-(n-1)c} \int_{x_1 + c}^{1-(n-2)c} \cdots \int_{x_{n-2} + c}^{1-c} 1\ dx_{n-1} dx_{n-2} \cdots dx_1
$$
To work through the first (n-2) integrals, we note the following equality for any 2 <= k <= n-1, which is simple enough to calculate:
$$
  \int_{x_{k-1} + c}^{1-(n-k)c} \frac{(1-(n-k)c-x_k)^{n-(k+1)}}{(n-(k+1))!} dx_k = \frac{(1-(n-(k-1))c - x_{k-1})^{n-k}}{(n-k)!}
$$
Then using induction, the integral we want to calculate reduces to integrating
$$
  \int_c^{1-(n-1)c} \frac{(1-(n-1)c - x_1)^{n-2}}{(n-2)!} dx_1
$$
But if we think of $x_0$ as equalling $0$, then the integral equality above for $2 \le k \le n-1$ also applies when $k=1$, 
which gives us the answer we want.

The size of the kth largest piece is given by
$$
  \frac{1}{n}\left(\frac{1}{n} + \cdots + \frac{1}{k}\right).
$$ %qed 


\problem{
Suppose I draw a ball one at a time with replacement from an urn with 100 balls and there's 
a unique label for each of the balls so I know which one has been drawn. 
Find the expected number of draws I have to make in order to get all 100 balls.}
\solution{
Recall that the problem concerns a prudent shopper who tries, in several attempts, 
to collect a complete set of $N$ different coupons. Each attempt provides the collector with a coupon randomly chosen from $N$ known kinds, and there is an unlimited supply of coupons 
of each kind. It is easy to estimate the expected waiting time to collect all $N$ coupons:
$$
  E\{ \mbox{time to collect $N$ coupons} \} = 1+\frac{N}{N-1}+\frac{N}{N-2}+\dots+N = N(\log N+\gamma+O(\frac{1}{N}))
 $$
Similarly,
\begin{align}
  E\{ \mbox{time to collect $\frac{N}{2}$ coupons} \} &= 1+\frac{N}{N-1}+\frac{N}{N-2}+\dots+\frac{N}{\frac{N}{2}+1} \notag\\
  &= N(\log N+\gamma+O(\frac{1}{N}) - \log\frac{N}{2}-\gamma-O(\frac{1}{N})) \notag\\
  &=N\log2+O(1) \notag
\end{align} }


\problem{ You are waiting for a bus. They arrive \par
a) with rate $\lambda = 1/10$ according to a Poisson process 
(so we have an average time of $10min$ between buses). \par
b) with time between bus arrivals distributed as $U[0,20]$ (the average is also $10min$). \par
If you take a random time what is your average wait for a bus? }
\solution{
a) For $U[0,20]$ distribution we have as follows.
Consider the lengths of times between buses, $X$, this has density $p_X(x) = 1/20$ on $[0,20]$
and $0$ outside $[0,20]$ interval. The probability density that you pick an interval of length $t$ is, 
$$
  p\{ \mbox{pick an interval of time } x \} = x p(x) /E X = t/100.
$$
The mean picked interval is thus $40/3$ (take the expectation of the above density).
But, when you pick an interval, the time you wait until the end of it is on average half of it's length, 
which gives the answer $20/3$. 

b) We use the basic property of Poisson processes with intensity $\lambda=1/10$ that time interval 
between two arrivals is distribited as $Exp(\lambda)$. 
Using the "lost of memory" property of Exponential random variable
$$
  p\{ X > y+x | X>y \} = p\{ X>x \}.
$$
we get that the average time till the next arrival is $10$ min.
Another funny consequence here is that even after waiting 10 mins, 
the expected time is still unchanged: it's 10 more minutes again. }



\problem{ Let $Y_n$ be the sum of n rolls of a fair 6 faced die and 0 for n=0.
Determine the probability that $Y_n$ divides $13$ as n approaches infinity.}
\solution{Решение 1. The answer is $1/13.$ We have 
$$
  p\{ Y_n \mbox{ divides } 13\} = 1/6 \cdot 
  ( p\{Y_{n-1} +1 \mbox{ divides } 13\} + 
    p\{Y_{n-1} +2 \mbox{ divides } 13\} + \dots + 
    p\{ Y_{n-1} + 6 \mbox{ divides }13\} ).
$$
The probability above is conditional to the first roll. Let 
$$
  U_n(0)=p\{ Y_n \mbox{ divides }13\}, U_n(1)=p\{Y_n +1 \mbox{ divides }13\},\dots,
  U_n(k)=p\{ Y_n +k \mbox{ divides } 13\} \mbox{ for } k=0,\dots,12.
$$
We have
$$
  U_n(0) = 1/6 \cdot (U_{n-1}(1)+\dots+ U_{n-1}(6)),\quad
  U_n(1) = 1/6 \cdot (U_{n-1}(2) + ...+ U_{n-1}(7) ),\quad \mbox{etc...}
$$
We also have $U_n(0)+\dots+U_n(12)=1$.
When $n\to\infty$, assuming that $U_n(k)$ converge to $U(k)$, we have:
$$
  U(0) = 1/6\cdot(U(1)+...+U(6)),\quad 
  \mbox{etc... }
  U(0)+...+U(12)=1.
$$
So $U(k) = 1/13$ for all $k=0,\dots,12.$

Решение 2. Consider a $13\times 13$ matrix, such that the $(i,j)$ entry represents the following probability: 
given that $Y_n$ is $i \mbox{ mod } 13$, what is the probability that $Y_{n+1}$ is $j mod 13$? 
This matrix is <<doubly stochastic>> because the sum of the entries in each row and each column equals 1. 
This means that the asymptotic distribution of $Y_n$ is uniform. 
This requires some application of Markov chain theory. }


\problem{ Let $X1,X2,\dots,X_n$ are independent $U(0,1)$ r.v.Let $Z$ is the random variable that equals $k$ 
for which the sum
$$
  S_k  = (X_1 + X_2 + \dots + X_n)
$$
exceeds $1$ for the first time. Find $E Z.$ }
\solution{
The probability that $n$ uniform random variables sum to less than $1$ is $1/n!$. 
So the probability the sum goes over $1$ for the first time on the $n$-th random variable is 
$1/(n-1)! - 1/n! = (n-1)/n!$. The expectation is the sum of $n*(n-1)/n! = 1/(n-2)!$ for $n = 2$ to $\infty$, 
which equals the sum of $1/n!$ for $n = 0$ to infinity, which is $e$.

Without much ado, here is a simpler and more general solution. Let $f(x)$ be the expectation of number of steps 
for exiting barrier $x$ for the first time. The expectation that is in excess of $1$ is 
$$
  \int_0^x p(t)f(x-t)\ dt = f(x)-1
$$
where $p(t)$ is the characteristic function for $(0,1)$. For $x\le 1$, we differentiate the above integral equation 
with respect to $x$ and turn it into a simple differential equation. Solve it, we have
$$
  f(x) = e^x, x\in [0,1].
$$
For $x>1$, we have
$$
  \int_{x-1}^x f(t)\,dt = f(x)-1.
$$
Differentiate and solve, we obtain the following recursive solution.
$$
  f(x) = e^x\big(f(n)-\int_n^x e^{-t}f(t-1)\,dt\big),\ x\in (n,n+1], n\in \mathbf N.
$$ }




\problem{ you throw a fair coin until you get 8 heads in a row. What's the probability you will see 8 consecutive tails (exactly) in the sequence prior to stopping?}
\solution{answer=1/2 }


\problem{You have an unfair coin. Probability of tossing a head is p. What is the expected number of tosses needed to get N consecutive heads? }

\solution{ Expected time to get N consecutive heads = Expected time to get (N-1) consecutive heads + p*1 + (Expected time to get N consecutive heads + 1)(1-p)

letting $x_n$ be the obvious (expected time for n consecutive heads):

$x_n = x_{n-1} + p + (x_n + 1)(1-p)$\par

so we get a recursion relation $x_n = (1/p)(x_{n-1} + 1)$\par

since $x_1 = 1/p$, we can find all the others.\par

This then seems to be the recursion relation for $x_n = \sum(1/p^i)$, where the sum is between i=1 and i=n }


\problem{ i offer to play a card game with you using a normal deck of 52 cards. the rules of the game are that we will turn over two cards at a time. if the cards are both black, they go into my pile. if they are both red, they go into your pile. if there is one red and one black, they go into the discard pile. we repeat the two card flipping until we've gone through all 52 cards. whoever has more cards in their pile at the end wins. i win if there is a tie. if you win, i pay you a dollar. how much would you pay to play this game? }
\solution{ tie is inevitable }




\problem{ You randomly pick three numbers on (0;1). Let's call them a,b,c. What is the probability that there is a triangle which edges of lenghts a,b,c?


Related questions are:

a,b are two random numbers picked from (0,1) -- non-correlated. What are the expected values of :

1. Min (a,b) (answer: 1/3)\par
2. Min(a+b,1) (answer: 5/6)\par
3. Min(a+b,1) - Abs(a-b) (answer: 1/2) ... This gives answer to the question below.


In this case, I think the probability of being able to form an n-simplex is 1-1/n! This is because we are given n+1 "areas" for the "faces", which gives us n+1 inequalities. Each inequality cuts out an n-simplex from the unit cube of volume 1/(n+1)!. These n-simplices are nonintersecting, and there are n+1 of them, so the the volume we have left is 1 - (n+1)/(n+1)! = 1 - 1/n!. Note that in 2 dimensions, this gives us the correct answer, 1/2.}
\solution{}


\problem{ WC 20XX final game Brazil-Germany 8:6 \par
Brazil scored the first goal. After that, the score follows a random
sequence until it reaches 8:6. What is the probability that Brazil
has been at least 1 goal ahead for the whole duration of the game? }
\solution{

we can also use the reflective principle to do the counting.

let N be the difference of games brazil won and german won, the status of the game can
be denoted as a pair (t, N). given brazil won the first game, the number of total possible
paths is C(13,6). the game starts at (1,1), and ends at (14,2). we need to count all paths
from (1,1) to (14,2) without touching N(t)=0. consider the complement and use N=0 as
the mirror, all paths ever reach N=0 can be viewed as starting from (1,-1) and ending at
(14,2), i.e., the number of such paths is C(13,5). so the probability is

1 - C(13,5) / C(13,7) = 1/4 \par

I don't know if this is legit but could you do it this way.

ordinarily, if you just know the results of two candidates the probability that the winner was always ahead is (w-l)/(w+l) where w = number votes for the winner, and l= number votes for the loser.

So in this case we have (8-6)/(8+6) = 2/14

however we are given that brazil won the first game.

so if we divide 2/14 by the conditonal probability of brazil winning the first game we get (2/14)/(8/14) = 1/4}


\problem{ What is the probability that the lottery draw will contain two or more consecutive integers? }
\solution{}


\problem{ Some components of a two-component system fail after receiving a shock. Shocks of three types arrive independently and in accordance with Poisson processes. Shocks of the first type arrive at a Poisson rate of ?1 and cause the first component to fail. Those of the second type arrive at a Poisson rate ?2 and cause the 2nd component to fail. The third type of shock arrives at a Poisson rate ?3 and causes both components to fail. Let X1 and X2 denote the survival times for the two components. \par
Show that X1 and X2 both have exponential distributions? }
\solution{

One can attack this successfully from a high level, with only a few key observations:\par
1) For any dt > 0 and any t, the conditional probability that, for example, 
X1 > t + dt given that X1 > t, is a function of dt alone, i.e., independent of t.\par
This should be clear without writing anything down.\par
2) Distributions satisfying this criterion can be shown to be exponential.}


\problem{ You're throwing a fair dice and each time you add the outcome to the total (you start from 0). What is the probability that the path of this process will visit a number N? What happens when $n\to\infty$?}
\solution{ A sequence p(n) is defined using recurrent formula:\par
p(n) = 1/k*sum(over j=1, j=k)(p[n-j]) (1)\par
The values of p[1]...p[k] represent the initial
conditions. \par

The sequence converges to some finite value 
at n->+inf for any initial conditions (easy).\par

k=6, p[1]=p[2]=...=p[5]=0, p[6]=1 is a special
case for the probability to get cumulative sum
"n" in a process of throwing a dice.\par

We shall prove a general formula that \par
p(n) -> 2/(k+1) at n->inf \par
if p[1]=...p[k-1]=0, p[k]=1\par

Proof:\par
------

Let's denote $S_m$ to be the sequence (1) with
initial conditions:\par

p[1]=...p[m-1]=0, p[m]=1, p[m+1]=...p[k]=0\par

In that notation, we are proving that $S_k -> 2/(k+1)$ \par

First, notice that $S_1 -> 1/k*S_k$\par

Now suppose we have proven that $S_m -> m/k*S_k$.
Consider a new sequence $S_(m+1)-S_m$ (i.e.
every element is a difference of the corresponding
elements). That new sequence has the following 
initial conditions:\par

0,0,...-1,1,..0 \par
p[1]=...p[m-1]=0, p[m]=-1, p[m+1]=1, p[m+2]=...p[k]=0\par

It is easy to see that $(S_(m+1)-S_m) -> 1/k*S_k$,\par
i.e. $S_(m+1) -> (m+1)/k*S_k$ and therefore\par
$S_m -> m/k*S_k$ for any m = 1,...k\par

Now consider a sequence $S_1+...+S_k$, i.e.\par
the one having initial conditions\par
p[1]=...p[k]=1 \par

That sequence is stable and equal to 1 at every point.\par

At the same time, that one also converges\par
to $(1/k+2/k+...k/k)*S_k = (k+1)/2*S_k$\par

We obtain: $(k+1)/2*S_k -> 1$ i.e. $S_k -> 2/(k+1)$ Q.E.D.\par

A special case k=6: p(n) -> 2/7 = 1/3.5 }

\problem{We have a call option on coin flips - the payoff of the "security" is the
number of heads that comes up after a number of flips. The strike price of
the option is 2. 

Need to value the option for 4 coin flips. (Interest rate is zero.)

Also, find the delta of this option. }
\solution{
You need one more thing to solve this problem, the price of bets on individual coin flips. If we assume that we can bet \$1 at even money, so we win \$1 if a flip is heads and pay \$1 if the flip is tails, then we have the following values:

4 - 0 \$2
3 - 1 \$1
2 - 2 \$0
1 - 3 \$0
0 - 4 \$0

3 - 0 \$1.50
2 - 1 \$0.50
1 - 2 \$0
0 - 3 \$0

2 - 0 \$1
1 - 1 \$0.25
0 - 2 \$0

1 - 0 \$0.625
0 - 1 \$0.125

0 - 0 \$0.375

The delta is \$0.25, that's how much you bet on the first flip to replicate the option \par
Yes we need to bet \$0.25 on the first flip, but the delta of the option is 0.5.
After the first flip, the underlying will be worth \$2.5 or \$1.5, whereas the call will be worth \$0.625 or \$0.125. So Delta=0.5. \par
Your answer is better than mine. You have computed the delta with respect to the underlying, which is the sum four coin flips. I was computing it with respect to a security that pays \$1 for heads and -\$1 for tails. So you would buy half a coin flip, paying \$0.25 to get \$0.50 for a head and \$0.00 for a tail. I would buy one-quarter of a \$1 coin flip bet, getting \$0.25 for a head and paying \$0.25 for a tail. It all comes out to the same thing, but it shows you always have to be careful what your delta is computed with respect to.}




----------------------------------------\par



Markov property - стоит взглянуть (считается интеграл с броуновским движением)\par

Pool Puzzle - соотношение скоростей плавающего и ловящего в круглом бассейне \par

An expectation problem - складываем равномерные величины, до тех пор пока сумма не станет больше 1 \par

Series gamble - возможно неплохая задача \par
shrimp by the pound - возможно неплохая задача \par
7 color hats puzzle - решение задачи для одновременного раздавания шляп\par

Harmonic tossing - про сходимости по вероятности - хорошая задача \par

Life is full of error and round off.\par

Для стохана - поиграться с какой-нибудь функцией типа реализации броуновского движения (попробовать посчитать от нее обычный интеграл), построить график \par

Там есть пробабилити: \par
http://www.mathematik.uni-bielefeld.de/~sillke/ \par
И досмотреть пункты 4, 6, 19 (8 возможно досмотреть) на cut-the-knot \par
http://www.mathpages.com/home/iprobabi.htm \par

Several old brainteaser (or math) questions -- good to work for fun - стоит посмотреть \par

123 theorem and its extensions \par

aops:\par
88391\par
97893, 88977 - максимизация ожидаемого сохраненного броска\par
154385 - expected number of rounds \par
152421 - про экспоненциальное распределение с страховые случаи\par
38284 - первый шаг \par

http://home.att.net/~numericana/answer/weighing.htm \par
Общее решение задачи про взвешивания \par

Prisoner's Dilemma by William Poundstone\par

про "случайные" с т.зр. человека последовательности: \par
$http://www.wilmott.com/messageview.cfm?catid=26\&threadid=17247$ \par

В одной пачке 56 эмэндэмсин (50 грамм). Эмэндэмсины бывают 5 цветов (не считая синего). \par


Only math nerds would call $2^{500}$ finite (Leonid Levin) \par




topic "simple fun" (про игру в половину от среднего и эмпирические данные) \par





\problem{ >You have a black box with N balls in it - each of a different color.
>Suppose you take turns as follows - randomly pick a ball in each hand,
>and paint the left hand ball the same color as the right hand ball.
>You replace both balls in the box before the next turn.
>
>How many turns do you expect before all balls are the same color?}
\solution{
I worked this out as follows. I think this might be more or
less along the lines of Robert Israel's analysis, which
I didn't actually follow, sad to say. \par

We just analyze one color, looking at the cases where it
"wins". After any draw and replacement ( a turn,) the urn
is in a state i with i red balls. Of course, red (say) "wins"
if i = N, and "loses" if i=0. We note that the probablity
of going from i -> i+1 ( i>0>N ) is always equal to the
probability of i -> i-1, and this value is i*(N-i)/(N-1)/N.

So, we can treat this as a random walk starting from 1
with absorbtion at i=0 and i=N. However,the expected number
of turns per step depends on the state i and is given by
N*(N-1)/i/(N-i)/2 . The 2 is because of the 2 equally
probable transitions.

Now we just have to evaluate the average number of visits
to each state 0<i<N, given that red wins. Since each visit
must terminate with a step, we multiply the visits by the
number of turns per step for each state to get the expected
number of turns taken in each state, and sum over states.

I get that the probability of winning starting from state
i is just i/N, and I get for the average number of visits
to a state, starting from i=1, ( win or lose ) 2*(N-i)/N.

Then the average number of "winning visits" to state i
per trial is :

 	i/N  * 2*(N-i)/N 

and the average number of turns accounted for by these
visits is:

	i/N  * 2*(N-i)/N  * N*(N-1)/i/(N-i)/2 = (N-1)/N
	
but the expected number of wins is 1/N times the number
of trials, so the expected number of turns per winning trial
in each state i is just N-1, and the expected number of turns
in a win for red, or for any other color, is just the number
of nonterminal states times the expected number of turns
in each state, or $(N-1)^2$.

Lew Mammel, Jr.

Yes, it is $(N-1)^2$.  
Let Si be the event that the balls eventually end up all coloured i, and
Ii the indicator of this event (1 if it occurs and 0 if not).
Let T be the number of steps until all balls are the same colour.
Let Ai be the initial number of balls of colour i (1 in the problem as 
given,
but let it vary).
Let V(k) = E(I1 T | A1=k) (as far as the random variable I1 T is 
concerned, 
it doesn't matter what Ai is for i <> 1).  We have V(0) = V(N) = 0.
Note that E(I1 | A1=k) = k/N.
By a "first-step analysis", you can show that
    $V(k) = k V(1) - (N-1) \sum_{j=1}^{k-1} j/(N-k+j)$ for 1 <= k <= N
For $k=N$ this means $V(1) = (N-1)^2/N$.
But what we want is
    $E(T | A1=A2=...=AN=1) = \sum_{i=1}^N E(Ii T | Ai=1) = N V(1) = (N-1)^2$.

Also interesting: if N is even and you start out with 2 balls of each of 
N/2
colours, the expected number of steps is $(N-1)^2 - N/2$.

-- 
Robert Israel}


\problem{ If W(t) is a standard Wiener process, the first passage time T(x) is defined as:

T(x) = inf{t: W(t) = x}

T(x) has density function $|x|Exp[-x^2/(2t)]/Sqrt[2 Pi t^3]$

I have a problem understanding why $E[T(x)]$ is Infinity for all x > 0.}
\solution{
Solution: \par
If $E[T(x)]$ were finite then according to Wald identities E[W(T(x))] were equal to 0, but it is not as W(T(x)) = x.}


Now let's say that you don't know the stochastic process followed by the stock price, but you are given the price at time 0 of a call with expiration at time T as a function of the strike X (that is, given any input strike from 0 to infinity, you can output the price of the call). How would you use this information to determine the risk-neutral distribution of the stock price at time T?\par
Let f(S, T) be the terminal risk-neutral pdf for the stock - the 2nd deriv of the call price with respect to strike is the discounted value of the risk-neutral pdf:\par
C''(K, T) = exp(-rT) f(K, T)\par
This is the well-known Breeden-Litzenberger result\par








\problem{ IBM: \par
Consider a loop of string of unit length. Suppose we cut the string independently and at random in n places. This will divide the loop into n pieces.\par
This month's puzzle asks\par
1. What is the expected (average) size of the smallest piece?\par
2. What is the expected (average) size of the largest piece?}
\solution{
Identify the loop of string with real numbers from the unit interval in the obvious way. We may assume without loss of generality that one of the cuts is at 0. Fix n. Let x be the expected size of the smallest piece. Let f(t) be the probability that the smallest piece has size at least t. Note it is easy to see that x equals the integral from 0 to 1 of f(t). We claim f(t)=(1-n*t)**(n-1) for t < 1/n, f(t)=0 otherwise. Let t < 1/n. Then we assert configurations of n points on a unit loop of string (with one point at 0) such that cutting at those points produces pieces of length at least t correspond to configurations of n points on a loop of length (1-n*t) with one point at 0 and otherwise unrestricted. This follows because the configurations map into each other by deleting (or adding) length t of string after each point. The claim follows because the density of the configurations on the shorter loop is (1-n*t)**(n-1). We can now compute x by evaluating the integral of (1-n*t)**(n-1) with respect to t from 0 to 1/n. Let s=(1-n*t). Then the integral becomes (1/n)*(integral from 0 to 1 of s**(n-1) with respect to s) or 1/(n*n). So the answer to the first part is 1/(n*n).\par

The answer to the second part can be derived by applying the above idea recursively and inductively. Consider a set of n points on an unit loop (with one point at 0). Let x be the size of the smallest piece if the loop is cut at those points. Consider deleting a piece of length x after each point. This will merge two of the points leaving a set of (n-1) points on a loop of length (1-n*x) (still with one point at 0). Since the expected size of x is 1/(n*n) the expected size of the smaller loop is (1-1/n). Let y be the size of the smallest piece when the smaller loop is cut at the remaining points. Clearly the expected size of y=(1-1/n)*(1/(n-1)*(n-1))=1/(n*(n-1)). Now the size of the second smallest piece in the original loop is x+y which has expected size (1/n)*((1/n)+(1/(n-1)). Continuing in this way we find that the expected size of the kth smallest piece is (1/n)*((1/n)+(1/(n-1))+ ... +(1/(n-k+1))). (Note the sum of the expected sizes of all the pieces is 1 as expected.) So the expected size of the largest piece (ie the nth smallest piece) is (1/n)(1+1/2+...+1/n). It is well known that the sum (1+1/2+...+1/n) behaves like ln(n) for large n so asymptotically the largest piece has size ln(n)/n.}






\problem{ The Volume of a Simplex \par
The length of the unit interval is 1.  The area of the triangle bounded by the x and y axes, and x+y ? 1, is 1/2.  The volume bounded by the xy, xz, and yz planes, and x+y+z ? 1, is 1/6.  Can we generalize this? }
\solution{

The volume of a hypersimplex is 1/n!.  Proceed by induction on n.  Assume an n dimensional simplex, where all n variables are greater than 0, and their sum is less than 1.  Let x be the variable of integration in a nested integral.  At the floor, when x = 0, we find a simplex in n-1 dimensions.  Its volume is 1/(n-1)!.  As we move along the x axis, the sum of the remaining variables is restricted to 1-x, rather than 1.  This is a scaled version of the n-1 simplex.  When all the variables are scaled by a factor of k, the volume is multiplied by kn-1.  Thus the volume of the simplex on the floor is multiplied by (1-x)n-1.  The integrand is therefore (1-x)n-1/(n-1)!.  We can replace 1-x by x; that just reflects the shape through a mirror.  Thus the integral is xn/n!.  Evaluate at 0 and 1 to get 1/n!. \par

The generalized octahedron in n dimensions consists of n variables such that the sum of their absolute values never exceeds 1.  This is actually a bunch of simplexes placed around the origin.  In fact we need 2n simplexes to make an octahedron.  In 3 dimensions we place 8 simplexes around the origin, one for each octant.  Each simplex presents one face of the octahedron.  The volume of the generalized octahedron is 2n/n!.  This approaches 0 as n approaches infinity.}






I just got a copy of Paul \& Dominic's Guide to Getting a Quant Job, which I hope everyone reads. I say not not for your benefit in getting a job, but for my benefit in saving time and annoyance when looking to fill a job. I'm posting here because it recommends this forum for practicing for brain teaser interview questions. That's an excellent idea. But if that's what you're here to do, you should also know how your answer will be scored. So here's my take on it, I invite other opinions and comments.\par

I ask three types of brain teaswer type questions.\par

(1) Really easy ones. This is the equivalent to moving the mouse around at random to see if the computer is frozen. An example might be, "Suppose I flip a fair coin 10 times and pay you \$1 if the first flip is heads, \$2 if the second flip is heads, up to \$10 if the 10th flip is heads. What is your expected payout?" I assume that anyone applying for a quant job knows how to sum the numbers from 1 to 10 (or has memorized the answer) and can divide by 2. If not, I don't want them anyway. I score the answers as follows:\par

10 - Listens carefully as the problem is posed, thinks for a minute, then answers \$27.50 without embellishment.\par
8 - Same as above, except asks for unnecessary clarification, appears suspicious of a trap, needs to use pen and paper or feels it necessary to explain the reasoning.\par
6 - Same as 8 or 10, but gets a wrong answer between \$20 and \$35, or a clearly unreasonable wrong answer but says it's unreasonable.\par
4 - Same as 8 or 10, but gets a clearly unreasonable wrong answer and doesn't mention it.\par
2 - Cannot understand the question, or pretends to misunderstand it to avoid giving a numerical answer.\par
0 - Throws out random phrases and numbers and watches my face hoping for some kind of hint, like Clever Hans, the horse that could do math.\par

The point here is that good quants are confident of their ability to recognize and solve simple problems. Most applicants are not good quants. You can do well on exams without this skill, because problems are often set at an expected moderate level of difficulty, if it's too easy or too hard you probably misunderstood it. Real work is not like that. If you're too scared to give a simple answer to a simple question, you're not going to work out.\par

(2) Hard ones, like many of the ones you will find here. Here the scoring is:

10 - Listens carefully as the problem is posed, thinks for a minute, then says anything intelligent that indicates they understand why the problem is hard, and shows some confidence at being able to solve it. Right or wrong doesn't matter.\par
8 - Answers immediately and correctly (I assume in this case they've heard it before).\par
6 - Says, "I've heard it before," and answers correctly\par
4 - Says, "I've heard it before," and answers incorrectly, or in a way that\par shows they don't understand the answer, or can't remember the answer\par
2 - Cannot understand the question, or pretends to misunderstand it to avoid giving an answer.\par
0 - Throws out random phrases and numbers and watches my face hoping for some kind of hint, like Clever Hans, the horse that could do math.\par

There are people who expect to get the right answer, and are happy to be judged on their actual level of intelligence. There are others who know in their hearts they will never get the right answer, and/or hope to mislead you into thinking they are smarter than they are. Only the first group will get considered for the job. I judge almost entirely on demeanor. Good people are eager to get these questions, bad people dread them. Good people are proud when they get it right and honest but unembarassed when they get it wrong. Bad people make excuses and complaints either way.\par

(3) Open-ended ones, like "Give me your personal subjective 50\% confidence interval for the Gross Domestic Product of Finland in USD."\par

10 - Without prompting comes up with a reasonable approach like, "I know it's a small country, but it's not so tiny I've never heard of it, so I'll guess 5 million population. It's not known as particularly poor or rich for Northern Europe, so I'll guess \$10,000 per capita income. That gives me a point estimate of \$50 billion. National income and GDP are close enough to each other for this approximation. For a 50\% confidence level I'd go 2 to 20 million population and \$5,000 to \$20,000 per capita income, that gives me \$10 billion to \$400 billion. Considering everything, I think that's too big a range, I'll go with \$25 billion to \$150 billion. If I had to bet, I'd be equally happy to bet the true number is inside as outside that range."\par
8 - Same as above but needs prompting and either uses some clearly unreasonable numbers or makes numerous math errors.\par
6 - Thinks for a while and comes up with a reasonable range, but can't discuss it intelligently.\par
4 - Cannot understand what a 50\% confidence interval is.\par
2 - Keeps asking for more information, or explains how they would look up the answer.\par
0 - Throws out random phrases and numbers and watches my face hoping for some kind of hint, like Clever Hans, the horse that could do math.\par

Here I'm looking for ability to understand and follow instructions, some small amount of practical sense and knowledge of the world and ability to use quantitative reasoning. I retired this question when a candidate outsmarted me. I'd used it for about two years and gotten scores from 0 to 10. Then someone said "My interval is all dollar amounts that round to odd integer amounts." He turned a (3) question into a (2) one, and got the job.\par





\problem{ There are n points 1, 2, 3, ... n arranging in order on a circle. If the i th point is n, then the i+1 th point is 1.\par
We choose a pair of adjacent points at random with equal probability of 1/(n-1). We continue to choose
pairs from the points on the circle at random. If one of the points of the pair has been chosen before, we disregard this pair.
The process is repeated until no new pair can be chosen and only isolated points remain.\par
What is the mean number of isolated points?}
\solution{Solution: may be a bad answer}





\problem{ A bar in a town has 25 seats in a row. The folks in this town are antisocial, so they only take a seat of which the adjacent seats are empty, or leave. \par
If you are the bartender, how can you seat your first guest to get more people (expected) at your bar?}
\solution{
Solution: (maybe computational)\par
Has anyone tried to calculate this maximum expectation? I see that seat num 3 and num 23 gives you the highest expectation, however, the value I am getting is 11.38 \par

Here is how I solve it.\par
Let E(n) be the expected number of antisocial people you can seat in n seats. For example,\par

E(0) = 0\par
E(1) = 1\par
E(2) = 1\par
E(3) = 1/3 * (1 + E(1)) + 1/3 * (1 + E(0)) + 1/3 * (1 + E(1)) = 1 + 2/3 * E(1)\par
E(4) = 1/4 * (1 + E(2)) + 1/4 * (1 + E(1)) + 1/4 * (1 + E(1)) + 1/4 * (1 + E(2)) = 1 + 2/4 * (E(1) + E(2))\par
E(5) = 1 + 2/5 * (E(1) + E(2) + E(3))\par
and
E(n) = 1 + 2/n * (E(1) + E(2) + ... + E(n-2)) n = 2, 3, ...\par

The last expression can also be written as\par

E(n) = E(n-2) + 2/n * (1 + E(n-3)) n = 3, 4, \par

Now, let A(m) be the expected number of antisocial people you can seat in 25 seats provided that the first person takes the mth seat. Then,\par

A(1) = 1 + E(23)\par
A(2) = 1 + E(22)\par
A(3) = E(1) + 1 + E(21)\par
A(4) = E(2) + 1 + E(20)\par
and so on\par

I am getting A(3) = 11.38 \par
It is interesting that A(m) is oscillating.}





\problem{ Найти $\sum (-1)^{k} / C_{n}^{k}$ \par
вопрос: где бы ее использовать? \par
answer: $n$-odd => 0, $n$ - even => $2-\frac{2}{n+2}$ }
\solution{
solution: \par
$n!S_{n}=\sum (-1)^k\cdot k!(n-k)!$ \par
$(n+1)!S_{n+1}=\sum (-1)^k\cdot k!(n+1-k)!=0$ \par
or:\par
$(n+1)!=\sum (-1)^k\cdot k!(n+1-k)!$ \par
$(n+2)n!S_{n}-(n+1)!=...=(n+1)!$ \par
Можно решать через арифметику (задача про $f(a,b)$ заменить биномиальный коэффициент на интеграл) }




\problem{ Joric and social egalitarianism\par
Source: Romanian TST 5 2007, Problem 2\par
The world-renowned Marxist theorist Joric is obsessed with both mathematics and social egalitarianism. Therefore, for any decimal representation of a positive integer $n$, he tries to partition its digits into two groups, such that the difference between the sums of the digits in each group be as small as possible. Joric calls this difference the defect of the number n. Determine the average value of the defect (over all positive integers), that is, if we denote by $\delta(n)$ the defect of $n$, compute $\lim_{n \rightarrow \infty}\frac{\sum_{k = 1}^{n}\delta(k)}{n}$. }
\solution{
Solution: \par
It is easy to observe that $\delta(n)$ is always less than 10. If you partition the digits in two groups with difference of sums $\delta(n)$ and $\delta(n)\geq 10$ then move one of the digits from the group with bigger sum to the other group. Hence $\delta(n)$ is bounded by 10. 
I claim that the average is $\frac{1}{2}$. It can be seen that this number is bigger than or equal to $\frac{1}{2}$, because if the sum of the digits of n is odd then $\delta(n)\geq 1$, and the probability of having an odd digital sum is $\frac{1}{2}$. 
One can also see that the probability of a number to have at least 10 digits of 1, is 1 (Really easy to show). Now the average of $\delta(n)$ over all numbers will be equal to the average over the special numbers I mentioned (The ones having at least ten digits of 1) 
For each of these numbers like n, just remove the first ten digits of 1 to obtain $m$. Then we know that $\delta(m)<10$. Now using the extra 1 digits we have at hand we can distribute them among the two groups we have for $\delta(m)$ to decrease the difference between sums to either 0 or 1. If the sum of digits of $m$ is odd we can reduce $\delta(m)$ to 1 and in the other case to 0. One can again easily observe that the probability of $m$ having an odd digital sum is $\frac{1}{2}$. So for half of the numbers (with respect to probability) we have $\delta(n)=0$ and for another half (disjoint from the last half) we have $\delta(n)\leq 1$. So the average we are trying to find is less than or equal to $\frac{1}{2}$ and hence my claim is proved.}







(wilmott) "вероятности" для натуральных чисел \par
Probability that $2^n$ begins with 603245? \par
Prime factor problem \par



\problem{ Открытая задача.
Правильную монетку подбрасывают неограниченное количество раз. После любого подбрасывания Аня можете сказать <<Стоп>>. Как только Аня говорит <<Стоп>>, игра оканчивается и Аня получает выигрыш равный доле выпавших <<орлов>>. Например, если Аня сказала стоп после последовательности <<РРОРО>> ее выигрыш равен $2/5$. Какая стратегия максимизирует ожидаемый выигрыш Ани? Чему равен наибольший ожидаемый выигрыш?

На настоящее (февраль 2010) ни один человек на Земле не знает ответа на этот вопрос. Предложите свою стратегию. С помощью большого количества случайных экспериментов на компьютере будет определена лучшая из предложенных стратегий. Авторы лучшей стратегии получают +2 балла к итоговой оценке. 

Детали: срок представления стратегий - 01 марта 2010.
В силу невозможности смоделировать бесконечное количество подбрасываний их количество ограничено $10^6$. таким образом, если в ходе эксперимента стратегия не сказала <<Стоп>> после $10^6$ подбрасываний, то она принудительно останавливается.
Для определения победителя будет проведено $10^4$ подбрасываний.  }

\solution{
Search: $S_n/n$ problem, current state $>E(best)>$ }



Wald had a number of important results, including the famous theorem that the expected value of the sum of a random number of random variates is equal to the product of the expected value of a single variate times the expected number in the sum; as long as the stopping rule is independent of the value of the sum \par
Like a lot of theorems, Wald's Theorem is valuable when it doesn't apply. That is, people often assume it is true, it's a handy trick for solving certain kinds of problems. Wald gave rigorous conditions under which it is true. When you come across an application, it's a good idea to check the three conditions.

Here is a simple example. In the mindless children's card game "War" two players split a deck of card between them. At each turn, both players reveal their top card, the player with the higher card takes both and puts them at the bottom of her deck. If the cards are the same there is a "war," meaning each player deals three cards face down, then turns up the next card. The player with the higher card takes all ten cards. In case of a tie there is another war, with the winner getting 18 cards. And so on until one player runs out of cards.

Question: what is the expected number of concealed Aces (the high card) that will change hands in the first play of the game? This is an important parameter for analyzing the game.

You could figure out all the possible combinations, but that would take a while. Wald's theorem tells us we can compute the expected number of wars on the first play, $16*(17^{-2} + 17^{-3} + 17^{-4})$, and multiply by the expected number of concealed Aces from the losing player per war, 3/13. The answer is 0.0144. By the way, you cannot have more than three wars. In the unlikely event (1/83,521) that the players tie four times in a row, both of them lose the game and no cards are exchanged.

The three conditions of Wald's theorem are:

(1) The number of trials is a non-negative integer with finite expectation.

(2) The outcome of the trials are i.i.d. with finite expectation.

(3) The outcome of the trial is independent of whether or not it is included in the total. 

In the War example, condition 2 is violated. The probability that the first two cards match is 3/51 = 1/17. If they do, the probability that the next two compared cards match is 2/50 * 1/49 + 48/50 * 3/49 = 146/2,450 = 0.0596 instead of 0.0588. \par

----\par
Version $\#1$ (tboafo): A die is rolled once. If the outcome is 1, 2, or 3, one stops; otherwise (i.e., if it is 4, 5, or 6) one rolls the die a second time. What is the total expected value? [A: 5.25] \par

Version $\#2$ (Wilbur): A die keeps being rolled until the outcome is 1, 2, or 3. What is the total expected value? [A: 7]\par

Now, on to version $\#3$:\par

One has an arbitrarily large number of dice at our disposal (this is just a conceptual convenience; the problem can easily be formulated with one single die). The first die is rolled. If the outcome is 1, 2, or 3, one stops; otherwise, if it is 4, 5, or 6, a corresponding number of dice are rolled. This procedure continues for every rolled dice whose outcome is 4, 5, or 6. Let n denote the n-th round of rolls. What is the total expected value at the end of the n-th round of rolls?\par

Let's clarify with a couple of examples:\par

Example I:\par
1st round of rolls (only one die is rolled): 3 => game ends.\par

Example II:\par
1st round of rolls (only one die is rolled): 4 => 4 dice will be rolled next
2nd round of rolls: 3, 5, 2, 6 => 1st and 3rd dice end their lives, 2nd and 4th dice will lead to 5 and 6 dice to be rolled next, respectively
3rd round of rolls: 1, 2, 4, 4, 3; 5, 3, 6, 1, 6, 2 => ... and so forth ...\par
solution: \par
expected value for 1st round: 3.5

expected number of dice for 2nd round: (4+5+6)/6 = 2.5\par
expected value for 2nd round: 2.5 * 3.5\par

...\par

expected number of dice for n-th round: $2.5^(n-1)$\par
expected value for n-th round: $2.5^(n-1)*3.5$\par
total expected value at n-th round = $(2.5^n -1)/(1.5) * 3.5 = 7/3*(5/2^n -1)$ \par

---\par




\problem{ In a large city the phone book comes in 4 volumes.\par
In a phone booth those for volumes are stacked one on top of the other.\par
Every t minutes someone takes out one volume from the stack, looks up a number and puts it back on the top of the stack.\par
Suppose that the probability that any user should pick volume i is $P_i$. (sum $P_i = 1$).\par
Define the depth $d_i$ of volume i as the distance from the top of the stack ($d_i = 1$ if it is on top of thestack, $d_i=2$ if there is one volume on top of it and so on).\par
Find the long-term average depths $d_i$.\par
a) Assume $P_1=0.4$ $P_2=0.3$ $P_3=0.2$ $P_4=0.1$\par
b) Find the expressions for $d_i$ for general $P_i$}
\solution{
The key is that each pair of books is independent. Using the 0.4, 0.3, 0.2 and 0.1 probabilities: book 1 spends 3/7 of the time below book 2, 2/6 of the time below book 3 and 1/5 of the time below book 4. So it has, on average, 202/210 books on top of it.\par
If the probabilities are p1, p2, . . ., pn, book i's average depth will be the sum for j = 1 to n (not = i) pj/(pi+pj).}

\problem{ If I start with nothing and I play this game whereby I throw a dice as many times as I want. For each throw, if 1 appears I win \$1, 2 appears I win \$2 ....but if 6 appears I lose all my money. So when is the optimal stopping time and what is the expected winning? }
\solution{
strategy - easy, Expectation - computational expensive (?) }



\problem{
Let $X$ and $Y$ be independent uniform random variables between 0 and 1.\par
Let $P(N,M) := P(max(X,Y)>M |min(x,y)<N)$ for any N,M in [0,1].\par
a) Find $P(N,M)$ for $M=0$ (trival case)\par
b) Find $P(N,M)$ for $M=1$ (trival case)\par
c) Find $P(N,M)$ for $N=M=1/2$\par
d) find $N$ s.t. $P(N,N)=1/2$\par
e) find $N$ and $M=f(N)$ s.t. $P(N,M)=1/2$\par
f) find $P(N,M)$ }
\solution{}


\problem{ In Bertrand's paradox one is given a circle and asked: What is the probability P that a chord chosen at random is longer than the side of an inscribed equilateral triangle.}
\solution{
The solution is highly dependent on how one interprets the phrase "chord chosen at random".\par

Solution 1: Assign a uniform probability distribution to the angles of intersection of the chord on the circumference -- then, clearly P = 1/3\par

Solution 2: Assign a uniform probability distribution to the center of the chord over the area of the circle -- then one can show that P = 1/4\par

Solution 3: Assign a uniform probability distribution to the linear distance between centers of the chord and circle midpoint, then p = 1/2\par

Remarkably, in a 1973 paper The Well-Posed Problem E.T. Jaynes argued that this problem is well-posed after all if one requires certain physical symmetries of the probability distribution: rotational invariance, translational invariance and scale invariance. As Jaynes demonstrates, the 3rd solution is actually superior on these physical grounds.}



\problem{ You stand by the bank of a straight river. You then walk 1 km straight in any direction that keeps you dry, i.e. without crossing the river and stop at a point P. \par
a) What is the expected value of the distance to the river?\par
b) If at point P you walk 1 km in any direction, what is the probability that you will get back to the river? \par
Comment: here random means uniform angle }
\solution{
I agree w/silverside on (a) 2/pi \par
for (b, given a random direction theta in [0, pi] that we walked to get to P, the probability of reaching the river by walking in a random direction is Abs(pi - 2 theta ) / 2pi. Averaging from 0 to pi gives 1/4.}


\problem{ Выбираются три независимые равномерные на [0;1] величины. Найдите функцию плотности средней величины, минимальной, максимальной. }
\solution{}



\problem{ Consider the ratio x/y of two positive reals x and y that are picked at random from [0;1]. What is the probability that the first non-zero digit in this ratio is a 4? }
\solution{}



\problem{ (досочинять) There is another example. The premier b-schools in India (IIMs) conduct an admission test to select students. The test usually contains 200 multiple choice (four alternatives per question) questions (1 mark each) with a 1/4 marks penalty for every incorrect answer. The test duration is 2 hours. It is generally reagrded as a tough one and a score of more than 100 is considered decent.\par

Suppose a student randomly selects one of the alternatives. Regarding the test as a sequence of 200 independent trials we can figure out the probability of scoring more than 100 marks is of the order of $10^(-25)$. Fair enough. Suppose you have an "average" student, who gets 50 questions correct. He can also narrow down the four alternatives of each question to two alternatives. Now he tosses a coin (or by anyother random method ) to select the correct answer. To get a score of 100 or more, net of negative marking, he needs to get atleast 70 of the remaining 150 questions correct. The probability, for this case, turns out to be an astounding 81\%! Even for a "below average" guy who gets 30 or less correct and tosses a coin to get the rest of the questions, has a significant probability of scoring more than 100 (40-50\%).\par

Now given score cards, what kind of conditional probability that a score of more than 100 came from a below average guy (who tossed a coin for marking answers) can we expect? I guess it should be substantial. I haven't actually calculated it. But it should be substantial.\par

отсутствует доля средних и глупых }
\solution{}

\problem{ Assume two binary sequences of lenght N generated by fair coin flips (independent, identically distributed with equal probability for H or T). What is the expected length of the longest common contiguous subsequence that appears in both sequences?\par

E.g.:\par
S1: 000101111\par
S2: 101010101\par
longest common contiguous subsequence is: 0101}
\solution{
I ask that people who post problems make some effort to be precise. I assume "random" binary sequence means each digit is 0 or 1 with probability 0.5, independent of all other digits. I assume subsequence means contiguous subsequence as specified by the subsequent post.

A squence of length N has N - k + 1 subsequences of length k. The probability of two sequences of length k being identical, given the assumptions above, is $2^{-k}$. So the expected number of identical subsequences of length k is $(N - k + 1)^2*2^{-k}$.\par

For large N, this is approximately $(N^2)*(2^{-k})$. Using that approximation, the expected number of identical subsequences of length k or longer is $(N^2)*2^{(1 - k)}$. This equals 1 if 2*ln(N)+(1 - k)*ln(2) = 0, or k = 2*ln(N)/ln(2) + 1.\par

For large N and k near 2*ln(N)/ln(2) + 1, the probability of having an identical subsequence of length k+1, given that you have have an identical subsequence of length k, is near 0.75. 1/(1 - 0.75) = 3.\par

This gives an approximation for the expected value of the longest subsequence to be 2*ln(N)/ln(2) + 4, for large N.\par

For N = 3 there are 64 possible combinations. 2 of them, 000/111 and 111/000 have longest identical subsequence equal to zero. 20 of them have longest identical subsequence equal to one. 34 of them have longest identical subsequence equal to two. 8 of them, anything/the same thing, have longest identical subsequence equal to three. The expected value is (2*0 + 20*1 + 34*2 + 8*3)/64 = 112/64 as Merlin81 said. That's not close to the 7.17 asymptotic approximation for large N.\par

variation:

the first sequence is 00000...000 }

\problem{
Is there a probability distribution on ${\bf Z}_{>0}$ such that the
probability of selecting an integer divisible by $p$ is $\frac{1}{p}$... \par
a) for all prime $p$? \par
b) for all $p$? }
\solution{
a) I believe the answer to the first question is yes. Let $p_i$ be the $i$-th
prime. For convenience, set $p_0= 1$. Let $q_i$ be the product of the
first $i$ primes. (In particular, $q_0=1$.) Assign the number $q_i$ a
probability of $\frac{1}{p_i} - \frac{1}{p_{i+1}}$ . Every other number
is assigned a probability of zero. I believe this probability
distribution satisfies your property. \par
b) I believe the answer to the second question is no. We can't even make
the condition work for all squarefree numbers. Here is a sketch of the
argument.\par
Suppose the condition did hold for all squarefree numbers. We can then
easily show that the events of being divisible by 2, 3, 5, etc., are
mutually independent. In particular, the probability of being
nondivisible by 17, 19, 23, 29, 31, etc., is 0. (Here I am using the
fact that the sum of inverse primes is infinite.) It follows that the
probability of being 1 through 16 is 0. By changing the 17 to an
arbitrarily large prime, we get that the probability of each integer
is 0, which is a contradiction. \par
Here's a remaining question. Can we make the condition hold for all
primes and products of two distinct primes? \par
source: aops, t=187407 }

\problem{ A particle is bouncing randomly in a two-dimensional box.  How far does it
travel between bounces, on average? \par
Suppose the particle is initially at some random position in the box and is
traveling in a straight line in a random direction and rebounds normally
at the edges. }
\solution{
Let $\theta$ be the angle of the point's initial vector.  After traveling a
distance $r$, the point has moved $r\cdot cos(\theta)$ horizontally and $r\cdot sin(\theta)$ vertically, and thus has struck $r\cdot (sin(\theta)+cos(\theta))+O(1)$ walls.  Hence the average distance between walls will be $1/(sin(\theta)+cos(\theta))$.  We now average this over all angles $\theta$: \par
$2/pi \cdot int_{\theta=0}^{\pi/2} (1/(sin(\theta)+cos(\theta))) d\theta=	2\cdot\sqrt{2}\cdot ln(1+\sqrt{2})/\pi \approx 0.79$ \par
source: probability puzzles, archive }



\problem{
==> probability/flips/once.in.run.p <== \par
What are the odds that a run of one H or T (i.e., THT or HTH) will occur
in n flips of a fair coin? }
\solution{
==> probability/flips/once.in.run.s <== \par
References: \par
    John P. Robinson, Transition Count and Syndrome are Uncorrelated, IEEE
    Transactions on Information Theory, Jan 1988. \par

First we define a function or enumerator P(n,k) as the number of length
"n" sequences that generate "k" successes.  For example, \par

     P(4,1)= 4  (HHTH, HTHH, TTHT, and THTT are 4 possible length 4 sequences). \par

I derived two generating functions g(x) and h(x) in order to enumerate
P(n,k), they are compactly represented by the following matrix
polynomial. \par

$\left(
\begin{array}{c}
g(x) \\
h(x) \\
\end{array}\right)=
\left(\begin{array}{cc}
1 & 1 \\
1 & x \\
\end{array}\right)^{(n-3)}
\left(\begin{array}{c}
4 \\
2+2x \\ 
\end{array}\right)$

The above is expressed as matrix generating function.  It can be shown
that $P(n,k)$ is the coefficient of the $x^k$ in the polynomial
$(g(x)+h(x))$. \par

For example, if $n=4$ we get $(g(x)+h(x))$ from the matrix generating
function as $(10+4x+2x^2)$.  Clearly, P(4,1) (coefficient of x) is 4 and
$P(4,2)=2$ ( There are two such sequences THTH, and HTHT). \par

We can show that \par

   mean(k) = (n-2)/ 4 and $sd= \sqrt{5n-12}/4$ \par
We need to generate "n" samples. This can be done by using sequences of length
(n+2).  Then our new statistics would be \par
   mean = n/4 \par
   $sd = \sqrt{5n-2}/4$ \par
Similar approach can be followed for higher dimensional cases. }




\problem{
==> probability/lights.p <== \par
Waldo and Basil are exactly m blocks west and n blocks north from
Central Park, and always go with the green light until they run out of
options.  Assuming that the probability of the light being green is 1/2
in each direction, that if the light is green in one direction it is
red in the other, and that the lights are not synchronized, find the
expected number of red lights that Waldo and Basil will encounter. }
\solution{
==> probability/lights.s <== \par
Let E(m,n) be this number, and let (x)C(y) = x!/(y! (x-y)!).  A model
for this problem is the following nxm grid:

%     ^         B---+---+---+ ... +---+---+---+ (m,0)
%     |         |   |   |   |     |   |   |   |
%     N         +---+---+---+ ... +---+---+---+ (m,1)
%<--W + E-->    :   :   :   :     :   :   :   :
%     S         +---+---+---+ ... +---+---+---+ (m,n-1)
%     |         |   |   |   |     |   |   |   |
%     v         +---+---+---+ ... +---+---+---E (m,n)

where each + represents a traffic light.  We can consider each
traffic light to be a direction pointer, with an equal chance of
pointing either east or south.

IMHO, the best way to approach this problem is to ask:  what is the
probability that edge-light (x,y) will be the first red edge-light
that the pedestrian encounters?  This is easy to answer; since the
only way to reach (x,y) is by going south x times and east y times,
in any order, we see that there are (x+y)C(x) possible paths from
(0,0) to (x,y).  Since each of these has probability $(1/2)^(x+y+1)$
of occuring, we see that the the probability we are looking for is
$(1/2)^(x+y+1)*(x+y)C(x)$.  Multiplying this by the expected number
of red lights that will be encountered from that point, (n-k+1)/2,
we see that

$E(m,n) = \sum_{k=0}^{m-1}  ( 1/2 )^(n+k+1) * (n+k)C(n) * (m-k+1)/2$
$+\sum_{k=0}^{n-1}( 1/2 )^(m+k+1) * (m+k)C(m) * (n-k+1)/2$


Are we done?  No!  Putting on our Captain Clever Cap, we define \par

$f(m,n) =\sum_{k=0}^{n-1}( 1/2 )^k * (m+k)C(m) * k $

and
$g(m,n) =\sum_{k=0}^{n-1} ( 1/2 )^k * (m+k)C(m).$

Now, we know that

$f(m,n)/2 =\sum_{k=1}^{n}   ( 1/2 )^k * (m+k-1)C(m) * (k-1) $

and since f(m,n)/2 = f(m,n) - f(m,n)/2, we get that

$f(m,n)/2 =\sum_{k=0}^{n-1} ( 1/2 )^k * ( (m+k)C(m) * k - (m+k-1)C(m) * (k-1) )$

$- (1/2)^n * (m+n-1)C(m) * (n-1)$ \par

 $=\sum_{k=0}^{n-2}    ( 1/2 )^(k+1) * (m+k)C(m) * (m+1)$
 
$- (1/2)^n * (m+n-1)C(m) * (n-1)$ \par

$= (m+1)/2 * (g(m,n) - (1/2)^(n-1)*(m+n-1)C(m)) - (1/2)^n*(m+n-1)C(m)*(n-1)$

therefore \par

$f(m,n) = (m+1) * g(m,n) - (n+m) * (1/2)^(n-1) * (m+n-1)C(m)$ .


Now, $E(m,n) = (n+1) * (1/2)^(m+2) * g(m,n) - (1/2)^(m+2) * f(m,n)+ (m+1) * (1/2)^(n+2) * g(n,m) - (1/2)^(n+2) * f(n,m)$ \par

$= (m+n) * (1/2)^(n+m+1) * (m+n)C(m) + (m-n) * (1/2)^(n+2) * g(n,m)+ (n-m) * (1/2)^(m+2) * g(m,n) $.


Setting m=n in this formula, we see that

           $E(n,n) = n * (1/2)^(2n) * (2n)C(n)$,

and applying Stirling's theorem we get the beautiful asymptotic formula

                  $E(n,n) ~ sqrt(n/pi)$.}


\problem{
==> probability/random.walk.p <==
Waldo has lost his car keys!  He's not using a very efficient search;
in fact, he's doing a random walk.  He starts at 0, and moves 1 unit
to the left or right, with equal probability.  On the next step, he
moves 2 units to the left or right, again with equal probability.  For
subsequent turns he follows the pattern 1, 2, 1, etc.

His keys, in truth, were right under his nose at point 0.  Assuming
that he'll spot them the next time he sees them, what is the
probability that poor Waldo will eventually return to 0?}
\solution{

==> probability/random.walk.s <==
I can show the probability that Waldo returns to 0 is 1.  Waldo's
wanderings map to an integer grid in the plane as follows.  Let
$(X_{t},Y_{t})$ be the cumulative sums of the length 1 and length 2 steps
respectively taken by Waldo through time t.  By looking only at even t,
we get the ordinary random walk in the plane, which returns to the
origin (0,0) with probability 1.  In fact, landing at (2n, n) for any n
will land Waldo on top of his keys too.  There's no need to look at odd
t. \par
Similar considerations apply for step sizes of arbitrary (fixed) size. }

\problem{
==> probability/transitivity.p <==
Can you number dice so that die A beats die B beats die C beats die A?
What is the largest probability p with which each event can occur? }
\solution{
==> probability/transitivity.s <==
Yes.  The actual values on the dice faces don't matter, only their
ordering.  WLOG we may assume that no two faces of the same or
different dice are equal.  We can assume "generalised dice", where the
faces need not be equally probable.  These can be approximated by dice
with equi-probable faces by having enough faces and marking some of
them the same.

Take the case of three dice, called A, B, and C.  Picture the different
values on the faces of the A die.  Suppose there are three:

            A       A       A

The values on the B die must lie in between those of the A die:

        B   A   B   A   B   A   B

With three different A values, we need only four different B values.

Similarly, the C values must lie in between these:

      C B C A C B C A C B C A C B C
      
Assume we want A to beat B, B to beat C, and C to beat A.  Then the above
scheme for the ordering of values can be simplified to:

      B C A B C A B C A B C

since for example, the first C in the previous arrangement can be moved
to the second with the effect that the probability that B beats C is
increased, and the probabilities that C beats A or A beats B are
unchanged.  Similarly for the other omitted faces.

In general we obtain for n dice A...Z the arrangement

    B ... Z A B ... Z ...... A B ... Z

where there are k complete cycles of B..ZA followed by B...Z.  k must be
at least 1.

CONJECTURE:  The optimum can be obtained for k=1.

So the arrangement of face values is B ... Z A B ... Z.  For three dice
it is BCABC.  Thus one die has just one face, all the other dice have two
(with in general different probabilities).

CONJECTURE:  At the optimum, the probabilities that each die beats the
next can be equal.

Now put probabilities into the BCABC arrangement:

    B  C  A  B  C
    x  y  1  x' y'

Clearly x+x' = y+y' = 1.

Prob. that A beats B = x'
           B beats C = x + x'y'
           C beats A = y

Therefore x' = y = x + x'y'

Solving for these gives x = y' = 1-y, x' = y = (-1 + sqrt(5))/2 = prob.
of each die beating the next = 0.618...

For four dice one obtains the probabilities:

    B  C  D  A  B  C  D
    x  y  z  1  x' y' z'

A beats B:  x'
B beats C:  x + x'y'
C beats D:  y + y'z'
D beats A:  z

CONJECTURE: for any number of dice, at the optimum, the sequence of
probabilities abc...z1a'b'c...z' is palindromic.

We thus have the equalities:

    x+x' = 1
    y+y' = 1
    z+z' = 1
    x' = z = x + x'y' = x + x'y'
    y = y' (hence both = 1/2)

Solving this gives x = 1/3, z = 2/3 = prob. of each die beating the next.
 Since all the numbers are rational, the limit is attainable with
finitely many equiprobable faces.  E.g. A has one face, marked 0.  C has
two faces, marked 2 and -2.  B has three faces, marked 3, -1, -1.  D has
three faces, marked 1, 1, -3.  Or all four dice can be given six faces,
marked with numbers in the range 0 to 6.

Finding the solution for 5, 6, or n dice is left as an exercise.

Source: Richard Kennaway, jrk\@sys.uea.ac.uk    \par


Martin Gardner (of course!) wrote about notransitive dice, see the Oct '74
issue of Scientific American, or his book "Wheels, Life and Other Mathematical
Amusements", ISBN 0-7167-1588-0 or ISBN 0-7167-1589-9 (paperback).

In the book, Gardner cites Bradley Efron of Stanford U. as stating that
the maximum number for three dice is approx .618, requiring dice with more
than six sides.  He also mentions that .75 is the limit approached as the
number of dice increases.  The book shows three sets of 6-sided dice, where
each set has 2/3 as the advantage probability. }






\section{Выборочное среднее, общая интуиция}
% mean_value

\problem{ Среднее и медиана. \par
Имеется пять чисел: $x$, $4$, $5$, $7$, $9$. При каком значении
$x$ медиана будет равна среднему? }
\solution{}

\problem{
Измерен рост 100 человек. Средний рост оказался равным 160
см. Медиана оказалась равной 155 см. Машин рост в 163 см был
ошибочно внесен как 173 см. Как изменятся медиана и среднее после
исправления ошибки? } 
\solution{} 

\problem{ This is a famous WWII story about the statistician Abraham Wald. He was asked by the Air Force to determine where to reinforce the armor on bombers. If you put armor everywhere, the plane is too heavy to take off. But the Air Force maintained detailed records of the location of every hit on every plane returning from missions in Germany. Wald looked at the tabulations and performed a simple transformation before using the distribution to place the armor. What transformation? \par
Wald inverted the distribution. The places where there were no holes were not places the anti-aircraft guns always missed, they were places where a hit was fatal. The places with lots of holes were places where hits didn't matter much. } 
\solution{} 

\problem{ 
The New York Times' weekly science supplement called <<Science Times>> on August 22, 1989 stated: \par
<<... From June 4 through November 4, 1984, for instance, 132 such victims were admitted to the Animal Medical Center.... Most of the cats landed on concrete. Most survived...\par 
... [Veterinarians] recorded the distance of the fall for 129 of the 132 cats. The falls ranged from 2 to 32 stories... 17 of the cats were put to sleep by their owners, in most cases not because of life-threatening injuries but because the owners said they could not afford medical treatment. Of the remaining 115, 8 died from shock and chest injuries... \par
... Even more surprising, the longer the fall, the greater the chance of survival. Only one of 22 cats that plunged from above 7 stories died, and there was only one fracture among the 13 that fell more than 9 stories. The cat that fell 32 stories on concrete, Sabrina, suffered [only] a mild lung puncture and a chipped tooth...>> \par
Is it true, that there is positive relation between length of the fall and chance of survival? } 
\solution{} 

\problem{
Возможно ли, что риск катастрофы в расчете на 1 час пути
больше для самолета, чем для автомобиля, а в расчете на 1 километр
пути -
наоборот? } 
\solution{} 

\problem{
Деканат утверждает, что если студента N перевести из группы
А в группу В, то средний рейтинг каждой группы возрастет. Возможно
ли
это? } 
\solution{} 

\problem{
Из класса А в класс Б перевели группу человек, затем из
класса Б в класс В перевели группу человек. После этой операции
рейтинг каждого класса возрос по сравнению с первоначальным. Затем
другие группы переводили в обратном направлении (из В в Б, потом
из Б в А). При этом рейтинг каждого класса снова вырос. Возможно
ли это?} 
\solution{} 

\problem{
Два лекарства испытывали на мужчинах и женщинах. Каждый
человек принимал только одно лекарство. Общий процент людей,
почувствовавших улучшение, больше среди принимавших лекарство А.
Процент мужчин, почувствовавших улучшение, больше среди принимавших лекарство В. Процент женщин, почувствовавших улучшение, больше среди принимавших лекарство В. Возможно ли это? } 
\solution{} 

\problem{
Пусть $X\in L^{1}$ и ее функция распределения $F(t)$
непрерывна. Величина $E|X-c|$ достигает своего минимума при
некотором $c_{0}$.
a) Найдите $F(c_{0})$. \par
b) Проинтерпретируйте \par
hint: удобно воспользоваться геометрической интерпретацией мат.
ожидания } \solution{} 

\problem{
\begin{figure}[h]
    \includegraphics{problems.1}
\end{figure}
Расположите по порядку: среднее, мода, медиана } 
\solution{} 

\problem{ 
На курсах 3 группы по 10 человек, 2 группы по 20 человек и
1
группа по 40 человек. \par
а) Каков средний размер группы, для которой читает лекции наугад
выбранный профессор? \par
б) Каков средний размер
группы, в которой учится наугад выбранный студент? \par
в) На других курсах Вы опросили $n$ человек и спросили у каждого размер группы, постройте несмещенную оценку для среднего размера группы. } 
\solution{} 

\problem{ $[$Mosteller$]$ Странное метро (шутка) \par
Мэрвин кончает работу в случайное время между 15 и 17 часами. Его
мать и его невеста живут в противоположных частях города. Мэрвин
садится в первый подошедший к платформе поезд, идущий в любом
направлении, и обедает с той из дам, к которой приедет. Мать
Мэрвина жалуется на то, что он редко у нее бывает, но юноша
утверждает, что его шансы обедать с ней и с невестой равны. Мэрвин
обедал с матерью дважды в течение 20 рабочих дней. \par
Объясните это явление. } 
\solution{} 

\problem{
Два эскалатора находятся рядом так, что человек может
перешагнуть с одного на другой без потери и набора высоты. Можно
ли сделать так, чтобы: коробка, стоящая на левом эскалаторе,
спускалась бы вниз; коробка, стоящая на правом эскалаторе в
среднем спускалась бы вниз; человек, переходящий с одного
эскалатора на другой без
изменения высоты, в среднем поднимался бы вверх? \par
Движение эскалаторов может быть не равномерным.} \solution{} 


\problem{
Пусть $X$ и $Y$ - две независимые случайные величины. Верно ли, что $Me(X+Y)=Me(X)+Me(Y)$, где $Me$ - это медиана.}
\solution{нет}


\problem{Приведите примеры, когда $Med(X+Y)=Med(X)+Med(Y)$ и $Med(X+Y)\neq Med(X)+Med(Y)$. $Med$ - медиана}
\solution{ два независимых симметричных распределения; практически любая сумма несимметричных распределений, например, два независимых с $p(x)=2-2x$ на $[0;1]$.}

\problem{Какой закон распределения лучше всего подходит для... (тут несколько реальных примеров)}
\solution{}


\section{Свойства оценок}
% estimators
\problem{  
Пусть  $X$  равномерна на  $\left[0;a\right]$. Придумайте
$Y=\alpha +\beta X$  так, чтобы  $Y$  была несмещенной оценкой
$a$.} 
\solution{} 

\problem{  
Пусть  $X_{i} $  - независимы и одинаково распределены.
При каком значении параметра  $\beta $

а)   $2X_{1} -5X_{2} +\beta X_{3} $  будет несмещенной оценкой для
$E\left(X_{i} \right)$?

б)   $\beta \left(X_{1} +X_{2} -2X_{3} \right)^{2} $  будет
несмещенной оценкой для  $Var\left(X_{i} \right)$?} 
\solution{} 

\problem{  
Пусть  $X_{1} $  и  $X_{2} $  независимы и равномерны на
$\left[0;a\right]$. При каком  $\beta $  оценка  $Y=\beta \cdot
\min \left\{X_{1},X_{2} \right\}$  для параметра  $a$  будет
несмещенной?} 
\solution{} 

\problem{ Пусть случайная величина  $X$  распределена равномерно на отрезке
$\left[0;a\right]$, где  $a>3$. Исследователь хочет оценить
параметр  $\theta =P\left(X<3\right)$. Рассмотрим следующую оценку
$\hat{\theta }=\left\{\begin{array}{l} {1,\; X<3}
\par {0,\; X\ge 3}
\end{array}\right. $. \par
а) Верно ли, что оценка $\hat{\theta}$ является несмещенной? \par
б) Найдите $E\left(\left(\hat{\theta }-\theta \right)^{2}
\right)$. } 
\solution{} 

\problem{ 
 Пусть  $X$  равномерна на  $\left[3a-2;3a+7\right]$. При
каких  $\alpha $  и  $\beta $  оценка  $Y=\alpha +\beta X$
неизвестного параметра  $a$  будет несмещенной?} 
\solution{} 

\problem{  
Закон распределения с.в.  $X$  имеет вид

а)  $\begin{array}{|c|ccc|}  \hline {x_{i} } & {0} & {1} & {a} \\
\hline {P\left(X=x_{i} \right)} & {{1\mathord{\left/ {\vphantom {1
4}} \right. \kern-\nulldelimiterspace} 4} } & {{1\mathord{\left/
{\vphantom {1 4}} \right. \kern-\nulldelimiterspace} 4} } &
{{2\mathord{\left/ {\vphantom {2 4}} \right.
\kern-\nulldelimiterspace} 4} } \\  \hline  \end{array}$ ; б)
$\begin{array}{|c|ccc|}  \hline {x_{i} } & {0} & {1} & {2} \\
\hline {P\left(X=x_{i} \right)} & {{1\mathord{\left/ {\vphantom {1
4}} \right. \kern-\nulldelimiterspace} 4} } & {a} &
{\left({3\mathord{\left/ {\vphantom {3 4}} \right.
\kern-\nulldelimiterspace} 4} -a\right)} \\  \hline  \end{array}$

Постройте несмещенную оценку вида  $Y=\alpha +\beta X$  для
неизвестного параметра  $a$} 
\solution{} 

\problem{  [т] \par
Время горения лампочки – экспоненциальная с.в. с ожиданием равным
$\theta $. Вася включил одновременно 20 лампочек. С.в.  $X$
обозначает время самого первого перегорания.
\par
а) Найдите $E(X)$ \par
б) Как с помощью  $X$  построить несмещенную оценку для  $\theta$? } 
\solution{} 

\problem{  
$X_{i} \sim iid$, какая из приведенных оценок для $E\left(X_{i}
\right)$  является несмещенной? наиболее эффективной? наиболее
эффективной среди несмещенных?

а)  $X_{1} $ ; б)  $X_{1} +3X_{2} -2X_{3} $ ; в)  ${\left(X_{1}
+X_{2} \right)\mathord{\left/ {\vphantom {\left(X_{1} +X_{2}
\right) 2}} \right. \kern-\nulldelimiterspace} 2} $ ; г)
${\left(X_{1} +X_{2} +X_{3} \right)\mathord{\left/ {\vphantom
{\left(X_{1} +X_{2} +X_{3} \right) 3}} \right.
\kern-\nulldelimiterspace} 3} $ ; д)  $\left(X_{1} +...+X_{20}
\right)/21$ ; е)  $X_{1} -2X_{2}$} 
\solution{} 

\problem{ (3 золотых слитка - задачу можно клонировать!!!) \par
Весы имеют ошибку со средним ноль и дисперсией $\sigma^2$.\par
Имеется три золотых монеты и результаты следующих 7 взвешиваний: \par
Каждую монету взвешивали по отдельности, все три монеты вместе, монеты взвешивали парами (1 и 2; 1 и 3; 2 и 3). \par
Вася предлагает сложить .... вычесть. \par
Будет ли оценка веса первого слитка смещенной? \par
Будет ли она лучше чем усреднить семь взвешиваний по 1-му слитку } 
\solution{} 

\problem{ Есть два золотых слитка, разных по весу. Сначала взвесили первый слиток и получили результат $X$. Затем взвесили второй слиток и получили результат $Y$. Затем взвесили оба слитка и получили результат $Z$. Допустим, что ошибка каждого взвешивания - это случайная величина с нулевым средним и дисперсией $\sigma^{2}$. \par
а) Придумайте наилучшую оценку веса первого слитка. \par
б) Сравните придуманную Вами оценку с оценкой, получаемой путем усреднения двух взвешиваний первого слитка. }
\solution{ a) Пусть истинные веса слитков равны $x$, $y$ и $z$. \par
Назовем оценку буквой $\hat{x}$ \par
$\hat{x}=aX+bY+cZ$ \par
Несмещенность: $E(\hat{x})=aE(X)+bE(Y)+cE(Z)=ax+by+c(x+y)=x$ \par
$a+c=1$, $b+c=0$ \par
$\hat{x}=(1-c)X+(-c)Y+cZ$ \par
Эффективность: \par
$Var(\hat{x})=((1-c)^{2}+c^{2}+c^{2})\cdot \sigma^{2}=(3c^{2}-2c+1)\sigma^{2}$ \par
Чтобы минимизировать дисперсию нужно выбрать $c=1/3$ \par
Т.е. $\hat{x}=\frac{2}{3}X-\frac{1}{3}Y+\frac{1}{3}Z$ \par
б) $Var(\hat{x})=\frac{2}{3}\sigma^{2}$ \par
$Var\left(\frac{X_{1}+X_{2}}{2}\right)=\frac{1}{2}\sigma^{2}$ \par
Усреднение двух взвешиваний первого слитка лучше. } 



\problem{ Задача по эконометрике? \par
Given a weighing device and two balls.
Weigh the first ball using the weighing device, we get a value X;
weigh the second ball using the weighing device, we get a value Y;
weigh the two balls together, we get a value Z.
Suppose that each time the error in the weights is an iid random variable,
for example, $X=a+R_a$, $Y=b+R_b$, $Z=a+b+R_c$, where $R_a$, $R_b$, $R_c$ are iid.
How do you get the best estimation of the first ball's the real weight (i.e., a).
We don't know the distribution of the R's. \par
best - unbiased, min var } 
\solution{} 

\problem{  
$X_{i} \sim iid$, какая из приведенных оценок для $Var\left(X_{i}
\right)$  является несмещенной

а)  $X_{1}^{2} -X_{1} X_{2} $ ; б)  $\frac{\sum
_{i=1}^{n}\left(X_{i} -\bar{X}\right)^{2}  }{n} $ ; в) $\frac{\sum
_{i=1}^{n}\left(X_{i} -\bar{X}\right)^{2}  }{n-1} $ ; г)
$\frac{1}{2} \left(X_{1} -X_{2} \right)^{2} $ ; д)  $X_{1} -2X_{2}
$ ; е)  $X_{1} X_{2}$} 

\solution{} 

\problem{
Случайные величины $X_{i}$ распределены равномерно на отрезке
$[0;a]$, известно, что $a>10$. Исследователь хочет оценить
параметр $\theta=\frac{1}{P(X_{i}<5)}$. \par
а) Используя $\bar{X_{n}}$ постройте несмещенную оценку
$\hat{\theta}$ для $\theta$ \par
б) Найдите дисперсию построенной оценки \par
в) Является ли построенная оценка состоятельной? } 

\solution{} 

\problem{
Пусть $X_{1}$, ... $X_{6}$ независимы и равномерно распределены на $[0;k]$, где $k\in [1;2]$ - неизвестный параметр. \par
а) Постройте несмещенную оценку для $k$ вида $\hat{k}=c\cdot \bar{X}$ \par
б) Постройте несмещенную оценку для $k$ вида $\hat{k}=c\cdot \min\{X_{1},...,X_{6}\}$ \par
в) Постройте несмещенную оценку для $k$ вида $\hat{k}=c\cdot \max\{X_{1},...,X_{6}\}$ \par
г) Какая из них является наиболее эффективной? } 

\solution{} 

\problem{
Пусть $X_{i}$ независимы, одинаково распределены с функцией плотности 
$p(t)=
\left\{
\begin{array}{c}
2t/k^{2}, t\in[0;k] \par
0, $ иначе$ \par
\end{array}
\right.$ \par
По имеющейся выборке $X_{1}$, $X_{2}$, ... $X_{n}$: \par
а) Оцените медиану этого распределение методом максимального правдоподобия \par
б) Оцените медиану этого распределения методом моментов, используя $E(X_{i})$ }
\solution{
Медиана равна $m=\frac{k}{\sqrt{2}}$ \par
$m_{ML}=\frac{\max\{X_{1},...,X_{n}\}}{\sqrt{2}}$ \par
Source: Suhov, 5.20 (i) }

\problem{
Let $X_{i}$ - iid with pdf given by: \par
$p(t)=\left\{\begin{array}{l}
\frac{e^{-t/\lambda}}{\mu+\lambda}, t\ge 0 \par
\frac{e^{t/\mu}}{\mu+\lambda}, t<0 \par
\end{array}\right.$ \par
You are given a sample of $X_{1}$, ... $X_{n}$ \par
a) Find ML estimators for $\mu$ and $\lambda$ \par
Hint: You may found useful the following notation: \par
$S^{+}$ - sum of all positive $X_{i}$ \par
$S^{-}$ - sum of all negative $X_{i}$ \par
b) Let $n=1$, is the estimator of $\lambda$ is unbiased? }
\solution{
a) $\hat{\lambda}=\frac{S^{+}+\sqrt{-S^{+}S^{-}}}{n}$ \par
$\hat{\mu}=\frac{-S^{-}+\sqrt{-S^{+}S^{-}}}{n}$ \par
b) only if $\mu=0$ \par
Source: Suhov, 5.27 (ii) } 

\problem{
Пусть $X_{1}$, $X_{2}$, $X_{3}$ - независимы, одинаково распределены с плотностью \par
$p(t)=\left\{\begin{array}{l}
e^{k-t}, t\ge k \par
0, t<k \par
\end{array}\right.$ \par
Являются ли следующие оценки для $k$ несмещенными? \par
Какая из оценок будет наиболее эффективной? }
\solution{ а) $\hat{k}=X_{3}-1$ \par
b) $\hat{k}=\min\{X_{1},X_{2},X_{3}\}-\frac{1}{3}$ \par
c) $\hat{k}=\frac{\bar{X}}{3}-1$ \par
Solution: \par
Все несмещенные, b - наиболее эффективная \par
Source: Suhov, 5.31 } 

\problem{
Пусть $X_{i}$ - iid, $U[-b;b]$. Имеется выборка из 2-х наблюдений. Вася строит оценку для $b$ по формуле $\hat{b}=c\cdot (|X_{1}|+|X_{2}|)$. \par
a) При каком $c$ оценка будет несмещенной? \par
б) При каком $c$ оценка будет минимизировать средне-квадратичную ошибку, $MSE=E((\hat{b}-b)^{2})$? \par
в) Решите задачу, если имеется $n$ наблюдений, и оценка строится по формуле $\hat{b}=c\sum_{i=1}^{n}|X_{i}|$ \par
Source: экзамен миэф, 2008, Пересецкий } 
\solution{} 

\problem{
Пусть величины $X_{1}$, ..., $X_{n}$ равномерны на $[k;k+1]$ и независимы. Есть две оценки параметра $k$: $\hat{k}_{1}=\bar{X}-\frac{1}{2}$ и $\hat{k}_{2}=max\{X_{i}\}-\frac{n}{n+1}$ \par
а) Являются ли оценки несмещенными? \par
б) У какой оценки ниже дисперсия (при большой выборке)? \par
в) Являются ли оценки состоятельными? }
\solution{
a) обе несмещенные \par
б) $Var(\hat{k}_{1})=\frac{1}{12n}<\frac{n}{n+2}-\left(\frac{n}{n+1}\right)^{2}=Var(\hat{k}_{2})$ \par
see \ref{max uniform} \par
в) да, обе }

\problem{
Пусть $X_{i}$ - независимы и имеют функцию плотности $p(t)=e^{a-t}$ при $t>a$, где $a$ - неизвестный параметр. В качестве оценки неизвестного $a$ используется $\hat{a}_{n}=\min\{X_{1},X_{2},...,X_{n}\}$. \par
а) Является ли предлагаемая оценка состоятельной? \par
б) Является ли предлагаемая оценка несмещенной? }
\solution{
Заметим, что $\hat{a}_{n}\geq a$. \par
$P(|\hat{a}_{n}-a|>\varepsilon)=P(\hat{a}_{n}-a>\varepsilon)=P(\hat{a}_{n}>a+\varepsilon)=P(\min\{X_{1},X_{2},...,X_{n}\}>a+\varepsilon)= \\
=P(X_{1}>a+\varepsilon \cap X_{2}> a+\varepsilon\cap ...)=
P(X_{1}>a+\varepsilon)\cdot P(X_{2}>a+\varepsilon)\cdot ...=
\left(\int_{a+\varepsilon}^{\infty}e^{a-t}dt\right)^{n}=\left(e^{-\varepsilon}\right)^{n}=e^{-n\varepsilon}$ \\
$\lim_{n\to\infty} e^{-n\varepsilon} =0$ \par
б) нет, не является ни при каких $n$, хотя смещение с ростом $n$ убывает } 

\problem{
Пусть $X_{i}$ - независимы и распределены равномерно на $[a-1;a]$, где $a$ - неизвестный параметр. В качестве оценки неизвестного $a$ используется $\hat{a}_{n}=\max\{X_{1},X_{2},...,X_{n}\}$. \par
а) Является ли предлагаемая оценка состоятельной? \par
б) Является ли предлагаемая оценка несмещенной? }
\solution{
Заметим, что $\hat{a}_{n}\leq a$. \par
$P(|\hat{a}_{n}-a|>\varepsilon)=P(-(\hat{a}_{n}-a)>\varepsilon)=P(\hat{a}_{n}<a-\varepsilon)=P(\max\{X_{1},X_{2},...,X_{n}\}<a-\varepsilon)= \\
=P(X_{1}<a-\varepsilon \cap X_{2}< a-\varepsilon\cap ...)=
P(X_{1}<a-\varepsilon)\cdot P(X_{2}<a-\varepsilon)\cdot ...=(1-\varepsilon)^{n}$ \\
$\lim_{n\to\infty} (1-\varepsilon)^{n} =0$ \par
б) нет, не является ни при каких $n$, хотя смещение с ростом $n$ убывает }

\problem{
Пусть $X_{1}$,... $X_{n}$ независимы и имеют функцию распределения $F(t)$. Обозначим $\hat{F}(t)$ - долю величин, оказавшихся не выше $t$. \par
%а) Найдите $E(\hat{F}(t))$, $Var(\hat{F}(t))$, $Cov(\hat{F}(t),\hat{F}(s))$ \par
Является ли оценка $\hat{F}(t)$ состоятельной? \par
% задача скопирована в РВС, там вопрос а)
} 
\solution{} 
\problem{
Пусть $\hat{a}_{n}$ - состоятельная и несмещенная оценка для неизвестного параметра $a$. Пусть $b=exp(a)$ и $\hat{b}_{n}=exp(\hat{a}_{n})$. \par
а) Будет ли $\hat{b}_{n}$ несмещенной оценкой для $b$? \par
б) Будет ли $\hat{b}_{n}$ состоятельной оценкой для $b$? } 
\solution{} 


\problem{ Васе нужно узнать вес двух слитков. У Васи есть очень интересные электронные весы с двумя чашами. Они показывают разницу в весе между левой и правой чашой. Например, вес одного слитка можно узнать, положив его на правую чашу, а на левую не положив ничего. Результат взвешивания является случайной величиной со средним равным истинному весу предмета и дисперсией $\sigma^{2}$, не зависящей от веса предмета. Васе можно воспользоваться чудо-весами всего два раза. Какая стратегия лучше?

а) Взвесить каждый слиток по отдельности

б) Сначала взвесить два слитка вместе (на одной чаше весов), затем оценить разность их веса и путем простых арифметических операций оценить вес каждого слитка}
\solution{как ни странно, б лучше}




\problem{Творческая задача по матстатистике.

Исследователь знает, что экономические переменные $X$ и $Y$ имеют совместное нормальное распределение с  нулевыми матожиданиями. Он хотел бы оценить коэффициент корреляции между $X$ и $Y$, но, к сожалению, выборка самих $X$ и $Y$ ему недоступна. Все, что у него есть --- это результаты опроса, в котором каждый респондент ответил на вопросы <<Ваш $X>0$?>>, <<Ваш $Y>0?$>>. Предложите состоятельную оценку для $corr(X,Y)$, которую можно рассчитать только на основе данных подобного опроса.

Источник: Алексей Суздальцев}

\solution{$\widehat{r}=\sin\left(\frac{\pi}{2}\widehat{r_0}\right)$, где $\widehat{r_0}$ --- выборочный коэффициент корреляции между имеющимися дамми. На основе наблюдений по дамми можно оценить вероятности типа $P(X>0\cap Y>0)$, а она выражется через арксинус корреляции. (дать ссылку на задачу...)}





\section{MM, ML, BA}
% mmmlba


\problem{ simple MM, ML и Bayesian \\
Допустим, что закон распределения $X_{n}$ имеет вид: \\
\begin{tabular}{|c|c|c|c|}
  \hline
  X & -1 & 0 & 2 \\
  \hline
  Prob & $\theta$ & $2\theta-0.2$ & $1.2-3\theta$ \\
  \hline
\end{tabular} \par
Имеется выборка: $X_{1}=0$, $X_{2}=2$. \par
a) Найдите оценки $\hat{\theta}_{ML}$ и $\hat{\theta}_{MM}$ \par
b) Первоначально ничего о $\theta$ не было известно и поэтому
предполагалось, что $\theta$ распределена равномерно на
$[0.1;0.4]$. Как выглядит условное распределение $\theta$? } 
\solution{} 

\problem{ 
Известно, что  $X_{i} $  распределены одинаково и независимо. С
помощью ML оцените значение неизвестного параметра $\theta $, если
функция плотности  $X_{i} $  имеет вид:

а)  $\theta y^{\theta -1} $  при  $y\in \left[0;1\right]$ ; б)
$\frac{2y}{\theta ^{2} } $  при  $y\in \left[0;\theta \right]$ ;
в)  $\frac{\theta e^{-\frac{\theta ^{2} }{2y} } }{\sqrt{2\pi y^{3}
} } $  при  $y\in \left[0;+\infty \right)$ ; г)  $\frac{\theta
\left(\ln ^{\theta -1} y\right)}{y} $  при  $y\in
\left[1;e\right]$

д)  $\frac{e^{-\left|y\right|} }{2\left(1-e^{-\theta } \right)} $
при  $y\in \left[-\theta ;\theta \right]$. В пунктах а), б) и д)
постройте также ММ оценку.} 
\solution{} 

\problem{  
Постройте MM и ML оценки параметра  $\lambda$ экспоненциального
распределения.\par

} \solution{} \problem{  
Постройте MM и ML оценки параметра  $\mu$  нормального
распределения при известном  $\sigma^{2} $.\par

} \solution{} \problem{  
Постройте MM и ML оценки параметра  $\sigma ^{2} $ нормального
распределения при известном  $\mu$.\par

} \solution{} \problem{ 
Пусть $X_{1}$, $X_{2}$,..., $X_{n}$ независимы и их функции
плотности имеет вид: \\
$ f(x)=
\left\{%
\begin{array}{ll}
    (k+1)x^{k}, & x \in [0;1]; \\
    0, & x \notin [0;1]. \\
\end{array}%
\right.     $ \par
Найдите оценки параметра $k$: \par
а) Методом максимального правдоподобия \par
б) Методом моментов \par

} \solution{} \problem{ 
 Пусть $X_{1}$, $X_{2}$,..., $X_{n}$ независимы и
равномерно
распределены на отрезке $[0;\theta]$, $\theta>1$ \par
а) Постройте $\hat{\theta}$ методом максимального правдоподобия \par
б) Постройте $\hat{\theta}$ методом моментов \par
в) Как изменятся ответы на <<a>> и <<б>>, если исследователь не
знает значений самих $X_{i}$, а знает только количество $X_{i}$
оказавшихся больше единицы? \par


} \solution{} \problem{ Про зайцев \par
В темно-синем лесу, где трепещут осины, отловлено $100$ зайцев.
Каждому из них на левое ухо завязали бант из красной ленточки и
отпустили. Через неделю будет снова отловлено $100$ зайцев. Из них
$N$ (св.) окажутся с бантами. Найдите $E(N)$, если
всего в лесу $z$ зайцев. \par
а) Придумайте MM оценку общего числа зайцев. \par
б) Придумайте ML оценку общего числа зайцев. \par

} \solution{} \problem{ Шутка \par
Вася утверждает, что оценки метода максимального правдоподобия
являются состоятельными. Является ли Васино
утверждение \par
а) максимально правдоподобным; \par
б) состоятельным? \par

} \solution{} \problem{
Найдите MM и ML оценки параметра $a$, если $X_{i}$ - независимы и
одинаково распределены с функцией плотности
$p(t)=a^{2}\cdot t\cdot e^{-a\cdot t}$ при $t>0$. \par

} \solution{} \problem{
Пусть с.в. $X_{i}$ независимы и имеют функцию плотности
$p(t)=\frac{a}{2}e^{-a\cdot |t|}$. Найдите оценку параметра $a$ \par
a) методом максимального правдоподобия \par
б) методом моментов <<приравняв>> теоретическую $Var(X_{i})$ и
эмпирическую
$\hat{\sigma}^{2}$; \par
в) методом моментов <<приравняв>> $E(X^{2})$ и соответствующий
эмпирический момент. \par

} \solution{} \problem{ [т?] \par
Пусть с.в. $X_{i}$ независимы и имеют функцию плотности
$p(t)=\frac{a}{2}e^{-a\cdot |t-b|}$. Найдите ML оценку параметров
$a$ и $b$. \par

} \solution{} \problem{ [т] \par
Пусть наблюдаются $Y_{1}=\beta+u_{1}$ и $Y_{2}=2\cdot\beta+u_{2}$,
где ненаблюдаемые $u_{i}$ независимо и одинаково распределены,
причем $E(u_{i})=0$ и $Var(u_{i})=\gamma$. Оказалось, что $y_{1}=0.9$, а $y_{2}=2.3$. Найдите оценки методом максимального правдоподобия для $\beta$ и $\gamma$, если: \par
а) $u_{i}$ - равномерно распределены; \par
б) $u_{i}$ - нормально распределены; \par

} \solution{} \problem{
Пусть $Y_{1}$ и $Y_{2}$ независимы и распределены по Пуассону. Известно также, что $E(Y_{1})=e^{a}$ и $E(Y_{2})=e^{a+b}$. Найдите ML оценки для $a$ и $b$. \par
Ответ: $\hat{a}=\ln(Y_{1})$, $\hat{b}=\ln(Y_{2})-\ln(Y_{1})$ \par
Source: Suhov, Probability and statistics \par


} \solution{} \problem{
Пусть $X_{i}$ независимы и одинаково распределены $N(\a,2\a)$ \par
По выборке $X_{1}$, ..., $X_{n}$ постройте оценку для $a$: \par
а) Методом моментов \par
б) Методом максимального правдоподобия \par


} \solution{} \problem{
На отрезке $[0;b]$ равномерно независимо друг от друга выбираются два числа. Пете сообщают величину $X$, максимум из этих двух чисел. Васе сообщают величину $X$, минимум из этих двух чисел. \par
а) Помогите Васе и Пете построить оценки неизвестного $b$ методом моментов и методом максимального правдоподобия. \par
б) Какие из четырех оценок будут несмещенными? \par
в) Какая из четырех оценок будет наиболее эффективной? Наиболее эффективной среди несмещенных? \par
г) Решите эту задачу, если вместо двух чисел на отрезке выбираются $n$ чисел. \par

} \solution{} \problem{ Корректоры очепяток-2 \par
Вася замечает опечатку с вероятностью $p$, Петя независимо от Васи замечает опечатку вероятностью $q$. В книге имеется $n$ опечаток. Известно, что Вася обнаружил $N_{1}$ опечаток, Петя - $N_{2}$ опечаток, причем $N_{12}$ опечаток были обнаружены и Петей, и Васей. \par
а) Предполагая, что параметры $p$ и $q$ известны, а $n$ - неизвестен, построить оценку $n$ методом максимального правдоподобия. \par
б) Предполагая, что $p$, $q$, $n$ - неизвестны, построить их оценки методом максимального правдоподобия. \par
Solution: \par
Задача \ref{korrektori ochepiatok} может быть полезна. \par
а) Находим вероятность получить заданные $N_{1}$, $N_{2}$, $N_{12}$ как функцию от $p$, $q$ и $n$ (получаем $f(p,q,n)$) \par
Максимизируем по $n$: значение $n$ увеличиваем пока $f(p,q,n+1)/f(p,q,n)\ge 1$ \par
Получаем $\hat{n}=\frac{N_{1}+N_{2}-N_{12}}{p+q-pq}$ (если забыть про целочисленность) \par
б) Оценки ML для $p$ и $q$ выглядят стандартно: \par
$\hat{p}=\frac{N_{1}}{\hat{n}}$, $\hat{q}=\frac{N_{2}}{\hat{n}}$ \par
Решая систему уравнений получаем: \par
$\hat{n}=\frac{N_{1}N_{2}}{N_{12}}$ \par
$\hat{p}=\frac{N_{12}}{N_{2}}$ \par
$\hat{q}=\frac{N_{12}}{N_{1}}$ \par

} \solution{} \problem{
Вася стрелял по мишени до $n$-го промаха. Вася записывал, только сколько попаданий подряд у него было. Осталась запись 4, 3, 5, 6, 8. Она означает что сначала у него было 4 попадания подряд, затем сколько-то промахов, затем 3 попадания подряд, затем сколько-то промахов и т.д. Предположим, что в течение этого времени меткость Васи не менялась. Постройте оценку для Васиной меткости \par
а) методом моментов \par
б) методом максимального правдоподобия \par
в) оцените $n$ методом максимального правдоподобия \par
Решение: \par
уточним, что $n$ - число промахов, $T$ - число записанных Васей цифр, $\bar{X}=\frac{\sum X_{i}}{T}$ \par
по смыслу задачи находим закон распределения: $P(X_{i}=k)=p^{k-1}q$ \par
а) $E(X_{i})=\frac{1}{q}$, отсюда $\hat{p}=1-\frac{1}{\bar{X}}$ \par
б) максимизируем по $p$ (при заданном числе записанных серий) получаем также $\hat{p}=1-\frac{1}{\bar{X}}$ \par
в) если максимизировать и по $n$: $L=P(X_{1},...,X_{T}|T)\cdot P(T)=\left(\frac{q}{p}\right)^{T}p^{X_{1}+...+X_{T}}C_{n}^{T}p^{T}q^{n-T}$ \par
Находим когда $L(n)$ растет, т.е. когда $L(n+1)\ge L(n)$ \par
Получаем $\hat{n}=\frac{T}{\hat{p}}$. \par

} \solution{} \problem{
Допустим, что поток посетителей ларька - Пуассоновский процесс с интенсивностью $\lambda$ человек в час. Продавщица каждый день закрывает ларек на обед с 13:00 до 14:00, выходных нет. \par
а) Сколько (в среднем) посетителей придут к закрытому ларьку за неделю? \par
% копия задачи в Пуассоновских \par
б) Имеются наблюдения $X_{1}$, $X_{2}$, ...$X_{n}$ (интервалы времени между отдельными посетителями, $n\ge2$). С помощью этих величин постройте несмещенную и с минимальной диспресией оценку среднего количества посетителей, приходящих к закрытому ларьку. \par
в) Что происходит в случае $n=1$? \par
Ответы: \par
а) $7\lambda$ \par
б) $\frac{7(n-1)}{\sum X_{i}}$ \par
в) несмещенной оценки не существует \par
Link: \par
10336, Expected number of sums in a given set, American Mathematical Monthly, Vol. 104, No 1 (Jan. 1997) \par
http://www.jstor.org/stable/2974832 \par
Авторы: Kotlarsky, Agnew, Schilling, Roters \par


} \solution{} \problem{ Серия 10-20-30 \par
В крупном банке 10 независимых клиентских <<окошек>>. В момент открытия в банк вошло 10 человек. Предположим, что время обслуживания одного клиента распределено экспоненциально с параметром $\lambda$. Оцените параметр $\lambda$ методом максимального правдоподобия в каждой из ситуаций: \par
а) Менеджер записал время обслуживания первого клиента в каждом окошке. Первое окошко обслужило своего первого клиента за 10 минут, второе (своего первого) - за 20 минут; оставшуюся часть записей менеджер благополучно затерял. \par
б) Менеджер наблюдал за окошками в течение получаса и записывал время обслуживания первого клиента. Первое окошко обслужило своего первого клиента за 10 минут, второе (своего первого) - за 20 минут; остальные окошки еще обслуживали своих первых клиентов в тот момент, когда менеджер удалился. \par
в) В момент открытия банка в банк вошло 10 человек. Менеджер наблюдал за окошками в течение получаса. За эти полчаса два окошка успели обслужить своих первых клиентов. Остальные окошки еще обслуживали своих первых клиентов в тот момент, когда менеджер удалился. \par
г) Менеджер наблюдал за окошками и решил записать время обслуживания первых двух клиентов. Первое окошко обслужило своего первого клиента за 10 минут, второе (своего первого) - за 20 минут. Сразу после того, как был обслужен второй клиент менеджер прекратил наблюдение. \par
д) Одновременно с открытием банка началась деловая встреча директора банка с инспектором по охране труда. Время проведения таких встреч - случайная величина, имеющая эскпоненциальное распределение со средним временем 30 минут. За время проведения встречи было обслужено два клиента. \par
е) Одновременно с открытием банка началась деловая встреча директора банка с инспектором по охране труда. Время проведения таких встреч - случайная величина, имеющая эскпоненциальное распределение со средним временем 30 минут. За время проведения встречи было обслужено два клиента, один за 10 минут, второй - за 20 минут. \par
ж) Изменим условие: в банке 11 окошек, при открытии банка вошло 11 клиентов. Клиент попавший в 11 окошко смотрел за остальными. Раньше клиента из 11 окошка освободилось двое клиентов: за 10 минут и за 20 минут. }

\solution{ а) $\hat{\lambda}=\frac{10+20}{2}$ \par
б) $L=p(10)p(20)\frac{P(X>30)^{8}}{P(X<30)^2}$ \par
в) $L=P(X<30)^2\cdot P(X>30)^{8}$ \par
ж) $L=p(10)p(20)P(X>20)^{9}$ \par 
г) $L=p(10)p(20)P(X>20)^{8}$ \par 
д) Вероятность быть обслуженным во время встречи равна $p(\lambda)=\frac{\lambda}{30+\lambda}$, поэтому $L=p(\lambda)^2(1-p(\lambda))^{8}$ \par
е) $L=...(1-p(\lambda))^8$ \par

} 


\problem{
На плоскости нарисована окружность с центром в начале координат и радиусом $r$. На ней равномерно выбирается $n$ точек. Оцените $r$ методом максимального правдоподобия, если известны координаты всех точек. \par
Ответ: окружность наименьшего радиуса, накрывающая все точки \par

} \solution{} \problem{
На плоскости нарисована окружность с центром $(a,b)$ и радиусом 1 см. На ней равномерно выбирается $n$ точек. Оцените вектор $(a,b)$ методом максимального правдоподобия, если известны координаты всех точек. \par
Ответ: множество таких пар $(a,b)$, что получающаяся окружность накрывает все точки \par
} \solution{} 



\problem{
Предположим, что доход жителей страны распределен экспоненциально с параметром $\lambda$. \par
Имеется выборка из 1000 наблюдений по жителям столицы. Постройте 90\% доверительный интервал для $\lambda$. Если...\par
а) столицу можно считать случайной выборкой из жителей страны  \par
б) в столице селятся только люди с доходом больше 100 тыс. рублей. \par
в) в столице селятся только люди с доходом больше $m$ тыс. рублей, где $m$ - неизвестная константа. При этом постройте также и 90\% доверительный интервал для $m$ \par
г) в столице живут 10\% самых богатых жителей страны. \par
д) в столице живут $p$\% самых богатых жителей страны, где $p$ - неизвестная константа. Постройте также 90\% доверительный интервал для $p$. \par

} \solution{} 

\problem{
$X_{i}$ - iid $U[0;a]$. Помимо числа $n$ мы знаем: \par
а) сколько $X$-ов из $n$ меньше 100 \par
б) сами значения иксов \par
в) сколько $X$-ов из $n$ меньше 100 и чему равны $X$ меньшие 100 \par
Оценить $a$ - методом моментов, максимального правдоподобия } \solution{} 



\problem{Как узнать ширину и длину, зная только площадь? <<Насяльника>> отправила Равшана и Джамшуда измерить ширину и длину земельного участка. Равшан и Джамшуд для надежности измеряеют длину и ширину 100 раз. Равшан меряет длину, результат измерений - случайная величина $X_{i}=X^{*}+\Delta X_{i}$, где $X^{*}$ - истинная длина участка, а $\Delta X_{i}\sim N(0;1)$ - ошибка измерения. Джамшуд меряет ширину, результат измерений - случайная величина $Y_{i}=Y^{*}+\Delta Y_{i}$, где $Y^{*}$ - истинная ширина, а $\Delta Y_{i}\sim N(0;1)$ - ошибка измерения. Предположим, что все ошибки измерений независимы. Предполагая, что <<насяльника>> хочет измерить площадь участка, каждый раз Равшан и Джамшуд сообщают <<насяльнику>> только величину $S_{i}=X_{i}Y_{i}$. 

Помогите <<насяльнику>> оценить $X^{*}$ и $Y^{*}$ методом моментов.}
\solution{Нужно оценить два параметра, значит нужно два уравнения, находим первый и второй моменты. $E(S_{i})=X^{*}Y^{*}$, $E(S_{i}^{2})=E(X_{i}^{2})E(Y_{i}^{2})=(1+(X^{*})^{2})(1+(Y^{*})^{2})=1+(X^{*})^{2}+(Y^{*})^{2}+(X^{*}Y^{*})^{2}$.

Пусть $A=\frac{\sum_{i}S_{i}}{n}$, $B=\frac{\sum_{i}S_{i}^{2}}{n}$. тогда, $\hat{X}\hat{Y}=A$ и $\hat{X}+\hat{Y}=\sqrt{B+2A-A^{2}-1}$.

Составляем квадратное уравнение 
\begin{equation}
t^{2}-\sqrt{B+2A-A^{2}-1}t+A=0
\end{equation}
и получаем (мы не можем определить, что конкретно является шириной, а что - длиной):
\begin{align}
\hat{X}=\frac{\sqrt{B+2A-A^{2}-1}-\sqrt{B-2A-A^{2}-1}}{2} \\
\hat{Y}=\frac{\sqrt{B+2A-A^{2}-1}+\sqrt{B-2A-A^{2}-1}}{2} 
\end{align}
}


\problem{В коробке 10 внешне не отличимых шоколадных конфет. Внутри $k$ штук из них есть орех. Мы выбирали конфеты наугад по одной и ели. Первый орех оказался в третьей по счету конфете. 

\begin{itemize}
\item Оцените неизвестный параметр $k$ методом моментов, методом максимального правдоподобия.
\item Постройте байесовскую оценку, если изначально мы верили, что $k$ равновероятно принимает значения от 0 до 10.
\end{itemize}

}
\solution{}

\problem{
Интервал времени в минутах между спам-письмами по электронной почте --- случайная величина с функцией плотности 
\begin{equation}
p(t)=
\begin{cases}
a^2\cdot t\cdot \exp(-at), \quad t\geq 0 \\
0, \quad t<0
\end{cases},
\end{equation}
где $a$ --- неизвестный параметр. По выборке из 20 наблюдений известно, что $\sum_{i=1}^{n}X_{i}=625$, $\sum_{i=1}^{n}\ln(X_i)=25$.


\begin{enumerate}
\item Оцените $a$ методом моментов
\item Оцените $a$ методом максимального правдоподобия
\item Постройте 95\% доверительный интервал для $a$ с помощью максимального правдоподобия
\end{enumerate}}


\solution{
\begin{equation}
E(\bar{X})=E(X_{i})=\int_{0}^{\infty} a^2\cdot t^2 \exp(-at)dt
\end{equation}

Для удобства заметим, что
\begin{equation}
\int_{0}^{\infty}t^n\exp(-at)dt=0+\int_{0}^{\infty}t^{n-1}\frac{n}{a}\exp(-at)dt
\end{equation}

Получаем мат. ожидание:
\begin{multline}
E(\bar{X})=\int_{0}^{\infty} a^2\cdot t\frac{2}{a}\cdot \exp(-at)dt=\\
=\int_{0}^{\infty} 2a\cdot t\cdot \exp(-at)dt=\int_{0}^{\infty} 2a\cdot \frac{1}{a} \exp(-at)dt=\\
=\int_{0}^{\infty} 2\cdot \exp(-at)dt=\frac{2}{a}
\end{multline}

Метод моментов
\begin{equation}
\bar{X}=\frac{2}{\hat{a}_{MM}}
\end{equation}

Получаем $\hat{a}_{MM}$:
\begin{equation}
\hat{a}_{MM}=\frac{2}{\bar{X}}=\frac{40}{625}=\frac{8}{125}=0.064
\end{equation}

Метод максимального правдоподобия
\begin{equation}
l=\sum_{i=1}^{n}\ln(p(x_{i}))=\sum_{i=1}^{n}(2\ln(a)+ln(x_{i})-ax_{i})=2n\ln(a)+\sum \ln(x_{i})-a\sum x_{i}
\end{equation}

\begin{equation}
l'(a)=\frac{2n}{a}-\sum x_{i}
\end{equation}

Получаем $\hat{a}_{ML}$:
\begin{equation}
\hat{a}_{ML}=\frac{2n}{\sum X_{i}}=\frac{2}{\bar{X}}=\hat{a}_{MM}
\end{equation}

Наблюдаемая информация Фишера
\begin{equation}
J_{n}(\hat{a})=-l''(\hat{a})=\frac{2n}{\hat{a}^{2}}=2n\cdot \left(\frac{\sum X_{i}}{2n}\right)^2=\frac{(\sum X_{i})^2}{2n}=\frac{625^2}{40}
\end{equation}

Доверительный интервал
\begin{equation}
\hat{a}\pm 1.96\cdot J_{n}^{-1/2}=0.064\pm 0.0198=[0.0442;0.0838]
\end{equation} }


\section{Гипотезы о среднем}
% mean_hypo

% ее можно давать гораздо раньше, но здесь она в тему..
\problem{
a) Пусть $X$ - равномерна на $[\mu-1;\mu+1]$, причем $\mu$ - неизвестна. Найдите $E(X)$. С какой вероятностью случайный интервал $[X-0.5;X+0.5]$ накрывает неизвестное $\mu$?

b) Пусть $X_{1}$, $X_{2}$ - нормальны $N(\mu,1)$, причем $\mu$ - неизвестна. Найдите $E(X_{1})$, $E(\bar{X})$. С какой вероятностью случайный интервал $[X_{1}-0.5;X+0.5]$ накрывает неизвестное $\mu$? А интервал $[\bar{X}-0.5;\bar{X}+0.5]$? Почему вторая вероятность не равна первой?}
\solution{ b) У $\bar{X}$ меньше дисперсия, поэтому она в среднем ближе к $\mu$, чем $X_{1}$.}

\problem{
До проведения рекламной компании в среднем 7 из 10
посетителей художественного магазина-салона делали покупки. После
рекламной компании из 200 посетителей покупки сделали 163. Можно
ли считать, что рекламная компания имела эффект на 5\%-ом уровне
значимости? \par

} \solution{} \problem{
Дневные расходы электроэнергии на предприятии составляли
1400 КВт со стандартным отклонением 50 КВт. За 50 дней прошедших
после ремонта и наладки оборудования средние расходы за день
составили 1340 КВт. Можно ли считать, что ремонт способствовал
экономии электроэнергии на 10\% уровне значимости? \par

} \solution{} \problem{
Монету подбросили 1000 раз, при этом 519 раз она выпала на орла.
Проверьте гипотезу о том, что монета <<правильная>> на уровне
значимости 5\%.\par

} \solution{} \problem{
Вася отвечает на 100 тестовых вопросов. В каждом вопросе
один правильный вариант ответа из пяти возможных. На 5\%-ом уровне
значимости проверьте гипотезу о том, что Вася ставит ответы
наугад, если он ответил правильно на 26 вопросов из теста.\par

} \solution{} \problem{
Некоторых студентов спросили, на какую оценку они рассчитывает по
теории вероятностей, 30 человек надеются на 4 балла, 20 человек –
на 6 баллов, 30 человек – на 8 баллов, 10 человек – на 10 баллов.
Проверьте гипотезу о том, что медиана равна 7 баллам на уровне
значимости 10\%.\par

} \solution{} \problem{
Кубик подбросили 160 раз, из них 29 он выпал на шестерку.
Проверьте гипотезу о том, что вероятность выпадения шестерки
правильная на уровне значимости 10\%.\par

} \solution{} \problem{
Двести домохозяек попробовали новый <<{\it Вовсе не обычный
порошок}>>, 110 из них получили более удачный результат, чем
раньше. На уровне значимости 5\% проверьте гипотезу о том, что
<<{\it Вовсе не обычный порошок}>> по эффективности не отличается
от старого средства (против альтернативной гипотезы о большей
эффективности).\par


} \solution{} \problem{
Величины $x_{1}$, $x_{2}$, ..., $x_{n}$ независимы и распределены $N(10,16)$. Вася знает дисперсию, но не знает среднего. Поэтому он строит 60\% доверительный интервал для истинного среднего значения. У него получаются две границы - левая и правая. \par
Какова вероятность того, что:\par
а) левая меньше 9?
б) левая и правая лежат по разные стороны от 9? \par
б) левая и правая лежат по разные стороны от настоящего среднего? \par

Доделать, может включить неизвестную дисперсию? \par


} \solution{} \problem{
По предварительному опросу 10000 человек на выборах в Думу 462
человека будут голосовать за партию <<{\it Обычная партия}>>. На
уровне значимости 0,05 проверьте гипотезу о том, что <<{\it
Обычная
партия}>> преодолеет 5\% барьер.\par

} \solution{} \problem{
Вася и Петя метают дротики по мишени. Каждый из них сделал
по 100 попыток. Вася оказался метче Пети в 59 попытках. На уровне
значимости 5\% проверьте гипотезу о том, что меткость Васи и Пети
одинаковая, против альтернативной гипотезы о том, что Вася метче
Пети. \par

} \solution{} \problem{
По 820 посетителям супермаркета средние расходы на одного
человека составили 340 рублей. Из достоверных источников известно,
что дисперсия равна 90000 руб. Постройте $95\%$ доверительные
интервалы для средних расходов одного посетителя
(двусторонний и два односторонних). \par

} \solution{} \problem{
В прошлом году средняя длина ушей зайцев в темно-синем
лесу была 20 см, $\sigma=4$. В этом году у случайно попавшихся 15
зайцев средняя длина оказалась 24 см. Предполагая нормальность
распределения, проверьте гипотезу о том, что средняя длина ушей не
изменилась (против альтернативной гипотезы о росте длины). \par

} \solution{} \problem{
Стандартное отклонение количества иголок у ежа равно 130.
По выборке из 12 ежей было получено среднее количество иголок
5120. Допустим, что количество иголок на одном еже можно считать
нормально распределенным. \par
a) Постройте $90\%$-ый доверительный интервал для среднего
количества иголок. \par
б) На $5\%$-ом уровне значимости проверьте гипотезу о том, что
среднее количество иголок равно 5000. \par

} \solution{} \problem{
Средний бал по диплому студента - c.в. $N(\mu;0.04)$.
Средний бал, рассчитанный по выборке из 25 абитуриентов этого
года, составил $4.30$. По данной выборке был построен
доверительный интервал для $\mu$: $(4.2424; 4.3576)$. Какой
уровень доверия соответствует этому
интервалу? \par

} \solution{} \problem{
Вася очень любит играть в преферанс. Предположим, что Васин
выигрыш распределен нормально. За последние 5 партий средний
выигрыш составил 1560 рублей, при оценке стандартного отклонения
равной 670 рублям. Постройте 90\%-ый доверительный интервал для
математического ожидания Васиного выигрыша. \par


} \solution{} \problem{
In 1882 Michelson performed experiments to measure the speed of
light. 23 trials gave an average of 299756.2 km/sec with a
standard deviation of 107.12. Find a $95\%$ confidence interval
for the speed of light. The correct answer is 299710.5 so there
must have be some bias in his experiments. \par

} \solution{} \problem{
An English biologist named Weldon was interested in the
'pip effect' in dice – the idea that the spots, or 'pips', which
on some dice are produced by cutting small holes in the surface,
make the sides with more spots lighter and more likely to turn up.
Weldon threw 12 dice 26306 times for a total of 315672 throws and
observed that a 5 or 6 came up on 106602 throws. Find a $95\%$
confidence interval for the true probability of getting 5
or 6 on a dice. \par

} \solution{} \problem{
On 384 out of 600 randomly selected farms, the operator was
also the owner. Find a $95\%$ confidence interval for the true
proportion of owner operated farms. \par

} \solution{} \problem{
During a two week period (10 weekdays) a parking garage
collected an average of \$126 with a standard deviation of \$15.
Find a 95\% confidence interval for the mean revenue. \par
Problems are borrowed from \url{www.math.cornell.edu/$\sim$durrett/ep4a/ep4a.html} \par


} \solution{} \problem{
In their last 100 chess games played against each other, Bill has
won 46 and Monica has won 54. Using this information and a 95\%
confidence level, what is the probability that Bill will win a
<<best of seven>> series with Monica? The first one to win 4 games
is the winner and no more games are played. \par

Hints: First determine a 95\% confidence interval for the
probability that Bill will win a game. Then, using the two
extremes of this interval, determine the probability that Bill
will win the series. This is a binomial experiment. Bill could win
the series in 4 games, 5 games, 6 games, or 7 games. Calculate the
probability of each and add them up. Do this for each of the two
interval extremes. \par
Source: (?):\par
\url{http://www.artofproblemsolving.com/Forum/viewtopic.php?highlight=probability+game\&t=87203}



} \solution{} \problem{
Имеются две монетки. Одна правильная, другая - выпадает орлом с
вероятностью $0<q<0.5$, значение $q$ известно. Монетки неотличимы
по внешним признакам. Одну из них (неизвестно какую) подкинули $N$
раз и сообщили Вам, сколько раз выпал орел. Опишите процедуру
тестирования гипотезы $H_{0}$: <<подбрасывалась правильная
монетка>>
против $H_{a}$: <<подбрасывалась неправильная монетка>>. \par
а) Каким должно быть $N$ чтобы вероятность ошибок первого и
второго рода не превышала 10 процентов, если $q=0.4$ \par
b) Ответьте вопрос <<a>> при произвольном $q$ \par

} \solution{} \problem{
Имеется две конкурирующие гипотезы: \par
$H_{0}$: Величина $X$ распределена равномерно на отрезке $[0;100]$ \par
$H_{a}$: Величина $X$ распределена равномерно на отрезке $[50;150]$ \par
Исследователь выбрал такой критерей: \par
Если $X<c$, то использовать $H_{0}$, иначе использовать $H_{a}$. \par
а) Что такое <<ошибка первого рода>>, <<ошибка второго рода>>,
<<мощность теста>>? \par
б) Постройте графики зависимостей ошибок первого и второго рода от
$c$. \par

} \solution{} \problem{
 Известно, что  $X_{i}$ iid $N\left(\mu ;900\right)$ .
Исследователь проверяет гипотезу $H_{0}$: $\mu =10$  против
$H_{A}$: $\mu =30$  по выборке из 20 наблюдений. Критерий выглядит
следующим образом: если  $\bar{X}>c$ , то выбрать  $H_{A} $ ,
иначе выбрать  $H_{0} $.\par
 а) Рассчитайте вероятности ошибок
первого и второго рода, мощность критерия для $c=25$. \par
б) Что произойдет с указанными вероятностями при росте количества
наблюдений ($c\in(10;30)$)? \par
в) Каким должно быть $c$, чтобы вероятность ошибки второго рода
равнялась $0,15$? \par
г) Как зависят от $c$ вероятности ошибок первого и второго рода
($c\in(10;30)$)? } 
\solution{} 

\problem{
Дама утверждает, что обладает особыми способностями и безошибочно
отличает <<бонакву>> без газа от <<святого источника>> без газа.
$H_{0}$: дама не обладает особыми способностями, $H_{a}$: дама
обладает особыми способностями. При даме в 3 стаканчика из 8-ми
налили <<бонакву>>, а в 5 оставшихся - <<святой источник>>. При
отгадывании стаканчики предлагаются даме в неизвестном ей порядке.
Критерий: принимается основная гипотеза, если дама ошиблась хотя
бы один раз и альтернативная иначе. \par
а) Рассчитайте вероятности ошибок первого и второго рода, мощность
критерия. \par
б) Сколько из 8 стаканчиков надо наполнить <<бонаквой>> и сколько
<<святым источником>>, чтобы вероятность ошибки первого рода была
минимальной? \par
Коммент: некоторые студенты утверждают, что отличить <<святой
источник>> от <<бонаквы>> - элементарно, а вот отличить
<<бонакву>> от <<акваминерале>> - трудно. } 
\solution{} 


\problem{Аня и Таня любят мыть посуду. Аня разбивает тарелку, которую моет, с вероятностью $ p_{a} $, Таня --- с вероятностью $ p_{t} $. Они хотят проверить гипотезу $ H_{0}: p_{a}=p_{t} $ против альтернативной $ H_{a}: p_{a}>p_{t} $. Способ проверки следующий. Они по очереди будут мыть по одной тарелки, до тех пор пока не разобьется две тарелки. Если обе эти тарелки будут разбиты Аней, то $ H_{0} $ будет отвергнута в пользу $ H_{a} $. Определите вероятность ошибки первого рода как функцию от $p_{a}$, если:
\begin{itemize}
\item Первой моет тарелку Аня
\item Первой моет тарелку Таня
\end{itemize} }
\solution{ Если начинает Аня, то $ \frac{1-p}{(2-p)^{2}} $; Если начинает Таня, то $ \frac{(1-p)^{2}}{(2-p)^{2}} $. Источник: Кирилл Фурманов}




\problem{
Школьник Вася аккуратно замерял время, которое ему требовалось, чтобы добраться от школы до дома. По результатам 90 наблюдений, среднее выборочное оказалось равным 14 мин, а несмещенная оценка дисперсии - 5 мин$^{2}$. \par
a) Постройте 90\% доверительный интервал для среднего времени на дорогу \par
б) На уровне значимости 10\% проверьте гипотезу о том, что среднее время равно 14,5 мин, против альтернативной гипотезы о меньшем времени. \par
в) Чему равно точное $P$-значение при проверке гипотезы в п. <<б>>? }
\solution{ a) $[13.61;14.39]$ \par
b) Отвергается ($Z_{observed}=-2.12$, $Z_{critical}=-1.28$) \par
c) $P_{value}=0.017$ \par
} 



\problem{
На днях Левада-Центр опубликовал итоги опроса, согласно которым 2/3 россиян поддерживают Путина и 2/5 россиян доверяют опросам Левада-Центра. Договоримся, что доверяющие опросам всегда отвечают искренне, а недоверяющие могли соврать в ответе на любой вопрос или оба. Исходя из этих данных, оцените реальную поддержку Путина россиянами. (Постройте 95\% доверительный интервал (для поддерживающих Путина и для верящих в опрос), если было опрошено 1000 человек). \par
source: лента ru-math } \solution{} 


\section{Остальные гипотезы}
% other_hypo



\problem{
Изучалось воздействие модератора на количество идей,
сочиняемых группой людей. По выборке из 4-х групп с модератором,
среднее количество идей оказалось равным 78, при стандартном
отклонении 24.4, по выборке из 4-х групп без модератора, среднее
количество идей оказалось равным 63.5, при стандартном отклонении
20.2. Предположим нормальность распределения. \par
а) Проверьте гипотезу о равенстве дисперсий. \par
б) Предполагая равенство дисперсий проверьте гипотезу о равенстве средних. \par

} \solution{} \problem{
Маркетинговый отдел банка опросил 300 женщин и 400 мужчин.
Оказалось, что реклама банка вызывает положительные эмоции у 74\%
опрошенных женщин и 69\% опрошенных мужчин. \par
а)  Можно ли считать, что реклама банка одинаково нравится
мужчинам и женщинам? \par
б) Постройте 90\% доверительный интервал для разницы долей мужчин
и женщин, одобряющих рекламу банка. \par
в) Проверьте гипотезу о том, что рекламу одобряет 70\% женщин
(против гипотезы о том, что рекламу одобряет более 70\% женщин).
\par
г) Предположим, что среди потребителей рекламы мужчин и женщин
поровну. Постройте 95\% доверительный интервал для доли людей,
которым нравится реклама
банка. \par

} \solution{} \problem{
Исследователь сравнивал суровость климата (дисперсию
температуры) в двух странах. Для этого случайным образом были
выбраны 37 наблюдений за среднедневной температурой в первой
стране и 46 наблюдений за среднедневной температурой во второй
стране. Известно, что  $\overline{X}_{I} =14$ , $\overline{X}_{II}
=11$ , $\hat{\sigma }_{I}^{2} =2341$  и $\hat{\sigma }_{II}^{2}
=3079$.
\par а)  Постройте 95\% доверительный интервал для разности
математических ожиданий среднедневных температур в двух странах.\par
б)  Предполагая, что среднедневная температура распределена
нормально, проверьте гипотезу об одинаковой суровости климата. С
помощью компьютера найдите точное Р-значение. \par
в)  В предположениях о нормальности постройте 90\%-ые
доверительные интервалы для дисперсий среднедневной температуры в
двух странах. Почему один из
интервалов оказался шире? \par

} \solution{} \problem{
Имеются две нормальные выборки: $\left\{X_{i}
\right\}=\left\{34,28,29,41,32\right\}$ , $\left\{Y_{i}
\right\}=\left\{32,30,31,25,24,29\right\}$. \par
а) Проверьте гипотезу о равенстве дисперсий. С помощью
компьютера укажите точное P-значение\par
б) В предположении о равенстве дисперсий проверьте гипотезу о
равенстве математических ожиданий. С помощью компьютера укажите
точное P-значение. \par


} \solution{} \problem{
Вася Сидоров хвастался перед Аней Ивановой, что в среднем
прыгает не меньше, чем на 2 метра. Напомним его результаты: 1,83;
1,64; 2,27; 1,78; 1,89; 2,33; 1,61; 2,31. Предположим, что длины
прыжков можно считать нормальными \par
а) Постройте $80\%$-ый доверительный интервал для дисперсии длины прыжка. \par
б) Дополнительно предположив, что $\sigma=0,3$ проверьте гипотезу
о том, что Вася действительно прыгает на 2 метра (Аня,
естественно, ему не верит, и утверждает, что он прыгает меньше,
чем на 2 метра). \par
в) Постройте двусторонний
$90\%$-ый доверительный интервал для $\mu$, если $\sigma=0,3$. \par

} \solution{} \problem{
Контрольные камеры ДПС на МКАД, зафиксировали скорость
движения 6-и автомобилей: 89, 83, 78, 96, 81, 79. В предположении
нормальности скоростей: \par
а) Постройте $90\%$-ый доверительный интервал для дисперсии скорости. \par
б) Постройте $90\%$ доверительный интервал для средней скорости автомобилей. \par
в) Постройте $90\%$ доверительный интервал для средней скорости
автомобилей, если известно, что настоящая дисперсия равна 50
(км/ч)$^{2}$ \par
г) На $10\%$-ом уровне значимости проверьте гипотезу о том, что
средняя скорость равна 90 км/ч. \par

} \solution{} \problem{
На курсе два потока, на первом потоке учатся 40 человек, на втором
потоке 50 человек. Средний балл за контрольную на первом потоке
равен 78 при (выборочном) стандартном отклонении в 7 баллов. На
втором потоке средний балл равен 74 при (выборочном) стандартном
отклонении в 8 баллов. \par
а) Постройте 90\% доверительный интервал для разницы баллов между
двумя потоками \par
б) На 10\%-ом уровне значимости проверьте гипотезу о том, что
результаты контрольной между потоками не отличаются. \par
в) Рассчитайте точное P-значение (P-value) теста в пункте 'б' \par

} \solution{} \problem{
Предположим, что время жизни лампочки распределено нормально. По
10 лампочкам оценка стандартного отклонения времени жизни
оказалась равной 120 часам. \par
а) Найдите 80\%-ый (двусторонний)
доверительный интервал для истинного стандартного отклонения. \par
б) Допустим, что выборку увеличат до 20 лампочек. Какова
вероятность того, что выборочная оценка дисперсии будет отличаться
от истинной дисперсии меньше, чем на 40\%? \par

} \solution{} \problem{
Допустим, что логарифм дохода семьи имеет нормальное распределение. В городе А была проведена случайная выборка 40 семей, показавшая выборочную дисперсию 20 (тыс.р.)$^{2}$. В городе Б по 30 семьям выборочная дисперсия оказалась равной 32 (тыс.р.)$^{2}$. \par
На уровне значимости 5\% проверьте гипотезу о том, что дисперсия одинакова, против альтернативной гипотезы о том, что город А более однородный. } 
\solution{} 

\problem{ Допустим, что логарифм дохода семьи имеет нормальное распределение. В городе А была проведена случайная выборка 40 семей, показавшая выборочную дисперсию 20 (тыс.р.)$^{2}$. В городе Б по 30 семьям выборочная дисперсия оказалась равной 32 (тыс.р.)$^{2}$. \par
На уровне значимости 5\% проверьте гипотезу о том, что дисперсия одинакова, против альтернативной гипотезы о том, что город А более однородный. } 
\solution{$F_{29,39}=\frac{32}{20}=1.6$ \par
$F_{critical}=1.74$ \par
Гипотеза о том, что дисперсия одинакова не отвергается. } 




\section{Дисперсионный анализ}
% anova

\problem{
Ниф-ниф ходит обедать в одно из трех близлежащих кафе.\par
Вот данные о его расходах (в рублях):\par
Кафе <<Четыре поросенка>>: 160, 100, 230, 200\par
Кафе <<Суши для поросят>>: 120, 100, 140, 150, 110, 160, 280\par
Кафе <<Здоровый хряк>>: 210, 160, 140, 140, 150\par
а) Проведите однофакторный дисперсионный анализ и сделайте выводы.\par
б) Постройте интервалы для разницы расходов между разными кафе\par

} \solution{} \problem{
Данные о продаже мороженого в киоске, сгруппированные по годам и сезонам:\par
Зима	Весна	Лето	Осень \par
1980	90	120	420	200\par
1981	80	150	380	210\par
1982	90	100	410	180\par
1983	100	130	540	180\par
а) Проведите двухфакторный дисперсионный анализ и сделайте выводы.\par
б) Можно ли сделать вывод о том, что продажи в 83 году выросли по сравнению с 82? } \solution{}

