% !Mode:: "TeX:UTF-8"
\section{Приёмы решения}
\subsection{Разложение в сумму}

\problem{ \label{sudba-don-juan-2}\zdt{Судьба Дон-Жуана-2} (см. тж. с.~\pageref{sudba-don-juan-1})

У Васи $n$  знакомых девушек, их всех зовут по-разному. Он пишет
им $n$  писем, но по рассеянности раскладывает их в конверты
наугад. Случайная величина $X$ обозначает количество девушек, получивших письма,
написанные лично для них. Найдите $\E(X)$, $\Var(X)$. }
\solution{$\E(X)=1$, $\Var(X)=1$. }



\problem{ \label{cube-cut-2}(см. тж. с.~\pageref{cube-cut-1}) \ENGs

A wooden cube that measures 3 cm along each edge is painted red. The painted cube is then cut into 27 pieces of 1-cm cubes. If I tossed all the small cubes in the air, so that they landed randomly on the table, how many cubes should I expect to land with a painted face up? \RUSs}
\solution{ $9$.}


\problem{Вокруг новогодней ёлки танцуют хороводом 27 детей. Мы считаем, что ребенок высокий, если он выше обоих своих соседей. Сколько высоких детей в среднем танцует вокруг елки? Вероятность совпадания роста будем считать равной нулю.}
\solution{Для трёх детей вероятность того, что тот, что посередине "--- самый высокий, равна $ \frac{1}{3} $, значит математическое ожидание равно $ \frac{27}{3}=9$. }

\problem{Маша собирает свою дамскую сумочку. Есть $n$ различных предметов, которые она туда может положить. Каждый предмет она кладёт независимо от других с вероятностью $p$.
\begin{enumerate}
\item Пусть $X$ "--- количество положенных предметов. Найдите $\E(X)$ и $\Var(X)$.
\item При каком $p$ вероятность положить в сумку любой заданный набор вещей не будет зависеть от конкретного набора?
\end{enumerate}}
\solution{ Биномиальное распределение, $\E(X)=np$, $\Var(X)=np(1-p) $. При $p=0{,}5$ все подмножества будут равновероятны.}


\problem{ Игральный кубик подбрасывается 100 раз. Найдите ожидаемую
сумму очков, дисперсию суммы, стандартное отклонение суммы.}
\solution{ }


\problem{ Гипергеометрическое распределение \\
В задачнике $N$ задач. Из них $a$ --- Вася умеет решать, а остальные не умеет. На экзамене предлагается равновероятно выбираемые $n$ задач. Величина $X$ --- число решенных Васей задач на экзамене.

Найдите $\E(X)$ и $\Var(X)$ }
\solution{
$X=X_{1}+...+X_{n}$, $\E(X)=n\frac{a}{N}$ \\
$\Var(X_{i})=\frac{a(N-a)}{N^{2}}$ \\
$\Cov(X_{i},X_{1}+...+X_{N})=0$ \\
$\Cov(X_{i},X_{j})=-\frac{\Var(X_{i}}{N-1}$ \\
$\Var(X)=n\Var(X_{i})\frac{N-n}{N-1}$}

\problem{
Кубик подбрасывается $n$ раз. Величина $X_{1}$ ---
число выпадений 1, а $X_{6}$ --- число выпадений 6. Найдите $\Corr(X_{1},X_{6})$ \\
Подсказка: $\Cov(X_{1},X_{1}+...+X_{6})$ вам в помощь... }
\solution{$\Cov(X_{1},X_{1}+...+X_{6})=0$, т.к. $X_{1}+...+X_{6}=const$
$\Corr(X_{1},X_{6})=-\frac{1}{5}$ }

\problem{ По 10 коробкам наугад раскладывают 7 карандашей. Каково
среднее количество пустых коробок? }
\solution{$10\cdot 0.9^7$ }

\problem{ $[$Mosteller$]$ Среднее число совпадений \\
Из хорошо перетасованной колоды на стол последовательно
выкладываются карты лицевой стороной наверх, после чего
Аналогичным образом выкладывается вторая колода, так что каждая
карта первой колоды лежит под картой из второй колоды. Каково
среднее  число совпадений   нижней  и верхней  карт?}
\solution{ $1$ }


\problem{ Grimmett, 3.3.3. \\
В группе 20 человек. Каждый из них подбрасывает по кубику. Найдите
ожидаемый выигрыш и дисперсию выигрыша группы, если:
\begin{enumerate}
\item за каждую пару игроков, выкинувших одинаковое количество очков,
группа получает один тугрик \\
\item за каждую пару игроков, выкинувших одинаковое количество очков,
группа получает эту сумму в тугриках
\end{enumerate}
}
\solution{ }

\problem{ Coupon's collector problem \\
Внутри упаковки шоколадки <<Веселые животные>> находится наклейка
с изображением одного из 30 животных. Предположим, что все
наклейки равновероятны. Большой приз получит каждый, кто соберет
наклейки всех животных. Какое количество шоколадок в среднем нужно
купить, чтобы выиграть большой приз? }
\solution{ }

\problem{ $[$Mosteller$]$ \\
В $n$ урн случайным образом бросают один за одним $k$ шаров.
Найдите математическое ожидание числа пустых урн. }
\solution{ }


\problem{
У Маши 30 разных пар туфель. И она говорит, что мало! Пес
Шарик утащил 17 туфель без разбору на левые и правые. Сколько
полных пар в среднем осталось у Маши? Сколько полных пар в среднем
досталось Шарику? }
\solution{ }

\problem{
Из колоды в 52 карты извлекается 5 карт. Сколько в среднем
извлекается мастей? Достоинств? Тузов? }
\solution{ Масть: $4\cdot (1-\frac{C_{39}^{5}}{C_{52}^{5}})$

Достоинство: $13\cdot (1-\frac{C_{48}^{5}}{C_{52}^{5}})$

Туз: $4\cdot \frac{5}{52}$ }


\problem{
На карточках написаны числа от 1 до $n$. В игре участвуют
$n$ человек. В первом туре каждый получает случайным образом по
одной карточке. Во втором туре карточки выдаются заново. Призы
раздаются по следующему принципу: Человек не получает приз, только
если найдется кто-то другой, кто получил большие числа в каждом
туре.
Каково среднее количество человек, получивших приз? \\
Взято с www.zaba.ru, какая-то олимпиада. }
\solution{ }

\problem{
А.А. Мамонтов сидит в 424 аудитории. Эконометрику
собираются сдавать несколько человек. На поиски пустых аудиторий
послано 3 студента-разведчика. На втором этаже 9 учебных
аудиторий, 5 из них заняты. Каждый из 3 студентов-разведчиков
независимо друг от друга заглядывает в 3 аудитории. Если студент
обнаруживает пустую аудиторию, то он сообщает ее номер А.А.
Мамонтову. Каково среднее
количество обнаруженных пустых аудиторий? }
\solution{ }


\problem{
Вася пишет друг за другом наугад 100 букв из латинского алфавита.
\begin{enumerate}
\item Каково ожидаемое количество букв, встречающихся в написанном <<слове>> ровно один раз?
\item Как изменилась бы искомая величина, $a_{k,n}$, если бы в алфавите было $k$ букв, а Вася писал бы <<слово>> из $n$ букв?
\item Найдите $\lim_{n\to\infty} a_{k,n}$, $\lim_{k\to\infty} a_{k,n}$
\end{enumerate}
}
\solution{ }

\problem{
За круглым столом сидят в случайном порядке $n$ супружеских пар, всего --- $2n$ человек. Величина $X$ --- число пар, где супруги оказались напротив друг друга.

Найдите $\E(X)$ и $\Var(X)$ }
\solution{ }

\problem{
Suppose there were $m$ married couples, but that $d$ of these $2m$ people have died. Regard the $d$ deaths as striking the $2m$ people at random. Let $X$ be the number of surviving couples. Find: $\E(X)$ and $\Var(X)$ }
\solution{ }



\problem{
Над озером взлетело 20 уток. Каждый из 10 охотников
стреляет в утку по своему выбору.
\begin{enumerate}
\item Каково ожидаемое количество убитых уток, если охотники стреляют без промаха?
\item Как изменится ответ, если вероятность попадания равна 0,7?
\item Каким будет ожидаемое количество охотников, попавших в цель?
\end{enumerate}
}
\solution{ }


\problem{
В каждой из двух урн находится по 50 белых и 50 черных шаров. Вася одновременно вытаскивает по шару из каждой урны и выбрасывает их.
Величина $X$ --- количество раз, когда из обеих урн были одновременно вытащены белые шары.

Найдите $\E(X)$, $\Var(X)$ }
\solution{ }

\problem{
На карточках написаны числа от 1 до $n$. Вася достает их одну за другой наугад. Если номер карточки является соседним с номером предыдущей карточки, то Вася получает 1 рубль. Величина $X$ --- Васин выигрыш. \\
Найдите $\E(X)$, $\Var(X)$ }
\solution{ }

\problem{
Вася называет наугад 50 чисел от 1 до 100, допускаются повторения, а Петя называет наугад 50 чисел от 1 до 100 без повторов. \\
Величины $X$ и $Y$ это суммы этих чисел.
\begin{enumerate}
\item Сравните $\E(X)$ и $\E(Y)$
\item Сравните $\Var(X)$ и $\Var(Y)$
\end{enumerate}
}
\solution{ $\E(X)=\E(Y)$, $\Var(X)>\Var(Y)$}



\problem{
Кубик подбрасывается до тех пор, пока каждая грань не
выпадет по разу. Найдите математическое ожидание и дисперсию числа
подбрасываний. }
\solution{ }

\problem{
Правильная монетка подбрасывается  $n$  раз. Серия --- это
последовательность подбрасываний из одинаковых результатов. К
примеру, в последовательности ОООРРО три серии.
\begin{enumerate}
\item Каково ожидаемое количество серий? \\
\item Дисперсия числа серий? \\
\item А если монетка неправильная и выпадает гербом с вероятностью  $p$?
\end{enumerate}
}
\solution{ }


\problem{ В здании 10 этажей, на каждом этаже 30 окон. Вечером в каждом окне независимо от других свет включается с вероятностью $p$.
\begin{enumerate}
\item Чему равно ожидаемое количество <<ноликов>> на фасаде здания?
\item Чему равно ожидаемое количество <<крестиков>> на фасаде здания?
\item При каких $p$ эти количества максимальны? Минимальны?
\end{enumerate}
Примечание: два разных нолика могут иметь общие точки \\
Вставить рисунок нолика и рисунок крестика, пример подсчета }
\solution{ }

\problem{
Если смотреть на корпус Ж здания Вышки с Дурасовского переулка, то видно 40 окон. (??? Видно 7 этажей, первый не видно, и 8 окон на каждом этаже, уточнить по месту). Допустим, что каждое из них освещено вечером независимо от других с вероятностью одна вторая. Назовем <<уголком>> комбинацию из 4-х окон, расположенных квадратом, в которой освещено ровно три окна (не важно, какие). Величина $X$ --- число <<уголков>>, возможно пересекающихся, на всем корпусе Ж. \\
Найдите  $\E(X)$ и $\Var(X)$ \\
Примечание - для наглядности: \\
\begin{tabular}{|c|c|}
  \hline
  X & X\\
  \hline
    & X \\
  \hline
\end{tabular},
\begin{tabular}{|c|c|}
  \hline
  X & \\
  \hline
  X & X \\
  \hline
\end{tabular},
\begin{tabular}{|c|c|}
  \hline
   & X\\
  \hline
  X & X \\
  \hline
\end{tabular},
\begin{tabular}{|c|c|}
  \hline
  X & X\\
  \hline
  X &  \\
  \hline
\end{tabular} - это <<уголки>>. \\
\begin{tabular}{|c|c|c|}
  \hline
  X & X & X\\
  \hline
    & X & \\
  \hline
  X & X & \\
  \hline
\end{tabular} - в этой конфигурации три <<уголка>>;
\begin{tabular}{|c|c|c|}
  \hline
  X &  & X\\
  \hline
    & X & \\
  \hline
  X &  & X\\
  \hline
\end{tabular} - а здесь - ни одного <<уголка>>.
}
\solution{ }

\problem{ В урне  $n$  шаров пронумерованных 1,2,... $n$. Наугад
вытаскивают $k$. Найдите ожидание и дисперсию суммы номеров. }
\solution{ }

\problem{ У Пети стопка из $n$ номеров газеты <<Вышка>> лежащих в случайном
порядке. Петя сортирует газеты следующим образом. Он
последовательно просматривает стопку сверху вниз. Если
просматриваемый выпуск более свежий, чем лежащий сверху стопки, то
Петя перекладывает более свежий выпуск наверх стопки и
начинает просматривать стопку заново. \par
Сколько <<переносов>> более свежих номеров наверх в среднем будет
сделано до того момента, когда наверху окажется первый выпуск
газеты? \par
\url{http://www.artofproblemsolving.com/Forum/viewtopic.php?t=124903 } }
\solution{Solution 1: \par
$p_{2}=\frac{1}{2}$ \par
С вероятностью $\frac{n-1}{n}$ сверху стопки лежит номер, меньший
$n$, в этом случае можно считать, что $n$-ый номер вообще
отсутствует в стопке. \par
С вероятностью $\frac{1}{n}$ сверху стопки лежит $n$-ый номер,
тогда обязательно происходит одно перекладывание, после которого
мысленно выкинув $n$-ый номер можно считать, что имеется случайно
упорядоченная стопка из $(n-1)$ выпуска.\par
$p_{n}=\frac{n-1}{n}p_{n-1}+\frac{1}{n}(p_{n-1}+1)$ \par
Итого: $p_{n}=\sum_{i=2}^{n}\frac{1}{i}\approx n\ln(n)$ \par
Solution 2: \par
Пусть $q_{i}$- вероятность того, что число $i$ <<уберут>> с верха стопки.\par
$q_{1}=0$ \par
Вероятность того, что число $i$ <<уберут>> с верха стопки равна
вероятности того, что среди чисел $1$, $2$,... $i$ число $i$ будет
первым, т.е. $\frac{1}{i}$. \par
$\E(X)=\E(X_{2})+...+\E(X_{n})=\sum_{i=2}^{n}\frac{1}{i}\approx
n\ln(n)$  }




\subsection{Первый шаг}


\problem{Илье Муромцу предстоит дорога к камню. И от камня начинаются ещё три дороги. Каждая из тех дорог снова оканчивается камнем. И от каждого камня начинаются ещё три дороги. И каждые те три дороги оканчиваются камнем\ldots И так далее до бесконечности. На каждой дороге можно встретить живущего на ней трёхголового Змея Горыныча с вероятностью (хм, Вы не поверите!) одна третья. Какова вероятность того, что у Ильи Муромца существует возможность пройти свой бесконечный жизненный путь, так ни разу и не встретив Змея Горыныча?}
\solution{$p=\frac{2}{3}(1-(1-p)^{3})$, нам подходит решение $ p=\frac{3-\sqrt{3}}{2} $. }


\problem{У Пети "--- монетка, выпадающая орлом с вероятностью $ p\in (0;1) $. У Васи "--- с вероятностью $ q\in (0;1) $. Они одновременно подбрасывают свои монетки до тех пор, пока у них не окажется набранным одинаковое количество орлов. В частности, они останавливаются после первого подбрасывания, если оно дало одинаковые результаты. Сколько в среднем раз им придётся подбросить монетку?}
\solution{}

\todo[inline]{А если 5 мастей подряд вообще нет?}
\problem{Сколько в среднем нужно взять из колоды в 52 карты, чтобы насобирать подряд 5 карт одной масти?

\begin{hint}
Ответ имеет вид произведения дробей очень простого вида.
\end{hint}
}
\solution{Если у нас $m=13$ достоинств и $n=4$ масти, то ответ имеет вид: $mn\prod\limits_{i=1}^{}\frac{in}{in+1}\approx 45{,}3$.}

\problem{Вася прыгает на один метр вперёд с вероятностью $p$ и на два метра вперёд с вероятностью $1-p$. Как только он пересечёт дистанцию в 100~метров, он останавливается. Получается, что он может остановиться на отметке либо в 100~метров, либо в 101~метр. Какова вероятность того, что он остановится ровно на отметке в 100~метров?}
% копия в задачах на остановку мартингала
\solution{ Обозначим за $P_n$ вероятность остановиться ровно на $n$ метрах. Мы ищем $P_{100}$.

\textit{Решение 1.} По методу первого шага:  $P_n=pP_{n-1}+(1-p)P_{n-2}$.

\textit{Решение 2.} Попасть ровно в $n$ можно двумя способами: перелетев $n-1$ или попав в $n-1$ и сделав шаг в один метр. Значит $P_n=(1-P_{n-1})+pP_{n-1}$.

\textit{Решение 3.} Обозначим Васину координату в момент времени $t$ как $X_t$. Можно найти $a$ так, чтобы процесс $Y_t=a^{X_t}$ был мартингалом. Момент остановки $T=\min\{t \min X_t\geq n\}$. Мартингал $Y_{t\wedge T}$ ограничен, теорема Дуба применима. $\E(Y_T)=\E(Y_0)=1$. Получаем уравнение $P_n a^{n}+(1-P_n) a^{n+1}=1$.}

% untyp
\problem{
Испытания по схеме Бернулли проводятся до первого успеха, вероятность успеха в
отдельном испытании равна $p$ \par
а) Чему равно ожидаемое количество испытаний?   \par
б) Чему равно ожидаемое количество неудач? \par
в) Чему равна дисперсия количества неудач? }
\solution{ $\frac{1}{p}$, $\frac{q}{p}$ \par
в) $E(X^{2})=p\cdot 1+q\cdot E((X+1)^{2})$, $Var(X)=\frac{q}{p^2}$ }

% untyp
\problem{ Отрицательное биномиальное \par
Испытания по схеме Бернулли проводятся до $r$-го успеха, вероятность успеха в
отдельном испытании равна $p$ \par
а) Чему равно ожидаемое количество неудач? \par
б) Чему равна дисперсия количества неудач? }
\solution{ (устно, при сделанной предыдущей задаче) $\frac{rq}{p}$, $Var(X)=\frac{rq}{p^2}$ }

% untyp
\problem{
Саша и Маша по очереди подбрасывают кубик. Посуду будет
мыть тот, кто первым выбросит шестерку. Маша бросает первой.
Какова вероятность того, что Маша будет мыть посуду? }
\solution{ }

% untyp
\problem{
Саша и Маша решили, что будут рожать нового ребенка, до тех
пор, пока в их семье не будут дети обоих полов. Каково ожидаемое
количество детей? }
\solution{ }

% untyp
\problem{
Четыре человека играют в игру <<белая ворона платит>>. Они
одновременно подкидывают монетки. Если три монетки выпали одной
стороной, а одна - по-другому, то <<белая ворона>> оплачивает всей
четверке ужин в ресторане. Если <<белая ворона>> не определилась,
то монетки подбрасывают снова. Сколько в среднем нужно
подбрасывания для определения <<белой вороны>>? }
\solution{ }

% untyp
\problem{
Саша и Маша каждую неделю ходят в кино. Саша доволен
фильмом с
вероятностью 1/4, Маша - с вероятностью 1/3. \par
a) Сколько недель в среднем пройдет до тех пор, пока кто-то не
будет доволен? \par
b) Какова вероятность того, что первым будет доволен Саша? \par
c) Сколько недель в среднем пройдет до тех пор, пока каждый не
будет доволен хотя бы одним просмотренным фильмом? }
\solution{ }

% untyp
\problem{
По ответу студента на вопрос преподаватель может сделать
один из трех выводов: ставить зачет, ставить незачет, задать еще
один вопрос. Допустим, что знания студента и характер
преподавателя таковы, что при ответе на отдельный вопрос зачет
получается с вероятностью $p_{1}=3/8$, незачет --- с вероятностью
$p_{2}=1/8$. Преподаватель задает вопросы до тех пор, пока не
определится
оценка. \par
а) Сколько вопросов в среднем будет задано? \par
б) Какова вероятность получения зачета? }
\solution{ }

% untyp
\problem{
Вы играете в следующую игру. Кубик подкидывается неограниченное число раз. Если на кубике выпадает 1, 2 или 3, то соответствующее количество монет добавляется на кон. Если выпадает 4 или 5, то игра оканчивается и Вы получаете сумму, лежащую на кону. Если выпадает 6, то игра оканчивается, а Вы не получаете ничего. \par
а) Чему равен ожидаемый выигрыш в эту игру? \par
б) Изменим условие: если выпадает 5, то набранная сумма сгорает, а игра начинается заново. Чему будет равен ожидаемый выигрыш? }
\solution{
a) $V(x)=\frac{1}{6}(V(x+1)+V(x+2)+V(x+3)+2x+0)$ \par
Ищем линейную $V(x)$, получаем $V(x)=\frac{2}{3}x+\frac{4}{3}$ \par
б) $V(x)=\frac{1}{6}(V(x+1)+V(x+2)+V(x+3)+x+V(0)+0)$ }

% untyp
\problem{ Вася подкидывает кубик. Если выпадает единица, или Вася говорит
<<стоп>>, то игра оканчивается, если нет, то начинается заново.
Васин выигрыш - последнее выпавшее число. Как выглядит оптимальная
стратегия? Как выглядит оптимальная стратегия, если за каждое
подбрасывание Вася платит 35 копеек?\cite{stirzaker:otep}}
\solution{ }

% untyp
\problem{
Саша и Маша подкидывают монетку бесконечное количество раз. Если сначала появится
РОРО, то выигрывает Саша, если сначала появится ОРОО, то - Маша. \par
а) У кого какие шансы выиграть? \par
b) Сколько в среднем времени ждать до появления РОРО? До ОРОО?
с) Сколько в среднем времени ждать до определения победителя? }
\solution{ }

% untyp
\problem{ \label{mishka ishet sir}
Есть три комнаты. В первой из них лежит сыр. Если мышка
попадает в первую комнату, то она находит сыр через одну минуту.
Если мышка попадает во вторую комнату, то она ищет сыр две минуты
и покидает комнату. Если мышка попадает в третью комнату, то она
ищет сыр три минуты и покидает комнату. Покинув комнату, мышка
выходит в коридор и выбирает новую комнату наугад (т.е. может
зайти в одну и ту же). Сейчас мышка в коридоре. Сколько времени ей
в среднем потребуется, чтобы найти сыр? }
\solution{ $m=\frac{1}{3}+\frac{1}{3}(2+m)+\frac{1}{3}(3+m)$, $m=6$ }

% untyp
\problem{
Иська и Еська по очереди подбрасывают два кубика. Иська
бросает первым. Иська выигрывает, если при своем броске получит 6
очков в сумме на двух кубиках. Еська выигрывает, если при своем
броске получит 7 очков в сумме на двух кубиков. Кубики
подбрасываются до
тех пор, пока не определится победитель. \par
а) Верно ли, что события $A=\{$на двух кубиках в сумме выпало
больше 5 очков$\}$ и $B=\{$на одном из кубиков выпала 1$\}$ являются независимыми? \par
б) Какова вероятность того, что Еська выиграет? }
\solution{ }

% untyp
\problem{
Players A and B play a (fair) dice game. <<A>> deposits one coin and
they take turns rolling a single dice, <<B>> rolling first. \par
If <<B>> rolls an even number, he collects a coin from the pot. If
he rolls an odd number, he put a coin (coins with same values
always). If <<A>> (plays and) rolls an even number, he collects a
coin but if he rolls an odd number, he does NOT add a coin. The
game continues until the pot is exhausted. \par
Question: what is the probability that <<A>> wins this game (that
is, exhaust the pot) ? \par
t=138358}
\solution{ }


\problem{
Вам предложена следующая игра. Изначально на кону 0 рублей. Раз за разом подбрасывается правильная монетка. Если она выпадает орлом, то казино добавляет на кон 100 рублей. Если монетка выпадает решкой, то все деньги, лежащие на кону, казино забирает себе, а Вы получаете красную карточку. Игра прекращается либо когда Вы получаете третью красную карточку, либо в любой момент времени до этого по Вашему выбору. Если Вы решили остановить игру до получения трех красных карточек, то Ваш выигрыш равен сумме на кону. При получении третьей красной карточки игра заканчивается и Вы не получаете ничего.
\begin{enumerate}
\item Как выглядит оптимальная стратегия в этой игре?
\item Чему при этом будет равен средний выигрыш?
\end{enumerate}
}
\solution{ }

\problem{ Китайский ресторан \par
Каждый момент времени в китайский ресторан приходит новый посетитель.
Если сейчас в ресторане сидит $n$ человек, а за конкретным столиком сидит $b$ человек, то вероятность того, что новый посетитель присоединится к этому столику равна $\frac{b}{n+\theta}$. С вероятностью $\frac{\theta}{n+\theta}$ посетитель сядет за отдельный столик. \par
Каково ожидаемое число занятых столиков к моменту времени $n$? }
\solution{ }


\problem{ Вася бьет мячом по воротам 100 раз. В первый раз вероятность
попасть равна $frac{1}{2}$, в каждый последующий раз вероятность
попасть увеличивается --- Вася становится метче; при этом разные
удары независимы. Какова вероятность того, что Вася попадет в
ворота четное число раз? }
\solution{$\frac{1}{2}$ }

\problem{ Вася нажимает на пульте телевизора кнопку <<On-Off>> 100 раз
подряд. Пульт старый, поэтому в первый раз кнопка срабатывает с
вероятностью $\frac{1}{2}$, затем вероятность срабатывания падает.
Какова вероятность того, что после всех нажатий телевизор будет
включен, если сейчас он выключен? }
\solution{$\frac{1}{2}$ }


\problem{ Вы в тире, и у Вас 100 патронов. С вероятностью $0.01$ Вы попадает в глаз Усамы Бен Ладена, за что получаете 20 дополнительных патронов, с вероятностью $0.05$ Вы попадаете в нос Усамы Бен Ладена, за что получаете 5 дополнительных патронов. Вы стреляете до тех пор, пока патроны не кончатся. Сколько в среднем Вы сделаете выстрелов?

\url{www.wilmott.com-forum-brainteasers} }
\solution{ Во первых, заметим, что ожидаемое количество выстрелов, если у Вас осталось $n$ патронов имеет вид $E_{n}=k\cdot n$. \par
Во-вторых, получим уравнение на $E_{n}$: \par
$E_{n}=1+E_{n-1}+kE(X)$, где $\E(X)$ - ожидаемый выигрыш патронов от одного выстрела. \par
Находим $k$: $k=\frac{1}{1-\E(X)}$ \par
Ответ задачи: $\frac{100}{1-0.45}$ \par
Solution2: Интуитивно: $100+100\cdot 20\cdot 0.01+100(\cdot 20\cdot)^{2}+...$  }



\problem{
В вершинах треугольника три ежика. С вероятностью $p$ каждый ежик
независимо от других двигается по часовой стрелке, с вероятностью
$(1-p)$ он двигается против часовой стрелки. Сколько в среднем
пройдет времени прежде,
чем они встретятся в одной вершине? \par
При каком $p$ ожидаемое время встречи минимально? }
\solution{
У системы 4 состояния (1-1-1, 1-2-0, 2-1-0, 3-0-0). \par
Пишем три уравнения на ожидаемые времена. \par
Решая находим $E(T)=\frac{3}{p(1-p)}$ \par
Минимум при $p=0.5$ \par
Ответ слишком красивый... красивое решение???? }


\problem{Монетка выпадает орлом с вероятностью $p$. Монетку подбрасывают до тех пор, пока впервые не выпадет орёл. Какова вероятность того, что будет сделано чётное число подбрасываний?}
\solution{$\P(A)=(1-p)/(2-p)$}

\problem{В зрительном зале $n$ мест. Первые $(n-1)$ зрителей расселись наугад. Последний зритель --- зануда и садиться на место строго согласно билету. Если на его месте сидит другой зритель, то этот зритель идёт пересаживаться на своё законное место. И так далее. Сколько в среднем будет пересаживаний? Сколько в среднем будет пересаживаний, если изначально в зале занято $k$ мест?}
\solution{}




\subsection{Аллюзии на принцип Белмана}


\problem{
Начинающая певица дает концерты каждый день. Каждый ее концерт приносит продюсеру 0.75 тысяч евро. После каждого концерта певица может впасть в депрессию с вероятностью 0.5. Самостоятельно выйти из депрессии певица не может. В депрессии она не в состоянии проводить концерты. Помочь ей могут только цветы от продюсера. Если подарить цветы на сумму $0\le x\le 1$ тысяч евро, то она выйдет из депрессии с вероятностью $\sqrt{x}$. Дисконт фактор равен $0.8$. \\
Какова оптимальная стратегия продюсера? }
\solution{
Рассмотрим совершенно конкурентный невольничий рынок начинающих певиц. Певицы в хорошем настроении продаются по $V_{1}$, в депрессии - по $V_{2}$. \\
$V_{1}=0.75+\delta(0.5V_{1}+0.5V_{2})$ \\
$V_{2}=max_{x}{-x+\delta(\sqrt{x}V_{1}+(1-\sqrt{x})V_{2})}$ }

\problem{ Будучи незамужней Маша испытывает отрицательную полезность $-c$ каждый день. Каждый день она знакомится с новым ухажером и может тут же выскочить за него замуж. Каждый ухажер характеризуется параметром $X$, полезностью, которую Маша получит в день свадьбы с ним. Вы о чем подумали? Величина $X$ распределено равномерно на $[0;1]$. Ежедневная полезность Маши от замужнего состояния после дня свадьбы равна 0. Дисконт фактор (с которым дисконтируется Машина полезность) равен $\delta$.
\begin{enumerate}
\item Как выглядит оптимальная стратегия Маши, если она выбирает мужа на всю жизнь?
\item Как выглядит оптимальная стратегия Маши, если она легко может развестись?
\end{enumerate} }
\solution{ }



\subsection{\textit{o}-малое}

\problem{Случайные величины $ X_{1} ,\ldots, X_{n} $ одинаково распределены с функцией плотности $ p(t) $ и независимы. Найдите функцию плотности третьего по величине $ X_{(3)}$.}
\solution{$ \PP(X_{(3)}\in [x;x+dx])= C_{n}^{2}C_{n-2}^{1} \bigl((F(x)+o(x)\bigr)^{n-3}\bigl(1-F(x)+o(x)\bigr)^{2}\bigl((f(x)+o(x)\bigr)dx $. Значит, искомая функция плотности равна $f_{X_{(3)}}(t)=f(t)F(t)^{n-3}\bigl(1-F(t)\bigr)^{2}$.}

\subsection{Вероятностный метод}
% задачи не по теории вероятностей, которые решаются с помощью теории вероятностей

% untyp
\problem{ На потоке 200 студентов. На контрольной было 6 задач. Известно, что каждую задачу решило не менее 120 человек.
Всегда ли преподаватель может выбрать двух студентов из потока так, что эти двое могут решить всю контрольную совместными усилиями?}
\solution{Выберем двух студентов из потока наугад. Вероятность того, что ни один из них не решил задачу \No\,1, не превосходит $\br{\frac{80}{200}}^2=0{,}16$.
Вероятность того, что ни один из них не решил задачу \No\,2, не превосходит 0{,}16 (по тем же причинам), и это справедливо для каждой из шести задач. Вероятность того, что хотя
бы одну задачу они на пару не решили, не превосходит суммы этих вероятностей, т.\,е. $0{,}16\cdot6=0{,}96$.
Значит, вероятность выбора пары студентов, которые совместными усилиями могут решить экзамен, не менее $0{,}04$. Значит, хотя бы одна такая пара существует.}


\subsection{Склеивание отрезка}


\problem{Машина может сломаться равновероятно в любой точке на дороге от города А до города Б. Когда машина сломается мы будем толкать ее до ближайшего сервиса. Где должны быть расположены три автосервиса чтобы минимизировать ожидаемую продолжительность толкания? А если автосервисов будет $n$?}
\solution{Проверить. Разбиваем отрезок на $n$ частей, ставим автосервис в центр каждой части
\url{http://math.stackexchange.com/questions/37254/} }



\problem{ Рулет \\
Длинный рулет разрезан на $n$ частей. Каждый из $k$ гостей по очереди забирает себе один кусочек, выбираемый случайным образом. В результате остается $n-k$ кусочков рулета. Оставшиеся кусочки рулета лежат <<сериями>>, разделенными <<дырками>> от забранных кусочков. Каково ожидаемое число <<серий>> оставшихся кусочков? К чему стремится эта величина при $n\to\infty$?\\
Aвтор: Алексей Суздальцев  }
\solution{ Решение 1: \\
Величина $X$ --- число <<серий>>, $X=X_{1}+...+X_{n}$, где $X_{i}$ - индикатор, показывающий, начинается ли новая серия с $i$-го кусочка. \\
Ответ: $\frac{n-k}{n}+(n-1)\frac{k}{n}\frac{n-k}{n-1}=(k+1)\frac{n-k}{n}$ \\
Решение 2: \\
Закольцуем рулет, добавив в него еще один кусочек, для хозяина дома --- для Алексея Суздальцева. Получаем $(k+1)$ потенциальную серию. Вероятность того, что некая серия непуста, равна $\frac{n-k}{n}$. Итого, $(k+1)\frac{n-k}{n}$ }

\problem{
Петя ищет 6 нужных ему книг в стопке из 30 книг. Книги внешне не отличимы. Сколько книг в среднем ему придется пересмотреть? Просмотренные книги Петя в общую кучу не возвращает.  }
\solution{Можно считать, что Петя берет книги подряд из хорошо перетасованной стопки. Соответственно он берет 6 книг и 6 интервалов (книги до 1-ой нужной, книги от 1-ой нужной до 2-ой нужной, и т.д.). Если считать, что средняя длина всех интервалов одинаковая, то получается такой ответ: $6+6\cdot\frac{24}{7}$. \\
Доказательство того, что средняя длина всех интервалов одинаковая: \\
Расположим 30 книг по кругу. Среди этих 30 книг отметим случайным образом 7 книг и случайным образом занумеруем их от 1 до 7. Эти 7 книг разбивают круг из книг на 7 частей. В силу симметрии средняя длина каждой части одинакова и равна $\frac{24}{7}$. Будем трактовать книгу номер 1 как разбивающую круг на стопку. А книги 2-7, как нужные Пети. }



\subsection{Мартингальный метод}

Справедливую игру стратегией не испортишь!

\problem{ У Васи $100$ рублей, у Пети --- $150$. Они играют в орлянку
правильной монеткой до тех пор, пока все деньги не перейдут к
одному игроку. Какова вероятность, что победит Вася? }
\solution{Пусть $X_{n}$ - благосостояние Васи после $n$-го хода, тогда
$\E(X_{n})=100$. $\E(X_{final})=250p+0(1-p)$.  }
