% !Mode:: "TeX:UTF-8"
\section{Компьютерные (use R or Python!)}

\subsection{Нахождение сложных сумм/поиск оптимальных стратегий}
\problem{Перед нами 10 коробок. Изначально в 1-й коробке 1 шар, во 2-й "--- 2 шара и т.\,д. Мы равновероятно выбираем одну из коробок, вытаскиваем из неё шар и кладём его равновероятно в одну из девяти оставшихся коробок. Мы повторяем это перекладывание до тех пор, пока одна из коробок не станет пустой. Пусть $N$ "--- число перекладываний. С помощью компьютера оцените $\E(N) $, $\Var(N)$.}
\solution{ $ \E(N)\approx 12{,}15$.}

\problem{В классе учатся $n$ человек. Нас интересует вероятность того, что хотя бы у двух из них дни рождения будут в соседние дни. (31 декабря и 1 января будем считать соседними). При каком $n$ эта вероятность впервые достигнет 0{,}5?}
\solution{ $\PP_{16}=0{,}482\,390\,182$, $\PP_{17}=0{,}525\,836\,596$.}

\problem{В классе 30 человек. Какова вероятность того, что есть три человека, у которых совпадают дни рождения? Найдите ответ с помощью симуляций и с помощью пуассоновского приближения. При каком количестве человек эта вероятность впервые превысит 50\,\%?}
\solution{Симуляции $p=0{,}028\,5 $, Пуассон: $p=0{,}03$.}

\problem{Сколько нужно людей, чтобы вероятность того, что в каждый день года у кого-то день рождения, впервые превысила 50\,\%?}
\solution{$\PP(T\leq k)= n^{-k}n!\left\{ \begin{array}{c} k \\ n \end{array} \right\}$ (число Стирлинга второго рода); 2287.}

\problem{Сколько в среднем нужно взять из колоды в 52 карты, чтобы насобирать подряд 5 карт одной масти? Не обязательно одной масти?}
\solution{Если у нас $m=13$ достоинств и $n=4$ масти, то ответ имеет вид: $mn\prod\limits_{i=1}^{}\frac{in}{in+1}\approx 45{,}3$; $\approx 28{,}0$.}

% untyp
\problem{Маленький мальчик торгует на улице еженедельной газетой. Покупает
он ее по 15 рублей, а продает по 30 рублей. Количество потенциальных покупателей --- случайная величина с распределением Пуассона и средним
значением равным 50. Нераспроданные газеты ничего не стоят. Пусть $n$ --- количество газет, максимизирующее ожидаемую прибыль мальчика.
\begin{enumerate}
\item Чему примерно должно быть равно значение функции распределения в
точке  $n$?
\item  С помощью компьютера найдите  $n$ и ожидаемую прибыль.
\end{enumerate}}
\solution{$n=50$, $665.51$

\inputminted{python}{src_python/newspapers_notext.py}
}


\subsection{Проведение симуляций}



% (к статистике) проверка простых гипотез на РЕАЛЬНЫХ (исторических) данных

\subsection{Statistics}
\problem{Петя подбрасывал две монетки неправильные монетки. Результаты подбрасывания:

Число подбрасываний первой. Число подбрасываний второй. Общее число орлов.
... ... ...

... ... ...
... ... ...


Оцените с помощью компьютера вероятности выпадения орлом для каждой монетки. Постройте доверительные интервалы.
(Красивого решения в явном виде нет).

Можно использовать нормальное приближение


}
\solution{}


\problem{Голосовать можно за трёх кандидатов: А, Б и В. Из 100 опрошенных 20 хотят голосовать за А, 40 "--- за Б, остальные "--- за В. В осях $p_{A}$, $p_{B}$ постройте 90-процентную доверительную двумерную область.}
\solution{ }



