% !Mode:: "TeX:UTF-8"
\documentclass[pdftex,11pt,a4paper]{article}

%%%%%%%%%%%%%%%%%%%%%%%  Загрузка пакетов  %%%%%%%%%%%%%%%%%%%%%%%%%%%%%%%%%%

%\usepackage{showkeys} % показывать метки в готовом pdf 

\usepackage{etex} % расширение классического tex
% в частности позволяет подгружать гораздо больше пакетов, чем мы и займёмся далее

\usepackage{cmap} % для поиска русских слов в pdf
\usepackage{verbatim} % для многострочных комментариев
\usepackage{makeidx} % для создания предметных указателей
\usepackage[X2,T2A]{fontenc}
\usepackage[utf8]{inputenc} % задание utf8 кодировки исходного tex файла
\usepackage{setspace}
\usepackage{amsmath,amsfonts,amssymb,amsthm}
\usepackage{mathrsfs} % sudo yum install texlive-rsfs
\usepackage{dsfont} % sudo yum install texlive-doublestroke
\usepackage{array,multicol,multirow,bigstrut} % sudo yum install texlive-multirow
\usepackage{indentfirst} % установка отступа в первом абзаце главы
\usepackage[british,russian]{babel} % выбор языка для документа
\usepackage{bm}
\usepackage{bbm} % шрифт с двойными буквами
\usepackage[perpage]{footmisc}

% создание гиперссылок в pdf
\usepackage[pdftex,unicode,colorlinks=true,urlcolor=blue,hyperindex,breaklinks]{hyperref} 

% свешиваем пунктуацию 
% теперь знаки пунктуации могут вылезать за правую границу текста, при этом текст выглядит ровнее
\usepackage{microtype}

\usepackage{textcomp}  % Чтобы в формулах можно было русские буквы писать через \text{}

% размер листа бумаги
\usepackage[paper=a4paper,top=13.5mm, bottom=13.5mm,left=16.5mm,right=13.5mm,includefoot]{geometry}

\usepackage{xcolor}

\usepackage[pdftex]{graphicx} % для вставки графики 

\usepackage{float,longtable}
\usepackage{soulutf8}

\usepackage{enumitem} % дополнительные плюшки для списков
%  например \begin{enumerate}[resume] позволяет продолжить нумерацию в новом списке

\usepackage{mathtools}
\usepackage{cancel,xspace} % sudo yum install texlive-cancel

\usepackage{minted} % display program code with syntax highlighting
% требует установки pygments и python 

\usepackage{numprint} % sudo yum install texlive-numprint
\npthousandsep{,}\npthousandthpartsep{}\npdecimalsign{.}

\usepackage{embedfile} % Чтобы код LaTeXа включился как приложение в PDF-файл

\usepackage{subfigure} % для создания нескольких рисунков внутри одного

\usepackage{tikz,pgfplots} % язык для рисования графики из latex'a
\usetikzlibrary{trees} % tikz-прибамбас для рисовки деревьев
\usepackage{tikz-qtree} % альтернативный tikz-прибамбас для рисовки деревьев
\usetikzlibrary{arrows} % tikz-прибамбас для рисовки стрелочек подлиннее

\usepackage{todonotes} % для вставки в документ заметок о том, что осталось сделать
% \todo{Здесь надо коэффициенты исправить}
% \missingfigure{Здесь будет Последний день Помпеи}
% \listoftodos --- печатает все поставленные \todo'шки


% более красивые таблицы
\usepackage{booktabs}
% заповеди из докупентации: 
% 1. Не используйте вертикальные линни
% 2. Не используйте двойные линии
% 3. Единицы измерения - в шапку таблицы
% 4. Не сокращайте .1 вместо 0.1
% 5. Повторяющееся значение повторяйте, а не говорите "то же"



%\usepackage{asymptote} % пакет для рисовки графики, должен идти после graphics
% но мы переходим на tikz :)

%\usepackage{sagetex} % для интеграции с Sage (вероятно тоже должен идти после graphics)

% metapost создает упрощенные eps файлы, которые можно напрямую включать в pdf 
% эта группа команд декларирует, что файлы будут этого упрощенного формата
% если metapost не используется, то этот блок не нужен
\usepackage{ifpdf} % для определения, запускается ли pdflatex или просто латех
\ifpdf
	\DeclareGraphicsRule{*}{mps}{*}{}
\fi
%%%%%%%%%%%%%%%%%%%%%%%%%%%%%%%%%%%%%%%%%%%%%%%%%%%%%%%%%%%%%%%%%%%%%%


%%%%%%%%%%%%%%%%%%%%%%%  Внедрение tex исходников в pdf файл  %%%%%%%%%%%%%%%%%%%%%%%%%%%%%%%%%%
\embedfile[desc={Main tex file}]{\jobname.tex} % Включение кода в выходной файл
\embedfile[desc={title_bor}]{/home/boris/science/tex_general/title_bor_utf8.tex}

%%%%%%%%%%%%%%%%%%%%%%%%%%%%%%%%%%%%%%%%%%%%%%%%%%%%%%%%%%%%%%%%%%%%%%



%%%%%%%%%%%%%%%%%%%%%%%  ПАРАМЕТРЫ  %%%%%%%%%%%%%%%%%%%%%%%%%%%%%%%%%%
\setstretch{1}                          % Межстрочный интервал
\flushbottom                            % Эта команда заставляет LaTeX чуть растягивать строки, чтобы получить идеально прямоугольную страницу
\righthyphenmin=2                       % Разрешение переноса двух и более символов
\pagestyle{plain}                       % Нумерация страниц снизу по центру.
\widowpenalty=300                     % Небольшое наказание за вдовствующую строку (одна строка абзаца на этой странице, остальное --- на следующей)
\clubpenalty=3000                     % Приличное наказание за сиротствующую строку (омерзительно висящая одинокая строка в начале страницы)
\setlength{\parindent}{1.5em}           % Красная строка.
%\captiondelim{. }
\setlength{\topsep}{0pt}
%%%%%%%%%%%%%%%%%%%%%%%%%%%%%%%%%%%%%%%%%%%%%%%%%%%%%%%%%%%%%%%%%%%%%%



%%%%%%%% Это окружение, которое выравнивает по центру без отступа, как у простого center
\newenvironment{center*}{%
  \setlength\topsep{0pt}
  \setlength\parskip{0pt}
  \begin{center}
}{%
  \end{center}
}
%%%%%%%%%%%%%%%%%%%%%%%%%%%%%%%%%%%%%%%%%%%%%%%%%%%%%%%%%%%%%%%%%%%%%%


%%%%%%%%%%%%%%%%%%%%%%%%%%% Правила переноса  слов
\hyphenation{ }
%%%%%%%%%%%%%%%%%%%%%%%%%%%%%%%%%%%%%%%%%%%%%%%%%%%%%%%%%%%%%%%%%%%%%%

\emergencystretch=2em


% DEFS
\def \mbf{\mathbf}
\def \msf{\mathsf}
\def \mbb{\mathbb}
\def \tbf{\textbf}
\def \tsf{\textsf}
\def \ttt{\texttt}
\def \tbb{\textbb}

\def \wh{\widehat}
\def \wt{\widetilde}
\def \ni{\noindent}
\def \ol{\overline}
\def \cd{\cdot}
\def \fr{\frac}
\def \bs{\backslash}
\def \lims{\limits}
\DeclareMathOperator{\dist}{dist}
\DeclareMathOperator{\VC}{VCdim}
\DeclareMathOperator{\card}{card}
\DeclareMathOperator{\sign}{sign}
\DeclareMathOperator{\sgn}{sign}
\DeclareMathOperator{\Tr}{\mbf{Tr}}
\def \xfs{(x_1,\ldots,x_{n-1})}
\DeclareMathOperator*{\argmin}{arg\,min}
\DeclareMathOperator*{\amn}{arg\,min}
\DeclareMathOperator*{\amx}{arg\,max}

\DeclareMathOperator{\Corr}{Corr}
\DeclareMathOperator{\Cov}{Cov}
\DeclareMathOperator{\Var}{Var}
\DeclareMathOperator{\corr}{Corr}
\DeclareMathOperator{\cov}{Cov}
\DeclareMathOperator{\var}{Var}
\DeclareMathOperator{\bin}{Bin}
\DeclareMathOperator{\Bin}{Bin}
\DeclareMathOperator{\rang}{rang}
\DeclareMathOperator*{\plim}{plim}
\DeclareMathOperator{\med}{med}


\providecommand{\iff}{\Leftrightarrow}
\providecommand{\hence}{\Rightarrow}

\def \ti{\tilde}
\def \wti{\widetilde}

\def \mA{\mathcal{A}}
\def \mB{\mathcal{B}}
\def \mC{\mathcal{C}}
\def \mE{\mathcal{E}}
\def \mF{\mathcal{F}}
\def \mH{\mathcal{H}}
\def \mL{\mathcal{L}}
\def \mN{\mathcal{N}}
\def \mU{\mathcal{U}}
\def \mV{\mathcal{V}}
\def \mW{\mathcal{W}}


\def \R{\mbb R}
\def \N{\mbb N}
\def \Z{\mbb Z}
\def \P{\mbb{P}}
\def \p{\mbb{P}}
\newcommand{\E}{\mathbb{E}}
\def \D{\msf{D}}
\def \I{\mbf{I}}

\def \a{\alpha}
\def \b{\beta}
\def \t{\tau}
\def \dt{\delta}
\newcommand{\e}{\varepsilon}
\def \ga{\gamma}
\def \kp{\varkappa}
\def \la{\lambda}
\def \sg{\sigma}
\def \sgm{\sigma}
\def \tt{\theta}
\def \ve{\varepsilon}
\def \Dt{\Delta}
\def \La{\Lambda}
\def \Sgm{\Sigma}
\def \Sg{\Sigma}
\def \Tt{\Theta}
\def \Om{\Omega}
\def \om{\omega}

%\newcommand{\p}{\partial}
\newcommand{\PP}{\mathbb{P}}

\def \ni{\noindent}
\def \lq{\glqq}
\def \rq{\grqq}
\def \lbr{\linebreak}
\def \vsi{\vspace{0.1cm}}
\def \vsii{\vspace{0.2cm}}
\def \vsiii{\vspace{0.3cm}}
\def \vsiv{\vspace{0.4cm}}
\def \vsv{\vspace{0.5cm}}
\def \vsvi{\vspace{0.6cm}}
\def \vsvii{\vspace{0.7cm}}
\def \vsviii{\vspace{0.8cm}}
\def \vsix{\vspace{0.9cm}}
\def \VSI{\vspace{1cm}}
\def \VSII{\vspace{2cm}}
\def \VSIII{\vspace{3cm}}

\newcommand{\bls}[1]{\boldsymbol{#1}}
\newcommand{\bsA}{\boldsymbol{A}}
\newcommand{\bsH}{\boldsymbol{H}}
\newcommand{\bsI}{\boldsymbol{I}}
\newcommand{\bsP}{\boldsymbol{P}}
\newcommand{\bsR}{\boldsymbol{R}}
\newcommand{\bsS}{\boldsymbol{S}}
\newcommand{\bsX}{\boldsymbol{X}}
\newcommand{\bsY}{\boldsymbol{Y}}
\newcommand{\bsZ}{\boldsymbol{Z}}
\newcommand{\bse}{\boldsymbol{e}}
\newcommand{\bsq}{\boldsymbol{q}}
\newcommand{\bsy}{\boldsymbol{y}}
\newcommand{\bsbeta}{\boldsymbol{\beta}}
\newcommand{\fish}{\mathrm{F}}
\newcommand{\Fish}{\mathrm{F}}
\renewcommand{\phi}{\varphi}
\newcommand{\ind}{\mathds{1}}
\newcommand{\inds}[1]{\mathds{1}_{\{#1\}}}
\renewcommand{\to}{\rightarrow}
\newcommand{\sumin}{\sum\limits_{i=1}^n}
\newcommand{\ofbr}[1]{\bigl( \{ #1 \} \bigr)}     % Например, вероятность события. Большие круглые, нормальные фигурные скобки вокруг аргумента
\newcommand{\Ofbr}[1]{\Bigl( \bigl\{ #1 \bigr\} \Bigr)} % Например, вероятность события. Больше больших круглые, большие фигурные скобки вокруг аргумента
\newcommand{\oeq}{{}\textcircled{\raisebox{-0.4pt}{{}={}}}{}} % Равно в кружке
\newcommand{\og}{\textcircled{\raisebox{-0.4pt}{>}}}  % Знак больше в кружке

% вместо горизонтальной делаем косую черточку в нестрогих неравенствах
\renewcommand{\le}{\leqslant}
\renewcommand{\ge}{\geqslant}
\renewcommand{\leq}{\leqslant}
\renewcommand{\geq}{\geqslant}


\newcommand{\figb}[1]{\bigl\{ #1  \bigr\}} % большие фигурные скобки вокруг аргумента
\newcommand{\figB}[1]{\Bigl\{ #1  \Bigr\}} % Больше больших фигурные скобки вокруг аргумента
\newcommand{\parb}[1]{\bigl( #1  \bigr)}   % большие скобки вокруг аргумента
\newcommand{\parB}[1]{\Bigl( #1  \Bigr)}   % Больше больших круглые скобки вокруг аргумента
\newcommand{\parbb}[1]{\biggl( #1  \biggr)} % большие-большие круглые скобки вокруг аргумента
\newcommand{\br}[1]{\left( #1  \right)}    % круглые скобки, подгоняемые по размеру аргумента
\newcommand{\fbr}[1]{\left\{ #1  \right\}} % фигурные скобки, подгоняемые по размеру аргумента
\newcommand{\eqdef}{\mathrel{\stackrel{\rm def}=}} % знак равно по определению
\newcommand{\const}{\mathrm{const}}        % const прямым начертанием
\newcommand{\zdt}[1]{\textit{#1}}
\newcommand{\ENG}[1]{\foreignlanguage{british}{#1}}
\newcommand{\ENGs}{\selectlanguage{british}}
\newcommand{\RUSs}{\selectlanguage{russian}}
\newcommand{\iid}{\text{i.\hspace{1pt}i.\hspace{1pt}d.}}

\newdimen\theoremskip
\theoremskip=0pt
\newenvironment{note}{\par\vskip\theoremskip\textbf{Замечание.\xspace}}{\par\vskip\theoremskip}
\newenvironment{hint}{\par\vskip\theoremskip\textbf{Подсказка.\xspace}}{\par\vskip\theoremskip}
\newenvironment{ist}{\par\vskip\theoremskip Источник:\xspace}{\par\vskip\theoremskip}

\newcommand*{\tabvrulel}[1]{\multicolumn{1}{|c}{#1}}
\newcommand*{\tabvruler}[1]{\multicolumn{1}{c|}{#1}}

\newcommand{\II}{{\fontencoding{X2}\selectfont\CYRII}}   % I десятеричное (английская i неуместна)
\newcommand{\ii}{{\fontencoding{X2}\selectfont\cyrii}}   % i десятеричное
\newcommand{\EE}{{\fontencoding{X2}\selectfont\CYRYAT}}  % ЯТЬ
\newcommand{\ee}{{\fontencoding{X2}\selectfont\cyryat}}  % ять
\newcommand{\FF}{{\fontencoding{X2}\selectfont\CYROTLD}} % ФИТА
\newcommand{\ff}{{\fontencoding{X2}\selectfont\cyrotld}} % фита
\newcommand{\YY}{{\fontencoding{X2}\selectfont\CYRIZH}}  % ИЖИЦА
\newcommand{\yy}{{\fontencoding{X2}\selectfont\cyrizh}}  % ижица

%%%%%%%%%%%%%%%%%%%%% Определение разрядки разреженного текста и задание красивых многоточий
\sodef\so{}{.15em}{1em plus1em}{.3em plus.05em minus.05em}
\newcommand{\ldotst}{\so{...}}
\newcommand{\ldotsq}{\so{?\hbox{\hspace{-0.61ex}}..}}
\newcommand{\ldotse}{\so{!..}}
%%%%%%%%%%%%%%%%%%%%%%%%%%%%%%%%%%%%%%%%%%%%%%%%%%%%%%%%%%%%%%%%%%%%%%

%%%%%%%%%%%%%%%%%%%%%%%%%%%%% Команда для переноса символов бинарных операций
\def\hm#1{#1\nobreak\discretionary{}{\hbox{$#1$}}{}}
%%%%%%%%%%%%%%%%%%%%%%%%%%%%%%%%%%%%%%%%%%%%%%%%%%%%%%%%%%%%%%%%%%%%%%

\setlist[enumerate,1]{label=\arabic*., ref=\arabic*, partopsep=0pt plus 2pt, topsep=0pt plus 1.5pt,itemsep=0pt plus .5pt,parsep=0pt plus .5pt}
\setlist[itemize,1]{partopsep=0pt plus 2pt, topsep=0pt plus 1.5pt,itemsep=0pt plus .5pt,parsep=0pt plus .5pt}

% Эти парни затем, если вдруг не захочется управлять списками из-под уютненького enumitem
% или если будет жизненно важно, чтобы в списках были именно русские буквы.
%\setlength{\partopsep}{0pt plus 3pt}
%\setlength{\topsep}{0pt plus 2pt}
%\setlength{\itemsep}{0 plus 1pt}
%\setlength{\parsep}{0 plus 1pt}

%на всякий случай пока есть
%теоремы без нумерации и имени
%\newtheorem*{theor}{Теорема}

%"Определения","Замечания"
%и "Гипотезы" не нумеруются
%\newtheorem*{defin}{Определение}
%\newtheorem*{rem}{Замечание}
%\newtheorem*{conj}{Гипотеза}

%"Теоремы" и "Леммы" нумеруются
%по главам и согласованно м/у собой
%\newtheorem{theorem}{Теорема}
%\newtheorem{lemma}[theorem]{Лемма}

% Утверждения нумеруются по главам
% независимо от Лемм и Теорем
%\newtheorem{prop}{Утверждение}
%\newtheorem{cor}{Следствие} 

% специальная штука под задачник
% создает команды:
% \problem{ текст задачи }
% \solution{ текст решения }
% \problemonly  - после этой команды будут печататься только \problem{} и \problemtext{}
% \solutiononly - после этой команды будут печататься только \solution{} и \solutiontext{}
% \problemandsolution - после этой команды печатается все
% \secsolution - задает новую (виртуальную) секцию для решений

% может потребоваться %\addtocounter{secsolution}{число глав без задач решений, не прогнанных через problemonly}

% как работать
% файл с решениями отдельной главы должен выглядеть так:
% \problem{ dddd} \solution{ddddddd}
% \problem{df sldk} \solution{ dfssd}

% главный файл может выглядеть двумя способами:

% Способ 1. (для решений контрольной, рядом задачи и ответы)
% \problemandsolution
% \input{file with problems}

% Способ 2. (для задачника, сначала все задачи, затем все ответы)
% \problemonly
% \input{file with problems}

% \solutiononly
% \input{file with problems}

% ААААААААААААААААААААААА надо делать!!!!
% Способ 3. - основной (вариант способа 2)

% \problemonly2
% \input{file with problems}

%\solutiononly - эта команда сама сделает все!




% файл с задачами:
% \section{Первая}
% \problem{ dddd} \solution{ddddddd}
% \problem{df sldk} \solution{ dfssd}
% \problemtext{Этот текст не будет напечатан после solutiononly}
% \section{Вторая}
% \problem{ ааа} \solution{dыва}
% \problem{ыавыв} \solution{ ыва}
% \solutiontext{Этот текст не будет напечатан после problemonly}



% начало кода:

\let\oldsection\section % сохраняем команду \section, т.к. мы ее переопределим
\let\oldsubsection\subsection % сохраняем команду \subsection, т.к. мы ее переопределим

\newcommand{\restoresection}{ % команда для восстановления \section \subsection
\renewcommand{\section}[1]{\oldsection{##1}}
\renewcommand{\subsection}[1]{\oldsubsection{##1}}
}

\newcounter{problem}[section]
%создаем новый счетчик "problem",
% будет автоматом сбрасываться на 0 при старте нового раздела
% при создании счетчик сам встанет на 0

\newcounter{secsolution}
% - это номер секции решаемой задачи (поскольку решения идут в одной секции, то номер секции надо менять в ручную)
\newcounter{solution}[secsolution]
% - это номер решаемой задачи, сам сбрасывается при увеличении secsolution на 1



\renewcommand{\thesecsolution}{\arabic{secsolution}}
% команда \thesecsolution просто выводит номер secsolution

\newcommand{\newsecsolution}{
\stepcounter{secsolution} % без создания ссылки увеличит secsolution на 1 со сбросом подчиненного счетчика
}
% команда \newsecsolution увеличит номер секции на 1 и установит номер решения внутри секции равным 0

\renewcommand{\theproblem}{\thesection.\arabic{problem}.}
\renewcommand{\thesolution}{\thesecsolution.\arabic{solution}.}
% обновляем команду \theproblem - она должна выводить номер секции и номер задачи внутри секции
% почему обновляем? - потому, что она создалась при создании счетчика problem

\newcommand{\problem}[1]{}
\newcommand{\solution}[1]{}
% создаем команды \problem, \solution с одним аргументом, которые ничего не делает
% ниже они будут переопределены


\newcommand{\problemtext}[1]{}
\newcommand{\solutiontext}[1]{}
% эти две команды будут выводить текст, заложенный внутри них только внутри соответствующей секции, в другой - ничего не будет делать
% в отличие от этой команды \problem \solution делают ссылки, слово "задача" и пр.


\newcommand{\problemonly}{
% эта команда переопределяет команду \problem

\setcounter{problem}{0}
\setcounter{solution}{0}
\setcounter{secsolution}{0}

\renewcommand{\problemtext}[1]{##1}
\renewcommand{\solutiontext}[1]{}


\renewcommand{\section}[1]{\oldsection{##1}\newsecsolution}
\renewcommand{\subsection}[1]{\oldsubsection{##1}}


\renewcommand{\problem}[1]{
\refstepcounter{problem}
% \phantomsection % создаем точку привязки для команды \label % не нужна, т.к. есть refstepcounter
\vspace{0.5ex plus 0.2ex minus 0.2ex}
Задача
\hyperref[s\theproblem]{\theproblem} % гиперссылка на метку "s1.1."
\label{p\theproblem} % метка "p1.1."
\par ##1}
\renewcommand{\solution}[1]{}
}

\newcommand{\solutiononly}{
% эта команда переопределяет команду \solution

\setcounter{problem}{0}
\setcounter{solution}{0}
\setcounter{secsolution}{0}

\renewcommand{\problemtext}[1]{}
\renewcommand{\solutiontext}[1]{##1}

\renewcommand{\section}[1]{\newsecsolution} % можно сюда чего-то добавить, чтобы решения отделялись как-то по секциям
\renewcommand{\subsection}[1]{}

\renewcommand{\problem}[1]{}
\renewcommand{\solution}[1]{
\refstepcounter{solution}
% \phantomsection
\hyperref[p\thesolution]{\thesolution} \label{s\thesolution}
##1}}



\newcommand{\problemandsolution}{
% эта команда переопределяет команды \solution, \problem

\setcounter{problem}{0}
\setcounter{solution}{0}
\setcounter{secsolution}{0}

\renewcommand{\problemtext}[1]{##1}
\renewcommand{\solutiontext}[1]{##1}

\renewcommand{\section}[1]{\oldsection{##1}\newsecsolution}
\renewcommand{\subsection}[1]{\oldsubsection{##1}}

\renewcommand{\problem}[1]{
\refstepcounter{problem}
% \phantomsection % создаем точку привязки для команды \label
Задача
\hyperref[s\theproblem]{\theproblem} % гиперссылка на метку "s1.1."
\label{p\theproblem} % метка "p1.1."
\par ##1}
\renewcommand{\solution}[1]{
\refstepcounter{solution}
% \phantomsection
\hyperref[p\thesolution]{\thesolution} \label{s\thesolution}
##1}
}




\title{Задачи по элементарной теории вероятностей и
матстатистике}
\author{Составитель: Борис Демешев, \\
\href{mailto:boris.demeshev@gmail.com}{boris.demeshev@gmail.com} }
\date{\today}

\newcommand{\teq}[1]{\stackrel{#1}{=}} % для задачи про одинаковый закон распределения от Алексея Суздальцева

\newcommand{\source}[1]{\par Источник: #1. \par}
\newcommand{\cat}[1]{}
\renewcommand{\labelenumi}{\arabic{enumi})}
\begin{document}

%\pagestyle{myheadings} \markboth{ТВИМС-задачник. Демешев Борис. boris.demeshev@gmail.com }{ТВИМС-задачник. Демешев Борис. boris.demeshev@gmail.com }

\maketitle
\tableofcontents{}

\section*{Предисловие} % подумать над номерами секций/задач (с 1, или с 2)
% !Mode:: "TeX:UTF-8"
%Идеи:
%1. Междисциплинарное взаимодействие \\
%1.1. включить вопросы на тему простых преобразований в известных моделях \\
%CAPM, rational expectation
%1.2. включить что-то на тему инструментальных переменных
%Задана (табличка-сделано!, ф. плотности-?? как бы по проще-то...???) для $X,u$ %придумайте переменную $Z$, которая была бы коррелирована с $X$, но не с $u$
%1.3. calculate partial correlation


%ver 23.09.10, добавлены задачи от Алексея Суздальцева
%ver 11.05.08, задачи про бассейн, оценки для геометрического распределения
%ver 12.03.08, корректоры очепяток, конь на шахматной доске \\
%ver 26.02.08, задача по статистике
%ver 26.04.07, about 500 problems \\
%ver 12.05.07, Расстояние между минимумом и максимумом \\
%Спелестолог и батарейки \\
%ver 18.05.07, minor correction, more small problems \\
%ver 20.05.07, Ковбои \\
%ver 29.05.07, Дополнительные патроны, Четыре шкатулки \\
%ver 10.06.07, Новый шаман \\
%ver 14.07.07, добавляются задачи из wilmott bt, стр 13 \\
%ver 16.10.07, Китайский ресторан

\subsection{Обращение к читателю}
Задачник находится в стадии разработки. Смелее спрашивайте и высказывайте
своё мнение Борису Демешеву, \href{mailto:boris.demeshev@gmail.com}{boris.demeshev@gmail.com}

Предлагайте свои задачи!


\subsection{Цель}
Есть много сборников задач. Зачем этот:

"--* Открытость и доступность. \url{http://demeshev.wordpress.com/materials/}

"--* Красивые задачи.

"--* Задачи под курс НИУ ВШЭ.

Всё то, что можно рассказать без теории меры.


\subsection{Об ответах и упрощениях}
Большинство ответов имеет совсем простой вид в духе
$\frac{a}{a+b}$, и их, очевидно, нельзя упростить. Некоторые ответы
простым образом выражаются через биномиальные коэффициенты. Не
упрощаются, но встречаются в ответах: $\sum_{i=1}^{n}\frac{1}{i}$,
$\sum_{i=1}^{n} i^{k}$. Ответы в виде громоздких сумм биномиальных
коэффициентов не используются, если это не оговорено в условии.
Используется сумма геометрической прогрессии, разложение в ряд
Тейлора для $e^{x}$.


%\textbf{Комменты прочие} \\
%Как пишется Еська-Иська?\\
%Забавные комментарии и рисунки \\
%Отработать классификацию \\
%Как быть если разные пункты задачи относятся к разным разделам? \\
%Открытые вопросы \\
%Ответы \\
%Как называются функции в английском excel? \\
%Разобраться с дефисом и тире\\
%Разобраться с Рыцари близнецы \\
%Отделенные точки и запятые от предыдущего текста \\
%Спросить про копирайт чужих задач \\
%Разбить в задачах с табличками распределения пункты а-б-в на
%отдельные задачи \\
%Задачи на корреляцию \\
% метки (тэги) - у одной задачи может быть несколько меток
% например: нормальное распределение - двойной интеграл

 % ok

\problemonly
% !Mode:: "TeX:UTF-8"

% есть идея проще:
% возле неоттипографленной (новой) задачи я ставлю такой комментарий

% untyp

% когда задача "готова" комментарий "untyp" рядом с ней можно убрать
% так проще потому, что при нажатии ctrl-f "untyp" мы сразу попадаем к нужному месту

% чтобы облегчить совместное редактирование - файл разделен на несколько более мелких частей
% кстати, у texmaker есть сквозной поиск по файлам. т.е. я могу искать "untyp" сразу во всех файлах



% прочее...


%про две шкатулки - вставить в стохан:
%E(Y|X=x_{i}) существует,
%но E(Y|X) - нет

%раздракониванию задач (полный дубляж условия) но вопросы разные - в разные разделы (да!)
%автоматические ссылки туда-сюда ???
%викифицирование ?

%метки
% die - про кубик
% coin - про монетку

%binomial
%uniform
%geom_d
%poisson

%circle_trick
%wrong_class - возможно неправильно классифицирована

%gen_fun - производящие функции


%
%Идеи решения (?):
% Превратить одношаговый эксперимент в двухшаговый: сначала выбрать предметы, затем выбрать их порядок

% !Mode:: "TeX:UTF-8"
\section{Простые эксперименты}
% simple_experiments

\subsection{Дискретные простые эксперименты}
%Эксперимент состоит из одного "этапа"

%Правило сложения вероятностей.
%Если события несовместны, то
%P(хотя бы одно)= сумма
%Р(все сразу)=0

%1.1. дискретные случайные величины (P, E)

\problem{
Подбрасываются два кубика. Какова вероятность выпадения хотя бы
одной шестёрки? Какова вероятность того, что шестёрка не выпадет
ни разу? }
\solution{
$\PP(N\geq 1)=1-\frac{5}{6}^2$; $\PP(N=0)=\frac{5}{6}^2$. }
\cat{die}

\problem{
$\Omega =\{a, b, c\}$, $\PP\ofbr{a, b}=0{,}8$, $\PP\ofbr{b,c}=0{,}7$. Найдите
$\PP\ofbr{a}$, $\PP\ofbr{b}$, $\PP\ofbr{c}$.}
\solution{$\PP\ofbr{b}=0{,}2$, $\PP\ofbr a=0{,}6$, $\PP\ofbr{c}=0{,}5$.  }

\problem{
$A$  и  $B$  несовместны,  $\PP(A)=0{,}3$, $\PP(B)=0{,}4$. Найдите
$\PP(A^{c} \cap B^{c} )$.}
\solution{ $\PP(A^{c} \cap B^{c} )=1-0{,}3-0{,}4$.}

\problem{
$\PP(A)=0{,}3$,  $\PP(B)=0{,}8$. В каких пределах
может лежать  $\PP(A\cap B)$? }
\solution{ $\PP(A\cap B)\in[0{,}1;0{,}3]$.}


\problem{
Кубик подбрасывается два раза. Какова вероятность того, что результат
второго броска будет строго больше, чем результат первого?
Какова вероятность того, что в сумме будет 6? Что в сумме будет 9? Что максимум равен
5? Что минимум равен 3? Что разница будет равна 1 или 0? }
\solution{ $\PP\ofbr{N_{2}>N_{1}}=\frac{15}{36}$; \par
$\PP\ofbr{N_{1}+N_{2}=6}=\frac{5}{36}$; \par
$\PP\ofbr{N_{1}+N_{2}=9}=\frac{4}{36}$; \par
$\PP\ofbr{\max\{N_{1},N_{2}\}=5}=\frac{9}{36}$; \par
$\PP\ofbr{\min\{N_{1},N_{2}\}=3}=\frac{7}{36}$; \par
$\PP\ofbr{|N_{1}-N_{2}|\leq 1}=\frac{16}{36}$. }
\cat{die}

\problem{ \label{shokoladnie konfeti}
На подносе лежит 20 шоколадных конфет, одинаковых с виду. В
четырёх из них есть орех внутри. Маша съела 5 конфет. Какова
вероятность того, что в наугад выбранной оставшейся конфете будет
орех? }
\solution{$\frac{4}{20}$. }

%%%%% пошло применение ожидания

\problem{ \label{ojidanie ot bernulli}
Пусть  $X$  принимает два значения, причём $\PP\ofbr{X=1}=p$ и
$\PP\ofbr{X=0}=1-p$. Найдите $\E(X)$.}
\solution{ $\E(X)=p$.}



\problem{
Пусть существует всего два момента времени, $t = 0$ и $t =
1$. Cтоимости облигаций и акций в момент времени $t$ обозначим соответственно
$B_{t}$ (bond) и $S_{t}$ (share). Известно, что $B_{0}=1$,
$B_{1}=1{,}1$, $S_{0}=5$, $S_{1}=
\begin{cases}
10, & p_{\text{high}}=0{,}7; \\
2, & p_{\text{low}}=0{,}3.
\end{cases}$ \\
Индивид может покупать акции и облигации по указанным ценам без
ограничений. Например, можно купить минус одну акцию: это
означает, что в момент времени $t=0$ индивид получает 5 рублей, а в момент $t = 1$ в
зависимости от состояния природы должен заплатить 10 рублей или 2
рубля.
\begin{enumerate}
\item Чему равна безрисковая процентная ставка за период?
\item Найдите дисконтированные математические ожидания будущих цен
акций и облигаций. Совпадают ли они с ценами нулевого периода?
\item Найдите такие вероятности $q_{\text{high}}$ и $q_{\text{low}}$, чтобы
дисконтированное математическое ожидание будущих цен
совпало с ценами нулевого периода.
\item Индивиду предлагают купить некий актив, который приносит 8
рублей в состоянии мира $\omega_{\text{high}}$ и 11 рублей в состоянии
мира $\omega_{\text{low}}$. Посчитайте ожидание стоимости этого актива с
помощью вероятностей $p$ и с помощью вероятностей $q$. Придумайте
такую комбинацию акций и облигаций, которая в будущем приносит 8 и
11 рублей соответственно, и найдите её стоимость.\end{enumerate} }
\solution{ }



\problem{
Игральный кубик подбрасывается два раза. Пусть  $X_{1}$ и $X_{2} $
"--- результаты подбрасывания. Найдите вероятности $\PP(\min
\left\{X_{1},X_{2} \right\}=4)$  и $\PP(\min
\left\{X_{1},X_{2} \right\}=2)$. }
\solution{ }


\problem{  \label{simple third}
На десяти карточках написаны числа от 1 до 9. Число 8 фигурирует
два раза, остальные числа "--- по одному разу. Карточки извлекают в
случайном порядке. Какова вероятность того, что девятка появится позже обеих
восьмёрок? }
\solution{ Устно: $\frac{1}{3}$.}

\problem{
17 заключённых, 5 камер. Заключённых распределяют по камерам по очереди, равновероятно в каждую. Какова вероятность, что Петя и Вася сидят в одной камере? }
\solution{ $0{,}2$. }
% решабельна ли более сложная задача, где конфигурации рассадок равновероятны?


\problem{
Кость подбрасывается два раза. Пусть  $X$  и  $Y$  "---
результаты
подбрасываний. Найдите  $\E\parb{|X\hm-Y|}$. }
\solution{ }

\problem{
\foreignlanguage{british}{We throw 3 dices one by one. What is the probability that we obtain 3 points in strictly increasing order?} }
\solution{ $\frac{C_{6}^{3}}{6^{3}}$. }

\problem{ \label{tri chisla}
Из 10 цифр (от 0 до 9) выбираются 3 наугад (возможны повторения).
Обозначим числа (в порядке появления): $X_{1}$, $X_{2}$, $X_{3}$.
Какова вероятность того, что $X_{1}>X_{2}>X_{3}$? }
\solution{ $\frac{C_{10}^{3}}{10^{3}}$, т.\,к. каждый способ выбрать три разных числа соответствует благоприятной
комбинации. }



\problem{
Кубик подбрасывается 3 раза. Какова вероятность того, что сумма первых двух подбрасываний будет больше третьего? }
\solution{ }

\problem{ \zdt{<<Масть>> при игре в бридж }

Часто приходится слышать, что некто при игре в бридж получил на
руки 13 пик. Какова вероятность (при условии, что карты хорошо
перетасованы) получить 13 карт одной масти?
\begin{note}
Каждый из четырёх игроков в бридж получает 13 карт из колоды в 52 карты.
\end{note}
\begin{ist}
Mosteller.
\end{ist}
}
\solution{ }


\problem{ \label{maksimum iz kartochek}
На карточках написаны числа от
1 до 100. В левую руку Маша берёт одну карточку, в правую "--- $k$~карточек.
Какова вероятность того, что число на карточке в левой руке
окажется больше числа на любой карточке из
правой руки? }
\solution{$\frac{1}{k+1}$, т.\,к. одна из $k+1$ карточек должна быть наибольшей.  }



\problem{ \label{sleeping beauty} \zdt{Спящая красавица}

Спящая красавица согласилась принять участие в научном
эксперименте. В воскресенье её специально уколют веретеном. Как
только она заснёт, будет подброшена правильная монетка. Если
монетка выпадет орлом, то спящую красавицу разбудят в понедельник
и спросят о том, как выпала монетка. Если монетка выпадет решкой,
то спящую царевну разбудят в понедельник, спросят о монетке, снова
уколют веретеном, разбудят во вторник и снова спросят о монетке.
Укол веретена вызывает легкую амнезию, и красавица не сможет
определить, просыпается ли она в первый раз или во второй.
Красавица только что проснулась.
\begin{enumerate}
\item Какова вероятность того, что сегодня понедельник?
\item Как следует отвечать красавице, если за каждый верный ответ ей
дарят молодильное яблоко?
\item Как следует отвечать красавице, если за неверный ответ её тут
же превращают в тыкву?
\end{enumerate}
\begin{note}
Осторожно! Некорректные вопросы!
\end{note}
 }
\solution{<<Сегодня понедельник>> "--- это \textbf{не} событие. Вероятность не
определена. Это функция от времени.

Вероятность того, что монетка выпала орлом, равна $0{,}5$. Поэтому ей
всё равно, как отвечать, если наказанием является превращение в
тыкву, и нужно отвечать: <<Решка!>> "--- если наградой является
молодильное яблоко. Предполагается, что красавица максимизирует
ожидаемое количество молодильных яблок.  }



\problem{
Пусть события $A_{0}$, $A_{1}$ и $A_{2}$ несовместны и вместе
покрывают всё $\Omega$. Обозначим $p_{0}=\P(A_{1}\cup A_{2})$, $p_{1}=\P(A_{0}\cup A_{2})$,
$p_{2}=\P(A_{0}\cup A_{1})$. Перечислите все условия, которым удовлетворяют $p_{0}$, $p_{1}$,
$p_{2}$. }
\solution{ }

\problem{
Найдите вероятность того, что произойдёт ровно одно событие из $A$ и $B$, если $\P(A)=0{,}3$, $\P(B)=0{,}2$, $\P(A\cap B)=0{,}1$.    }
\solution{ }

\problem{
Вася наугад выбирает два разных натуральных числа от 1 до 4.
\begin{enumerate}
\item Какова вероятность того, что будет выбрано число 3?
\item Какова вероятность того, что сумма выбранных чисел будет чётная?
\item Каково математическое ожидание суммы выбранных чисел?
\end{enumerate}
 }
\solution{ $\PP\ofbr{3}=\frac{1}{2}$, $\PP\ofbr{\Sigma\text{ чёт.}}=\frac{1}{3}$, $\E(\Sigma)=5$. }


\problem{
Известно, что когда соревнуются А и Б, то А побеждает с вероятностью $x$ (Б, соотвественно, с вероятностью $(1-x)$). Когда соревнуются А и В, то А побеждает с вероятностью $y$ (В, соответственно, с вероятностью $(1-y)$).
\begin{enumerate}
\item  Придумайте модель, которая бы позволяла узнать вероятность победы Б над В.
\item  Покажите, что можно придумать другую модель и получить другую вероятность.
\end{enumerate} }
\solution{
Если предположить, что у каждого игрока есть своя сила (константа), а вероятности победить в схватке для двух игроков относятся так же, как их силы, то $x=\frac{a}{a+b}$, $y=\frac{a}{a+c}$. Легко находим, что $\frac{b}{b+c}=\frac{y-xy}{x+y-2xy}$. }



\problem{  В клубе 25 человек.
\begin{enumerate}
\item  Сколькими способами можно выбрать комитет
из четырёх человек?
\item  Сколькими способами можно выбрать руководство, состоящее из
директора, зама и кассира?
\end{enumerate}
 }
\solution{ Комитет можно выбрать $C_{25}^{4}$ способами, руководство "--- $C_{25}^{3}3!$.}

\problem{ Сколькими способами можно расставить 5 человек в очередь?}
\solution{$5!$. }

\problem{ Сколькими способами можно покрасить 12 комнат, если требуется 4
покрасить жёлтым цветом, 5 "--- голубым и 3 "--- зелёным?}
\solution{ $C_{12}^{4}C_{8}^{5}$. }

\problem{ Шесть студентов (три юноши и три девушки), стоят в очереди за
пирожками в случайном порядке. Какова вероятность того, что юноши
и девушки чередуются?}
\solution{$2\cdot\frac{3!3!}{6!}$. }


\problem{
Где-то в начале 17 века Галилея попросили объяснить следующее:
количество троек натуральных чисел, дающих в сумме 9, такое же,
как количество троек, дающих в сумме 10; но при трёхкратном
подбрасывании кубика 9 в сумме выпадает реже, чем 10. Дайте корректное объяснение. }
\solution{ }


\problem{В классе 30 человек, и все разного роста. Учитель физкультуры хочет отобрать и поставить в порядке возрастания роста 5 человек. Сколькими способами это можно сделать?}
\solution{$C_{30}^{5}$. Расположить по росту можно только в одном порядке. }

\problem{Случайная величина $X$ равновероятно принимает одно из пяти значений: 1, 2, 3, 8 и 9. 
\begin{enumerate}
\item Найдите математическое ожидание и медиану $X$
\item Найдите значение $u$ при котором функция $f(u)=\E(|X-u|)$ достигает минимума
\item Найдите значение $u$ при котором функция $g(u)=\E((X-u)^2)$ достигает минимума
\item Сделайте выводы
\end{enumerate}
}
\solution{$\min f(u)=\med(X)$, $\min g(u)=\E(X)$}


\subsection{Непрерывные простые эксперименты}
%1.2. непрерывные случайные величины (P, E для равномерной)
\problem{
Поезда метро идут регулярно с интервалом 3 минуты. Пассажир
приходит на платформу в случайный момент времени. Пусть $X$
"--- время ожидания поезда в минутах.

Найдите $\P(X<1)$, $\E(X)$. }
\solution{$\frac{1}{3}$, $1{,}5$. }


\problem{
Светофор 40 секунд горит зелёным светом, 3 секунды "--- жёлтым, 30
секунд "--- красным, затем цикл повторяется. Петя подъезжает к светофору. На жёлтый свет Петя предпочитает остановиться.
\begin{enumerate}
\item  Какова вероятность, что Петя сможет проехать сразу?
\item  Какова средняя задержка Пети на светофоре?
\item  Вася, стоящий рядом со светофором, смотрит на него в течение 3
секунд. Какова вероятность того, что он увидит смену цвета?
\end{enumerate}
 }
\solution{ }

\problem{
Случайные величины $X$, $Y$, и $Z$ независимы и равномерны на $[0;1]$. Какова вероятность того, что $X+Y>Z$? }
\solution{ }


\problem{
\foreignlanguage{british}{At a bus stop you can take bus \#1 and bus \#2. Bus \#1 passes 10 minutes after bus \#2 has passed whereas bus \#2 passes 20 mins after bus \#1 has passed. What is the average waiting time to get on a bus at that bus stop?}

\begin{ist}
Wilmott forum, \texttt{catid=26\&threadid=55617}.
\end{ist}
 }
\solution{ $\frac{25}{3}$. }


\problem{
На множестве $A:=\{x\geq 0,\ 0\leq y\leq e^{-x}\}$ случайно (равномерно) выбирается точка. Пусть $X$ "--- абсцисса этой точки. Найдите следующие вероятности: $\PP(X>1)$, $\PP(X\in (1;5))$, $\PP(X \in [1;5])$. }
\solution{ $\int_{1}^{\infty}e^{-x}\,dx$; $\PP(X\in (1;5))=\PP(X \in [1;5])=\int_{1}^{5}e^{-x}\,dx$. }


\problem{На плоскости нарисован треугольник с вершинами $(0,0)$, $(2,0)$ и $(1,1)$. Случайным образом, равномерно, выбирается точка внутри этого треугольника. Случайная величина $X$ --- абсцисса полученной точки. Найдите
\begin{enumerate}
\item $\P(X>1)$, $\P(X<0.5)$, $\P(X=0.2)$
\item $\E(X)$
\end{enumerate}
}
\solution{$\E(X)=1$}
\todo[inline]{Тут спросить про минимизацию $f(u)=\E(|X-u|)$? }

\problem{Предположим, что завтрашний курс тугриков к луидорам --- случайная величина $X$, равномерная на отрезке $[0;1]$. Финансовый аналитик Вовочка строит прогноз $a$. За неправильный прогноз Вовочка заплатит штраф. Какой прогноз следует сделать Вовочке чтобы минимизировать ожидаемое значение штрафа, если
\begin{enumerate}
\item штраф считается по формуле $|X-a|$
\item штраф считается по формуле $0.75(X-a)$ при $X>a$ и $0.25(a-X)$ при $X<a$
\end{enumerate} 
}
\solution{Квантили распределения}

\subsection{Смешанные простые эксперименты, или содержание эксперимента неясно}
%1.3. смешанные случайные величины (P, E для смеси с равномерной)
\problem{
Как связаны между собой $\PP(A)$ и $\E(\inds{A})$? }
\solution{Равны.}


% !Mode:: "TeX:UTF-8"
\section{Сложные эксперименты}

% test comment
\subsection{<<Продвинутая>> комбинаторика }
%(использование "голого" биномиального коэффициента без каких бы то ни было заморочек можно делать в главе 1. Здесь всё, где надо применять комбинаторику с умом.
%Эту главу будут читать не все.
% возможно, что комбинаторная формула проста, но нужно догадаться до неё

\problem{
\ENG{Jenny and Alex flip $n$ fair coins each.}
\begin{enumerate}
\item \ENG{What is the probability that they get the same number of tails?}
\item  Пусть $a_{n}=\sqrt{n}\cdot p_{n}$, где $p_{n}$ "--- найденная вероятность. Найдите $\lim a_{n}$.
\end{enumerate}
 }
\solution{ $C_{2n}^{n}/2^{2n}$. Из $2n$ подбрасываний выберем $n$. Выбранные в зоне Jenny соответствуют невыбранным в зоне Alex. б) Через формулу Стирлинга: $\frac{1}{\sqrt{\pi}}$. }

\problem{
\ENG{2 couples and a single person are to be randomly placed in 5 seats in a row. What is the probability that no person that belongs to one of the couples sits next to his/her pair?} }
\solution{ }


\problem{
Встретились 6 друзей. Каждый дарит подарок одному из других 5 человек. Какова вероятность того, что найдется хотя бы одна пара человек, которая вручит подарки друг другу? }
\solution{ }


\problem{\zdt{Хулиган и случайная система}

На плоскости нарисовано $n$ прямых. Среди них нет параллельных, и никакие три не пересекаются в одной точке. Хулиган берёт первую прямую. Эта прямая делит плоскость на две полуплоскости. Хулиган случайным образом закрашивает одну из этих двух полуплоскостей. Затем хулиган поступает аналогично с каждой прямой. Какова вероятность того, что на плоскости останется хотя бы один незакрашенный кусочек?


\begin{ist}
Алексей Суздальцев.
\end{ist}
}

\solution{$\frac{n^2+n+2}{2^{n+1}}$
В оригинале у Алексея: Пусть $A$ "--- матрица $n\times2$, $\vec x$ и $\vec b$ "--- векторы подходящих размерностей. Известно, что как в матрице $A$, так и в расширенной матрице $\begin{pmatrix}
A & b  \end{pmatrix}$ все миноры максимального порядка ненулевые. Вася берёт систему $Ax=b$ и в каждой строке независимо от других ставит вместо знака равенства равновероятно знак <<больше>> или знак <<меньше>>. Какова вероятность того, что полученная система неравенств имеет решение?

Решение: остаться может либо один кусочек, либо ни одного. Всего есть $2^{n}$ способов выбирать полуплоскости. Некоторые из этих способов приводят к тому, что закрашена вся плоскость. Кусочков всего $\frac{n^{2}+n+2}{2}$. Каждый кусочек однозначно определяет способ выбора полуплоскостей. }

\problem{
На столе есть следующие предметы:
\begin{itemize}
\item 4 отличающихся друг от друга чашки;
\item 4 одинаковых гранёных стакана;
\item 10 одинаковых кусков сахара;
\item 7 соломинок разных цветов.
\end{itemize}
Сколькими способами можно разложить:
\begin{enumerate}
\item Сахар по чашкам;
\item Сахар по стаканам;
\item Соломинки по чашкам;
\item Соломинки по стаканам?
\end{enumerate}
Как изменятся ответы, если требуется, чтобы пустых ёмкостей не
оставалось? }
\solution{ }

\problem{
Сколькими способами можно разложить $k$ кусков сахара по
$n$ различающимся чашкам?

Подсказка: ответ "--- всего лишь биномиальный коэффициент. }
\solution{ $C_{k+n-1}^{n-1}$. }

% test comment 2

\problem{ \zdt{Генуэзская лотерея (задача Леонарда Эйлера)}

Из 90 чисел выбираются 5 наугад. Назовем серией последовательность
из нескольких чисел, идущих подряд. Например, если выпали числа
23, 24, 77, 78 и 79 (неважно в каком порядке), то есть две серии
(23-24, 77-78-79). Определите вероятность того, что будет ровно $k$ серий.

\begin{note}
Сама лотерея возникла в 17 веке.
\end{note}
 }
\solution{ }

\problem{ \label{sudba-don-juan-1} \zdt{Судьба Дон-Жуана} (см. тж. с.~\pageref{sudba-don-juan-2})

У Васи $n$  знакомых девушек (их всех зовут по-разному). Он пишет
им $n$  писем, но по рассеянности раскладывает их в конверты
наугад. Случайная величина $X$ обозначает количество девушек, получивших письма,
написанные лично для них.
\begin{enumerate}
\item Какова вероятность того, что Маша получит письмо, адресованное ей?
\item Какова вероятность того, что Маша и Лена получат письма, адресованные им?
\item Какова вероятность того, что хотя бы одна девушка получит
письмо, адресованное именно ей? Каков предел этой вероятности при
$n\rightarrow +\infty$?
\item Какова вероятность того, что произойдет ровно $k$ совпадений?
\end{enumerate}
}
\solution{ $1-\frac{1}{2!}+\frac{1}{3!}-\ldots$; $\frac{1}{e}$.  }


\problem{  \zdt{Покер}

Выбирается 5 карт из колоды (52 карты без джокеров, достоинством
от 2 до туза, всего 13 достоинств). Рассчитайте вероятности
комбинаций:\label{combo}
\begin{enumerate}
\item Pair (пара) "--- две карты одного достоинства;
\item Two pairs (две пары) "--- две карты одного достоинства и две другого;
\item Three of Kind (тройка) "--- три карты одного достоинства (две другие "--- разного достоинства);
\item Straight (стрит) "--- пять последовательных карт, не обязательно одной масти;
\item Flush (масть) "--- все карты одной масти;
\item Full House (фул-хаус) "--- три карты одного достоинства и две другого;
\item Four of Kind  (каре) "--- четыре карты одного достоинства;
\item Straight Flush (стрит-флэш) "--- пять последовательных карт одной масти;
\item Royal Flush (роял-флэш) "--- старшие пять последовательных карт одной масти?
\end{enumerate}

Примечание: более слабая комбинация не содержит в себе более
сильной.
}
\solution{ }

\problem{ \label{vilkodir} \zdt{<<Вилкодыр>>}

Есть $n$ дырочек, расположенных в линию, на расстоянии в 1~см друг от друга. У каждой вилки два штырька на расстоянии в 1 см.
\begin{enumerate}
\item Сколькими способами можно воткнуть $k$ одинаковых вилок?
\item Как изменится ответ, если дырочки расположены по окружности?
\end{enumerate}
}
\solution{Представим себе мифический объект <<вилкодыр>>. Он может
превращаться либо в вилку, либо в дырку. Вилкодыров у нас $n-k$.
Из них нужно выбрать $k$ вилок. Ответ: $C_{n-k}^{k}$ для линейного и
$C_{n-k}^{k}\frac{n}{n-k}$ для кругового расположения дырочек. }

\problem{ \label{lampochki v riad}
В ряду $n$ лампочек. Из них надо
зажечь 8, причём так, чтобы было три серии (по
2, 3 и 3 горящих лампочки). Сколькими способами это можно сделать? }
\solution{$3\cdot C_{n+1-d}^{k}$, где $k$ "--- число серий, $d$ "--- суммарная
длина серий. Домножение на 3 взялось из трёх вариантов (2-3-3,
3-2-3, 3-3-2). Мифический объект "--- серия из горящих лампочек
плюс негорящая справа.}

\problem{
Вася играет в преферанс. Он взял прикуп, снёс две карты и
выбрал козыря. У Васи на руках четыре козыря. Какова вероятность,
что
оставшиеся четыре козыря разделились как 4:0, 3:1, 2:2?

\begin{note}
Для тех, кто не знает, как играть в преферанс: 32 карты, из
которых 8 "--- будущие козыри, раздаются по 10 между 3 игроками, ещё
две кладутся в прикуп.
\end{note}
 }
\solution{ }

\problem{
Перетасована колода в 52 карты.
\begin{enumerate}
\item Какова вероятность того, что какие-нибудь туз и король будут лежать рядом?
\item Какова вероятность того, что какой-нибудь туз будет лежать за
каким-нибудь королем?
\end{enumerate}
 }
\solution{ }

\problem{
Чему равна сумма $C_{n}^{0}-C_{n}^{1}+C_{n}^{2}-...$?

Её применение к matching problem. }
\solution{ }

\problem{
В линию выложено $n$ предметов друг за другом. Сколькими способами
можно выбрать $k$ предметов из линии так, чтобы не были выбраны
соседние предметы? }
\solution{ $C_{n-k+1}^{k}$.

Решение 1. Отдельно рассмотрим два случая: самый правый предмет
выбран и самый правый предмет не выбран. В каждом случае
склеиваем предмет и примыкающий к нему справа предмет <<разделитель>>.

Решение 2. Удалим $k-1$ предмет из линии. Из оставшихся предметов
выберем $k$. Вернём удалённые как <<разделители>>. }

\problem{
\ENGs Given eight distinguishable rings, let $n$ be the number of
possible five-ring arrangements on the four fingers (not the
thumb) of one hand. The order of rings on each finger is
significant, but it is not required that each finger have a ring. Find the leftmost three nonzero digits of $n$. \RUSs
\begin{ist}
AIME 2000, 5.
\end{ist}
 }
\solution{Всего расположений $\binom{8}{5}\binom{8}{3}5! = 376\,320$, и три левые цифры "--- это \boxed{376}. }

\problem{ \ENGs
5 numbers are randomly picked from 90. In your bet cards, you get to choose 5 numbers.  How many cards have you got to fill in,
to guarantee that at least one of them has 4 right numbers? \RUSs
\begin{ist}
Wilmott forum.
\end{ist}
 }
\solution{ \ENGs
The answer to the original problem (\numprint{43948843} bet cards) was quoted already several times assuming that positioning of right numbers is irrelevant:
\begin{itemize}
\item there is eactly one bet card choice with 5 right numbers and
\item there are $5 \times (90-5) = 425$ bet card choices with exactly 4 right numbers.
\end{itemize}

Since the total number of different bet card choices is $\binom{90}{5}$, we have to fill out $\binom{90}{5} - 425 - 1 +1 = \numprint{43948843}$ bet cards to have at least 4 right numbers with probability 1.
\RUSs
}


\problem{
В контрольной 20 вопросов. Все ответы разные. Вася успел переписать у друга все верные ответы, но не знает, в каком порядке они идут. Отлично ставят ответившим верно на не менее чем 15 вопросов. Какова вероятность того, что Вася получит отлично? }
\solution{
$p=\frac{1}{20!}\cdot \bigl(C^{0}_{20}+C^{1}_{20}\cdot 0+C^{2}_{20}+C^{3}_{20}\cdot 2+C^{4}_{20}\cdot(C^{2}_{4}+3)+C^{5}_{20}\cdot(2C^{2}_{5}+4)\bigr)$. }

\problem{
There are $k$ books of mine among $n$ books. We put them in a shelf randomly. Which is the possibility that there are $p$ books of my who are placed continuously? (At least? Exactly?)
\begin{ist}
AoPS, \texttt{f=498\&t=192257}.
\end{ist}
 }
\solution{ Ugly sum? }


\problem{
На каждой карточке вы можете отметить любые 5 чисел из 100. Сколько карточек нужно купить, чтобы гарантированно угадать 3 числа из выпадающих в лотереи 7 чисел?

\begin{note}
Могут быть громоздкие вычисления.
\end{note}
 }
\solution{ }

\problem{
Сколькими способами можно поставить в очередь $a$ мужчин и $b$ женщин так, чтобы нигде двое мужчин не стояли рядом? }
\solution{ }

\problem{
Известно, что функция $f(n,k)$ удовлетворяет условиям:
\begin{itemize}
\item $f(n,k) = f(n-1,k) + f(n, k-1)$;
\item $f(n,0)=f(n,n)=1$.
\end{itemize}

Что это за функция такая?}
\solution{$C_{n+m}^{n}$ }

\problem{ \zdt{Усталые влюблённые}

В вагоне метро на длинную скамейку в $n$ мест садятся случайным образом $k>\frac{n}{2}$ пассажиров. Какова вероятность того, что после этого на скамейку сможет сесть влюблённая пара (влюблённым обязательно надо сидеть рядом)?

\begin{ist}
Алексей Суздальцев.
\end{ist}
}

\solution{Любую рассадку пассажиров можно представить в виде последовательности из нулей и единиц длины $n$, в которой единиц ровно $k$. Всего таких последовательностей $C_n^k$. Найдем количество последовательностей, \emph{не} удовлетворяющих условию, то есть не содержащих сдвоенных нулей (назовем такие последовательности \emph{плохими}).

Припишем к каждой плохой последовательности фиктивную единицу справа. Тогда в новой последовательности после любого нуля стоит единица, а значит, вся последовательность состоит из паттернов <<1>> и <<01>>. Паттернов <<01>> ровно $n-k$, (столько же, сколько и нулей), всего же паттернов столько же, сколько и единиц, то есть $k+1$. Таким образом, мы имеем $k+1$ позиций, из которых надо выбрать $n-k$, куда встанет паттерн <<01>>. Способов сделать это всего $C_{k+1}^{n-k}$, что и равно числу плохих последовательностей.

Значит, искомая вероятность равна $1-\frac{C_{k+1}^{n-k}}{C_n^k}$.}


\problem{
В классе 28 человек, среди них 18 девочек. Класс построили в 4 ряда по 7 человек. Какова вероятность того, что рядом с Вовочкой будет стоять хотя бы одна девочка?
\begin{note}
Для Вовочки любая девушка "--- рядом :).
\end{note}
 }
\solution{ }



\subsection{Геометрическое распределение (и близкие по духу)}
% первое упоминание о методе первого шага


\problem{
Равной силы команды играют до трёх побед. Какова вероятность того,
что будет ровно 3 партии? Ровно 4? Ровно 5? }
\solution{ $\PP(N=3)=2\frac{1}{2}^{3}$; $\PP(N=4)=2C_{3}^{1}\frac{1}{2}^{4}$; $\PP(N=5)=2C_{4}^{2}\frac{1}{2}^{5}$. }
\cat{geom_d}

\problem{Вася стреляет по мишени бесконечное количество раз. Он попадает по мишени с очень маленькой вероятностью. Какова вероятность того, что до первого попадания по мишени Васе потребуется больше времени, чем в среднем уходит на одно попадание?}
\solution{Путь $p$ "--- вероятность. Тогда $\E(X)=\frac{1}{p}$. Нас интересует $\PP\left(X>\frac{1}{p}\right)=\left(1-\frac{1}{p}\right)^{\frac{1}{p}}\approx \frac{1}{e}$.}
\cat{geom_d}


\problem{  \zdt{Геометрическое распределение}
Кубик подбрасывают до первого выпадения шестерки. Случайная величина  $N$ "---
число подбрасываний.
\begin{itemize}
\item Найдите $\PP(N=6)$, $\PP(N=k)$, $\PP(N>10)$ и $\PP(N>30\mid N>20)$, $\E(N)$.
\item Найдите $\E\left(\frac{1}{N}\right)$.
\end{itemize}
 }
\solution{ }
\cat{geom_d}


\problem{ \label{s chego vse nachinalos}\zdt{С чего всё начиналось\ldotst{}}

Париж, Людовик XIV, 1654 год, высшее общество говорит о рождении
новой науки "--- теории вероятностей. Ах, кавалер де Мере, <<fort
honn\^{e}te homme sans \^{e}tre math\'{e}maticien>>\ldotst{} (<<благородный
человек, хотя и не математик>>). Старая задача, неправильные
решения которой предлагались тысячелетиями (например, одно из
неправильных решений предлагал изобретатель двойной записи, кумир
бухгалтеров, Лука Пачоли), наконец решена правильно! Два игрока
играют в честную игру до шести побед. Игрок, первым выигравший
шесть партий (не обязательно подряд), получает 800 рублей. К
текущему моменту первый игрок выиграл 5 партий, а второй "--- 3
партии. Они вынуждены прервать игру в данной ситуации. Как им поделить приз по справедливости? }
\solution{ $700:100$.}


\problem{ \zdt{Von Neumann. Что делать, если монетка неправильная?}

Имеется <<несправедливая>> монетка, выпадающая гербом с некоторой
вероятностью. Под раундом будем подразумевать двукратное
подбрасывание монеты. Проводим первый раунд. Если результат раунда
"--- ГР (сначала герб, затем решка), то считаем, что выиграл первый
игрок. Если результат раунда "--- РГ, то считаем, что выиграл второй
игрок. Если результат раунда "--- ГГ или РР, то проводим ещё один
раунд. И так далее, пока либо не определится победитель, либо
количество раундов не достигнет числа $n$.
\begin{enumerate}
\item Найдите вероятности <<ничьей>>, выигрыша первого игрока, выигрыша
второго игрока в зависимости от $n$. Найдите пределы этих
вероятностей при $n\rightarrow +\infty$.
\item Как с помощью неправильной монетки сымитировать правильную?
\end{enumerate}
 }
\solution{ }

\problem{ \label{Monty Hell problem} \textit{Monty Hell problem} (не путать с Monty Hall)

\textbf{Сказка.} Ежедневно Кощей Бессмертный получает пенсию в размере 10~золотых монет. Затрат у Кощея нет никаких. Поэтому с начала пенсионного возраста он аккуратно нумерует
каждую полученную монету и кладет её в сундук. Ночью Мышка-норушка крадёт одну золотую монету из сундука.
\begin{enumerate}
\item Какова вероятность того, что $i$-я монета когда-либо исчезнет из Сундука?
\item Какова вероятность того, что хотя бы одна монета пролежит в сундуке бесконечно долго?
\item Дни сокращаются в продолжительности (каждый последующий "--- в два раза короче, чем предыдущий). Сколько монет будет в сундуке в конце времени?
\end{enumerate}
\begin{hint}
$(1-x) \le e^{-x}$.
\end{hint}
\begin{note}
А где надсказка?
\end{note}
 }
\solution{ Вероятность, что $i$-я монета когда-либо исчезнет, равна $1$, а того, что пролежит бесконечно долго, "--- 0. }

\problem{
Случайным образом выбирается натуральное число $X$. Вероятность выбора числа $n$ такова: $\PP(X=n)=2^{-n}$.
\begin{enumerate}
\item  Какова вероятность того, что будет выбрано чётное число? Нечётное число? Число, большее пяти? Число от 3 до 11?
\item Пусть независимо друг от друга выбираются $c$ чисел. Пусть $K_{c}$ "--- количество невыбранных чисел на отрезке от одного до наибольшего выбранного числа. Найдите $\PP(K_{c}=k).$
\end{enumerate}
\begin{ist}
AMM E3061, T.~Ferguson and C.~Melolidakis.
\end{ist}
 }
\solution{ $P(K_{c}=k)=2^{-(k+1)}$ вне зависимости от $c$. Для начала обнулим значение $K_{c}$ и возьмём в руку $c$ монеток. Подкинем монетки. Если все выпали орлом, мы прибавляем единичку к $K_{c}$. Если все выпали решкой, то мы объявляем значение $K_{c}$. Если часть выпала орлом, часть решкой, то выкинем те, что выпали решкой, и снова перейдем к подкидыванию монеток. В результате имее: рост $K_{c}$ на единицу или глобальная остановка процесса происходит равновероятно. Значит, $K_{c}$ распределено геометрически. }
\cat{hard}

\problem{Величины $X_1$, $X_2$, \ldots независимы и одинаково распределены с некоторой функцией плотности $f$. Величина $X_i$ --- это количество осадков в $i$-ый год. Пусть $Y$ --- номер года, когда впервые будет превышено количество осадков, выпавших в первом году.
Найдите закон распределения $Y$ и $\E(Y)$}
\solution{Найдем $\P(Y>k)$. Это вероятность того, что в первые $k$ лет не будет достигнут уровень первого года. Значит это вероятность того, что первый год дал наибольшее количество осадков за первые $k$ лет. В силу симметрии $\P(Y>k)=1/k$. Отсюда $\P(Y=k)=\P(Y>k-1)-\P(Y>k)=1/k(k-1)$ и $\E(Y)=+\infty$}

% untyp
\problem{Преподаватель по теории вероятностей пообещал своим студентам, что включит задачу на геометрическое распределение в экзамен с вероятностью 1/3. Чтобы исполнить своё обещание он подбросил одну монетку два раза и посчитал количество орлов. Оказалось, что орлов было ровно два. На основании этого количества он принял решение. Какое решение он принял и почему?}
\solution{Подбрасывая монетки детерминированное количество раз нельзя получить вероятность 1/3, значит преподаватель заранее не знал, сколько раз он будет подбрасывать монетку.  Простое правило принятия решения может иметь примерно такой вид: если выпало А, то давать задачу, если выпало Б, то не давать задачу, если не выпало ни А, ни Б, то повторить эксперимент. Орлов может быть либо два, либо один, либо 0. На два орла преподаватель повторять эксперимент не стал. Разумно предположить, что он использовал простую стратегию. Значит два орла означают <<включать>>, один орел --- <<не включать>>, ноль орлов --- подбрасывать монетку еще два раза. }
Идея: Николай Арефьев

\problem{Вася прыгает в длину несколько раз подряд. Результаты васиных прыжков --- независимые одинаково распределенные непрерывные случайные величины. Прыгнув в первый раз он записывает результат. И прыгает до тех пор, пока не перепрыгнет свой первый результат. Величина $X$ --- сколько прыжков Васе потребуется сделать дополнительно, чтобы перепрыгнуть первый результат. Найдите $\P(X=k)$ и $\E(X)$.}
\solution{$\P(X=k)=\frac{1}{k(k+1)}$, т.к. последний прыжок должен быть самым длинным из $k+1$ прыжка, а первый --- самым длинным из $k$ оставшихся. $\E(X)=\infty$.}




\subsection{Из \textit{n} предметов выбирается \textit{k}}
\problem{ \label{id008}
Из 50 деталей 4 бракованные. Выбирается наугад 10 деталей на проверку.
Какова вероятность не заметить брак? }
\solution{$\frac{C_{46}^{10}}{C_{50}^{10}}$. }

\problem{
Есть 4 карты одного достоинства. Наугад выбираются две.
Какова вероятность того, что они будут разного цвета? }
\solution{$\frac{1}{3}$. }

\problem{ \label{5 iz 36}
Какова вероятность полностью угадать комбинацию в лотерее 5 из
36?}
\solution{$\frac{1}{C_{36}^{5}}$. }

\problem{
В мешке 50 орехов, из них 5 пустые. Вы выбираете наугад 10
орехов. Какова вероятность того, что ровно один из них будет
пустой?}
\solution{ $\frac{C_{45}^{9}C_{5}^{1}}{C_{50}^{10}}$.}


\problem{ \label{tri shara iz korobki}
Из коробки с 4 синими и 5 зелёными шарами достают 3 шара. Пусть
$B$  и  $G$  "--- количество извлечённых синих и зелёных шаров.
Найдите  $\E(B)$,  $\E(G)$,  $\E(B\cdot G)$,  $\E(B-G)$. }
\solution{$\E(B)=3\cdot\frac{4}{9}=\frac{4}{3}$; $\E(G)=3-\E(B)=\frac{8}{3}$; $\E(B-G)=-\frac{4}{3}$; $\E(B\cdot G)=2\cdot\frac{5}{6}$.  }




\problem{
На факультете $n$ студентов. Из них наугад выбирают $a$ человек. Через год $b_{-}$ студентов покидают факультет, $b_{+}$ студентов приходят на факультет. Из них снова наугад выбирают $a$. Какова вероятность того, что хотя бы одного выберут два раза? }
\solution{ }


\problem{ \label{cube-cut-1}(см. тж. с.~\pageref{cube-cut-2})

 \ENGs A wooden cube that measures 3 cm along each edge is painted red. The painted cube is then cut into 27 pieces of 1-cm cubes.
\begin{enumerate}
\item If I choose one of the small cubes at random and toss it in the air, what is the probability that it will land red-painted side up?
\item If I put all the small cubes in a bag and randomly draw out 3, what is the probability that at least 3 faces on the cubes I choose are painted red?
\item If I put the small cubes in a bag and randomly draw out 3, what is the probability that exactly 3 of the faces are painted red?
\item Invent a new question!
\end{enumerate} \RUSs
\begin{ist}
\url{http://letsplaymath.wordpress.com/2007/07/25/puzzle-random-blocks/}
\end{ist}
 }
\solution{ Вероятность выпадения красной стороны сверху равна $\frac{1}{3}$. }

\problem{
Контрольную пишут 40 человек. Половина пишет первый вариант, половина "--- второй. Время написания работы каждым студентом "--- независимые непрерывные случайные величины. Какова вероятность того, что в тот момент, когда будет сдана последняя работа первого варианта, останется ещё 5 человек, пишущих второй вариант? }
\solution{ $\frac{C_{20}^{1}C_{20}^{5}}{C_{40}^{6}}\frac{1}{6}$ (вероятность заданной шестёрки финалистов помножить на вероятность выбора одного человека из шести). }


\problem{В бридж играют четыре игрока: Юг, Восток, Север, Запад. Перемешанная колода в 52 карты раздаётся игрокам по очереди по одной карте. Юг и Север получили 11~пик. Какова вероятность того, что две оставшиеся пики оказались у одного игрока? Разделились между остальными игроками? Каковы вероятности различных раскладов пик между остальными игроками, если Юг и Север получили 8~пик?}
\solution{}

\problem{ \label{korrektori ochepiatok} \zdt{Корректоры очепяток}

Вася замечает очепятку с вероятностью $0{,}7$; Петя независимо от Васи замечает очепятку с вероятностью $0{,}8$. В книге содержится 100 опечаток. Какова вероятность того, что Вася заметит 30 опечаток, Петя "--- 50, причём 14 опечаток будут замечены обоими корректорами? }
\solution{
$\frac{100!}{14!16!36!34!}0{,}7^{30}0{,}3^{70}0{,}8^{50}0{,}2^{50}$. }



\subsection{Биномиальное распределение (до дисперсии)}
%1.4. Эксперимент состоит из множества одинаковых этапов
%(сюда можно отнести простые задачи на биномиальное распределение и совсем простую комбинаторику)

% спорные случаи - эксперимент повторяется два раза - можно отнести в одношаговые (если обозримо в явном виде выписать все исходы)

\problem{
Монетка подбрасывается 5 раз. Какова вероятность того, что будет
выпадет ровно один орёл? Ровно два? Ни одного? }
\solution{$\PP(N=1)=C_{5}^{1}(\frac{1}{2})^{5}$; $\PP(N=2)=C_{5}^{2}(\frac{1}{2})^{5}$; $\PP(N=0)=C_{5}^{0}(\frac{1}{2})^{5}$. }
\cat{die} \cat{binomial}

\problem{
Какова вероятность при шести подбрасываниях кубика получить ровно
две шестёрки? }
\solution{$\PP(N=2)=C_{6}^{2}(\frac{1}{6})^{2}(\frac{5}{6})^{4}$. }
\cat{binomial}


\problem{
Какова вероятность того, что у десяти человек не будет ни одного совпадения дней рождений? Каков минимальный размер компании, чтобы вероятность одинакового дня рождения была больше половины? }
\solution{$\frac{365\cdot 364\cdot 363\cdot \ldots \cdot 356}{365^{n}}$; минимальная компания состоит из $23$ человек. }


\problem{
Маша подбрасывает монетку три раза, а Саша "--- два раза. Какова
вероятность того, что у Маши герб выпадет больше раз, чем у
Саши?}
\solution{ $\PP=\frac{1}{4}(1-\frac{1}{8})+\frac{1}{2}(\frac{1}{8}+3\frac{1}{8})+\frac{1}{4}\frac{1}{8}=\frac{1}{2}$.}


\problem{ \label{deti raznih polov}
Сколько детей должно быть в семье, чтобы вероятность того,
что имеется по крайней мере один ребенок каждого пола, была больше
0,95? }
\solution{$(\frac{1}{2})^{(n-1)}\le 0{,}15$.  }


\problem{ \zdt{Осторожный фальшивомонетчик}

Дворцовый чеканщик кладёт в каждый ящик вместимостью в сто монет
одну фальшивую. Король подозревает чеканщика и подвергает проверке
монеты, взятые наудачу по одной в каждом из 100 ящиков.
\begin{enumerate}
\item Какова вероятность того, что чеканщик не будет разоблачён?
\item Каков ответ в предыдущей задаче, если 100 заменить на $n$?
\end{enumerate}
\begin{ist}
Mosteller.
\end{ist}
}
\solution{ $0{,}99^{n}$. }


\problem{ \label{strategia udvoenia} \zdt{Стратегия удвоения}

В казино имеется рулетка, которая с вероятностью $0{,}5$ выпадает
или на чёрное, или на красное. Игрок, поставивший сумму $n$ и угадавший
цвет, получает обратно сумму $2n$. Вася решил играть по следующей
схеме. Сначала он ставит доллар. Если он выигрывает, то покидает
казино, если проигрывает, то удваивает ставку и ставит два
доллара. Если выигрывает, то покидает казино, если проигрывает, то
снова удваивает ставку и ставит четыре доллара и т.\,д., пока не
выиграет в первый раз или впервые не хватит денег на новую
удвоенную ставку. У Васи имеется 1\,050 долларов.
\begin{enumerate}
\item Какова вероятность того, что Вася покинет казино после выигрыша?
\item Каков ожидаемый выигрыш Васи?
\end{enumerate}
\begin{note}
В реальности вероятность меньше $0{,}5$, т.\,к. на
рулетке есть 0 и (иногда) 00. Их наличие, естественно, уменьшает и
вероятность, и ожидаемый выигрыш.
\end{note}
 }
\solution{ $1-\frac{1}{1024}$; $0$. }


\problem{ \ENGs
When the $n$'s dice are thrown at the one time, find the probability such that the sum of the cast is $n+3$? \RUSs }
\solution{ }


\problem{
Пусть $X_{1}$, $X_{2}$,..., $X_{n}$ "--- НОРСВ, такие, что $X_{i}= \begin{cases}  1, & p; \\ 0, & (1-p).\end{cases}$ Пусть $k$ "--- такая константа, что $2k\ge n$. Найдите вероятность того, что самая длинная серия из единиц имеет длину $k$. Что делать при $2k<n$? }
\solution{ }
\cat{wrong_class}


\problem{ \ENGs
Suppose you are given a random number generator, which draws samples from an uniform distribution between $0$ and $1$.
The question is: how many samples you have to draw, so that you are 95\% sure that at least 1 sample lies between $0.70$ and $0.72$? \RUSs }
\solution{ }


\problem{  \zdt{Биномиальное распределение}

Кубик подбрасывают 5 раз. Пусть $N$ "--- количество выпадений шестёрки. Найдите $\PP(N=3)$, $\PP(N=k)$  и
$\PP(N>4 \mid N>3)$, $\E(N)$.}
\solution{ }

\problem{ \zdt{Максимальная вероятность для биномиального распределения}

Пусть $X$ распределена биномиально. Общее число экспериментов
равно $n$, вероятность успеха в отдельном испытании равна $p$.
\begin{enumerate}
\item Найдите $\frac{\PP(X=k)}{\PP(X=k-1)}$.
\item При каких $k$ дробь $\frac{\PP(X=k)}{\PP(X=k-1)}$ будет не меньше 1?
\item Каким должно быть $k$, чтобы $\PP(X=k)$ была максимальной?
\end{enumerate}
 }
\solution{ }

\problem{ Известно, что предварительно зарезервированный билет на автобус
дальнего следования выкупается с вероятностью 0{,}9. В обычном
автобусе 18~мест, в микроавтобусе 9~мест. Компания <<Микро>>,
перевозящая людей в микроавтобусах, допускает резервирование 10~билетов на один микроавтобус. Компания <<Макро>>, перевозящая
людей в обычных автобусах допускает резервирование 20~мест на один автобус. У какой компании больше вероятность оказаться в ситуации нехватки
мест? }
\solution{ }

\problem{ \ENGs
The psychologist Tversky and his colleagues say that about four
out of five people will answer (a) to the following question:
\begin{quote}
A certain town is served by two hospitals. In the larger
hospital about 45 babies are born each day, and in the smaller
hospital 15 babies are born each day. Although the overall
proportion of boys is about 50 percent, the actual proportion at
either hospital may be more or less than 50 percent on any day. At
the end of a year, which hospital will have the greater number of
days on which more than 60 percent of the babies born were boys?
\end{quote}
\begin{center}
\begin{tabular}{ccc}
(a) the large hospital & (b) the small hospital & (c) neither (about the same) \\
\end{tabular}
\end{center}\RUSs

Дайте верный ответ и попытайтесь объяснить, почему большинство
людей ошибается при ответе на этот вопрос. }
\solution{В маленьком роддоме <<мальчиковых>> дней больше. В силу закона больших чисел, чем больше число наблюдений, тем сильнее выборочная доля мальчиков должна быть похожа на вероятность рождения мальчика.}

\problem{
В забеге участвуют 12 лошадей. Каждый из 10 зрителей пытается составить свой прогноз для трёх призовых мест. Какова вероятность того, что хотя бы один из них окажется прав? }
\solution{ $ 1-(\frac{1319}{1320})^{10}\approx 0{,}008 $. }


\problem{
Есть $N$ монеток. Каждая из них может быть фальшивой с
вероятностью $p$. Известно, сколько весят настоящие. Известно, что
фальшивые весят меньше, чем настоящие. Каждая фальшивая может иметь
своё отклонение от правильного веса. Задача "---
определить, является ли фальшивой каждая монета. Предлагается следующий способ:
\begin{quote}
Разбить монеты на группы по $n$ монет в каждой группе. Взвесить
каждую группу. Если вес группы совпадает с эталонным, то вся
группа признается настоящей. Если вес группы меньше эталонного, то
каждая монеты из
группы взвешивается отдельно.
\end{quote}

Предположим, что $N$ делится на $n$. Пусть $X$ "--- требуемое число взвешиваний.
\begin{enumerate}
\item Найдите $\E(X)$;
\item При каком условии на $p$ и $n$ предложенный способ более
эффективен чем взвешивание каждой монеты?
\item Исследуйте поведение функции $\frac{\E(X)}{N}$ от $n$ (есть ли минимум, максимум и т.\,д.).
\end{enumerate}
 }
\solution{ }


\problem{ \zdt{Задача Банаха (Banach's matchbox problem)}

У Маши есть две коробки, в каждой из которых осталось по $n$~конфет. Когда Маша хочет конфету, она выбирает наугад одну из
коробок и берёт конфету оттуда. Рано или поздно Маша впервые
откроет пустую коробку. В этот момент другая коробка содержит
некоторое количество конфет. Обозначим за $u_r$ вероятность того, что
в другой коробке ровно $r$ конфет.
\begin{enumerate}
\item  Найдите $u_r$.
\item  Найдите вероятности $v_r$ того, что в тот момент, когда из
одной коробки возьмут последнюю конфету (она только станет
пустой!), в другой будет находится ровно $r$ конфет.
\item  Найдите вероятность того, что коробка, которая была опустошена
раньше, не будет первой коробкой, открытой пустой.
\end{enumerate}
 }
\solution{
Пункт 1. Последняя попытка взять конфету "--- из пустой коробки. Назовём
эту коробку $A$. Из предыдущих $n+(n-r)$ конфет $n$ приходятся на
коробку $A$. Вероятности равны $\frac{1}{2}$. Получаем:
$u_{r}=\frac{C_{2n-r}^{n}}{2^{n+(n-r)}}$ }


\problem{
В уездном городе $N$ два родильных дома, в одном ежедневно рождается 50 человек, в другом "--- 100 человек. В каком роддоме чаще рождается одинаковое количество мальчиков и девочек?}
\solution{В маленьком. }

\problem{
\ENGs Let you choose an infinite sequence of integers between 1 and 10, what is the possibility that your sequence doesn't have any ``1''? \RUSs }
\solution{ 0. }


\problem{ \ENGs
There are three coins in a box.  These coins when flipped, will
come up heads with respective probabilities $0.3$, $0.5$, $0.7$.  A
coin is randomly selected (meaning uniform distribution!) from among
these three and then flipped $10$ times.  Let $N$ be the number of
heads obtained on the first ten flips. \RUSs
\begin{enumerate}
\item Найдите $\PP(N=0)$.
\item If you win \$1 each time a head appears and you lose \$1 each time a tail appears, is this a fair game?  Explain.
\end{enumerate}
 }
\solution{ }


% серия задач, которые могут казаться интуитивно противоречивыми...
\problem{Как почувствовать разницу в 0{,}01\cite{sekei:paradox}? Пусть вероятность того, что Маша находится целый день в хорошем настроении, равна 0{,}99, а вероятность того, что Саша находится в хорошем настроении, равна 0{,}999\,9. Какова вероятность того, что Маша будет целый год непрерывно в хорошем настроении? Саша?}
\solution{0{,}025\,5 и 0{,}964\,2.}


\problem{Петя подбрасывает 10 монеток. Если из этих 10~подбрасываний будет как минимум 8~одинаковых, то мы назовём это чудом. Какова вероятность чуда? Какова вероятность хотя бы одного чуда, если, кроме Пети, ещё 9~человек подбрасывает по 10~монеток?}
\solution{$\frac{7}{64}$; около $\frac{2}{3}$.}

\problem{Вероятность того, что прошлогодний грецкий орех будет червивым, равна 0{,}25. Сколько минимум нужно взять грецких орехов, чтобы среди них был хотя бы один нормальный с вероятностью 99,9\,\%?}
\solution{5.}

\problem{ \zdt{Биномиальные числа Фибоначчи}

Пусть $\{F_k\}$ "--- последовательность Фибоначчи, а $X$ "--- число выпавших орлов при $n$ подбрасываниях правильной монетки. Вычислить $\E(F_{1+X})$.
\begin{ist}
Алексей Суздальцев.
\end{ist}
}

\solution{$\frac{F_{2n+1}}{2^n}$. У чисел Фибоначчи есть свойство: $F_{n}=F_{n-1}+F_{n-2}\hm =L(1+L)F_{n} \hm=\ldots \hm =L^{k}(1+l)^{k}F_{n} \hm=(1+L)^{k}F_{n-k}$.}

\problem{Семеро друзей выбирают, пойти им в кино на фильм ужасов или на комедию. Каждый из них предпочтёт комедию независимо от других с вероятностью $0{,}6$. Есть два способа голосования, А и Б. Способ~А "--- все голосуют одновременно, выбирается альтернатива, набравшая больше голосов. Способ~Б "--- голосование в два тура. Первый тур: трое самых старших друзей голосуют между собой и большинством решают, за что они втроём будут голосовать единогласно во втором туре: за комедию или за ужасы. Второй тур: голосуют все семеро, но трое старших голосуют так, как согласованно договорились на первом туре. При каком способе голосования выше шансы пойти на комедию? }
\solution{}


\problem{Маша и Саша учатся в одном классе. Маша и Саша учатся по одним и тем же $n$ учебникам. В один день Маша и Саша независимо друг от друга приносят случайное подмножество своих учебников в школу.
\begin{enumerate}
\item Какова вероятность того, что у Маши не будет ни одного учебника, которого бы не было у Саши?
\item Какова вероятность того, что вместе у Саши и Маши будут все $n$ учебников хотя бы в одном экземпляре?
\end{enumerate}}
\solution{Можно считать, что каждый учебник Саша и Маша берут с вероятностью $0{,}5$. Ответ в обоих пунктах: $0{,}75^n$.}


\subsection{Деревья и прочее без условных вероятностей}

\problem{
Подбрасывается кубик, а затем монетка подбрасывается столько раз,
сколько очков на выпавшей грани. Какова вероятность того, что
орёл выпадет ровно 4 раза?}
\solution{ $\PP=\frac{1}{6}\left(\left(\frac{1}{2}\right)^{4}+C_{5}^{4}\left(\frac{1}{2}\right)^{5}+C_{6}^{4}\left(\frac{1}{2}\right)^{6}\right).$
}


\problem{ \label{ritsari-bliznetsi} \zdt{Рыцари-близнецы }

Король Артур проводит рыцарский турнир, в котором, так же как и в
теннисе, порядок состязания определяется жребием. Среди восьми рыцарей, одинаково искусных в
ратном деле, два близнеца.
\begin{enumerate}
\item Какова вероятность того, что они встретятся в поединке?
\item Каков ответ в случае $2^n$ рыцарей?
\end{enumerate}
 }
\solution{$\PP_1=\frac{1}{7}+\frac{1}{14}+\frac{1}{28}$; $\PP_2=\frac{1}{2^{n}-1}\cdot 2\cdot\left(1-0{,}5^{n}\right)$.  }
\begin{ist}
Mosteller.
\end{ist}

\problem{ \label{Vasia i Petia na lektsii}
Вася посещает 60\,\% лекций по теории вероятностей, Петя "--- 70\,\%. Они
из разных групп и посещают лекции независимо друг от друга. Какова
вероятность, что на следующую лекцию придут оба? Хотя бы один из
них?}
\solution{ $\PP(N=2)=0{,}7\cdot 0{,}6=0{,}42$. $\PP(N\geq 1)=1-(1-0{,}7)\cdot (1-0{,}6)=0{,}88$. }

\problem{ \label{vtoroi v finale} \zdt{Выйдет ли второй в финал?} \par
В теннисном турнире участвуют 8~игроков. Есть три тура
(четвертьфинал "--- полуфинал "--- финал). Противники в первом туре
определяются случайным образом. Предположим, что лучший игрок
всегда побеждает второго по мастерству, а тот, в свою очередь
побеждает всех остальных. Проигрывающий в финале занимает второе
место. Какова вероятность
того, что это место займет второй по мастерству игрок?
\begin{ist}
обработка Mosteller.
\end{ist}
}
\solution{ $\PP=\frac{4}{7}$. }

\problem{ \ENGs
The Wimbledon Men's Singles Tournament has 128 players. The first round pairings are completely random, subject to the constraint that none of the top 32 players can be paired against each other. Two competitors, Olivier Rochus, and his brother Christophe are competing, and neither are in the elite group of 32 players. What is the probability that these brothers play in the first round (as actually occurred)? \RUSs }
\solution{ }

\problem{
Первый автобус отходит от остановки в 5:00. Далее интервалы между
автобусами равновероятно составляют 10 или 15 минут, независимо от
прошлых интервалов. Вася приходит на остановку в 5:42.
\begin{enumerate}
\item Какова ожидаемая длина интервала, в который он попадает?
\item  Какова ожидаемая длина следующего интервала?
\item  Пусть $t\to\infty$ (???)
\end{enumerate}
 }
\solution{ Ожидаемая длина следующего интервала "--- 12{,}5 минут. }

\problem{ \ENGs
There are two ants on opposite corners of a cube. On each move, they can travel along an edge to an adjacent vertex. What is the probability that they both return to their starting point after 4 moves? \RUSs }
\solution{$(\frac{7}{27})^{2}$. }


\problem{ \label{legkomislennii chlen juri} \zdt{Легкомысленный член жюри} \par
В жюри из трёх человек два члена независимо друг от друга
принимают правильное решение с вероятностью $p$, а  третий для
вынесения    решения бросает монету (окончательное решение
выносится большинством голосов). Жюри из одного человека выносит
справедливое решение с вероятностью $p$. Какое из этих жюри
выносит справедливое решение с большей вероятностью?
\begin{ist}
Mosteller.
\end{ist}
 }
\solution{ $p-\frac{p^{2}}{2}<p$, т.\,е. жюри из одного человека лучше. }


\problem{ \label{Simpson's paradox} \zdt{Simpson's paradox} \par
Тренер хочет отправить на соревнование самого сильного из своих
спортсменов. Самым сильным игроком тренер считает того, у кого
больше всех шансов получить первое место, если бы соревнование
проводилось среди своих. У тренера два спортсмена: А, постоянно
набирающий 3~штрафных очка при выполнении упражнения, и Б,
набирающий 1~штрафное очко с вероятностью 0{,}54 и 5~штрафных очков
с вероятностью 0{,}46.
\begin{enumerate}
\item Кого отправит тренер на соревнования?
\item Кого отправит тренер на соревнования, если, помимо А и Б, у него
тренируется спортсмен В, набирающий 2 штрафных очка с вероятностью
0,56, 4 штрафных очка с вероятностью 0,22 и 6 штрафных очков с
вероятностью 0,22.
\item Мораль?
\end{enumerate}
 }
\solution{ Спортсмена Б, если нет спортсмена В; спортсмена А, если есть спортсмен В. Мораль "--- зависимость от третьей альтернативы. }


\problem{
В турнире участвуют 8 человек, разных по силе. Более сильный побеждает более слабого. Проигравший выбывает, победитель выходит в следующий тур.
Какова вероятность того, что $i$-й по силе игрок дойдет до финала? }
\solution{ }


\problem{ \ENGs
A bag contains a total of $N$ balls either blue or red. If $5$ balls are chosen from the bag the probability all of them being blue is 0.5. What are the values of $N$ for which this is possible? \RUSs}
\solution{ }

\problem{ \ENGs
Each of two boxes contains both black and white marbles, and the total number of marbles in the two boxes is 25. One marble is taken out of each box randomly. The probability that both marbles are black is $\frac{27}{50}$. What is the probability that both marbles are white? \RUSs }
\solution{ }


\problem{ \label{gadanie v pole}
Маша с подружкой гуляют в поле. Подружка предлагает погадать на
суженого. Она зажимает в руке шесть травинок так, чтобы концы
травинок торчали сверху и снизу. Маша сначала связывает эти
травинки попарно между собой сверху, а затем и снизу (получается
три завязывания сверху и три завязывания снизу). Если при этом все
шесть травинок окажутся связанными в одно кольцо, то это означает,
что Маша в текущем году выйдет замуж.
Какие шансы у Маши?
\begin{note}
Будем считать, что завязывание травинок в <<трилистник>>, <<восьмерку>> и прочие нетривиальные узлы также
означает замужество.
\end{note}
\begin{ist}
Баврин, Фрибус, <<Старинные задачи>>.
\end{ist}
 }
\solution{$\frac{8}{15}$. }


\problem{
Две урны содержат одно и то же количество
шаров, несколько чёрных и несколько белых каждая. Из них
извлекаются $n$ ($n>3$) шаров с возвращением. Найти число $n$ и
содержимое обеих урн, если вероятность того, что все белые шары
извлечены из первой урны, равна вероятности того, что из второй
извлечены либо все белые, либо все чёрные шары.
\begin{ist}
Mosteller.
\end{ist}
}
\solution{ }
\cat{wrong_class}

\problem{ \label{po rublu za 6}
Кость подбрасывается 3 раза. Размер ставки "--- 1 рубль. Если
шестёрка не выпадает ни разу, то ставка проиграна, если шестёрка
выпадает хотя бы один раз, то ставка возвращается, плюс
выплачивается выигрыш по 1 рублю за каждую шестёрку. Найдите
стоимость этой лотереи. }
\solution{$\E(X)=3\cdot\frac{1}{6}+1+(-2)\cdot\left(\frac{5}{6}\right)^{3}$ }

\problem{
\label{Parrondo's game} \zdt{Parrondo's game}

Назовем <<рублёвой игрой с вероятностью $p$>> игру, в которой
игрок выигрывает 1~рубль с вероятностью $p$ и проигрывает один
рубль с вероятностью $(1-p)$. Игра $A$ "--- это рублёвая игра с вероятностью 0,45.
Игра $B$ состоит в следующем: если сумма в твоём кошельке делится
на три, то ты играешь в рублёвую игру с вероятностью 0,05; если же
сумма в твоем кошельке не делится на три, то ты играешь в рублёвую
игру с вероятностью 0{,}7. Что будет происходить с ожидаемым благосостоянием игрока, если он
\begin{enumerate}
\item Будет постоянно играть в игру $A$?
\item  Будет постоянно играть в игру $B$?
\item  Будет постоянно играть $A$ или $B$ с вероятностью по 0,5?
\item  Придумайте <<лохотрон>> для интеллектуалов.
\end{enumerate}
 }
% d - идея Ромы Мартусевича
\solution{ При игре только в $A$ "--- убывать; только в $B$ убывать; в вероятностную комбинацию "--- возрастать. }

\problem{ \zdt{Parrondo's game --- альтернатива} \ENGs

A much simpler example is dealing cards from a well-shuffled deck. Suppose I get \$14 if two cards in a row match in rank (two 2's or two Kings for examples), and pay \$1 if they don't. The chance of two cards in a row matching is $\frac{1}{17}$, so I pay \$16 for each \$14 I win.

Now I play the same game, alternating the deal between two decks. Now the chance of two successive cards matching is $\frac{1}{13}$, so I pay \$12 for every \$14 I win.

Each game individually loses money, but alternate them and you win money. Eureka! We're all rich. \RUSs
\begin{ist}
Wilmott forum.
\end{ist}
 }
\solution{ }

\problem{ \zdt{Триэль }

Три гусара "--- $A$, $B$ и $C$ "--- стреляются за прекрасную даму. Стреляют
они по очереди ($A$, $B$, $C$, $A$, $B$, $C$, \ldotst), каждый стреляет в
противника по своему выбору. $A$ попадает с вероятностью 0.1, $B$
"--- 0.5, $C$ "--- 0.9. Триэль продолжается до тех пор, пока в живых не
останется только один. Предположим, что стрелять в воздух нельзя.
\begin{enumerate}
\item Какой должна быть стратегия $A$?
\item У кого какие шансы на победу?
\end{enumerate}
 }
\solution{ }

\problem{ \zdt{Триэль-2}

Три гусара "--- $A$, $B$ и $C$ "--- стреляются за прекрасную даму. Стреляют
они одновременно, каждый стреляет в противника по своему выбору.
$A$ попадает с вероятностью 0,1, $B$ "--- 0,5, $C$ "--- 0,9. Триэль
продолжается до тех пор, пока в живых не
останется только один или никого.
\begin{enumerate}
\item Какой должна быть стратегия $A$?
\item У кого какие шансы на прекрасную даму?
\end{enumerate} }
\solution{ }


\problem{
% переписать (у Менделя - не горошины вроде бы?)
У диплоидных организмов наследственные характеристики определяются
парой генов. Вспомним знакомые нам с 9-го класса горошины чешского
монаха Менделя. Ген, определяющий форму горошины, имеет две
аллели:  <<А>> (гладкая) и <<а>> (морщинистая). <<А>> доминирует над
<<а>>. В популяции бесконечное количество организмов. Родители
каждого потомка определяются случайным образом. Одна аллель
потомка выбирается наугад из аллелей матери, другая "--- из аллелей
отца. Начальное распределение
генотипов имеет вид: <<АА>> "--- 30\,\%, <<Аа>> "--- 60\,\%, <<аа>> "--- 10\,\%.
\begin{enumerate}
\item  Каким будет распределение генотипов в $n$-м поколении?
\item  Заметив закономерность, сформулируйте и докажите теорему
Харди"--~Вайнберга для произвольного начального распределения
генотипов.
\end{enumerate}
 }
\solution{ }

\problem{
У диплоидных организмов наследственные характеристики определяются
парой генов. Некий ген, сцепленный с полом, имеет две аллели:
<<А>> и <<а>>, т.\,е. девочка может иметь один из трёх генотипов
(<<АА>>, <<Аа>>, <<аа>>), а мальчик "--- только два (<<А>> и <<а>>; хромосома,
определяющая мужской пол, короче и не содержит нужного участка).
От мамы ребёнок наследует одну из двух аллелей (равновероятно), а
от отца либо наследует (тогда получается девочка), либо нет (тогда
получается мальчик). <<А>> доминирует <<а>>. В популяции
бесконечное количество организмов. Родители каждого
потомка определяются случайным образом.
\begin{enumerate}
\item Верно ли, что численность генотипов стабилизируется со временем?
\item Известно, что дальтонизм является признаком, сцепленным с
полом. Догадавшись, является ли он рецессивным или доминантным,
определите, среди кого (мужчин или женщин) дальтоников больше.
\end{enumerate}

/проверить биологию/ }
\solution{ }



\problem{
В коробке находится четыре внешне одинаковые лампочки. Две
лампочки исправны, две "--- нет. Лампочки извлекают из коробки по
одной до тех пор, пока не будут извлечены обе исправные.
\begin{enumerate}
\item Какова вероятность того, что опыт закончится извлечением трёх
лампочек?
\item  Каково ожидаемое количество извлеченных лампочек?
\end{enumerate}
 }
\solution{ }


\problem{ \label{spelestolog} \zdt{Спелестолог и батарейки}

У спелестолога в каменоломнях сели батарейки в налобном фонаре, и он оказался в абсолютной темноте. В рюкзаке у него 8~батареек: 5 новых и 3 старых. Для работы фонаря требуется две новые батарейки. Спелестолог вытаскивает из рюкзака две батарейки наугад и вставляет их в фонарь. Если фонарь не начинает работать, то спелестолог откладывает эти две батарейки и пробует следующие и т.\,д.
\begin{enumerate}
\item Сколько попыток в среднем потребуется?
\item Какая попытка вероятнее всего будет первой удачной?
\item Творческая часть. Поиграйтесь с задачей. Случайна ли прогрессия в ответе? Сравните с вариантом «6 новых $+$ 4 старых» и т.\,д.
\end{enumerate}
}

\solution{ \begin{tabular}{|c|c|c|c|c|}\hline
$N$ & 1 & 2 & 3 & 4 \bigstrut \\ \hline
$\PP$ & $\frac{5}{14}$ & $\frac{4}{14}$ & $\frac{3}{14}$ & $\frac{2}{14}$ \bigstrut \\ \hline
\end{tabular}

Решение для $6=4+2$: $\PP(N=1)=\frac{C_{4}^{2}}{C_{6}^{2}}=\frac{6}{15}$; $\PP(N=3)=\frac{4\cdot 2}{C_{6}^{2}}\frac{3\cdot 1}{C_{5}^{2}}=\frac{4}{15}$; $\PP(N=2)=\frac{5}{15}$; $\E(N)=\frac{28}{15}$. }



\problem{ \label{dva ferzia}
Два ферзя (чёрный и белый) ставятся наугад на шахматную доску.
\begin{enumerate}
\item Какова вероятность того, что они будут <<бить>> друг друга?
\item К чему стремится эта вероятность для шахматной доски со
стороной, стремящейся к бесконечности?
\end{enumerate}
 }
\solution{Вероятность того, что ферзи будут угрожать друг другу, равна $\frac{14}{63}+\frac{1}{64}\frac{1}{63}4(7\cdot 7+5\cdot 9+3\cdot 11+ 1\cdot 13)$.
Шахматная доска делится на четыре квадратных зоны с одинаковым числом клеток, покрываемых ферзём. Если длина стороны будет стремиться в бесконечность, то эта вероятность будет стремиться к
0, так как она равна отношению длин нескольких линий ко всей площади.  }


\problem{
На день рождения к Васе пришли две Маши, два Саши, Петя и Коля. Все вместе с Васей сели за круглый стол. Какова вероятность, что Вася окажется между двумя тёзками? }
\solution{ Слева должен сесть тот, у кого есть тёзка. $p_{1}=\frac{4}{6}$. Справа должен сесть его парный. $p_{2}=\frac{1}{5}$, итого $p=p_{1}\cdot p_{2}=\frac{2}{15}$. }



\problem{
Равновероятно независимо друг от друга выбираются три числа от 1 до 20. Какова вероятность того, что третье попадет между двух первых? }
\solution{ $\frac{57}{200}=0{,}285$. }

\problem{ \ENGs Five distinct numbers are randomly distributed to players numbered 1 through 5. Whenever two players compare their numbers, the one with the higher one is declared the winner. Initially, players 1 and 2 compare their numbers; the winner then compares with player 3. Let $X$ denote the number of times player 1 is a winner. Find the distribution of $X$. \RUSs }
\solution{ }



\problem{ \label{simple optimization}
Подбрасывается правильный кубик. Узнав результат, игрок выбирает,
подкидывать ли кубик второй раз. Игрок получает сумму денег, равную
количеству очков при последнем подбрасывании.
\begin{enumerate}
\item Каков ожидаемый выигрыш игрока при оптимальной стратегии?
\item Каков ожидаемый выигрыш игрока, если максимальное количество подбрасываний равно трём?
\end{enumerate}
 }
\solution{$\frac{1}{2}\cdot5+\frac{1}{2}\frac{7}{2}=4{,}25$. }

\problem{На столе стоят 42~коробки, они занумерованы от 0 до 41. В каждой коробке 41~шар, в коробке с номером $i$ лежат $i$ белых шаров, а остальные чёрные. Мы наугад выбираем коробку, а затем из неё достаём три шара. Какова вероятность того, что они будут одного цвета?
\begin{ist}
\url{http://math.stackexchange.com/questions/70760/}
\end{ist}
} % 42 --- это потому что это число является ответом на вопрос Жизни, Вселенной и Всего Такого?
\solution{Можно представить себе другое условие: в коробке 42 занумерованных шара, мы выбираем один наугад. Красим шары с меньшим номером в белый, остальные "--- в чёрный. Затем берём три шара. Это равносильно тому, что мы возьмём 4~шара с номерами 1, 2, 3, 4 и выберем из них разделитель цветов случайно. Значит, вероятность равна $\frac{1}{2}$.}



% !Mode:: "TeX:UTF-8"
\section{Условные вероятности и ожидания. Дополнительная информация}
%Правило умножения вероятностей:
%Если A B независимы, то

\subsection{Условная вероятность}

\problem{ \ENGs
A bag contains a counter, known to be either white or black. A white counter is put in, the bag is shaken, and a counter is drawn out, which proves to be white. What is now the chance of drawing a white counter? \RUSs}
\solution{ }

\problem{ \ENGs
You have a hat in which there are three pancakes: one is golden on both sides, one is brown on both sides, and one is golden on one side and brown on the other. You withdraw one pancake, look at one side, and see that it is brown. What is the probability that the other side is brown? \RUSs}
\solution{ }

\problem{ \ENGs
The inhabitants of an island tell truth one third of the time. They lie with the probability of $\frac{2}{3}$. On an occasion, after one of them made a statement, another fellow stepped forward and declared the statement true. What is the probability that it was indeed true? \RUSs }
\solution{ }


\problem{
На кубиках написаны числа от 1 до 100. Кубики свалены в кучу. Вася выбирает наугад из кучи по очереди три кубика.
\begin{enumerate}
\item Какова вероятность, что полученные три числа будут идти в возрастающем порядке?
\item Какова вероятность, что полученные три числа будут идти в возрастающем порядке, если известно, что первое меньше последнего?
\end{enumerate}
 }
\solution{ $\frac{1}{6}$; $\frac{1}{3}$. }


\problem{ (дописать)

Наследование группы крови контролируется аутосомным геном. Три его аллеля обозначаются буквами А, В и 0. Аллели А и В доминантны в одинаковой степени, а аллель 0 рецессивен по отношению к ним обоим. Поэтому существует четыре группы крови. Им соответствуют следующие генотипы:
\begin{itemize}
\item Первая (I) "--- 00;
\item Вторая (II) "--- АА, А0;
\item Третья (III) "--- ВВ, В0;
\item Четвёртая (IV) "--- АВ.
\end{itemize}

Наследование резус-фактора кодируется тремя парами генов и происходит независимо от наследования группы крови. Наиболее значимый ген имеет два аллеля, аллель D доминантный, аллель d рецессивный. Таким образом, получаем следующие генотипы:
\begin{itemize}
\item Резус-положительный "--- DD, Dd;
\item Резус-отрицательный "--- dd.
\end{itemize}

Если у беременной женщины резус"=отрицательная кровь, а у плода резус"=положительная, то есть риск возникновения гемолитической болезни (у матери образуются антитела к резус фактору, безвредные для неё, но вызывающие разрушение эритроцитов плода).

Перед нами два семейства: Монтекки и Капулетти. \\
...}
\solution{ }


\problem{ \label{rekordnaia volna}
Пусть $X_{i}$ "--- НОРСВ, такие, что $\PP(X_{i}=X_{j})=0$. Обозначим за
$E_{k}$ событие, состоящее в том, что $X_{k}$ оказалась
<<рекордом>>, т.\,е. больше, чем все предыдущие $X_{i}$ ($i<k$). Для
определённости будем считать, что $E_{1}=\Omega$.
\begin{enumerate}
\item Найдите $\PP(E_{k})$.
\item  Верно ли, что $E_{k}$ независимы?
\item  Какова вероятность того, что второй рекорд будет зафиксирован в $n$-й момент времени?
\item  Сколько в среднем времени пройдёт до второго рекорда?
\end{enumerate}

\begin{ist}
Williams, 4.3.
\end{ist}
 }
\solution{ Какая-то из первых $k$ величин будет наибольшей. В силу \iid{}
получаем, что $\PP(E_{k})=\frac{1}{k}$. $E_k$ независимы: например, если известно,
что 10-е наблюдение было рекордом, это ничего не говорит о рекордах в первых 9-ти
наблюдениях. Вероятность второго рекорда в $n$-й момент равна $\frac{1}{n(n-1)}$,
а в среднем времени до второго рекорда пройдёт $\infty$. }

\problem{Известно, что $\PP(A \mid B)=\PP(A \mid B^{c})$. Верно ли, что $A$ и $B$ независимы?}
\solution{Да.}


\problem{ \zdt{Randomized response technique}

В анкету для чиновников включён скользкий вопрос: <<Берёте ли Вы
взятки?>>. Чтобы стимулировать чиновников отвечать правдиво,
используется следующий прием. Перед ответом на вопрос чиновник втайне от анкетирующего подкидывает специальную монетку, на гранях
которой написано <<правда>>, <<ложь>>. Если монетка выпадает
<<правдой>>, то предлагается отвечать на вопрос правдиво, если
монетка выпадает на <<ложь>>, то предлагается солгать. Таким
образом, ответ <<да>> не обязательно означает, что чиновник берёт
взятки.

Допустим, что треть чиновников берёт взятки, а монетка
неправильная и выпадает <<правдой>> с вероятностью 0{,}2.
\begin{enumerate}
\item Какова вероятность того, что чиновник ответит <<да>>?
\item  Какова вероятность того, что чиновник берёт взятки, если он
ответил <<да>>? Если ответил <<нет>>?
\end{enumerate}
\todo[inline]{Вставить построение несмещённой оценки?}
}
\solution{ }

\problem{
Пусть события  $A$  и  $B$  независимы и $\PP(B)>0$.
Чему равна  $\PP(A \mid B)$? }
\solution{ $ \PP(A \mid B)=\PP(A)$. }

\problem{
Из колоды в 52 карты извлекается одна карта наугад. Верно ли, что
события <<извлечён туз>> и <<извлечена пика>> являются
независимыми? }
\solution{ Да. }

\problem{
Из колоды в 52 карты извлекаются по очереди две карты наугад.
Верно ли, что события <<первая карта "--- туз>> и <<вторая карта "---
туз>> являются независимыми? }
\solution{ Нет. }

\problem{
Известно, что $\PP(A)=0{,}3$, $\PP(B)=0,{4}$, $\PP(C)=0{,}5$. События
$A$ и $B$ несовместны, события $A$ и $C$ независимы и
$\PP(B\mid C)=0{,}1$.
Найдите $\PP(A\cup B\cup C)$. }
\solution{ }

\problem{
Имеется три монетки. Две <<правильных>> и одна "--- с орлами по
обеим сторонам. Петя выбирает одну монетку наугад и подкидывает её
два раза. Оба раза выпадает орёл. Какова вероятность того, что
монетка <<неправильная>>? }
\solution{ }

\problem{
Самолёт упал либо в горах, либо на равнине. Вероятность того, что самолёт упал в горах, равна 0{,}75. Для поиска пропавшего самолёта выделено 10 вертолётов. Каждый вертолёт можно использовать только в одном месте. Как распределить имеющиеся вертолёты, если вероятность обнаружения пропавшего самолёта отдельно взятым вертолётом равна: $0{,}95$? $0,6$ (пасмурно)? $0{,}1$ (туман)? }
\begin{ist}
Айвазян, экзамен РЭШ.
\end{ist}
\solution{ }

\problem{
Предположим, что социологическим опросам доверяют 70\,\% жителей. Те, кто доверяет опросам, всегда отвечают искренне; те, кто не доверяет, отвечают наугад, равновероятно выбирая <<да>> или <<нет>>. Социолог Петя  в анкету очередного опроса включил вопрос: <<Доверяете ли Вы социологическим опросам?>>
\begin{enumerate}
\item Какова вероятность, что случайно выбранный респондент ответит <<Да>>?
\item  Какова вероятность того, что он действительно доверяет, если известно, что он ответил <<Да>>?
\end{enumerate}
 }
\solution{ }

\problem{
Два охотника выстрелили в одну утку. Первый попадает с
вероятностью 0{,}4, второй "--- с вероятностью 0{,}6. В утку попала ровно
одна пуля. Какова вероятность того, что утка была убита первым
охотником? }
\solution{
$p=\frac{0{,}4\cdot 0{,}4}{0{,}4\cdot 0{,}4+0{,}6\cdot 0{,}6}=\frac{4}{13}$.}

\problem{
С вероятностью 0{,}3 Вася оставил конспект в одной из 10
посещённых им сегодня аудиторий. Вася осмотрел 7 из 10 аудиторий и
конспекта в них не нашёл.
\begin{enumerate}
\item  Какова вероятность того, что конспект будет найден в следующей
осматриваемой им аудитории?
\item  Какова (условная) вероятность того, что конспект оставлен
где-то в другом месте?
\end{enumerate}
 }
\solution{ }

\problem{
Вася гоняет на мотоцикле по единичной окружности с центром в
начале координат. В случайный момент времени он останавливается.
Пусть случайные величины  $X$  и  $Y$  "--- это Васины абсцисса и
ордината в момент остановки. Найдите  $\PP\left(X>\frac{1}{2} \right)$,
$\PP\left(X>\frac{1}{2} \bigm| Y<\frac{1}{2} \right)$. Являются ли события
$A=\left\{X>\frac{1}{2} \right\}$  и
$B=\left\{Y<\frac{1}{2} \right\}$  независимыми?
\begin{hint}
$\cos\left(\frac{\pi }{3} \right)=\frac{1}{2}$, длина окружности $l=2\pi r$.
\end{hint}
}
\solution{ }

\problem{
Пусть  $\PP(A)=1/4$,  $\PP(A \mid B)=\frac{1}{2}$  и $\PP(B\mid A)=\frac{1}{3}$. Найдите $\PP(A\cap
B)$, $\PP(B)$  и  $\PP(A\cup B)$.}
\solution{ }

\problem{
Примерно\footnote{Цифры условные. Celui qui ne mange pas de
bifsteak au cause de la vache folle --- il est fou! Jolivaldt.} 4\,\%
коров заражены <<коровьим бешенством>>. Имеется тест, позволяющий
с определённой степенью достоверности установить, заражено ли мясо
прионом или нет. С вероятностью $0{,}9$ заражённое мясо будет признано
заражённым. <<Чистое>> мясо будет признано заражённым с
вероятностью 0{,}1. Судя по тесту, эта партия мяса заражена. Какова
вероятность того, что она действительно заражена?}
\solution{ }

\problem{
\emph{Роме Протасевичу, искавшему со мной у Мутновского
вулкана в
тумане серую палатку...}

Есть две тёмные комнаты, $A$ и $B$. В одной из них сидит чёрная кошка.
Первоначально предполагается, что вероятность нахождения кошки в
комнате $A$ равна $\alpha$. Вероятность найти чёрную кошку в темной
комнате (если она там есть) с одной попытки равна $p$.  Допустим,
что вы сделали $a$ неудачных попыток поиска кошки в комнате $A$ и
$b$ неудачных попыток в комнате $B$.
\begin{enumerate}
\item Чему равна условная вероятность нахождения кошки в комнате $A$?
\item  При каком условии на $(a-b)$ эта вероятность будет больше
$0{,}5$?
\end{enumerate}
 }
\solution{ }

\problem{
Кубик подбрасывается два раза. Найдите вероятность
получить сумму, равную 8, если на первом кубике выпало 3.}
\solution{ $\frac{1}{6}$. }

\problem{
В коробке 10 пронумерованных монеток, $i$-я монетка выпадает
орлом с вероятностью $\frac{i}{10}$. Из коробки была вытащена одна
монетка наугад. Она выпала орлом. Какова вероятность того, что это
была пятая монетка? }
\solution{
$\frac{1}{11}$.  }

\problem{ Вы играете две партии в шахматы против незнакомца. Равновероятно
незнакомец может оказаться новичком, любителем или профессионалом.
Вероятности вашего выигрыша в отдельной партии, соответственно,
будут равны 0{,}9; 0{,}5; 0{,}3.
\begin{enumerate}
\item Какова вероятность выиграть первую партию?
\item Какова вероятность выиграть вторую партию, если вы выиграли
первую?
\end{enumerate}
 }

\solution{ $p_{a}=\frac{1}{3}(0{,}9+0{,}5+0{,}3)=\frac{17}{30}$, $p_{b}=\frac{1}{3}(0{,}9^{2}+0{,}5^{2}+0{,}3^{2})/p_{a}=\frac{115}{170}$. }

\problem{
В каких из перечисленных случаев вероятность наличия флэша (см. \hyperref[combo]{карточные комбинации} на стр.~\pageref{combo}) больше, чем при полном отсутствии информации:
\begin{enumerate}
\item Первая карта из имеющихся "--- это туз;
\item Первая карта из имеющихся "--- это туз бубей;
\item На руках имеется хотя бы один туз;
\item На руках имеется туз бубей.
\end{enumerate}
 }

\solution{ \ENGs Unverified, but no calculation:

An arbitrary hand can have two aces but a flush hand can't.  The
average number of aces that appear in flush hands is the same as the
average number of aces in arbitrary hands, but the aces are spread out
more evenly for the flush hands, so set (3) contains a higher fraction
of flushes.

Aces of spades, on the other hand, are spread out the same way over
possible hands as they are over flush hands, since there is only one of
them in the deck.  Whether or not a hand is flush is based solely on a
comparison between different cards in the hand, so looking at just one
card is necessarily uninformative.  So the other sets contain the same
fraction of flushes as the set of all possible hands. \RUSs }


\problem{\ENGs A man has 3 equally favorite seats to fish at. The probability with which the man can succeed at catching at each seat is 0.6, 0.7, 0.8 respectively. It is known that the man dropped the hint at one seat three times and just caught one fish. Find the probability that the fish was caught at the first seat. \RUSs}
\solution{$\approx 0{,}503$. }


\subsection{Условное среднее}


\problem{ \label{dve shkatulki} \zdt{Две шкатулки}

Васе предлагают две шкатулки и обещают, что в одной из них денег
в два раз больше, чем в другой. Вася открывает наугад одну из них
"--- в ней $a$ рублей. Вася может взять либо деньги, либо
оставшуюся шкатулку.
\begin{enumerate}
\item Правильно ли Вася считает, что ожидаемое количество денег в
неоткрытой шкатулке равно $\frac{1}{2}\left( {\frac{1} {2}a}
\right)+\frac{1}{2}( {2a} ) = 1\frac{1} {4}a$ и что
поэтому нужно изменить свой выбор?
\item Пусть известно, что в пару шкатулок кладут $3^k$ и $3^{k+1}$
рублей с вероятностью $p_k  = ( {\frac{1} {2}} )^k $. Стоит ли
Васе изменить свой выбор после того, как он открыл
первую шкатулку? Почему?
\end{enumerate}
 }
\solution{Вася считает неправильно: условное распределение суммы можно определить, только зная безусловное.

Концепция условного ожидания неприменима? Вставить это в иллюстрацию условного ожидания? При заданном безусловном распределении Васе следует сменить
свой выбор вне зависимости от того, что он увидел в первой шкатулке. Вторая открытая лучше первой открытой. Это возможно
из-за того, что безусловная ожидаемая сумма равна бесконечности
для обеих шкатулок.  }


\problem{В кабинет бюрократа скопилась очередь ещё до его открытия. Пусть время обслуживания страждущих "--- независимые экспоненциальные случайные величины. Посетитель, пришедший через $t$ минут после открытия, узнал, что первый посетитель уже ушёл, а второй ещё сидит в кабинете. Найдите ожидаемое время обслуживания первого посетителя, $\E(X_{1} \mid X_{1}\le t < X_{1}+X_{2}) $.}
\solution{$\frac{t}{2}$.}
\cat{poisson} \cat{exp} % может перекинуть в Пуассоновский процесс?


\problem{
Пете и Васе предложили одну и ту же задачу. Они могут правильно решить её с вероятностями 0{,}7 и 0{,}8 соответственно. К задаче предлагается 5 ответов на выбор, поэтому будем считать, что выбор каждого из пяти ответов равновероятен, если задача решена неправильно.
\begin{enumerate}
\item Какова вероятность несовпадения ответов Пети и Васи?
\item  Какова вероятность того, что Петя ошибся, если ответы совпали?
\item  Каково ожидаемое количество правильных решений, если ответы совпали?
\end{enumerate}
 }
\solution{ }

\problem{Автобусы ходят регулярно с интервалом в 10~минут. Вася приходит на остановку в случайный момент времени и ждёт автобуса не больше $a$ минут. Величина $a$ "--- константа из интервала $(0;10)$. Если автобус приходит меньше чем за $a$ минут, то Вася уезжает на нём. Если автобуса нет в течение $a$ минут, то Вася заходит в ближайшую кафешку перекусить и через случайное время возвращается на остановку. На второй раз он ждёт до прихода автобуса.
\begin{enumerate}
\item Какое время Вася в среднем проводит в ожидании автобуса?
\item  Постройте график получившейся функции от $a$.
\end{enumerate}
}
\solution{$f(a)=5+\frac{a(10-a)}{20}$.}

\problem{
Игрок получает 13 карт из колоды в 52 карты. \\
\begin{enumerate}
\item Какова вероятность того, что у него как минимум два туза, если
известно, что у него есть хотя бы один туз? 
\item Какова вероятность того, что у него как минимум два туза, если
известно, что у него есть туз пик? 
\item Объясните, почему эти две вероятности отличаются. 
\end{enumerate}
}
\solution{ }

\problem{ В уездном городе  $N$  проживают  $10^{7}$  человек. Каждый из них
может обладать редким даром ясновидения с вероятностью
$p=10^{-7}$ независимо от других. 
\begin{enumerate}
\item Каково ожидаемое количество ясновидящих? 
\item Известно, что Петя --- ясновидящий. Какова вероятность
найти еще одного ясновидящего в городе $N$?
\end{enumerate}
}
\solution{ 1, почти 1. }

\problem{  
Цвет глаз кодируется несколькими генами. В целом более темный цвет доминирует более светлый. Ген карих глаз доминирует ген синих. Т.е. у носителя пары bb глаза
синие, а у носителя пар BB и Bb --- карие. У диплоидных организмов
(а мы такие :)) одна аллель наследуется от папы, а одна --- от мамы.
В семье у кареглазых родителей два сына --- кареглазый и синеглазый.
Кареглазый женился на синеглазой девушке. Какова вероятность
рождения у них синеглазого ребенка?}
\solution{ }

\problem{
 У тети Маши --- двое детей, один старше другого. Предположим, что вероятности рождения мальчика и девочки равны и не зависят от дня недели, а пол первого и второго ребенка независимы. 
\begin{enumerate}
\item Известно, что хотя бы один ребенок --- мальчик. Какова
вероятность того, что другой ребенок --- девочка?
\item Тетя Маша наугад выбирает одного своего
ребенка и посылает к тете Оле, вернуть учебник по теории
вероятностей. Это оказывается мальчик. Какова вероятность того,
что другой ребенок --- девочка? 
\item Известно, что старший ребенок --- мальчик. Какова вероятность того, что другой ребенок --- девочка? 
\item На вопрос: <<А правда ли тетя Маша, что у вас есть сын, родившийся в пятницу?>>. Она ответила: <<Да>>. Какова вероятность того, что другой ребенок --- девочка?
\end{enumerate}
}
\solution{$ 2/3 $, $1/2$, $ 1/2 $, $ 14/27 $ }

\problem{ В урне 5 белых и 11 черных шаров. Два шара извлекаются по
очереди. Какова вероятность того, что второй шар будет черным?
Какова вероятность того, что первый шар --- белый, если известно,
что второй шар --- черный?}
\solution{ }

\problem{ Monty-hall \\
Вы играете в <<Поле Чудес>> и Вам предлагают <<3 шкатулки>>.
Назовем их a, b и c. В одной из трех шкатулок лежит 1000 рублей.
(Введем соответственно события A, B и C, где A означает <<деньги
лежат в
шкатулке a>>). Вы выбираете наугад одну из трех шкатулок. \\
Ведущий, который знает, где лежат деньги, убирает одну пустую
шкатулку, не выбранную Вами (среди двух не выбранных Вами
обязательно есть пустая, если таковых две, то ведущий убирает
любую наугад). Допустим, Вы выбрали шкатулку b, а ведущий после
этого убрал шкатулку c. \\
Найдите условную вероятность того, что приз лежит в выбранной Вами
шкатулке. Имеет ли Вам смысл изменить Ваш выбор?

\emph{Альтернативный вариант условия-1} \\
После того, как Вы выбрали шкатулку, ведущий открывает наугад одну
из пустых шкатулок (при этом он может открыть Вашу и разочаровать
Вас). Допустим, Вы выбрали шкатулку b, а ведущий после этого
открыл шкатулку c. Найдите условную вероятность того, что приз
лежит в выбранной Вами шкатулке. Имеет ли Вам смысл изменить Ваш
выбор? \\
\emph{Альтернативный вариант условия-2} \\
После того, как Вы выбрали шкатулку, ведущий открывает наугад одну
из оставшихся шкатулок (при этом он может оказаться открытой
шкатулка с деньгами). Допустим, Вы выбрали шкатулку b, а ведущий
после этого открыл шкатулку c и она оказалась пустой. Найдите
условную вероятность того, что приз лежит в выбранной Вами
шкатулке. Имеет ли Вам смысл изменить Ваш выбор? }

\solution{solution 1: \\
Задача эквивалентна следующей: игрок выбирает шкатулку. Затем (она не открывается) игрок выбирает оставить ее или взять обе другие. Очевидно, во втором случае шансы в два раза выше. \\
Solution 2: \\
Игрок не получает информации --- вероятность не меняется. Лучше сменить выбор.  }

\problem{ multi-stage monty hall \\
Suppose there are four doors, one of which is a winner. The host says:
<<You point to one of the doors, and then I will open one of the other non-winners. Then you decide whether to stick with your original pick or switch to one of the remaining doors. Then I will open another (other than the current pick) non-winner. You will then make your final decision by sticking with the door picked on the previous decision or by switching to the only other remaining door>>
Optimal strategy? \\
source: cut-the-knot -- Bhaskara Rao }
\solution{ stick-switch }

\problem{ В школе три девятых класса, <<А>>, <<Б>> и <<В>>, одинаковые по
численности. В <<А>> классе 30\% обожают учителя географии, в
<<Б>> классе --- 40\% и в <<В>> классе --- 70\%. Девятиклассник Петя
обожает учителя географии. Какова вероятность того, что он из
<<Б>>
класса?}
\solution{ }

\problem{ В урне 7 красных, 5 желтых и 11 белых шаров. Два шара
выбирают наугад. Какова вероятность, что это красный и белый, если
известно, что они разного цвета.}
\solution{ }

\problem{ Саша едет на день рождения к Маше и ищет её дом. Её дом находится
южнее по улице. Одна треть встречных прохожих --- местные. Местные всегда
лгут, неместные говорят правду с вероятностью $\frac{3}{4}$.
Изначально Саша оценивает вероятность того, что дом находится
южнее, как $a$. Саша спросил первого встречного прохожего и
получил ответ <<севернее>>. Как Саша изменит свою
субъективную вероятность? }
\solution{ }

\problem{ Самолет упал в горах, в степи или в море. Вероятности,
соответственно, равны $0,5$, $0,3$ и $0,2$. Если он упал в горах,
то при поиске его найдут с вероятностью $0,7$. В степи --- $0,8$, на
море --- $0,2$. Самолет искали в горах, в степи и не нашли. Какова
вероятность того, что он упал в море? }
\solution{ }

\problem{ Русская рулетка. \\
Давайте сыграем в русскую рулетку\ldots Вы привязаны к стулу и не
можете встать. Вот пистолет. Вот его барабан --- в нем шесть гнезд
для патронов, и они все пусты. Смотрите: у меня два патрона. Вы
обратили внимание, что я их вставил в соседние гнезда барабана?
Теперь я ставлю барабан на место и вращаю его. Я подношу пистолет
к вашему виску и нажимаю на спусковой крючок. Щелк! Вы еще живы.
Вам повезло! Сейчас я собираюсь еще раз нажать на крючок. Что вы
предпочитаете: чтобы я снова провернул барабан или чтобы просто
нажал на спусковой крючок? \\
\url{http://forum.eldaniz.ru/index.php?topic=293.60} }
\solution{ }

\problem{ Четыре свидетеля, A, B, C и D, говорят правду независимо
друг от друга с вероятностью $\frac{1}{3}$. A утверждает, что B
отрицает, что C заявил, что D солгал. Какова условная
вероятность того, что D сказал правду? }
\solution{ }

\problem{
Подробности о пожаре (Ах, а правда ли, что тетя Соня забыла
выключить утюг?) передаются по цепочке из четырех человек
(А-B-C-D), каждый из которых говорит следующему имеющуюся у него
информацию с вероятностью $p$, а с вероятностью $1-p$ говорит
совершенно
противоположное. D говорит, что тетя Соня утюг выключила. \\
Как зависит от $p$ условная вероятность того, что тетя Соня
действительно выключила утюг? }
\solution{ }

\problem{
Есть четыре населенных пункта $A$, $B$, $C$ и $D$. Прямая
дорога между каждыми двумя существует с вероятностью $p$. 
\begin{enumerate}
\item Какова вероятность того, что можно добраться из $A$ в $D$?
\item Какова вероятность того, что можно добраться из $A$ в $D$, если
между $B$ и $C$ нет прямой дороги? 
\end{enumerate}
}
\solution{ }



\problem{ В урне лежат 5 пронумерованных от одного до пяти шаров. По
очереди вытаскиваются два шара. Какова вероятность того, что
разница в номерах будет больше двух? Какова вероятность того, что
первым был вытащен шар с номером 2, если разница в номерах была
больше двух?}
\solution{ }

\problem{ A regular $n$-polygon has vertices numbered 0, 1, 2,\ldots, $n-1$ in clockwise. Let the vertex  0 be a starting point. When you roll a dice, you will move the coin clockwise by the number on the dice. Denote the number of the arriving vertice by $X$. Again roll a dice, you will move from the vertex $X$ to the vertex $Y$. 
\begin{enumerate}
\item Are  $X$ and $Y$ independent?
\item Find the value of $n$ such that $X$ and $Y$ are independent  
\end{enumerate}
Source: Kyoto University entrance exam/Science , Problem 6, 1st Round, 1990 }
\solution{ }

\problem{
Будем говорить, что событие $A$ благоприятствует событию $B$, если $\P(B\mid A)>\P(B)$. \\
Известно, что $A$ благоприятствует $B$, $B$ благоприятствует $C$. \\
Верно ли, что $A$ благоприятствует $C$? }
\solution{ не обязательно }

\problem{ Два неравенства
\begin{enumerate}
\item Известно, что $\P(A\mid B)>\P(A)$. Верно ли, что $\P(B\mid A)>\P(B)$?  
\item Известно, что $\P(A\mid B)>\P(B)$. Верно ли, что $\P(B\mid A)>\P(A)$?   
\end{enumerate}
}
\solution{ а) да; б) нет }

\problem{ На Древе познания Добра и Зла растет 6 плодов познания Добра и 5 плодов познания Зла. Адам и Ева съели по 2 плода. Какова вероятность того, что Ева познала Зло, если Адам познал Добро? }
\solution{ }

\problem{ A sniper has 0.8 chance to hit the target if he hit his last shot and 0.7 chance to hit the target if he missed his last shot. It is known he missed on the 1st shot and hit on the 3rd shot. \\
What is the probability he hit the second shot?}
\solution{ $8/11$ }

\problem{
Снайпер попадает в <<яблочко>> с вероятностью 0.8, если в предыдущий раз он попал в <<яблочко>>; и с вероятностью 0.7, если в предыдущий раз он не попал в <<яблочко>> или если это был первый выстрел. Снайпер стрелял по мишени 3 раза.
\begin{enumerate}
\item Какова вероятность попадания в <<яблочко>> при втором выстреле? \\
\item Какова вероятность попадания в <<яблочко>> при втором выстреле, если при первом снайпер попал, а при третьем --- промазал?
\end{enumerate}
}
\solution{
a) $p=0.7\cdot 0.8+ 0.3\cdot 0.7=0.77$ \\
b) $p=\frac{0.7\cdot0.8\cdot0.2}{0.7\cdot 0.8\cdot 0.2 + 0.7\cdot 0.2 \cdot 0.3}=\frac{8}{11}$ }

\problem{
Есть две неправильные монетки. Первая выпадает орлом с вероятностью 0.1, вторая выпадает орлом с вероятностью 0.9. Из этих двух монеток равновероятно выбирают одну и подбрасывают ее 2 раза. 
\begin{enumerate}
\item Верно ли, что результат первого и второго подбрасывания независимы? \\
\item Известно, что выбрали первую монетку. Верно ли, что результат первого и второго подбрасывания независимы? 
\end{enumerate}
}
\solution{нет, да}

\problem{
Вы равновероятно могли получить письмо из Москвы или из Игарки. Все буквы в названии города в обратном адресе кроме одной нечитаемы из-за загрязнения на конверте. Единственная различимая буква --- это буква <<а>>. Какова условная вероятность того, что письмо пришло из Москвы? }
\solution{
Из названия города случайным образом оставляем одну букву. \\ $p=\frac{0.5\frac{1}{6}}{0.5\frac{1}{6}+0.5\frac{2}{6}}=1/3$}

\problem{
Вася кидает дротик в мишень три раза. Его броски независимы друг от друга. Известно, что во второй раз он попал дальше от центра, чем в первый раз. Какова условная вероятность того, что в третий раз он попадет ближе к центру, чем в первый раз? }
\solution{
$\frac{1}{3}$ }

\problem{
В одном мешке лежат только спелые яблоки, в другом --- одинаковое количество спелых и зеленых. Вы случайным образом вытаскиваете яблоко из мешка, оно --- спелое, вы кладете его обратно. Какова вероятность, что следующее яблоко из того же мешка будет зеленым? \\
Какова вероятность, что следующее яблоко из того же мешка будет зеленым, если было $n$ попыток достать яблоко, и каждый раз вытаскивалось и клалось обратно спелое яблоко? }
\solution{
 $\frac{1}{2+2^{n}}$ }

\problem{
Three dice are rolled.  If no two show the same face, what is the probability
that one is an ``ace'' (one spot showing.)? }
\solution{ }

\problem{ Given that a throw with ten dice produced at least one ace, what is
the probability of two or more aces? }
\solution{ }

\problem{
Определение. События $A_{1}$ и $A_{2}$ называются \emph{условно
независимыми} относительно события $B$, если $\P(A_{1}\cap
A_{2}\mid B)=\P(A_{1}\mid B)\cdot \P(A_{2}\mid B)$. 
\begin{enumerate}
\item Приведите пример таких $A_{1}$, $A_{2}$ и $B$, что $A_{1}$ и
$A_{2}$, независимы, но не являются условно независимыми
относительно $B$. \\
\item Приведите пример таких $A_{1}$, $A_{2}$ и $B$, что $A_{1}$ и
$A_{2}$, зависимы, но являются условно независимыми
относительно $B$. 
\end{enumerate}
}
\solution{ }

\problem{
В урне 99 белых и один черный шар. Один шар извлекается из урны наугад. Петя сказал, что шар --- белый. Вася сказал, что шар --- белый. Какова вероятность того, что шар действительно белый, если Петя говорит правду с вероятностью 0.8, а Вася --- с вероятностью 0.9, независимо от Пети? }
\solution{ }

\problem{
У нас ходят два автобуса --- 10-ый и 20-ый. Десятый приходит через десять минут после 20-го; 20-ый --- через 20 минут после десятого. Я прихожу на остановку в случайный момент времени. 
\begin{enumerate}
\item Сколько мне в среднем ждать автобуса? 
\item Сколько мне еще в среднем ждать автобуса, если я уже прождал $m$ минут? 
\end{enumerate}
}
\solution{
 а) $frac{25}{3}$ }

% !Mode:: "TeX:UTF-8"
\section{$\Var$, $\sigma$, conditional $\Var$, conditional $\sigma$}
\subsection{Дискретный случай}

\problem{ \label{do 2-h v summe}
Кубик подбрасывают до тех пор, пока накопленная сумма очков на
гранях не превысит 2. Пусть  $X$  "--- число подбрасываний кубика.
Найдите  $\E(X)$, $\Var(X)$,
$\Var(36X-5)$, $\E(36X-17)$. }
\solution{ \begin{tabular}{|c|c|c|c|} \hline
  % after \\: \hline or \cline{col1-col2} \cline{col3-col4} ...
  $X$ & 1 & 2 & 3 \\ \hline
  $\PP$ & $\frac{24}{36}$ & $\frac{11}{36}$ & $\frac{1}{36}$ \bigstrut \\  \hline
\end{tabular}

$\E(X)=\frac{49}{36}$, $\E(36X-17)=32$. $\Var(X)=\frac{371}{1296}$, $\Var(36X-5)=371$. }


\problem{ \label{iz 5 detalei 2 brakovannih}
Из 5-ти деталей 3 бракованных. Сколько потребуется в среднем
попыток, прежде чем обнаружится первая дефектная деталь? Какова
дисперсия числа попыток?}
\solution{\begin{tabular}{|c|c|c|c|} \hline
  % after \par: \hline or \cline{col1-col2} \cline{col3-col4} ...
  $X$ & 1 & 2 & 3 \\  \hline
 $\PP$ & $\frac{12}{20}$ & $\frac{6}{20}$ & $\frac{2}{20}$ \bigstrut \\
  \hline
\end{tabular} \par
$\E(X)=\frac{3}{2}$, $\Var(X)=\frac{9}{20}$.  }


\problem{
Бросают два правильных игральных кубика. Пусть  $X$  "--- наименьшая
из выпавших граней, а  $Y$  "--- наибольшая.
\begin{enumerate}
\item  Рассчитайте  $\PP(X=3\cap Y=5)$;
\item  Найдите  $\E(X)$,  $\Var(X)$, $\E(3X-2Y)$.
\end{enumerate}
\begin{ist}
Cut the knot.
\end{ist}
 }
\solution{ }

\problem{
Из колоды в 52 карты извлекаются две. Пусть  $X$  "--- количество
тузов. Найдите закон распределения  $X$, $\E(X)$, $\Var(X)$.}
\solution{ }

\problem{  \label{Iska priglasil 3 druzei}
Иська пригласил трёх друзей навестить его. Каждый из них появится
независимо от другого с вероятностью $0{,}9$, $0{,}7$ и $0{,}5$
соответственно. Пусть $N$ "--- количество пришедших гостей.
\begin{enumerate}
\item Рассчитайте вероятности $\PP(N=0)$, $\PP(N=1)$, $\PP(N=2)$ и $\PP(N=3)$;
\item Найдите $\E(N)$ и $\Var(N)$.
\end{enumerate}
 }
\solution{$\PP(N=0)=0{,}1\cdot 0{,}3\cdot 0{,}5$; $\PP(N=3)=0{,}9\cdot 0{,}7\cdot 0{,}5$;
$\PP(N=1)=0{,}9\cdot 0{,}3\cdot 0{,}5+0{,}1\cdot 0{,}7\cdot 0{,}5+0{,}1\cdot0{,}3\cdot 0{,}5$. $\E(N)=0{,}9+0{,}7+0{,}5$; $\Var(N)=0{,}9\cdot 0{,}1+0{,}7\cdot 0{,}3+0{,}5\cdot
0{,}5$.  }


\problem{
В коробке лежат три монеты: их достоинство "--- 1, 5 и 10 копеек соответственно. Они извлекаются в случайном порядке. Пусть $X_{1}$,  $X_{2}$  и  $X_{3}$  "--- достоинства монет в порядке их
появления из коробки.
\begin{enumerate}
\item Верно ли, что  $X_{1}$  и  $X_{3}$ одинаково распределены?
\item  Верно ли, что  $X_{1}$  и  $X_{3}$ независимы?
\item  Найдите $\E(X_{2})$
\item  Найдите дисперсию $\bar{X}_{2} =\frac{X_{1} +X_{2} }{2} $.
\end{enumerate}
 }
\solution{ }

\problem{  \zdt{Easy}

Пусть $X$ "--- сумма очков, выпавших в результате двукратного подбрасывания кубика. Найдите $\E(X)$, $\Var(X)$. }
\solution{ }

\problem{
Охотник, имеющий 4 патрона, стреляет по дичи до первого
попадания или до израсходования всех патронов. Вероятность
попадания при первом выстреле равна 0{,}6, а при каждом последующем
уменьшается на 0{,}1. Найдите
\begin{enumerate}
\item  Закон распределения числа патронов, израсходованных охотником;
\item  Математическое ожидание и дисперсию этой случайной величины.
\end{enumerate}
 }
\solution{ \small

\begin{tabular}{|c|c|c|c|c|}  \hline
  % after \\: \hline or \cline{col1-col2} \cline{col3-col4} ...
  $x_{i}$ & 1 & 2 & 3 & 4 \\  \hline
  $p_i=\PP(X=x_{i})$ & $0{,}6$& $(1-0{,}6)\cdot 0{,}5 = 0{,}2$ & $(1-0{,}6)\cdot(1-0{,}5)\cdot 0{,}4 = 0{,}08$ & $1-p_{1}-p_{2}-p_{3} = 0{,}12$ \\ \hline
\end{tabular}

\normalsize $\E(X)=1{,}7$, $\Var(X)\approx 1{,}08$. }


\subsection{Непрерывный случай}
%Здесь появляется:
%Е когда задана функция плотности

\problem{  \ENGs
A large quantity of pebbles lies scattered uniformly over a
circular field; compare the labour of collecting them on by one
\begin{enumerate}
\item At the center $O$ of the field;
\item At a point $A$ on the circumference.
\end{enumerate}\RUSs
\begin{ist}
Grimmett, экзамен 1858 года в St John's College, Кембридж.
\end{ist}
 }
\solution{ }

\subsection{? Способ расчёта ожидания и дисперсии через условную}

\problem{
Число $x$ выбирается равномерно на отрезке $[0; 1]$. Затем случайно выбираются числа из отрезка $[0; 1]$ до тех пор, пока не появится число больше $x$.
\begin{enumerate}
\item  Сколько в среднем потребуется попыток?
\item  Сколько в среднем потребуется попыток, если $x$ выбирается равномерно на отрезке $[0;r]$, $r<1$?
\item  Сколько в среднем потребуется попыток, $x$ не выбирается равномерно, а имеет функцию плотности $p(t)=2(1-t)$ для $t\in[0;1]$?
\end{enumerate}
 }
\solution{ }




% !Mode:: "TeX:UTF-8"
\section{Связь между случайными величинами, Cov, Corr}
\subsection{Дискретный случай}
%(частная/маржинальная/... ф распределения)


\problem{ \label{vtoroie podkidivanie}
 У Васи есть $n$ монеток,
каждая из которых выпадает орлом с вероятностью $p$. В первом
раунде Вася подкидывает все монетки, во втором раунде Вася
подкидывает только те монетки, которые выпали орлом в первом
раунде. Пусть $R_{i}$ "--- количество монеток,
подкидывавшихся и выпавших орлом во $i$-м раунде.
\begin{enumerate}
\item Каков закон распределения величины $R_{2}$?
\item  Найдите $\Corr(R_{1},R_{2})$.
\item  Как ведет себя корреляция при $p\to 0$ и $p\to 1$? Почему?
\end{enumerate}
 }
\solution{ Закон распределения по сути эксперимента: $\bin(n; p^{2}$); $\Corr(R_{1},R_{2}) =\sqrt{\frac{p}{1+p}}$. При $p\to 0$ корреляция равна 0; при  $p\to 1$ составляет 0{,}5. Почему "--- чез = чёрт его знает.
}



\problem{<<Корреляция --- это мера линейной связи>>

Найдите все случайные величины $X$ такие, что $corr(X,X^2)=1$.


Источник: Алексей Суздальцев}

\solution{Случайные величины, принимающие два значения $x_1<x_2$, такие, что $|x_1|<|x_2|$.}



\subsection{Непрерывный случай (Cov, Corr)}


\problem{Пусть $ X $ равномерно на $ [0;1] $. Если известно, что $ X=x $, то случайная величина $ Y $ равномерна на $ [x;x+1] $. Найдите $ P(X+Y<1) $ и $ f_{X|Y}(x|y) $. Как распределено $ Y-X $?}
\solution{$ P(X+Y<1)=1/4 $, $ f_{X|Y}(x|y)=1/f(y) $ при $ y\in [x;x+1], x\in[0;1] $. Равномерно.}


\subsection{Общие свойства Cov и Var}

\problem{Верно ли, что $X$ и $Y$  независимы, если известно, что
\begin{enumerate}
\item Величина $X$ и любая функция $g(Y)$ некоррелированы?
\item Любая функция $f(X)$ и любая функция $g(Y)$ некоррелированы?
\end{enumerate}
}
\solution{В первом случае "--- нет, например, $X \sim \mN(0;1)$ и $Y=X^2$. Во втором "--- да: возьмём индикаторы и получим стандартное определение независимости}


\subsection{Преобразование случайных величин (преобразования)}


\problem{ \label{simmetria razbitia otrezka}
На отрезке равномерно и независимо выбираются две точки. Верно ли,
что длины получающихся трёх отрезков распределены одинаково? }
\solution{Да. Возьмём окружность. Наугад отметим три точки. Одну будем
трактовать как разрезающую окружность на отрезок. Две других "---  как
разрезающие отрезок на три части. }
\cat{uniform} \cat{circle_trick}


\problem{ \zdt{Птички на проводе-1}

На провод, отрезок $[0; 1]$, равномерно и независимо друг от друга
садятся $n$ птичек. Пусть $Y_{1}$,\ldots, $Y_{n+1}$ "--- расстояния
от левого столба до первой птички, от первой птички от второй и т.\,д.
\begin{enumerate}
\item Найдите функцию плотности $Y_{1}$;
\item Верно ли, что все $Y_{i}$ одинаково распределены?
\item  Верно ли, что все $Y_{i}$ независимы?
\item Найдите $\Cov(Y_{i},Y_{j})$ (вроде бы ковариации равны?);
\item  Как распределена величина $n\cdot Y_{1}$ при больших $n$? Почему?
\end{enumerate}
 }
\solution{ Пусть $X_{i}$ "--- координата $i$-ой птички. $\PP(Y_{1}\le t)=1-\PP(\min\{X_{i}\}>t)=1-(1-t)^{n}$. Далее, $\lim \PP(nY_{1}\le t)=\lim 1-\left(1-\frac{t}{n}\right)^{n}=1-e^{-t}$. Распределение экспоненциальное с параметром $\lambda=1$. }
\cat{uniform} \cat{circle_trick} \cat{exponential}


\problem{ \zdt{Птички на проводе-2}

На провод, отрезок $[0;1]$, равномерно и независимо друг от друга
садятся $n$ птичек. Мы берем ведро жёлтой краски и для каждой
птички красим участок провода от неё до ближайшей к ней соседки.

Какая часть провода будет окрашена при больших $n$? }
\solution{ Пусть $n$ велико, тогда $Y_{i}$ можно считать независимыми и
$nY_{i}$ "--- экспоненциально распределёнными. Не красятся только <<большие>> интервалы, т.\,е. интервалы, чья
длина больше, чем каждого из двух соседних интервалов. <<Больших>>
интервалов примерно треть. Находим $\E(B)=\E(\max\{Y_{1},Y_{2},Y_{3}\})$. $\delta=1-\E(B)\frac{n}{3}=\frac{7}{18}$. }
\begin{ist}
Marcin Kuczma.
\end{ist}

\problem{Длительность разговора клиента в минутах $X$ "--- экспоненциальная случайная величина со средним две минуты. Стоимость разговора, $Y$, составляет $5$ рублей за весь разговор, если разговор короче двух минут, и $2{,}5$ рубля за минуту, если разговор длиннее двух минут. Неполные минуты оплачиваются пропорционально, например, за 3{,}5 минуты нужно заплатить $2{,}5\cdot 3{,}5$ рублей. Найдите $\E(Y)$, $\Var(Y)$, постройте функцию распределения $Y$. }
\solution{}

\problem{Вася пришёл на остановку. Ему нужен 42-й или 21-й автобус. Время до прихода 42-го равномерно на отрезке $[0;10]$ минут, время до прихода 21-го равномерно на отрезке $[0;20]$ минут. Время прихода 42-го и время прихода 21-го "--- независимые величины. Обозначим за $Y$  время, которое Вася проведёт в ожидании на остановке. Найдите функцию плотности $Y$, $\E(Y)$, $\Var(Y)$. }
\solution{}


\subsection{Прочее про несколько случайных величин}

\problem{\zdt{Парадокс голосования}
\par
Пусть $X$, $Y$, $Z$ "--- дискретные
случайные величины, их значения попарно различны с вероятностью 1.
Докажите, что $\min\left\{\PP(X>Y),\PP(Y>Z),\PP(Z>X)\right\}\le \frac{2}{3}$.
Приведите пример, при котором эта граница точно достигается. }
\solution{}


% !Mode:: "TeX:UTF-8"
\section{Приёмы решения}
\subsection{Разложение в сумму}

\problem{ \label{sudba-don-juan-2}\zdt{Судьба Дон-Жуана-2} (см. тж. с.~\pageref{sudba-don-juan-1})

У Васи $n$  знакомых девушек, их всех зовут по-разному. Он пишет
им $n$  писем, но по рассеянности раскладывает их в конверты
наугад. Случайная величина $X$ обозначает количество девушек, получивших письма,
написанные лично для них. Найдите $\E(X)$, $\Var(X)$. }
\solution{$\E(X)=1$, $\Var(X)=1$. }



\problem{ \label{cube-cut-2}(см. тж. с.~\pageref{cube-cut-1}) \ENGs

A wooden cube that measures 3 cm along each edge is painted red. The painted cube is then cut into 27 pieces of 1-cm cubes. If I tossed all the small cubes in the air, so that they landed randomly on the table, how many cubes should I expect to land with a painted face up? \RUSs}
\solution{ $9$.}


\problem{Вокруг новогодней ёлки танцуют хороводом 27 детей. Мы считаем, что ребенок высокий, если он выше обоих своих соседей. Сколько высоких детей в среднем танцует вокруг елки? Вероятность совпадания роста будем считать равной нулю.}
\solution{Для трёх детей вероятность того, что тот, что посередине "--- самый высокий, равна $ \frac{1}{3} $, значит математическое ожидание равно $ \frac{27}{3}=9$. }

\problem{Маша собирает свою дамскую сумочку. Есть $n$ различных предметов, которые она туда может положить. Каждый предмет она кладёт независимо от других с вероятностью $p$.
\begin{enumerate}
\item Пусть $X$ "--- количество положенных предметов. Найдите $\E(X)$ и $\Var(X)$.
\item При каком $p$ вероятность положить в сумку любой заданный набор вещей не будет зависеть от конкретного набора?
\end{enumerate}}
\solution{ Биномиальное распределение, $\E(X)=np$, $\Var(X)=np(1-p) $. При $p=0{,}5$ все подмножества будут равновероятны.}


\problem{ Игральный кубик подбрасывается 100 раз. Найдите ожидаемую
сумму очков, дисперсию суммы, стандартное отклонение суммы.}
\solution{ }


\problem{ Гипергеометрическое распределение \\
В задачнике $N$ задач. Из них $a$ --- Вася умеет решать, а остальные не умеет. На экзамене предлагается равновероятно выбираемые $n$ задач. Величина $X$ --- число решенных Васей задач на экзамене. 

Найдите $\E(X)$ и $\Var(X)$ }
\solution{ 
$X=X_{1}+...+X_{n}$, $\E(X)=n\frac{a}{N}$ \\
$\Var(X_{i})=\frac{a(N-a)}{N^{2}}$ \\
$\Cov(X_{i},X_{1}+...+X_{N})=0$ \\
$\Cov(X_{i},X_{j})=-\frac{\Var(X_{i}}{N-1}$ \\
$\Var(X)=n\Var(X_{i})\frac{N-n}{N-1}$}

\problem{ 
Кубик подбрасывается $n$ раз. Величина $X_{1}$ ---
число выпадений 1, а $X_{6}$ --- число выпадений 6. Найдите $\Corr(X_{1},X_{6})$ \\
Подсказка: $\Cov(X_{1},X_{1}+...+X_{6})$ вам в помощь... }
\solution{$\Cov(X_{1},X_{1}+...+X_{6})=0$, т.к. $X_{1}+...+X_{6}=const$ 
$\Corr(X_{1},X_{6})=-\frac{1}{5}$ }

\problem{ По 10 коробкам наугад раскладывают 7 карандашей. Каково
среднее количество пустых коробок? }
\solution{$10\cdot (1-\frac{9}{10}^{7})$ }

\problem{ $[$Mosteller$]$ Среднее число совпадений \\
Из хорошо перетасованной колоды на стол последовательно
выкладываются карты лицевой стороной наверх, после чего
Аналогичным образом выкладывается вторая колода, так что каждая
карта первой колоды лежит под картой из второй колоды. Каково
среднее  число совпадений   нижней  и верхней  карт?}
\solution{ $1$ }


\problem{ Grimmett, 3.3.3. \\
В группе 20 человек. Каждый из них подбрасывает по кубику. Найдите
ожидаемый выигрыш и дисперсию выигрыша группы, если: 
\begin{enumerate}
\item за каждую пару игроков, выкинувших одинаковое количество очков,
группа получает один тугрик \\
\item за каждую пару игроков, выкинувших одинаковое количество очков,
группа получает эту сумму в тугриках  
\end{enumerate}
}
\solution{ }

\problem{ Coupon's collector problem \\
Внутри упаковки шоколадки <<Веселые животные>> находится наклейка
с изображением одного из 30 животных. Предположим, что все
наклейки равновероятны. Большой приз получит каждый, кто соберет
наклейки всех животных. Какое количество шоколадок в среднем нужно
купить, чтобы выиграть большой приз? }
\solution{ }

\problem{ $[$Mosteller$]$ \\
В $n$ урн случайным образом бросают один за одним $k$ шаров.
Найдите математическое ожидание числа пустых урн. }
\solution{ }


\problem{
У Маши 30 разных пар туфель. И она говорит, что мало! Пес
Шарик утащил 17 туфель без разбору на левые и правые. Сколько
полных пар в среднем осталось у Маши? Сколько полных пар в среднем
досталось Шарику? }
\solution{ }

\problem{
Из колоды в 52 карты извлекается 5 карт. Сколько в среднем
извлекается мастей? Достоинств? Тузов? }
\solution{ Масть: $4\cdot (1-\frac{C_{39}^{5}}{C_{52}^{5}})$ 

Достоинство: $13\cdot (1-\frac{C_{48}^{5}}{C_{52}^{5}})$ 

Туз: $4\cdot \frac{5}{52}$ }


\problem{
На карточках написаны числа от 1 до $n$. В игре участвуют
$n$ человек. В первом туре каждый получает случайным образом по
одной карточке. Во втором туре карточки выдаются заново. Призы
раздаются по следующему принципу: Человек не получает приз, только
если найдется кто-то другой, кто получил большие числа в каждом
туре.
Каково среднее количество человек, получивших приз? \\
Взято с www.zaba.ru, какая-то олимпиада. }
\solution{ }

\problem{
А.А. Мамонтов сидит в 424 аудитории. Эконометрику
собираются сдавать несколько человек. На поиски пустых аудиторий
послано 3 студента-разведчика. На втором этаже 9 учебных
аудиторий, 5 из них заняты. Каждый из 3 студентов-разведчиков
независимо друг от друга заглядывает в 3 аудитории. Если студент
обнаруживает пустую аудиторию, то он сообщает ее номер А.А.
Мамонтову. Каково среднее
количество обнаруженных пустых аудиторий? }
\solution{ }


\problem{
Вася пишет друг за другом наугад 100 букв из латинского алфавита. 
\begin{enumerate}
\item Каково ожидаемое количество букв, встречающихся в написанном <<слове>> ровно один раз?
\item Как изменилась бы искомая величина, $a_{k,n}$, если бы в алфавите было $k$ букв, а Вася писал бы <<слово>> из $n$ букв?
\item Найдите $\lim_{n\to\infty} a_{k,n}$, $\lim_{k\to\infty} a_{k,n}$    
\end{enumerate}
}
\solution{ }

\problem{
За круглым столом сидят в случайном порядке $n$ супружеских пар, всего --- $2n$ человек. Величина $X$ --- число пар, где супруги оказались напротив друг друга. 

Найдите $\E(X)$ и $\Var(X)$ }
\solution{ }

\problem{
Suppose there were $m$ married couples, but that $d$ of these $2m$ people have died. Regard the $d$ deaths as striking the $2m$ people at random. Let $X$ be the number of surviving couples. Find: $\E(X)$ and $\Var(X)$ }
\solution{ }



\problem{
Над озером взлетело 20 уток. Каждый из 10 охотников
стреляет в утку по своему выбору. 
\begin{enumerate}
\item Каково ожидаемое количество убитых уток, если охотники стреляют без промаха? 
\item Как изменится ответ, если вероятность попадания равна 0,7?
\item Каким будет ожидаемое количество охотников, попавших в цель?
\end{enumerate}
}
\solution{ }


\problem{
В каждой из двух урн находится по 50 белых и 50 черных шаров. Вася одновременно вытаскивает по шару из каждой урны и выбрасывает их.
Величина $X$ --- количество раз, когда из обеих урн были одновременно вытащены белые шары. 

Найдите $\E(X)$, $\Var(X)$ }
\solution{ }

\problem{
На карточках написаны числа от 1 до $n$. Вася достает их одну за другой наугад. Если номер карточки является соседним с номером предыдущей карточки, то Вася получает 1 рубль. Величина $X$ --- Васин выигрыш. \\
Найдите $\E(X)$, $\Var(X)$ }
\solution{ }

\problem{
Вася называет наугад 50 чисел от 1 до 100, допускаются повторения, а Петя называет наугад 50 чисел от 1 до 100 без повторов. \\
Величины $X$ и $Y$ это суммы этих чисел. 
\begin{enumerate}
\item Сравните $\E(X)$ и $\E(Y)$ 
\item Сравните $\Var(X)$ и $\Var(Y)$  
\end{enumerate}
}
\solution{ $\E(X)=\E(Y)$, $\Var(X)>\Var(Y)$}



\problem{
Кубик подбрасывается до тех пор, пока каждая грань не
выпадет по разу. Найдите математическое ожидание и дисперсию числа
подбрасываний. }
\solution{ }

\problem{
Правильная монетка подбрасывается  $n$  раз. Серия --- это
последовательность подбрасываний из одинаковых результатов. К
примеру, в последовательности ОООРРО три серии. 
\begin{enumerate}
\item Каково ожидаемое количество серий? \\
\item Дисперсия числа серий? \\
\item А если монетка неправильная и выпадает гербом с вероятностью  $p$? 
\end{enumerate}
}
\solution{ }


\problem{ В здании 10 этажей, на каждом этаже 30 окон. Вечером в каждом окне независимо от других свет включается с вероятностью $p$. 
\begin{enumerate}
\item Чему равно ожидаемое количество <<ноликов>> на фасаде здания? 
\item Чему равно ожидаемое количество <<крестиков>> на фасаде здания?   
\item При каких $p$ эти количества максимальны? Минимальны?  
\end{enumerate}
Примечание: два разных нолика могут иметь общие точки \\
Вставить рисунок нолика и рисунок крестика, пример подсчета }
\solution{ }

\problem{ 
Если смотреть на корпус Ж здания Вышки с Дурасовского переулка, то видно 40 окон. (??? Видно 7 этажей, первый не видно, и 8 окон на каждом этаже, уточнить по месту). Допустим, что каждое из них освещено вечером независимо от других с вероятностью одна вторая. Назовем <<уголком>> комбинацию из 4-х окон, расположенных квадратом, в которой освещено ровно три окна (не важно, какие). Величина $X$ --- число <<уголков>>, возможно пересекающихся, на всем корпусе Ж. \\
Найдите  $\E(X)$ и $\Var(X)$ \\
Примечание - для наглядности: \\
\begin{tabular}{|c|c|}
  \hline
  X & X\\
  \hline
    & X \\
  \hline
\end{tabular},
\begin{tabular}{|c|c|}
  \hline
  X & \\
  \hline
  X & X \\
  \hline
\end{tabular},
\begin{tabular}{|c|c|}
  \hline
   & X\\
  \hline
  X & X \\
  \hline
\end{tabular},
\begin{tabular}{|c|c|}
  \hline
  X & X\\
  \hline
  X &  \\
  \hline
\end{tabular} - это <<уголки>>. \\
\begin{tabular}{|c|c|c|}
  \hline
  X & X & X\\
  \hline
    & X & \\
  \hline
  X & X & \\
  \hline
\end{tabular} - в этой конфигурации три <<уголка>>;
\begin{tabular}{|c|c|c|}
  \hline
  X &  & X\\
  \hline
    & X & \\
  \hline
  X &  & X\\
  \hline
\end{tabular} - а здесь - ни одного <<уголка>>. 
}
\solution{ }

\problem{ В урне  $n$  шаров пронумерованных 1,2,... $n$. Наугад
вытаскивают $k$. Найдите ожидание и дисперсию суммы номеров. }
\solution{ }

\problem{ У Пети стопка из $n$ номеров газеты <<Вышка>> лежащих в случайном
порядке. Петя сортирует газеты следующим образом. Он
последовательно просматривает стопку сверху вниз. Если
просматриваемый выпуск более свежий, чем лежащий сверху стопки, то
Петя перекладывает более свежий выпуск наверх стопки и
начинает просматривать стопку заново. \par
Сколько <<переносов>> более свежих номеров наверх в среднем будет
сделано до того момента, когда наверху окажется первый выпуск
газеты? \par
\url{http://www.artofproblemsolving.com/Forum/viewtopic.php?t=124903 } }
\solution{Solution 1: \par
$p_{2}=\frac{1}{2}$ \par
С вероятностью $\frac{n-1}{n}$ сверху стопки лежит номер, меньший
$n$, в этом случае можно считать, что $n$-ый номер вообще
отсутствует в стопке. \par
С вероятностью $\frac{1}{n}$ сверху стопки лежит $n$-ый номер,
тогда обязательно происходит одно перекладывание, после которого
мысленно выкинув $n$-ый номер можно считать, что имеется случайно
упорядоченная стопка из $(n-1)$ выпуска.\par
$p_{n}=\frac{n-1}{n}p_{n-1}+\frac{1}{n}(p_{n-1}+1)$ \par
Итого: $p_{n}=\sum_{i=2}^{n}\frac{1}{i}\approx n\ln(n)$ \par
Solution 2: \par
Пусть $q_{i}$- вероятность того, что число $i$ <<уберут>> с верха стопки.\par
$q_{1}=0$ \par
Вероятность того, что число $i$ <<уберут>> с верха стопки равна
вероятности того, что среди чисел $1$, $2$,... $i$ число $i$ будет
первым, т.е. $\frac{1}{i}$. \par
$\E(X)=\E(X_{2})+...+\E(X_{n})=\sum_{i=2}^{n}\frac{1}{i}\approx
n\ln(n)$  }




\subsection{Первый шаг}


\problem{Илье Муромцу предстоит дорога к камню. И от камня начинаются ещё три дороги. Каждая из тех дорог снова оканчивается камнем. И от каждого камня начинаются ещё три дороги. И каждые те три дороги оканчиваются камнем\ldots И так далее до бесконечности. На каждой дороге можно встретить живущего на ней трёхголового Змея Горыныча с вероятностью (хм, Вы не поверите!) одна третья. Какова вероятность того, что у Ильи Муромца существует возможность пройти свой бесконечный жизненный путь, так ни разу и не встретив Змея Горыныча?}
\solution{$p=\frac{2}{3}(1-(1-p)^{3})$, нам подходит решение $ p=\frac{3-\sqrt{3}}{2} $. }


\problem{У Пети "--- монетка, выпадающая орлом с вероятностью $ p\in (0;1) $. У Васи "--- с вероятностью $ q\in (0;1) $. Они одновременно подбрасывают свои монетки до тех пор, пока у них не окажется набранным одинаковое количество орлов. В частности, они останавливаются после первого подбрасывания, если оно дало одинаковые результаты. Сколько в среднем раз им придётся подбросить монетку?}
\solution{}

\todo[inline]{А если 5 мастей подряд вообще нет?}
\problem{Сколько в среднем нужно взять из колоды в 52 карты, чтобы насобирать подряд 5 карт одной масти?

\begin{hint}
Ответ имеет вид произведения дробей очень простого вида.
\end{hint}
}
\solution{Если у нас $m=13$ достоинств и $n=4$ масти, то ответ имеет вид: $mn\prod\limits_{i=1}^{}\frac{in}{in+1}\approx 45{,}3$.}

\problem{Вася прыгает на один метр вперёд с вероятностью $p$ и на два метра вперёд с вероятностью $1-p$. Как только он пересечёт дистанцию в 100~метров, он останавливается. Получается, что он может остановиться на отметке либо в 100~метров, либо в 101~метр. Какова вероятность того, что он остановится ровно на отметке в 100~метров?}
% копия в задачах на остановку мартингала
\solution{ Обозначим за $P_n$ вероятность остановиться ровно на $n$ метрах. Мы ищем $P_{100}$.

\textit{Решение 1.} По методу первого шага:  $P_n=pP_{n-1}+(1-p)P_{n-2}$.

\textit{Решение 2.} Попасть ровно в $n$ можно двумя способами: перелетев $n-1$ или попав в $n-1$ и сделав шаг в один метр. Значит $P_n=(1-P_{n-1})+pP_{n-1}$.

\textit{Решение 3.} Обозначим Васину координату в момент времени $t$ как $X_t$. Можно найти $a$ так, чтобы процесс $Y_t=a^{X_t}$ был мартингалом. Момент остановки $T=\min\{t \min X_t\geq n\}$. Мартингал $Y_{t\wedge T}$ ограничен, теорема Дуба применима. $\E(Y_T)=\E(Y_0)=1$. Получаем уравнение $P_n a^{n}+(1-P_n) a^{n+1}=1$.}

% untyp
\problem{
Испытания по схеме Бернулли проводятся до первого успеха, вероятность успеха в
отдельном испытании равна $p$ \par
а) Чему равно ожидаемое количество испытаний?   \par
б) Чему равно ожидаемое количество неудач? \par
в) Чему равна дисперсия количества неудач? }
\solution{ $\frac{1}{p}$, $\frac{q}{p}$ \par
в) $E(X^{2})=p\cdot 1+q\cdot E((X+1)^{2})$, $Var(X)=\frac{q}{p^2}$ }

% untyp
\problem{ Отрицательное биномиальное \par
Испытания по схеме Бернулли проводятся до $r$-го успеха, вероятность успеха в
отдельном испытании равна $p$ \par
а) Чему равно ожидаемое количество неудач? \par
б) Чему равна дисперсия количества неудач? }
\solution{ (устно, при сделанной предыдущей задаче) $\frac{rq}{p}$, $Var(X)=\frac{rq}{p^2}$ }

% untyp
\problem{
Саша и Маша по очереди подбрасывают кубик. Посуду будет
мыть тот, кто первым выбросит шестерку. Маша бросает первой.
Какова вероятность того, что Маша будет мыть посуду? }
\solution{ }

% untyp
\problem{
Саша и Маша решили, что будут рожать нового ребенка, до тех
пор, пока в их семье не будут дети обоих полов. Каково ожидаемое
количество детей? }
\solution{ }

% untyp
\problem{
Четыре человека играют в игру <<белая ворона платит>>. Они
одновременно подкидывают монетки. Если три монетки выпали одной
стороной, а одна - по-другому, то <<белая ворона>> оплачивает всей
четверке ужин в ресторане. Если <<белая ворона>> не определилась,
то монетки подбрасывают снова. Сколько в среднем нужно
подбрасывания для определения <<белой вороны>>? }
\solution{ }

% untyp
\problem{
Саша и Маша каждую неделю ходят в кино. Саша доволен
фильмом с
вероятностью 1/4, Маша - с вероятностью 1/3. \par
a) Сколько недель в среднем пройдет до тех пор, пока кто-то не
будет доволен? \par
b) Какова вероятность того, что первым будет доволен Саша? \par
c) Сколько недель в среднем пройдет до тех пор, пока каждый не
будет доволен хотя бы одним просмотренным фильмом? }
\solution{ }

% untyp
\problem{
По ответу студента на вопрос преподаватель может сделать
один из трех выводов: ставить зачет, ставить незачет, задать еще
один вопрос. Допустим, что знания студента и характер
преподавателя таковы, что при ответе на отдельный вопрос зачет
получается с вероятностью $p_{1}=3/8$, незачет --- с вероятностью
$p_{2}=1/8$. Преподаватель задает вопросы до тех пор, пока не
определится
оценка. \par
а) Сколько вопросов в среднем будет задано? \par
б) Какова вероятность получения зачета? }
\solution{ }

% untyp
\problem{
Вы играете в следующую игру. Кубик подкидывается неограниченное число раз. Если на кубике выпадает 1, 2 или 3, то соответствующее количество монет добавляется на кон. Если выпадает 4 или 5, то игра оканчивается и Вы получаете сумму, лежащую на кону. Если выпадает 6, то игра оканчивается, а Вы не получаете ничего. \par
а) Чему равен ожидаемый выигрыш в эту игру? \par
б) Изменим условие: если выпадает 5, то набранная сумма сгорает, а игра начинается заново. Чему будет равен ожидаемый выигрыш? }
\solution{ 
a) $V(x)=\frac{1}{6}(V(x+1)+V(x+2)+V(x+3)+2x+0)$ \par
Ищем линейную $V(x)$, получаем $V(x)=\frac{2}{3}x+\frac{4}{3}$ \par
б) $V(x)=\frac{1}{6}(V(x+1)+V(x+2)+V(x+3)+x+V(0)+0)$ }

% untyp
\problem{ Вася подкидывает кубик. Если выпадает единица, или Вася говорит
<<стоп>>, то игра оканчивается, если нет, то начинается заново.
Васин выигрыш - последнее выпавшее число. Как выглядит оптимальная
стратегия? Как выглядит оптимальная стратегия, если за каждое
подбрасывание Вася платит 35 копеек?\cite{stirzaker:otep}}
\solution{ }

% untyp
\problem{
Саша и Маша подкидывают монетку бесконечное количество раз. Если сначала появится
РОРО, то выигрывает Саша, если сначала появится ОРОО, то - Маша. \par
а) У кого какие шансы выиграть? \par
b) Сколько в среднем времени ждать до появления РОРО? До ОРОО?
с) Сколько в среднем времени ждать до определения победителя? }
\solution{ }

% untyp
\problem{ \label{mishka ishet sir}
Есть три комнаты. В первой из них лежит сыр. Если мышка
попадает в первую комнату, то она находит сыр через одну минуту.
Если мышка попадает во вторую комнату, то она ищет сыр две минуты
и покидает комнату. Если мышка попадает в третью комнату, то она
ищет сыр три минуты и покидает комнату. Покинув комнату, мышка
выходит в коридор и выбирает новую комнату наугад (т.е. может
зайти в одну и ту же). Сейчас мышка в коридоре. Сколько времени ей
в среднем потребуется, чтобы найти сыр? }
\solution{ $m=\frac{1}{3}+\frac{1}{3}(2+m)+\frac{1}{3}(3+m)$, $m=6$ }

% untyp
\problem{
Иська и Еська по очереди подбрасывают два кубика. Иська
бросает первым. Иська выигрывает, если при своем броске получит 6
очков в сумме на двух кубиках. Еська выигрывает, если при своем
броске получит 7 очков в сумме на двух кубиков. Кубики
подбрасываются до
тех пор, пока не определится победитель. \par
а) Верно ли, что события $A=\{$на двух кубиках в сумме выпало
больше 5 очков$\}$ и $B=\{$на одном из кубиков выпала 1$\}$ являются независимыми? \par
б) Какова вероятность того, что Еська выиграет? }
\solution{ }

% untyp
\problem{
Players A and B play a (fair) dice game. <<A>> deposits one coin and
they take turns rolling a single dice, <<B>> rolling first. \par
If <<B>> rolls an even number, he collects a coin from the pot. If
he rolls an odd number, he put a coin (coins with same values
always). If <<A>> (plays and) rolls an even number, he collects a
coin but if he rolls an odd number, he does NOT add a coin. The
game continues until the pot is exhausted. \par
Question: what is the probability that <<A>> wins this game (that
is, exhaust the pot) ? \par
t=138358}
\solution{ }


\problem{
Вам предложена следующая игра. Изначально на кону 0 рублей. Раз за разом подбрасывается правильная монетка. Если она выпадает орлом, то казино добавляет на кон 100 рублей. Если монетка выпадает решкой, то все деньги, лежащие на кону, казино забирает себе, а Вы получаете красную карточку. Игра прекращается либо когда Вы получаете третью красную карточку, либо в любой момент времени до этого по Вашему выбору. Если Вы решили остановить игру до получения трех красных карточек, то Ваш выигрыш равен сумме на кону. При получении третьей красной карточки игра заканчивается и Вы не получаете ничего. 
\begin{enumerate}
\item Как выглядит оптимальная стратегия в этой игре? 
\item Чему при этом будет равен средний выигрыш?  
\end{enumerate}
}
\solution{ }

\problem{ Китайский ресторан \par
Каждый момент времени в китайский ресторан приходит новый посетитель.
Если сейчас в ресторане сидит $n$ человек, а за конкретным столиком сидит $b$ человек, то вероятность того, что новый посетитель присоединится к этому столику равна $\frac{b}{n+\theta}$. С вероятностью $\frac{\theta}{n+\theta}$ посетитель сядет за отдельный столик. \par
Каково ожидаемое число занятых столиков к моменту времени $n$? }
\solution{ }


\problem{ Вася бьет мячом по воротам 100 раз. В первый раз вероятность
попасть равна $frac{1}{2}$, в каждый последующий раз вероятность
попасть увеличивается --- Вася становится метче; при этом разные
удары независимы. Какова вероятность того, что Вася попадет в
ворота четное число раз? }
\solution{$\frac{1}{2}$ }

\problem{ Вася нажимает на пульте телевизора кнопку <<On-Off>> 100 раз
подряд. Пульт старый, поэтому в первый раз кнопка срабатывает с
вероятностью $\frac{1}{2}$, затем вероятность срабатывания падает.
Какова вероятность того, что после всех нажатий телевизор будет
включен, если сейчас он выключен? }
\solution{$\frac{1}{2}$ }


\problem{ Вы в тире, и у Вас 100 патронов. С вероятностью $0.01$ Вы попадает в глаз Усамы Бен Ладена, за что получаете 20 дополнительных патронов, с вероятностью $0.05$ Вы попадаете в нос Усамы Бен Ладена, за что получаете 5 дополнительных патронов. Вы стреляете до тех пор, пока патроны не кончатся. Сколько в среднем Вы сделаете выстрелов? 

\url{www.wilmott.com-forum-brainteasers} }
\solution{ Во первых, заметим, что ожидаемое количество выстрелов, если у Вас осталось $n$ патронов имеет вид $E_{n}=k\cdot n$. \par
Во-вторых, получим уравнение на $E_{n}$: \par
$E_{n}=1+E_{n-1}+kE(X)$, где $\E(X)$ - ожидаемый выигрыш патронов от одного выстрела. \par
Находим $k$: $k=\frac{1}{1-\E(X)}$ \par
Ответ задачи: $\frac{100}{1-0.45}$ \par
Solution2: Интуитивно: $100+100\cdot 20\cdot 0.01+100(\cdot 20\cdot)^{2}+...$  }



\problem{
В вершинах треугольника три ежика. С вероятностью $p$ каждый ежик
независимо от других двигается по часовой стрелке, с вероятностью
$(1-p)$ он двигается против часовой стрелки. Сколько в среднем
пройдет времени прежде,
чем они встретятся в одной вершине? \par
При каком $p$ ожидаемое время встречи минимально? }
\solution{ 
У системы 4 состояния (1-1-1, 1-2-0, 2-1-0, 3-0-0). \par
Пишем три уравнения на ожидаемые времена. \par
Решая находим $E(T)=\frac{3}{p(1-p)}$ \par
Минимум при $p=0.5$ \par
Ответ слишком красивый... красивое решение???? }


\problem{Монетка выпадает орлом с вероятностью $p$. Монетку подбрасывают до тех пор, пока впервые не выпадет орёл. Какова вероятность того, что будет сделано чётное число подбрасываний?}
\solution{$\P(A)=(1-p)/(2-p)$}



\subsection{Аллюзии на принцип Белмана}


\problem{
Начинающая певица дает концерты каждый день. Каждый ее концерт приносит продюсеру 0.75 тысяч евро. После каждого концерта певица может впасть в депрессию с вероятностью 0.5. Самостоятельно выйти из депрессии певица не может. В депрессии она не в состоянии проводить концерты. Помочь ей могут только цветы от продюсера. Если подарить цветы на сумму $0\le x\le 1$ тысяч евро, то она выйдет из депрессии с вероятностью $\sqrt{x}$. Дисконт фактор равен $0.8$. \\
Какова оптимальная стратегия продюсера? }
\solution{ 
Рассмотрим совершенно конкурентный невольничий рынок начинающих певиц. Певицы в хорошем настроении продаются по $V_{1}$, в депрессии - по $V_{2}$. \\
$V_{1}=0.75+\delta(0.5V_{1}+0.5V_{2})$ \\
$V_{2}=max_{x}{-x+\delta(\sqrt{x}V_{1}+(1-\sqrt{x})V_{2})}$ }

\problem{ Будучи незамужней Маша испытывает отрицательную полезность $-c$ каждый день. Каждый день она знакомится с новым ухажером и может тут же выскочить за него замуж. Каждый ухажер характеризуется параметром $X$, полезностью, которую Маша получит в день свадьбы с ним. Вы о чем подумали? Величина $X$ распределено равномерно на $[0;1]$. Ежедневная полезность Маши от замужнего состояния после дня свадьбы равна 0. Дисконт фактор (с которым дисконтируется Машина полезность) равен $\delta$. 
\begin{enumerate}
\item Как выглядит оптимальная стратегия Маши, если она выбирает мужа на всю жизнь?
\item Как выглядит оптимальная стратегия Маши, если она легко может развестись? 
\end{enumerate} }
\solution{ }



\subsection{\textit{o}-малое}

\problem{Случайные величины $ X_{1} ,\ldots, X_{n} $ одинаково распределены с функцией плотности $ p(t) $ и независимы. Найдите функцию плотности третьего по величине $ X_{(3)}$.}
\solution{$ \PP(X_{(3)}\in [x;x+dx])= C_{n}^{2}C_{n-2}^{1} \bigl((F(x)+o(x)\bigr)^{n-3}\bigl(1-F(x)+o(x)\bigr)^{2}\bigl((f(x)+o(x)\bigr)dx $. Значит, искомая функция плотности равна $f_{X_{(3)}}(t)=f(t)F(t)^{n-3}\bigl(1-F(t)\bigr)^{2}$.}

\subsection{Вероятностный метод}
% задачи не по теории вероятностей, которые решаются с помощью теории вероятностей

% untyp
\problem{ На потоке 200 студентов. На контрольной было 6 задач. Известно, что каждую задачу решило не менее 120 человек.
Всегда ли преподаватель может выбрать двух студентов из потока так, что эти двое могут решить всю контрольную совместными усилиями?}
\solution{Выберем двух студентов из потока наугад. Вероятность того, что ни один из них не решил задачу \No\,1, не превосходит $\br{\frac{80}{200}}^2=0{,}16$.
Вероятность того, что ни один из них не решил задачу \No\,2, не превосходит 0{,}16 (по тем же причинам), и это справедливо для каждой из шести задач. Вероятность того, что хотя
бы одну задачу они на пару не решили, не превосходит суммы этих вероятностей, т.\,е. $0{,}16\cdot6=0{,}96$.
Значит, вероятность выбора пары студентов, которые совместными усилиями могут решить экзамен, не менее $0{,}04$. Значит, хотя бы одна такая пара существует.}


\subsection{Склеивание отрезка}


\problem{Машина может сломаться равновероятно в любой точке на дороге от города А до города Б. Когда машина сломается мы будем толкать ее до ближайшего сервиса. Где должны быть расположены три автосервиса чтобы минимизировать ожидаемую продолжительность толкания? А если автосервисов будет $n$?}
\solution{Проверить. Разбиваем отрезок на $n$ частей, ставим автосервис в центр каждой части
\url{http://math.stackexchange.com/questions/37254/} }



\problem{ Рулет \\
Длинный рулет разрезан на $n$ частей. Каждый из $k$ гостей по очереди забирает себе один кусочек, выбираемый случайным образом. В результате остается $n-k$ кусочков рулета. Оставшиеся кусочки рулета лежат <<сериями>>, разделенными <<дырками>> от забранных кусочков. Каково ожидаемое число <<серий>> оставшихся кусочков? К чему стремится эта величина при $n\to\infty$?\\
Aвтор: Алексей Суздальцев  }
\solution{ Решение 1: \\
Величина $X$ --- число <<серий>>, $X=X_{1}+...+X_{n}$, где $X_{i}$ - индикатор, показывающий, начинается ли новая серия с $i$-го кусочка. \\
Ответ: $\frac{n-k}{n}+(n-1)\frac{k}{n}\frac{n-k}{n-1}=(k+1)\frac{n-k}{n}$ \\
Решение 2: \\
Закольцуем рулет, добавив в него еще один кусочек, для хозяина дома --- для Алексея Суздальцева. Получаем $(k+1)$ потенциальную серию. Вероятность того, что некая серия непуста, равна $\frac{n-k}{n}$. Итого, $(k+1)\frac{n-k}{n}$ }

\problem{
Петя ищет 6 нужных ему книг в стопке из 30 книг. Книги внешне не отличимы. Сколько книг в среднем ему придется пересмотреть? Просмотренные книги Петя в общую кучу не возвращает.  }
\solution{Можно считать, что Петя берет книги подряд из хорошо перетасованной стопки. Соответственно он берет 6 книг и 6 интервалов (книги до 1-ой нужной, книги от 1-ой нужной до 2-ой нужной, и т.д.). Если считать, что средняя длина всех интервалов одинаковая, то получается такой ответ: $6+6\cdot\frac{24}{7}$. \\
Доказательство того, что средняя длина всех интервалов одинаковая: \\
Расположим 30 книг по кругу. Среди этих 30 книг отметим случайным образом 7 книг и случайным образом занумеруем их от 1 до 7. Эти 7 книг разбивают круг из книг на 7 частей. В силу симметрии средняя длина каждой части одинакова и равна $\frac{24}{7}$. Будем трактовать книгу номер 1 как разбивающую круг на стопку. А книги 2-7, как нужные Пети. }



\subsection{Мартингальный метод}

Справедливую игру стратегией не испортишь!

\problem{ У Васи $100$ рублей, у Пети --- $150$. Они играют в орлянку
правильной монеткой до тех пор, пока все деньги не перейдут к
одному игроку. Какова вероятность, что победит Вася? }
\solution{Пусть $X_{n}$ - благосостояние Васи после $n$-го хода, тогда
$\E(X_{n})=100$. $\E(X_{final})=250p+0(1-p)$.  }



% !Mode:: "TeX:UTF-8"
\section{Неравенства, связанные с ожиданием}

\subsection{Чебышёв/Марков/Кантелли/Чернов}
\problem{
У последовательности неотрицательных случайных величин $X_1$, $X_2$,\ldots дисперсия постоянна, а  математическое ожидание стремится к бесконечности, $\lim \E(X)=+\infty$. Найдите $\lim \P(X_n>a)$. }
\solution{ $\P(X_n\leq q)=\P(X_n-\E(X_n)\leq a-\E(X_n))$. При больших $n$ величина $a-\E(X_n)$ отрицательна, поэтому $\P(X_n-\E(X_n)\leq a-\E(X_n))\leq \P(|X_n-\E(X_n)|\geq |a-\E(X_n)|)\leq c/(a-\E(X_n))^2$.}


\subsection{Йенсен}

\subsection{Коши"--~Шварц}


% !Mode:: "TeX:UTF-8"
\section{Компьютерные (use R or Python!)}

\subsection{Нахождение сложных сумм/поиск оптимальных стратегий}
\problem{Перед нами 10 коробок. Изначально в 1-й коробке 1 шар, во 2-й "--- 2 шара и т.\,д. Мы равновероятно выбираем одну из коробок, вытаскиваем из неё шар и кладём его равновероятно в одну из девяти оставшихся коробок. Мы повторяем это перекладывание до тех пор, пока одна из коробок не станет пустой. Пусть $N$ "--- число перекладываний. С помощью компьютера оцените $\E(N) $, $\Var(N)$.}
\solution{ $ \E(N)\approx 12{,}15$.}

\problem{В классе учатся $n$ человек. Нас интересует вероятность того, что хотя бы у двух из них дни рождения будут в соседние дни. (31 декабря и 1 января будем считать соседними). При каком $n$ эта вероятность впервые достигнет 0{,}5?}
\solution{ $\PP_{16}=0{,}482\,390\,182$, $\PP_{17}=0{,}525\,836\,596$.}

\problem{В классе 30 человек. Какова вероятность того, что есть три человека, у которых совпадают дни рождения? Найдите ответ с помощью симуляций и с помощью пуассоновского приближения. При каком количестве человек эта вероятность впервые превысит 50\,\%?}
\solution{Симуляции $p=0{,}028\,5 $, Пуассон: $p=0{,}03$.}

\problem{Сколько нужно людей, чтобы вероятность того, что в каждый день года у кого-то день рождения, впервые превысила 50\,\%?}
\solution{$\PP(T\leq k)= n^{-k}n!\left\{ \begin{array}{c} k \\ n \end{array} \right\}$ (число Стирлинга второго рода); 2287.}

\problem{Сколько в среднем нужно взять из колоды в 52 карты, чтобы насобирать подряд 5 карт одной масти? Не обязательно одной масти?}
\solution{Если у нас $m=13$ достоинств и $n=4$ масти, то ответ имеет вид: $mn\prod\limits_{i=1}^{}\frac{in}{in+1}\approx 45{,}3$; $\approx 28{,}0$.}

% untyp
\problem{Маленький мальчик торгует на улице еженедельной газетой. Покупает
он ее по 15 рублей, а продает по 30 рублей. Количество потенциальных покупателей --- случайная величина с распределением Пуассона и средним
значением равным 50. Нераспроданные газеты ничего не стоят. Пусть $n$ --- количество газет, максимизирующее ожидаемую прибыль мальчика.
\begin{enumerate}
\item Чему примерно должно быть равно значение функции распределения в
точке  $n$?
\item  С помощью компьютера найдите  $n$ и ожидаемую прибыль.
\end{enumerate}}
\solution{$n=50$, $665.51$

\inputminted{python}{src_python/newspapers_notext.py}
}



\problem{Подбрасывается правильная монетка. В любой момент вы можете сказать <<Хватит>>. Ваш выигрыш равен доле орлов на момент остановки. С помощью компьютера определите, чему равен ожидаемый выигрыш при использовании оптимальной стратегии? При решении на компьютере можно считать, что число подбрасываний ограничено скажем 500.}
\solution{Около 0.7925}

\problem{У Васи есть 100 рублей. Вася открывает карты из колоды одну за одной в случайном порядке. В колоде 26 красных и 26 чёрных карт. Перед открытием каждой карты Вася может поставить на цвет любую целую сумму рублей в пределах своего капитала. Если он угадал цвет, то его ставка возвращается удвоенной, если нет, то он теряет ставку. Задача Васи --- максимизировать ожидаемый финальный выигрыш. С помощью компьютера определите, как выглядит оптимальная стратегия и какую сумму он в среднем выигрывает? }
\solution{Ожидаемая сумму в концу игры - 808 рублей}



\subsection{Проведение симуляций}



% (к статистике) проверка простых гипотез на РЕАЛЬНЫХ (исторических) данных

\subsection{Statistics}
\problem{Петя подбрасывал две монетки неправильные монетки. Результаты подбрасывания:

Число подбрасываний первой. Число подбрасываний второй. Общее число орлов.
... ... ...

... ... ...
... ... ...


Оцените с помощью компьютера вероятности выпадения орлом для каждой монетки. Постройте доверительные интервалы.
(Красивого решения в явном виде нет).

Можно использовать нормальное приближение


}
\solution{}


\problem{Голосовать можно за трёх кандидатов: А, Б и В. Из 100 опрошенных 20 хотят голосовать за А, 40 "--- за Б, остальные "--- за В. В осях $p_{A}$, $p_{B}$ постройте 90-процентную доверительную двумерную область.}
\solution{ }




% !Mode:: "TeX:UTF-8"
\section{Пуассоновский поток и экспоненциальное распределение}

\subsection{Пуассоновский поток}


\problem{ Предположим, что кузнечики на большой поляне распределены
по пуассоновскому закону с  $\lambda=3$  на квадратный метр. Какой
следует взять сторону квадрата, чтобы вероятность найти в нем хотя
бы одного кузнечика была равна  $0,8$? }
\solution{ }


\problem{Саша красит стены в своей комнате, а Алёша "--- в своей. У каждой комнаты четыре стены. Предположим, что время покраски одной стены и для Саши, и для Алёши "--- экспоненциальная случайная величина с параметром $\lambda$. Какова вероятность того, что Саша успеет покрасить 3 стены раньше, чем Алёша "--- две?}

\solution{Каждая следующая стена равновероятно покрашена Сашей и Алёшей. Значит, нам нужны $\PP(SSS)+\PP(SSAS)+\PP(SASS)+\PP(ASSS)=\frac{5}{16}$. По другому: для простоты положим $\lambda=1$. Пусть $T$ "--- время, когда Саша закончит 3 стены. Функция плотности гамма-распределения (сумма трёх экспоненциальных) $f(t)=0{,}5t^{2}e^{-t}$. Нам нужна вероятность того, что к тому времени Алёша успеет меньше двух стен: $\int_{0}^{\infty} \PP(N_{t}<2 \mid T=t)\,dt =\ldots=\frac{5}{16}$.}

\problem{Машины подъезжают к светофору пуассоновским потоком с интенсивностью $\lambda $. Для простоты будем считать, что первая машина подъезжает в $ t=0 $. Светофор горит зелёным только в целые моменты времени, и этого достаточно чтобы пропустить одну машину, т.\,е. светофор горит красным при $ t\in(0;1) $, $ t\in(1;2) $, $ t\in(2;3) $ и т.\,д. Какой будет средняя длина очереди через продолжительное время? Чему будет равна вероятность, что очередь пуста?}
\solution{Производящая функция удовлетворяет соотношению:
\[ g(t)=\exp(\lambda (t-1))\frac{g(t)+tg(0)-g(0)}{t} \]
\[ g(t)=g(0)\frac{(t-1)\exp(\lambda (t-1))}{t-\exp(\lambda (t-1))} \]
Из условия $ g(1)\to 1 $ находим $ g(0)=1-\lambda $ и, помучившись, $\E(X_{\infty})=g'(1)=\frac{\lambda(2-\lambda)}{2\cdot(1-\lambda)} $.}

\cat{Poisson} \cat{gen_fun}

\problem{В офисе два телефона: зелёный и красный. Входящие звонки на красный "--- Пуассоновский поток событий с интенсивностью $\lambda_{1}=4$ звонка в час, входящие на зелёный "--- с интенсивностью $\lambda_2=5$ звонков в час. Секретарша Василиса Премудрая одна в офисе. Время разговора "--- случайная величина, имеющая экспоненциальное распределение со средним временем $5$ минут. Если Василиса занята разговором, то на второй телефон она не отвечает. Сколько звонков в час в среднем пропустит Василиса, потому что будет занята разговором по другому телефону? Являются ли пропущенные звонки Пуассоновским потоком? }
\solution{}

\problem{В офисе два телефона: зелёный и красный. Входящие звонки на красный "--- Пуассоновский поток событий с интенсивностью $\lambda_{1}=4$ звонка в час, входящие на зелёный "--- с интенсивностью $\lambda_2=5$ звонков в час. Секретарша Василиса Премудрая одна в офисе. Перед началом рабочего дня она подбрасывает монетку и отключает один из телефонов: зелёный "--- если выпала решка, красный "--- если выпал орёл. Обозначим за $Y$ время от начала дня до первого звонка. Найдите функцию плотности $Y$. }
\solution{}

\problem{Случайная величина $X$ имеет экспоненциальное распределение с параметром $\lambda$. Найдите медиану $X$. }
\solution{}

\subsection{Пуассоновское приближение}
% при замене на Poisson(\lambda=np) ошибка не превосходит
% min(1,1/\lambda)\sum p_{i}^{2}

\problem{ Используя пуассоновское предупреждение найдите вероятности
\begin{enumerate}
\item В гирлянде 25 лампочек. Вероятность брака для отдельной
лампочки равна 0,01. Какова вероятность того, что гирлянда
полностью исправна? 
\item По некоему предмету незачет получило всего 2\% студентов.
Какова вероятность того, что в группе из 50 студентов будет ровно
1 человек с незачетом? 
\item Вася испек 40 булочек. В каждую из них он кладет изюминку с
$p=0,02$. Какова вероятность того, что всего окажется 3 булочки с
изюмом? 
\end{enumerate} }
\solution{ }

\problem{ Вася каждый день подбрасывает монетку 10 раз. Монетка с
вероятностью 0,005 встает на ребро. Используя пуассоновскую
аппроксимацию, оцените вероятность того, что за 100 дней монетка
встанет на ребро ровно 3 раза. }
\solution{ }

\problem{ Страховая компания <<Ой>> заключает договор страхования от
<<невыезда>> (не выдачи визы) с туристами, покупающими туры в
Европу. Из предыдущей практики известно, что в среднем отказывают
в визе одному из 130 человек. Найдите вероятность того, что из 200
застраховавшихся в <<Ой>> туристов, четверым потребуется страховое
возмещение. }
\solution{ }

\problem{
Вася, владелец крупного Интернет-портала, вывесил на главной
странице рекламный баннер. Ежедневно его страницу посещают 1000
человек. Вероятность того, что посетитель портала кликнет по
баннеру равна 0,003. С помощью пуассоноского приближения оцените
вероятность того, что за один день не будет ни одного клика по
баннеру.}
\solution{ }




% !Mode:: "TeX:UTF-8"
\section{Нормальное распределение и ЦПТ}
\subsection{Одномерное нормальное распределение}

\problem{Где находятся точки перегиба функции плотности для нормального распределения?}
\solution{$ \mu \pm \sigma $.}

\problem{Пусть $X\sim \mN(\mu;\sigma^{2})$ и $t>\mu$. В какой точке функция $\PP(X\in [t;t+dt])$ убывает быстрее всего?}
\solution{$ \mu + \sigma $.}

\problem{Имеются две акции с доходностями $X$ и $Y$ на один вложенный рубль. Доходности некоррелированы, $\E(X)=0{,}09$, $\E(Y)=0{,}04$, $\sigma_X=0{,}5$, $\sigma_Y=0{,}1$. У инвестора есть 1~рубль. Он покупает на $a\in (0;1)$ рубля первых акций и на $(1-a)$ вторых акций. Обозначим за $\mu(a)$ и $\sigma(a)$ ожидаемую доходность и стандартное отклонение доходности полученного портфеля.
\begin{enumerate}
\item Постройте множество возможных $\mu(a)$ и $\sigma(a)$
\item Найдите наименее рисковый портфель. Чему равна его ожидаемая доходность?
\end{enumerate} }
\solution{}

\problem{Имеются две акции с доходностями $X$ и $Y$ на один вложенный рубль. Доходности некоррелированы, $\E(X)=0{,}09$, $\E(Y)=0{,}04$, $\sigma_X=0{,}5$, $\sigma_Y=0{,}1$. У инвестора есть 1~рубль. Инвестор подкидывает неправильную монетку, выпадающую орлом с вероятностью $p$. Если монетка выпадает орлом, он покупает первые акции, если решкой, то вторые. Обозначим за $\mu(p)$ и $\sigma(p)$ ожидаемую доходность и стандартное отклонение доходности полученной стратегии.
\begin{enumerate}
\item Постройте множество возможных $\mu(p)$ и $\sigma(p)$.
\item Найдите функцию плотности доходности полученной стратегии.
\end{enumerate} }
\solution{}


\problem{ Известна функция плотности случайной величины,  $p_{X}(t)=c\cdot \exp (-8t^{2} +5t)$. Найдите $E(X)$,  $\sigma _{X} $. }
\solution{ выделяем полный квадрат, $\E(X)=\frac{5}{16}$, $\sigma_{X}=\frac{1}{4}$  }



\subsection{ЦПТ}
\problem{
Вася играет в компьютерную игру "--- <<стрелялку"=бродилку>>. По
сюжету ему нужно убить 60 монстров. На один выстрел уходит ровно 1
минута. Вероятность убить монстра с одного выстрела равна 0{,}25.
Количество выстрелов не ограничено.
\begin{enumerate}
\item Сколько времени в среднем Вася тратит на одного монстра?
\item  Найдите дисперсию этого времени.
\item  Какова вероятность того, что Вася закончит игру меньше, чем за
3 часа?
\end{enumerate}
 }
\solution{ }


\subsection{Многомерное нормальное распределение}

\problem{
Ермолай Лопахин решил приступить к вырубке вишневого сада. Однако выяснилось, что растут в нём не только вишни, но и яблони. Причём, по словам Любови Андреевны Раневской, среднее количество деревьев (а они периодически погибают от холода или жары, либо из семян вырастают новые) в саду распределено в соответствии с нормальным законом ($X$ "--- число яблонь, $Y$ "--- число вишен) со следующими параметрами:
\begin{equation}
\begin{pmatrix}	X \\ 	Y 	\end{pmatrix}
\sim \mN
\left(
\begin{pmatrix}
25 \\ 125
\end{pmatrix}
;
\begin{pmatrix}
	5 & 4 \\
	4 & 10
	\end{pmatrix}
\right)
\end{equation}

Найдите вероятность того, что Ермолаю Лопахину придется вырубить более 150~деревьев.
Каково ожидаемое число подлежащих вырубке вишен, если известно, что предприимчивый и последовательный Лопахин, не затронув ни одного вишнёвого дерева, начал очистку сада с яблонь и все 35~яблонь уже вырубил?

Автор: Кирилл Фурманов, Ира ...}
\solution{ }


\problem{
В самолете пассажирам предлагают на выбор <<мясо>> или <<курицу>>. В самолет 250 мест. Каждый пассажир с вероятностью 0.6 выбирает курицу, и с вероятностью 0.4 - мясо. Сколько порций курицы и мяса нужно взять, чтобы с вероятностью 99\% каждый пассажир получил предпочитаемое блюдо, а стоимость <<мяса>> и <<курицы>> для компании одинаковая? \\
Как изменится ответ, если компания берет на борт одинаковое количество <<мяса>> и <<курицы>>? }
\solution{ 
$K=170$, $M=120$ (симметричный интервал) или $K=M=168$ (площадь с одного края можно принять за 0) \\
Вариант: театр, два входа, два гардероба а) только пары, б) по одному }


\subsection{Распределения связанные с нормальным}

% задачи без статистики на свойства t, F, chi распределений


\problem{Сравните $\E(F_{k,n})$ и $Var(t_n)$.}
\solution{$\E(F_{k,n})=Var(t_n)$}


\problem{Пусть $X\sim t_{n}$. Как распределена величина $Y=X^{2}$? }
\solution{ $F_{1,n}$ }


\problem{  На плоскости выбирается точка со случайными координатами. Абсцисса
и ордината независимы и распределены $N(0;1)$. Какова вероятность
того, что расстояние от точки до начала координат будет больше
2,45? }
\solution{ Квадрат расстояния имеет $\chi^2_2$ распределение  }



% !Mode:: "TeX:UTF-8"
\section{Случайное блуждание}
\subsection{Дискретное случайное блуждание}


\problem{Какова вероятность того, что трёхмерное случайное блуждание бесконечное количество раз пересечёт вертикальную ось?}
\solution{1. Про вертикальные шаги можно забыть, а вероятность бесконечное количество раз посетить 0 для двумерного случайного блуждания равна 1.}


\problem{Пусть $X_{n}$ "--- симметричное случайное блуждание. Сколько времени в среднем придётся ждать, пока остаток от деления $X_{n}$ на 183 окажется равным 39?}
\solution{$39^{2}$.}



\subsection{Принцип отражения}
\problem{  \zdt{Выборы} \par
После выборов, в которых участвуют два кандидата, A и B, за них
поступило $a$ и $b$ ($a>b$) бюллетеней соответственно, скажем, 3 и
2. Если подсчёт голосов производится последовательным извлечением
бюллетеней из урны, то какова вероятность того, что хотя бы один
раз число вынутых бюллетеней, поданных за A и B, было одинаково? Какова вероятность того, что A всё время лидировал?
\begin{ist}
Mosteller.
\end{ist}
}
\solution{ }

\problem{ \zdt{Ничьи при бросании монеты} \par
Игроки A и B в орлянку играют $n$ раз. После первого бросания
каковы шансы на то, что в течение всей игры их выигрыши не
совпадут?
\begin{ist}
Mosteller.
\end{ist}}
\solution{ }

\problem{Доходность акции следует симметричному дискретному случайному блужданию. Какова вероятность того, что в момент времени $2k+1$ доходность будет выше, чем когда-либо в прошлом?

Источник: Алексей Суздальцев}
\solution{$\frac{C_{2k}^{k}}{2^{2k+1}}$. Совсем простого решения не знаю, хотя ответ простой.}



\subsection{Броуновское движение}


% !Mode:: "TeX:UTF-8"


% !Mode:: "TeX:UTF-8"

% !Mode:: "TeX:UTF-8"
% !Mode:: "TeX:UTF-8"
% MM, ML
\section{Метод максимального правдоподобия, метод моментов}

\problem{Падал первый снег и 30 школьников ловили снежинки. В среднем каждый поймал $3{,}3$ снежинки. Количества снежинок пойманные каждым --- независимые Пуассоновские случайные величины с общим параметром $\lambda$. 
\begin{enumerate}
\item Оцените $\lambda$ используя метод максимального правдоподобия
\item Оцените дисперсию полученной оценки
\item На уровне значимости 5\% проверьте гипотезу о том, что $\lambda=3$ используя тест множителей Лагранжа
\item ... используя тест отношения правдоподобия
\item ... используя тест Вальда 
\item Постройте 95\% доверительный интервал Вальда для $\lambda$
\end{enumerate}}
\solution{ Логарифмическая функция правдоподобия
\begin{equation}
l(\lambda)=-n\lambda+\sum x_i \ln\lambda-\sum \ln(x_i !)
\end{equation}
Первая производная:
\begin{equation}
l'(\lambda)=-n+\frac{\sum x_i}{\lambda}
\end{equation}
Оценка метода максимального правдоподобия имеет вид $\hat{\lambda}=\bar{X}_n$
Ожидаемае информация Фишера $I(\lambda)=n/\lambda$, наблюдаемая информация Фишера $I(\hat{\lambda})=n/\hat{\lambda}$.
Тест Вальда $W=(\hat{\lambda}-3)^2\cdot I(\hat{\lambda})$ \\
Тест множителей Лагранжа $LM=l'(3)\cdot I^{-1}(3)$ \\
Тест отношения правдоподобия $LR=2(l(\hat{\lambda})-l(3))$ 
}



\problem{В коробке 10 внешне не отличимых шоколадных конфет. Внутри $k$ штук из них есть орех. Мы выбирали конфеты наугад по одной и ели. Первый орех оказался в третьей по счету конфете. 

Оцените неизвестный параметр $k$ методом моментов, методом максимального правдоподобия.}
\solution{}



\problem{
Интервал времени в минутах между спам-письмами по электронной почте --- случайная величина с функцией плотности 
\begin{equation}
p(t)=
\begin{cases}
a^2\cdot t\cdot \exp(-at), \quad t\geq 0 \\
0, \quad t<0
\end{cases},
\end{equation}
где $a$ --- неизвестный параметр. По выборке из 20 наблюдений известно, что $\sum_{i=1}^{n}X_{i}=625$, $\sum_{i=1}^{n}\ln(X_i)=25$.


\begin{enumerate}
\item Оцените $a$ методом моментов
\item Оцените $a$ методом максимального правдоподобия
\item Постройте 95\% доверительный интервал для $a$ с помощью максимального правдоподобия
\end{enumerate}}


\solution{
\begin{equation}
E(\bar{X})=E(X_{i})=\int_{0}^{\infty} a^2\cdot t^2 \exp(-at)dt
\end{equation}

Для удобства заметим, что
\begin{equation}
\int_{0}^{\infty}t^n\exp(-at)dt=0+\int_{0}^{\infty}t^{n-1}\frac{n}{a}\exp(-at)dt
\end{equation}

Получаем мат. ожидание:
\begin{multline}
E(\bar{X})=\int_{0}^{\infty} a^2\cdot t\frac{2}{a}\cdot \exp(-at)dt=\\
=\int_{0}^{\infty} 2a\cdot t\cdot \exp(-at)dt=\int_{0}^{\infty} 2a\cdot \frac{1}{a} \exp(-at)dt=\\
=\int_{0}^{\infty} 2\cdot \exp(-at)dt=\frac{2}{a}
\end{multline}

Метод моментов
\begin{equation}
\bar{X}=\frac{2}{\hat{a}_{MM}}
\end{equation}

Получаем $\hat{a}_{MM}$:
\begin{equation}
\hat{a}_{MM}=\frac{2}{\bar{X}}=\frac{40}{625}=\frac{8}{125}=0.064
\end{equation}

Метод максимального правдоподобия
\begin{equation}
l=\sum_{i=1}^{n}\ln(p(x_{i}))=\sum_{i=1}^{n}(2\ln(a)+ln(x_{i})-ax_{i})=2n\ln(a)+\sum \ln(x_{i})-a\sum x_{i}
\end{equation}

\begin{equation}
l'(a)=\frac{2n}{a}-\sum x_{i}
\end{equation}

Получаем $\hat{a}_{ML}$:
\begin{equation}
\hat{a}_{ML}=\frac{2n}{\sum X_{i}}=\frac{2}{\bar{X}}=\hat{a}_{MM}
\end{equation}

Наблюдаемая информация Фишера
\begin{equation}
J_{n}(\hat{a})=-l''(\hat{a})=\frac{2n}{\hat{a}^{2}}=2n\cdot \left(\frac{\sum X_{i}}{2n}\right)^2=\frac{(\sum X_{i})^2}{2n}=\frac{625^2}{40}
\end{equation}

Доверительный интервал
\begin{equation}
\hat{a}\pm 1.96\cdot J_{n}^{-1/2}=0.064\pm 0.0198=[0.0442;0.0838]
\end{equation} }

\problem{В банке 10 независимых клиентских <<окошек>>. В момент открытия в банк вошло 10 человек. Каждый клиент встал к отдельному окошку. Других клиентов банке в этот день не было. Предположим, что время обслуживания одного клиента распределено экспоненциально с параметром $\lambda$. Оцените параметр $\lambda$  и оцените дисперсию оценки методом максимального правдоподобия в каждой из ситуаций 

\begin{enumerate}
\item Менеджер записал время обслуживания клиента в каждом окошке. Окошко \No 1 обслужило своего клиента за 10 минут, окошко \No 2 обслужило своего клиента за 20 минут; оставшуюся часть записей менеджер благополучно затерял. 
\item Менеджер наблюдал за окошками в течение получаса и записывал время обслуживания клиента. Окошко \No 1 обслужило своего клиента за 10 минут, окошко \No 2 обслужило своего клиента за 20 минут; остальные окошки еще обслуживали своих первых клиентов в тот момент, когда менеджер удалился. 
\item  Менеджер наблюдал за окошками в течение получаса. За эти полчаса два окошка успели обслужить своих клиентов. Остальные окошки еще обслуживали своих первых клиентов в тот момент, когда менеджер удалился. 
\item Менеджер наблюдал за окошками и решил записать время обслуживания первых двух клиентов. Через  10 минут от начала работы был обслужен первый клиент в одном из окошек, через 20 минут от начала работы был обслужен второй клиент. Сразу после того, как был обслужен второй клиент менеджер прекратил наблюдение. 
\item Одновременно с открытием банка началась деловая встреча директора банка с инспектором по охране труда. Время проведения таких встреч --- случайная величина, имеющая экспоненциальное распределение со средним временем 30 минут. За время проведения встречи было обслужено два клиента. 
\item Одновременно с открытием банка началась деловая встреча директора банка с инспектором по охране труда. Время проведения таких встреч - случайная величина, имеющая экспоненциальное распределение со средним временем 30 минут. За время проведения встречи было обслужено два клиента, один за 10 минут, второй --- за 20 минут. 
\item Изменим условие: в банке 11 окошек, при открытии банка вошло 11 клиентов. Других клиентов в этот день не было. Клиент попавший в окошко \No 11 смотрел за остальными. Раньше клиента из окошка  \No 11 освободилось двое клиентов: за 10 минут и за 20 минут. 
\end{enumerate} }
\solution{}


\problem{Предположим, что доход жителей страны распределен экспоненциально. 
Имеется выборка из 1000 наблюдений по жителям столицы. Если возможно, постройте 90\% доверительный интервал для $\lambda$ в следующих случаях
\begin{enumerate}
\item Столицу можно считать случайной выборкой из жителей страны
\item В столице селятся только люди с доходом больше 100 тыс. рублей. 
\item В столице селятся только люди с доходом больше $m$ тыс. рублей, где $m$ - неизвестная константа. При этом постройте также и 90\% доверительный интервал для $m$ 
\item В столице живут 10\% самых богатых жителей страны
\item В столице живут $p$\% самых богатых жителей страны, где $p$ - неизвестная константа. Постройте также 90\% доверительный интервал для $p$
\end{enumerate} }
\solution{}

\problem{Случайные величины $X_1$, $X_2$, \ldots, $X_n$ независимы и нормальны $N(\mu,\sigma^2)$ с неизвестным математический ожиданием и дисперсией. При помощи теста множителей Лагранжа, теста отношения правдоподобия и теста Вальда проверьте гипотезы
\begin{enumerate}
\item $H_0$: $\mu=0$
\item $H_0$: $\sigma^2=1$
\item 
$H_0$: $\left\{\begin{array}{l}
\mu=0 \\
\sigma^2=1
\end{array}\right.$
\end{enumerate} }
\solution{}


% !Mode:: "TeX:UTF-8"

% !Mode:: "TeX:UTF-8"
% тестирование гипотез. общие свойства


\problem{Вовочка тестирует гипотезу $H_{0}$ против гипотезы $H_{a}$. Предположим, что $H_{0}$ на самом деле верна. По своей сути p-value является случайной величиной. Какое распределение оно имеет?}
\solution{Равномерное на $[0;1]$ }


\problem{Гражданин Фёдор решает проверить, не жульничает ли напёрсточник Афанасий, для чего предлагает Афанасию сыграть 5 партий в напёрстки. Фёдор решает, что в каждой партии будет выбирать один из трёх напёрстков наугад, не смотря на движения рук ведущего. Основная гипотеза: Афанасий честен, и вероятность правильно угадать напёрсток, под которым спрятан шарик, равна 1/3. Альтернативная гипотеза: Афанасий каким-то образом жульничает (например, незаметно прячет шарик), так что вероятность угадать нужный напёрсток меньше, чем 1/3. Статистический критерий: основная гипотеза отвергается, если Фёдор ни разу не угадает, где шарик.
\begin{enumerate}
\item Найдите уровень значимости критерия.
\item Найдите мощность критерия в том случае, когда Афанасий жульничает, так что вероятность угадать нужный напёрсток равна 1/5.
\end{enumerate}}
\solution{}



% !Mode:: "TeX:UTF-8"
\section{Гипотезы о среднем и сравнении среднего при большом $n$}


\problem{Предположим, что исходные наблюдения $X_{i}$ нормальны $N(\mu,\sigma^{2})$ и независимы. Константы $\mu$ и $\sigma$ неизвестны. Вовочка строит доверительный интервал для $\mu$ по первой половине доступных наблюдений. Петя --- по всем наблюдениям. Может ли получится у Вовочки интервал меньшей ширины, чем у Пети?}
\solution{Да. Например, если первая половина наблюдений попала рядом с $\mu$, а вторая --- далеко. Ну не повезло Пете.}

% untyp
\problem{Вася и Петя выясняют, кто лучше умеет знакомиться с девушками. Вася попытался познакомиться с 100 девушек, из них 54 девушки дали ему номер своего телефона. Петя  попытался познакомиться с 900 девушек, из них 495 дали ему номер своего телефона. 

Вася и Петя изучили курс матстата и начали спорить. Петя утверждает, что ему в среднем чаще девушки дают свой номер телефона и аргументирует это так: давай проверим гипотезу, что в среднем ровно половина девушек даёт номер своего телефона, против альтернативной гипотезы, что больше половины. По твоим данным эта гипотеза не отвергается, а по моим --- отвергается.

Вася утверждает, что он лучше убеждает девушек. Аргументирует это так: давай проверим гипотезу, что в среднем 60\% девушек даёт номер своего телефона, против альтернативной гипотезы, что меньше 60\:. По твоим данным эта гипотеза отвергается, а по моим --- не отвергается. 

Кто из них прав?}
\solution{Оба они делают одну ошибку: если $H_0$ не отвергается, это не значит, что она --- верна. Корректнее было бы провести тест на сравнение средних. }
Идея: Кирилл Фурманов



% !Mode:: "TeX:UTF-8"

% !Mode:: "TeX:UTF-8"

% untyp
\problem{ 
При подбрасывании кубика грани выпали 234, 229, 240, 219,
236 и 231 раз соответственно. Проверьте гипотезу о том, что кубик
<<правильный>>. } 
\solution{} 

% untyp
\problem{
Проверьте независимость дохода и пола по таблице: \\
\begin{tabular}{|c|c|c|c|}
  \hline
   & $<500$ & $500-1000$ & $>1000$ \\
  \hline
  М & 112 & 266 & 34 \\
  Ж & 140 & 152 & 11 \\
  \hline
\end{tabular} } 
\solution{} 

% untyp
\problem{
Вася Сидоров утверждает, что ходит в кино в два раза чаще, чем в
спортзал, а в спортзал в два раза чаще, чем в театр. За последние
полгода он 10 раз был в театре, 17 раз - в спортзале и
39 раз - в кино. Правдоподобно ли Васино утверждение? } 
\solution{} 

% untyp
\problem{
Проверьте независимость пола респондента и предпочитаемого
им сока: \par
\begin{tabular}{|c|c|c|c|}
  \hline
   & Апельсиновый & Томатный & Вишневый \\
  \hline
  М & 69 & 40 & 23 \\
  Ж & 74 & 62 & 34 \\
  \hline
\end{tabular} } 
\solution{} 


% untyp
\problem{
У 200 человек записали цвет глаз и волос. На уровне значимости
10\% проверьте гипотезу о независимости этих признаков. \par
\begin{tabular}{|c|c|c|c|}
  \hline
  Цвет глаз/волос & Светлые & Темные & Итого \\
  \hline
  Зеленые & 49 & 25 & 74 \\
  Другие & 30 & 96 & 126 \\
  \hline
  Итого & 79 & 121 & 200 \\
  \hline
\end{tabular} } 
\solution{} 

% untyp
\problem{
Идея задачи на хи-квадрат. \par
Если предложить голосовать за 3 альтернативы... \par
Если предложить голосовать за 4 альтернативы... \par
Выполняется ли предпосылка независимости от посторонних
альтернатив? } 
\solution{} 



% untyp
\problem{Когда Пирсон придумал хи-квадрат тест на независимость признаков (около 1900 г.), он не был уверен в правильном количестве степеней свободы. Он разошелся во мнениях с Фишером. Фишер считал, что для таблицы сопряженности размера два на два хи-квадрат статистика будет иметь три степени свободы, а Пирсон - что одну. Чтобы выяснить истину, Фишер взял большое количество таблиц два на два с заведомо независимыми признаками и посчитал среднее значение хи-квадрат статистики. Чему оно оказалось равно? Почему этот эксперимент помог выяснить истину?}
\solution{Среднее значение хи-квадрат случайной величины равно числу степеней свободы. Единице. Historical Note (as told by Chris Olsen): 
The Chi-square statistic was invented by Karl Pearson about 1900. Pearson knew what the Chi-square distribution looks like, but he was unsure about the degrees of freedom. 

About 15 years later, Fisher got involved. He and Pearson were unable to agree on the degrees of freedom for the two-by-two table, and they could not settle the issue mathematically. Pearson believed there was 1 degree of freedom and Fisher 3 degrees of freedom. 

They had no nice way to do simulations, which would be the modern approach, so Fisher looked at lots of data in two-by-two tables where the variables were thought to be
independent. For each table he calculated the Chi-square statistic. Recall that the expected value for the Chi-square statistic is the degrees of freedom. After collecting many Chi-square values, Fisher averaged all the values and got a result he described as <<embarrassingly close to 1>> 

This confirmed that there is one degree of freedom for a two-by-two table. Some years later this result was proved mathematically. }


% !Mode:: "TeX:UTF-8"

% untyp
\problem{
Из 10 опрошенных студентов часть предпочитала готовиться по
синему учебнику, а часть - по зеленому. В таблице представлены их
итоговые баллы.  \\
\begin{tabular}{|c|c|c|c|c|c|c|}
  \hline
  Синий & 76 & 45 & 57 & 65 &  &  \\
  \hline
  Зеленый & 49 & 59 & 66 & 81 & 38 & 88 \\
  \hline
\end{tabular} \\
а) С помощью теста Манна-Уитни (Mann-Whitney) проверьте гипотезу о
том, что выбор учебника не меняет закона распределения оценки. \par
\emph{Разрешается использование нормальной аппроксимации} \par
б) Возможно ли в этой задаче использовать (Wilcoxon Signed Rank Test)? } 
\solution{} 

% untyp
\problem{
Имеются результаты экзамена в двух группах. Группа 1: 45,
67, 87, 71, 34, 12, 54, 57; группа 2: 46, 66, 81, 72, 11, 47, 55,
51, 9, 99. С помощью теста Манна-Уитни на уровне значимости $5\%$ проверьте гипотезу о том, что результаты двух групп не отличаются. } 
\solution{} 

% untyp
\problem{
Имеются результаты нескольких студентов до и после
апелляции (в скобках указан результат до апелляции):  48(47),
54(52), 67(60), 56(60), 55(58), 55(60), 90(70), 71(81), 72(87),
69(60). Предполагая, что изменение оценки на апелляции симметрично распределено, на уровне значимости $5\%$ проверьте гипотезу о том, что
апелляция в среднем не сказывается на результатах. } 
\solution{} 

% untyp
\problem{
Имеются наблюдения за говорливостью 30 попугаев
(слов/день): 34, 56, 32, 45, 34, 45, 67, 1, 34, 12, 123, ... , 37
(всего 13 наблюдений меньше 40). Проверить гипотезу о том, что
медиана равна 40 (слов/день). } 
\solution{} 

% untyp
\problem{
Вашему вниманию представлены результаты прыжков в длину
Васи Сидорова на двух соревнованиях. На первых среди болельщиц
присутствовала Аня Иванова (его первая любовь): 1,83; 1,64; 2,27;
1,78; 1,89; 2,33; 1,61; 2,31. На вторых Аня среди болельщиц не
присутствовала: 1,26; 1,41; 2,05; 1,07; 1,59; 1,96; 1,29; 1,52;
1,18; 1,47. С помощью теста (Mann-Whitney) проверьте гипотезу о
том, что присутствие Ани Ивановой положительно влияет на
результаты Васи Сидорова. Уровень значимости $\alpha=0.05$. } 
\solution{} 

% untyp
\problem{
Некоторые результаты 2-х контрольных по теории вероятностей
выглядят следующим образом (указан результат за вторую контрольную
и в скобках результат за первую): 43(55), 113(108), 97(53),
68(42), 94(67), 90.5(97), 35(91), 126(127), 102(78), 89(83). Можно
ли считать (при $\alpha=0.05$), что вторую контрольную написали
лучше? Предположим, что разница в баллах распределена симметрично.} 
\solution{} 

% untyp
\problem{
Садовник осматривал по очереди розовые кусты вдоль ограды. Всего вдоль ограды растет 30 розовых кустов. Из них оказалось 20 здоровых и 10 больных. \par
Вот заметки садовника: $+++\ominus++\ominus\ominus\ominus++++\ominus\ominus ++\ominus+++++\ominus\ominus\ominus++++$ \par
(+ - здоровый куст, $\ominus$ - больной куст) \par
а) С помощью теста серий проверьте гипотезу о независимости испытаний \par
б) Какой естественный смысл имеет эта гипотеза? \par
Подсказка: можно использовать нормальное распределение }
\solution{}








\section{Решения}
\solutiononly

"--* Профессор, я решал эту задачу 3 часа, а ответ не совпадает. Где я сделал ошибку?

"--* Предположив, что ответы к задачнику верные.

По мотивам AoPS, \texttt{f=265\&t=275053}.
\par\bigskip


% !Mode:: "TeX:UTF-8"

% есть идея проще:
% возле неоттипографленной (новой) задачи я ставлю такой комментарий

% untyp

% когда задача "готова" комментарий "untyp" рядом с ней можно убрать
% так проще потому, что при нажатии ctrl-f "untyp" мы сразу попадаем к нужному месту

% чтобы облегчить совместное редактирование - файл разделен на несколько более мелких частей
% кстати, у texmaker есть сквозной поиск по файлам. т.е. я могу искать "untyp" сразу во всех файлах



% прочее...


%про две шкатулки - вставить в стохан:
%E(Y|X=x_{i}) существует,
%но E(Y|X) - нет

%раздракониванию задач (полный дубляж условия) но вопросы разные - в разные разделы (да!)
%автоматические ссылки туда-сюда ???
%викифицирование ?

%метки
% die - про кубик
% coin - про монетку

%binomial
%uniform
%geom_d
%poisson

%circle_trick
%wrong_class - возможно неправильно классифицирована

%gen_fun - производящие функции


%
%Идеи решения (?):
% Превратить одношаговый эксперимент в двухшаговый: сначала выбрать предметы, затем выбрать их порядок

% !Mode:: "TeX:UTF-8"
\section{Простые эксперименты}
% simple_experiments

\subsection{Дискретные простые эксперименты}
%Эксперимент состоит из одного "этапа"

%Правило сложения вероятностей.
%Если события несовместны, то
%P(хотя бы одно)= сумма
%Р(все сразу)=0

%1.1. дискретные случайные величины (P, E)

\problem{
Подбрасываются два кубика. Какова вероятность выпадения хотя бы
одной шестёрки? Какова вероятность того, что шестёрка не выпадет
ни разу? }
\solution{
$\PP(N\geq 1)=1-\frac{5}{6}^2$; $\PP(N=0)=\frac{5}{6}^2$. }
\cat{die}

\problem{
$\Omega =\{a, b, c\}$, $\PP\ofbr{a, b}=0{,}8$, $\PP\ofbr{b,c}=0{,}7$. Найдите
$\PP\ofbr{a}$, $\PP\ofbr{b}$, $\PP\ofbr{c}$.}
\solution{$\PP\ofbr{b}=0{,}2$, $\PP\ofbr a=0{,}6$, $\PP\ofbr{c}=0{,}5$.  }

\problem{
$A$  и  $B$  несовместны,  $\PP(A)=0{,}3$, $\PP(B)=0{,}4$. Найдите
$\PP(A^{c} \cap B^{c} )$.}
\solution{ $\PP(A^{c} \cap B^{c} )=1-0{,}3-0{,}4$.}

\problem{
$\PP(A)=0{,}3$,  $\PP(B)=0{,}8$. В каких пределах
может лежать  $\PP(A\cap B)$? }
\solution{ $\PP(A\cap B)\in[0{,}1;0{,}3]$.}


\problem{
Кубик подбрасывается два раза. Какова вероятность того, что результат
второго броска будет строго больше, чем результат первого?
Какова вероятность того, что в сумме будет 6? Что в сумме будет 9? Что максимум равен
5? Что минимум равен 3? Что разница будет равна 1 или 0? }
\solution{ $\PP\ofbr{N_{2}>N_{1}}=\frac{15}{36}$; \par
$\PP\ofbr{N_{1}+N_{2}=6}=\frac{5}{36}$; \par
$\PP\ofbr{N_{1}+N_{2}=9}=\frac{4}{36}$; \par
$\PP\ofbr{\max\{N_{1},N_{2}\}=5}=\frac{9}{36}$; \par
$\PP\ofbr{\min\{N_{1},N_{2}\}=3}=\frac{7}{36}$; \par
$\PP\ofbr{|N_{1}-N_{2}|\leq 1}=\frac{16}{36}$. }
\cat{die}

\problem{ \label{shokoladnie konfeti}
На подносе лежит 20 шоколадных конфет, одинаковых с виду. В
четырёх из них есть орех внутри. Маша съела 5 конфет. Какова
вероятность того, что в наугад выбранной оставшейся конфете будет
орех? }
\solution{$\frac{4}{20}$. }

%%%%% пошло применение ожидания

\problem{ \label{ojidanie ot bernulli}
Пусть  $X$  принимает два значения, причём $\PP\ofbr{X=1}=p$ и
$\PP\ofbr{X=0}=1-p$. Найдите $\E(X)$.}
\solution{ $\E(X)=p$.}



\problem{
Пусть существует всего два момента времени, $t = 0$ и $t =
1$. Cтоимости облигаций и акций в момент времени $t$ обозначим соответственно
$B_{t}$ (bond) и $S_{t}$ (share). Известно, что $B_{0}=1$,
$B_{1}=1{,}1$, $S_{0}=5$, $S_{1}=
\begin{cases}
10, & p_{\text{high}}=0{,}7; \\
2, & p_{\text{low}}=0{,}3.
\end{cases}$ \\
Индивид может покупать акции и облигации по указанным ценам без
ограничений. Например, можно купить минус одну акцию: это
означает, что в момент времени $t=0$ индивид получает 5 рублей, а в момент $t = 1$ в
зависимости от состояния природы должен заплатить 10 рублей или 2
рубля.
\begin{enumerate}
\item Чему равна безрисковая процентная ставка за период?
\item Найдите дисконтированные математические ожидания будущих цен
акций и облигаций. Совпадают ли они с ценами нулевого периода?
\item Найдите такие вероятности $q_{\text{high}}$ и $q_{\text{low}}$, чтобы
дисконтированное математическое ожидание будущих цен
совпало с ценами нулевого периода.
\item Индивиду предлагают купить некий актив, который приносит 8
рублей в состоянии мира $\omega_{\text{high}}$ и 11 рублей в состоянии
мира $\omega_{\text{low}}$. Посчитайте ожидание стоимости этого актива с
помощью вероятностей $p$ и с помощью вероятностей $q$. Придумайте
такую комбинацию акций и облигаций, которая в будущем приносит 8 и
11 рублей соответственно, и найдите её стоимость.\end{enumerate} }
\solution{ }



\problem{
Игральный кубик подбрасывается два раза. Пусть  $X_{1}$ и $X_{2} $
"--- результаты подбрасывания. Найдите вероятности $\PP(\min
\left\{X_{1},X_{2} \right\}=4)$  и $\PP(\min
\left\{X_{1},X_{2} \right\}=2)$. }
\solution{ }


\problem{  \label{simple third}
На десяти карточках написаны числа от 1 до 9. Число 8 фигурирует
два раза, остальные числа "--- по одному разу. Карточки извлекают в
случайном порядке. Какова вероятность того, что девятка появится позже обеих
восьмёрок? }
\solution{ Устно: $\frac{1}{3}$.}

\problem{
17 заключённых, 5 камер. Заключённых распределяют по камерам по очереди, равновероятно в каждую. Какова вероятность, что Петя и Вася сидят в одной камере? }
\solution{ $0{,}2$. }
% решабельна ли более сложная задача, где конфигурации рассадок равновероятны?


\problem{
Кость подбрасывается два раза. Пусть  $X$  и  $Y$  "---
результаты
подбрасываний. Найдите  $\E\parb{|X\hm-Y|}$. }
\solution{ }

\problem{
\foreignlanguage{british}{We throw 3 dices one by one. What is the probability that we obtain 3 points in strictly increasing order?} }
\solution{ $\frac{C_{6}^{3}}{6^{3}}$. }

\problem{ \label{tri chisla}
Из 10 цифр (от 0 до 9) выбираются 3 наугад (возможны повторения).
Обозначим числа (в порядке появления): $X_{1}$, $X_{2}$, $X_{3}$.
Какова вероятность того, что $X_{1}>X_{2}>X_{3}$? }
\solution{ $\frac{C_{10}^{3}}{10^{3}}$, т.\,к. каждый способ выбрать три разных числа соответствует благоприятной
комбинации. }



\problem{
Кубик подбрасывается 3 раза. Какова вероятность того, что сумма первых двух подбрасываний будет больше третьего? }
\solution{ }

\problem{ \zdt{<<Масть>> при игре в бридж }

Часто приходится слышать, что некто при игре в бридж получил на
руки 13 пик. Какова вероятность (при условии, что карты хорошо
перетасованы) получить 13 карт одной масти?
\begin{note}
Каждый из четырёх игроков в бридж получает 13 карт из колоды в 52 карты.
\end{note}
\begin{ist}
Mosteller.
\end{ist}
}
\solution{ }


\problem{ \label{maksimum iz kartochek}
На карточках написаны числа от
1 до 100. В левую руку Маша берёт одну карточку, в правую "--- $k$~карточек.
Какова вероятность того, что число на карточке в левой руке
окажется больше числа на любой карточке из
правой руки? }
\solution{$\frac{1}{k+1}$, т.\,к. одна из $k+1$ карточек должна быть наибольшей.  }



\problem{ \label{sleeping beauty} \zdt{Спящая красавица}

Спящая красавица согласилась принять участие в научном
эксперименте. В воскресенье её специально уколют веретеном. Как
только она заснёт, будет подброшена правильная монетка. Если
монетка выпадет орлом, то спящую красавицу разбудят в понедельник
и спросят о том, как выпала монетка. Если монетка выпадет решкой,
то спящую царевну разбудят в понедельник, спросят о монетке, снова
уколют веретеном, разбудят во вторник и снова спросят о монетке.
Укол веретена вызывает легкую амнезию, и красавица не сможет
определить, просыпается ли она в первый раз или во второй.
Красавица только что проснулась.
\begin{enumerate}
\item Какова вероятность того, что сегодня понедельник?
\item Как следует отвечать красавице, если за каждый верный ответ ей
дарят молодильное яблоко?
\item Как следует отвечать красавице, если за неверный ответ её тут
же превращают в тыкву?
\end{enumerate}
\begin{note}
Осторожно! Некорректные вопросы!
\end{note}
 }
\solution{<<Сегодня понедельник>> "--- это \textbf{не} событие. Вероятность не
определена. Это функция от времени.

Вероятность того, что монетка выпала орлом, равна $0{,}5$. Поэтому ей
всё равно, как отвечать, если наказанием является превращение в
тыкву, и нужно отвечать: <<Решка!>> "--- если наградой является
молодильное яблоко. Предполагается, что красавица максимизирует
ожидаемое количество молодильных яблок.  }



\problem{
Пусть события $A_{0}$, $A_{1}$ и $A_{2}$ несовместны и вместе
покрывают всё $\Omega$. Обозначим $p_{0}=\P(A_{1}\cup A_{2})$, $p_{1}=\P(A_{0}\cup A_{2})$,
$p_{2}=\P(A_{0}\cup A_{1})$. Перечислите все условия, которым удовлетворяют $p_{0}$, $p_{1}$,
$p_{2}$. }
\solution{ }

\problem{
Найдите вероятность того, что произойдёт ровно одно событие из $A$ и $B$, если $\P(A)=0{,}3$, $\P(B)=0{,}2$, $\P(A\cap B)=0{,}1$.    }
\solution{ }

\problem{
Вася наугад выбирает два разных натуральных числа от 1 до 4.
\begin{enumerate}
\item Какова вероятность того, что будет выбрано число 3?
\item Какова вероятность того, что сумма выбранных чисел будет чётная?
\item Каково математическое ожидание суммы выбранных чисел?
\end{enumerate}
 }
\solution{ $\PP\ofbr{3}=\frac{1}{2}$, $\PP\ofbr{\Sigma\text{ чёт.}}=\frac{1}{3}$, $\E(\Sigma)=5$. }


\problem{
Известно, что когда соревнуются А и Б, то А побеждает с вероятностью $x$ (Б, соотвественно, с вероятностью $(1-x)$). Когда соревнуются А и В, то А побеждает с вероятностью $y$ (В, соответственно, с вероятностью $(1-y)$).
\begin{enumerate}
\item  Придумайте модель, которая бы позволяла узнать вероятность победы Б над В.
\item  Покажите, что можно придумать другую модель и получить другую вероятность.
\end{enumerate} }
\solution{
Если предположить, что у каждого игрока есть своя сила (константа), а вероятности победить в схватке для двух игроков относятся так же, как их силы, то $x=\frac{a}{a+b}$, $y=\frac{a}{a+c}$. Легко находим, что $\frac{b}{b+c}=\frac{y-xy}{x+y-2xy}$. }



\problem{  В клубе 25 человек.
\begin{enumerate}
\item  Сколькими способами можно выбрать комитет
из четырёх человек?
\item  Сколькими способами можно выбрать руководство, состоящее из
директора, зама и кассира?
\end{enumerate}
 }
\solution{ Комитет можно выбрать $C_{25}^{4}$ способами, руководство "--- $C_{25}^{3}3!$.}

\problem{ Сколькими способами можно расставить 5 человек в очередь?}
\solution{$5!$. }

\problem{ Сколькими способами можно покрасить 12 комнат, если требуется 4
покрасить жёлтым цветом, 5 "--- голубым и 3 "--- зелёным?}
\solution{ $C_{12}^{4}C_{8}^{5}$. }

\problem{ Шесть студентов (три юноши и три девушки), стоят в очереди за
пирожками в случайном порядке. Какова вероятность того, что юноши
и девушки чередуются?}
\solution{$2\cdot\frac{3!3!}{6!}$. }


\problem{
Где-то в начале 17 века Галилея попросили объяснить следующее:
количество троек натуральных чисел, дающих в сумме 9, такое же,
как количество троек, дающих в сумме 10; но при трёхкратном
подбрасывании кубика 9 в сумме выпадает реже, чем 10. Дайте корректное объяснение. }
\solution{ }


\problem{В классе 30 человек, и все разного роста. Учитель физкультуры хочет отобрать и поставить в порядке возрастания роста 5 человек. Сколькими способами это можно сделать?}
\solution{$C_{30}^{5}$. Расположить по росту можно только в одном порядке. }

\problem{Случайная величина $X$ равновероятно принимает одно из пяти значений: 1, 2, 3, 8 и 9. 
\begin{enumerate}
\item Найдите математическое ожидание и медиану $X$
\item Найдите значение $u$ при котором функция $f(u)=\E(|X-u|)$ достигает минимума
\item Найдите значение $u$ при котором функция $g(u)=\E((X-u)^2)$ достигает минимума
\item Сделайте выводы
\end{enumerate}
}
\solution{$\min f(u)=\med(X)$, $\min g(u)=\E(X)$}


\subsection{Непрерывные простые эксперименты}
%1.2. непрерывные случайные величины (P, E для равномерной)
\problem{
Поезда метро идут регулярно с интервалом 3 минуты. Пассажир
приходит на платформу в случайный момент времени. Пусть $X$
"--- время ожидания поезда в минутах.

Найдите $\P(X<1)$, $\E(X)$. }
\solution{$\frac{1}{3}$, $1{,}5$. }


\problem{
Светофор 40 секунд горит зелёным светом, 3 секунды "--- жёлтым, 30
секунд "--- красным, затем цикл повторяется. Петя подъезжает к светофору. На жёлтый свет Петя предпочитает остановиться.
\begin{enumerate}
\item  Какова вероятность, что Петя сможет проехать сразу?
\item  Какова средняя задержка Пети на светофоре?
\item  Вася, стоящий рядом со светофором, смотрит на него в течение 3
секунд. Какова вероятность того, что он увидит смену цвета?
\end{enumerate}
 }
\solution{ }

\problem{
Случайные величины $X$, $Y$, и $Z$ независимы и равномерны на $[0;1]$. Какова вероятность того, что $X+Y>Z$? }
\solution{ }


\problem{
\foreignlanguage{british}{At a bus stop you can take bus \#1 and bus \#2. Bus \#1 passes 10 minutes after bus \#2 has passed whereas bus \#2 passes 20 mins after bus \#1 has passed. What is the average waiting time to get on a bus at that bus stop?}

\begin{ist}
Wilmott forum, \texttt{catid=26\&threadid=55617}.
\end{ist}
 }
\solution{ $\frac{25}{3}$. }


\problem{
На множестве $A:=\{x\geq 0,\ 0\leq y\leq e^{-x}\}$ случайно (равномерно) выбирается точка. Пусть $X$ "--- абсцисса этой точки. Найдите следующие вероятности: $\PP(X>1)$, $\PP(X\in (1;5))$, $\PP(X \in [1;5])$. }
\solution{ $\int_{1}^{\infty}e^{-x}\,dx$; $\PP(X\in (1;5))=\PP(X \in [1;5])=\int_{1}^{5}e^{-x}\,dx$. }


\problem{На плоскости нарисован треугольник с вершинами $(0,0)$, $(2,0)$ и $(1,1)$. Случайным образом, равномерно, выбирается точка внутри этого треугольника. Случайная величина $X$ --- абсцисса полученной точки. Найдите
\begin{enumerate}
\item $\P(X>1)$, $\P(X<0.5)$, $\P(X=0.2)$
\item $\E(X)$
\end{enumerate}
}
\solution{$\E(X)=1$}
\todo[inline]{Тут спросить про минимизацию $f(u)=\E(|X-u|)$? }

\problem{Предположим, что завтрашний курс тугриков к луидорам --- случайная величина $X$, равномерная на отрезке $[0;1]$. Финансовый аналитик Вовочка строит прогноз $a$. За неправильный прогноз Вовочка заплатит штраф. Какой прогноз следует сделать Вовочке чтобы минимизировать ожидаемое значение штрафа, если
\begin{enumerate}
\item штраф считается по формуле $|X-a|$
\item штраф считается по формуле $0.75(X-a)$ при $X>a$ и $0.25(a-X)$ при $X<a$
\end{enumerate} 
}
\solution{Квантили распределения}

\subsection{Смешанные простые эксперименты, или содержание эксперимента неясно}
%1.3. смешанные случайные величины (P, E для смеси с равномерной)
\problem{
Как связаны между собой $\PP(A)$ и $\E(\inds{A})$? }
\solution{Равны.}


% !Mode:: "TeX:UTF-8"
\section{Сложные эксперименты}

% test comment
\subsection{<<Продвинутая>> комбинаторика }
%(использование "голого" биномиального коэффициента без каких бы то ни было заморочек можно делать в главе 1. Здесь всё, где надо применять комбинаторику с умом.
%Эту главу будут читать не все.
% возможно, что комбинаторная формула проста, но нужно догадаться до неё

\problem{
\ENG{Jenny and Alex flip $n$ fair coins each.}
\begin{enumerate}
\item \ENG{What is the probability that they get the same number of tails?}
\item  Пусть $a_{n}=\sqrt{n}\cdot p_{n}$, где $p_{n}$ "--- найденная вероятность. Найдите $\lim a_{n}$.
\end{enumerate}
 }
\solution{ $C_{2n}^{n}/2^{2n}$. Из $2n$ подбрасываний выберем $n$. Выбранные в зоне Jenny соответствуют невыбранным в зоне Alex. б) Через формулу Стирлинга: $\frac{1}{\sqrt{\pi}}$. }

\problem{
\ENG{2 couples and a single person are to be randomly placed in 5 seats in a row. What is the probability that no person that belongs to one of the couples sits next to his/her pair?} }
\solution{ }


\problem{
Встретились 6 друзей. Каждый дарит подарок одному из других 5 человек. Какова вероятность того, что найдется хотя бы одна пара человек, которая вручит подарки друг другу? }
\solution{ }


\problem{\zdt{Хулиган и случайная система}

На плоскости нарисовано $n$ прямых. Среди них нет параллельных, и никакие три не пересекаются в одной точке. Хулиган берёт первую прямую. Эта прямая делит плоскость на две полуплоскости. Хулиган случайным образом закрашивает одну из этих двух полуплоскостей. Затем хулиган поступает аналогично с каждой прямой. Какова вероятность того, что на плоскости останется хотя бы один незакрашенный кусочек?


\begin{ist}
Алексей Суздальцев.
\end{ist}
}

\solution{$\frac{n^2+n+2}{2^{n+1}}$
В оригинале у Алексея: Пусть $A$ "--- матрица $n\times2$, $\vec x$ и $\vec b$ "--- векторы подходящих размерностей. Известно, что как в матрице $A$, так и в расширенной матрице $\begin{pmatrix}
A & b  \end{pmatrix}$ все миноры максимального порядка ненулевые. Вася берёт систему $Ax=b$ и в каждой строке независимо от других ставит вместо знака равенства равновероятно знак <<больше>> или знак <<меньше>>. Какова вероятность того, что полученная система неравенств имеет решение?

Решение: остаться может либо один кусочек, либо ни одного. Всего есть $2^{n}$ способов выбирать полуплоскости. Некоторые из этих способов приводят к тому, что закрашена вся плоскость. Кусочков всего $\frac{n^{2}+n+2}{2}$. Каждый кусочек однозначно определяет способ выбора полуплоскостей. }

\problem{
На столе есть следующие предметы:
\begin{itemize}
\item 4 отличающихся друг от друга чашки;
\item 4 одинаковых гранёных стакана;
\item 10 одинаковых кусков сахара;
\item 7 соломинок разных цветов.
\end{itemize}
Сколькими способами можно разложить:
\begin{enumerate}
\item Сахар по чашкам;
\item Сахар по стаканам;
\item Соломинки по чашкам;
\item Соломинки по стаканам?
\end{enumerate}
Как изменятся ответы, если требуется, чтобы пустых ёмкостей не
оставалось? }
\solution{ }

\problem{
Сколькими способами можно разложить $k$ кусков сахара по
$n$ различающимся чашкам?

Подсказка: ответ "--- всего лишь биномиальный коэффициент. }
\solution{ $C_{k+n-1}^{n-1}$. }

% test comment 2

\problem{ \zdt{Генуэзская лотерея (задача Леонарда Эйлера)}

Из 90 чисел выбираются 5 наугад. Назовем серией последовательность
из нескольких чисел, идущих подряд. Например, если выпали числа
23, 24, 77, 78 и 79 (неважно в каком порядке), то есть две серии
(23-24, 77-78-79). Определите вероятность того, что будет ровно $k$ серий.

\begin{note}
Сама лотерея возникла в 17 веке.
\end{note}
 }
\solution{ }

\problem{ \label{sudba-don-juan-1} \zdt{Судьба Дон-Жуана} (см. тж. с.~\pageref{sudba-don-juan-2})

У Васи $n$  знакомых девушек (их всех зовут по-разному). Он пишет
им $n$  писем, но по рассеянности раскладывает их в конверты
наугад. Случайная величина $X$ обозначает количество девушек, получивших письма,
написанные лично для них.
\begin{enumerate}
\item Какова вероятность того, что Маша получит письмо, адресованное ей?
\item Какова вероятность того, что Маша и Лена получат письма, адресованные им?
\item Какова вероятность того, что хотя бы одна девушка получит
письмо, адресованное именно ей? Каков предел этой вероятности при
$n\rightarrow +\infty$?
\item Какова вероятность того, что произойдет ровно $k$ совпадений?
\end{enumerate}
}
\solution{ $1-\frac{1}{2!}+\frac{1}{3!}-\ldots$; $\frac{1}{e}$.  }


\problem{  \zdt{Покер}

Выбирается 5 карт из колоды (52 карты без джокеров, достоинством
от 2 до туза, всего 13 достоинств). Рассчитайте вероятности
комбинаций:\label{combo}
\begin{enumerate}
\item Pair (пара) "--- две карты одного достоинства;
\item Two pairs (две пары) "--- две карты одного достоинства и две другого;
\item Three of Kind (тройка) "--- три карты одного достоинства (две другие "--- разного достоинства);
\item Straight (стрит) "--- пять последовательных карт, не обязательно одной масти;
\item Flush (масть) "--- все карты одной масти;
\item Full House (фул-хаус) "--- три карты одного достоинства и две другого;
\item Four of Kind  (каре) "--- четыре карты одного достоинства;
\item Straight Flush (стрит-флэш) "--- пять последовательных карт одной масти;
\item Royal Flush (роял-флэш) "--- старшие пять последовательных карт одной масти?
\end{enumerate}

Примечание: более слабая комбинация не содержит в себе более
сильной.
}
\solution{ }

\problem{ \label{vilkodir} \zdt{<<Вилкодыр>>}

Есть $n$ дырочек, расположенных в линию, на расстоянии в 1~см друг от друга. У каждой вилки два штырька на расстоянии в 1 см.
\begin{enumerate}
\item Сколькими способами можно воткнуть $k$ одинаковых вилок?
\item Как изменится ответ, если дырочки расположены по окружности?
\end{enumerate}
}
\solution{Представим себе мифический объект <<вилкодыр>>. Он может
превращаться либо в вилку, либо в дырку. Вилкодыров у нас $n-k$.
Из них нужно выбрать $k$ вилок. Ответ: $C_{n-k}^{k}$ для линейного и
$C_{n-k}^{k}\frac{n}{n-k}$ для кругового расположения дырочек. }

\problem{ \label{lampochki v riad}
В ряду $n$ лампочек. Из них надо
зажечь 8, причём так, чтобы было три серии (по
2, 3 и 3 горящих лампочки). Сколькими способами это можно сделать? }
\solution{$3\cdot C_{n+1-d}^{k}$, где $k$ "--- число серий, $d$ "--- суммарная
длина серий. Домножение на 3 взялось из трёх вариантов (2-3-3,
3-2-3, 3-3-2). Мифический объект "--- серия из горящих лампочек
плюс негорящая справа.}

\problem{
Вася играет в преферанс. Он взял прикуп, снёс две карты и
выбрал козыря. У Васи на руках четыре козыря. Какова вероятность,
что
оставшиеся четыре козыря разделились как 4:0, 3:1, 2:2?

\begin{note}
Для тех, кто не знает, как играть в преферанс: 32 карты, из
которых 8 "--- будущие козыри, раздаются по 10 между 3 игроками, ещё
две кладутся в прикуп.
\end{note}
 }
\solution{ }

\problem{
Перетасована колода в 52 карты.
\begin{enumerate}
\item Какова вероятность того, что какие-нибудь туз и король будут лежать рядом?
\item Какова вероятность того, что какой-нибудь туз будет лежать за
каким-нибудь королем?
\end{enumerate}
 }
\solution{ }

\problem{
Чему равна сумма $C_{n}^{0}-C_{n}^{1}+C_{n}^{2}-...$?

Её применение к matching problem. }
\solution{ }

\problem{
В линию выложено $n$ предметов друг за другом. Сколькими способами
можно выбрать $k$ предметов из линии так, чтобы не были выбраны
соседние предметы? }
\solution{ $C_{n-k+1}^{k}$.

Решение 1. Отдельно рассмотрим два случая: самый правый предмет
выбран и самый правый предмет не выбран. В каждом случае
склеиваем предмет и примыкающий к нему справа предмет <<разделитель>>.

Решение 2. Удалим $k-1$ предмет из линии. Из оставшихся предметов
выберем $k$. Вернём удалённые как <<разделители>>. }

\problem{
\ENGs Given eight distinguishable rings, let $n$ be the number of
possible five-ring arrangements on the four fingers (not the
thumb) of one hand. The order of rings on each finger is
significant, but it is not required that each finger have a ring. Find the leftmost three nonzero digits of $n$. \RUSs
\begin{ist}
AIME 2000, 5.
\end{ist}
 }
\solution{Всего расположений $\binom{8}{5}\binom{8}{3}5! = 376\,320$, и три левые цифры "--- это \boxed{376}. }

\problem{ \ENGs
5 numbers are randomly picked from 90. In your bet cards, you get to choose 5 numbers.  How many cards have you got to fill in,
to guarantee that at least one of them has 4 right numbers? \RUSs
\begin{ist}
Wilmott forum.
\end{ist}
 }
\solution{ \ENGs
The answer to the original problem (\numprint{43948843} bet cards) was quoted already several times assuming that positioning of right numbers is irrelevant:
\begin{itemize}
\item there is eactly one bet card choice with 5 right numbers and
\item there are $5 \times (90-5) = 425$ bet card choices with exactly 4 right numbers.
\end{itemize}

Since the total number of different bet card choices is $\binom{90}{5}$, we have to fill out $\binom{90}{5} - 425 - 1 +1 = \numprint{43948843}$ bet cards to have at least 4 right numbers with probability 1.
\RUSs
}


\problem{
В контрольной 20 вопросов. Все ответы разные. Вася успел переписать у друга все верные ответы, но не знает, в каком порядке они идут. Отлично ставят ответившим верно на не менее чем 15 вопросов. Какова вероятность того, что Вася получит отлично? }
\solution{
$p=\frac{1}{20!}\cdot \bigl(C^{0}_{20}+C^{1}_{20}\cdot 0+C^{2}_{20}+C^{3}_{20}\cdot 2+C^{4}_{20}\cdot(C^{2}_{4}+3)+C^{5}_{20}\cdot(2C^{2}_{5}+4)\bigr)$. }

\problem{
There are $k$ books of mine among $n$ books. We put them in a shelf randomly. Which is the possibility that there are $p$ books of my who are placed continuously? (At least? Exactly?)
\begin{ist}
AoPS, \texttt{f=498\&t=192257}.
\end{ist}
 }
\solution{ Ugly sum? }


\problem{
На каждой карточке вы можете отметить любые 5 чисел из 100. Сколько карточек нужно купить, чтобы гарантированно угадать 3 числа из выпадающих в лотереи 7 чисел?

\begin{note}
Могут быть громоздкие вычисления.
\end{note}
 }
\solution{ }

\problem{
Сколькими способами можно поставить в очередь $a$ мужчин и $b$ женщин так, чтобы нигде двое мужчин не стояли рядом? }
\solution{ }

\problem{
Известно, что функция $f(n,k)$ удовлетворяет условиям:
\begin{itemize}
\item $f(n,k) = f(n-1,k) + f(n, k-1)$;
\item $f(n,0)=f(n,n)=1$.
\end{itemize}

Что это за функция такая?}
\solution{$C_{n+m}^{n}$ }

\problem{ \zdt{Усталые влюблённые}

В вагоне метро на длинную скамейку в $n$ мест садятся случайным образом $k>\frac{n}{2}$ пассажиров. Какова вероятность того, что после этого на скамейку сможет сесть влюблённая пара (влюблённым обязательно надо сидеть рядом)?

\begin{ist}
Алексей Суздальцев.
\end{ist}
}

\solution{Любую рассадку пассажиров можно представить в виде последовательности из нулей и единиц длины $n$, в которой единиц ровно $k$. Всего таких последовательностей $C_n^k$. Найдем количество последовательностей, \emph{не} удовлетворяющих условию, то есть не содержащих сдвоенных нулей (назовем такие последовательности \emph{плохими}).

Припишем к каждой плохой последовательности фиктивную единицу справа. Тогда в новой последовательности после любого нуля стоит единица, а значит, вся последовательность состоит из паттернов <<1>> и <<01>>. Паттернов <<01>> ровно $n-k$, (столько же, сколько и нулей), всего же паттернов столько же, сколько и единиц, то есть $k+1$. Таким образом, мы имеем $k+1$ позиций, из которых надо выбрать $n-k$, куда встанет паттерн <<01>>. Способов сделать это всего $C_{k+1}^{n-k}$, что и равно числу плохих последовательностей.

Значит, искомая вероятность равна $1-\frac{C_{k+1}^{n-k}}{C_n^k}$.}


\problem{
В классе 28 человек, среди них 18 девочек. Класс построили в 4 ряда по 7 человек. Какова вероятность того, что рядом с Вовочкой будет стоять хотя бы одна девочка?
\begin{note}
Для Вовочки любая девушка "--- рядом :).
\end{note}
 }
\solution{ }



\subsection{Геометрическое распределение (и близкие по духу)}
% первое упоминание о методе первого шага


\problem{
Равной силы команды играют до трёх побед. Какова вероятность того,
что будет ровно 3 партии? Ровно 4? Ровно 5? }
\solution{ $\PP(N=3)=2\frac{1}{2}^{3}$; $\PP(N=4)=2C_{3}^{1}\frac{1}{2}^{4}$; $\PP(N=5)=2C_{4}^{2}\frac{1}{2}^{5}$. }
\cat{geom_d}

\problem{Вася стреляет по мишени бесконечное количество раз. Он попадает по мишени с очень маленькой вероятностью. Какова вероятность того, что до первого попадания по мишени Васе потребуется больше времени, чем в среднем уходит на одно попадание?}
\solution{Путь $p$ "--- вероятность. Тогда $\E(X)=\frac{1}{p}$. Нас интересует $\PP\left(X>\frac{1}{p}\right)=\left(1-\frac{1}{p}\right)^{\frac{1}{p}}\approx \frac{1}{e}$.}
\cat{geom_d}


\problem{  \zdt{Геометрическое распределение}
Кубик подбрасывают до первого выпадения шестерки. Случайная величина  $N$ "---
число подбрасываний.
\begin{itemize}
\item Найдите $\PP(N=6)$, $\PP(N=k)$, $\PP(N>10)$ и $\PP(N>30\mid N>20)$, $\E(N)$.
\item Найдите $\E\left(\frac{1}{N}\right)$.
\end{itemize}
 }
\solution{ }
\cat{geom_d}


\problem{ \label{s chego vse nachinalos}\zdt{С чего всё начиналось\ldotst{}}

Париж, Людовик XIV, 1654 год, высшее общество говорит о рождении
новой науки "--- теории вероятностей. Ах, кавалер де Мере, <<fort
honn\^{e}te homme sans \^{e}tre math\'{e}maticien>>\ldotst{} (<<благородный
человек, хотя и не математик>>). Старая задача, неправильные
решения которой предлагались тысячелетиями (например, одно из
неправильных решений предлагал изобретатель двойной записи, кумир
бухгалтеров, Лука Пачоли), наконец решена правильно! Два игрока
играют в честную игру до шести побед. Игрок, первым выигравший
шесть партий (не обязательно подряд), получает 800 рублей. К
текущему моменту первый игрок выиграл 5 партий, а второй "--- 3
партии. Они вынуждены прервать игру в данной ситуации. Как им поделить приз по справедливости? }
\solution{ $700:100$.}


\problem{ \zdt{Von Neumann. Что делать, если монетка неправильная?}

Имеется <<несправедливая>> монетка, выпадающая гербом с некоторой
вероятностью. Под раундом будем подразумевать двукратное
подбрасывание монеты. Проводим первый раунд. Если результат раунда
"--- ГР (сначала герб, затем решка), то считаем, что выиграл первый
игрок. Если результат раунда "--- РГ, то считаем, что выиграл второй
игрок. Если результат раунда "--- ГГ или РР, то проводим ещё один
раунд. И так далее, пока либо не определится победитель, либо
количество раундов не достигнет числа $n$.
\begin{enumerate}
\item Найдите вероятности <<ничьей>>, выигрыша первого игрока, выигрыша
второго игрока в зависимости от $n$. Найдите пределы этих
вероятностей при $n\rightarrow +\infty$.
\item Как с помощью неправильной монетки сымитировать правильную?
\end{enumerate}
 }
\solution{ }

\problem{ \label{Monty Hell problem} \textit{Monty Hell problem} (не путать с Monty Hall)

\textbf{Сказка.} Ежедневно Кощей Бессмертный получает пенсию в размере 10~золотых монет. Затрат у Кощея нет никаких. Поэтому с начала пенсионного возраста он аккуратно нумерует
каждую полученную монету и кладет её в сундук. Ночью Мышка-норушка крадёт одну золотую монету из сундука.
\begin{enumerate}
\item Какова вероятность того, что $i$-я монета когда-либо исчезнет из Сундука?
\item Какова вероятность того, что хотя бы одна монета пролежит в сундуке бесконечно долго?
\item Дни сокращаются в продолжительности (каждый последующий "--- в два раза короче, чем предыдущий). Сколько монет будет в сундуке в конце времени?
\end{enumerate}
\begin{hint}
$(1-x) \le e^{-x}$.
\end{hint}
\begin{note}
А где надсказка?
\end{note}
 }
\solution{ Вероятность, что $i$-я монета когда-либо исчезнет, равна $1$, а того, что пролежит бесконечно долго, "--- 0. }

\problem{
Случайным образом выбирается натуральное число $X$. Вероятность выбора числа $n$ такова: $\PP(X=n)=2^{-n}$.
\begin{enumerate}
\item  Какова вероятность того, что будет выбрано чётное число? Нечётное число? Число, большее пяти? Число от 3 до 11?
\item Пусть независимо друг от друга выбираются $c$ чисел. Пусть $K_{c}$ "--- количество невыбранных чисел на отрезке от одного до наибольшего выбранного числа. Найдите $\PP(K_{c}=k).$
\end{enumerate}
\begin{ist}
AMM E3061, T.~Ferguson and C.~Melolidakis.
\end{ist}
 }
\solution{ $P(K_{c}=k)=2^{-(k+1)}$ вне зависимости от $c$. Для начала обнулим значение $K_{c}$ и возьмём в руку $c$ монеток. Подкинем монетки. Если все выпали орлом, мы прибавляем единичку к $K_{c}$. Если все выпали решкой, то мы объявляем значение $K_{c}$. Если часть выпала орлом, часть решкой, то выкинем те, что выпали решкой, и снова перейдем к подкидыванию монеток. В результате имее: рост $K_{c}$ на единицу или глобальная остановка процесса происходит равновероятно. Значит, $K_{c}$ распределено геометрически. }
\cat{hard}

\problem{Величины $X_1$, $X_2$, \ldots независимы и одинаково распределены с некоторой функцией плотности $f$. Величина $X_i$ --- это количество осадков в $i$-ый год. Пусть $Y$ --- номер года, когда впервые будет превышено количество осадков, выпавших в первом году.
Найдите закон распределения $Y$ и $\E(Y)$}
\solution{Найдем $\P(Y>k)$. Это вероятность того, что в первые $k$ лет не будет достигнут уровень первого года. Значит это вероятность того, что первый год дал наибольшее количество осадков за первые $k$ лет. В силу симметрии $\P(Y>k)=1/k$. Отсюда $\P(Y=k)=\P(Y>k-1)-\P(Y>k)=1/k(k-1)$ и $\E(Y)=+\infty$}

% untyp
\problem{Преподаватель по теории вероятностей пообещал своим студентам, что включит задачу на геометрическое распределение в экзамен с вероятностью 1/3. Чтобы исполнить своё обещание он подбросил одну монетку два раза и посчитал количество орлов. Оказалось, что орлов было ровно два. На основании этого количества он принял решение. Какое решение он принял и почему?}
\solution{Подбрасывая монетки детерминированное количество раз нельзя получить вероятность 1/3, значит преподаватель заранее не знал, сколько раз он будет подбрасывать монетку.  Простое правило принятия решения может иметь примерно такой вид: если выпало А, то давать задачу, если выпало Б, то не давать задачу, если не выпало ни А, ни Б, то повторить эксперимент. Орлов может быть либо два, либо один, либо 0. На два орла преподаватель повторять эксперимент не стал. Разумно предположить, что он использовал простую стратегию. Значит два орла означают <<включать>>, один орел --- <<не включать>>, ноль орлов --- подбрасывать монетку еще два раза. }
Идея: Николай Арефьев

\problem{Вася прыгает в длину несколько раз подряд. Результаты васиных прыжков --- независимые одинаково распределенные непрерывные случайные величины. Прыгнув в первый раз он записывает результат. И прыгает до тех пор, пока не перепрыгнет свой первый результат. Величина $X$ --- сколько прыжков Васе потребуется сделать дополнительно, чтобы перепрыгнуть первый результат. Найдите $\P(X=k)$ и $\E(X)$.}
\solution{$\P(X=k)=\frac{1}{k(k+1)}$, т.к. последний прыжок должен быть самым длинным из $k+1$ прыжка, а первый --- самым длинным из $k$ оставшихся. $\E(X)=\infty$.}




\subsection{Из \textit{n} предметов выбирается \textit{k}}
\problem{ \label{id008}
Из 50 деталей 4 бракованные. Выбирается наугад 10 деталей на проверку.
Какова вероятность не заметить брак? }
\solution{$\frac{C_{46}^{10}}{C_{50}^{10}}$. }

\problem{
Есть 4 карты одного достоинства. Наугад выбираются две.
Какова вероятность того, что они будут разного цвета? }
\solution{$\frac{1}{3}$. }

\problem{ \label{5 iz 36}
Какова вероятность полностью угадать комбинацию в лотерее 5 из
36?}
\solution{$\frac{1}{C_{36}^{5}}$. }

\problem{
В мешке 50 орехов, из них 5 пустые. Вы выбираете наугад 10
орехов. Какова вероятность того, что ровно один из них будет
пустой?}
\solution{ $\frac{C_{45}^{9}C_{5}^{1}}{C_{50}^{10}}$.}


\problem{ \label{tri shara iz korobki}
Из коробки с 4 синими и 5 зелёными шарами достают 3 шара. Пусть
$B$  и  $G$  "--- количество извлечённых синих и зелёных шаров.
Найдите  $\E(B)$,  $\E(G)$,  $\E(B\cdot G)$,  $\E(B-G)$. }
\solution{$\E(B)=3\cdot\frac{4}{9}=\frac{4}{3}$; $\E(G)=3-\E(B)=\frac{8}{3}$; $\E(B-G)=-\frac{4}{3}$; $\E(B\cdot G)=2\cdot\frac{5}{6}$.  }




\problem{
На факультете $n$ студентов. Из них наугад выбирают $a$ человек. Через год $b_{-}$ студентов покидают факультет, $b_{+}$ студентов приходят на факультет. Из них снова наугад выбирают $a$. Какова вероятность того, что хотя бы одного выберут два раза? }
\solution{ }


\problem{ \label{cube-cut-1}(см. тж. с.~\pageref{cube-cut-2})

 \ENGs A wooden cube that measures 3 cm along each edge is painted red. The painted cube is then cut into 27 pieces of 1-cm cubes.
\begin{enumerate}
\item If I choose one of the small cubes at random and toss it in the air, what is the probability that it will land red-painted side up?
\item If I put all the small cubes in a bag and randomly draw out 3, what is the probability that at least 3 faces on the cubes I choose are painted red?
\item If I put the small cubes in a bag and randomly draw out 3, what is the probability that exactly 3 of the faces are painted red?
\item Invent a new question!
\end{enumerate} \RUSs
\begin{ist}
\url{http://letsplaymath.wordpress.com/2007/07/25/puzzle-random-blocks/}
\end{ist}
 }
\solution{ Вероятность выпадения красной стороны сверху равна $\frac{1}{3}$. }

\problem{
Контрольную пишут 40 человек. Половина пишет первый вариант, половина "--- второй. Время написания работы каждым студентом "--- независимые непрерывные случайные величины. Какова вероятность того, что в тот момент, когда будет сдана последняя работа первого варианта, останется ещё 5 человек, пишущих второй вариант? }
\solution{ $\frac{C_{20}^{1}C_{20}^{5}}{C_{40}^{6}}\frac{1}{6}$ (вероятность заданной шестёрки финалистов помножить на вероятность выбора одного человека из шести). }


\problem{В бридж играют четыре игрока: Юг, Восток, Север, Запад. Перемешанная колода в 52 карты раздаётся игрокам по очереди по одной карте. Юг и Север получили 11~пик. Какова вероятность того, что две оставшиеся пики оказались у одного игрока? Разделились между остальными игроками? Каковы вероятности различных раскладов пик между остальными игроками, если Юг и Север получили 8~пик?}
\solution{}

\problem{ \label{korrektori ochepiatok} \zdt{Корректоры очепяток}

Вася замечает очепятку с вероятностью $0{,}7$; Петя независимо от Васи замечает очепятку с вероятностью $0{,}8$. В книге содержится 100 опечаток. Какова вероятность того, что Вася заметит 30 опечаток, Петя "--- 50, причём 14 опечаток будут замечены обоими корректорами? }
\solution{
$\frac{100!}{14!16!36!34!}0{,}7^{30}0{,}3^{70}0{,}8^{50}0{,}2^{50}$. }



\subsection{Биномиальное распределение (до дисперсии)}
%1.4. Эксперимент состоит из множества одинаковых этапов
%(сюда можно отнести простые задачи на биномиальное распределение и совсем простую комбинаторику)

% спорные случаи - эксперимент повторяется два раза - можно отнести в одношаговые (если обозримо в явном виде выписать все исходы)

\problem{
Монетка подбрасывается 5 раз. Какова вероятность того, что будет
выпадет ровно один орёл? Ровно два? Ни одного? }
\solution{$\PP(N=1)=C_{5}^{1}(\frac{1}{2})^{5}$; $\PP(N=2)=C_{5}^{2}(\frac{1}{2})^{5}$; $\PP(N=0)=C_{5}^{0}(\frac{1}{2})^{5}$. }
\cat{die} \cat{binomial}

\problem{
Какова вероятность при шести подбрасываниях кубика получить ровно
две шестёрки? }
\solution{$\PP(N=2)=C_{6}^{2}(\frac{1}{6})^{2}(\frac{5}{6})^{4}$. }
\cat{binomial}


\problem{
Какова вероятность того, что у десяти человек не будет ни одного совпадения дней рождений? Каков минимальный размер компании, чтобы вероятность одинакового дня рождения была больше половины? }
\solution{$\frac{365\cdot 364\cdot 363\cdot \ldots \cdot 356}{365^{n}}$; минимальная компания состоит из $23$ человек. }


\problem{
Маша подбрасывает монетку три раза, а Саша "--- два раза. Какова
вероятность того, что у Маши герб выпадет больше раз, чем у
Саши?}
\solution{ $\PP=\frac{1}{4}(1-\frac{1}{8})+\frac{1}{2}(\frac{1}{8}+3\frac{1}{8})+\frac{1}{4}\frac{1}{8}=\frac{1}{2}$.}


\problem{ \label{deti raznih polov}
Сколько детей должно быть в семье, чтобы вероятность того,
что имеется по крайней мере один ребенок каждого пола, была больше
0,95? }
\solution{$(\frac{1}{2})^{(n-1)}\le 0{,}15$.  }


\problem{ \zdt{Осторожный фальшивомонетчик}

Дворцовый чеканщик кладёт в каждый ящик вместимостью в сто монет
одну фальшивую. Король подозревает чеканщика и подвергает проверке
монеты, взятые наудачу по одной в каждом из 100 ящиков.
\begin{enumerate}
\item Какова вероятность того, что чеканщик не будет разоблачён?
\item Каков ответ в предыдущей задаче, если 100 заменить на $n$?
\end{enumerate}
\begin{ist}
Mosteller.
\end{ist}
}
\solution{ $0{,}99^{n}$. }


\problem{ \label{strategia udvoenia} \zdt{Стратегия удвоения}

В казино имеется рулетка, которая с вероятностью $0{,}5$ выпадает
или на чёрное, или на красное. Игрок, поставивший сумму $n$ и угадавший
цвет, получает обратно сумму $2n$. Вася решил играть по следующей
схеме. Сначала он ставит доллар. Если он выигрывает, то покидает
казино, если проигрывает, то удваивает ставку и ставит два
доллара. Если выигрывает, то покидает казино, если проигрывает, то
снова удваивает ставку и ставит четыре доллара и т.\,д., пока не
выиграет в первый раз или впервые не хватит денег на новую
удвоенную ставку. У Васи имеется 1\,050 долларов.
\begin{enumerate}
\item Какова вероятность того, что Вася покинет казино после выигрыша?
\item Каков ожидаемый выигрыш Васи?
\end{enumerate}
\begin{note}
В реальности вероятность меньше $0{,}5$, т.\,к. на
рулетке есть 0 и (иногда) 00. Их наличие, естественно, уменьшает и
вероятность, и ожидаемый выигрыш.
\end{note}
 }
\solution{ $1-\frac{1}{1024}$; $0$. }


\problem{ \ENGs
When the $n$'s dice are thrown at the one time, find the probability such that the sum of the cast is $n+3$? \RUSs }
\solution{ }


\problem{
Пусть $X_{1}$, $X_{2}$,..., $X_{n}$ "--- НОРСВ, такие, что $X_{i}= \begin{cases}  1, & p; \\ 0, & (1-p).\end{cases}$ Пусть $k$ "--- такая константа, что $2k\ge n$. Найдите вероятность того, что самая длинная серия из единиц имеет длину $k$. Что делать при $2k<n$? }
\solution{ }
\cat{wrong_class}


\problem{ \ENGs
Suppose you are given a random number generator, which draws samples from an uniform distribution between $0$ and $1$.
The question is: how many samples you have to draw, so that you are 95\% sure that at least 1 sample lies between $0.70$ and $0.72$? \RUSs }
\solution{ }


\problem{  \zdt{Биномиальное распределение}

Кубик подбрасывают 5 раз. Пусть $N$ "--- количество выпадений шестёрки. Найдите $\PP(N=3)$, $\PP(N=k)$  и
$\PP(N>4 \mid N>3)$, $\E(N)$.}
\solution{ }

\problem{ \zdt{Максимальная вероятность для биномиального распределения}

Пусть $X$ распределена биномиально. Общее число экспериментов
равно $n$, вероятность успеха в отдельном испытании равна $p$.
\begin{enumerate}
\item Найдите $\frac{\PP(X=k)}{\PP(X=k-1)}$.
\item При каких $k$ дробь $\frac{\PP(X=k)}{\PP(X=k-1)}$ будет не меньше 1?
\item Каким должно быть $k$, чтобы $\PP(X=k)$ была максимальной?
\end{enumerate}
 }
\solution{ }

\problem{ Известно, что предварительно зарезервированный билет на автобус
дальнего следования выкупается с вероятностью 0{,}9. В обычном
автобусе 18~мест, в микроавтобусе 9~мест. Компания <<Микро>>,
перевозящая людей в микроавтобусах, допускает резервирование 10~билетов на один микроавтобус. Компания <<Макро>>, перевозящая
людей в обычных автобусах допускает резервирование 20~мест на один автобус. У какой компании больше вероятность оказаться в ситуации нехватки
мест? }
\solution{ }

\problem{ \ENGs
The psychologist Tversky and his colleagues say that about four
out of five people will answer (a) to the following question:
\begin{quote}
A certain town is served by two hospitals. In the larger
hospital about 45 babies are born each day, and in the smaller
hospital 15 babies are born each day. Although the overall
proportion of boys is about 50 percent, the actual proportion at
either hospital may be more or less than 50 percent on any day. At
the end of a year, which hospital will have the greater number of
days on which more than 60 percent of the babies born were boys?
\end{quote}
\begin{center}
\begin{tabular}{ccc}
(a) the large hospital & (b) the small hospital & (c) neither (about the same) \\
\end{tabular}
\end{center}\RUSs

Дайте верный ответ и попытайтесь объяснить, почему большинство
людей ошибается при ответе на этот вопрос. }
\solution{В маленьком роддоме <<мальчиковых>> дней больше. В силу закона больших чисел, чем больше число наблюдений, тем сильнее выборочная доля мальчиков должна быть похожа на вероятность рождения мальчика.}

\problem{
В забеге участвуют 12 лошадей. Каждый из 10 зрителей пытается составить свой прогноз для трёх призовых мест. Какова вероятность того, что хотя бы один из них окажется прав? }
\solution{ $ 1-(\frac{1319}{1320})^{10}\approx 0{,}008 $. }


\problem{
Есть $N$ монеток. Каждая из них может быть фальшивой с
вероятностью $p$. Известно, сколько весят настоящие. Известно, что
фальшивые весят меньше, чем настоящие. Каждая фальшивая может иметь
своё отклонение от правильного веса. Задача "---
определить, является ли фальшивой каждая монета. Предлагается следующий способ:
\begin{quote}
Разбить монеты на группы по $n$ монет в каждой группе. Взвесить
каждую группу. Если вес группы совпадает с эталонным, то вся
группа признается настоящей. Если вес группы меньше эталонного, то
каждая монеты из
группы взвешивается отдельно.
\end{quote}

Предположим, что $N$ делится на $n$. Пусть $X$ "--- требуемое число взвешиваний.
\begin{enumerate}
\item Найдите $\E(X)$;
\item При каком условии на $p$ и $n$ предложенный способ более
эффективен чем взвешивание каждой монеты?
\item Исследуйте поведение функции $\frac{\E(X)}{N}$ от $n$ (есть ли минимум, максимум и т.\,д.).
\end{enumerate}
 }
\solution{ }


\problem{ \zdt{Задача Банаха (Banach's matchbox problem)}

У Маши есть две коробки, в каждой из которых осталось по $n$~конфет. Когда Маша хочет конфету, она выбирает наугад одну из
коробок и берёт конфету оттуда. Рано или поздно Маша впервые
откроет пустую коробку. В этот момент другая коробка содержит
некоторое количество конфет. Обозначим за $u_r$ вероятность того, что
в другой коробке ровно $r$ конфет.
\begin{enumerate}
\item  Найдите $u_r$.
\item  Найдите вероятности $v_r$ того, что в тот момент, когда из
одной коробки возьмут последнюю конфету (она только станет
пустой!), в другой будет находится ровно $r$ конфет.
\item  Найдите вероятность того, что коробка, которая была опустошена
раньше, не будет первой коробкой, открытой пустой.
\end{enumerate}
 }
\solution{
Пункт 1. Последняя попытка взять конфету "--- из пустой коробки. Назовём
эту коробку $A$. Из предыдущих $n+(n-r)$ конфет $n$ приходятся на
коробку $A$. Вероятности равны $\frac{1}{2}$. Получаем:
$u_{r}=\frac{C_{2n-r}^{n}}{2^{n+(n-r)}}$ }


\problem{
В уездном городе $N$ два родильных дома, в одном ежедневно рождается 50 человек, в другом "--- 100 человек. В каком роддоме чаще рождается одинаковое количество мальчиков и девочек?}
\solution{В маленьком. }

\problem{
\ENGs Let you choose an infinite sequence of integers between 1 and 10, what is the possibility that your sequence doesn't have any ``1''? \RUSs }
\solution{ 0. }


\problem{ \ENGs
There are three coins in a box.  These coins when flipped, will
come up heads with respective probabilities $0.3$, $0.5$, $0.7$.  A
coin is randomly selected (meaning uniform distribution!) from among
these three and then flipped $10$ times.  Let $N$ be the number of
heads obtained on the first ten flips. \RUSs
\begin{enumerate}
\item Найдите $\PP(N=0)$.
\item If you win \$1 each time a head appears and you lose \$1 each time a tail appears, is this a fair game?  Explain.
\end{enumerate}
 }
\solution{ }


% серия задач, которые могут казаться интуитивно противоречивыми...
\problem{Как почувствовать разницу в 0{,}01\cite{sekei:paradox}? Пусть вероятность того, что Маша находится целый день в хорошем настроении, равна 0{,}99, а вероятность того, что Саша находится в хорошем настроении, равна 0{,}999\,9. Какова вероятность того, что Маша будет целый год непрерывно в хорошем настроении? Саша?}
\solution{0{,}025\,5 и 0{,}964\,2.}


\problem{Петя подбрасывает 10 монеток. Если из этих 10~подбрасываний будет как минимум 8~одинаковых, то мы назовём это чудом. Какова вероятность чуда? Какова вероятность хотя бы одного чуда, если, кроме Пети, ещё 9~человек подбрасывает по 10~монеток?}
\solution{$\frac{7}{64}$; около $\frac{2}{3}$.}

\problem{Вероятность того, что прошлогодний грецкий орех будет червивым, равна 0{,}25. Сколько минимум нужно взять грецких орехов, чтобы среди них был хотя бы один нормальный с вероятностью 99,9\,\%?}
\solution{5.}

\problem{ \zdt{Биномиальные числа Фибоначчи}

Пусть $\{F_k\}$ "--- последовательность Фибоначчи, а $X$ "--- число выпавших орлов при $n$ подбрасываниях правильной монетки. Вычислить $\E(F_{1+X})$.
\begin{ist}
Алексей Суздальцев.
\end{ist}
}

\solution{$\frac{F_{2n+1}}{2^n}$. У чисел Фибоначчи есть свойство: $F_{n}=F_{n-1}+F_{n-2}\hm =L(1+L)F_{n} \hm=\ldots \hm =L^{k}(1+l)^{k}F_{n} \hm=(1+L)^{k}F_{n-k}$.}

\problem{Семеро друзей выбирают, пойти им в кино на фильм ужасов или на комедию. Каждый из них предпочтёт комедию независимо от других с вероятностью $0{,}6$. Есть два способа голосования, А и Б. Способ~А "--- все голосуют одновременно, выбирается альтернатива, набравшая больше голосов. Способ~Б "--- голосование в два тура. Первый тур: трое самых старших друзей голосуют между собой и большинством решают, за что они втроём будут голосовать единогласно во втором туре: за комедию или за ужасы. Второй тур: голосуют все семеро, но трое старших голосуют так, как согласованно договорились на первом туре. При каком способе голосования выше шансы пойти на комедию? }
\solution{}


\problem{Маша и Саша учатся в одном классе. Маша и Саша учатся по одним и тем же $n$ учебникам. В один день Маша и Саша независимо друг от друга приносят случайное подмножество своих учебников в школу.
\begin{enumerate}
\item Какова вероятность того, что у Маши не будет ни одного учебника, которого бы не было у Саши?
\item Какова вероятность того, что вместе у Саши и Маши будут все $n$ учебников хотя бы в одном экземпляре?
\end{enumerate}}
\solution{Можно считать, что каждый учебник Саша и Маша берут с вероятностью $0{,}5$. Ответ в обоих пунктах: $0{,}75^n$.}


\subsection{Деревья и прочее без условных вероятностей}

\problem{
Подбрасывается кубик, а затем монетка подбрасывается столько раз,
сколько очков на выпавшей грани. Какова вероятность того, что
орёл выпадет ровно 4 раза?}
\solution{ $\PP=\frac{1}{6}\left(\left(\frac{1}{2}\right)^{4}+C_{5}^{4}\left(\frac{1}{2}\right)^{5}+C_{6}^{4}\left(\frac{1}{2}\right)^{6}\right).$
}


\problem{ \label{ritsari-bliznetsi} \zdt{Рыцари-близнецы }

Король Артур проводит рыцарский турнир, в котором, так же как и в
теннисе, порядок состязания определяется жребием. Среди восьми рыцарей, одинаково искусных в
ратном деле, два близнеца.
\begin{enumerate}
\item Какова вероятность того, что они встретятся в поединке?
\item Каков ответ в случае $2^n$ рыцарей?
\end{enumerate}
 }
\solution{$\PP_1=\frac{1}{7}+\frac{1}{14}+\frac{1}{28}$; $\PP_2=\frac{1}{2^{n}-1}\cdot 2\cdot\left(1-0{,}5^{n}\right)$.  }
\begin{ist}
Mosteller.
\end{ist}

\problem{ \label{Vasia i Petia na lektsii}
Вася посещает 60\,\% лекций по теории вероятностей, Петя "--- 70\,\%. Они
из разных групп и посещают лекции независимо друг от друга. Какова
вероятность, что на следующую лекцию придут оба? Хотя бы один из
них?}
\solution{ $\PP(N=2)=0{,}7\cdot 0{,}6=0{,}42$. $\PP(N\geq 1)=1-(1-0{,}7)\cdot (1-0{,}6)=0{,}88$. }

\problem{ \label{vtoroi v finale} \zdt{Выйдет ли второй в финал?} \par
В теннисном турнире участвуют 8~игроков. Есть три тура
(четвертьфинал "--- полуфинал "--- финал). Противники в первом туре
определяются случайным образом. Предположим, что лучший игрок
всегда побеждает второго по мастерству, а тот, в свою очередь
побеждает всех остальных. Проигрывающий в финале занимает второе
место. Какова вероятность
того, что это место займет второй по мастерству игрок?
\begin{ist}
обработка Mosteller.
\end{ist}
}
\solution{ $\PP=\frac{4}{7}$. }

\problem{ \ENGs
The Wimbledon Men's Singles Tournament has 128 players. The first round pairings are completely random, subject to the constraint that none of the top 32 players can be paired against each other. Two competitors, Olivier Rochus, and his brother Christophe are competing, and neither are in the elite group of 32 players. What is the probability that these brothers play in the first round (as actually occurred)? \RUSs }
\solution{ }

\problem{
Первый автобус отходит от остановки в 5:00. Далее интервалы между
автобусами равновероятно составляют 10 или 15 минут, независимо от
прошлых интервалов. Вася приходит на остановку в 5:42.
\begin{enumerate}
\item Какова ожидаемая длина интервала, в который он попадает?
\item  Какова ожидаемая длина следующего интервала?
\item  Пусть $t\to\infty$ (???)
\end{enumerate}
 }
\solution{ Ожидаемая длина следующего интервала "--- 12{,}5 минут. }

\problem{ \ENGs
There are two ants on opposite corners of a cube. On each move, they can travel along an edge to an adjacent vertex. What is the probability that they both return to their starting point after 4 moves? \RUSs }
\solution{$(\frac{7}{27})^{2}$. }


\problem{ \label{legkomislennii chlen juri} \zdt{Легкомысленный член жюри} \par
В жюри из трёх человек два члена независимо друг от друга
принимают правильное решение с вероятностью $p$, а  третий для
вынесения    решения бросает монету (окончательное решение
выносится большинством голосов). Жюри из одного человека выносит
справедливое решение с вероятностью $p$. Какое из этих жюри
выносит справедливое решение с большей вероятностью?
\begin{ist}
Mosteller.
\end{ist}
 }
\solution{ $p-\frac{p^{2}}{2}<p$, т.\,е. жюри из одного человека лучше. }


\problem{ \label{Simpson's paradox} \zdt{Simpson's paradox} \par
Тренер хочет отправить на соревнование самого сильного из своих
спортсменов. Самым сильным игроком тренер считает того, у кого
больше всех шансов получить первое место, если бы соревнование
проводилось среди своих. У тренера два спортсмена: А, постоянно
набирающий 3~штрафных очка при выполнении упражнения, и Б,
набирающий 1~штрафное очко с вероятностью 0{,}54 и 5~штрафных очков
с вероятностью 0{,}46.
\begin{enumerate}
\item Кого отправит тренер на соревнования?
\item Кого отправит тренер на соревнования, если, помимо А и Б, у него
тренируется спортсмен В, набирающий 2 штрафных очка с вероятностью
0,56, 4 штрафных очка с вероятностью 0,22 и 6 штрафных очков с
вероятностью 0,22.
\item Мораль?
\end{enumerate}
 }
\solution{ Спортсмена Б, если нет спортсмена В; спортсмена А, если есть спортсмен В. Мораль "--- зависимость от третьей альтернативы. }


\problem{
В турнире участвуют 8 человек, разных по силе. Более сильный побеждает более слабого. Проигравший выбывает, победитель выходит в следующий тур.
Какова вероятность того, что $i$-й по силе игрок дойдет до финала? }
\solution{ }


\problem{ \ENGs
A bag contains a total of $N$ balls either blue or red. If $5$ balls are chosen from the bag the probability all of them being blue is 0.5. What are the values of $N$ for which this is possible? \RUSs}
\solution{ }

\problem{ \ENGs
Each of two boxes contains both black and white marbles, and the total number of marbles in the two boxes is 25. One marble is taken out of each box randomly. The probability that both marbles are black is $\frac{27}{50}$. What is the probability that both marbles are white? \RUSs }
\solution{ }


\problem{ \label{gadanie v pole}
Маша с подружкой гуляют в поле. Подружка предлагает погадать на
суженого. Она зажимает в руке шесть травинок так, чтобы концы
травинок торчали сверху и снизу. Маша сначала связывает эти
травинки попарно между собой сверху, а затем и снизу (получается
три завязывания сверху и три завязывания снизу). Если при этом все
шесть травинок окажутся связанными в одно кольцо, то это означает,
что Маша в текущем году выйдет замуж.
Какие шансы у Маши?
\begin{note}
Будем считать, что завязывание травинок в <<трилистник>>, <<восьмерку>> и прочие нетривиальные узлы также
означает замужество.
\end{note}
\begin{ist}
Баврин, Фрибус, <<Старинные задачи>>.
\end{ist}
 }
\solution{$\frac{8}{15}$. }


\problem{
Две урны содержат одно и то же количество
шаров, несколько чёрных и несколько белых каждая. Из них
извлекаются $n$ ($n>3$) шаров с возвращением. Найти число $n$ и
содержимое обеих урн, если вероятность того, что все белые шары
извлечены из первой урны, равна вероятности того, что из второй
извлечены либо все белые, либо все чёрные шары.
\begin{ist}
Mosteller.
\end{ist}
}
\solution{ }
\cat{wrong_class}

\problem{ \label{po rublu za 6}
Кость подбрасывается 3 раза. Размер ставки "--- 1 рубль. Если
шестёрка не выпадает ни разу, то ставка проиграна, если шестёрка
выпадает хотя бы один раз, то ставка возвращается, плюс
выплачивается выигрыш по 1 рублю за каждую шестёрку. Найдите
стоимость этой лотереи. }
\solution{$\E(X)=3\cdot\frac{1}{6}+1+(-2)\cdot\left(\frac{5}{6}\right)^{3}$ }

\problem{
\label{Parrondo's game} \zdt{Parrondo's game}

Назовем <<рублёвой игрой с вероятностью $p$>> игру, в которой
игрок выигрывает 1~рубль с вероятностью $p$ и проигрывает один
рубль с вероятностью $(1-p)$. Игра $A$ "--- это рублёвая игра с вероятностью 0,45.
Игра $B$ состоит в следующем: если сумма в твоём кошельке делится
на три, то ты играешь в рублёвую игру с вероятностью 0,05; если же
сумма в твоем кошельке не делится на три, то ты играешь в рублёвую
игру с вероятностью 0{,}7. Что будет происходить с ожидаемым благосостоянием игрока, если он
\begin{enumerate}
\item Будет постоянно играть в игру $A$?
\item  Будет постоянно играть в игру $B$?
\item  Будет постоянно играть $A$ или $B$ с вероятностью по 0,5?
\item  Придумайте <<лохотрон>> для интеллектуалов.
\end{enumerate}
 }
% d - идея Ромы Мартусевича
\solution{ При игре только в $A$ "--- убывать; только в $B$ убывать; в вероятностную комбинацию "--- возрастать. }

\problem{ \zdt{Parrondo's game --- альтернатива} \ENGs

A much simpler example is dealing cards from a well-shuffled deck. Suppose I get \$14 if two cards in a row match in rank (two 2's or two Kings for examples), and pay \$1 if they don't. The chance of two cards in a row matching is $\frac{1}{17}$, so I pay \$16 for each \$14 I win.

Now I play the same game, alternating the deal between two decks. Now the chance of two successive cards matching is $\frac{1}{13}$, so I pay \$12 for every \$14 I win.

Each game individually loses money, but alternate them and you win money. Eureka! We're all rich. \RUSs
\begin{ist}
Wilmott forum.
\end{ist}
 }
\solution{ }

\problem{ \zdt{Триэль }

Три гусара "--- $A$, $B$ и $C$ "--- стреляются за прекрасную даму. Стреляют
они по очереди ($A$, $B$, $C$, $A$, $B$, $C$, \ldotst), каждый стреляет в
противника по своему выбору. $A$ попадает с вероятностью 0.1, $B$
"--- 0.5, $C$ "--- 0.9. Триэль продолжается до тех пор, пока в живых не
останется только один. Предположим, что стрелять в воздух нельзя.
\begin{enumerate}
\item Какой должна быть стратегия $A$?
\item У кого какие шансы на победу?
\end{enumerate}
 }
\solution{ }

\problem{ \zdt{Триэль-2}

Три гусара "--- $A$, $B$ и $C$ "--- стреляются за прекрасную даму. Стреляют
они одновременно, каждый стреляет в противника по своему выбору.
$A$ попадает с вероятностью 0,1, $B$ "--- 0,5, $C$ "--- 0,9. Триэль
продолжается до тех пор, пока в живых не
останется только один или никого.
\begin{enumerate}
\item Какой должна быть стратегия $A$?
\item У кого какие шансы на прекрасную даму?
\end{enumerate} }
\solution{ }


\problem{
% переписать (у Менделя - не горошины вроде бы?)
У диплоидных организмов наследственные характеристики определяются
парой генов. Вспомним знакомые нам с 9-го класса горошины чешского
монаха Менделя. Ген, определяющий форму горошины, имеет две
аллели:  <<А>> (гладкая) и <<а>> (морщинистая). <<А>> доминирует над
<<а>>. В популяции бесконечное количество организмов. Родители
каждого потомка определяются случайным образом. Одна аллель
потомка выбирается наугад из аллелей матери, другая "--- из аллелей
отца. Начальное распределение
генотипов имеет вид: <<АА>> "--- 30\,\%, <<Аа>> "--- 60\,\%, <<аа>> "--- 10\,\%.
\begin{enumerate}
\item  Каким будет распределение генотипов в $n$-м поколении?
\item  Заметив закономерность, сформулируйте и докажите теорему
Харди"--~Вайнберга для произвольного начального распределения
генотипов.
\end{enumerate}
 }
\solution{ }

\problem{
У диплоидных организмов наследственные характеристики определяются
парой генов. Некий ген, сцепленный с полом, имеет две аллели:
<<А>> и <<а>>, т.\,е. девочка может иметь один из трёх генотипов
(<<АА>>, <<Аа>>, <<аа>>), а мальчик "--- только два (<<А>> и <<а>>; хромосома,
определяющая мужской пол, короче и не содержит нужного участка).
От мамы ребёнок наследует одну из двух аллелей (равновероятно), а
от отца либо наследует (тогда получается девочка), либо нет (тогда
получается мальчик). <<А>> доминирует <<а>>. В популяции
бесконечное количество организмов. Родители каждого
потомка определяются случайным образом.
\begin{enumerate}
\item Верно ли, что численность генотипов стабилизируется со временем?
\item Известно, что дальтонизм является признаком, сцепленным с
полом. Догадавшись, является ли он рецессивным или доминантным,
определите, среди кого (мужчин или женщин) дальтоников больше.
\end{enumerate}

/проверить биологию/ }
\solution{ }



\problem{
В коробке находится четыре внешне одинаковые лампочки. Две
лампочки исправны, две "--- нет. Лампочки извлекают из коробки по
одной до тех пор, пока не будут извлечены обе исправные.
\begin{enumerate}
\item Какова вероятность того, что опыт закончится извлечением трёх
лампочек?
\item  Каково ожидаемое количество извлеченных лампочек?
\end{enumerate}
 }
\solution{ }


\problem{ \label{spelestolog} \zdt{Спелестолог и батарейки}

У спелестолога в каменоломнях сели батарейки в налобном фонаре, и он оказался в абсолютной темноте. В рюкзаке у него 8~батареек: 5 новых и 3 старых. Для работы фонаря требуется две новые батарейки. Спелестолог вытаскивает из рюкзака две батарейки наугад и вставляет их в фонарь. Если фонарь не начинает работать, то спелестолог откладывает эти две батарейки и пробует следующие и т.\,д.
\begin{enumerate}
\item Сколько попыток в среднем потребуется?
\item Какая попытка вероятнее всего будет первой удачной?
\item Творческая часть. Поиграйтесь с задачей. Случайна ли прогрессия в ответе? Сравните с вариантом «6 новых $+$ 4 старых» и т.\,д.
\end{enumerate}
}

\solution{ \begin{tabular}{|c|c|c|c|c|}\hline
$N$ & 1 & 2 & 3 & 4 \bigstrut \\ \hline
$\PP$ & $\frac{5}{14}$ & $\frac{4}{14}$ & $\frac{3}{14}$ & $\frac{2}{14}$ \bigstrut \\ \hline
\end{tabular}

Решение для $6=4+2$: $\PP(N=1)=\frac{C_{4}^{2}}{C_{6}^{2}}=\frac{6}{15}$; $\PP(N=3)=\frac{4\cdot 2}{C_{6}^{2}}\frac{3\cdot 1}{C_{5}^{2}}=\frac{4}{15}$; $\PP(N=2)=\frac{5}{15}$; $\E(N)=\frac{28}{15}$. }



\problem{ \label{dva ferzia}
Два ферзя (чёрный и белый) ставятся наугад на шахматную доску.
\begin{enumerate}
\item Какова вероятность того, что они будут <<бить>> друг друга?
\item К чему стремится эта вероятность для шахматной доски со
стороной, стремящейся к бесконечности?
\end{enumerate}
 }
\solution{Вероятность того, что ферзи будут угрожать друг другу, равна $\frac{14}{63}+\frac{1}{64}\frac{1}{63}4(7\cdot 7+5\cdot 9+3\cdot 11+ 1\cdot 13)$.
Шахматная доска делится на четыре квадратных зоны с одинаковым числом клеток, покрываемых ферзём. Если длина стороны будет стремиться в бесконечность, то эта вероятность будет стремиться к
0, так как она равна отношению длин нескольких линий ко всей площади.  }


\problem{
На день рождения к Васе пришли две Маши, два Саши, Петя и Коля. Все вместе с Васей сели за круглый стол. Какова вероятность, что Вася окажется между двумя тёзками? }
\solution{ Слева должен сесть тот, у кого есть тёзка. $p_{1}=\frac{4}{6}$. Справа должен сесть его парный. $p_{2}=\frac{1}{5}$, итого $p=p_{1}\cdot p_{2}=\frac{2}{15}$. }



\problem{
Равновероятно независимо друг от друга выбираются три числа от 1 до 20. Какова вероятность того, что третье попадет между двух первых? }
\solution{ $\frac{57}{200}=0{,}285$. }

\problem{ \ENGs Five distinct numbers are randomly distributed to players numbered 1 through 5. Whenever two players compare their numbers, the one with the higher one is declared the winner. Initially, players 1 and 2 compare their numbers; the winner then compares with player 3. Let $X$ denote the number of times player 1 is a winner. Find the distribution of $X$. \RUSs }
\solution{ }



\problem{ \label{simple optimization}
Подбрасывается правильный кубик. Узнав результат, игрок выбирает,
подкидывать ли кубик второй раз. Игрок получает сумму денег, равную
количеству очков при последнем подбрасывании.
\begin{enumerate}
\item Каков ожидаемый выигрыш игрока при оптимальной стратегии?
\item Каков ожидаемый выигрыш игрока, если максимальное количество подбрасываний равно трём?
\end{enumerate}
 }
\solution{$\frac{1}{2}\cdot5+\frac{1}{2}\frac{7}{2}=4{,}25$. }

\problem{На столе стоят 42~коробки, они занумерованы от 0 до 41. В каждой коробке 41~шар, в коробке с номером $i$ лежат $i$ белых шаров, а остальные чёрные. Мы наугад выбираем коробку, а затем из неё достаём три шара. Какова вероятность того, что они будут одного цвета?
\begin{ist}
\url{http://math.stackexchange.com/questions/70760/}
\end{ist}
} % 42 --- это потому что это число является ответом на вопрос Жизни, Вселенной и Всего Такого?
\solution{Можно представить себе другое условие: в коробке 42 занумерованных шара, мы выбираем один наугад. Красим шары с меньшим номером в белый, остальные "--- в чёрный. Затем берём три шара. Это равносильно тому, что мы возьмём 4~шара с номерами 1, 2, 3, 4 и выберем из них разделитель цветов случайно. Значит, вероятность равна $\frac{1}{2}$.}



% !Mode:: "TeX:UTF-8"
\section{Условные вероятности и ожидания. Дополнительная информация}
%Правило умножения вероятностей:
%Если A B независимы, то

\subsection{Условная вероятность}

\problem{ \ENGs
A bag contains a counter, known to be either white or black. A white counter is put in, the bag is shaken, and a counter is drawn out, which proves to be white. What is now the chance of drawing a white counter? \RUSs}
\solution{ }

\problem{ \ENGs
You have a hat in which there are three pancakes: one is golden on both sides, one is brown on both sides, and one is golden on one side and brown on the other. You withdraw one pancake, look at one side, and see that it is brown. What is the probability that the other side is brown? \RUSs}
\solution{ }

\problem{ \ENGs
The inhabitants of an island tell truth one third of the time. They lie with the probability of $\frac{2}{3}$. On an occasion, after one of them made a statement, another fellow stepped forward and declared the statement true. What is the probability that it was indeed true? \RUSs }
\solution{ }


\problem{
На кубиках написаны числа от 1 до 100. Кубики свалены в кучу. Вася выбирает наугад из кучи по очереди три кубика.
\begin{enumerate}
\item Какова вероятность, что полученные три числа будут идти в возрастающем порядке?
\item Какова вероятность, что полученные три числа будут идти в возрастающем порядке, если известно, что первое меньше последнего?
\end{enumerate}
 }
\solution{ $\frac{1}{6}$; $\frac{1}{3}$. }


\problem{ (дописать)

Наследование группы крови контролируется аутосомным геном. Три его аллеля обозначаются буквами А, В и 0. Аллели А и В доминантны в одинаковой степени, а аллель 0 рецессивен по отношению к ним обоим. Поэтому существует четыре группы крови. Им соответствуют следующие генотипы:
\begin{itemize}
\item Первая (I) "--- 00;
\item Вторая (II) "--- АА, А0;
\item Третья (III) "--- ВВ, В0;
\item Четвёртая (IV) "--- АВ.
\end{itemize}

Наследование резус-фактора кодируется тремя парами генов и происходит независимо от наследования группы крови. Наиболее значимый ген имеет два аллеля, аллель D доминантный, аллель d рецессивный. Таким образом, получаем следующие генотипы:
\begin{itemize}
\item Резус-положительный "--- DD, Dd;
\item Резус-отрицательный "--- dd.
\end{itemize}

Если у беременной женщины резус"=отрицательная кровь, а у плода резус"=положительная, то есть риск возникновения гемолитической болезни (у матери образуются антитела к резус фактору, безвредные для неё, но вызывающие разрушение эритроцитов плода).

Перед нами два семейства: Монтекки и Капулетти. \\
...}
\solution{ }


\problem{ \label{rekordnaia volna}
Пусть $X_{i}$ "--- НОРСВ, такие, что $\PP(X_{i}=X_{j})=0$. Обозначим за
$E_{k}$ событие, состоящее в том, что $X_{k}$ оказалась
<<рекордом>>, т.\,е. больше, чем все предыдущие $X_{i}$ ($i<k$). Для
определённости будем считать, что $E_{1}=\Omega$.
\begin{enumerate}
\item Найдите $\PP(E_{k})$.
\item  Верно ли, что $E_{k}$ независимы?
\item  Какова вероятность того, что второй рекорд будет зафиксирован в $n$-й момент времени?
\item  Сколько в среднем времени пройдёт до второго рекорда?
\end{enumerate}

\begin{ist}
Williams, 4.3.
\end{ist}
 }
\solution{ Какая-то из первых $k$ величин будет наибольшей. В силу \iid{}
получаем, что $\PP(E_{k})=\frac{1}{k}$. $E_k$ независимы: например, если известно,
что 10-е наблюдение было рекордом, это ничего не говорит о рекордах в первых 9-ти
наблюдениях. Вероятность второго рекорда в $n$-й момент равна $\frac{1}{n(n-1)}$,
а в среднем времени до второго рекорда пройдёт $\infty$. }

\problem{Известно, что $\PP(A \mid B)=\PP(A \mid B^{c})$. Верно ли, что $A$ и $B$ независимы?}
\solution{Да.}


\problem{ \zdt{Randomized response technique}

В анкету для чиновников включён скользкий вопрос: <<Берёте ли Вы
взятки?>>. Чтобы стимулировать чиновников отвечать правдиво,
используется следующий прием. Перед ответом на вопрос чиновник втайне от анкетирующего подкидывает специальную монетку, на гранях
которой написано <<правда>>, <<ложь>>. Если монетка выпадает
<<правдой>>, то предлагается отвечать на вопрос правдиво, если
монетка выпадает на <<ложь>>, то предлагается солгать. Таким
образом, ответ <<да>> не обязательно означает, что чиновник берёт
взятки.

Допустим, что треть чиновников берёт взятки, а монетка
неправильная и выпадает <<правдой>> с вероятностью 0{,}2.
\begin{enumerate}
\item Какова вероятность того, что чиновник ответит <<да>>?
\item  Какова вероятность того, что чиновник берёт взятки, если он
ответил <<да>>? Если ответил <<нет>>?
\end{enumerate}
\todo[inline]{Вставить построение несмещённой оценки?}
}
\solution{ }

\problem{
Пусть события  $A$  и  $B$  независимы и $\PP(B)>0$.
Чему равна  $\PP(A \mid B)$? }
\solution{ $ \PP(A \mid B)=\PP(A)$. }

\problem{
Из колоды в 52 карты извлекается одна карта наугад. Верно ли, что
события <<извлечён туз>> и <<извлечена пика>> являются
независимыми? }
\solution{ Да. }

\problem{
Из колоды в 52 карты извлекаются по очереди две карты наугад.
Верно ли, что события <<первая карта "--- туз>> и <<вторая карта "---
туз>> являются независимыми? }
\solution{ Нет. }

\problem{
Известно, что $\PP(A)=0{,}3$, $\PP(B)=0,{4}$, $\PP(C)=0{,}5$. События
$A$ и $B$ несовместны, события $A$ и $C$ независимы и
$\PP(B\mid C)=0{,}1$.
Найдите $\PP(A\cup B\cup C)$. }
\solution{ }

\problem{
Имеется три монетки. Две <<правильных>> и одна "--- с орлами по
обеим сторонам. Петя выбирает одну монетку наугад и подкидывает её
два раза. Оба раза выпадает орёл. Какова вероятность того, что
монетка <<неправильная>>? }
\solution{ }

\problem{
Самолёт упал либо в горах, либо на равнине. Вероятность того, что самолёт упал в горах, равна 0{,}75. Для поиска пропавшего самолёта выделено 10 вертолётов. Каждый вертолёт можно использовать только в одном месте. Как распределить имеющиеся вертолёты, если вероятность обнаружения пропавшего самолёта отдельно взятым вертолётом равна: $0{,}95$? $0,6$ (пасмурно)? $0{,}1$ (туман)? }
\begin{ist}
Айвазян, экзамен РЭШ.
\end{ist}
\solution{ }

\problem{
Предположим, что социологическим опросам доверяют 70\,\% жителей. Те, кто доверяет опросам, всегда отвечают искренне; те, кто не доверяет, отвечают наугад, равновероятно выбирая <<да>> или <<нет>>. Социолог Петя  в анкету очередного опроса включил вопрос: <<Доверяете ли Вы социологическим опросам?>>
\begin{enumerate}
\item Какова вероятность, что случайно выбранный респондент ответит <<Да>>?
\item  Какова вероятность того, что он действительно доверяет, если известно, что он ответил <<Да>>?
\end{enumerate}
 }
\solution{ }

\problem{
Два охотника выстрелили в одну утку. Первый попадает с
вероятностью 0{,}4, второй "--- с вероятностью 0{,}6. В утку попала ровно
одна пуля. Какова вероятность того, что утка была убита первым
охотником? }
\solution{
$p=\frac{0{,}4\cdot 0{,}4}{0{,}4\cdot 0{,}4+0{,}6\cdot 0{,}6}=\frac{4}{13}$.}

\problem{
С вероятностью 0{,}3 Вася оставил конспект в одной из 10
посещённых им сегодня аудиторий. Вася осмотрел 7 из 10 аудиторий и
конспекта в них не нашёл.
\begin{enumerate}
\item  Какова вероятность того, что конспект будет найден в следующей
осматриваемой им аудитории?
\item  Какова (условная) вероятность того, что конспект оставлен
где-то в другом месте?
\end{enumerate}
 }
\solution{ }

\problem{
Вася гоняет на мотоцикле по единичной окружности с центром в
начале координат. В случайный момент времени он останавливается.
Пусть случайные величины  $X$  и  $Y$  "--- это Васины абсцисса и
ордината в момент остановки. Найдите  $\PP\left(X>\frac{1}{2} \right)$,
$\PP\left(X>\frac{1}{2} \bigm| Y<\frac{1}{2} \right)$. Являются ли события
$A=\left\{X>\frac{1}{2} \right\}$  и
$B=\left\{Y<\frac{1}{2} \right\}$  независимыми?
\begin{hint}
$\cos\left(\frac{\pi }{3} \right)=\frac{1}{2}$, длина окружности $l=2\pi r$.
\end{hint}
}
\solution{ }

\problem{
Пусть  $\PP(A)=1/4$,  $\PP(A \mid B)=\frac{1}{2}$  и $\PP(B\mid A)=\frac{1}{3}$. Найдите $\PP(A\cap
B)$, $\PP(B)$  и  $\PP(A\cup B)$.}
\solution{ }

\problem{
Примерно\footnote{Цифры условные. Celui qui ne mange pas de
bifsteak au cause de la vache folle --- il est fou! Jolivaldt.} 4\,\%
коров заражены <<коровьим бешенством>>. Имеется тест, позволяющий
с определённой степенью достоверности установить, заражено ли мясо
прионом или нет. С вероятностью $0{,}9$ заражённое мясо будет признано
заражённым. <<Чистое>> мясо будет признано заражённым с
вероятностью 0{,}1. Судя по тесту, эта партия мяса заражена. Какова
вероятность того, что она действительно заражена?}
\solution{ }

\problem{
\emph{Роме Протасевичу, искавшему со мной у Мутновского
вулкана в
тумане серую палатку...}

Есть две тёмные комнаты, $A$ и $B$. В одной из них сидит чёрная кошка.
Первоначально предполагается, что вероятность нахождения кошки в
комнате $A$ равна $\alpha$. Вероятность найти чёрную кошку в темной
комнате (если она там есть) с одной попытки равна $p$.  Допустим,
что вы сделали $a$ неудачных попыток поиска кошки в комнате $A$ и
$b$ неудачных попыток в комнате $B$.
\begin{enumerate}
\item Чему равна условная вероятность нахождения кошки в комнате $A$?
\item  При каком условии на $(a-b)$ эта вероятность будет больше
$0{,}5$?
\end{enumerate}
 }
\solution{ }

\problem{
Кубик подбрасывается два раза. Найдите вероятность
получить сумму, равную 8, если на первом кубике выпало 3.}
\solution{ $\frac{1}{6}$. }

\problem{
В коробке 10 пронумерованных монеток, $i$-я монетка выпадает
орлом с вероятностью $\frac{i}{10}$. Из коробки была вытащена одна
монетка наугад. Она выпала орлом. Какова вероятность того, что это
была пятая монетка? }
\solution{
$\frac{1}{11}$.  }

\problem{ Вы играете две партии в шахматы против незнакомца. Равновероятно
незнакомец может оказаться новичком, любителем или профессионалом.
Вероятности вашего выигрыша в отдельной партии, соответственно,
будут равны 0{,}9; 0{,}5; 0{,}3.
\begin{enumerate}
\item Какова вероятность выиграть первую партию?
\item Какова вероятность выиграть вторую партию, если вы выиграли
первую?
\end{enumerate}
 }

\solution{ $p_{a}=\frac{1}{3}(0{,}9+0{,}5+0{,}3)=\frac{17}{30}$, $p_{b}=\frac{1}{3}(0{,}9^{2}+0{,}5^{2}+0{,}3^{2})/p_{a}=\frac{115}{170}$. }

\problem{
В каких из перечисленных случаев вероятность наличия флэша (см. \hyperref[combo]{карточные комбинации} на стр.~\pageref{combo}) больше, чем при полном отсутствии информации:
\begin{enumerate}
\item Первая карта из имеющихся "--- это туз;
\item Первая карта из имеющихся "--- это туз бубей;
\item На руках имеется хотя бы один туз;
\item На руках имеется туз бубей.
\end{enumerate}
 }

\solution{ \ENGs Unverified, but no calculation:

An arbitrary hand can have two aces but a flush hand can't.  The
average number of aces that appear in flush hands is the same as the
average number of aces in arbitrary hands, but the aces are spread out
more evenly for the flush hands, so set (3) contains a higher fraction
of flushes.

Aces of spades, on the other hand, are spread out the same way over
possible hands as they are over flush hands, since there is only one of
them in the deck.  Whether or not a hand is flush is based solely on a
comparison between different cards in the hand, so looking at just one
card is necessarily uninformative.  So the other sets contain the same
fraction of flushes as the set of all possible hands. \RUSs }


\problem{\ENGs A man has 3 equally favorite seats to fish at. The probability with which the man can succeed at catching at each seat is 0.6, 0.7, 0.8 respectively. It is known that the man dropped the hint at one seat three times and just caught one fish. Find the probability that the fish was caught at the first seat. \RUSs}
\solution{$\approx 0{,}503$. }


\subsection{Условное среднее}


\problem{ \label{dve shkatulki} \zdt{Две шкатулки}

Васе предлагают две шкатулки и обещают, что в одной из них денег
в два раз больше, чем в другой. Вася открывает наугад одну из них
"--- в ней $a$ рублей. Вася может взять либо деньги, либо
оставшуюся шкатулку.
\begin{enumerate}
\item Правильно ли Вася считает, что ожидаемое количество денег в
неоткрытой шкатулке равно $\frac{1}{2}\left( {\frac{1} {2}a}
\right)+\frac{1}{2}( {2a} ) = 1\frac{1} {4}a$ и что
поэтому нужно изменить свой выбор?
\item Пусть известно, что в пару шкатулок кладут $3^k$ и $3^{k+1}$
рублей с вероятностью $p_k  = ( {\frac{1} {2}} )^k $. Стоит ли
Васе изменить свой выбор после того, как он открыл
первую шкатулку? Почему?
\end{enumerate}
 }
\solution{Вася считает неправильно: условное распределение суммы можно определить, только зная безусловное.

Концепция условного ожидания неприменима? Вставить это в иллюстрацию условного ожидания? При заданном безусловном распределении Васе следует сменить
свой выбор вне зависимости от того, что он увидел в первой шкатулке. Вторая открытая лучше первой открытой. Это возможно
из-за того, что безусловная ожидаемая сумма равна бесконечности
для обеих шкатулок.  }


\problem{В кабинет бюрократа скопилась очередь ещё до его открытия. Пусть время обслуживания страждущих "--- независимые экспоненциальные случайные величины. Посетитель, пришедший через $t$ минут после открытия, узнал, что первый посетитель уже ушёл, а второй ещё сидит в кабинете. Найдите ожидаемое время обслуживания первого посетителя, $\E(X_{1} \mid X_{1}\le t < X_{1}+X_{2}) $.}
\solution{$\frac{t}{2}$.}
\cat{poisson} \cat{exp} % может перекинуть в Пуассоновский процесс?


\problem{
Пете и Васе предложили одну и ту же задачу. Они могут правильно решить её с вероятностями 0{,}7 и 0{,}8 соответственно. К задаче предлагается 5 ответов на выбор, поэтому будем считать, что выбор каждого из пяти ответов равновероятен, если задача решена неправильно.
\begin{enumerate}
\item Какова вероятность несовпадения ответов Пети и Васи?
\item  Какова вероятность того, что Петя ошибся, если ответы совпали?
\item  Каково ожидаемое количество правильных решений, если ответы совпали?
\end{enumerate}
 }
\solution{ }

\problem{Автобусы ходят регулярно с интервалом в 10~минут. Вася приходит на остановку в случайный момент времени и ждёт автобуса не больше $a$ минут. Величина $a$ "--- константа из интервала $(0;10)$. Если автобус приходит меньше чем за $a$ минут, то Вася уезжает на нём. Если автобуса нет в течение $a$ минут, то Вася заходит в ближайшую кафешку перекусить и через случайное время возвращается на остановку. На второй раз он ждёт до прихода автобуса.
\begin{enumerate}
\item Какое время Вася в среднем проводит в ожидании автобуса?
\item  Постройте график получившейся функции от $a$.
\end{enumerate}
}
\solution{$f(a)=5+\frac{a(10-a)}{20}$.}

\problem{
Игрок получает 13 карт из колоды в 52 карты. \\
\begin{enumerate}
\item Какова вероятность того, что у него как минимум два туза, если
известно, что у него есть хотя бы один туз? 
\item Какова вероятность того, что у него как минимум два туза, если
известно, что у него есть туз пик? 
\item Объясните, почему эти две вероятности отличаются. 
\end{enumerate}
}
\solution{ }

\problem{ В уездном городе  $N$  проживают  $10^{7}$  человек. Каждый из них
может обладать редким даром ясновидения с вероятностью
$p=10^{-7}$ независимо от других. 
\begin{enumerate}
\item Каково ожидаемое количество ясновидящих? 
\item Известно, что Петя --- ясновидящий. Какова вероятность
найти еще одного ясновидящего в городе $N$?
\end{enumerate}
}
\solution{ 1, почти 1. }

\problem{  
Цвет глаз кодируется несколькими генами. В целом более темный цвет доминирует более светлый. Ген карих глаз доминирует ген синих. Т.е. у носителя пары bb глаза
синие, а у носителя пар BB и Bb --- карие. У диплоидных организмов
(а мы такие :)) одна аллель наследуется от папы, а одна --- от мамы.
В семье у кареглазых родителей два сына --- кареглазый и синеглазый.
Кареглазый женился на синеглазой девушке. Какова вероятность
рождения у них синеглазого ребенка?}
\solution{ }

\problem{
 У тети Маши --- двое детей, один старше другого. Предположим, что вероятности рождения мальчика и девочки равны и не зависят от дня недели, а пол первого и второго ребенка независимы. 
\begin{enumerate}
\item Известно, что хотя бы один ребенок --- мальчик. Какова
вероятность того, что другой ребенок --- девочка?
\item Тетя Маша наугад выбирает одного своего
ребенка и посылает к тете Оле, вернуть учебник по теории
вероятностей. Это оказывается мальчик. Какова вероятность того,
что другой ребенок --- девочка? 
\item Известно, что старший ребенок --- мальчик. Какова вероятность того, что другой ребенок --- девочка? 
\item На вопрос: <<А правда ли тетя Маша, что у вас есть сын, родившийся в пятницу?>>. Она ответила: <<Да>>. Какова вероятность того, что другой ребенок --- девочка?
\end{enumerate}
}
\solution{$ 2/3 $, $1/2$, $ 1/2 $, $ 14/27 $ }

\problem{ В урне 5 белых и 11 черных шаров. Два шара извлекаются по
очереди. Какова вероятность того, что второй шар будет черным?
Какова вероятность того, что первый шар --- белый, если известно,
что второй шар --- черный?}
\solution{ }

\problem{ Monty-hall \\
Вы играете в <<Поле Чудес>> и Вам предлагают <<3 шкатулки>>.
Назовем их a, b и c. В одной из трех шкатулок лежит 1000 рублей.
(Введем соответственно события A, B и C, где A означает <<деньги
лежат в
шкатулке a>>). Вы выбираете наугад одну из трех шкатулок. \\
Ведущий, который знает, где лежат деньги, убирает одну пустую
шкатулку, не выбранную Вами (среди двух не выбранных Вами
обязательно есть пустая, если таковых две, то ведущий убирает
любую наугад). Допустим, Вы выбрали шкатулку b, а ведущий после
этого убрал шкатулку c. \\
Найдите условную вероятность того, что приз лежит в выбранной Вами
шкатулке. Имеет ли Вам смысл изменить Ваш выбор?

\emph{Альтернативный вариант условия-1} \\
После того, как Вы выбрали шкатулку, ведущий открывает наугад одну
из пустых шкатулок (при этом он может открыть Вашу и разочаровать
Вас). Допустим, Вы выбрали шкатулку b, а ведущий после этого
открыл шкатулку c. Найдите условную вероятность того, что приз
лежит в выбранной Вами шкатулке. Имеет ли Вам смысл изменить Ваш
выбор? \\
\emph{Альтернативный вариант условия-2} \\
После того, как Вы выбрали шкатулку, ведущий открывает наугад одну
из оставшихся шкатулок (при этом он может оказаться открытой
шкатулка с деньгами). Допустим, Вы выбрали шкатулку b, а ведущий
после этого открыл шкатулку c и она оказалась пустой. Найдите
условную вероятность того, что приз лежит в выбранной Вами
шкатулке. Имеет ли Вам смысл изменить Ваш выбор? }

\solution{solution 1: \\
Задача эквивалентна следующей: игрок выбирает шкатулку. Затем (она не открывается) игрок выбирает оставить ее или взять обе другие. Очевидно, во втором случае шансы в два раза выше. \\
Solution 2: \\
Игрок не получает информации --- вероятность не меняется. Лучше сменить выбор.  }

\problem{ multi-stage monty hall \\
Suppose there are four doors, one of which is a winner. The host says:
<<You point to one of the doors, and then I will open one of the other non-winners. Then you decide whether to stick with your original pick or switch to one of the remaining doors. Then I will open another (other than the current pick) non-winner. You will then make your final decision by sticking with the door picked on the previous decision or by switching to the only other remaining door>>
Optimal strategy? \\
source: cut-the-knot -- Bhaskara Rao }
\solution{ stick-switch }

\problem{ В школе три девятых класса, <<А>>, <<Б>> и <<В>>, одинаковые по
численности. В <<А>> классе 30\% обожают учителя географии, в
<<Б>> классе --- 40\% и в <<В>> классе --- 70\%. Девятиклассник Петя
обожает учителя географии. Какова вероятность того, что он из
<<Б>>
класса?}
\solution{ }

\problem{ В урне 7 красных, 5 желтых и 11 белых шаров. Два шара
выбирают наугад. Какова вероятность, что это красный и белый, если
известно, что они разного цвета.}
\solution{ }

\problem{ Саша едет на день рождения к Маше и ищет её дом. Её дом находится
южнее по улице. Одна треть встречных прохожих --- местные. Местные всегда
лгут, неместные говорят правду с вероятностью $\frac{3}{4}$.
Изначально Саша оценивает вероятность того, что дом находится
южнее, как $a$. Саша спросил первого встречного прохожего и
получил ответ <<севернее>>. Как Саша изменит свою
субъективную вероятность? }
\solution{ }

\problem{ Самолет упал в горах, в степи или в море. Вероятности,
соответственно, равны $0,5$, $0,3$ и $0,2$. Если он упал в горах,
то при поиске его найдут с вероятностью $0,7$. В степи --- $0,8$, на
море --- $0,2$. Самолет искали в горах, в степи и не нашли. Какова
вероятность того, что он упал в море? }
\solution{ }

\problem{ Русская рулетка. \\
Давайте сыграем в русскую рулетку\ldots Вы привязаны к стулу и не
можете встать. Вот пистолет. Вот его барабан --- в нем шесть гнезд
для патронов, и они все пусты. Смотрите: у меня два патрона. Вы
обратили внимание, что я их вставил в соседние гнезда барабана?
Теперь я ставлю барабан на место и вращаю его. Я подношу пистолет
к вашему виску и нажимаю на спусковой крючок. Щелк! Вы еще живы.
Вам повезло! Сейчас я собираюсь еще раз нажать на крючок. Что вы
предпочитаете: чтобы я снова провернул барабан или чтобы просто
нажал на спусковой крючок? \\
\url{http://forum.eldaniz.ru/index.php?topic=293.60} }
\solution{ }

\problem{ Четыре свидетеля, A, B, C и D, говорят правду независимо
друг от друга с вероятностью $\frac{1}{3}$. A утверждает, что B
отрицает, что C заявил, что D солгал. Какова условная
вероятность того, что D сказал правду? }
\solution{ }

\problem{
Подробности о пожаре (Ах, а правда ли, что тетя Соня забыла
выключить утюг?) передаются по цепочке из четырех человек
(А-B-C-D), каждый из которых говорит следующему имеющуюся у него
информацию с вероятностью $p$, а с вероятностью $1-p$ говорит
совершенно
противоположное. D говорит, что тетя Соня утюг выключила. \\
Как зависит от $p$ условная вероятность того, что тетя Соня
действительно выключила утюг? }
\solution{ }

\problem{
Есть четыре населенных пункта $A$, $B$, $C$ и $D$. Прямая
дорога между каждыми двумя существует с вероятностью $p$. 
\begin{enumerate}
\item Какова вероятность того, что можно добраться из $A$ в $D$?
\item Какова вероятность того, что можно добраться из $A$ в $D$, если
между $B$ и $C$ нет прямой дороги? 
\end{enumerate}
}
\solution{ }



\problem{ В урне лежат 5 пронумерованных от одного до пяти шаров. По
очереди вытаскиваются два шара. Какова вероятность того, что
разница в номерах будет больше двух? Какова вероятность того, что
первым был вытащен шар с номером 2, если разница в номерах была
больше двух?}
\solution{ }

\problem{ A regular $n$-polygon has vertices numbered 0, 1, 2,\ldots, $n-1$ in clockwise. Let the vertex  0 be a starting point. When you roll a dice, you will move the coin clockwise by the number on the dice. Denote the number of the arriving vertice by $X$. Again roll a dice, you will move from the vertex $X$ to the vertex $Y$. 
\begin{enumerate}
\item Are  $X$ and $Y$ independent?
\item Find the value of $n$ such that $X$ and $Y$ are independent  
\end{enumerate}
Source: Kyoto University entrance exam/Science , Problem 6, 1st Round, 1990 }
\solution{ }

\problem{
Будем говорить, что событие $A$ благоприятствует событию $B$, если $\P(B\mid A)>\P(B)$. \\
Известно, что $A$ благоприятствует $B$, $B$ благоприятствует $C$. \\
Верно ли, что $A$ благоприятствует $C$? }
\solution{ не обязательно }

\problem{ Два неравенства
\begin{enumerate}
\item Известно, что $\P(A\mid B)>\P(A)$. Верно ли, что $\P(B\mid A)>\P(B)$?  
\item Известно, что $\P(A\mid B)>\P(B)$. Верно ли, что $\P(B\mid A)>\P(A)$?   
\end{enumerate}
}
\solution{ а) да; б) нет }

\problem{ На Древе познания Добра и Зла растет 6 плодов познания Добра и 5 плодов познания Зла. Адам и Ева съели по 2 плода. Какова вероятность того, что Ева познала Зло, если Адам познал Добро? }
\solution{ }

\problem{ A sniper has 0.8 chance to hit the target if he hit his last shot and 0.7 chance to hit the target if he missed his last shot. It is known he missed on the 1st shot and hit on the 3rd shot. \\
What is the probability he hit the second shot?}
\solution{ $8/11$ }

\problem{
Снайпер попадает в <<яблочко>> с вероятностью 0.8, если в предыдущий раз он попал в <<яблочко>>; и с вероятностью 0.7, если в предыдущий раз он не попал в <<яблочко>> или если это был первый выстрел. Снайпер стрелял по мишени 3 раза.
\begin{enumerate}
\item Какова вероятность попадания в <<яблочко>> при втором выстреле? \\
\item Какова вероятность попадания в <<яблочко>> при втором выстреле, если при первом снайпер попал, а при третьем --- промазал?
\end{enumerate}
}
\solution{
a) $p=0.7\cdot 0.8+ 0.3\cdot 0.7=0.77$ \\
b) $p=\frac{0.7\cdot0.8\cdot0.2}{0.7\cdot 0.8\cdot 0.2 + 0.7\cdot 0.2 \cdot 0.3}=\frac{8}{11}$ }

\problem{
Есть две неправильные монетки. Первая выпадает орлом с вероятностью 0.1, вторая выпадает орлом с вероятностью 0.9. Из этих двух монеток равновероятно выбирают одну и подбрасывают ее 2 раза. 
\begin{enumerate}
\item Верно ли, что результат первого и второго подбрасывания независимы? \\
\item Известно, что выбрали первую монетку. Верно ли, что результат первого и второго подбрасывания независимы? 
\end{enumerate}
}
\solution{нет, да}

\problem{
Вы равновероятно могли получить письмо из Москвы или из Игарки. Все буквы в названии города в обратном адресе кроме одной нечитаемы из-за загрязнения на конверте. Единственная различимая буква --- это буква <<а>>. Какова условная вероятность того, что письмо пришло из Москвы? }
\solution{
Из названия города случайным образом оставляем одну букву. \\ $p=\frac{0.5\frac{1}{6}}{0.5\frac{1}{6}+0.5\frac{2}{6}}=1/3$}

\problem{
Вася кидает дротик в мишень три раза. Его броски независимы друг от друга. Известно, что во второй раз он попал дальше от центра, чем в первый раз. Какова условная вероятность того, что в третий раз он попадет ближе к центру, чем в первый раз? }
\solution{
$\frac{1}{3}$ }

\problem{
В одном мешке лежат только спелые яблоки, в другом --- одинаковое количество спелых и зеленых. Вы случайным образом вытаскиваете яблоко из мешка, оно --- спелое, вы кладете его обратно. Какова вероятность, что следующее яблоко из того же мешка будет зеленым? \\
Какова вероятность, что следующее яблоко из того же мешка будет зеленым, если было $n$ попыток достать яблоко, и каждый раз вытаскивалось и клалось обратно спелое яблоко? }
\solution{
 $\frac{1}{2+2^{n}}$ }

\problem{
Three dice are rolled.  If no two show the same face, what is the probability
that one is an ``ace'' (one spot showing.)? }
\solution{ }

\problem{ Given that a throw with ten dice produced at least one ace, what is
the probability of two or more aces? }
\solution{ }

\problem{
Определение. События $A_{1}$ и $A_{2}$ называются \emph{условно
независимыми} относительно события $B$, если $\P(A_{1}\cap
A_{2}\mid B)=\P(A_{1}\mid B)\cdot \P(A_{2}\mid B)$. 
\begin{enumerate}
\item Приведите пример таких $A_{1}$, $A_{2}$ и $B$, что $A_{1}$ и
$A_{2}$, независимы, но не являются условно независимыми
относительно $B$. \\
\item Приведите пример таких $A_{1}$, $A_{2}$ и $B$, что $A_{1}$ и
$A_{2}$, зависимы, но являются условно независимыми
относительно $B$. 
\end{enumerate}
}
\solution{ }

\problem{
В урне 99 белых и один черный шар. Один шар извлекается из урны наугад. Петя сказал, что шар --- белый. Вася сказал, что шар --- белый. Какова вероятность того, что шар действительно белый, если Петя говорит правду с вероятностью 0.8, а Вася --- с вероятностью 0.9, независимо от Пети? }
\solution{ }

\problem{
У нас ходят два автобуса --- 10-ый и 20-ый. Десятый приходит через десять минут после 20-го; 20-ый --- через 20 минут после десятого. Я прихожу на остановку в случайный момент времени. 
\begin{enumerate}
\item Сколько мне в среднем ждать автобуса? 
\item Сколько мне еще в среднем ждать автобуса, если я уже прождал $m$ минут? 
\end{enumerate}
}
\solution{
 а) $frac{25}{3}$ }

% !Mode:: "TeX:UTF-8"
\section{$\Var$, $\sigma$, conditional $\Var$, conditional $\sigma$}
\subsection{Дискретный случай}

\problem{ \label{do 2-h v summe}
Кубик подбрасывают до тех пор, пока накопленная сумма очков на
гранях не превысит 2. Пусть  $X$  "--- число подбрасываний кубика.
Найдите  $\E(X)$, $\Var(X)$,
$\Var(36X-5)$, $\E(36X-17)$. }
\solution{ \begin{tabular}{|c|c|c|c|} \hline
  % after \\: \hline or \cline{col1-col2} \cline{col3-col4} ...
  $X$ & 1 & 2 & 3 \\ \hline
  $\PP$ & $\frac{24}{36}$ & $\frac{11}{36}$ & $\frac{1}{36}$ \bigstrut \\  \hline
\end{tabular}

$\E(X)=\frac{49}{36}$, $\E(36X-17)=32$. $\Var(X)=\frac{371}{1296}$, $\Var(36X-5)=371$. }


\problem{ \label{iz 5 detalei 2 brakovannih}
Из 5-ти деталей 3 бракованных. Сколько потребуется в среднем
попыток, прежде чем обнаружится первая дефектная деталь? Какова
дисперсия числа попыток?}
\solution{\begin{tabular}{|c|c|c|c|} \hline
  % after \par: \hline or \cline{col1-col2} \cline{col3-col4} ...
  $X$ & 1 & 2 & 3 \\  \hline
 $\PP$ & $\frac{12}{20}$ & $\frac{6}{20}$ & $\frac{2}{20}$ \bigstrut \\
  \hline
\end{tabular} \par
$\E(X)=\frac{3}{2}$, $\Var(X)=\frac{9}{20}$.  }


\problem{
Бросают два правильных игральных кубика. Пусть  $X$  "--- наименьшая
из выпавших граней, а  $Y$  "--- наибольшая.
\begin{enumerate}
\item  Рассчитайте  $\PP(X=3\cap Y=5)$;
\item  Найдите  $\E(X)$,  $\Var(X)$, $\E(3X-2Y)$.
\end{enumerate}
\begin{ist}
Cut the knot.
\end{ist}
 }
\solution{ }

\problem{
Из колоды в 52 карты извлекаются две. Пусть  $X$  "--- количество
тузов. Найдите закон распределения  $X$, $\E(X)$, $\Var(X)$.}
\solution{ }

\problem{  \label{Iska priglasil 3 druzei}
Иська пригласил трёх друзей навестить его. Каждый из них появится
независимо от другого с вероятностью $0{,}9$, $0{,}7$ и $0{,}5$
соответственно. Пусть $N$ "--- количество пришедших гостей.
\begin{enumerate}
\item Рассчитайте вероятности $\PP(N=0)$, $\PP(N=1)$, $\PP(N=2)$ и $\PP(N=3)$;
\item Найдите $\E(N)$ и $\Var(N)$.
\end{enumerate}
 }
\solution{$\PP(N=0)=0{,}1\cdot 0{,}3\cdot 0{,}5$; $\PP(N=3)=0{,}9\cdot 0{,}7\cdot 0{,}5$;
$\PP(N=1)=0{,}9\cdot 0{,}3\cdot 0{,}5+0{,}1\cdot 0{,}7\cdot 0{,}5+0{,}1\cdot0{,}3\cdot 0{,}5$. $\E(N)=0{,}9+0{,}7+0{,}5$; $\Var(N)=0{,}9\cdot 0{,}1+0{,}7\cdot 0{,}3+0{,}5\cdot
0{,}5$.  }


\problem{
В коробке лежат три монеты: их достоинство "--- 1, 5 и 10 копеек соответственно. Они извлекаются в случайном порядке. Пусть $X_{1}$,  $X_{2}$  и  $X_{3}$  "--- достоинства монет в порядке их
появления из коробки.
\begin{enumerate}
\item Верно ли, что  $X_{1}$  и  $X_{3}$ одинаково распределены?
\item  Верно ли, что  $X_{1}$  и  $X_{3}$ независимы?
\item  Найдите $\E(X_{2})$
\item  Найдите дисперсию $\bar{X}_{2} =\frac{X_{1} +X_{2} }{2} $.
\end{enumerate}
 }
\solution{ }

\problem{  \zdt{Easy}

Пусть $X$ "--- сумма очков, выпавших в результате двукратного подбрасывания кубика. Найдите $\E(X)$, $\Var(X)$. }
\solution{ }

\problem{
Охотник, имеющий 4 патрона, стреляет по дичи до первого
попадания или до израсходования всех патронов. Вероятность
попадания при первом выстреле равна 0{,}6, а при каждом последующем
уменьшается на 0{,}1. Найдите
\begin{enumerate}
\item  Закон распределения числа патронов, израсходованных охотником;
\item  Математическое ожидание и дисперсию этой случайной величины.
\end{enumerate}
 }
\solution{ \small

\begin{tabular}{|c|c|c|c|c|}  \hline
  % after \\: \hline or \cline{col1-col2} \cline{col3-col4} ...
  $x_{i}$ & 1 & 2 & 3 & 4 \\  \hline
  $p_i=\PP(X=x_{i})$ & $0{,}6$& $(1-0{,}6)\cdot 0{,}5 = 0{,}2$ & $(1-0{,}6)\cdot(1-0{,}5)\cdot 0{,}4 = 0{,}08$ & $1-p_{1}-p_{2}-p_{3} = 0{,}12$ \\ \hline
\end{tabular}

\normalsize $\E(X)=1{,}7$, $\Var(X)\approx 1{,}08$. }


\subsection{Непрерывный случай}
%Здесь появляется:
%Е когда задана функция плотности

\problem{  \ENGs
A large quantity of pebbles lies scattered uniformly over a
circular field; compare the labour of collecting them on by one
\begin{enumerate}
\item At the center $O$ of the field;
\item At a point $A$ on the circumference.
\end{enumerate}\RUSs
\begin{ist}
Grimmett, экзамен 1858 года в St John's College, Кембридж.
\end{ist}
 }
\solution{ }

\subsection{? Способ расчёта ожидания и дисперсии через условную}

\problem{
Число $x$ выбирается равномерно на отрезке $[0; 1]$. Затем случайно выбираются числа из отрезка $[0; 1]$ до тех пор, пока не появится число больше $x$.
\begin{enumerate}
\item  Сколько в среднем потребуется попыток?
\item  Сколько в среднем потребуется попыток, если $x$ выбирается равномерно на отрезке $[0;r]$, $r<1$?
\item  Сколько в среднем потребуется попыток, $x$ не выбирается равномерно, а имеет функцию плотности $p(t)=2(1-t)$ для $t\in[0;1]$?
\end{enumerate}
 }
\solution{ }




% !Mode:: "TeX:UTF-8"
\section{Связь между случайными величинами, Cov, Corr}
\subsection{Дискретный случай}
%(частная/маржинальная/... ф распределения)


\problem{ \label{vtoroie podkidivanie}
 У Васи есть $n$ монеток,
каждая из которых выпадает орлом с вероятностью $p$. В первом
раунде Вася подкидывает все монетки, во втором раунде Вася
подкидывает только те монетки, которые выпали орлом в первом
раунде. Пусть $R_{i}$ "--- количество монеток,
подкидывавшихся и выпавших орлом во $i$-м раунде.
\begin{enumerate}
\item Каков закон распределения величины $R_{2}$?
\item  Найдите $\Corr(R_{1},R_{2})$.
\item  Как ведет себя корреляция при $p\to 0$ и $p\to 1$? Почему?
\end{enumerate}
 }
\solution{ Закон распределения по сути эксперимента: $\bin(n; p^{2}$); $\Corr(R_{1},R_{2}) =\sqrt{\frac{p}{1+p}}$. При $p\to 0$ корреляция равна 0; при  $p\to 1$ составляет 0{,}5. Почему "--- чез = чёрт его знает.
}



\problem{<<Корреляция --- это мера линейной связи>>

Найдите все случайные величины $X$ такие, что $corr(X,X^2)=1$.


Источник: Алексей Суздальцев}

\solution{Случайные величины, принимающие два значения $x_1<x_2$, такие, что $|x_1|<|x_2|$.}



\subsection{Непрерывный случай (Cov, Corr)}


\problem{Пусть $ X $ равномерно на $ [0;1] $. Если известно, что $ X=x $, то случайная величина $ Y $ равномерна на $ [x;x+1] $. Найдите $ P(X+Y<1) $ и $ f_{X|Y}(x|y) $. Как распределено $ Y-X $?}
\solution{$ P(X+Y<1)=1/4 $, $ f_{X|Y}(x|y)=1/f(y) $ при $ y\in [x;x+1], x\in[0;1] $. Равномерно.}


\subsection{Общие свойства Cov и Var}

\problem{Верно ли, что $X$ и $Y$  независимы, если известно, что
\begin{enumerate}
\item Величина $X$ и любая функция $g(Y)$ некоррелированы?
\item Любая функция $f(X)$ и любая функция $g(Y)$ некоррелированы?
\end{enumerate}
}
\solution{В первом случае "--- нет, например, $X \sim \mN(0;1)$ и $Y=X^2$. Во втором "--- да: возьмём индикаторы и получим стандартное определение независимости}


\subsection{Преобразование случайных величин (преобразования)}


\problem{ \label{simmetria razbitia otrezka}
На отрезке равномерно и независимо выбираются две точки. Верно ли,
что длины получающихся трёх отрезков распределены одинаково? }
\solution{Да. Возьмём окружность. Наугад отметим три точки. Одну будем
трактовать как разрезающую окружность на отрезок. Две других "---  как
разрезающие отрезок на три части. }
\cat{uniform} \cat{circle_trick}


\problem{ \zdt{Птички на проводе-1}

На провод, отрезок $[0; 1]$, равномерно и независимо друг от друга
садятся $n$ птичек. Пусть $Y_{1}$,\ldots, $Y_{n+1}$ "--- расстояния
от левого столба до первой птички, от первой птички от второй и т.\,д.
\begin{enumerate}
\item Найдите функцию плотности $Y_{1}$;
\item Верно ли, что все $Y_{i}$ одинаково распределены?
\item  Верно ли, что все $Y_{i}$ независимы?
\item Найдите $\Cov(Y_{i},Y_{j})$ (вроде бы ковариации равны?);
\item  Как распределена величина $n\cdot Y_{1}$ при больших $n$? Почему?
\end{enumerate}
 }
\solution{ Пусть $X_{i}$ "--- координата $i$-ой птички. $\PP(Y_{1}\le t)=1-\PP(\min\{X_{i}\}>t)=1-(1-t)^{n}$. Далее, $\lim \PP(nY_{1}\le t)=\lim 1-\left(1-\frac{t}{n}\right)^{n}=1-e^{-t}$. Распределение экспоненциальное с параметром $\lambda=1$. }
\cat{uniform} \cat{circle_trick} \cat{exponential}


\problem{ \zdt{Птички на проводе-2}

На провод, отрезок $[0;1]$, равномерно и независимо друг от друга
садятся $n$ птичек. Мы берем ведро жёлтой краски и для каждой
птички красим участок провода от неё до ближайшей к ней соседки.

Какая часть провода будет окрашена при больших $n$? }
\solution{ Пусть $n$ велико, тогда $Y_{i}$ можно считать независимыми и
$nY_{i}$ "--- экспоненциально распределёнными. Не красятся только <<большие>> интервалы, т.\,е. интервалы, чья
длина больше, чем каждого из двух соседних интервалов. <<Больших>>
интервалов примерно треть. Находим $\E(B)=\E(\max\{Y_{1},Y_{2},Y_{3}\})$. $\delta=1-\E(B)\frac{n}{3}=\frac{7}{18}$. }
\begin{ist}
Marcin Kuczma.
\end{ist}

\problem{Длительность разговора клиента в минутах $X$ "--- экспоненциальная случайная величина со средним две минуты. Стоимость разговора, $Y$, составляет $5$ рублей за весь разговор, если разговор короче двух минут, и $2{,}5$ рубля за минуту, если разговор длиннее двух минут. Неполные минуты оплачиваются пропорционально, например, за 3{,}5 минуты нужно заплатить $2{,}5\cdot 3{,}5$ рублей. Найдите $\E(Y)$, $\Var(Y)$, постройте функцию распределения $Y$. }
\solution{}

\problem{Вася пришёл на остановку. Ему нужен 42-й или 21-й автобус. Время до прихода 42-го равномерно на отрезке $[0;10]$ минут, время до прихода 21-го равномерно на отрезке $[0;20]$ минут. Время прихода 42-го и время прихода 21-го "--- независимые величины. Обозначим за $Y$  время, которое Вася проведёт в ожидании на остановке. Найдите функцию плотности $Y$, $\E(Y)$, $\Var(Y)$. }
\solution{}


\subsection{Прочее про несколько случайных величин}

\problem{\zdt{Парадокс голосования}
\par
Пусть $X$, $Y$, $Z$ "--- дискретные
случайные величины, их значения попарно различны с вероятностью 1.
Докажите, что $\min\left\{\PP(X>Y),\PP(Y>Z),\PP(Z>X)\right\}\le \frac{2}{3}$.
Приведите пример, при котором эта граница точно достигается. }
\solution{}


% !Mode:: "TeX:UTF-8"
\section{Приёмы решения}
\subsection{Разложение в сумму}

\problem{ \label{sudba-don-juan-2}\zdt{Судьба Дон-Жуана-2} (см. тж. с.~\pageref{sudba-don-juan-1})

У Васи $n$  знакомых девушек, их всех зовут по-разному. Он пишет
им $n$  писем, но по рассеянности раскладывает их в конверты
наугад. Случайная величина $X$ обозначает количество девушек, получивших письма,
написанные лично для них. Найдите $\E(X)$, $\Var(X)$. }
\solution{$\E(X)=1$, $\Var(X)=1$. }



\problem{ \label{cube-cut-2}(см. тж. с.~\pageref{cube-cut-1}) \ENGs

A wooden cube that measures 3 cm along each edge is painted red. The painted cube is then cut into 27 pieces of 1-cm cubes. If I tossed all the small cubes in the air, so that they landed randomly on the table, how many cubes should I expect to land with a painted face up? \RUSs}
\solution{ $9$.}


\problem{Вокруг новогодней ёлки танцуют хороводом 27 детей. Мы считаем, что ребенок высокий, если он выше обоих своих соседей. Сколько высоких детей в среднем танцует вокруг елки? Вероятность совпадания роста будем считать равной нулю.}
\solution{Для трёх детей вероятность того, что тот, что посередине "--- самый высокий, равна $ \frac{1}{3} $, значит математическое ожидание равно $ \frac{27}{3}=9$. }

\problem{Маша собирает свою дамскую сумочку. Есть $n$ различных предметов, которые она туда может положить. Каждый предмет она кладёт независимо от других с вероятностью $p$.
\begin{enumerate}
\item Пусть $X$ "--- количество положенных предметов. Найдите $\E(X)$ и $\Var(X)$.
\item При каком $p$ вероятность положить в сумку любой заданный набор вещей не будет зависеть от конкретного набора?
\end{enumerate}}
\solution{ Биномиальное распределение, $\E(X)=np$, $\Var(X)=np(1-p) $. При $p=0{,}5$ все подмножества будут равновероятны.}


\problem{ Игральный кубик подбрасывается 100 раз. Найдите ожидаемую
сумму очков, дисперсию суммы, стандартное отклонение суммы.}
\solution{ }


\problem{ Гипергеометрическое распределение \\
В задачнике $N$ задач. Из них $a$ --- Вася умеет решать, а остальные не умеет. На экзамене предлагается равновероятно выбираемые $n$ задач. Величина $X$ --- число решенных Васей задач на экзамене. 

Найдите $\E(X)$ и $\Var(X)$ }
\solution{ 
$X=X_{1}+...+X_{n}$, $\E(X)=n\frac{a}{N}$ \\
$\Var(X_{i})=\frac{a(N-a)}{N^{2}}$ \\
$\Cov(X_{i},X_{1}+...+X_{N})=0$ \\
$\Cov(X_{i},X_{j})=-\frac{\Var(X_{i}}{N-1}$ \\
$\Var(X)=n\Var(X_{i})\frac{N-n}{N-1}$}

\problem{ 
Кубик подбрасывается $n$ раз. Величина $X_{1}$ ---
число выпадений 1, а $X_{6}$ --- число выпадений 6. Найдите $\Corr(X_{1},X_{6})$ \\
Подсказка: $\Cov(X_{1},X_{1}+...+X_{6})$ вам в помощь... }
\solution{$\Cov(X_{1},X_{1}+...+X_{6})=0$, т.к. $X_{1}+...+X_{6}=const$ 
$\Corr(X_{1},X_{6})=-\frac{1}{5}$ }

\problem{ По 10 коробкам наугад раскладывают 7 карандашей. Каково
среднее количество пустых коробок? }
\solution{$10\cdot (1-\frac{9}{10}^{7})$ }

\problem{ $[$Mosteller$]$ Среднее число совпадений \\
Из хорошо перетасованной колоды на стол последовательно
выкладываются карты лицевой стороной наверх, после чего
Аналогичным образом выкладывается вторая колода, так что каждая
карта первой колоды лежит под картой из второй колоды. Каково
среднее  число совпадений   нижней  и верхней  карт?}
\solution{ $1$ }


\problem{ Grimmett, 3.3.3. \\
В группе 20 человек. Каждый из них подбрасывает по кубику. Найдите
ожидаемый выигрыш и дисперсию выигрыша группы, если: 
\begin{enumerate}
\item за каждую пару игроков, выкинувших одинаковое количество очков,
группа получает один тугрик \\
\item за каждую пару игроков, выкинувших одинаковое количество очков,
группа получает эту сумму в тугриках  
\end{enumerate}
}
\solution{ }

\problem{ Coupon's collector problem \\
Внутри упаковки шоколадки <<Веселые животные>> находится наклейка
с изображением одного из 30 животных. Предположим, что все
наклейки равновероятны. Большой приз получит каждый, кто соберет
наклейки всех животных. Какое количество шоколадок в среднем нужно
купить, чтобы выиграть большой приз? }
\solution{ }

\problem{ $[$Mosteller$]$ \\
В $n$ урн случайным образом бросают один за одним $k$ шаров.
Найдите математическое ожидание числа пустых урн. }
\solution{ }


\problem{
У Маши 30 разных пар туфель. И она говорит, что мало! Пес
Шарик утащил 17 туфель без разбору на левые и правые. Сколько
полных пар в среднем осталось у Маши? Сколько полных пар в среднем
досталось Шарику? }
\solution{ }

\problem{
Из колоды в 52 карты извлекается 5 карт. Сколько в среднем
извлекается мастей? Достоинств? Тузов? }
\solution{ Масть: $4\cdot (1-\frac{C_{39}^{5}}{C_{52}^{5}})$ 

Достоинство: $13\cdot (1-\frac{C_{48}^{5}}{C_{52}^{5}})$ 

Туз: $4\cdot \frac{5}{52}$ }


\problem{
На карточках написаны числа от 1 до $n$. В игре участвуют
$n$ человек. В первом туре каждый получает случайным образом по
одной карточке. Во втором туре карточки выдаются заново. Призы
раздаются по следующему принципу: Человек не получает приз, только
если найдется кто-то другой, кто получил большие числа в каждом
туре.
Каково среднее количество человек, получивших приз? \\
Взято с www.zaba.ru, какая-то олимпиада. }
\solution{ }

\problem{
А.А. Мамонтов сидит в 424 аудитории. Эконометрику
собираются сдавать несколько человек. На поиски пустых аудиторий
послано 3 студента-разведчика. На втором этаже 9 учебных
аудиторий, 5 из них заняты. Каждый из 3 студентов-разведчиков
независимо друг от друга заглядывает в 3 аудитории. Если студент
обнаруживает пустую аудиторию, то он сообщает ее номер А.А.
Мамонтову. Каково среднее
количество обнаруженных пустых аудиторий? }
\solution{ }


\problem{
Вася пишет друг за другом наугад 100 букв из латинского алфавита. 
\begin{enumerate}
\item Каково ожидаемое количество букв, встречающихся в написанном <<слове>> ровно один раз?
\item Как изменилась бы искомая величина, $a_{k,n}$, если бы в алфавите было $k$ букв, а Вася писал бы <<слово>> из $n$ букв?
\item Найдите $\lim_{n\to\infty} a_{k,n}$, $\lim_{k\to\infty} a_{k,n}$    
\end{enumerate}
}
\solution{ }

\problem{
За круглым столом сидят в случайном порядке $n$ супружеских пар, всего --- $2n$ человек. Величина $X$ --- число пар, где супруги оказались напротив друг друга. 

Найдите $\E(X)$ и $\Var(X)$ }
\solution{ }

\problem{
Suppose there were $m$ married couples, but that $d$ of these $2m$ people have died. Regard the $d$ deaths as striking the $2m$ people at random. Let $X$ be the number of surviving couples. Find: $\E(X)$ and $\Var(X)$ }
\solution{ }



\problem{
Над озером взлетело 20 уток. Каждый из 10 охотников
стреляет в утку по своему выбору. 
\begin{enumerate}
\item Каково ожидаемое количество убитых уток, если охотники стреляют без промаха? 
\item Как изменится ответ, если вероятность попадания равна 0,7?
\item Каким будет ожидаемое количество охотников, попавших в цель?
\end{enumerate}
}
\solution{ }


\problem{
В каждой из двух урн находится по 50 белых и 50 черных шаров. Вася одновременно вытаскивает по шару из каждой урны и выбрасывает их.
Величина $X$ --- количество раз, когда из обеих урн были одновременно вытащены белые шары. 

Найдите $\E(X)$, $\Var(X)$ }
\solution{ }

\problem{
На карточках написаны числа от 1 до $n$. Вася достает их одну за другой наугад. Если номер карточки является соседним с номером предыдущей карточки, то Вася получает 1 рубль. Величина $X$ --- Васин выигрыш. \\
Найдите $\E(X)$, $\Var(X)$ }
\solution{ }

\problem{
Вася называет наугад 50 чисел от 1 до 100, допускаются повторения, а Петя называет наугад 50 чисел от 1 до 100 без повторов. \\
Величины $X$ и $Y$ это суммы этих чисел. 
\begin{enumerate}
\item Сравните $\E(X)$ и $\E(Y)$ 
\item Сравните $\Var(X)$ и $\Var(Y)$  
\end{enumerate}
}
\solution{ $\E(X)=\E(Y)$, $\Var(X)>\Var(Y)$}



\problem{
Кубик подбрасывается до тех пор, пока каждая грань не
выпадет по разу. Найдите математическое ожидание и дисперсию числа
подбрасываний. }
\solution{ }

\problem{
Правильная монетка подбрасывается  $n$  раз. Серия --- это
последовательность подбрасываний из одинаковых результатов. К
примеру, в последовательности ОООРРО три серии. 
\begin{enumerate}
\item Каково ожидаемое количество серий? \\
\item Дисперсия числа серий? \\
\item А если монетка неправильная и выпадает гербом с вероятностью  $p$? 
\end{enumerate}
}
\solution{ }


\problem{ В здании 10 этажей, на каждом этаже 30 окон. Вечером в каждом окне независимо от других свет включается с вероятностью $p$. 
\begin{enumerate}
\item Чему равно ожидаемое количество <<ноликов>> на фасаде здания? 
\item Чему равно ожидаемое количество <<крестиков>> на фасаде здания?   
\item При каких $p$ эти количества максимальны? Минимальны?  
\end{enumerate}
Примечание: два разных нолика могут иметь общие точки \\
Вставить рисунок нолика и рисунок крестика, пример подсчета }
\solution{ }

\problem{ 
Если смотреть на корпус Ж здания Вышки с Дурасовского переулка, то видно 40 окон. (??? Видно 7 этажей, первый не видно, и 8 окон на каждом этаже, уточнить по месту). Допустим, что каждое из них освещено вечером независимо от других с вероятностью одна вторая. Назовем <<уголком>> комбинацию из 4-х окон, расположенных квадратом, в которой освещено ровно три окна (не важно, какие). Величина $X$ --- число <<уголков>>, возможно пересекающихся, на всем корпусе Ж. \\
Найдите  $\E(X)$ и $\Var(X)$ \\
Примечание - для наглядности: \\
\begin{tabular}{|c|c|}
  \hline
  X & X\\
  \hline
    & X \\
  \hline
\end{tabular},
\begin{tabular}{|c|c|}
  \hline
  X & \\
  \hline
  X & X \\
  \hline
\end{tabular},
\begin{tabular}{|c|c|}
  \hline
   & X\\
  \hline
  X & X \\
  \hline
\end{tabular},
\begin{tabular}{|c|c|}
  \hline
  X & X\\
  \hline
  X &  \\
  \hline
\end{tabular} - это <<уголки>>. \\
\begin{tabular}{|c|c|c|}
  \hline
  X & X & X\\
  \hline
    & X & \\
  \hline
  X & X & \\
  \hline
\end{tabular} - в этой конфигурации три <<уголка>>;
\begin{tabular}{|c|c|c|}
  \hline
  X &  & X\\
  \hline
    & X & \\
  \hline
  X &  & X\\
  \hline
\end{tabular} - а здесь - ни одного <<уголка>>. 
}
\solution{ }

\problem{ В урне  $n$  шаров пронумерованных 1,2,... $n$. Наугад
вытаскивают $k$. Найдите ожидание и дисперсию суммы номеров. }
\solution{ }

\problem{ У Пети стопка из $n$ номеров газеты <<Вышка>> лежащих в случайном
порядке. Петя сортирует газеты следующим образом. Он
последовательно просматривает стопку сверху вниз. Если
просматриваемый выпуск более свежий, чем лежащий сверху стопки, то
Петя перекладывает более свежий выпуск наверх стопки и
начинает просматривать стопку заново. \par
Сколько <<переносов>> более свежих номеров наверх в среднем будет
сделано до того момента, когда наверху окажется первый выпуск
газеты? \par
\url{http://www.artofproblemsolving.com/Forum/viewtopic.php?t=124903 } }
\solution{Solution 1: \par
$p_{2}=\frac{1}{2}$ \par
С вероятностью $\frac{n-1}{n}$ сверху стопки лежит номер, меньший
$n$, в этом случае можно считать, что $n$-ый номер вообще
отсутствует в стопке. \par
С вероятностью $\frac{1}{n}$ сверху стопки лежит $n$-ый номер,
тогда обязательно происходит одно перекладывание, после которого
мысленно выкинув $n$-ый номер можно считать, что имеется случайно
упорядоченная стопка из $(n-1)$ выпуска.\par
$p_{n}=\frac{n-1}{n}p_{n-1}+\frac{1}{n}(p_{n-1}+1)$ \par
Итого: $p_{n}=\sum_{i=2}^{n}\frac{1}{i}\approx n\ln(n)$ \par
Solution 2: \par
Пусть $q_{i}$- вероятность того, что число $i$ <<уберут>> с верха стопки.\par
$q_{1}=0$ \par
Вероятность того, что число $i$ <<уберут>> с верха стопки равна
вероятности того, что среди чисел $1$, $2$,... $i$ число $i$ будет
первым, т.е. $\frac{1}{i}$. \par
$\E(X)=\E(X_{2})+...+\E(X_{n})=\sum_{i=2}^{n}\frac{1}{i}\approx
n\ln(n)$  }




\subsection{Первый шаг}


\problem{Илье Муромцу предстоит дорога к камню. И от камня начинаются ещё три дороги. Каждая из тех дорог снова оканчивается камнем. И от каждого камня начинаются ещё три дороги. И каждые те три дороги оканчиваются камнем\ldots И так далее до бесконечности. На каждой дороге можно встретить живущего на ней трёхголового Змея Горыныча с вероятностью (хм, Вы не поверите!) одна третья. Какова вероятность того, что у Ильи Муромца существует возможность пройти свой бесконечный жизненный путь, так ни разу и не встретив Змея Горыныча?}
\solution{$p=\frac{2}{3}(1-(1-p)^{3})$, нам подходит решение $ p=\frac{3-\sqrt{3}}{2} $. }


\problem{У Пети "--- монетка, выпадающая орлом с вероятностью $ p\in (0;1) $. У Васи "--- с вероятностью $ q\in (0;1) $. Они одновременно подбрасывают свои монетки до тех пор, пока у них не окажется набранным одинаковое количество орлов. В частности, они останавливаются после первого подбрасывания, если оно дало одинаковые результаты. Сколько в среднем раз им придётся подбросить монетку?}
\solution{}

\todo[inline]{А если 5 мастей подряд вообще нет?}
\problem{Сколько в среднем нужно взять из колоды в 52 карты, чтобы насобирать подряд 5 карт одной масти?

\begin{hint}
Ответ имеет вид произведения дробей очень простого вида.
\end{hint}
}
\solution{Если у нас $m=13$ достоинств и $n=4$ масти, то ответ имеет вид: $mn\prod\limits_{i=1}^{}\frac{in}{in+1}\approx 45{,}3$.}

\problem{Вася прыгает на один метр вперёд с вероятностью $p$ и на два метра вперёд с вероятностью $1-p$. Как только он пересечёт дистанцию в 100~метров, он останавливается. Получается, что он может остановиться на отметке либо в 100~метров, либо в 101~метр. Какова вероятность того, что он остановится ровно на отметке в 100~метров?}
% копия в задачах на остановку мартингала
\solution{ Обозначим за $P_n$ вероятность остановиться ровно на $n$ метрах. Мы ищем $P_{100}$.

\textit{Решение 1.} По методу первого шага:  $P_n=pP_{n-1}+(1-p)P_{n-2}$.

\textit{Решение 2.} Попасть ровно в $n$ можно двумя способами: перелетев $n-1$ или попав в $n-1$ и сделав шаг в один метр. Значит $P_n=(1-P_{n-1})+pP_{n-1}$.

\textit{Решение 3.} Обозначим Васину координату в момент времени $t$ как $X_t$. Можно найти $a$ так, чтобы процесс $Y_t=a^{X_t}$ был мартингалом. Момент остановки $T=\min\{t \min X_t\geq n\}$. Мартингал $Y_{t\wedge T}$ ограничен, теорема Дуба применима. $\E(Y_T)=\E(Y_0)=1$. Получаем уравнение $P_n a^{n}+(1-P_n) a^{n+1}=1$.}

% untyp
\problem{
Испытания по схеме Бернулли проводятся до первого успеха, вероятность успеха в
отдельном испытании равна $p$ \par
а) Чему равно ожидаемое количество испытаний?   \par
б) Чему равно ожидаемое количество неудач? \par
в) Чему равна дисперсия количества неудач? }
\solution{ $\frac{1}{p}$, $\frac{q}{p}$ \par
в) $E(X^{2})=p\cdot 1+q\cdot E((X+1)^{2})$, $Var(X)=\frac{q}{p^2}$ }

% untyp
\problem{ Отрицательное биномиальное \par
Испытания по схеме Бернулли проводятся до $r$-го успеха, вероятность успеха в
отдельном испытании равна $p$ \par
а) Чему равно ожидаемое количество неудач? \par
б) Чему равна дисперсия количества неудач? }
\solution{ (устно, при сделанной предыдущей задаче) $\frac{rq}{p}$, $Var(X)=\frac{rq}{p^2}$ }

% untyp
\problem{
Саша и Маша по очереди подбрасывают кубик. Посуду будет
мыть тот, кто первым выбросит шестерку. Маша бросает первой.
Какова вероятность того, что Маша будет мыть посуду? }
\solution{ }

% untyp
\problem{
Саша и Маша решили, что будут рожать нового ребенка, до тех
пор, пока в их семье не будут дети обоих полов. Каково ожидаемое
количество детей? }
\solution{ }

% untyp
\problem{
Четыре человека играют в игру <<белая ворона платит>>. Они
одновременно подкидывают монетки. Если три монетки выпали одной
стороной, а одна - по-другому, то <<белая ворона>> оплачивает всей
четверке ужин в ресторане. Если <<белая ворона>> не определилась,
то монетки подбрасывают снова. Сколько в среднем нужно
подбрасывания для определения <<белой вороны>>? }
\solution{ }

% untyp
\problem{
Саша и Маша каждую неделю ходят в кино. Саша доволен
фильмом с
вероятностью 1/4, Маша - с вероятностью 1/3. \par
a) Сколько недель в среднем пройдет до тех пор, пока кто-то не
будет доволен? \par
b) Какова вероятность того, что первым будет доволен Саша? \par
c) Сколько недель в среднем пройдет до тех пор, пока каждый не
будет доволен хотя бы одним просмотренным фильмом? }
\solution{ }

% untyp
\problem{
По ответу студента на вопрос преподаватель может сделать
один из трех выводов: ставить зачет, ставить незачет, задать еще
один вопрос. Допустим, что знания студента и характер
преподавателя таковы, что при ответе на отдельный вопрос зачет
получается с вероятностью $p_{1}=3/8$, незачет --- с вероятностью
$p_{2}=1/8$. Преподаватель задает вопросы до тех пор, пока не
определится
оценка. \par
а) Сколько вопросов в среднем будет задано? \par
б) Какова вероятность получения зачета? }
\solution{ }

% untyp
\problem{
Вы играете в следующую игру. Кубик подкидывается неограниченное число раз. Если на кубике выпадает 1, 2 или 3, то соответствующее количество монет добавляется на кон. Если выпадает 4 или 5, то игра оканчивается и Вы получаете сумму, лежащую на кону. Если выпадает 6, то игра оканчивается, а Вы не получаете ничего. \par
а) Чему равен ожидаемый выигрыш в эту игру? \par
б) Изменим условие: если выпадает 5, то набранная сумма сгорает, а игра начинается заново. Чему будет равен ожидаемый выигрыш? }
\solution{ 
a) $V(x)=\frac{1}{6}(V(x+1)+V(x+2)+V(x+3)+2x+0)$ \par
Ищем линейную $V(x)$, получаем $V(x)=\frac{2}{3}x+\frac{4}{3}$ \par
б) $V(x)=\frac{1}{6}(V(x+1)+V(x+2)+V(x+3)+x+V(0)+0)$ }

% untyp
\problem{ Вася подкидывает кубик. Если выпадает единица, или Вася говорит
<<стоп>>, то игра оканчивается, если нет, то начинается заново.
Васин выигрыш - последнее выпавшее число. Как выглядит оптимальная
стратегия? Как выглядит оптимальная стратегия, если за каждое
подбрасывание Вася платит 35 копеек?\cite{stirzaker:otep}}
\solution{ }

% untyp
\problem{
Саша и Маша подкидывают монетку бесконечное количество раз. Если сначала появится
РОРО, то выигрывает Саша, если сначала появится ОРОО, то - Маша. \par
а) У кого какие шансы выиграть? \par
b) Сколько в среднем времени ждать до появления РОРО? До ОРОО?
с) Сколько в среднем времени ждать до определения победителя? }
\solution{ }

% untyp
\problem{ \label{mishka ishet sir}
Есть три комнаты. В первой из них лежит сыр. Если мышка
попадает в первую комнату, то она находит сыр через одну минуту.
Если мышка попадает во вторую комнату, то она ищет сыр две минуты
и покидает комнату. Если мышка попадает в третью комнату, то она
ищет сыр три минуты и покидает комнату. Покинув комнату, мышка
выходит в коридор и выбирает новую комнату наугад (т.е. может
зайти в одну и ту же). Сейчас мышка в коридоре. Сколько времени ей
в среднем потребуется, чтобы найти сыр? }
\solution{ $m=\frac{1}{3}+\frac{1}{3}(2+m)+\frac{1}{3}(3+m)$, $m=6$ }

% untyp
\problem{
Иська и Еська по очереди подбрасывают два кубика. Иська
бросает первым. Иська выигрывает, если при своем броске получит 6
очков в сумме на двух кубиках. Еська выигрывает, если при своем
броске получит 7 очков в сумме на двух кубиков. Кубики
подбрасываются до
тех пор, пока не определится победитель. \par
а) Верно ли, что события $A=\{$на двух кубиках в сумме выпало
больше 5 очков$\}$ и $B=\{$на одном из кубиков выпала 1$\}$ являются независимыми? \par
б) Какова вероятность того, что Еська выиграет? }
\solution{ }

% untyp
\problem{
Players A and B play a (fair) dice game. <<A>> deposits one coin and
they take turns rolling a single dice, <<B>> rolling first. \par
If <<B>> rolls an even number, he collects a coin from the pot. If
he rolls an odd number, he put a coin (coins with same values
always). If <<A>> (plays and) rolls an even number, he collects a
coin but if he rolls an odd number, he does NOT add a coin. The
game continues until the pot is exhausted. \par
Question: what is the probability that <<A>> wins this game (that
is, exhaust the pot) ? \par
t=138358}
\solution{ }


\problem{
Вам предложена следующая игра. Изначально на кону 0 рублей. Раз за разом подбрасывается правильная монетка. Если она выпадает орлом, то казино добавляет на кон 100 рублей. Если монетка выпадает решкой, то все деньги, лежащие на кону, казино забирает себе, а Вы получаете красную карточку. Игра прекращается либо когда Вы получаете третью красную карточку, либо в любой момент времени до этого по Вашему выбору. Если Вы решили остановить игру до получения трех красных карточек, то Ваш выигрыш равен сумме на кону. При получении третьей красной карточки игра заканчивается и Вы не получаете ничего. 
\begin{enumerate}
\item Как выглядит оптимальная стратегия в этой игре? 
\item Чему при этом будет равен средний выигрыш?  
\end{enumerate}
}
\solution{ }

\problem{ Китайский ресторан \par
Каждый момент времени в китайский ресторан приходит новый посетитель.
Если сейчас в ресторане сидит $n$ человек, а за конкретным столиком сидит $b$ человек, то вероятность того, что новый посетитель присоединится к этому столику равна $\frac{b}{n+\theta}$. С вероятностью $\frac{\theta}{n+\theta}$ посетитель сядет за отдельный столик. \par
Каково ожидаемое число занятых столиков к моменту времени $n$? }
\solution{ }


\problem{ Вася бьет мячом по воротам 100 раз. В первый раз вероятность
попасть равна $frac{1}{2}$, в каждый последующий раз вероятность
попасть увеличивается --- Вася становится метче; при этом разные
удары независимы. Какова вероятность того, что Вася попадет в
ворота четное число раз? }
\solution{$\frac{1}{2}$ }

\problem{ Вася нажимает на пульте телевизора кнопку <<On-Off>> 100 раз
подряд. Пульт старый, поэтому в первый раз кнопка срабатывает с
вероятностью $\frac{1}{2}$, затем вероятность срабатывания падает.
Какова вероятность того, что после всех нажатий телевизор будет
включен, если сейчас он выключен? }
\solution{$\frac{1}{2}$ }


\problem{ Вы в тире, и у Вас 100 патронов. С вероятностью $0.01$ Вы попадает в глаз Усамы Бен Ладена, за что получаете 20 дополнительных патронов, с вероятностью $0.05$ Вы попадаете в нос Усамы Бен Ладена, за что получаете 5 дополнительных патронов. Вы стреляете до тех пор, пока патроны не кончатся. Сколько в среднем Вы сделаете выстрелов? 

\url{www.wilmott.com-forum-brainteasers} }
\solution{ Во первых, заметим, что ожидаемое количество выстрелов, если у Вас осталось $n$ патронов имеет вид $E_{n}=k\cdot n$. \par
Во-вторых, получим уравнение на $E_{n}$: \par
$E_{n}=1+E_{n-1}+kE(X)$, где $\E(X)$ - ожидаемый выигрыш патронов от одного выстрела. \par
Находим $k$: $k=\frac{1}{1-\E(X)}$ \par
Ответ задачи: $\frac{100}{1-0.45}$ \par
Solution2: Интуитивно: $100+100\cdot 20\cdot 0.01+100(\cdot 20\cdot)^{2}+...$  }



\problem{
В вершинах треугольника три ежика. С вероятностью $p$ каждый ежик
независимо от других двигается по часовой стрелке, с вероятностью
$(1-p)$ он двигается против часовой стрелки. Сколько в среднем
пройдет времени прежде,
чем они встретятся в одной вершине? \par
При каком $p$ ожидаемое время встречи минимально? }
\solution{ 
У системы 4 состояния (1-1-1, 1-2-0, 2-1-0, 3-0-0). \par
Пишем три уравнения на ожидаемые времена. \par
Решая находим $E(T)=\frac{3}{p(1-p)}$ \par
Минимум при $p=0.5$ \par
Ответ слишком красивый... красивое решение???? }


\problem{Монетка выпадает орлом с вероятностью $p$. Монетку подбрасывают до тех пор, пока впервые не выпадет орёл. Какова вероятность того, что будет сделано чётное число подбрасываний?}
\solution{$\P(A)=(1-p)/(2-p)$}



\subsection{Аллюзии на принцип Белмана}


\problem{
Начинающая певица дает концерты каждый день. Каждый ее концерт приносит продюсеру 0.75 тысяч евро. После каждого концерта певица может впасть в депрессию с вероятностью 0.5. Самостоятельно выйти из депрессии певица не может. В депрессии она не в состоянии проводить концерты. Помочь ей могут только цветы от продюсера. Если подарить цветы на сумму $0\le x\le 1$ тысяч евро, то она выйдет из депрессии с вероятностью $\sqrt{x}$. Дисконт фактор равен $0.8$. \\
Какова оптимальная стратегия продюсера? }
\solution{ 
Рассмотрим совершенно конкурентный невольничий рынок начинающих певиц. Певицы в хорошем настроении продаются по $V_{1}$, в депрессии - по $V_{2}$. \\
$V_{1}=0.75+\delta(0.5V_{1}+0.5V_{2})$ \\
$V_{2}=max_{x}{-x+\delta(\sqrt{x}V_{1}+(1-\sqrt{x})V_{2})}$ }

\problem{ Будучи незамужней Маша испытывает отрицательную полезность $-c$ каждый день. Каждый день она знакомится с новым ухажером и может тут же выскочить за него замуж. Каждый ухажер характеризуется параметром $X$, полезностью, которую Маша получит в день свадьбы с ним. Вы о чем подумали? Величина $X$ распределено равномерно на $[0;1]$. Ежедневная полезность Маши от замужнего состояния после дня свадьбы равна 0. Дисконт фактор (с которым дисконтируется Машина полезность) равен $\delta$. 
\begin{enumerate}
\item Как выглядит оптимальная стратегия Маши, если она выбирает мужа на всю жизнь?
\item Как выглядит оптимальная стратегия Маши, если она легко может развестись? 
\end{enumerate} }
\solution{ }



\subsection{\textit{o}-малое}

\problem{Случайные величины $ X_{1} ,\ldots, X_{n} $ одинаково распределены с функцией плотности $ p(t) $ и независимы. Найдите функцию плотности третьего по величине $ X_{(3)}$.}
\solution{$ \PP(X_{(3)}\in [x;x+dx])= C_{n}^{2}C_{n-2}^{1} \bigl((F(x)+o(x)\bigr)^{n-3}\bigl(1-F(x)+o(x)\bigr)^{2}\bigl((f(x)+o(x)\bigr)dx $. Значит, искомая функция плотности равна $f_{X_{(3)}}(t)=f(t)F(t)^{n-3}\bigl(1-F(t)\bigr)^{2}$.}

\subsection{Вероятностный метод}
% задачи не по теории вероятностей, которые решаются с помощью теории вероятностей

% untyp
\problem{ На потоке 200 студентов. На контрольной было 6 задач. Известно, что каждую задачу решило не менее 120 человек.
Всегда ли преподаватель может выбрать двух студентов из потока так, что эти двое могут решить всю контрольную совместными усилиями?}
\solution{Выберем двух студентов из потока наугад. Вероятность того, что ни один из них не решил задачу \No\,1, не превосходит $\br{\frac{80}{200}}^2=0{,}16$.
Вероятность того, что ни один из них не решил задачу \No\,2, не превосходит 0{,}16 (по тем же причинам), и это справедливо для каждой из шести задач. Вероятность того, что хотя
бы одну задачу они на пару не решили, не превосходит суммы этих вероятностей, т.\,е. $0{,}16\cdot6=0{,}96$.
Значит, вероятность выбора пары студентов, которые совместными усилиями могут решить экзамен, не менее $0{,}04$. Значит, хотя бы одна такая пара существует.}


\subsection{Склеивание отрезка}


\problem{Машина может сломаться равновероятно в любой точке на дороге от города А до города Б. Когда машина сломается мы будем толкать ее до ближайшего сервиса. Где должны быть расположены три автосервиса чтобы минимизировать ожидаемую продолжительность толкания? А если автосервисов будет $n$?}
\solution{Проверить. Разбиваем отрезок на $n$ частей, ставим автосервис в центр каждой части
\url{http://math.stackexchange.com/questions/37254/} }



\problem{ Рулет \\
Длинный рулет разрезан на $n$ частей. Каждый из $k$ гостей по очереди забирает себе один кусочек, выбираемый случайным образом. В результате остается $n-k$ кусочков рулета. Оставшиеся кусочки рулета лежат <<сериями>>, разделенными <<дырками>> от забранных кусочков. Каково ожидаемое число <<серий>> оставшихся кусочков? К чему стремится эта величина при $n\to\infty$?\\
Aвтор: Алексей Суздальцев  }
\solution{ Решение 1: \\
Величина $X$ --- число <<серий>>, $X=X_{1}+...+X_{n}$, где $X_{i}$ - индикатор, показывающий, начинается ли новая серия с $i$-го кусочка. \\
Ответ: $\frac{n-k}{n}+(n-1)\frac{k}{n}\frac{n-k}{n-1}=(k+1)\frac{n-k}{n}$ \\
Решение 2: \\
Закольцуем рулет, добавив в него еще один кусочек, для хозяина дома --- для Алексея Суздальцева. Получаем $(k+1)$ потенциальную серию. Вероятность того, что некая серия непуста, равна $\frac{n-k}{n}$. Итого, $(k+1)\frac{n-k}{n}$ }

\problem{
Петя ищет 6 нужных ему книг в стопке из 30 книг. Книги внешне не отличимы. Сколько книг в среднем ему придется пересмотреть? Просмотренные книги Петя в общую кучу не возвращает.  }
\solution{Можно считать, что Петя берет книги подряд из хорошо перетасованной стопки. Соответственно он берет 6 книг и 6 интервалов (книги до 1-ой нужной, книги от 1-ой нужной до 2-ой нужной, и т.д.). Если считать, что средняя длина всех интервалов одинаковая, то получается такой ответ: $6+6\cdot\frac{24}{7}$. \\
Доказательство того, что средняя длина всех интервалов одинаковая: \\
Расположим 30 книг по кругу. Среди этих 30 книг отметим случайным образом 7 книг и случайным образом занумеруем их от 1 до 7. Эти 7 книг разбивают круг из книг на 7 частей. В силу симметрии средняя длина каждой части одинакова и равна $\frac{24}{7}$. Будем трактовать книгу номер 1 как разбивающую круг на стопку. А книги 2-7, как нужные Пети. }



\subsection{Мартингальный метод}

Справедливую игру стратегией не испортишь!

\problem{ У Васи $100$ рублей, у Пети --- $150$. Они играют в орлянку
правильной монеткой до тех пор, пока все деньги не перейдут к
одному игроку. Какова вероятность, что победит Вася? }
\solution{Пусть $X_{n}$ - благосостояние Васи после $n$-го хода, тогда
$\E(X_{n})=100$. $\E(X_{final})=250p+0(1-p)$.  }



% !Mode:: "TeX:UTF-8"
\section{Неравенства, связанные с ожиданием}

\subsection{Чебышёв/Марков/Кантелли/Чернов}
\problem{
У последовательности неотрицательных случайных величин $X_1$, $X_2$,\ldots дисперсия постоянна, а  математическое ожидание стремится к бесконечности, $\lim \E(X)=+\infty$. Найдите $\lim \P(X_n>a)$. }
\solution{ $\P(X_n\leq q)=\P(X_n-\E(X_n)\leq a-\E(X_n))$. При больших $n$ величина $a-\E(X_n)$ отрицательна, поэтому $\P(X_n-\E(X_n)\leq a-\E(X_n))\leq \P(|X_n-\E(X_n)|\geq |a-\E(X_n)|)\leq c/(a-\E(X_n))^2$.}


\subsection{Йенсен}

\subsection{Коши"--~Шварц}


% !Mode:: "TeX:UTF-8"
\section{Компьютерные (use R or Python!)}

\subsection{Нахождение сложных сумм/поиск оптимальных стратегий}
\problem{Перед нами 10 коробок. Изначально в 1-й коробке 1 шар, во 2-й "--- 2 шара и т.\,д. Мы равновероятно выбираем одну из коробок, вытаскиваем из неё шар и кладём его равновероятно в одну из девяти оставшихся коробок. Мы повторяем это перекладывание до тех пор, пока одна из коробок не станет пустой. Пусть $N$ "--- число перекладываний. С помощью компьютера оцените $\E(N) $, $\Var(N)$.}
\solution{ $ \E(N)\approx 12{,}15$.}

\problem{В классе учатся $n$ человек. Нас интересует вероятность того, что хотя бы у двух из них дни рождения будут в соседние дни. (31 декабря и 1 января будем считать соседними). При каком $n$ эта вероятность впервые достигнет 0{,}5?}
\solution{ $\PP_{16}=0{,}482\,390\,182$, $\PP_{17}=0{,}525\,836\,596$.}

\problem{В классе 30 человек. Какова вероятность того, что есть три человека, у которых совпадают дни рождения? Найдите ответ с помощью симуляций и с помощью пуассоновского приближения. При каком количестве человек эта вероятность впервые превысит 50\,\%?}
\solution{Симуляции $p=0{,}028\,5 $, Пуассон: $p=0{,}03$.}

\problem{Сколько нужно людей, чтобы вероятность того, что в каждый день года у кого-то день рождения, впервые превысила 50\,\%?}
\solution{$\PP(T\leq k)= n^{-k}n!\left\{ \begin{array}{c} k \\ n \end{array} \right\}$ (число Стирлинга второго рода); 2287.}

\problem{Сколько в среднем нужно взять из колоды в 52 карты, чтобы насобирать подряд 5 карт одной масти? Не обязательно одной масти?}
\solution{Если у нас $m=13$ достоинств и $n=4$ масти, то ответ имеет вид: $mn\prod\limits_{i=1}^{}\frac{in}{in+1}\approx 45{,}3$; $\approx 28{,}0$.}

% untyp
\problem{Маленький мальчик торгует на улице еженедельной газетой. Покупает
он ее по 15 рублей, а продает по 30 рублей. Количество потенциальных покупателей --- случайная величина с распределением Пуассона и средним
значением равным 50. Нераспроданные газеты ничего не стоят. Пусть $n$ --- количество газет, максимизирующее ожидаемую прибыль мальчика.
\begin{enumerate}
\item Чему примерно должно быть равно значение функции распределения в
точке  $n$?
\item  С помощью компьютера найдите  $n$ и ожидаемую прибыль.
\end{enumerate}}
\solution{$n=50$, $665.51$

\inputminted{python}{src_python/newspapers_notext.py}
}



\problem{Подбрасывается правильная монетка. В любой момент вы можете сказать <<Хватит>>. Ваш выигрыш равен доле орлов на момент остановки. С помощью компьютера определите, чему равен ожидаемый выигрыш при использовании оптимальной стратегии? При решении на компьютере можно считать, что число подбрасываний ограничено скажем 500.}
\solution{Около 0.7925}

\problem{У Васи есть 100 рублей. Вася открывает карты из колоды одну за одной в случайном порядке. В колоде 26 красных и 26 чёрных карт. Перед открытием каждой карты Вася может поставить на цвет любую целую сумму рублей в пределах своего капитала. Если он угадал цвет, то его ставка возвращается удвоенной, если нет, то он теряет ставку. Задача Васи --- максимизировать ожидаемый финальный выигрыш. С помощью компьютера определите, как выглядит оптимальная стратегия и какую сумму он в среднем выигрывает? }
\solution{Ожидаемая сумму в концу игры - 808 рублей}



\subsection{Проведение симуляций}



% (к статистике) проверка простых гипотез на РЕАЛЬНЫХ (исторических) данных

\subsection{Statistics}
\problem{Петя подбрасывал две монетки неправильные монетки. Результаты подбрасывания:

Число подбрасываний первой. Число подбрасываний второй. Общее число орлов.
... ... ...

... ... ...
... ... ...


Оцените с помощью компьютера вероятности выпадения орлом для каждой монетки. Постройте доверительные интервалы.
(Красивого решения в явном виде нет).

Можно использовать нормальное приближение


}
\solution{}


\problem{Голосовать можно за трёх кандидатов: А, Б и В. Из 100 опрошенных 20 хотят голосовать за А, 40 "--- за Б, остальные "--- за В. В осях $p_{A}$, $p_{B}$ постройте 90-процентную доверительную двумерную область.}
\solution{ }




% !Mode:: "TeX:UTF-8"
\section{Пуассоновский поток и экспоненциальное распределение}

\subsection{Пуассоновский поток}


\problem{ Предположим, что кузнечики на большой поляне распределены
по пуассоновскому закону с  $\lambda=3$  на квадратный метр. Какой
следует взять сторону квадрата, чтобы вероятность найти в нем хотя
бы одного кузнечика была равна  $0,8$? }
\solution{ }


\problem{Саша красит стены в своей комнате, а Алёша "--- в своей. У каждой комнаты четыре стены. Предположим, что время покраски одной стены и для Саши, и для Алёши "--- экспоненциальная случайная величина с параметром $\lambda$. Какова вероятность того, что Саша успеет покрасить 3 стены раньше, чем Алёша "--- две?}

\solution{Каждая следующая стена равновероятно покрашена Сашей и Алёшей. Значит, нам нужны $\PP(SSS)+\PP(SSAS)+\PP(SASS)+\PP(ASSS)=\frac{5}{16}$. По другому: для простоты положим $\lambda=1$. Пусть $T$ "--- время, когда Саша закончит 3 стены. Функция плотности гамма-распределения (сумма трёх экспоненциальных) $f(t)=0{,}5t^{2}e^{-t}$. Нам нужна вероятность того, что к тому времени Алёша успеет меньше двух стен: $\int_{0}^{\infty} \PP(N_{t}<2 \mid T=t)\,dt =\ldots=\frac{5}{16}$.}

\problem{Машины подъезжают к светофору пуассоновским потоком с интенсивностью $\lambda $. Для простоты будем считать, что первая машина подъезжает в $ t=0 $. Светофор горит зелёным только в целые моменты времени, и этого достаточно чтобы пропустить одну машину, т.\,е. светофор горит красным при $ t\in(0;1) $, $ t\in(1;2) $, $ t\in(2;3) $ и т.\,д. Какой будет средняя длина очереди через продолжительное время? Чему будет равна вероятность, что очередь пуста?}
\solution{Производящая функция удовлетворяет соотношению:
\[ g(t)=\exp(\lambda (t-1))\frac{g(t)+tg(0)-g(0)}{t} \]
\[ g(t)=g(0)\frac{(t-1)\exp(\lambda (t-1))}{t-\exp(\lambda (t-1))} \]
Из условия $ g(1)\to 1 $ находим $ g(0)=1-\lambda $ и, помучившись, $\E(X_{\infty})=g'(1)=\frac{\lambda(2-\lambda)}{2\cdot(1-\lambda)} $.}

\cat{Poisson} \cat{gen_fun}

\problem{В офисе два телефона: зелёный и красный. Входящие звонки на красный "--- Пуассоновский поток событий с интенсивностью $\lambda_{1}=4$ звонка в час, входящие на зелёный "--- с интенсивностью $\lambda_2=5$ звонков в час. Секретарша Василиса Премудрая одна в офисе. Время разговора "--- случайная величина, имеющая экспоненциальное распределение со средним временем $5$ минут. Если Василиса занята разговором, то на второй телефон она не отвечает. Сколько звонков в час в среднем пропустит Василиса, потому что будет занята разговором по другому телефону? Являются ли пропущенные звонки Пуассоновским потоком? }
\solution{}

\problem{В офисе два телефона: зелёный и красный. Входящие звонки на красный "--- Пуассоновский поток событий с интенсивностью $\lambda_{1}=4$ звонка в час, входящие на зелёный "--- с интенсивностью $\lambda_2=5$ звонков в час. Секретарша Василиса Премудрая одна в офисе. Перед началом рабочего дня она подбрасывает монетку и отключает один из телефонов: зелёный "--- если выпала решка, красный "--- если выпал орёл. Обозначим за $Y$ время от начала дня до первого звонка. Найдите функцию плотности $Y$. }
\solution{}

\problem{Случайная величина $X$ имеет экспоненциальное распределение с параметром $\lambda$. Найдите медиану $X$. }
\solution{}

\subsection{Пуассоновское приближение}
% при замене на Poisson(\lambda=np) ошибка не превосходит
% min(1,1/\lambda)\sum p_{i}^{2}

\problem{ Используя пуассоновское предупреждение найдите вероятности
\begin{enumerate}
\item В гирлянде 25 лампочек. Вероятность брака для отдельной
лампочки равна 0,01. Какова вероятность того, что гирлянда
полностью исправна? 
\item По некоему предмету незачет получило всего 2\% студентов.
Какова вероятность того, что в группе из 50 студентов будет ровно
1 человек с незачетом? 
\item Вася испек 40 булочек. В каждую из них он кладет изюминку с
$p=0,02$. Какова вероятность того, что всего окажется 3 булочки с
изюмом? 
\end{enumerate} }
\solution{ }

\problem{ Вася каждый день подбрасывает монетку 10 раз. Монетка с
вероятностью 0,005 встает на ребро. Используя пуассоновскую
аппроксимацию, оцените вероятность того, что за 100 дней монетка
встанет на ребро ровно 3 раза. }
\solution{ }

\problem{ Страховая компания <<Ой>> заключает договор страхования от
<<невыезда>> (не выдачи визы) с туристами, покупающими туры в
Европу. Из предыдущей практики известно, что в среднем отказывают
в визе одному из 130 человек. Найдите вероятность того, что из 200
застраховавшихся в <<Ой>> туристов, четверым потребуется страховое
возмещение. }
\solution{ }

\problem{
Вася, владелец крупного Интернет-портала, вывесил на главной
странице рекламный баннер. Ежедневно его страницу посещают 1000
человек. Вероятность того, что посетитель портала кликнет по
баннеру равна 0,003. С помощью пуассоноского приближения оцените
вероятность того, что за один день не будет ни одного клика по
баннеру.}
\solution{ }




% !Mode:: "TeX:UTF-8"
\section{Нормальное распределение и ЦПТ}
\subsection{Одномерное нормальное распределение}

\problem{Где находятся точки перегиба функции плотности для нормального распределения?}
\solution{$ \mu \pm \sigma $.}

\problem{Пусть $X\sim \mN(\mu;\sigma^{2})$ и $t>\mu$. В какой точке функция $\PP(X\in [t;t+dt])$ убывает быстрее всего?}
\solution{$ \mu + \sigma $.}

\problem{Имеются две акции с доходностями $X$ и $Y$ на один вложенный рубль. Доходности некоррелированы, $\E(X)=0{,}09$, $\E(Y)=0{,}04$, $\sigma_X=0{,}5$, $\sigma_Y=0{,}1$. У инвестора есть 1~рубль. Он покупает на $a\in (0;1)$ рубля первых акций и на $(1-a)$ вторых акций. Обозначим за $\mu(a)$ и $\sigma(a)$ ожидаемую доходность и стандартное отклонение доходности полученного портфеля.
\begin{enumerate}
\item Постройте множество возможных $\mu(a)$ и $\sigma(a)$
\item Найдите наименее рисковый портфель. Чему равна его ожидаемая доходность?
\end{enumerate} }
\solution{}

\problem{Имеются две акции с доходностями $X$ и $Y$ на один вложенный рубль. Доходности некоррелированы, $\E(X)=0{,}09$, $\E(Y)=0{,}04$, $\sigma_X=0{,}5$, $\sigma_Y=0{,}1$. У инвестора есть 1~рубль. Инвестор подкидывает неправильную монетку, выпадающую орлом с вероятностью $p$. Если монетка выпадает орлом, он покупает первые акции, если решкой, то вторые. Обозначим за $\mu(p)$ и $\sigma(p)$ ожидаемую доходность и стандартное отклонение доходности полученной стратегии.
\begin{enumerate}
\item Постройте множество возможных $\mu(p)$ и $\sigma(p)$.
\item Найдите функцию плотности доходности полученной стратегии.
\end{enumerate} }
\solution{}


\problem{ Известна функция плотности случайной величины,  $p_{X}(t)=c\cdot \exp (-8t^{2} +5t)$. Найдите $E(X)$,  $\sigma _{X} $. }
\solution{ выделяем полный квадрат, $\E(X)=\frac{5}{16}$, $\sigma_{X}=\frac{1}{4}$  }



\subsection{ЦПТ}
\problem{
Вася играет в компьютерную игру "--- <<стрелялку"=бродилку>>. По
сюжету ему нужно убить 60 монстров. На один выстрел уходит ровно 1
минута. Вероятность убить монстра с одного выстрела равна 0{,}25.
Количество выстрелов не ограничено.
\begin{enumerate}
\item Сколько времени в среднем Вася тратит на одного монстра?
\item  Найдите дисперсию этого времени.
\item  Какова вероятность того, что Вася закончит игру меньше, чем за
3 часа?
\end{enumerate}
 }
\solution{ }


\subsection{Многомерное нормальное распределение}

\problem{
Ермолай Лопахин решил приступить к вырубке вишневого сада. Однако выяснилось, что растут в нём не только вишни, но и яблони. Причём, по словам Любови Андреевны Раневской, среднее количество деревьев (а они периодически погибают от холода или жары, либо из семян вырастают новые) в саду распределено в соответствии с нормальным законом ($X$ "--- число яблонь, $Y$ "--- число вишен) со следующими параметрами:
\begin{equation}
\begin{pmatrix}	X \\ 	Y 	\end{pmatrix}
\sim \mN
\left(
\begin{pmatrix}
25 \\ 125
\end{pmatrix}
;
\begin{pmatrix}
	5 & 4 \\
	4 & 10
	\end{pmatrix}
\right)
\end{equation}

Найдите вероятность того, что Ермолаю Лопахину придется вырубить более 150~деревьев.
Каково ожидаемое число подлежащих вырубке вишен, если известно, что предприимчивый и последовательный Лопахин, не затронув ни одного вишнёвого дерева, начал очистку сада с яблонь и все 35~яблонь уже вырубил?

Автор: Кирилл Фурманов, Ира ...}
\solution{ }


\problem{
В самолете пассажирам предлагают на выбор <<мясо>> или <<курицу>>. В самолет 250 мест. Каждый пассажир с вероятностью 0.6 выбирает курицу, и с вероятностью 0.4 - мясо. Сколько порций курицы и мяса нужно взять, чтобы с вероятностью 99\% каждый пассажир получил предпочитаемое блюдо, а стоимость <<мяса>> и <<курицы>> для компании одинаковая? \\
Как изменится ответ, если компания берет на борт одинаковое количество <<мяса>> и <<курицы>>? }
\solution{ 
$K=170$, $M=120$ (симметричный интервал) или $K=M=168$ (площадь с одного края можно принять за 0) \\
Вариант: театр, два входа, два гардероба а) только пары, б) по одному }


\subsection{Распределения связанные с нормальным}

% задачи без статистики на свойства t, F, chi распределений


\problem{Сравните $\E(F_{k,n})$ и $Var(t_n)$.}
\solution{$\E(F_{k,n})=Var(t_n)$}


\problem{Пусть $X\sim t_{n}$. Как распределена величина $Y=X^{2}$? }
\solution{ $F_{1,n}$ }


\problem{  На плоскости выбирается точка со случайными координатами. Абсцисса
и ордината независимы и распределены $N(0;1)$. Какова вероятность
того, что расстояние от точки до начала координат будет больше
2,45? }
\solution{ Квадрат расстояния имеет $\chi^2_2$ распределение  }



% !Mode:: "TeX:UTF-8"
\section{Случайное блуждание}
\subsection{Дискретное случайное блуждание}


\problem{Какова вероятность того, что трёхмерное случайное блуждание бесконечное количество раз пересечёт вертикальную ось?}
\solution{1. Про вертикальные шаги можно забыть, а вероятность бесконечное количество раз посетить 0 для двумерного случайного блуждания равна 1.}


\problem{Пусть $X_{n}$ "--- симметричное случайное блуждание. Сколько времени в среднем придётся ждать, пока остаток от деления $X_{n}$ на 183 окажется равным 39?}
\solution{$39^{2}$.}



\subsection{Принцип отражения}
\problem{  \zdt{Выборы} \par
После выборов, в которых участвуют два кандидата, A и B, за них
поступило $a$ и $b$ ($a>b$) бюллетеней соответственно, скажем, 3 и
2. Если подсчёт голосов производится последовательным извлечением
бюллетеней из урны, то какова вероятность того, что хотя бы один
раз число вынутых бюллетеней, поданных за A и B, было одинаково? Какова вероятность того, что A всё время лидировал?
\begin{ist}
Mosteller.
\end{ist}
}
\solution{ }

\problem{ \zdt{Ничьи при бросании монеты} \par
Игроки A и B в орлянку играют $n$ раз. После первого бросания
каковы шансы на то, что в течение всей игры их выигрыши не
совпадут?
\begin{ist}
Mosteller.
\end{ist}}
\solution{ }

\problem{Доходность акции следует симметричному дискретному случайному блужданию. Какова вероятность того, что в момент времени $2k+1$ доходность будет выше, чем когда-либо в прошлом?

Источник: Алексей Суздальцев}
\solution{$\frac{C_{2k}^{k}}{2^{2k+1}}$. Совсем простого решения не знаю, хотя ответ простой.}



\subsection{Броуновское движение}


% !Mode:: "TeX:UTF-8"


% !Mode:: "TeX:UTF-8"

% !Mode:: "TeX:UTF-8"
% !Mode:: "TeX:UTF-8"
% MM, ML
\section{Метод максимального правдоподобия, метод моментов}

\problem{Падал первый снег и 30 школьников ловили снежинки. В среднем каждый поймал $3{,}3$ снежинки. Количества снежинок пойманные каждым --- независимые Пуассоновские случайные величины с общим параметром $\lambda$. 
\begin{enumerate}
\item Оцените $\lambda$ используя метод максимального правдоподобия
\item Оцените дисперсию полученной оценки
\item На уровне значимости 5\% проверьте гипотезу о том, что $\lambda=3$ используя тест множителей Лагранжа
\item ... используя тест отношения правдоподобия
\item ... используя тест Вальда 
\item Постройте 95\% доверительный интервал Вальда для $\lambda$
\end{enumerate}}
\solution{ Логарифмическая функция правдоподобия
\begin{equation}
l(\lambda)=-n\lambda+\sum x_i \ln\lambda-\sum \ln(x_i !)
\end{equation}
Первая производная:
\begin{equation}
l'(\lambda)=-n+\frac{\sum x_i}{\lambda}
\end{equation}
Оценка метода максимального правдоподобия имеет вид $\hat{\lambda}=\bar{X}_n$
Ожидаемае информация Фишера $I(\lambda)=n/\lambda$, наблюдаемая информация Фишера $I(\hat{\lambda})=n/\hat{\lambda}$.
Тест Вальда $W=(\hat{\lambda}-3)^2\cdot I(\hat{\lambda})$ \\
Тест множителей Лагранжа $LM=l'(3)\cdot I^{-1}(3)$ \\
Тест отношения правдоподобия $LR=2(l(\hat{\lambda})-l(3))$ 
}



\problem{В коробке 10 внешне не отличимых шоколадных конфет. Внутри $k$ штук из них есть орех. Мы выбирали конфеты наугад по одной и ели. Первый орех оказался в третьей по счету конфете. 

Оцените неизвестный параметр $k$ методом моментов, методом максимального правдоподобия.}
\solution{}



\problem{
Интервал времени в минутах между спам-письмами по электронной почте --- случайная величина с функцией плотности 
\begin{equation}
p(t)=
\begin{cases}
a^2\cdot t\cdot \exp(-at), \quad t\geq 0 \\
0, \quad t<0
\end{cases},
\end{equation}
где $a$ --- неизвестный параметр. По выборке из 20 наблюдений известно, что $\sum_{i=1}^{n}X_{i}=625$, $\sum_{i=1}^{n}\ln(X_i)=25$.


\begin{enumerate}
\item Оцените $a$ методом моментов
\item Оцените $a$ методом максимального правдоподобия
\item Постройте 95\% доверительный интервал для $a$ с помощью максимального правдоподобия
\end{enumerate}}


\solution{
\begin{equation}
E(\bar{X})=E(X_{i})=\int_{0}^{\infty} a^2\cdot t^2 \exp(-at)dt
\end{equation}

Для удобства заметим, что
\begin{equation}
\int_{0}^{\infty}t^n\exp(-at)dt=0+\int_{0}^{\infty}t^{n-1}\frac{n}{a}\exp(-at)dt
\end{equation}

Получаем мат. ожидание:
\begin{multline}
E(\bar{X})=\int_{0}^{\infty} a^2\cdot t\frac{2}{a}\cdot \exp(-at)dt=\\
=\int_{0}^{\infty} 2a\cdot t\cdot \exp(-at)dt=\int_{0}^{\infty} 2a\cdot \frac{1}{a} \exp(-at)dt=\\
=\int_{0}^{\infty} 2\cdot \exp(-at)dt=\frac{2}{a}
\end{multline}

Метод моментов
\begin{equation}
\bar{X}=\frac{2}{\hat{a}_{MM}}
\end{equation}

Получаем $\hat{a}_{MM}$:
\begin{equation}
\hat{a}_{MM}=\frac{2}{\bar{X}}=\frac{40}{625}=\frac{8}{125}=0.064
\end{equation}

Метод максимального правдоподобия
\begin{equation}
l=\sum_{i=1}^{n}\ln(p(x_{i}))=\sum_{i=1}^{n}(2\ln(a)+ln(x_{i})-ax_{i})=2n\ln(a)+\sum \ln(x_{i})-a\sum x_{i}
\end{equation}

\begin{equation}
l'(a)=\frac{2n}{a}-\sum x_{i}
\end{equation}

Получаем $\hat{a}_{ML}$:
\begin{equation}
\hat{a}_{ML}=\frac{2n}{\sum X_{i}}=\frac{2}{\bar{X}}=\hat{a}_{MM}
\end{equation}

Наблюдаемая информация Фишера
\begin{equation}
J_{n}(\hat{a})=-l''(\hat{a})=\frac{2n}{\hat{a}^{2}}=2n\cdot \left(\frac{\sum X_{i}}{2n}\right)^2=\frac{(\sum X_{i})^2}{2n}=\frac{625^2}{40}
\end{equation}

Доверительный интервал
\begin{equation}
\hat{a}\pm 1.96\cdot J_{n}^{-1/2}=0.064\pm 0.0198=[0.0442;0.0838]
\end{equation} }

\problem{В банке 10 независимых клиентских <<окошек>>. В момент открытия в банк вошло 10 человек. Каждый клиент встал к отдельному окошку. Других клиентов банке в этот день не было. Предположим, что время обслуживания одного клиента распределено экспоненциально с параметром $\lambda$. Оцените параметр $\lambda$  и оцените дисперсию оценки методом максимального правдоподобия в каждой из ситуаций 

\begin{enumerate}
\item Менеджер записал время обслуживания клиента в каждом окошке. Окошко \No 1 обслужило своего клиента за 10 минут, окошко \No 2 обслужило своего клиента за 20 минут; оставшуюся часть записей менеджер благополучно затерял. 
\item Менеджер наблюдал за окошками в течение получаса и записывал время обслуживания клиента. Окошко \No 1 обслужило своего клиента за 10 минут, окошко \No 2 обслужило своего клиента за 20 минут; остальные окошки еще обслуживали своих первых клиентов в тот момент, когда менеджер удалился. 
\item  Менеджер наблюдал за окошками в течение получаса. За эти полчаса два окошка успели обслужить своих клиентов. Остальные окошки еще обслуживали своих первых клиентов в тот момент, когда менеджер удалился. 
\item Менеджер наблюдал за окошками и решил записать время обслуживания первых двух клиентов. Через  10 минут от начала работы был обслужен первый клиент в одном из окошек, через 20 минут от начала работы был обслужен второй клиент. Сразу после того, как был обслужен второй клиент менеджер прекратил наблюдение. 
\item Одновременно с открытием банка началась деловая встреча директора банка с инспектором по охране труда. Время проведения таких встреч --- случайная величина, имеющая экспоненциальное распределение со средним временем 30 минут. За время проведения встречи было обслужено два клиента. 
\item Одновременно с открытием банка началась деловая встреча директора банка с инспектором по охране труда. Время проведения таких встреч - случайная величина, имеющая экспоненциальное распределение со средним временем 30 минут. За время проведения встречи было обслужено два клиента, один за 10 минут, второй --- за 20 минут. 
\item Изменим условие: в банке 11 окошек, при открытии банка вошло 11 клиентов. Других клиентов в этот день не было. Клиент попавший в окошко \No 11 смотрел за остальными. Раньше клиента из окошка  \No 11 освободилось двое клиентов: за 10 минут и за 20 минут. 
\end{enumerate} }
\solution{}


\problem{Предположим, что доход жителей страны распределен экспоненциально. 
Имеется выборка из 1000 наблюдений по жителям столицы. Если возможно, постройте 90\% доверительный интервал для $\lambda$ в следующих случаях
\begin{enumerate}
\item Столицу можно считать случайной выборкой из жителей страны
\item В столице селятся только люди с доходом больше 100 тыс. рублей. 
\item В столице селятся только люди с доходом больше $m$ тыс. рублей, где $m$ - неизвестная константа. При этом постройте также и 90\% доверительный интервал для $m$ 
\item В столице живут 10\% самых богатых жителей страны
\item В столице живут $p$\% самых богатых жителей страны, где $p$ - неизвестная константа. Постройте также 90\% доверительный интервал для $p$
\end{enumerate} }
\solution{}

\problem{Случайные величины $X_1$, $X_2$, \ldots, $X_n$ независимы и нормальны $N(\mu,\sigma^2)$ с неизвестным математический ожиданием и дисперсией. При помощи теста множителей Лагранжа, теста отношения правдоподобия и теста Вальда проверьте гипотезы
\begin{enumerate}
\item $H_0$: $\mu=0$
\item $H_0$: $\sigma^2=1$
\item 
$H_0$: $\left\{\begin{array}{l}
\mu=0 \\
\sigma^2=1
\end{array}\right.$
\end{enumerate} }
\solution{}


% !Mode:: "TeX:UTF-8"

% !Mode:: "TeX:UTF-8"
% тестирование гипотез. общие свойства


\problem{Вовочка тестирует гипотезу $H_{0}$ против гипотезы $H_{a}$. Предположим, что $H_{0}$ на самом деле верна. По своей сути p-value является случайной величиной. Какое распределение оно имеет?}
\solution{Равномерное на $[0;1]$ }


\problem{Гражданин Фёдор решает проверить, не жульничает ли напёрсточник Афанасий, для чего предлагает Афанасию сыграть 5 партий в напёрстки. Фёдор решает, что в каждой партии будет выбирать один из трёх напёрстков наугад, не смотря на движения рук ведущего. Основная гипотеза: Афанасий честен, и вероятность правильно угадать напёрсток, под которым спрятан шарик, равна 1/3. Альтернативная гипотеза: Афанасий каким-то образом жульничает (например, незаметно прячет шарик), так что вероятность угадать нужный напёрсток меньше, чем 1/3. Статистический критерий: основная гипотеза отвергается, если Фёдор ни разу не угадает, где шарик.
\begin{enumerate}
\item Найдите уровень значимости критерия.
\item Найдите мощность критерия в том случае, когда Афанасий жульничает, так что вероятность угадать нужный напёрсток равна 1/5.
\end{enumerate}}
\solution{}



% !Mode:: "TeX:UTF-8"
\section{Гипотезы о среднем и сравнении среднего при большом $n$}


\problem{Предположим, что исходные наблюдения $X_{i}$ нормальны $N(\mu,\sigma^{2})$ и независимы. Константы $\mu$ и $\sigma$ неизвестны. Вовочка строит доверительный интервал для $\mu$ по первой половине доступных наблюдений. Петя --- по всем наблюдениям. Может ли получится у Вовочки интервал меньшей ширины, чем у Пети?}
\solution{Да. Например, если первая половина наблюдений попала рядом с $\mu$, а вторая --- далеко. Ну не повезло Пете.}

% untyp
\problem{Вася и Петя выясняют, кто лучше умеет знакомиться с девушками. Вася попытался познакомиться с 100 девушек, из них 54 девушки дали ему номер своего телефона. Петя  попытался познакомиться с 900 девушек, из них 495 дали ему номер своего телефона. 

Вася и Петя изучили курс матстата и начали спорить. Петя утверждает, что ему в среднем чаще девушки дают свой номер телефона и аргументирует это так: давай проверим гипотезу, что в среднем ровно половина девушек даёт номер своего телефона, против альтернативной гипотезы, что больше половины. По твоим данным эта гипотеза не отвергается, а по моим --- отвергается.

Вася утверждает, что он лучше убеждает девушек. Аргументирует это так: давай проверим гипотезу, что в среднем 60\% девушек даёт номер своего телефона, против альтернативной гипотезы, что меньше 60\:. По твоим данным эта гипотеза отвергается, а по моим --- не отвергается. 

Кто из них прав?}
\solution{Оба они делают одну ошибку: если $H_0$ не отвергается, это не значит, что она --- верна. Корректнее было бы провести тест на сравнение средних. }
Идея: Кирилл Фурманов



% !Mode:: "TeX:UTF-8"

% !Mode:: "TeX:UTF-8"

% untyp
\problem{ 
При подбрасывании кубика грани выпали 234, 229, 240, 219,
236 и 231 раз соответственно. Проверьте гипотезу о том, что кубик
<<правильный>>. } 
\solution{} 

% untyp
\problem{
Проверьте независимость дохода и пола по таблице: \\
\begin{tabular}{|c|c|c|c|}
  \hline
   & $<500$ & $500-1000$ & $>1000$ \\
  \hline
  М & 112 & 266 & 34 \\
  Ж & 140 & 152 & 11 \\
  \hline
\end{tabular} } 
\solution{} 

% untyp
\problem{
Вася Сидоров утверждает, что ходит в кино в два раза чаще, чем в
спортзал, а в спортзал в два раза чаще, чем в театр. За последние
полгода он 10 раз был в театре, 17 раз - в спортзале и
39 раз - в кино. Правдоподобно ли Васино утверждение? } 
\solution{} 

% untyp
\problem{
Проверьте независимость пола респондента и предпочитаемого
им сока: \par
\begin{tabular}{|c|c|c|c|}
  \hline
   & Апельсиновый & Томатный & Вишневый \\
  \hline
  М & 69 & 40 & 23 \\
  Ж & 74 & 62 & 34 \\
  \hline
\end{tabular} } 
\solution{} 


% untyp
\problem{
У 200 человек записали цвет глаз и волос. На уровне значимости
10\% проверьте гипотезу о независимости этих признаков. \par
\begin{tabular}{|c|c|c|c|}
  \hline
  Цвет глаз/волос & Светлые & Темные & Итого \\
  \hline
  Зеленые & 49 & 25 & 74 \\
  Другие & 30 & 96 & 126 \\
  \hline
  Итого & 79 & 121 & 200 \\
  \hline
\end{tabular} } 
\solution{} 

% untyp
\problem{
Идея задачи на хи-квадрат. \par
Если предложить голосовать за 3 альтернативы... \par
Если предложить голосовать за 4 альтернативы... \par
Выполняется ли предпосылка независимости от посторонних
альтернатив? } 
\solution{} 



% untyp
\problem{Когда Пирсон придумал хи-квадрат тест на независимость признаков (около 1900 г.), он не был уверен в правильном количестве степеней свободы. Он разошелся во мнениях с Фишером. Фишер считал, что для таблицы сопряженности размера два на два хи-квадрат статистика будет иметь три степени свободы, а Пирсон - что одну. Чтобы выяснить истину, Фишер взял большое количество таблиц два на два с заведомо независимыми признаками и посчитал среднее значение хи-квадрат статистики. Чему оно оказалось равно? Почему этот эксперимент помог выяснить истину?}
\solution{Среднее значение хи-квадрат случайной величины равно числу степеней свободы. Единице. Historical Note (as told by Chris Olsen): 
The Chi-square statistic was invented by Karl Pearson about 1900. Pearson knew what the Chi-square distribution looks like, but he was unsure about the degrees of freedom. 

About 15 years later, Fisher got involved. He and Pearson were unable to agree on the degrees of freedom for the two-by-two table, and they could not settle the issue mathematically. Pearson believed there was 1 degree of freedom and Fisher 3 degrees of freedom. 

They had no nice way to do simulations, which would be the modern approach, so Fisher looked at lots of data in two-by-two tables where the variables were thought to be
independent. For each table he calculated the Chi-square statistic. Recall that the expected value for the Chi-square statistic is the degrees of freedom. After collecting many Chi-square values, Fisher averaged all the values and got a result he described as <<embarrassingly close to 1>> 

This confirmed that there is one degree of freedom for a two-by-two table. Some years later this result was proved mathematically. }


% !Mode:: "TeX:UTF-8"

% untyp
\problem{
Из 10 опрошенных студентов часть предпочитала готовиться по
синему учебнику, а часть - по зеленому. В таблице представлены их
итоговые баллы.  \\
\begin{tabular}{|c|c|c|c|c|c|c|}
  \hline
  Синий & 76 & 45 & 57 & 65 &  &  \\
  \hline
  Зеленый & 49 & 59 & 66 & 81 & 38 & 88 \\
  \hline
\end{tabular} \\
а) С помощью теста Манна-Уитни (Mann-Whitney) проверьте гипотезу о
том, что выбор учебника не меняет закона распределения оценки. \par
\emph{Разрешается использование нормальной аппроксимации} \par
б) Возможно ли в этой задаче использовать (Wilcoxon Signed Rank Test)? } 
\solution{} 

% untyp
\problem{
Имеются результаты экзамена в двух группах. Группа 1: 45,
67, 87, 71, 34, 12, 54, 57; группа 2: 46, 66, 81, 72, 11, 47, 55,
51, 9, 99. С помощью теста Манна-Уитни на уровне значимости $5\%$ проверьте гипотезу о том, что результаты двух групп не отличаются. } 
\solution{} 

% untyp
\problem{
Имеются результаты нескольких студентов до и после
апелляции (в скобках указан результат до апелляции):  48(47),
54(52), 67(60), 56(60), 55(58), 55(60), 90(70), 71(81), 72(87),
69(60). Предполагая, что изменение оценки на апелляции симметрично распределено, на уровне значимости $5\%$ проверьте гипотезу о том, что
апелляция в среднем не сказывается на результатах. } 
\solution{} 

% untyp
\problem{
Имеются наблюдения за говорливостью 30 попугаев
(слов/день): 34, 56, 32, 45, 34, 45, 67, 1, 34, 12, 123, ... , 37
(всего 13 наблюдений меньше 40). Проверить гипотезу о том, что
медиана равна 40 (слов/день). } 
\solution{} 

% untyp
\problem{
Вашему вниманию представлены результаты прыжков в длину
Васи Сидорова на двух соревнованиях. На первых среди болельщиц
присутствовала Аня Иванова (его первая любовь): 1,83; 1,64; 2,27;
1,78; 1,89; 2,33; 1,61; 2,31. На вторых Аня среди болельщиц не
присутствовала: 1,26; 1,41; 2,05; 1,07; 1,59; 1,96; 1,29; 1,52;
1,18; 1,47. С помощью теста (Mann-Whitney) проверьте гипотезу о
том, что присутствие Ани Ивановой положительно влияет на
результаты Васи Сидорова. Уровень значимости $\alpha=0.05$. } 
\solution{} 

% untyp
\problem{
Некоторые результаты 2-х контрольных по теории вероятностей
выглядят следующим образом (указан результат за вторую контрольную
и в скобках результат за первую): 43(55), 113(108), 97(53),
68(42), 94(67), 90.5(97), 35(91), 126(127), 102(78), 89(83). Можно
ли считать (при $\alpha=0.05$), что вторую контрольную написали
лучше? Предположим, что разница в баллах распределена симметрично.} 
\solution{} 

% untyp
\problem{
Садовник осматривал по очереди розовые кусты вдоль ограды. Всего вдоль ограды растет 30 розовых кустов. Из них оказалось 20 здоровых и 10 больных. \par
Вот заметки садовника: $+++\ominus++\ominus\ominus\ominus++++\ominus\ominus ++\ominus+++++\ominus\ominus\ominus++++$ \par
(+ - здоровый куст, $\ominus$ - больной куст) \par
а) С помощью теста серий проверьте гипотезу о независимости испытаний \par
б) Какой естественный смысл имеет эта гипотеза? \par
Подсказка: можно использовать нормальное распределение }
\solution{}







% \section{Функции Экселя}
% % !Mode:: "TeX:UTF-8"
% f_excel
% возможно имеет смысл бросить его в пользу r/matlab?

СЛЧИС(), RAND() = генерирует с.в. равномерно распределенную на
$[0;1]$ \\
СРЗНАЧ(Набор чисел) = $\bar{X}=\frac{\sum X_{i}}{n}$ \\
ДИСП(Набор чисел) = $\frac{\sum (X_{i}-\bar{X})^{2}}{n-1}$ \\
СТЬЮДРАСП(x,n,1) = $P(T_{n}>x)$, где $T_{n}$ - с.в. имеющая $t$
распределение c $n$ степенями свободы \\
СТЬЮДРАСП(x,n,2) = $P(|T_{n}|>x)=2\cdot P(T_{n}>x)$, где $T_{n}$ -
с.в. имеющая $t$ распределение c $n$ степенями свободы \\
СТЬЮДРАСПОБР(p,n) = обратная к СТЬЮДРАСП(x,n,2) \\
НОРМРАСП(x,$\mu$,$\sigma$,1) = $P(N\le x)$, где $N\sim
N(\mu,\sigma^{2})$ \\
НОРМСТРАСП(x) = $P(N\le x)$, где $N\sim
N(0,1)$ \\
НОРМРАСПОБР(p,$\mu$,$\sigma$) = обратная к
НОРМРАСП(x,$\mu$,$\sigma$,1) \\
НОРМСТОБР(p) = обратная к НОРМСТРАСП(x) \\
ХИ2РАСП(x,n) = $P(C>x)$, где $C\sim \chi_{n}^{2}$ \\
ХИ2ОБР(p,n) = обратная к ХИ2РАСП(x,n) \\
FРАСП(x,$n_{1}$,$n_{2}$) = $P(F>x)$, где $F\sim F_{n_{1},n_{2}}$ \\
FРАСПОБР(p, $n_{1}$, $n_{2}$) = обратная к
FРАСП(x,$n_{1}$,$n_{2}$) \\
  % ok

% \section{Обозначения} % ok
% % !Mode:: "TeX:UTF-8"
% notations

$X\sim U[a;b]$ - случайная величина $X$ распределена равномерно на
отрезке $[a;b]$, буква U - от слова uniform \\
$X\sim N(\mu,\sigma^{2})$ \\
$X\sim N(\mu,\Sigma)$ $\Sigma$ - ковариационная матрица \\
iid - independent and identically distributed, независимы и
одинаково распределенные \\


% \section{Минитеория} % ok
% % minitheory

$\displaystyle \Omega $  = множество всех исходов

событие = набор исходов

свойства вероятности:  \\
$P(\Omega)=1$, $P(\emptyset)=0$,  $0\le P(C)\le 1$ \\
Если  $A$  и  $B$  несовместны (не могут произойти одновременно,
$A\cap B=\emptyset $ ), то  $P(A\cup B)=P(A)+P(B)$. \\

Существуют такое событие $D$, что $P(D)=0$, но $P\ne\emptyset$. \\

Число способов выбрать  $k$  предметов из $n$ ($C$ из  $n$  по
$k$), если не важен порядок: $C_{n}^{k} = (
\begin{array}{c}
  n \\
  k \\
\end{array}
) =\frac{n!}{k!(n-k)!}$. Maple: binomial(n,k); n!; \\


Условная вероятность наступления события  $A$, если известно, что
$B$ наступило =  $P(A|B)=\frac{P(A\cap B)}{P(B)} $ \\
условная вероятность определена при  $P(B)>0$ \\

события  $A$  и  $B$  называются независимыми, если  $P(A\cap
B)=P(A)\cdot P(B)$ или (для $P(B)>0$) $P(A|B)=P(A)$ \\


$A^{c} =\Omega \backslash A$  - дополнение к событию  $A$ \\

формула полной вероятности  $P(A)=P(A|B_{1} )\cdot P(B_{1}
)+...+P(A|B_{n} )\cdot P(B_{n} )$, если $B_{n} $  - не
пересекаются, и в сумме исчерпывают все возможные варианты. \\

Функция распределения  $F(a)=P(X\le a)$. \\

Функция  $p(t)$  называется функцией плотности для случайной
величины  $X$, если  $P(a\le X\le b)=\int _{a}^{b}p(t)dt $.
Функция плотности - это не вероятность. Это - почти :)
вероятность. \\

Для случайных величин с функцией плотности  $P(X=t)=0$. \\
Для всех $P(-\infty <X<+\infty )=1$. \\

Условная функция плотности  $p_{X|Y} (x|y)=\frac{p_{X,Y}
(x,y)}{p_{Y} (y)} $ \\
Для непрерывных с.в.  $F(a)=\int _{-\infty }^{a}p(t)dt $. \\


Для дискретных величин  $E(X)=\sum x_{i} \cdot P(X=x_{i} ) $,\\
Для непрерывных - $E(X)=\int x\cdot p(x)\cdot dx $. \\

Условное ожидание: \\
$E(X|A)=\sum x_{i} \cdot P(X=x_{i} |A) $.

Для удобства  $E(f(X))=\sum f(x_{i} )\cdot P(X=x_{i} ) $. \\


$E(aX+Y)=aE(X)+E(Y)$. \\


Дисперсия $Var(X)=E((X-E(X))^{2} )$. \\
Ковариация $Cov(X,Y)=E((X-E(X))(Y-E(Y)))$. \\
Cтандартное отклонение $\sigma _{X} =\sqrt{Var(X)} $ \\
Корреляция $cor(X,Y)=\frac{Cov(X,Y)}{\sigma _{X} \cdot \sigma
_{Y}}$ \\


С.в. независимы, если никакая информация о  $X$  не позволяет
сделать никаких выводов о  $Y$.

Для независимых с.в. $E(XY)=E(X)E(Y)$. Для дискретных независимых
случайных величин  $P(X=x\cap Y=y)=P(X=x)P(Y=y)$, для непрерывных
независимых $p_{X,Y} (x,y)=p_{X} (x)p_{Y} (y)$.

Выборочное среднее  $\bar{X}=\frac{X_{1} +...+X_{n} }{n} $, \\
Несмещенная оценка дисперсии  $\hat{\sigma }^{2} =\frac{\sum
(X_{i} -\bar{X})^{2}  }{n-1} $ \\

Центральная предельная теорема: \\
Если  $X_{i} $  - iid, $0<Var(X_{i} )<\infty $, то $\frac{S_{n}
-E(S_{n} )}{\sqrt{Var(S_{n} )} } \xrightarrow[{n\to \infty
}]{distribution} N(0;1)$. \\
Пуассоновское приближение: \\
Если  $L_{n} $  - биномиальные с.в. с параметрами  $(n,p_{n} )$  и
$np_{n} \stackrel{n\to \infty }{\longrightarrow}\lambda $, то
$P(L_{n} =k)\to e^{-\lambda } {\lambda ^{k} \mathord{\left/
{\vphantom {\lambda ^{k}  k!}} \right. \kern-\nulldelimiterspace}
k!} $ \\





Зоопарк:  \\
$p(t)=\lambda e^{-\lambda t}$, Экспоненциальное, $E(X)=\frac{1}{\lambda } $,  $Var(X)=\frac{1}{\lambda ^{2} } $ \\
$p(t)=\frac{1}{\sqrt{2\pi } \sigma } \exp (-\frac{(t-\mu )^{2}
}{2\sigma ^{2} } )$,
Нормальное \\
 $P(X=t)=e^{-\lambda } \frac{\lambda ^{t}
}{t!} $, Пуассон, $E(X)=\lambda $,  $Var(X)=\lambda $ \\
$P(X=t)=C_{N}^{t} p^{t} (1-p)^{N-t} $,
Биномиальное \\
$p(\vec{x}^{T} )=c \exp (-\frac{1}{2} (x-\mu )^{T} \Omega
^{-1} (x-\mu ))$, Многомерное нормальное \\

$X\sim N(\vec{\mu };\Omega )$,  $\Omega $ - ковариационная
матрица, $c=\frac{1}{(2\pi )^{{\tfrac{n}{2}} } \det
^{{\tfrac{1}{2}} } (\Omega
)}$ \\
Для двумерного нормального: \\
$E(X_{1}|X_{2})=\mu_{1}+\rho\frac{\sigma_{1}}{\sigma_{2}}(X_{2}-\mu_{2})$ \\
$Var(X_{1}|X_{2})=\sigma_{1}^{2}(1-\rho^{2})$ \\


События представляют собой Пуассоновский поток с параметром
$\lambda $: \\
Количество событий, происходящих за время  $t$  - с.в. имеющая
распределение Пуассона с ожиданием  $(\lambda \cdot
t)$.\\
Время между двумя событиями в потоке распределено экспоненциально
с ожиданием  $\frac{1}{\lambda } $.\\

Уровень значимости = <<порог редкости>>. Если происходит редкое
событие, то  $H_{0} $  отвергается.\\

Метод максимального правдоподобия:  Maximum likelihood, ML. \\
$\mathop{\max }\limits_{\theta } \prod p(x_{i},\theta
) $ \\
Полезен переход к логарифму. \\


Метод моментов: Method of moments, MM. \\
Если $E(\bar{X})=f(\theta)$, то находим  $\hat{\theta }$  из
уравнения $\bar{X}=f(\hat{\theta})$. \\


Оценка $\hat{\theta}$ неизвестного параметра $\theta$ называется: \\
несмещенной, если  $E(\hat{\theta})=\theta$ \\
состоятельной, если для  $\forall \varepsilon
>0$   $\lim P(|\theta -\hat{\theta }_{n}|>\varepsilon )=0$. \\
эффективной среди некоторого набора оценок, если
у нее минимальная дисперсия. \\

MSE, mean squared error,
$MSE(\hat{\theta})=E((\hat{\theta}-\theta)^{2})$ \\


If $X_{i}$ - iid, $N(\mu,\sigma^{2})$, then
$\frac{(n-1)\cdot\hat{\sigma}^{2}}{\sigma^{2}}=
\frac{\sum (X_{i}-\overline{X})^{2}}{\sigma^{2}}$ - $\chi_{(n-1)}$. \\

$\sum \frac{(X_{i}-n p_{i})^{2}}{n p_{i}}\sim \chi_{r-1}^{2}$;
$\sum \frac{(X_{i,j}-n \hat{p}_{i,j})^{2}}{n\hat{p}_{i,j}}\sim
\chi_{(r-1)(c-1)}^{2}$. If $X\sim N(0;1)$ and $K\sim \chi_{n}^{2}$
then $Y=\frac{X}{\sqrt{\frac{K}{n}}}$ is called $t_{n}$.  If
$X_{i}$ - iid $N(\mu,\sigma^2)$, then
$\frac{X_{n}-\mu}{\sqrt{\frac{\hat{\sigma}^2}{n}}}\sim t_{n-1}$. \\


\bibliographystyle{plain}
\bibliography{opit.bib}
\end{document} 