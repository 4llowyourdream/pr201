\documentclass[pdftex,12pt,a4paper]{article}

%%%%%%%%%%%%%%%%%%%%%%%  Загрузка пакетов  %%%%%%%%%%%%%%%%%%%%%%%%%%%%%%%%%%

%\usepackage{showkeys} % показывать метки в готовом pdf 

\usepackage{etex} % расширение классического tex
% в частности позволяет подгружать гораздо больше пакетов, чем мы и займёмся далее

\usepackage{cmap} % для поиска русских слов в pdf
\usepackage{verbatim} % для многострочных комментариев
\usepackage{makeidx} % для создания предметных указателей
\usepackage[X2,T2A]{fontenc}
\usepackage[utf8]{inputenc} % задание utf8 кодировки исходного tex файла
\usepackage{setspace}
\usepackage{amsmath,amsfonts,amssymb,amsthm}
\usepackage{mathrsfs} % sudo yum install texlive-rsfs
\usepackage{dsfont} % sudo yum install texlive-doublestroke
\usepackage{array,multicol,multirow,bigstrut} % sudo yum install texlive-multirow
\usepackage{indentfirst} % установка отступа в первом абзаце главы
\usepackage[british,russian]{babel} % выбор языка для документа
\usepackage{bm}
\usepackage{bbm} % шрифт с двойными буквами
\usepackage[perpage]{footmisc}

% создание гиперссылок в pdf
\usepackage[pdftex,unicode,colorlinks=true,urlcolor=blue,hyperindex,breaklinks]{hyperref} 

% свешиваем пунктуацию 
% теперь знаки пунктуации могут вылезать за правую границу текста, при этом текст выглядит ровнее
\usepackage{microtype}

\usepackage{textcomp}  % Чтобы в формулах можно было русские буквы писать через \text{}

% размер листа бумаги
\usepackage[paper=a4paper,top=13.5mm, bottom=13.5mm,left=16.5mm,right=13.5mm,includefoot]{geometry}

\usepackage{xcolor}

\usepackage[pdftex]{graphicx} % для вставки графики 

\usepackage{float,longtable}
\usepackage{soulutf8}

\usepackage{enumitem} % дополнительные плюшки для списков
%  например \begin{enumerate}[resume] позволяет продолжить нумерацию в новом списке

\usepackage{mathtools}
\usepackage{cancel,xspace} % sudo yum install texlive-cancel

\usepackage{minted} % display program code with syntax highlighting
% требует установки pygments и python 

\usepackage{numprint} % sudo yum install texlive-numprint
\npthousandsep{,}\npthousandthpartsep{}\npdecimalsign{.}

\usepackage{embedfile} % Чтобы код LaTeXа включился как приложение в PDF-файл

\usepackage{subfigure} % для создания нескольких рисунков внутри одного

\usepackage{tikz,pgfplots} % язык для рисования графики из latex'a
\usetikzlibrary{trees} % tikz-прибамбас для рисовки деревьев
\usepackage{tikz-qtree} % альтернативный tikz-прибамбас для рисовки деревьев
\usetikzlibrary{arrows} % tikz-прибамбас для рисовки стрелочек подлиннее

\usepackage{todonotes} % для вставки в документ заметок о том, что осталось сделать
% \todo{Здесь надо коэффициенты исправить}
% \missingfigure{Здесь будет Последний день Помпеи}
% \listoftodos --- печатает все поставленные \todo'шки


% более красивые таблицы
\usepackage{booktabs}
% заповеди из докупентации: 
% 1. Не используйте вертикальные линни
% 2. Не используйте двойные линии
% 3. Единицы измерения - в шапку таблицы
% 4. Не сокращайте .1 вместо 0.1
% 5. Повторяющееся значение повторяйте, а не говорите "то же"



%\usepackage{asymptote} % пакет для рисовки графики, должен идти после graphics
% но мы переходим на tikz :)

%\usepackage{sagetex} % для интеграции с Sage (вероятно тоже должен идти после graphics)

% metapost создает упрощенные eps файлы, которые можно напрямую включать в pdf 
% эта группа команд декларирует, что файлы будут этого упрощенного формата
% если metapost не используется, то этот блок не нужен
\usepackage{ifpdf} % для определения, запускается ли pdflatex или просто латех
\ifpdf
	\DeclareGraphicsRule{*}{mps}{*}{}
\fi
%%%%%%%%%%%%%%%%%%%%%%%%%%%%%%%%%%%%%%%%%%%%%%%%%%%%%%%%%%%%%%%%%%%%%%


%%%%%%%%%%%%%%%%%%%%%%%  Внедрение tex исходников в pdf файл  %%%%%%%%%%%%%%%%%%%%%%%%%%%%%%%%%%
\embedfile[desc={Main tex file}]{\jobname.tex} % Включение кода в выходной файл
\embedfile[desc={title_bor}]{/home/boris/science/tex_general/title_bor_utf8.tex}

%%%%%%%%%%%%%%%%%%%%%%%%%%%%%%%%%%%%%%%%%%%%%%%%%%%%%%%%%%%%%%%%%%%%%%



%%%%%%%%%%%%%%%%%%%%%%%  ПАРАМЕТРЫ  %%%%%%%%%%%%%%%%%%%%%%%%%%%%%%%%%%
\setstretch{1}                          % Межстрочный интервал
\flushbottom                            % Эта команда заставляет LaTeX чуть растягивать строки, чтобы получить идеально прямоугольную страницу
\righthyphenmin=2                       % Разрешение переноса двух и более символов
\pagestyle{plain}                       % Нумерация страниц снизу по центру.
\widowpenalty=300                     % Небольшое наказание за вдовствующую строку (одна строка абзаца на этой странице, остальное --- на следующей)
\clubpenalty=3000                     % Приличное наказание за сиротствующую строку (омерзительно висящая одинокая строка в начале страницы)
\setlength{\parindent}{1.5em}           % Красная строка.
%\captiondelim{. }
\setlength{\topsep}{0pt}
%%%%%%%%%%%%%%%%%%%%%%%%%%%%%%%%%%%%%%%%%%%%%%%%%%%%%%%%%%%%%%%%%%%%%%



%%%%%%%% Это окружение, которое выравнивает по центру без отступа, как у простого center
\newenvironment{center*}{%
  \setlength\topsep{0pt}
  \setlength\parskip{0pt}
  \begin{center}
}{%
  \end{center}
}
%%%%%%%%%%%%%%%%%%%%%%%%%%%%%%%%%%%%%%%%%%%%%%%%%%%%%%%%%%%%%%%%%%%%%%


%%%%%%%%%%%%%%%%%%%%%%%%%%% Правила переноса  слов
\hyphenation{ }
%%%%%%%%%%%%%%%%%%%%%%%%%%%%%%%%%%%%%%%%%%%%%%%%%%%%%%%%%%%%%%%%%%%%%%

\emergencystretch=2em


% DEFS
\def \mbf{\mathbf}
\def \msf{\mathsf}
\def \mbb{\mathbb}
\def \tbf{\textbf}
\def \tsf{\textsf}
\def \ttt{\texttt}
\def \tbb{\textbb}

\def \wh{\widehat}
\def \wt{\widetilde}
\def \ni{\noindent}
\def \ol{\overline}
\def \cd{\cdot}
\def \fr{\frac}
\def \bs{\backslash}
\def \lims{\limits}
\DeclareMathOperator{\dist}{dist}
\DeclareMathOperator{\VC}{VCdim}
\DeclareMathOperator{\card}{card}
\DeclareMathOperator{\sign}{sign}
\DeclareMathOperator{\sgn}{sign}
\DeclareMathOperator{\Tr}{\mbf{Tr}}
\def \xfs{(x_1,\ldots,x_{n-1})}
\DeclareMathOperator*{\argmin}{arg\,min}
\DeclareMathOperator*{\amn}{arg\,min}
\DeclareMathOperator*{\amx}{arg\,max}

\DeclareMathOperator{\Corr}{Corr}
\DeclareMathOperator{\Cov}{Cov}
\DeclareMathOperator{\Var}{Var}
\DeclareMathOperator{\corr}{Corr}
\DeclareMathOperator{\cov}{Cov}
\DeclareMathOperator{\var}{Var}
\DeclareMathOperator{\bin}{Bin}
\DeclareMathOperator{\Bin}{Bin}
\DeclareMathOperator{\rang}{rang}
\DeclareMathOperator*{\plim}{plim}
\DeclareMathOperator{\med}{med}


\providecommand{\iff}{\Leftrightarrow}
\providecommand{\hence}{\Rightarrow}

\def \ti{\tilde}
\def \wti{\widetilde}

\def \mA{\mathcal{A}}
\def \mB{\mathcal{B}}
\def \mC{\mathcal{C}}
\def \mE{\mathcal{E}}
\def \mF{\mathcal{F}}
\def \mH{\mathcal{H}}
\def \mL{\mathcal{L}}
\def \mN{\mathcal{N}}
\def \mU{\mathcal{U}}
\def \mV{\mathcal{V}}
\def \mW{\mathcal{W}}


\def \R{\mbb R}
\def \N{\mbb N}
\def \Z{\mbb Z}
\def \P{\mbb{P}}
\def \p{\mbb{P}}
\newcommand{\E}{\mathbb{E}}
\def \D{\msf{D}}
\def \I{\mbf{I}}

\def \a{\alpha}
\def \b{\beta}
\def \t{\tau}
\def \dt{\delta}
\newcommand{\e}{\varepsilon}
\def \ga{\gamma}
\def \kp{\varkappa}
\def \la{\lambda}
\def \sg{\sigma}
\def \sgm{\sigma}
\def \tt{\theta}
\def \ve{\varepsilon}
\def \Dt{\Delta}
\def \La{\Lambda}
\def \Sgm{\Sigma}
\def \Sg{\Sigma}
\def \Tt{\Theta}
\def \Om{\Omega}
\def \om{\omega}

%\newcommand{\p}{\partial}
\newcommand{\PP}{\mathbb{P}}

\def \ni{\noindent}
\def \lq{\glqq}
\def \rq{\grqq}
\def \lbr{\linebreak}
\def \vsi{\vspace{0.1cm}}
\def \vsii{\vspace{0.2cm}}
\def \vsiii{\vspace{0.3cm}}
\def \vsiv{\vspace{0.4cm}}
\def \vsv{\vspace{0.5cm}}
\def \vsvi{\vspace{0.6cm}}
\def \vsvii{\vspace{0.7cm}}
\def \vsviii{\vspace{0.8cm}}
\def \vsix{\vspace{0.9cm}}
\def \VSI{\vspace{1cm}}
\def \VSII{\vspace{2cm}}
\def \VSIII{\vspace{3cm}}

\newcommand{\bls}[1]{\boldsymbol{#1}}
\newcommand{\bsA}{\boldsymbol{A}}
\newcommand{\bsH}{\boldsymbol{H}}
\newcommand{\bsI}{\boldsymbol{I}}
\newcommand{\bsP}{\boldsymbol{P}}
\newcommand{\bsR}{\boldsymbol{R}}
\newcommand{\bsS}{\boldsymbol{S}}
\newcommand{\bsX}{\boldsymbol{X}}
\newcommand{\bsY}{\boldsymbol{Y}}
\newcommand{\bsZ}{\boldsymbol{Z}}
\newcommand{\bse}{\boldsymbol{e}}
\newcommand{\bsq}{\boldsymbol{q}}
\newcommand{\bsy}{\boldsymbol{y}}
\newcommand{\bsbeta}{\boldsymbol{\beta}}
\newcommand{\fish}{\mathrm{F}}
\newcommand{\Fish}{\mathrm{F}}
\renewcommand{\phi}{\varphi}
\newcommand{\ind}{\mathds{1}}
\newcommand{\inds}[1]{\mathds{1}_{\{#1\}}}
\renewcommand{\to}{\rightarrow}
\newcommand{\sumin}{\sum\limits_{i=1}^n}
\newcommand{\ofbr}[1]{\bigl( \{ #1 \} \bigr)}     % Например, вероятность события. Большие круглые, нормальные фигурные скобки вокруг аргумента
\newcommand{\Ofbr}[1]{\Bigl( \bigl\{ #1 \bigr\} \Bigr)} % Например, вероятность события. Больше больших круглые, большие фигурные скобки вокруг аргумента
\newcommand{\oeq}{{}\textcircled{\raisebox{-0.4pt}{{}={}}}{}} % Равно в кружке
\newcommand{\og}{\textcircled{\raisebox{-0.4pt}{>}}}  % Знак больше в кружке

% вместо горизонтальной делаем косую черточку в нестрогих неравенствах
\renewcommand{\le}{\leqslant}
\renewcommand{\ge}{\geqslant}
\renewcommand{\leq}{\leqslant}
\renewcommand{\geq}{\geqslant}


\newcommand{\figb}[1]{\bigl\{ #1  \bigr\}} % большие фигурные скобки вокруг аргумента
\newcommand{\figB}[1]{\Bigl\{ #1  \Bigr\}} % Больше больших фигурные скобки вокруг аргумента
\newcommand{\parb}[1]{\bigl( #1  \bigr)}   % большие скобки вокруг аргумента
\newcommand{\parB}[1]{\Bigl( #1  \Bigr)}   % Больше больших круглые скобки вокруг аргумента
\newcommand{\parbb}[1]{\biggl( #1  \biggr)} % большие-большие круглые скобки вокруг аргумента
\newcommand{\br}[1]{\left( #1  \right)}    % круглые скобки, подгоняемые по размеру аргумента
\newcommand{\fbr}[1]{\left\{ #1  \right\}} % фигурные скобки, подгоняемые по размеру аргумента
\newcommand{\eqdef}{\mathrel{\stackrel{\rm def}=}} % знак равно по определению
\newcommand{\const}{\mathrm{const}}        % const прямым начертанием
\newcommand{\zdt}[1]{\textit{#1}}
\newcommand{\ENG}[1]{\foreignlanguage{british}{#1}}
\newcommand{\ENGs}{\selectlanguage{british}}
\newcommand{\RUSs}{\selectlanguage{russian}}
\newcommand{\iid}{\text{i.\hspace{1pt}i.\hspace{1pt}d.}}

\newdimen\theoremskip
\theoremskip=0pt
\newenvironment{note}{\par\vskip\theoremskip\textbf{Замечание.\xspace}}{\par\vskip\theoremskip}
\newenvironment{hint}{\par\vskip\theoremskip\textbf{Подсказка.\xspace}}{\par\vskip\theoremskip}
\newenvironment{ist}{\par\vskip\theoremskip Источник:\xspace}{\par\vskip\theoremskip}

\newcommand*{\tabvrulel}[1]{\multicolumn{1}{|c}{#1}}
\newcommand*{\tabvruler}[1]{\multicolumn{1}{c|}{#1}}

\newcommand{\II}{{\fontencoding{X2}\selectfont\CYRII}}   % I десятеричное (английская i неуместна)
\newcommand{\ii}{{\fontencoding{X2}\selectfont\cyrii}}   % i десятеричное
\newcommand{\EE}{{\fontencoding{X2}\selectfont\CYRYAT}}  % ЯТЬ
\newcommand{\ee}{{\fontencoding{X2}\selectfont\cyryat}}  % ять
\newcommand{\FF}{{\fontencoding{X2}\selectfont\CYROTLD}} % ФИТА
\newcommand{\ff}{{\fontencoding{X2}\selectfont\cyrotld}} % фита
\newcommand{\YY}{{\fontencoding{X2}\selectfont\CYRIZH}}  % ИЖИЦА
\newcommand{\yy}{{\fontencoding{X2}\selectfont\cyrizh}}  % ижица

%%%%%%%%%%%%%%%%%%%%% Определение разрядки разреженного текста и задание красивых многоточий
\sodef\so{}{.15em}{1em plus1em}{.3em plus.05em minus.05em}
\newcommand{\ldotst}{\so{...}}
\newcommand{\ldotsq}{\so{?\hbox{\hspace{-0.61ex}}..}}
\newcommand{\ldotse}{\so{!..}}
%%%%%%%%%%%%%%%%%%%%%%%%%%%%%%%%%%%%%%%%%%%%%%%%%%%%%%%%%%%%%%%%%%%%%%

%%%%%%%%%%%%%%%%%%%%%%%%%%%%% Команда для переноса символов бинарных операций
\def\hm#1{#1\nobreak\discretionary{}{\hbox{$#1$}}{}}
%%%%%%%%%%%%%%%%%%%%%%%%%%%%%%%%%%%%%%%%%%%%%%%%%%%%%%%%%%%%%%%%%%%%%%

\setlist[enumerate,1]{label=\arabic*., ref=\arabic*, partopsep=0pt plus 2pt, topsep=0pt plus 1.5pt,itemsep=0pt plus .5pt,parsep=0pt plus .5pt}
\setlist[itemize,1]{partopsep=0pt plus 2pt, topsep=0pt plus 1.5pt,itemsep=0pt plus .5pt,parsep=0pt plus .5pt}

% Эти парни затем, если вдруг не захочется управлять списками из-под уютненького enumitem
% или если будет жизненно важно, чтобы в списках были именно русские буквы.
%\setlength{\partopsep}{0pt plus 3pt}
%\setlength{\topsep}{0pt plus 2pt}
%\setlength{\itemsep}{0 plus 1pt}
%\setlength{\parsep}{0 plus 1pt}

%на всякий случай пока есть
%теоремы без нумерации и имени
%\newtheorem*{theor}{Теорема}

%"Определения","Замечания"
%и "Гипотезы" не нумеруются
%\newtheorem*{defin}{Определение}
%\newtheorem*{rem}{Замечание}
%\newtheorem*{conj}{Гипотеза}

%"Теоремы" и "Леммы" нумеруются
%по главам и согласованно м/у собой
%\newtheorem{theorem}{Теорема}
%\newtheorem{lemma}[theorem]{Лемма}

% Утверждения нумеруются по главам
% независимо от Лемм и Теорем
%\newtheorem{prop}{Утверждение}
%\newtheorem{cor}{Следствие} 

%\usepackage{showkeys} % показывать метки

% специальная штука под задачник
% создает команды:
% \problem{ текст задачи }
% \solution{ текст решения }
% \problemonly  - после этой команды будут печататься только \problem{} и \problemtext{}
% \solutiononly - после этой команды будут печататься только \solution{} и \solutiontext{}
% \problemandsolution - после этой команды печатается все
% \secsolution - задает новую (виртуальную) секцию для решений

% может потребоваться %\addtocounter{secsolution}{число глав без задач решений, не прогнанных через problemonly}

% как работать
% файл с решениями отдельной главы должен выглядеть так:
% \problem{ dddd} \solution{ddddddd}
% \problem{df sldk} \solution{ dfssd}

% главный файл может выглядеть двумя способами:

% Способ 1. (для решений контрольной, рядом задачи и ответы)
% \problemandsolution
% \input{file with problems}

% Способ 2. (для задачника, сначала все задачи, затем все ответы)
% \problemonly
% \input{file with problems}

% \solutiononly
% \input{file with problems}

% ААААААААААААААААААААААА надо делать!!!!
% Способ 3. - основной (вариант способа 2)

% \problemonly2
% \input{file with problems}

%\solutiononly - эта команда сама сделает все!




% файл с задачами:
% \section{Первая}
% \problem{ dddd} \solution{ddddddd}
% \problem{df sldk} \solution{ dfssd}
% \problemtext{Этот текст не будет напечатан после solutiononly}
% \section{Вторая}
% \problem{ ааа} \solution{dыва}
% \problem{ыавыв} \solution{ ыва}
% \solutiontext{Этот текст не будет напечатан после problemonly}



% начало кода:

\let\oldsection\section % сохраняем команду \section, т.к. мы ее переопределим
\let\oldsubsection\subsection % сохраняем команду \subsection, т.к. мы ее переопределим

\newcommand{\restoresection}{ % команда для восстановления \section \subsection
\renewcommand{\section}[1]{\oldsection{##1}}
\renewcommand{\subsection}[1]{\oldsubsection{##1}}
}

\newcounter{problem}[section]
%создаем новый счетчик "problem",
% будет автоматом сбрасываться на 0 при старте нового раздела
% при создании счетчик сам встанет на 0

\newcounter{secsolution}
% - это номер секции решаемой задачи (поскольку решения идут в одной секции, то номер секции надо менять в ручную)
\newcounter{solution}[secsolution]
% - это номер решаемой задачи, сам сбрасывается при увеличении secsolution на 1



\renewcommand{\thesecsolution}{\arabic{secsolution}}
% команда \thesecsolution просто выводит номер secsolution

\newcommand{\newsecsolution}{
\stepcounter{secsolution} % без создания ссылки увеличит secsolution на 1 со сбросом подчиненного счетчика
}
% команда \newsecsolution увеличит номер секции на 1 и установит номер решения внутри секции равным 0

\renewcommand{\theproblem}{\thesection.\arabic{problem}.}
\renewcommand{\thesolution}{\thesecsolution.\arabic{solution}.}
% обновляем команду \theproblem - она должна выводить номер секции и номер задачи внутри секции
% почему обновляем? - потому, что она создалась при создании счетчика problem

\newcommand{\problem}[1]{}
\newcommand{\solution}[1]{}
% создаем команды \problem, \solution с одним аргументом, которые ничего не делает
% ниже они будут переопределены


\newcommand{\problemtext}[1]{}
\newcommand{\solutiontext}[1]{}
% эти две команды будут выводить текст, заложенный внутри них только внутри соответствующей секции, в другой - ничего не будет делать
% в отличие от этой команды \problem \solution делают ссылки, слово "задача" и пр.


\newcommand{\problemonly}{
% эта команда переопределяет команду \problem

\setcounter{problem}{0}
\setcounter{solution}{0}
\setcounter{secsolution}{0}

\renewcommand{\problemtext}[1]{##1}
\renewcommand{\solutiontext}[1]{}


\renewcommand{\section}[1]{\oldsection{##1}\newsecsolution}
\renewcommand{\subsection}[1]{\oldsubsection{##1}}


\renewcommand{\problem}[1]{
\refstepcounter{problem}
% \phantomsection % создаем точку привязки для команды \label % не нужна, т.к. есть refstepcounter
\vspace{0.5ex plus 0.2ex minus 0.2ex}
Задача
\hyperref[s\theproblem]{\theproblem} % гиперссылка на метку "s1.1."
\label{p\theproblem} % метка "p1.1."
\par ##1}
\renewcommand{\solution}[1]{}
}

\newcommand{\solutiononly}{
% эта команда переопределяет команду \solution

\setcounter{problem}{0}
\setcounter{solution}{0}
\setcounter{secsolution}{0}

\renewcommand{\problemtext}[1]{}
\renewcommand{\solutiontext}[1]{##1}

\renewcommand{\section}[1]{\newsecsolution} % можно сюда чего-то добавить, чтобы решения отделялись как-то по секциям
\renewcommand{\subsection}[1]{}

\renewcommand{\problem}[1]{}
\renewcommand{\solution}[1]{
\refstepcounter{solution}
% \phantomsection
\hyperref[p\thesolution]{\thesolution} \label{s\thesolution}
##1}}



\newcommand{\problemandsolution}{
% эта команда переопределяет команды \solution, \problem

\setcounter{problem}{0}
\setcounter{solution}{0}
\setcounter{secsolution}{0}

\renewcommand{\problemtext}[1]{##1}
\renewcommand{\solutiontext}[1]{##1}

\renewcommand{\section}[1]{\oldsection{##1}\newsecsolution}
\renewcommand{\subsection}[1]{\oldsubsection{##1}}

\renewcommand{\problem}[1]{
\refstepcounter{problem}
% \phantomsection % создаем точку привязки для команды \label
Задача
\hyperref[s\theproblem]{\theproblem} % гиперссылка на метку "s1.1."
\label{p\theproblem} % метка "p1.1."
\par ##1}
\renewcommand{\solution}[1]{
\refstepcounter{solution}
% \phantomsection
\hyperref[p\thesolution]{\thesolution} \label{s\thesolution}
##1}
}




%\title{Задачи по элементарной теории вероятностей и матстатистике}
%\author{Составитель: Борис Демешев, boris.demeshev@gmail.com}
%\date{\today}

\begin{document}


\parindent=0 pt % Отступ равен 0

Коротко о некоторых параметрических тестах. (ver 01.08.06). \\
Текст можно скачать на www.xion.ru (учеба-2 курс-теория
вероятностей). \\
Вопросы/комментарии/предложения/найденные ошибки можно смело
отправлять на
roah@yandex.ru (Борису Демешеву) \\


\textbf{Quote} \\
Во избежание несчастных случаев торпеды хранить так, чтобы верхняя
их часть находилась внизу, а нижняя наверху. Дабы персонал не
путал верх с низом, на верхней части торпеды сделать надпись
<<верх>>
\begin{flushright}
\emph{из инструкции} \\
\end{flushright}


\textbf{Обозначения} \\

$\bar{X}=\frac{X_{1}+...+X_{n}}{n}$ - несмещенная оценка математического ожидания \\
$\hat{\sigma}^{2}=\frac{\sum_{i}(X_{i}-\bar{X})^{2}}{n-1}$ - несмещенная оценка дисперсии \\
$\hat{v}^{2}=\frac{\sum_{i}(X_{i}-\bar{X})^{2}}{n}$ - оценка
дисперсии,
при больших $n$ слабо отличается от $\hat{\sigma}^{2}$ \\

\textbf{Случай 0-1}\\
Часто рассматривается случай, когда $X_{i}$ принимают только
значения 0 и 1. Значение 1 с вероятностью $p$, значение 0 с
вероятностью $q=1-p$. \\
В этом случае:\\
1. $E(X_{i})=p$, $Var(X_{i})=pq=p(1-p)$. \\
2. Вместо $\bar{X}$ часто используется обозначение $\hat{p}$ \\
3. $\hat{v}^{2}=\hat{p}(1-\hat{p})$ \\
Доказательство 3: \\
$\sum(X_{i}-\bar{X})^{2}=\sum(X_{i}^{2}+\bar{X}^{2}-2X_{i}\bar{X})= \\
= \sum X_{i}+n\bar{X}^{2}-2\bar{X}\sum
X_{i}=n\bar{X}+n\bar{X}^{2}-2n\bar{X}^{2}=n\bar{X}-n\bar{X}^{2}=n\hat{p}(1-\hat{p})$
\\

\textbf{Definitions} \\
Определение 1. \\
Распределение случайной величины $K$ называется $\chi^{2}$
распределением с $n$ степенями свободы, если величину можно
представить в виде $K=Z_{1}^{2}+...+Z_{n}^{2}$, где $Z_{i}$ iid,
$N(0;1)$ \\
Т.е. есть $\chi^{2}$-распределение номер 1, есть
$\chi^{2}$-распределение номер 2 и т.д.

Определение 2. \\
Распределение случайной величины $T$ называется $t$-распределением
с $n$ степенями свободы, если величину можно представить в виде
$T=\frac{Z}{\sqrt{\frac{K}{n}}}$, где $Z \sim N(0;1)$, $K\sim
\chi_{n}^{2}$, $Z$ и $K$ независимы \\

Определение 3. \\
Распределение случайной величины $F$ называется $F$-распределением
с $n$, $k$ степенями свободы, если величину можно представить в
виде $F=\frac{X/n}{Y/k}$, где $X\sim \chi_{n}^{2}$ и $Y\sim
\chi_{k}^{2}$ и $X$ и $Y$ независимы \\

\textbf{Одна выборка} \\

Асимптотический результат. \\
Если: \\
1. $X_{i}$ - независимы, одинаково распределены, \\
2. $E(X_{i})=\mu$, $Var(X_{i})=\sigma^{2}$, \\
То: \\
1. $Z=\frac{\bar{X}_{n}-\mu}{\sqrt{\frac{\sigma^{2}}{n}}}$ имеет
асимптотически  (т.е. при $n\to\infty$) нормальное распределение $N(0;1)$ \\
2. $Z=\frac{\bar{X}_{n}-\mu}{\sqrt{\frac{\hat{\sigma}^{2}}{n}}}$
имеет
асимптотически  (т.е. при $n\to\infty$) нормальное распределение $N(0;1)$ \\
Примечание: \\
В пункте 2 вместо $\hat{\sigma}^{2}$ можно взять любую другую
состоятельную оценку дисперсии, например $\hat{v}^{2}$ \\

Точный результат. \\
Добавив дополнительное условие нормальности отдельных $X_{i}$
получаем: \\
Если: \\
1. $X_{i}$ - независимы, одинаково распределены, \\
2. $E(X_{i})=\mu$, $Var(X_{i})=\sigma^{2}$, \\
3. $X_{i}$ - нормально распределены, \\
То: \\
1. $Z=\frac{\bar{X}_{n}-\mu}{\sqrt{\frac{\sigma^{2}}{n}}}$ имеет
 нормальное распределение $N(0;1)$ \\
2. $Z=\frac{\bar{X}_{n}-\mu}{\sqrt{\frac{\hat{\sigma}^{2}}{n}}}$
имеет
 $t$-распределение с $n-1$ степенью свободы \\
3. $Q=\frac{\hat{(n-1)\sigma}^{2}}{\sigma^{2}}$ имеет $\chi^{2}$
распределение с $n-1$ степенью свободы. \\

Частный случай асимптотического результата. \\
Отдельно рассмотрим случай 0-1: \\
Если: \\
1. $X_{i}$ - независимы, одинаково распределены, \\
2. $X_{i}$ принимает значение 1 с вероятностью $p$ и значение 0 c вероятностью $(1-p)$ \\
То: \\
0. $E(X_{i})=p$, $Var(X_{i})=p(1-p)$,
$\hat{v}^{2}=\hat{p}(1-\hat{p})$, вместо $\bar{X}$ используем
$\hat{p}$  \\
1. $Z=\frac{\hat{p}-p}{\sqrt{\frac{p(1-p)}{n}}}$
имеет
 асимптотически  (т.е. при $n\to\infty$) нормальное распределение $N(0;1)$ \\
2. $Z=\frac{\hat{p}-p}{\sqrt{\frac{\hat{p}(1-\hat{p})}{n}}}$ имеет
 асимптотически  (т.е. при $n\to\infty$) нормальное распределение $N(0;1)$ \\

\textbf{Две выборки} \\


Если: \\
1. $X_{i}$ - одинаково распределены между собой, \\
2. $E(X_{i})=\mu_{x}$, $Var(X_{i})=\sigma_{x}^{2}$, \\
3. $Y_{i}$ - одинаково распределены между собой, \\
4. $E(Y_{i})=\mu_{y}$, $Var(Y_{i})=\sigma_{y}^{2}$, \\
5. Все случайные величины $X_{i}$ и $Y_{j}$ независимы. \\
То: \\
1. $\frac{\bar{X}-\bar{Y}-(\mu_{x}-\mu_{y})} {\sqrt
    {
    \frac{\sigma_{x}^{2}}{n_{x}}+\frac{\sigma_{y}^{2}}{n_{y}}
    }
}$ имеет
 асимптотически  (т.е. при $n\to\infty$) нормальное распределение $N(0;1)$ \\
2. $\frac{\bar{X}-\bar{Y}-(\mu_{x}-\mu_{y})} {\sqrt
    {
    \frac{\hat{\sigma}_{x}^{2}}{n_{x}}+\frac{\hat{\sigma}_{y}^{2}}{n_{y}}
    }
}$ имеет
 асимптотически  (т.е. при $n\to\infty$) нормальное распределение $N(0;1)$ \\
3. $\frac{\bar{X}-\bar{Y}-(\mu_{x}-\mu_{y})} { \sqrt{
    \frac{(n_{x}-1)\hat{\sigma}_{x}^{2}+(n_{y}-1)\hat{\sigma}_{y}^{2}}{n_{x}+n_{y}-2}
    \cdot
    \left(\frac{1}{n_{x}}+\frac{1}{n_{y}}\right)
    }
}$ имеет асимптотически  (т.е. при $n\to\infty$) нормальное распределение $N(0;1)$ \\



Если: \\
1. $X_{i}$ - одинаково распределены между собой, \\
2. $E(X_{i})=\mu_{x}$, $Var(X_{i})=\sigma_{x}^{2}$, \\
3. $Y_{i}$ - одинаково распределены между собой, \\
4. $E(Y_{i})=\mu_{y}$, $Var(Y_{i})=\sigma_{y}^{2}$, \\
5. Все случайные величины $X_{i}$ и $Y_{j}$ независимы. \\
6. Все случайные величины $X_{i}$ и $Y_{j}$ нормально распределены \\
То: \\
1. $t=\frac{\bar{X}-\bar{Y}-(\mu_{x}-\mu_{y})} { \sqrt{
    \frac{(n_{x}-1)\hat{\sigma}_{x}^{2}+(n_{y}-1)\hat{\sigma}_{y}^{2}}{n_{x}+n_{y}-2}
    \cdot
    \left(\frac{1}{n_{x}}+\frac{1}{n_{y}}\right)
    }
}$ имеет $t$-распределение c $(n_{x}+n_{y}-2)$ степенью свободы \\
2.
$F=\frac{\hat{\sigma}_{x}^{2}/\sigma_{x}^{2}}{\hat{\sigma}_{y}^{2}/\sigma_{y}^{2}}$
имеет $F$ распределение c $(n_{x}-1)$ и $(n_{y}-1)$ степенями
свободы. \\


\textbf{Summary} \\
Для проверки гипотез о среднем:\\
Если точный закон распределения $X_{i}$ неизвестен или доподлинно
известно, что он отличается от нормального (случай 0-1), то
 используется асимптотически нормальное распределение. Т.е. требуется
 большое $n$. \\
Если известно, что $X_{i}$ нормальны, то (при любых $n$): \\
а) если известна оценка дисперсии $\hat{\sigma}^{2}$ - используем
$t$. При больших $n$ $t$-распределение перестанет отличаться от нормального. \\
б) если известна точная дисперсия (что бывает редко) - используем
нормальное распределение \\
в) можно проверять гипотезы о дисперсии с помощью $\chi^{2}$ \\



\textbf{Про} $\chi^{2}$ \textbf{распределение} \\

Проверка гипотезы о соответствии наблюдаемых частот заданному
закону распределения \\

Предполагаемые частоты: $p_{i}$ \\
Выборочные частоты: $\hat{p}_{i}$ \\

Статистика (имеет $\chi^{2}$ распределение с $(c-1)$ степенями
свободы):
$K=\sum \frac{(n\hat{p}_{i}-np_{i})^{2}}{np_{i}}$ \\
Можно, конечно, считать по иному: \\
$K=n \sum \frac{(\hat{p}_{i}-p_{i})^{2}}{p_{i}}$ \\
Или, если напрячь немного арифметический мускул: \\
$K=n \left(\sum{\frac{\hat{p}_{i}^{2}}{p_{i}}} - 1 \right)$ \\


Проверка гипотезы о независимости двух признаков \\

Выборочные частоты: $\hat{p}_{ij}$ \\
Частоты, рассчитанные в предположении независимости: $p_{ij}$ \\

Способ расчета эталонных <<независимых>> частот: \\
Сначала считаем оценки вероятностей по каждому признаку: \\
$p_{i\cdot}=\sum_{j}\hat{p}_{ij}$ \\
$p_{\cdot j}=\sum_{i}\hat{p}_{ij}$ \\
Если события $A$ и $B$ независимы, то $P(A\cap B)=P(A)P(B)$:\\
$p_{ij}=p_{\cdot j}p_{i\cdot}$ \\

Статистика (имеет $\chi^{2}$ распределение с $(r-1)(c-1)$
степенями свободы):
$K=\sum \frac{(n\hat{p}_{ij}-np_{ij})^{2}}{np_{ij}}$ \\
Или:
$K=n \sum \frac{(\hat{p}_{ij}-p_{ij})^{2}}{p_{ij}}$ \\
Или:
$K=n \left(\sum{\frac{\hat{p}_{ij}^{2}}{p_{ij}}} - 1 \right)$ \\




- Сколько степеней свободы? \\
- Столько, сколько вероятностей можно расставить <<от фонаря>> при соблюдении ограничений\\
В первом случае: \\
В обеих табличках (для $p_{i}$ и $\hat{p}_{i}$) имеется $c$ ячеек.
\\
Действует единственное общее ограничение $\sum p_{i}=1$ (и $\sum
\hat{p}_{i}=1$) \\
Значит, $(c-1)$ вероятность может быть любой, а последняя
считается исходя из того, что сумма вероятностей равна 1. \\

Во втором случае: \\
В обеих табличках (для $p_{ij}$ и $\hat{p}_{ij}$) имеется $r\cdot
c$ ячеек \\
Общие ограничения: сумма вероятностей по каждой строке и по
каждому столбцу должна быть одинакова для
$p_{ij}$ и для $\hat{p}_{ij}$. \\
Значит в каждом столбце кроме последнего можно поставить <<от
фонаря>> $r-1$ число. Последний столбец рассчитается сам собой.
Следовательно, получается $(r-1)(c-1)$ степень свободы. \\

Большинство авторов использует в формуле не частоты, а количества.
Мне этот подход кажется менее удачным, потому, что нужно
объяснять, что такое <<эталонное>> количество. При этом более
туманным (imho) становится вычисление эталонного количества для
проверки гипотезы о независимости признаков. \\
Переход от вероятностей к количествам прозрачен: \\
Эталонное количество: $E_{i}=np_{i}$ \\
Выборочное количество: $V_{i}=n\hat{p}_{i}$ \\
Соответственно меняются формулы. \\



\textbf{Some proofs}


Теорема 1. \\
Пусть $X_{i}$ iid, $N(\mu;1)$. Величину
$\sum_{i=1}^{n}(X_{i}-\bar{X}_{n})^{2}$ можно представить в виде
$\sum_{i=1}^{n-1}Z_{i}^{2}$, где $Var(Z_{i})=1$ и
$Cov(Z_{i},Z_{j})=0$ для $i \ne j$. \\

Доказательство: \\
В качестве $Z_{k}$ возьмем $\frac{\sum_{i=1}^{k}X_{i}-kX_{k+1}}{\sqrt{n(n+1)}}$ \\
То, что $Var(Z_{i})=1$ и $Cov(Z_{i},Z_{j})=0$ проверяется "в лоб".
\\
Остается убедится в том, что $\sum_{i=1}^{n}(X_{i}-\bar{X}_{n})^{2}=\sum_{i=1}^{n-1}Z_{i}^{2}$ \\
Доказательство (предложила Алина Дурдыева) \\

Для $n=1$ формула верна. \\
Пусть для некоторого $n$ формула верна, т.е.
$\sum_{i}^{n}(X_{i}-\bar{X}_{n})^{2}=\sum_{i}^{n-1}Z_{i}^{2}$ \\

$\sum_{i=1}^{n+1} (X_{i}-\bar{X}_{n+1})^{2}=
\sum_{i=1}^{n+1}(X_{i}-\frac{X_{n+1}+n\bar{X}_{n}}{n+1})^{2}=
\sum_{i=1}^{n+1}(X_{i}-\bar{X}_{n}-\frac{X_{n+1}-\bar{X}_{n}}{n+1})^{2}=\\$
$=\sum_{i=1}^{n}(X_{i}-\bar{X}_{n}-\frac{X_{n+1}-\bar{X}_{n}}{n+1})^{2}+
(X_{n+1}-\bar{X}_{n}-\frac{X_{n+1}-\bar{X}_{n}}{n+1})^{2}$ \\

Обозначим $b=\frac{X_{n+1}-\bar{X}_{n}}{n+1}$. \\

$\sum_{i=1}^{n+1} (X_{i}-\bar{X}_{n+1})^{2}=
\sum_{i=1}^{n}(X_{i}-\bar{X}_{n}-b)^{2}+((n+1)b-b))^2= \\
=\left[\sum_{i=1}^{n}(X_{i}-\bar{X}_{n})^{2}+nb^{2}-2b\sum_{i=1}^{n}(X_{i}-\bar{X}_{n})\right]
+n^{2}b^{2}= \\
=\sum_{i=1}^{n}(X_{i}-\bar{X}_{n})^{2}+b^{2}n(n+1)$ \\
Осталось вспомнить, что
$Z_{n}=\frac{\sum_{i=1}^{n}X_{i}-nX_{n+1}}{\sqrt{n(n+1)}}$. \\
Получаем, что $\sum_{i=1}^{n+1}
(X_{i}-\bar{X}_{n+1})^{2}=\sum_{i=1}^{n-1}Z_{i}^{2}+Z_{n}^{2}$. \\


Следствие. \\
Пусть $X_{i}$ iid, $N(\mu;\sigma^{2})$.
Из теоремы и определения нетрудно видеть, что: \\
$\frac{\sum_{i}^{n}(X_{i}-\bar{X}_{n})^{2}}{\sigma^{2}}\sim
\chi_{n-1}^{2}$ \\
Если $\hat{\sigma}^{2}=\frac{\sum(X_{i}-\bar{X})^{2}}{n-1}$, то
$\frac{(n-1)\hat{\sigma}^{2}}{\sigma^{2}}\sim
\chi_{n-1}^{2}$ \\


Теорема 2. \\
Если $X_{i}$ iid, $N(\mu_{x};\sigma^{2})$ и $Y_{i}$ iid,
$N(\mu_{y};\sigma^{2})$. \\
То: $\frac{\bar{X}-\bar{Y}-(\mu_{x}-\mu_{y})} { \sqrt{
    \frac{(n_{x}-1)\hat{\sigma}_{x}^{2}+(n_{y}-1)\hat{\sigma}_{y}^{2}}{n_{x}+n_{y}-2}
    \cdot
    \left(\frac{1}{n_{x}}+\frac{1}{n_{y}}\right)
    }
}\sim t_{n_{x}+n_{y}-2}$ \\
Доказательство: \\
Вспомним, что $\frac{\sum (X_{i}-\bar{X})^{2}}{\sigma^{2}}\sim
\chi_{n_{x}-1}^{2}$ и $\frac{\sum
(Y_{j}-\bar{Y})^{2}}{\sigma^{2}}\sim
\chi_{n_{y}-1}^{2}$. \\
Следовательно, $\frac{\sum (X_{i}-\bar{X})^{2}+\sum
(Y_{j}-\bar{Y})^{2}}{\sigma^{2}}\sim
\chi_{n_{x}+n_{y}-2}^{2}$. \\
Также известно, что $Var(\bar{X}-\bar{Y})=\sigma^{2}\cdot
(\frac{1}{n_{x}}+\frac{1}{n_{y}})$. \\
Следовательно,
$\frac{\bar{X}-\bar{Y}-(\mu_{x}-\mu_{y})}{\sqrt{\sigma^{2}\cdot
(\frac{1}{n_{x}}+\frac{1}{n_{y}})}}\sim N(0;1)$. \\
По определению, $t_{k}=\frac{N(0;1)}{\sqrt{\frac{\chi^{k}}{k}}}$
\\
Получаем, что:
$$\frac{\bar{X}-\bar{Y}-(\mu_{x}-\mu_{y})} {
\sqrt{
    \frac{(n_{x}-1)\hat{\sigma}_{x}^{2}+(n_{y}-1)\hat{\sigma}_{y}^{2}}{n_{x}+n_{y}-2}
    \cdot
    \left(\frac{1}{n_{x}}+\frac{1}{n_{y}}\right)
    }
}\sim t_{n_{x}+n_{y}-2}$$


Можно получить более простой асимптотический результат. \\
Заметим, что $\frac{\bar{X}-\bar{Y}-(\mu_{x}-\mu_{y})}
{\sqrt{Var(\bar{X}-\bar{Y})}}\sim N(0;1)$. \\
$Var(\bar{X}-\bar{Y})=Var(\bar{X})+Var(\bar{Y})
=\frac{\sigma_{x}^{2}}{n_{x}}+\frac{\sigma_{y}^{2}}{n_{y}}$. \\
Заменим настоящие (неизвестные) дисперсии, на их оценки: \\
$\hat{\sigma}_{(\bar{X}-\bar{Y})}^{2}=
\frac{\hat{\sigma}_{x}^{2}}{n_{x}}+\frac{\hat{\sigma}_{y}^{2}}{n_{y}}$. \\
Получим асимптотический результат: \\
$\frac{\bar{X}-\bar{Y}-(\mu_{x}-\mu_{y})} {\sqrt
    {
    \frac{\hat{\sigma}_{x}^{2}}{n_{x}}+\frac{\hat{\sigma}_{y}^{2}}{n_{y}}
    }
}\rightarrow N(0;1)$. \\ \\


\textbf{Про упрямство и эконометрику} \\
Я упрямо обозначаю несмещенную оценку дисперсии знаком
$\hat{\sigma}^{2}$, а не $s^{2}$, как большинство европейских
авторов. \\
Почему? \\
В курсе эконометрики решается задача нахождения $\hat{\beta}_{1}$
и $\hat{\beta}_{2}$, если \\
$Y_{i}=\beta_{1}+\beta_{2}X_{i}+u_{i}$ \\
Попутно находится оценка дисперсии $u_{i}$. Она равна
$\hat{\sigma}^{2}=\frac{RSS}{n-2}$. \\
Там оценивается два параметра, поэтому $(n-2)$. А здесь мы
оцениваем один параметр - мат. ожидание $Y_{i}$: \\
$Y_{i}=\beta_{1}+u_{i}$ \\
МНК даст $\hat{\beta}_{1}=\bar{Y}$, а оценка дисперсии будет
$\hat{\sigma}^{2}=\frac{RSS}{n-1}$



\end{document}
