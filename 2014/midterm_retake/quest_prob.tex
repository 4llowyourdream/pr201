
\element{probability1}{
  \begin{questionmult}{1}
Случайным образом выбирается семья с двумя детьми. Событие $A$ --- в семье старший ребенок --- мальчик,  событие $B$ --- в семье только один из детей --- мальчик, событие $C$ --- в семье хотя бы один из детей --- мальчик.  Вероятность $\P(C)$ равна
\begin{multicols}{3}
   \begin{choices}
      \correctchoice{$3/4$}
      \wrongchoice{$1/4$}
      \wrongchoice{$1/2$}
      \wrongchoice{$1$}
      \wrongchoice{$2/3$}
      
       \end{choices}
  \end{multicols}
  \end{questionmult}
}

\element{probability1}{
  \begin{questionmult}{2}
Случайным образом выбирается семья с двумя детьми. Событие $A$ --- в семье старший ребенок --- мальчик,  событие $B$ --- в семье только один из детей --- мальчик, событие $C$ --- в семье хотя бы один из детей --- мальчик.  Вероятность $\P(A \cup C)$ равна
   \begin{multicols}{3}
   \begin{choices}
      \correctchoice{$3/4$}
      \wrongchoice{$3/8$}
      \wrongchoice{$2/3$}
      \wrongchoice{$1$}
      \wrongchoice{$1/2$}
     
       \end{choices}
  \end{multicols}
  \end{questionmult}
}

\element{probability1}{
  \begin{questionmult}{3}
Случайным образом выбирается семья с двумя детьми. Событие $A$ --- в семье старший ребенок --- мальчик,  событие $B$ --- в семье только один из детей --- мальчик, событие $C$ --- в семье хотя бы один из детей --- мальчик.  Вероятность $\P(A | C)$ равна
  \begin{multicols}{3}
   \begin{choices}
      \correctchoice{$2/3$}
      \wrongchoice{$1/4$}
      \wrongchoice{$1/2$}
      \wrongchoice{$1$}
      \wrongchoice{$3/4$}
     
       \end{choices}
  \end{multicols}
  \end{questionmult}
}

\element{probability1}{
  \begin{questionmult}{4}
    Случайным образом выбирается семья с двумя детьми. Событие $A$ --- в семье старший ребенок --- мальчик,  событие $B$ --- в семье только один из детей --- мальчик, событие $C$ --- в семье хотя бы один из детей --- мальчик. 
    \begin{choices}
      \correctchoice{$A$ и $B$ --- независимы, $A$ и $C$ --- зависимы, $B$ и $C$ --- зависимы}
      \wrongchoice{События $A$, $B$, $C$ --- независимы попарно, но зависимы в совокупности}
      \wrongchoice{Любые два события из $A$, $B$, $C$ --- зависимы}
      \wrongchoice{События $A$, $B$, $C$ --- независимы в совокупности}
      \wrongchoice{$\P(A\cap B\cap C)=\P(A)\P(B)\P(C)$}
     
    \end{choices}
  \end{questionmult}
}


\element{probability1}{
  \begin{questionmult}{5}
Имеется три монетки. Две <<правильных>> и одна --- с <<орлами>> по обеим сторонам. Вася выбирает одну монетку наугад и подкидывает ее один раз. Вероятность того, что выпадет орел равна
    \begin{multicols}{3}
   \begin{choices}
      \correctchoice{$2/3$}
      \wrongchoice{$1/2$}
      \wrongchoice{$3/5$}
      \wrongchoice{$1/3$}
      \wrongchoice{$2/5$}
     
       \end{choices}
  \end{multicols}
  \end{questionmult}
}

\element{probability}{
  \begin{questionmult}{6}
Имеется три монетки. Две <<правильных>> и одна --- с <<орлами>> по обеим сторонам. Вася выбирает одну монетку наугад и подкидывает ее один раз. Вероятность того, что была выбрана неправильная монетка, если выпал орел, равна
  \begin{multicols}{3}
   \begin{choices}
      \correctchoice{$1/2$}
      \wrongchoice{$3/5$}
      \wrongchoice{$2/3$}
      \wrongchoice{$1/3$}
       \wrongchoice{$3/2$}
     
       \end{choices}
  \end{multicols}
  \end{questionmult}
}


\element{probability}{
  \begin{questionmult}{7}
Вася бросает 7 правильных игральных кубиков. Наиболее вероятное количество выпавших шестёрок равно
    \begin{multicols}{3}
   \begin{choices}
      \correctchoice{$1$}
      \wrongchoice{$6/7$}
      \wrongchoice{$7/6$}
      \wrongchoice{$0$}
      \wrongchoice{$2$}
     
       \end{choices}
  \end{multicols}
  \end{questionmult}
}


\element{probability}{
  \begin{questionmult}{8}
Вася бросает 7 правильных игральных кубиков. Вероятность того, что ровно на пяти из кубиков выпадет шестёрка равна
    \begin{multicols}{3}
   \begin{choices}
    \correctchoice{$525\left(\frac{1}{6}\right)^7$}
      \wrongchoice{$\frac{7}{12}\left(\frac{1}{6}\right)^5$}
      \wrongchoice{$\frac{525}{12}\left(\frac{1}{6}\right)^7$}
      \wrongchoice{$\left(\frac{1}{6}\right)^5$}
       \wrongchoice{$\left(\frac{1}{6}\right)^7$}
     
       \end{choices}
  \end{multicols}
  \end{questionmult}
}


\element{probability}{
  \begin{questionmult}{9}
Вася бросает 7 правильных игральных кубиков. Математическое ожидание суммы выпавших очков равно
    \begin{multicols}{3}
   \begin{choices}
      \correctchoice{$24.5$}
      \wrongchoice{$7/6$}
      \wrongchoice{$21$}
      \wrongchoice{$30$}
      \wrongchoice{$42$}
     
      \end{choices}
  \end{multicols}
  \end{questionmult}
}


\element{probability}{
  \begin{questionmult}{10}
Вася бросает 7 правильных игральных кубиков. Дисперсия суммы выпавших очков равна
\begin{multicols}{3}
   \begin{choices}
      \correctchoice{$7\cdot\frac{35}{12}$}
      \wrongchoice{$7$}
      \wrongchoice{$7\cdot \frac{35}{36}$}
      \wrongchoice{$7/6$}
      \wrongchoice{$35/36$}      
     
    \end{choices}
  \end{multicols}
  \end{questionmult}
}

\element{probability}{
  \begin{questionmult}{11}
Вася бросает 7 правильных игральных кубиков. Пусть величина  $X$ --- сумма очков, выпавших на первых двух кубиках, а величина  $Y$ --- сумма очков, выпавших на следующих пяти кубиках. Ковариация $\Cov(X,Y)$ равна
\begin{multicols}{3}
   \begin{choices}
      \correctchoice{$0$}
      \wrongchoice{$1$}
      \wrongchoice{$0.5$}
      \wrongchoice{$2/5$}
      \wrongchoice{$-2/5$}      
     
    \end{choices}
  \end{multicols}
  \end{questionmult}
}


\element{probability}{
  \begin{questionmult}{12}
Число изюминок в булочке --- случайная величина, имеющая распределение Пуассона. Известно, что в среднем каждая булочка содержит 13 изюминок. Вероятность того, что в случайно выбранной булочке окажется только одна изюминка равна:
\begin{multicols}{3}
   \begin{choices}
      \correctchoice{$13e^{-13}$}
      \wrongchoice{$1/13$}
      \wrongchoice{$e^{-13}$}
      \wrongchoice{$e^{-13}/13$}
      \wrongchoice{$e^{13}/13!$}      
     
    \end{choices}
  \end{multicols}
  \end{questionmult}
}

\element{commontext}{
\newpage
\rule{\textwidth}{1pt}
\textbf{В вопросах 13-16} совместное распределение пары величин $X$ и $Y$ задано таблицей:

\begin{tabular}{c|ccc}
 & $Y=-1$ & $Y=0$ & $Y=1$ \\ 
\hline 
$X=-1$ & $1/4$ & $0$  &  $1/4$\\ 
$X=1$ & $1/6$ & $1/6$ &  $1/6$ \\ 
\end{tabular} 

\vspace{0.5cm}
}

\element{1316}{
  \begin{questionmult}{13}
Математическое ожидание случайной величины $X$ при условии, что $Y=-1$ равно
\begin{multicols}{3}
   \begin{choices}
      \correctchoice{$-1/5$}
      \wrongchoice{$-1/12$}
      \wrongchoice{$0$}
      \wrongchoice{$-1/3$}
      \wrongchoice{$1/10$}      
     
    \end{choices}
  \end{multicols}
  \end{questionmult}
}


\element{1316}{
  \begin{questionmult}{14}
Вероятность того, что $X=1$ при условии, что $Y<0$ равна
\begin{multicols}{3}
   \begin{choices}
      \correctchoice{$2/5$}
      \wrongchoice{$1/6$}
      \wrongchoice{$1/12$}
      \wrongchoice{$5/12$}
      \wrongchoice{$1/3$}      
     
    \end{choices}
  \end{multicols}
  \end{questionmult}
}

\element{1316}{
  \begin{questionmult}{15}
Дисперсия случайной величины $Y$  равна
\begin{multicols}{3}
   \begin{choices}
      \correctchoice{$5/6$}
      \wrongchoice{$5/12$}
      \wrongchoice{$1/3$}
      \wrongchoice{$1/2$}
      \wrongchoice{$12/5$}      
     
    \end{choices}
  \end{multicols}
  \end{questionmult}
}

\element{1316}{
  \begin{questionmult}{16}
Ковариация, $\Cov(X,Y)$, равна
\begin{multicols}{3}
   \begin{choices}
      \correctchoice{$0$}
      \wrongchoice{$1$}
      \wrongchoice{$0.5$}
      \wrongchoice{$-0.5$}
      \wrongchoice{$-1$}      
     
    \end{choices}
  \end{multicols}
  \end{questionmult}
}


\element{commontext2}{
%\newpage
\rule{\textwidth}{1pt}
\textbf{В вопросах 17-19} функция распределения случайной величины $X$ имеет вид
\[
F(x)=\begin{cases}
0, \; \text{ если } x<0 \\
cx^2, \; \text{ если } x\in [0;1] \\
1, \; \text{ если } x>1
\end{cases}
\]

\vspace{0.5cm}
}


\element{1719}{
  \begin{questionmult}{17}
Константа $c$ равна
\begin{multicols}{3}
   \begin{choices}
      \correctchoice{$1$}
      \wrongchoice{$0.5$}
      \wrongchoice{$1.5$}
      \wrongchoice{$2$}
      \wrongchoice{$2/3$}      
     
    \end{choices}
  \end{multicols}
  \end{questionmult}
}

\element{1719}{
  \begin{questionmult}{18}
Вероятность того, что величина $X$ примет значение из интервала  $[0.5, 1.5]$ равна
\begin{multicols}{3}
   \begin{choices}
      \correctchoice{$3/4$}
      \wrongchoice{$1$}
      \wrongchoice{$2/3$}
      \wrongchoice{$1/2$}
      \wrongchoice{$3/2$}      
     
    \end{choices}
  \end{multicols}
  \end{questionmult}
}

\element{1719}{
  \begin{questionmult}{19}
Математическое ожидание $\E(X)$ равно
\begin{multicols}{3}
   \begin{choices}
      \correctchoice{$2/3$}
      \wrongchoice{$1/4$}
      \wrongchoice{$1/2$}
      \wrongchoice{$3/4$}
      \wrongchoice{$2$}      
     
    \end{choices}
  \end{multicols}
  \end{questionmult}
}

\element{commontext3}{
\newpage
\rule{\textwidth}{1pt}
\textbf{В вопросах 20-23} совместная функция плотности пары $X$ и $Y$ имеет вид
\[
f(x,y)=\begin{cases}
cx^2y^2, \; \text{ если } x\in[0;1], y\in [0;1] \\
0, \; \text{ иначе} 
\end{cases}
\]

\vspace{0.5cm}

}

\element{2023}{
  \begin{questionmult}{20}
Константа $c$ равна
\begin{multicols}{3}
   \begin{choices}
      \correctchoice{$9$}
      \wrongchoice{$1$}
      \wrongchoice{$1/2$}
      \wrongchoice{$1/4$}
      \wrongchoice{$2$}      
     
    \end{choices}
  \end{multicols}
  \end{questionmult}
}

\element{2023}{
  \begin{questionmult}{21}
Вероятность $\P(X<0.5, Y<0.5)$ равна
\begin{multicols}{3}
   \begin{choices}
      \correctchoice{$1/64$}
      \wrongchoice{$1/8$}
      \wrongchoice{$1/16$}
      \wrongchoice{$1/4$}
      \wrongchoice{$9/16$}      
     
    \end{choices}
  \end{multicols}
  \end{questionmult}
}

\element{2023}{
  \begin{questionmult}{22}
Условная функция плотности  $f_{X|Y=2}(x)$ равна 
\begin{multicols}{2}
   \begin{choices}
      \correctchoice{не определена}
      \wrongchoice{$f_{X|Y=2}(x)=\begin{cases} 9x^2\, \text{ если } x\in [0;1] \\ 0, \text{ иначе }    \end{cases}$}
      \wrongchoice{$f_{X|Y=2}(x)=\begin{cases} 3x^2\, \text{ если } x\in [0;1] \\ 0, \text{ иначе }    \end{cases}$}
      \wrongchoice{$f_{X|Y=2}(x)=\begin{cases} 36x^2\, \text{ если } x\in [0;1] \\ 0, \text{ иначе }    \end{cases}$}
      \wrongchoice{$f_{X|Y=2}(x)=\begin{cases} x^2\, \text{ если } x\in [0;1] \\ 0, \text{ иначе }    \end{cases}$}
     
    \end{choices}
  \end{multicols}
  \end{questionmult}
}

\element{2023}{
  \begin{questionmult}{23}
Математическое ожидание $\E(X/Y)$ равно
\begin{multicols}{3}
   \begin{choices}
      \correctchoice{$9/8$}
      \wrongchoice{$3$}
      \wrongchoice{$1$}
      \wrongchoice{$1/2$}
      \wrongchoice{$2$}      
     
    \end{choices}
  \end{multicols}
  \end{questionmult}
}

\element{commontext4}{
%\newpage
\rule{\textwidth}{1pt}
\textbf{В вопросах 24-25} известно, что $\E(X)=1$, $\Var(X)=1$, $\E(Y)=4$, $\Var(Y)=9$, $\Cov(X,Y)=-3$

\vspace{0.5cm}

}

\element{2425}{
  \begin{questionmult}{24}
Ковариация $\Cov(2X-Y,X+3Y)$ равна
\begin{multicols}{3}
   \begin{choices}
      \correctchoice{$-40$}
      \wrongchoice{$-18$}
      \wrongchoice{$22$}
      \wrongchoice{$40$}
      \wrongchoice{$18$}      
     
    \end{choices}
  \end{multicols}
  \end{questionmult}
}

\element{2425}{
  \begin{questionmult}{25}
Корреляция $\Corr(2X+3,4Y-5)$ равна
\begin{multicols}{3}
   \begin{choices}
      \correctchoice{$-1$}
      \wrongchoice{$-1/8$}
      \wrongchoice{$1/3$}
      \wrongchoice{$1$}
      \wrongchoice{$1/6$}      
     
    \end{choices}
  \end{multicols}
  \end{questionmult}
}


\element{2630}{
  \begin{questionmult}{26}
  \AMCnoCompleteMulti
Пусть случайные величины $X$ и $Y$ --- независимы, тогда \textbf{НЕ ВЕРНЫМ} является утверждение
\begin{multicols}{2}
   \begin{choices}
      \correctchoice{$\Var(X-Y)<\Var(X)+\Var(Y)$ }
      \wrongchoice{$\Cov(X,Y) = 0$}
      \wrongchoice{$\E(XY)=\E(X)\E(Y)$}
      \wrongchoice{$\E(X|Y)=\E(X)$}
      \wrongchoice{$\P(X<a, Y<b)=\P(X<a)\P(Y<b)$}      
      \wrongchoice{$\P(X<a | Y<b)=\P(X<a)$}
    \end{choices}
  \end{multicols}
  \end{questionmult}
}


\element{2630}{
  \begin{questionmult}{27}
Если $\E(X)=0$, то, согласно неравенству Чебышева, $\P(|X| \leq 5 \sqrt{\Var(X)})$ лежит в интервале
\begin{multicols}{3}
   \begin{choices}
      \correctchoice{$[0.96;1]$ }
      \wrongchoice{$[0;0.04]$}
      \wrongchoice{$[0;0.2]$}
      \wrongchoice{$[0.8;1]$}
      \wrongchoice{$[0.5;1]$}      
     
    \end{choices}
  \end{multicols}
  \end{questionmult}
}


\element{2630}{
  \begin{questionmult}{28}
Пусть $X_1$, $X_2$, \ldots, $X_n$ --- последовательность независимых одинаково распределенных случайных величин, $\E(X_i)=3$ и $\Var(X_i)=9$. Следующая величина имеет асимптотически стандартное нормальное распределение
\begin{multicols}{3}
   \begin{choices}
      \correctchoice{ $\sqrt{n}\frac{\bar{X}-3}{3}$ }
      \wrongchoice{$\frac{\bar{X}_n-3}{3}$}
      \wrongchoice{$\frac{\bar{X}_n-3}{3\sqrt{n}}$}
      \wrongchoice{$\frac{X_n-3}{3}$}
      
      \wrongchoice{$\sqrt{n}(\bar{X}-3)$}
    \end{choices}
  \end{multicols}
  \end{questionmult}
}

\element{2630}{
  \begin{questionmult}{29}
  \AMCnoCompleteMulti
Случайная величина $X$ имеет функцию плотности $f(x)=\frac{1}{3\sqrt{2\pi}} \exp\left(-\frac{(x-1)^2}{18} \right)$. Следующее утверждение \textbf{НЕ ВЕРНО}
\begin{multicols}{3}
   \begin{choices}
      \correctchoice{Случайная величина $X$ дискретна }
      \wrongchoice{$\Var(X)=9$}
      \wrongchoice{$\E(X)=1$}
      \wrongchoice{$\P(X>1)=0.5$}
      \wrongchoice{$\P(X=0)=0$}      
      \wrongchoice{$\P(X<0)>0$}
    \end{choices}
  \end{multicols}
  \end{questionmult}
}

\element{2630}{
  \begin{questionmult}{30}
  \AMCnoCompleteMulti
Пусть $X_1$, $X_2$, \ldots, $X_n$ --- последовательность независимых одинаково распределенных случайных величин, $\E(X_i)=\mu$ и $\Var(X_i)=\sigma^2$. Следующее утверждение в общем случае \textbf{НЕ ВЕРНО}:
%\begin{multicols}{3}
   \begin{choices}
      \correctchoice{$\frac{X_n-\mu}{\sigma} \overset{F}{\to} N(0;1)$ при $n\to\infty$ }
      \wrongchoice{$\lim_{n\to\infty} \Var(\bar{X}_n)=0$}
      \wrongchoice{$\bar{X}_n \overset{P}{\to} \mu$ при $n\to\infty$}
      \wrongchoice{$\frac{\bar{X}_n-\mu}{\sigma /\sqrt{n}} \overset{F}{\to } N(0,1) $ при $n\to\infty$}
       \wrongchoice{$\frac{\bar{X}_n-\mu}{\sqrt{n} \sigma } \overset{P}{\to } 0 $ при $n\to\infty$}
      %\wrongchoice{Центральная предельная теорема не применима к последовательности бернуллиевских случайных величин}      
       \wrongchoice{$\bar{X}_n-\mu \overset{F}{\to } 0 $ при $n\to\infty$}
    \end{choices}
  %\end{multicols}
  \end{questionmult}
}

\element{ruler}{
\noindent\rule{\textwidth}{1pt}
}

\element{newpage}{
\newpage\null
}



\element{exam_15}{
  \begin{questionmult}{1}
Пусть $X_1$, \ldots, $X_n$ --- выборка объема $n$ из равномерного на $[a, b]$ распределения. Оценка $X_1+X_2$ параметра $c=a+b$ является
\begin{multicols}{2}
   \begin{choices}
      \correctchoice{несмещенной и несостоятельной}
      \wrongchoice{несмещенной и состоятельной}
      \wrongchoice{смещенной и состоятельной}
      \wrongchoice{смещенной и несостоятельной}
      \wrongchoice{асимптотически несмещенной и состоятельной}
      \end{choices}
  \end{multicols}
  \end{questionmult}
}


\element{exam_15}{
  \begin{questionmult}{2}
Пусть $X_1$, \ldots, $X_n$ --- выборка объема $n$ из некоторого распределения с конечным математическим ожиданием. Несмещенной и состоятельной оценкой математического ожидания является
\begin{multicols}{3}
   \begin{choices}
      \correctchoice{$\frac{X_1}{2 n}+\frac{X_2+\ldots+X_{n-1}}{n-2}-\frac{X_n}{2 n}$}
      \wrongchoice{$\frac{1}{3} X_1 + \frac{2}{3} X_2$}
      \wrongchoice{$\frac{X_1}{2 n}+\frac{X_2+\ldots+X_{n-2}}{n-2}+\frac{X_n}{2 n}$}
      \wrongchoice{$\frac{X_1}{2 n}+\frac{X_2+\ldots+X_{n-2}}{n-1}+\frac{X_n}{2 n}$}    
      \wrongchoice{$\frac{X_1+X_2}{2}$} 
      \end{choices}
  \end{multicols}
  \end{questionmult}
}

\element{exam_15}{
  \begin{questionmult}{3}
Пусть $X_1$,\ldots, $X_n$ --- выборка объема $n$ из равномерного на $[0, \theta]$ распределения. Оценка параметра $\theta$ методом моментов по $k$-му моменту имеет вид:
\begin{multicols}{3}
   \begin{choices}
      \correctchoice{$\sqrt[k]{(k+1) \overline{X^k}}$}
      \wrongchoice{$\sqrt[k]{(k+1) \overline{X}^k}$}
      \wrongchoice{$\sqrt[k]{k \overline{X^k}}$}
      \wrongchoice{$\sqrt[k]{k \overline{X}^k}$}     
       \wrongchoice{$\sqrt[k+1]{(k+1) \overline{X}^k}$}     
      \end{choices}
  \end{multicols}
  \end{questionmult}
}


\element{exam_15}{
  \begin{questionmult}{4}
Пусть $X_1$, \ldots, $X_n$ --- выборка объема $n$ из равномерного на $[0, \theta]$ распределения. Состоятельной оценкой параметра $\theta$ является:
\begin{multicols}{3}
   \begin{choices}[o] % не рандомизирует порядок ответов      
      \wrongchoice{$X_{(n)}$}
      \wrongchoice{$X_{(n-1)}$}
      \wrongchoice{$\frac{n}{n+1} X_{(n-1)}$}     
       \wrongchoice{$\frac{n^2}{n^2-n+3} X_{(n-3)}$}     
       \correctchoice{все перечисленные случайные величины}
      \end{choices}
  \end{multicols}
  \end{questionmult}
}


\element{exam_15}{ % в фигурных скобках название группы вопросов
  \begin{questionmult}{5} % тип вопроса (questionmult --- множественный выбор) и в фигурных --- номер вопроса
Пусть $X_1$, \ldots, $X_{2 n}$ --- выборка объема $2 n$ из некоторого распределения. Какая из нижеперечисленных оценок математического ожидания имеет наименьшую дисперсию?
\begin{multicols}{3} % располагаем ответы в 3 колонки
   \begin{choices} % опция [o] не рандомизирует порядок ответов      
      \wrongchoice{$X_1$}
      \wrongchoice{$\frac{X_1+X_2}{2}$}
      \wrongchoice{$\frac{1}{n} \sum_{i=1}^n X_i$}     
       \wrongchoice{$\frac{1}{n} \sum_{i=n+1}^{2 n} X_i$}     
       \correctchoice{$\frac{1}{2 n} \sum_{i=1}^{2 n} X_i$}
      \end{choices}
  \end{multicols}
  \end{questionmult}
}


\element{exam_15}{ % в фигурных скобках название группы вопросов
  \begin{questionmult}{6} % тип вопроса (questionmult --- множественный выбор) и в фигурных --- номер вопроса
Пусть $X_1$, \ldots, $X_n$ --- выборка объема $n$ из распределения Бернулли с параметром $p$. Статистика $X_2 X_{n-2}$ является
\begin{multicols}{2} % располагаем ответы в [k] колонки
   \begin{choices} % опция [o] не рандомизирует порядок ответов      
      \wrongchoice{оценкой максимального правдоподобия}
      \wrongchoice{асимптотически нормальной оценкой $p^2$}
      \wrongchoice{эффективной оценкой $p^2$}     
       \wrongchoice{состоятельной оценкой $p^2$}     
       \correctchoice{несмещенной оценкой $p^2$}
      \end{choices}
  \end{multicols}
  \end{questionmult}
}

\element{exam_15}{ % в фигурных скобках название группы вопросов
  \begin{questionmult}{7} % тип вопроса (questionmult --- множественный выбор) и в фигурных --- номер вопроса
Пусть $X_1$, \ldots, $X_n$ --- выборка объема $n$ из равномерного на $[a, b]$ распределения. Выберите наиболее точный ответ из предложенных. Оценка $\theta^*_n = X_{(n)}-X_{(1)}$ длины отрезка $[a,b]$ является
\begin{multicols}{3} % располагаем ответы в 3 колонки
   \begin{choices} % опция [o] не рандомизирует порядок ответов      
      \wrongchoice{несмещенной}
      \wrongchoice{состоятельной и асимптотически смещённой}
      \wrongchoice{несостоятельной и асимптотически несмещенной}     
       \wrongchoice{нормально распределённой}     
       \correctchoice{состоятельной и асимптотически несмещенной}
      \end{choices}
  \end{multicols}
  \end{questionmult}
}


\element{exam_15}{ % в фигурных скобках название группы вопросов
  \begin{questionmult}{8} % тип вопроса (questionmult --- множественный выбор) и в фигурных --- номер вопроса
Мощностью теста называется
%\begin{multicols}{2} % располагаем ответы в [k] колонки
   \begin{choices} % опция [o] не рандомизирует порядок ответов      
      \wrongchoice{Вероятность отвергнуть основную гипотезу, когда она верна}
      \wrongchoice{Вероятность отвергнуть альтернативную гипотезу, когда она верна}
      \wrongchoice{Вероятность принять неверную гипотезу}     
       \wrongchoice{Единица минус  вероятность отвергнуть основную гипотезу, когда она верна}     
       \correctchoice{Единица минус  вероятность отвергнуть альтернативную гипотезу, когда она верна}
      \end{choices}
%  \end{multicols}
  \end{questionmult}
}

\element{exam_15}{ % в фигурных скобках название группы вопросов
  \begin{questionmult}{9} % тип вопроса (questionmult --- множественный выбор) и в фигурных --- номер вопроса
Если P-значение (P-value) больше уровня значимости  $\alpha$, то гипотеза  $H_0: \; \sigma=1$
\begin{multicols}{2} % располагаем ответы в {k} колонки
   \begin{choices} % опция [o] не рандомизирует порядок ответов      
      \wrongchoice{Отвергается}
      \wrongchoice{Отвергается, только если  $H_a: \; \sigma>1$}
      \wrongchoice{Отвергается, только если  $H_a: \; \sigma\neq 1$}     
       \wrongchoice{ Отвергается, только если  $H_a: \; \sigma<1$}     
       \correctchoice{Не отвергается}
      \end{choices}
  \end{multicols}
  \end{questionmult}
}


\element{exam_15}{ % в фигурных скобках название группы вопросов
  \begin{questionmult}{10} % тип вопроса (questionmult --- множественный выбор) и в фигурных --- номер вопроса
Имеется случайная выборка размера $n$ из нормального распределения. При проверке гипотезы о равенстве математического ожидания заданному значению при известной дисперсии используется статистика, имеющая распределение
\begin{multicols}{3} % располагаем ответы в {k} колонки
   \begin{choices} % опция [o] не рандомизирует порядок ответов      
      \wrongchoice{$t_n$}
      \wrongchoice{ $t_{n-1}$}
      \wrongchoice{$\chi^2_n$}     
       \wrongchoice{$\chi^2_{n-1}$}     
       \correctchoice{$N(0,1)$}
      \end{choices}
  \end{multicols}
  \end{questionmult}
}




\element{exam_15}{ % в фигурных скобках название группы вопросов
  \begin{questionmult}{11} % тип вопроса (questionmult --- множественный выбор) и в фигурных --- номер вопроса
Имеется случайная выборка размера $n$ из нормального распределения. При проверке гипотезы о равенстве дисперсии заданному значению при неизвестном математическом ожидании используется статистика, имеющая распределение
\begin{multicols}{3} % располагаем ответы в {k} колонки
   \begin{choices} % опция [o] не рандомизирует порядок ответов      
      \wrongchoice{$t_n$}
      \wrongchoice{ $t_{n-1}$}
      \wrongchoice{$\chi^2_n$}     
       \wrongchoice{$N(0,1)$}     
       \correctchoice{$\chi^2_{n-1}$}
      \end{choices}
  \end{multicols}
  \end{questionmult}
}


\element{exam_15}{ % в фигурных скобках название группы вопросов
  \begin{questionmult}{12} % тип вопроса (questionmult --- множественный выбор) и в фигурных --- номер вопроса
По случайной выборке из 100 наблюдений было оценено выборочное среднее $\bar{X}=20$  и несмещенная оценка дисперсии  $\hat{\sigma}^2=25$. В рамках проверки гипотезы $H_0: \; \mu=15$  против альтернативной гипотезы $H_a: \; \mu>15$  можно сделать следующее заключение
%\begin{multicols}{2} % располагаем ответы в {k} колонки
   \begin{choices} % опция [o] не рандомизирует порядок ответов      
      \wrongchoice{Гипотеза $H_0$  отвергается на уровне значимости 5\%, но не  на уровне значимости 1\%}
      \wrongchoice{Гипотеза  $H_0$ отвергается на уровне значимости 10\%, но не на уровне значимости 5\%}
      \wrongchoice{Гипотеза  $H_0$ отвергается на уровне значимости 20\%, но не  на уровне значимости 10\%}     
       \wrongchoice{ Гипотеза $H_0$  не отвергается на любом разумном уровне значимости}     
       \correctchoice{Гипотеза $H_0$  отвергается на любом разумном уровне значимости}
      \end{choices}
%  \end{multicols}
  \end{questionmult}
}


\element{exam_15}{ % в фигурных скобках название группы вопросов
  \begin{questionmult}{13} % тип вопроса (questionmult --- множественный выбор) и в фигурных --- номер вопроса
На основе случайной выборки, содержащей одно наблюдение  $X_1$, тестируется гипотеза $H_0: \; X_1 \sim U[0;1]$  против альтернативной гипотезы  $H_a: \; X_1 \sim U[0.5;1.5]$. Рассматривается критерий: если $X_1>0.8$, то гипотеза $H_0$  отвергается в пользу гипотезы  $H_a$. Вероятность ошибки 2-го рода для этого критерия равна:
\begin{multicols}{3} % располагаем ответы в {k} колонки
   \begin{choices} % опция [o] не рандомизирует порядок ответов      
      \wrongchoice{0.1}
      \wrongchoice{0.2}
      \wrongchoice{0.4}     
       \wrongchoice{0.5}     
       \correctchoice{0.3}
      \end{choices}
 \end{multicols}
  \end{questionmult}
}

\element{exam_15}{ % в фигурных скобках название группы вопросов
  \begin{questionmult}{14} % тип вопроса (questionmult --- множественный выбор) и в фигурных --- номер вопроса
Пусть $X_1$, $X_2$, \ldots, $X_n$ --- случайная выборка размера 36 из нормального распределения $N(\mu, 9)$. Для тестирования основной гипотезы  $H_0: \; \mu=0$  против альтернативной $H_a: \; \mu=-2$   вы используете критерий: если  $\bar{X}\geq -1$, то вы не отвергаете гипотезу $H_0$, в противном случае вы отвергаете гипотезу  $H_0$ в пользу гипотезы  $H_a$. Мощность критерия равна
\begin{multicols}{3} % располагаем ответы в {k} колонки
   \begin{choices} % опция [o] не рандомизирует порядок ответов      
      \wrongchoice{0.58}
      \wrongchoice{0.85}
      \wrongchoice{0.78}     
       \wrongchoice{0.87}     
       \correctchoice{0.98}
      \end{choices}
 \end{multicols}
  \end{questionmult}
}

\element{exam_15}{ % в фигурных скобках название группы вопросов
  \begin{questionmult}{15} % тип вопроса (questionmult --- множественный выбор) и в фигурных --- номер вопроса
Николай Коперник подбросил бутерброд 200 раз. Бутерброд упал маслом вниз 95 раз, а маслом вверх --- 105 раз. Значение критерия $\chi^2$ Пирсона для проверки гипотезы о равной вероятности данных событий равно 
\begin{multicols}{3} % располагаем ответы в {k} колонки
   \begin{choices} % опция [o] не рандомизирует порядок ответов      
      \wrongchoice{0.25}
      \wrongchoice{0.75}
      \wrongchoice{0.5}     
       \wrongchoice{7.5}     
       \correctchoice{0.5}
      \end{choices}
 \end{multicols}
  \end{questionmult}
}

\element{exam_15}{ % в фигурных скобках название группы вопросов
  \begin{questionmult}{16} % тип вопроса (questionmult --- множественный выбор) и в фигурных --- номер вопроса
Каждое утро в 8:00 Иван Андреевич Крылов, либо завтракает, либо уже позавтракал. В это же время кухарка либо заглядывает к Крылову, либо нет. По таблице сопряженности вычислите  статистику $\chi^2$ Пирсона для тестирования гипотезы о том, что визиты кухарки не зависят от того, позавтракал ли уже Крылов или нет.
\begin{tabular}{c|cc}
Время 8:00 & кухарка заходит & кухарка не заходит \\ 
\hline 
Крылов завтракает & 200 & 40 \\ 
Крылов уже позавтракал & 25 & 100 \\ 
\end{tabular} 
\begin{multicols}{3} % располагаем ответы в {k} колонки
   \begin{choices} % опция [o] не рандомизирует порядок ответов      
      \wrongchoice{39}
      \wrongchoice{79}
      \wrongchoice{100}     
       \wrongchoice{179}     
       \correctchoice{139}
      \end{choices}
 \end{multicols}
  \end{questionmult}
}

%% Боря Демешев:

\element{exam_15}{ % в фигурных скобках название группы вопросов
  \begin{questionmult}{17} % тип вопроса (questionmult --- множественный выбор) и в фигурных --- номер вопроса
Ковариационная матрица вектора $X=(X_1,X_2)$ имеет вид
\[
\begin{pmatrix}
10 & 3 \\
3 & 8
\end{pmatrix}
\]
Дисперсия разности элементов вектора, $\Var(X_1-X_2)$, равняется
\begin{multicols}{3} % располагаем ответы в {k} колонки
   \begin{choices} % опция [o] не рандомизирует порядок ответов      
      \wrongchoice{18}
      \wrongchoice{15}
      \wrongchoice{2}     
       \wrongchoice{6}     
       \correctchoice{12}
      \end{choices}
 \end{multicols}
  \end{questionmult}
}


\element{exam_15}{ % в фигурных скобках название группы вопросов
  \begin{questionmult}{18} % тип вопроса (questionmult --- множественный выбор) и в фигурных --- номер вопроса
Все условия регулярности для применения метода максимального правдоподобия выполнены. Вторая производная лог-функции правдоподобия равна $\ell''(\hat{\theta})=-100$. Оценка стандартной ошибки для $\hat{\theta}$ равна
\begin{multicols}{3} % располагаем ответы в {k} колонки
   \begin{choices} % опция [o] не рандомизирует порядок ответов      
      \wrongchoice{100}
      \wrongchoice{10}
      \wrongchoice{1}     
       \wrongchoice{0.01}     
       \correctchoice{0.1}
      \end{choices}
 \end{multicols}
  \end{questionmult}
}

\element{exam_15}{ % в фигурных скобках название группы вопросов
  \begin{questionmult}{19} % тип вопроса (questionmult --- множественный выбор) и в фигурных --- номер вопроса
Геродот Геликарнасский проверяет гипотезу $H_0: \; \mu=0, \; \sigma^2=1$ с помощью $LR$ статистики теста отношения правдоподобия. При подстановке оценок метода максимального правдоподобия в лог-функцию правдоподобия он получил $\ell=-177$, а при подстановке $\mu=0$ и $\sigma=1$ оказалось, что $\ell=-211$. Найдите значение $LR$ статистики и укажите её закон распределения при верной $H_0$ 
\begin{multicols}{3} % располагаем ответы в {k} колонки
   \begin{choices} % опция [o] не рандомизирует порядок ответов      
      \wrongchoice{$LR=34$, $\chi^2_2$}
      \wrongchoice{$LR=34$, $\chi^2_{n-1}$}
      \wrongchoice{$LR=\ln 68$, $\chi^2_{n-2}$}     
       \wrongchoice{$LR=\ln 34$, $\chi^2_{n-2}$}     
       \correctchoice{$LR=68$, $\chi^2_2$}
      \end{choices}
 \end{multicols}
  \end{questionmult}
}


\element{exam_15}{ % в фигурных скобках название группы вопросов
  \begin{questionmult}{20} % тип вопроса (questionmult --- множественный выбор) и в фигурных --- номер вопроса
Геродот Геликарнасский проверяет гипотезу $H_0: \; \mu=2$. Лог-функция правдоподобия имеет вид $\ell(\mu,\nu)=-\frac{n}{2}\ln (2\pi)-\frac{n}{2}\ln \nu -\frac{\sum_{i=1}^n(x_i-\mu)^2}{2\nu}$. Оценка максимального правдоподобия для $\nu$ при предположении, что $H_0$ верна, равна
\begin{multicols}{3} % располагаем ответы в {k} колонки
   \begin{choices} % опция [o] не рандомизирует порядок ответов      
      \wrongchoice{$\frac{\sum x_i^2 - 4\sum x_i}{n}$}
      \wrongchoice{$\frac{\sum x_i^2 - 4\sum x_i}{n}+1$}
      \wrongchoice{$\frac{\sum x_i^2 - 4\sum x_i}{n}+2$}     
       \wrongchoice{$\frac{\sum x_i^2 - 4\sum x_i}{n}+3$}     
       \correctchoice{$\frac{\sum x_i^2 - 4\sum x_i}{n}+4$}
      \end{choices}
 \end{multicols}
  \end{questionmult}
}


\element{exam_15}{ % в фигурных скобках название группы вопросов
  \begin{questionmult}{21} % тип вопроса (questionmult --- множественный выбор) и в фигурных --- номер вопроса
Ацтек Монтесума Илуикамина хочет оценить параметр $a$ методом максимального правдоподобия по выборке из неотрицательного распределения с функцией плотности $f(x)=\frac{1}{2}a^3x^2e^{-ax}$ при $x\geq 0$. Для этой цели ему достаточно максимизировать функцию  
\begin{multicols}{3} % располагаем ответы в {k} колонки
   \begin{choices} % опция [o] не рандомизирует порядок ответов      
      \wrongchoice{$3n\ln a - a \prod \ln x_i$}
      \wrongchoice{$3n\prod \ln a - a x^n$}
      \wrongchoice{$3n \ln a - an \ln x_i$}     
       \wrongchoice{$3n \sum \ln a_i - a \sum \ln x_i$}     
       \correctchoice{$3n \ln a - a \sum x_i$}
      \end{choices}
 \end{multicols}
  \end{questionmult}
}

\element{exam_15}{ % в фигурных скобках название группы вопросов
  \begin{questionmult}{22} % тип вопроса (questionmult --- множественный выбор) и в фигурных --- номер вопроса
Бессмертный гений поэзии Ли Бо оценивает математическое ожидание  по выборка размера $n$ из нормального распределения. Он построил оценку метода моментов, $\hat{\mu}_{MM}$, и оценку максимального правдоподобия, $\hat{\mu}_{ML}$. Про эти оценки можно утверждать, что 
\begin{multicols}{2} % располагаем ответы в {k} колонки
   \begin{choices} % опция [o] не рандомизирует порядок ответов      
      \wrongchoice{они не равны, но сближаются при $n\to \infty$}
      \wrongchoice{они не равны, и не сближаются при $n\to \infty$}
      \wrongchoice{ $\hat{\mu}_{MM}>\hat{\mu}_{ML}$}     
       \wrongchoice{$\hat{\mu}_{MM}<\hat{\mu}_{ML}$ }     
       \correctchoice{они равны}
      \end{choices}
 \end{multicols}
  \end{questionmult}
}


%% Ваня Станкевич

\element{exam_15}{ % в фигурных скобках название группы вопросов
  \begin{questionmult}{23} % тип вопроса (questionmult --- множественный выбор) и в фигурных --- номер вопроса
Проверяя гипотезу о равенстве дисперсий в двух выборках (размером в 3 и 5 наблюдений), Анаксимандр Милетский получил значение тестовой статистики 10. Если оценка дисперсии по первой выборке равна 8, то вторая оценка дисперсии может быть равна
\begin{multicols}{3} % располагаем ответы в {k} колонки
   \begin{choices} % опция [o] не рандомизирует порядок ответов      
      \wrongchoice{$25$}
      \wrongchoice{$4/3$}
      \wrongchoice{$3/4$}     
       \wrongchoice{$4$}     
       \correctchoice{$80$}
      \end{choices}
 \end{multicols}
  \end{questionmult}
}

\element{exam_15}{ % в фигурных скобках название группы вопросов
  \begin{questionmult}{24} % тип вопроса (questionmult --- множественный выбор) и в фигурных --- номер вопроса
Пусть  $\hat{\sigma}^2_1$ --- несмещенная оценка дисперсии, полученная по первой выборке размером $n_1$,   $\hat{\sigma}^2_2$ --- несмещенная оценка дисперсии, полученная по второй выборке, с меньшим размером  $n_2$. Тогда статистика $\frac{\hat{\sigma}^2_1/n_1}{\hat{\sigma}^2_2/n_2}$  имеет распределение
\begin{multicols}{3} % располагаем ответы в {k} колонки
   \begin{choices} % опция [o] не рандомизирует порядок ответов      
      \wrongchoice{$N(0;1)$}
      \wrongchoice{$\chi^2_{n_1+n_2}$}
      \wrongchoice{$F_{n_1,n_2}$}     
       \wrongchoice{$t_{n_1+n_2-1}$}     
       \wrongchoice{$F_{n_1-1,n_2-1}$}
      \end{choices}
 \end{multicols}
  \end{questionmult}
}

\element{exam_15}{ % в фигурных скобках название группы вопросов
  \begin{questionmult}{25} % тип вопроса (questionmult --- множественный выбор) и в фигурных --- номер вопроса
Зулус Чака каСензангакона проверяет гипотезу  о равенстве математических ожиданий в двух нормальных выборках небольших размеров $n_1$   и  $n_2$. Если дисперсии неизвестны, но равны, то тестовая статистика имеет распределение
\begin{multicols}{3} % располагаем ответы в {k} колонки
   \begin{choices} % опция [o] не рандомизирует порядок ответов      
      \wrongchoice{$t_{n_1+n_2-2}$}
      \wrongchoice{$t_{n_1+n_2}$}
      \wrongchoice{$F_{n_1,n_2}$}     
       \correctchoice{$t_{n_1+n_2-1}$}     
       \wrongchoice{$\chi^2_{n_1+n_2-1}$}
      \end{choices}
 \end{multicols}
  \end{questionmult}
}

\element{exam_15}{ % в фигурных скобках название группы вопросов
  \begin{questionmult}{26} % тип вопроса (questionmult --- множественный выбор) и в фигурных --- номер вопроса
Критерий знаков проверяет нулевую гипотезу
%\begin{multicols}{3} % располагаем ответы в {k} колонки
   \begin{choices} % опция [o] не рандомизирует порядок ответов      
      \wrongchoice{о равенстве математических ожиданий двух нормально распределенных случайных величин}
      \wrongchoice{о совпадении функции распределения случайной величины с заданной теоретической функцией распределения}
      \wrongchoice{о равенстве нулю вероятности того, что случайная величина $X$ окажется больше случайной величины $Y$, если альтернативная гипотеза записана как $\mu_X>\mu_Y$ }     
       \correctchoice{о равенстве нулю вероятности того, что случайная величина $X$ окажется больше случайной величины $Y$, если альтернативная гипотеза записана как $\mu_X>\mu_Y$}     
       \wrongchoice{о равенстве $1/2$ вероятности того, что случайная величина $X$ окажется больше случайной величины $Y$, если альтернативная гипотеза записана как $\mu_X>\mu_Y$}
      \end{choices}
% \end{multicols}
  \end{questionmult}
}


\element{exam_15}{ % в фигурных скобках название группы вопросов
  \begin{questionmult}{27} % тип вопроса (questionmult --- множественный выбор) и в фигурных --- номер вопроса
Вероятность ошибки первого рода, $\alpha$, и вероятность ошибки второго рода, $\beta$, всегда связаны соотношением
\begin{multicols}{3} % располагаем ответы в {k} колонки
   \begin{choices} % опция [o] не рандомизирует порядок ответов      
      \wrongchoice{$\alpha+\beta=1$}
      \wrongchoice{$\alpha+\beta \leq 1$}
      \wrongchoice{$\alpha+\beta \geq 1$}     
       \wrongchoice{$\alpha\leq \beta $}     
       \wrongchoice{$\alpha\geq \beta $}
      \end{choices}
 \end{multicols}
  \end{questionmult}
}


\element{exam_15}{ % в фигурных скобках название группы вопросов
  \begin{questionmult}{28} % тип вопроса (questionmult --- множественный выбор) и в фигурных --- номер вопроса
Среди 100 случайно выбранных ацтеков 20 платят дань Кулуакану, а 80 --- Аскапоцалько. Соответственно, оценка доли ацтеков, платящих дань Кулуакану, равна $\hat{p}=0.2$. Разумная оценка стандартного отклонения случайной величины $\hat{p}$ равна
\begin{multicols}{3} % располагаем ответы в {k} колонки
   \begin{choices} % опция [o] не рандомизирует порядок ответов      
      \wrongchoice{$0.4$}
      \wrongchoice{$0.16$}
      \wrongchoice{$1.6$}     
       \wrongchoice{$0.016$}     
       \correctchoice{$0.04$}
      \end{choices}
 \end{multicols}
  \end{questionmult}
}



%% ЕВ Коссова

\element{exam_15}{ % в фигурных скобках название группы вопросов
  \begin{questionmult}{29} % тип вопроса (questionmult --- множественный выбор) и в фигурных --- номер вопроса
Датчик случайных чисел выдал следующие значения псевдо случайной величины: $0.78$, $0.48$. Вычислите значение критерия Колмогорова и проверьте гипотезу $H_0$ о соответствии распределения равномерному на $[0;1]$. Критическое значение статистики Колмогорова для уровня значимости 0.1 и двух наблюдений равно $0.776$.
\begin{multicols}{3} % располагаем ответы в {k} колонки
   \begin{choices} % опция [o] не рандомизирует порядок ответов      
      \wrongchoice{0.3, $H_0$ не отвергается}
      \wrongchoice{1.26, $H_0$ отвергается}
      \wrongchoice{0.48, $H_0$ не отвергается}     
       \wrongchoice{0.37, $H_0$ не отвергается}     
       \correctchoice{0.78, $H_0$ отвергается}
      \end{choices}
 \end{multicols}
  \end{questionmult}
}

\element{exam_15}{ % в фигурных скобках название группы вопросов
  \begin{questionmult}{30} % тип вопроса (questionmult --- множественный выбор) и в фигурных --- номер вопроса
У пяти случайно выбранных студентов первого потока результаты за контрольную по статистике оказались равны  82, 47, 20, 43 и 73. У четырёх случайно выбранных студентов второго потока --- 68, 83, 60 и 52. Вычислите статистику Вилкоксона и проверьте гипотезу $H_0$ об однородности результатов студентов двух потоков. Критические значения статистики Вилкоксона равны $T_L=12$ и $T_R=28$.
\begin{multicols}{3} % располагаем ответы в {k} колонки
   \begin{choices} % опция [o] не рандомизирует порядок ответов      
      \wrongchoice{20, $H_0$ не отвергается}
      \wrongchoice{65.75, $H_0$ отвергается}
      \wrongchoice{53, $H_0$ отвергается}     
       \wrongchoice{12.75, $H_0$ не отвергается}     
       \correctchoice{24, $H_0$ не отвергается}
      \end{choices}
 \end{multicols}
  \end{questionmult}
}

\element{exam_15}{ % в фигурных скобках название группы вопросов
  \begin{questionmult}{31} % тип вопроса (questionmult --- множественный выбор) и в фигурных --- номер вопроса
 Производитель мороженного попросил оценить по 10-бальной шкале два вида мороженного: с кусочками шоколада и с орешками. Было опрошено 5 человек.
 
 
 \begin{tabular}{c|ccccc}
  & Евлампий & Аристарх & Капитолина & Аграфена & Эвридика \\ 
 \hline 
С крошкой & 10 & 6 & 7 & 5 & 4 \\ 
С орехами & 9 & 8 & 8 & 7 & 6 \\ 
 \end{tabular} 
 
 
Вычислите модуль значения статистики теста знаков. Используя нормальную аппроксимацию, проверьте на уровне значимости $0.05$ гипотезу об отсутствии предпочтения мороженного с орешками против альтернативы, что мороженное с орешками вкуснее.
\begin{multicols}{3} % располагаем ответы в {k} колонки
   \begin{choices} % опция [o] не рандомизирует порядок ответов      
      \wrongchoice{1.65, $H_0$ отвергается}
      \wrongchoice{1.34, $H_0$ не отвергается}
      \wrongchoice{1.29, $H_0$ отвергается}     
       \wrongchoice{1.29, $H_0$ не отвергается}     
       \correctchoice{1.96, $H_0$ отвергается}
      \end{choices}
 \end{multicols}
  \end{questionmult}
}


\element{exam_15}{ % в фигурных скобках название группы вопросов
  \begin{questionmult}{32} % тип вопроса (questionmult --- множественный выбор) и в фигурных --- номер вопроса
По 10 наблюдениям проверяется гипотеза $H_0: \; \mu=10$ против $H_a: \; \mu \neq 10$ на выборке из нормального распределения с неизвестной дисперсией. Величина $\sqrt{n}\cdot (\bar{X}-\mu)/\hat{\sigma}$ оказалась равной $1$. P-значение примерно равно
\begin{multicols}{3} % располагаем ответы в {k} колонки
   \begin{choices} % опция [o] не рандомизирует порядок ответов      
      \wrongchoice{$0.83$}
      \wrongchoice{$0.17$}
      \wrongchoice{$0.34$}     
       \wrongchoice{$0.32$}     
       \correctchoice{$0.16$}
      \end{choices}
 \end{multicols}
  \end{questionmult}
}


