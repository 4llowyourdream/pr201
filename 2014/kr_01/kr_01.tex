\documentclass[12pt,a4paper]{article}
\usepackage[utf8]{inputenc}
\usepackage[russian]{babel}

\usepackage{amsmath}
\usepackage{amsfonts}
\usepackage{amssymb}
\usepackage[left=2cm,right=2cm,top=2cm,bottom=2cm]{geometry}

\def \P{\mathbb{P}}
\def \Var{\mathbb{V}\mathrm{ar}}
\def \Cov{\mathbb{C}\mathrm{ov}}
\def \E{\mathbb{E}}

\begin{document}

\thispagestyle{empty}
\subsubsection*{Праздник номер один по теории вероятностей. Часть 1}


\begin{enumerate}
%\item Винни-Пух собирается играть в Пустяки и готовит для игры палочки. Он нашел палку длиной 1 м, а дальше поступает следующим образом. Разламывает палку равномерно в случайном месте, одну полученную часть использует для игры, а вторую снова случайным образом делит на две части. Далее одну новую часть Винни-Пух снова использует для игры, а вторую новую часть снова делит на две. И так далее. Обозначим $X_i$ --- длину палочки, использованной Винни-Пухом в $i$-ых Пустяках.

%Найдите функцию плотности $X_i$, $\E(X_i)$, $\Var(X_i)$

\item Вася купил два арбуза у торговки тети Маши и один арбуз у торговки тети Оли. Арбузы у тети Маши спелые с вероятностью 90\% (независимо друг от друга), арбузы у тети Оли спелые с вероятностью 70\%.

\begin{enumerate}
\item Какова вероятность того, что все Васины арбузы спелые?
\item Придя домой Вася выбрал случайным образом один из трех арбузов и разрезал его. Какова вероятность того, что это арбуз от тёти Маши, если он оказался спелым?
\item Какова вероятность того, что второй и третий съеденные Васей арбузы были от тёти Маши, если все три арбуза оказались спелыми? 
\end{enumerate}


\item В большой большой стране живет очень большое количество $n>>0$ семей. Количества детей в разных семьях независимы. Количество детей в каждой семье --- случайная величина с распределением заданным табличкой:


\begin{tabular}{ccccc}
$X_i$ & 0 & 1 & 2 & 3 \\ 
\hline 
$\P()$ & 0.1 & 0.3 & 0.2 & 0.4 \\ 
\end{tabular} 

\begin{enumerate}
\item Исследователь Афанасий выбирает одну семью из всех семей наугад, пусть $X$ --- число детей в этой семье. Найдите $\E(X)$ и $\Var(X)$.
\item Исследователь Бенедикт выбирает одного ребенка из всех детей наугад, пусть $Y$ --- число детей в семье этого ребёнка. Как распределена величина $Y$? Что больше, $\E(Y)$ или $\E(X)$?
\end{enumerate}

\item Функция плотности случайной величины $X$ имеет вид 
\[
f(x)=
\begin{cases}
\frac{3}{8} x^2, \text{ если } x\in [0;2] \\
0, \text{ иначе }
\end{cases}
\]
\begin{enumerate}
\item Не производя вычислений найдите $\int_{-\infty}^{+\infty}f(x)\,dx$
\item Найдите $\E(X)$, $\E(X^2)$ и дисперсию $\Var(X)$
\item Найдите $\P(X>1.5)$, $\P(X>1.5 \mid X>1)$
\item При каком $c$ функция $g(x)=c x f(x)$ будет функцией плотности некоторой случайной величины?
\end{enumerate}

\item Известно, что  $\E\left(Z\right)=-3$. $\E\left(Z^{2} \right)=15$,  $\Var\left(X+Y\right)=20$  и  $\Var\left(X-Y\right)=10$. 
\begin{enumerate}
\item Найдите  $\Var\left(Z\right)$,  $\Var\left(4-3Z\right)$  и  $\E\left(5+3Z-Z^{2} \right)$
\item Найдите  $\Cov\left(X,Y\right)$  и  $\Cov\left(6-X,3Y\right)$ 
\item Можно ли утверждать, что случайные величины $X$ и $Y$ независимы?
\end{enumerate}

\item Листая сборник задач по теории вероятностей Вася наткнулся на задачу:


\fbox{%
\parbox{15cm}{%
Какова вероятность того, что наугад выбранный ответ на этот вопрос окажется верным?

1) 0.25		2) 0.5		3) 0.6		4) 0.25 }
}

Чему же равна вероятность выбора верного ответа?

\item Книга в 500 страниц содержит 400 опечаток. Предположим, что каждая из них независимо от остальных опечаток может с одинаковой вероятностью оказаться на любой странице книги. 
\begin{enumerate}
\item Определите вероятность того, что на 13-й странице будет не менее двух опечаток, в явном виде и с помощью приближения Пуассона.
\item Определите наиболее вероятное число, математическое ожидание и дисперсию числа опечаток на 13-ой странице.
\item Является ли 13-ая страница более <<несчастливой>>, чем все остальные (в том смысле, что на 13-ой странице ожидается большее количество очепяток, чем на любой другой)?
\end{enumerate}
\item Вася случайным образом посещает лекции по ОВП (Очень Важному Предмету). С вероятностью 0.9 произвольно выбранная лекция полезна, и с вероятностью 0.7 она интересна. Полезность и интересность --- независимые друг от друга и от номера лекции свойства. Всего Вася прослушал 30 лекций. 
\begin{enumerate}
\item Определите математическое ожидание и дисперсию числа полезных лекций, прослушанных Васей
\item Определите математическое ожидание числа одновременно бесполезных и неинтересных лекций, прослушанных Васей, и математическое ожидание числа лекций, обладающих хотя бы одним из свойств (полезность,  интересность) 
\end{enumerate}
\item Функция распределения случайной величины X задана следующей формулой:
 \[
 F(x)=\frac{ae^x}{1+e^x}+b
 \]
Определите: константы $a$ и $b$, математическое ожидание и третий начальный момент случайной величины $X$, медиану и моду распределения.

%\item Вы хотите приобрести некую фирму. Стоимость фирмы для ее нынешних владельцев --- случайная величина, равномерно распределенная на отрезке $[0;1]$. Вы предлагаете владельцам продать ее за называемую Вами сумму. Владельцы либо соглашаются, либо нет. Если владельцы согласны, то Вы платите обещанную сумму и получаете фирму. Когда фирма переходит в Ваши руки, ее стоимость сразу возрастает на 20\%.

%\begin{enumerate}
%\item Чему равен Ваш ожидаемый выигрыш, если Вы предлагаете цену 0.5?
%\item Какова оптимальная предлагаемая цена?
%\end{enumerate}




\end{enumerate}


\newpage
\thispagestyle{empty}
\subsubsection*{Праздник номер один по теории вероятностей. Часть 2}


\begin{enumerate}
\item Маша подкидывает кубик до тех пор, пока два последних броска в сумме не дадут 10. Обозначим случайные величины: $N$ --- количество бросков, а $S$ --- сумма набранных за всю игру очков.  
\begin{enumerate}
\item Найдите $\P(N=2)$, $\P(N=3)$
\item Найдите $\E(N)$, $\E(S)$, $\E(N^2)$
\item Пусть $X_N$ --- результат последнего броска. Как распределена случайная величина $X_N$?
\end{enumerate}


\item В столовую пришли 30 студентов и встали в очередь в случайном порядке. Среди них есть Вовочка и Машенька. Пусть $V$ --- это количество человек в очереди перед Вовочкой, а $M\geq 0$ --- количество человек между Вовочкой и Машенькой. 
\begin{enumerate}
\item Найдите $\P(V=1)$, $\P(M=1)$, $\P(M=V)$
\item Найдите $\E(V)$, $\E(M)$, $\Var(M)$
\end{enumerate}

\item Польский математик Стефан Банах имел привычку носить в каждом из двух карманов пальто по коробку спичек. Всякий раз, когда ему хотелось закурить трубку, он выбирал наугад один из коробков и доставал из него спичку. Первоначально в каждом коробке было по $n$ спичек. Но когда-то наступает момент, когда выбранный наугад коробок оказывается пустым.

\begin{enumerate}
\item Какова вероятность того, что в другом коробке в этот момент осталось ровно $k$ спичек?
\item Каково среднее количество спичек в другом коробке?
\end{enumerate}

\item Производитель чудо-юдо-йогуртов наклеивает на каждую упаковку одну из 50 случайно выбираемых наклеек. Покупатель собравший все наклейки получает приз от производителя. Пусть $X$ --- это количество упаковок йогурта, которое нужно купить, чтобы собрать все наклейки.

Найдите $\P(X=50)$, $\E(X)$, $\Var(X)$

Hint: $\ln(50)\approx 3.91$, а $\sum_{i=1}^n \frac{1}{i} \approx \int_1^n \frac{1}{x}\, dx$ :)


\item В самолете $n$ мест и все билеты проданы. Первой в очереди на посадку стоит Сумасшедшая Старушка. Сумасшедшая Старушка несмотря на билет садиться на случайно выбираемое место. Каждый оставшийся пассажир садится на своё место, если оно свободно и на случайное выбираемое место, если его место уже кем-то занято.

\begin{enumerate}
\item Какова вероятность того, что все пассажиры сядут на свои места?
%\item Какова вероятность того, что второй пассажир в очереди сядет на своё место? 
\item Какова вероятность того, что последний пассажир сядет на своё место?
\item Чему примерно равно среднее количество пассажиров севших на свои места?
\end{enumerate}


\end{enumerate}

\end{document}