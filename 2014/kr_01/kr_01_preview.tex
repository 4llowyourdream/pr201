\documentclass[12pt,a4paper]{article}
\usepackage[utf8]{inputenc}
\usepackage[russian]{babel}

\usepackage{amsmath}
\usepackage{amsfonts}
\usepackage{amssymb}
\usepackage[left=2cm,right=2cm,top=2cm,bottom=2cm]{geometry}

\def \P{\mathbb{P}}
\def \Var{\mathbb{V}\mathrm{ar}}
\def \Cov{\mathbb{C}\mathrm{ov}}
\def \E{\mathbb{E}}

\begin{document}
\thispagestyle{empty}
\subsubsection*{Праздник номер один по теории вероятностей. Часть 2}


\begin{enumerate}
\item Маша ..... ................ ........ ............. ....... ......... ............... в сумме ....... ................ ............ ......... ........ ....... .......... .... случайные величины: $N$ --- ........ ........ ............. .............. ........... .......... ........., а $S$ --- ............ .............. ............ ................. ........ ...........  
\begin{enumerate}
\item Найдите $\P(N=1)$, $\P(N=2)$, $\P(N=3)$, 
\item Найдите $\E(N)$, $\E(S)$, $\E(N^2)$
\item Пусть $X_N$ ---  .......... .......... ....... .......... ................. ......... ........... Как распределена случайная величина $X_N$?
\end{enumerate}


\item ................. ..............  пришли ....... ............... в случайном порядке. Среди .. ......... ......... .............. ...............  .................. ......... Пусть $V$ --- это количество ......... ............... ............. ............ ............... .... ....., а $M\geq 0$ --- количество ............ .......... ........... .............. .......... ...... ....... .... 
\begin{enumerate}
\item Найдите $\P(V=1)$, $\P(M=1)$, $\P(M=V)$
\item Найдите $\E(V)$, $\E(M)$, $\Var(M)$
\end{enumerate}

\item .............. ............ .................... ............... ............ имел ....... ............... .............. ....... ........ ............ ..... ...... ....... раз, ................ .... ................. ........... ..... .............. ......, .... выбирал наугад один из ............ .............. ............ ............. .......... ............. Первоначально ................ ................ .................   $n$ ...... .......... .......... ....... .............. ............. ....... ...... выбранный наугад ............. .... .............. ............. ........ ....... .......... .....

\begin{enumerate}
\item Какова вероятность того, что в .............. ............ .............  ................. осталось ровно $k$ ......?
\item Каково среднее количество ......... ......... .....................  .............. ....................?
\end{enumerate}

\item  .......... ........... ................... ............. ....... на каждую .............  ... . . ......... 50 случайно выбираемых ......... .. ........ ......... все ...... .. . ......... ............. ............................... ............. ....... ......... . Пусть $X$ --- это количество ........ .................... .................. ............. ............... ............. ............... ................ ................ .............

Найдите $\P(X=50)$, $\E(X)$, $\Var(X)$

Hint:  ....... ....... ................. ............ ..........  ............ ........ ......... ............... ................... :)


\item В ............  ................. ............все ..... ........ проданы.  ................ ......... ............. ................. .................. несмотря ...................  ...... .................... ............ ................ случайно выбираемое ........... .......... .... ........ ....... .......... ........ ........ ....... Каждый ......... ........... ........... ............... .......................... ....., если ........... ...... ......... ............. ............... ............. ...... случайное ................. .......... ......... ................ ............ ........., если   ............. ....... ............ ........ .......... ..........

\begin{enumerate}
\item Какова вероятность того, что .........................................................................?
%\item Какова вероятность того, что второй пассажир в очереди сядет на своё место? 
\item Какова вероятность того, что .........................................................................?
\item Чему примерно равно среднее .........................................................................?
\end{enumerate}


\end{enumerate}

\end{document}