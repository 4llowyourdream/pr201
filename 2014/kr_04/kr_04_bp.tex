\documentclass[12pt,a4paper]{article}
\usepackage[utf8]{inputenc}
\usepackage[russian]{babel}

\usepackage{amsmath}
\usepackage{amsfonts}
\usepackage{amssymb}
\usepackage[left=1cm,right=1cm,top=1cm,bottom=1cm]{geometry}
\usepackage{enumitem}

\setlist[enumerate,1]{label=\arabic*., ref=\arabic*, partopsep=0pt plus 2pt, topsep=0pt plus 1.5pt,itemsep=0pt plus .5pt,parsep=0pt plus .5pt}
\setlist[itemize,1]{partopsep=0pt plus 2pt, topsep=0pt plus 1.5pt,itemsep=0pt plus .5pt,parsep=0pt plus .5pt}



\begin{document}
\thispagestyle{empty}
Вариант А.

\vspace{0.5cm}
%Задача 5. 
%
%Пусть $X_1$, \ldots, $X_n$ --- случайная выборка из нормального распределения с математическим ожиданием $\mu$ и дисперсией $\nu$, где $\mu$ и $\nu$ --- неизвестные параметры. Реализация случайной выборки из 5 наблюдений: -1.2, -2.3, 0.5, -1.3, -0.5.
%
%При помощи теста отношения правдоподобия протестируйте гипотезу $H_0: \mu=0$ на уровне значимости 5\%.


\textbf{Задача 1 (для первого потока).}

Проверка  40 случайно выбранных лекций показала, что студент Халявин присутствовал только на 16 из них. 
\begin{enumerate}
\item Найдите 95\% доверительный интервал для вероятности увидеть Халявина на лекции.
\item На уровне значимости 5\% проверьте гипотезу о том, что Халявин посещает в среднем половину лекций. 
\item Вычислите минимальный уровень значимости, при котором основная гипотеза отвергается (P-значение).
\end{enumerate}

\vspace{0.5cm}
 
\textbf{Задача 1 (для второго потока).}

Вес упаковки с лекарством является нормальной случайной величиной. Взвешивание 20~упаковок показало, что выборочное среднее равно 51 г., а  несмещенная оценка дисперсии равна~4. 
\begin{enumerate}
\item На уровне значимости 10\% проверьте гипотезу, что в среднем вес упаковки составляет~55 г.    
\item Контрольное взвешивание 30 упаковок такого же лекарства другого производителя показало, что несмещенная оценка дисперсии веса равна 6. На уровне значимости 10\% проверьте гипотезу о равенстве дисперсий веса упаковки двух производителей. 
\end{enumerate}
	
\vspace{0.5cm}

\textbf{Задача 2 (для первого потока).}

В ходе анкетирования  15 сотрудников банка «Альфа» ответили на вопрос о том, сколько времени они проводят на работе ежедневно. Среднее выборочное оказалось равно 9.5 часам при выборочном стандартном отклонении 0.5 часа. Аналогичные показатели для 12 сотрудников банка «Бета» составили 9.8 и 0.6 часа соответственно. 

Считая распределение времени нормальным, на уровне значимости 5\% проверьте гипотезу о том, что сотрудники банка «Альфа» в среднем проводят на работе столько же времени, сколько и сотрудники банка «Бета». 

\vspace{0.5cm}

\textbf{Задача 2 (для второго потока).}

Экзамен принимают два преподавателя, случайным образом выбирая студентов. По выборке из 85 и 100 наблюдений, выборочные доли не сдавших экзамен студентов составили соответственно 0.2  и 0.17. 
\begin{enumerate}
\item Можно ли при уровне значимости в 1\% утверждать, что преподаватели предъявляют к студентам одинаковый уровень требований? 
\item Вычислите минимальный уровень значимости, при котором основная гипотеза отвергается (P-значение).
\end{enumerate}

\vspace{0.5cm}

\textbf{Задача 3 (общая).}


Методом максимального правдоподобия найдите оценку параметра $\theta$ для выборки $X_1$, \ldots, $X_n$ из распределения с функцией плотности
\[
f(x)=\begin{cases}
\frac{1}{\theta^2}xe^{-\frac{x}{\theta}}, \; x>0 \\
0, \; x\leq 0
\end{cases}
\]

\vspace{0.5cm}

\textbf{Задача 4 (общая).}

Пусть $X_1$, \ldots, $X_{100}$ --- случайная выборка из нормального распределения с математическим ожиданием $\mu$ и дисперсией $\nu$, где $\mu$ и $\nu$ --- неизвестные параметры. По 100 наблюдениям $\sum x_i=30$, $\sum x_i^2=146$, $\sum x_i^3=122$. 

При помощи теста отношения правдоподобия протестируйте гипотезу $H_0: \nu=1$ на уровне значимости 5\%. 

\newpage
\thispagestyle{empty}
Вариант B.

%Задача 5. 
%
%Пусть $X_1$, \ldots, $X_n$ --- случайная выборка из нормального распределения с математическим ожиданием $\mu$ и дисперсией $\nu$, где $\mu$ и $\nu$ --- неизвестные параметры. Реализация случайной выборки из 5 наблюдений: -1.2, -2.3, 0.5, -1.3, -0.5.
%
%При помощи теста отношения правдоподобия протестируйте гипотезу $H_0: \mu=0$ на уровне значимости 5\%.

\vspace{0.5cm}

\textbf{Задача 1 (для первого потока).}

Проверка  40 случайно выбранных лекций показала, что студент Халявин присутствовал только на 16 из них. 
\begin{enumerate}
\item Найдите 95\% доверительный интервал для вероятности увидеть Халявина на лекции.
\item На уровне значимости 5\% проверьте гипотезу о том, что Халявин посещает в среднем половину лекций. 
\item Вычислите минимальный уровень значимости, при котором основная гипотеза отвергается (P-значение).
\end{enumerate}

\vspace{0.5cm}
 
\textbf{Задача 1 (для второго потока).}

Вес упаковки с лекарством является нормальной случайной величиной. Взвешивание 20~упаковок показало, что выборочное среднее равно 51 г., а  несмещенная оценка дисперсии равна~4. 
\begin{enumerate}
\item На уровне значимости 10\% проверьте гипотезу, что в среднем вес упаковки составляет~55 г.    
\item Контрольное взвешивание 30 упаковок такого же лекарства другого производителя показало, что несмещенная оценка дисперсии веса равна 6. На уровне значимости 10\% проверьте гипотезу о равенстве дисперсий веса упаковки двух производителей. 
\end{enumerate}
	
\vspace{0.5cm} 

\textbf{Задача 2 (для первого потока).}

В ходе анкетирования  15 сотрудников банка «Альфа» ответили на вопрос о том, сколько времени они проводят на работе ежедневно. Среднее выборочное оказалось равно 9.5 часам при выборочном стандартном отклонении 0.5 часа. Аналогичные показатели для 12 сотрудников банка «Бета» составили 9.8 и 0.6 часа соответственно. 

Считая распределение времени нормальным, на уровне значимости 5\% проверьте гипотезу о том, что сотрудники банка «Альфа» в среднем проводят на работе столько же времени, сколько и сотрудники банка «Бета». 

\vspace{0.5cm}

\textbf{Задача 2 (для второго потока).}

Экзамен принимают два преподавателя, случайным образом выбирая студентов. По выборке из 85 и 100 наблюдений, выборочные доли не сдавших экзамен студентов составили соответственно 0.2  и 0.17. 
\begin{enumerate}
\item Можно ли при уровне значимости в 1\% утверждать, что преподаватели предъявляют к студентам одинаковый уровень требований? 
\item Вычислите минимальный уровень значимости, при котором основная гипотеза отвергается (P-значение).
\end{enumerate}

\vspace{0.5cm}

\textbf{Задача 3 (общая).}


Методом максимального правдоподобия найдите оценку параметра $\theta$ для выборки $X_1$, \ldots, $X_n$ из распределения с функцией плотности
\[
f(x)=\begin{cases}
\frac{1}{\theta}xe^{-\frac{x}{\sqrt{\theta}}}, \; x>0 \\
0, \; x\leq 0
\end{cases}
\]

\vspace{0.5cm}

\textbf{Задача 4 (общая).}

Пусть $X_1$, \ldots, $X_{100}$ --- случайная выборка из нормального распределения с математическим ожиданием $\mu$ и дисперсией $\nu$, где $\mu$ и $\nu$ --- неизвестные параметры. По 100 наблюдениям $\sum x_i=30$, $\sum x_i^2=146$, $\sum x_i^3=122$. 

При помощи теста отношения правдоподобия протестируйте гипотезу $H_0: \mu=0$ на уровне значимости 5\%.

\newpage

\textbf{Задача 5 (исследовательская).}

\vspace{0.1cm}

Пусть $X_1$, \ldots, $X_{n}$ --- случайная выборка из нормального распределения с математическим ожиданием $\mu$ и дисперсией $\nu$, где $\mu$ и $\nu$ --- неизвестные параметры. Рассмотрим три классических теста, отношения правдоподобия, $LR$, множителей Лагранжа, $LM$ и Вальда, $W$, для тестирования гипотезы $H_0: \; \mu=0$. 

\begin{enumerate}
\item Сравните  статистики $LR$, $LM$ и $W$ между собой. Какая --- наибольшая, какая --- наименьшая? 
\item Изменится ли упорядоченность статистик, если проверять гипотезу $H_0: \; \mu=\mu_0$?
\end{enumerate}

\vspace{0.5cm}

Подсказка: $
\frac{x}{1+x} \leq \ln(1+x) \leq x\, \; \text{ при } x>-1
$

\vspace{0.5cm}

\textbf{Задача 6 (исследовательская).}
\thispagestyle{empty}
\vspace{0.1cm}

Величины $X_1$, \ldots, $X_n$ независимы и одинаково распределены с функцией плотности
\[
f(x)=\begin{cases}
a^2xe^{-ax}, \; x>0 \\
0, \; x\leq 0
\end{cases}
\]

По выборке из 100 наблюдений оказалось, что $\sum x_i =300$, $\sum x_i^2=1000$, $\sum x_i^3=3700$.

\begin{enumerate}
\item Найдите оценку неизвестного параметра $a$ методом моментов
\item Используя дельта-метод или иначе оцените дисперсию полученной оценки $a$
\item Постройте 95\%-ый доверительный интервал используя оценку метода моментов
\end{enumerate}




\end{document}
