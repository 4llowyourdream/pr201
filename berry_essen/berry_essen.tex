На бытовом языке центральная предельная теорема формулируется так:

При большом $n$ распределение $\bar{X}_{n}$ похоже на нормальное.

Возникает естественный вопрос - {}``А большое $n$ - это сколько?''.
Кто-то говорит 30, кто-то 60... Пора покончить с этим безобразием!

Ответ даёт теорема Берри-Эссена (Berry-Essen):

Если $X_{i}$ независимы и одинаково распределены, $Z_{n}=\frac{\bar{X}_{n}-\mu}{\sqrt{\frac{\sigma^{2}}{n}}}$,
$Z\sim N(0;1)$, то: 

\begin{equation}
|\P(Z_{n}\leq t)-\P(Z\leq t)|\leq c\cdot\frac{\E(|X_{i}-\mu|^{3})}{\sigma^{3}\sqrt{n}}
\end{equation}


На момент создания этих заметок (март 2011) про константу $c$ известно
, что она лежит где-то в интервале $[0,4097;0,4784]$. 

Сама центральная предельная теорема утверждает только то, что:

\[
\P(Z_{n}\leq t)\underset{n\to\infty}{\longrightarrow}\P(Z\leq t)
\]


Поскольку 

\begin{multline}
|\P(Z_{n}\in[a;b])-\P(Z\in[a;b])|=|\P(Z_{n}\leq b)-\P(Z_{n}\leq a)-(\P(Z\leq b)-\P(Z\leq a))|=\\
=|\P(Z_{n}\leq b)-\P(Z\leq b)+\P(Z\leq a)-\P(Z_{n}\leq a)|\leq\\
\leq|\P(Z_{n}\leq b)-\P(Z\leq b)|+|\P(Z\leq a)-\P(Z_{n}\leq a)|\leq2c\cdot\frac{\E(|X_{i}-\mu|^{3})}{\sigma^{3}\sqrt{n}}
\end{multline}


Для простоты можно завысить $c$ и считать его равным $0,5$. Тогда
мы получаем:

\begin{equation}
|\P(Z_{n}\in[a;b])-\P(Z\in[a;b])|\leq\frac{\E(|X_{i}-\mu|^{3})}{\sigma^{3}\sqrt{n}}\label{eq:ab_error}
\end{equation}


Давайте применим эту теорему к биномиальному распределению:

\begin{tabular}{|c|c|c|}
\hline 
$X_{i}$ & 0 & 1\tabularnewline
\hline 
\hline 
Prob & $1-p$ & $p$\tabularnewline
\hline 
\end{tabular}

В этом случае:$\E(X_{i})=p$, $\Var(X_{i})=p(1-p)$, $\sigma=\sqrt{p(1-p)}$
и $\E(|X_{i}-p|^{3})=p(1-p)(p^{2}+(1-p)^{2})$:

\begin{equation}
\frac{\E(|X_{i}-\mu|^{3})}{\sigma^{3}\sqrt{n}}=\frac{p(1-p)(p^{2}+(1-p)^{2})}{(p(1-p))^{3/2}\sqrt{n}}=\frac{p^{2}+(1-p)^{2}}{\sqrt{p(1-p)n}}
\end{equation}


Для наглядности несколько цифр%
\footnote{Погрешность посчитана по формуле \ref{eq:ab_error}, т.е. при завышенном
$c$. Фактическая погрешность может быть гораздо меньше.%
}:

\begin{tabular}{|c|c|c|}
\hline 
$n$ & $p$ & Погрешность при оцеке $\P(\bar{X}_{n}\in[a;b])$\tabularnewline
\hline 
\hline 
50 & 0.5 & \tabularnewline
\hline 
100 & 0.5 & \tabularnewline
\hline 
500 & 0.1 & \tabularnewline
\hline 
1000 & 0.1 & \tabularnewline
\hline 
\end{tabular}

Аналогичный вопрос возникает при замене биномиального распределения
на распределение Пуассона. В этом случае аналогичная теорема имеет
вид:

Если $X\sim Bin(n,p)$ и $Y\sim Poisson(\lambda=np)$, то:

$|\P(X\in A)-\P(Y\in A)|\leq$

Кстати говоря, на пуассоновское можно заменять не только биномиальное
распределение, но и другие похожие

...

Доказательства для любопытных...

Упражнения
