\documentclass[12pt,a4paper]{amsart}

\include{title_bor}


\begin{document}
Предварительный тест \\
1. В лотерее случайным образом выбираются 6 номеров из 100. Петя поставил на 1, 2, 3, 4, 5, 6, а Паша - на 7, 23, 37, 59, 62, 91. У кого больше шанс выиграть? \\
а) У Пети \\
б) У Паши \\
в) Одинаковы \\
2. Обычную рублевую монетку подбрасывают четыре раза. Первые три раза она выпала орлом. Вероятность того, что она выпадет орлом в четвертый раз: \\
а) больше 0.5 \\
б) меньше 0.5 \\
в) равна 0.5 \\
3. Два обычных игральных кубика подбрасываются одновременно. Больше шансы выпасть у комбинации: \\
а) две шестерки \\
б) одна шестерка, одна пятерка \\
в) одинаковые шансы \\
4. У Пети связка ключей. Один из них подходит к замку. Петя не знает, какой ключ подходит к замку и перебирает их по очереди. У какого ключа выше шансы подойти? \\
а) у первого \\
б) у последнего \\
в) одинаковы \\
5. В маленьком городке в среднем рождается 15 человек ежедневно. В большом городке - в среднем 45 человек ежедневно. Вероятность рождения мальчика примерно равна 0.5. На протяжении длительного времени в обоих городках считали дни, когда рождается больше 65\% мальчиков. \\
а) таких дней больше в маленьком городке \\
б) таких дней больше в большом городке \\
в) примерно одинаково \\
6. Какая вероятность выше? \\
а) выпадения как минимум двух орлов при трех подбрасываниях монетки \\
б) выпадения как минимум 200 орлов при 300 подбрасываниях монетки \\
в) одинаковы \\
7. В каком случае больше вариантов? \\
а) выбрать 2-х человек из группы в 10 человек \\
б) выбрать 8-х человек из группы в 10 человек \\
в) количество вариантов совпадает \\
8. У Пети 4 ореха. Из них два, не ясно какие, пустые. Петя разбивает первый орех, он оказывается пустым. Вероятность того, что второй орех будет пустым\\
а) больше 0.5 \\
б) меньше 0.5 \\
в) равна 0.5 \\
9. У Паши 4 ореха. Из них два, не ясно какие, пустые. Паша разбивает первый орех, и затем, не глядя на результат, разбивает второй. Второй разбитый орех - пустой. Вероятность того, что первый разбитый орех был пустым? \\
а) больше 0.5 \\
б) меньше 0.5 \\
в) равна 0.5 \\
10. Редкой болезнью болеет 0.01\% населения. Существующий тест ошибается в 10\% случаев. У первого встречного берут тест. Судя по тесту, человек болен. Какова вероятность того, что он действительно болен? \\
а) больше 0.5 \\
б) меньше 0.5 \\
в) равна 0.5 \\

По мотивам: \\
Fischbein, Schnarch, Evolution with age of probabilistic, intuitively based misconceptions \\
http://www.jstor.org/stable/749665 \\
Journal for research in mathematics education, vol 28, n 1, (jan 1997) \\


\end{document}