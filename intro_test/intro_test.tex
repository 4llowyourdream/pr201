\documentclass{article}
\usepackage[paper=a4paper,top=15mm, bottom=13.5mm,left=16.5mm,right=13.5mm,includefoot]{geometry}

\usepackage[box, % запрет на перенос вопросов
nopage, % убираем колонтитулы страницы
insidebox, % ставим буквы в квадратики
separateanswersheet, % добавляем бланк ответов
nowatermark, % отсутствие надписи "Черновик"
%indivanswers,  % показываем верные ответы
%answers,
lang=RU,
completemulti]{automultiplechoice}
\usepackage{multicol}
\usepackage[utf8]{inputenc}
\usepackage[russian]{babel}
\usepackage{comment}
\usepackage{amsmath, amssymb}

\renewcommand{\P}{\mathbb{P}}
\newcommand{\E}{\mathbb{E}}
\newcommand{\Var}{\mathrm{Var}}
\newcommand{\Cov}{\mathrm{Cov}}
\newcommand{\Corr}{\mathrm{Corr}}

\begin{document}

% \AMCnoCompleteMulti % отменяет "Нет верного ответа"
% [o] не тасует ответы


\begin{question}{intro_01}
В лотерее случайным образом выбираются 6 номеров из 100. Петя поставил на 1, 2, 3, 4, 5, 6, а Паша — на 7, 23, 37, 59, 62, 91. У кого больше шанс выиграть?
  \begin{choiceshoriz}
    \correctchoice{Одинаковы}
    \wrongchoice{У Паши}
    \wrongchoice{У Пети}
  \end{choiceshoriz}
\end{question}


\begin{question}{intro_02}
Обычную рублевую монетку подбрасывают четыре раза. Первые три раза она выпала орлом. Вероятность того, что она выпадет орлом в четвертый раз:
  \begin{choiceshoriz}
    \correctchoice{равна 0.5}
    \wrongchoice{больше 0.5}
    \wrongchoice{меньше 0.5}
  \end{choiceshoriz}
\end{question}


\begin{question}{intro_03}
Два обычных игральных кубика подбрасываются одновременно. Больше шансы выпасть у комбинации:
  \begin{choiceshoriz}[o]
  \wrongchoice{две шестерки}
    \correctchoice{одна шестерка, одна пятерка}
    \wrongchoice{одинаковые шансы}
  \end{choiceshoriz}
\end{question}


\begin{question}{intro_04}
У Пети связка ключей. Один из них подходит к замку. Петя не знает, какой ключ подходит к замку и перебирает их по очереди. У какого ключа выше шансы подойти?
  \begin{choiceshoriz}[o]
    \wrongchoice{у первого}
    \wrongchoice{у последнего}
    \correctchoice{одинаковы}
  \end{choiceshoriz}
\end{question}

\begin{question}{intro_05}
Вероятность рождения мальчика примерно равна 0.5. На протяжении длительного времени в маленьком городе и большом городе считали дни, когда рождается больше 65\% мальчиков. Таких дней окажется больше
  \begin{choiceshoriz}
    \correctchoice{в маленьком городе}
    \wrongchoice{в большом городе}
    \wrongchoice{примерно одинаково}
  \end{choiceshoriz}
\end{question}


\begin{question}{intro_06}
Какая вероятность выше?
  \begin{choices}
    \correctchoice{выпадения как минимум двух орлов при трех подбрасываниях монетки}
    \wrongchoice{выпадения как минимум 200 орлов при 300 подбрасываниях монетки}
    \wrongchoice{примерно одинаковы}
  \end{choices}
\end{question}


\begin{question}{intro_07}
У Пети и у Паши по 10 книг. Петя с собой берёт две, а Паша --- восемь. У кого больше возможных способов выбора?
  \begin{choiceshoriz}
    \correctchoice{одинаково}
    \wrongchoice{у Пети}
    \wrongchoice{у Паши}
  \end{choiceshoriz}
\end{question}


\begin{question}{intro_08}
У Пети 4 ореха. Из них два, не ясно какие, пустые. Петя разбивает первый орех, он оказывается пустым. Вероятность того, что второй орех будет пустым
  \begin{choiceshoriz}
    \correctchoice{меньше 0.5}
    \wrongchoice{больше 0.5}
    \wrongchoice{равна 0.5}
  \end{choiceshoriz}
\end{question}


\begin{question}{intro_09}
У Паши 4 ореха. Из них два, не ясно какие, пустые. Паша разбивает первый орех, и затем, не глядя на результат, разбивает второй. Второй разбитый орех --- пустой. Вероятность того, что первый разбитый орех был пустым?
  \begin{choiceshoriz}
    \correctchoice{меньше 0.5}
    \wrongchoice{больше 0.5}
    \wrongchoice{равна 0.5}
  \end{choiceshoriz}
\end{question}


\begin{question}{intro_10}
Редкой болезнью болеет 0.01\% населения. Существующий тест ошибается в 10\% случаев. У первого встречного берут тест. Судя по тесту, человек болен. Какова вероятность того, что он действительно болен?
  \begin{choiceshoriz}
    \correctchoice{меньше 0.5}
    \wrongchoice{больше 0.5}
    \wrongchoice{равна 0.5}
  \end{choiceshoriz}
\end{question}


\textbf{Открытые вопросы}:

\begin{enumerate}
\item Что такое $\pi$? Чему примерно равно $\pi$?
\item Что такое $e$? Чему примерно равно $e$?
\end{enumerate}

\newpage
По мотивам: \\
Fischbein, Schnarch, Evolution with age of probabilistic, intuitively based misconceptions \\
http://www.jstor.org/stable/749665 \\
Journal for research in mathematics education, vol 28, n 1, (jan 1997) \\


\end{document}
