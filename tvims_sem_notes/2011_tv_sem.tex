\documentclass[pdftex,12pt,a4paper]{article}

%%%%%%%%%%%%%%%%%%%%%%%  Загрузка пакетов  %%%%%%%%%%%%%%%%%%%%%%%%%%%%%%%%%%

%\usepackage{showkeys} % показывать метки в готовом pdf 

\usepackage{etex} % расширение классического tex
% в частности позволяет подгружать гораздо больше пакетов, чем мы и займёмся далее

\usepackage{cmap} % для поиска русских слов в pdf
\usepackage{verbatim} % для многострочных комментариев
\usepackage{makeidx} % для создания предметных указателей
\usepackage[X2,T2A]{fontenc}
\usepackage[utf8]{inputenc} % задание utf8 кодировки исходного tex файла
\usepackage{setspace}
\usepackage{amsmath,amsfonts,amssymb,amsthm}
\usepackage{mathrsfs} % sudo yum install texlive-rsfs
\usepackage{dsfont} % sudo yum install texlive-doublestroke
\usepackage{array,multicol,multirow,bigstrut} % sudo yum install texlive-multirow
\usepackage{indentfirst} % установка отступа в первом абзаце главы
\usepackage[british,russian]{babel} % выбор языка для документа
\usepackage{bm}
\usepackage{bbm} % шрифт с двойными буквами
\usepackage[perpage]{footmisc}

% создание гиперссылок в pdf
\usepackage[pdftex,unicode,colorlinks=true,urlcolor=blue,hyperindex,breaklinks]{hyperref} 

% свешиваем пунктуацию 
% теперь знаки пунктуации могут вылезать за правую границу текста, при этом текст выглядит ровнее
\usepackage{microtype}

\usepackage{textcomp}  % Чтобы в формулах можно было русские буквы писать через \text{}

% размер листа бумаги
\usepackage[paper=a4paper,top=13.5mm, bottom=13.5mm,left=16.5mm,right=13.5mm,includefoot]{geometry}

\usepackage{xcolor}

\usepackage[pdftex]{graphicx} % для вставки графики 

\usepackage{float,longtable}
\usepackage{soulutf8}

\usepackage{enumitem} % дополнительные плюшки для списков
%  например \begin{enumerate}[resume] позволяет продолжить нумерацию в новом списке

\usepackage{mathtools}
\usepackage{cancel,xspace} % sudo yum install texlive-cancel

\usepackage{minted} % display program code with syntax highlighting
% требует установки pygments и python 

\usepackage{numprint} % sudo yum install texlive-numprint
\npthousandsep{,}\npthousandthpartsep{}\npdecimalsign{.}

\usepackage{embedfile} % Чтобы код LaTeXа включился как приложение в PDF-файл

\usepackage{subfigure} % для создания нескольких рисунков внутри одного

\usepackage{tikz,pgfplots} % язык для рисования графики из latex'a
\usetikzlibrary{trees} % tikz-прибамбас для рисовки деревьев
\usepackage{tikz-qtree} % альтернативный tikz-прибамбас для рисовки деревьев
\usetikzlibrary{arrows} % tikz-прибамбас для рисовки стрелочек подлиннее

\usepackage{todonotes} % для вставки в документ заметок о том, что осталось сделать
% \todo{Здесь надо коэффициенты исправить}
% \missingfigure{Здесь будет Последний день Помпеи}
% \listoftodos --- печатает все поставленные \todo'шки


% более красивые таблицы
\usepackage{booktabs}
% заповеди из докупентации: 
% 1. Не используйте вертикальные линни
% 2. Не используйте двойные линии
% 3. Единицы измерения - в шапку таблицы
% 4. Не сокращайте .1 вместо 0.1
% 5. Повторяющееся значение повторяйте, а не говорите "то же"



%\usepackage{asymptote} % пакет для рисовки графики, должен идти после graphics
% но мы переходим на tikz :)

%\usepackage{sagetex} % для интеграции с Sage (вероятно тоже должен идти после graphics)

% metapost создает упрощенные eps файлы, которые можно напрямую включать в pdf 
% эта группа команд декларирует, что файлы будут этого упрощенного формата
% если metapost не используется, то этот блок не нужен
\usepackage{ifpdf} % для определения, запускается ли pdflatex или просто латех
\ifpdf
	\DeclareGraphicsRule{*}{mps}{*}{}
\fi
%%%%%%%%%%%%%%%%%%%%%%%%%%%%%%%%%%%%%%%%%%%%%%%%%%%%%%%%%%%%%%%%%%%%%%


%%%%%%%%%%%%%%%%%%%%%%%  Внедрение tex исходников в pdf файл  %%%%%%%%%%%%%%%%%%%%%%%%%%%%%%%%%%
\embedfile[desc={Main tex file}]{\jobname.tex} % Включение кода в выходной файл
\embedfile[desc={title_bor}]{/home/boris/science/tex_general/title_bor_utf8.tex}

%%%%%%%%%%%%%%%%%%%%%%%%%%%%%%%%%%%%%%%%%%%%%%%%%%%%%%%%%%%%%%%%%%%%%%



%%%%%%%%%%%%%%%%%%%%%%%  ПАРАМЕТРЫ  %%%%%%%%%%%%%%%%%%%%%%%%%%%%%%%%%%
\setstretch{1}                          % Межстрочный интервал
\flushbottom                            % Эта команда заставляет LaTeX чуть растягивать строки, чтобы получить идеально прямоугольную страницу
\righthyphenmin=2                       % Разрешение переноса двух и более символов
\pagestyle{plain}                       % Нумерация страниц снизу по центру.
\widowpenalty=300                     % Небольшое наказание за вдовствующую строку (одна строка абзаца на этой странице, остальное --- на следующей)
\clubpenalty=3000                     % Приличное наказание за сиротствующую строку (омерзительно висящая одинокая строка в начале страницы)
\setlength{\parindent}{1.5em}           % Красная строка.
%\captiondelim{. }
\setlength{\topsep}{0pt}
%%%%%%%%%%%%%%%%%%%%%%%%%%%%%%%%%%%%%%%%%%%%%%%%%%%%%%%%%%%%%%%%%%%%%%



%%%%%%%% Это окружение, которое выравнивает по центру без отступа, как у простого center
\newenvironment{center*}{%
  \setlength\topsep{0pt}
  \setlength\parskip{0pt}
  \begin{center}
}{%
  \end{center}
}
%%%%%%%%%%%%%%%%%%%%%%%%%%%%%%%%%%%%%%%%%%%%%%%%%%%%%%%%%%%%%%%%%%%%%%


%%%%%%%%%%%%%%%%%%%%%%%%%%% Правила переноса  слов
\hyphenation{ }
%%%%%%%%%%%%%%%%%%%%%%%%%%%%%%%%%%%%%%%%%%%%%%%%%%%%%%%%%%%%%%%%%%%%%%

\emergencystretch=2em


% DEFS
\def \mbf{\mathbf}
\def \msf{\mathsf}
\def \mbb{\mathbb}
\def \tbf{\textbf}
\def \tsf{\textsf}
\def \ttt{\texttt}
\def \tbb{\textbb}

\def \wh{\widehat}
\def \wt{\widetilde}
\def \ni{\noindent}
\def \ol{\overline}
\def \cd{\cdot}
\def \fr{\frac}
\def \bs{\backslash}
\def \lims{\limits}
\DeclareMathOperator{\dist}{dist}
\DeclareMathOperator{\VC}{VCdim}
\DeclareMathOperator{\card}{card}
\DeclareMathOperator{\sign}{sign}
\DeclareMathOperator{\sgn}{sign}
\DeclareMathOperator{\Tr}{\mbf{Tr}}
\def \xfs{(x_1,\ldots,x_{n-1})}
\DeclareMathOperator*{\argmin}{arg\,min}
\DeclareMathOperator*{\amn}{arg\,min}
\DeclareMathOperator*{\amx}{arg\,max}

\DeclareMathOperator{\Corr}{Corr}
\DeclareMathOperator{\Cov}{Cov}
\DeclareMathOperator{\Var}{Var}
\DeclareMathOperator{\corr}{Corr}
\DeclareMathOperator{\cov}{Cov}
\DeclareMathOperator{\var}{Var}
\DeclareMathOperator{\bin}{Bin}
\DeclareMathOperator{\Bin}{Bin}
\DeclareMathOperator{\rang}{rang}
\DeclareMathOperator*{\plim}{plim}
\DeclareMathOperator{\med}{med}


\providecommand{\iff}{\Leftrightarrow}
\providecommand{\hence}{\Rightarrow}

\def \ti{\tilde}
\def \wti{\widetilde}

\def \mA{\mathcal{A}}
\def \mB{\mathcal{B}}
\def \mC{\mathcal{C}}
\def \mE{\mathcal{E}}
\def \mF{\mathcal{F}}
\def \mH{\mathcal{H}}
\def \mL{\mathcal{L}}
\def \mN{\mathcal{N}}
\def \mU{\mathcal{U}}
\def \mV{\mathcal{V}}
\def \mW{\mathcal{W}}


\def \R{\mbb R}
\def \N{\mbb N}
\def \Z{\mbb Z}
\def \P{\mbb{P}}
\def \p{\mbb{P}}
\newcommand{\E}{\mathbb{E}}
\def \D{\msf{D}}
\def \I{\mbf{I}}

\def \a{\alpha}
\def \b{\beta}
\def \t{\tau}
\def \dt{\delta}
\newcommand{\e}{\varepsilon}
\def \ga{\gamma}
\def \kp{\varkappa}
\def \la{\lambda}
\def \sg{\sigma}
\def \sgm{\sigma}
\def \tt{\theta}
\def \ve{\varepsilon}
\def \Dt{\Delta}
\def \La{\Lambda}
\def \Sgm{\Sigma}
\def \Sg{\Sigma}
\def \Tt{\Theta}
\def \Om{\Omega}
\def \om{\omega}

%\newcommand{\p}{\partial}
\newcommand{\PP}{\mathbb{P}}

\def \ni{\noindent}
\def \lq{\glqq}
\def \rq{\grqq}
\def \lbr{\linebreak}
\def \vsi{\vspace{0.1cm}}
\def \vsii{\vspace{0.2cm}}
\def \vsiii{\vspace{0.3cm}}
\def \vsiv{\vspace{0.4cm}}
\def \vsv{\vspace{0.5cm}}
\def \vsvi{\vspace{0.6cm}}
\def \vsvii{\vspace{0.7cm}}
\def \vsviii{\vspace{0.8cm}}
\def \vsix{\vspace{0.9cm}}
\def \VSI{\vspace{1cm}}
\def \VSII{\vspace{2cm}}
\def \VSIII{\vspace{3cm}}

\newcommand{\bls}[1]{\boldsymbol{#1}}
\newcommand{\bsA}{\boldsymbol{A}}
\newcommand{\bsH}{\boldsymbol{H}}
\newcommand{\bsI}{\boldsymbol{I}}
\newcommand{\bsP}{\boldsymbol{P}}
\newcommand{\bsR}{\boldsymbol{R}}
\newcommand{\bsS}{\boldsymbol{S}}
\newcommand{\bsX}{\boldsymbol{X}}
\newcommand{\bsY}{\boldsymbol{Y}}
\newcommand{\bsZ}{\boldsymbol{Z}}
\newcommand{\bse}{\boldsymbol{e}}
\newcommand{\bsq}{\boldsymbol{q}}
\newcommand{\bsy}{\boldsymbol{y}}
\newcommand{\bsbeta}{\boldsymbol{\beta}}
\newcommand{\fish}{\mathrm{F}}
\newcommand{\Fish}{\mathrm{F}}
\renewcommand{\phi}{\varphi}
\newcommand{\ind}{\mathds{1}}
\newcommand{\inds}[1]{\mathds{1}_{\{#1\}}}
\renewcommand{\to}{\rightarrow}
\newcommand{\sumin}{\sum\limits_{i=1}^n}
\newcommand{\ofbr}[1]{\bigl( \{ #1 \} \bigr)}     % Например, вероятность события. Большие круглые, нормальные фигурные скобки вокруг аргумента
\newcommand{\Ofbr}[1]{\Bigl( \bigl\{ #1 \bigr\} \Bigr)} % Например, вероятность события. Больше больших круглые, большие фигурные скобки вокруг аргумента
\newcommand{\oeq}{{}\textcircled{\raisebox{-0.4pt}{{}={}}}{}} % Равно в кружке
\newcommand{\og}{\textcircled{\raisebox{-0.4pt}{>}}}  % Знак больше в кружке

% вместо горизонтальной делаем косую черточку в нестрогих неравенствах
\renewcommand{\le}{\leqslant}
\renewcommand{\ge}{\geqslant}
\renewcommand{\leq}{\leqslant}
\renewcommand{\geq}{\geqslant}


\newcommand{\figb}[1]{\bigl\{ #1  \bigr\}} % большие фигурные скобки вокруг аргумента
\newcommand{\figB}[1]{\Bigl\{ #1  \Bigr\}} % Больше больших фигурные скобки вокруг аргумента
\newcommand{\parb}[1]{\bigl( #1  \bigr)}   % большие скобки вокруг аргумента
\newcommand{\parB}[1]{\Bigl( #1  \Bigr)}   % Больше больших круглые скобки вокруг аргумента
\newcommand{\parbb}[1]{\biggl( #1  \biggr)} % большие-большие круглые скобки вокруг аргумента
\newcommand{\br}[1]{\left( #1  \right)}    % круглые скобки, подгоняемые по размеру аргумента
\newcommand{\fbr}[1]{\left\{ #1  \right\}} % фигурные скобки, подгоняемые по размеру аргумента
\newcommand{\eqdef}{\mathrel{\stackrel{\rm def}=}} % знак равно по определению
\newcommand{\const}{\mathrm{const}}        % const прямым начертанием
\newcommand{\zdt}[1]{\textit{#1}}
\newcommand{\ENG}[1]{\foreignlanguage{british}{#1}}
\newcommand{\ENGs}{\selectlanguage{british}}
\newcommand{\RUSs}{\selectlanguage{russian}}
\newcommand{\iid}{\text{i.\hspace{1pt}i.\hspace{1pt}d.}}

\newdimen\theoremskip
\theoremskip=0pt
\newenvironment{note}{\par\vskip\theoremskip\textbf{Замечание.\xspace}}{\par\vskip\theoremskip}
\newenvironment{hint}{\par\vskip\theoremskip\textbf{Подсказка.\xspace}}{\par\vskip\theoremskip}
\newenvironment{ist}{\par\vskip\theoremskip Источник:\xspace}{\par\vskip\theoremskip}

\newcommand*{\tabvrulel}[1]{\multicolumn{1}{|c}{#1}}
\newcommand*{\tabvruler}[1]{\multicolumn{1}{c|}{#1}}

\newcommand{\II}{{\fontencoding{X2}\selectfont\CYRII}}   % I десятеричное (английская i неуместна)
\newcommand{\ii}{{\fontencoding{X2}\selectfont\cyrii}}   % i десятеричное
\newcommand{\EE}{{\fontencoding{X2}\selectfont\CYRYAT}}  % ЯТЬ
\newcommand{\ee}{{\fontencoding{X2}\selectfont\cyryat}}  % ять
\newcommand{\FF}{{\fontencoding{X2}\selectfont\CYROTLD}} % ФИТА
\newcommand{\ff}{{\fontencoding{X2}\selectfont\cyrotld}} % фита
\newcommand{\YY}{{\fontencoding{X2}\selectfont\CYRIZH}}  % ИЖИЦА
\newcommand{\yy}{{\fontencoding{X2}\selectfont\cyrizh}}  % ижица

%%%%%%%%%%%%%%%%%%%%% Определение разрядки разреженного текста и задание красивых многоточий
\sodef\so{}{.15em}{1em plus1em}{.3em plus.05em minus.05em}
\newcommand{\ldotst}{\so{...}}
\newcommand{\ldotsq}{\so{?\hbox{\hspace{-0.61ex}}..}}
\newcommand{\ldotse}{\so{!..}}
%%%%%%%%%%%%%%%%%%%%%%%%%%%%%%%%%%%%%%%%%%%%%%%%%%%%%%%%%%%%%%%%%%%%%%

%%%%%%%%%%%%%%%%%%%%%%%%%%%%% Команда для переноса символов бинарных операций
\def\hm#1{#1\nobreak\discretionary{}{\hbox{$#1$}}{}}
%%%%%%%%%%%%%%%%%%%%%%%%%%%%%%%%%%%%%%%%%%%%%%%%%%%%%%%%%%%%%%%%%%%%%%

\setlist[enumerate,1]{label=\arabic*., ref=\arabic*, partopsep=0pt plus 2pt, topsep=0pt plus 1.5pt,itemsep=0pt plus .5pt,parsep=0pt plus .5pt}
\setlist[itemize,1]{partopsep=0pt plus 2pt, topsep=0pt plus 1.5pt,itemsep=0pt plus .5pt,parsep=0pt plus .5pt}

% Эти парни затем, если вдруг не захочется управлять списками из-под уютненького enumitem
% или если будет жизненно важно, чтобы в списках были именно русские буквы.
%\setlength{\partopsep}{0pt plus 3pt}
%\setlength{\topsep}{0pt plus 2pt}
%\setlength{\itemsep}{0 plus 1pt}
%\setlength{\parsep}{0 plus 1pt}

%на всякий случай пока есть
%теоремы без нумерации и имени
%\newtheorem*{theor}{Теорема}

%"Определения","Замечания"
%и "Гипотезы" не нумеруются
%\newtheorem*{defin}{Определение}
%\newtheorem*{rem}{Замечание}
%\newtheorem*{conj}{Гипотеза}

%"Теоремы" и "Леммы" нумеруются
%по главам и согласованно м/у собой
%\newtheorem{theorem}{Теорема}
%\newtheorem{lemma}[theorem]{Лемма}

% Утверждения нумеруются по главам
% независимо от Лемм и Теорем
%\newtheorem{prop}{Утверждение}
%\newtheorem{cor}{Следствие} 

%\emergencystretch=2em \voffset=-3cm \hoffset=-3cm
%\unitlength=0.6mm \textwidth=19cm \textheight=27cm

% делаем короче интервал в списках 
\setlength{\itemsep}{0pt} 
\setlength{\parskip}{0pt} 
\setlength{\parsep}{0pt}

% свешиваем пунктуацию (т.е. знаки пунктуации могут вылезать за правую границу текста, при этом текст выглядит ровнее)
\usepackage{microtype}



\newcounter{zadacha}[section]
%новые счетчик "zadacha" будет автоматом сбрасываться на 0 при старте нового раздела
\newcommand{\zad}{\par\refstepcounter{zadacha} \arabic{zadacha}. }
\renewcommand{\thezadacha}{\thesection.\arabic{zadacha}}

\begin{document}
\parindent=0 pt % отступ равен 0
% семинар 1 - 
св. --- случайная величина (rv. --- random variable) \\
$\Omega$ --- список всех возможных исходов случайного эксперимента \\
$\P(A)$ --- вероятность события $A$; $0\le \P(A)\le 1$; $A^{c}=\bar{A}$ --- отрицание события $A$ \\
Если $A$ и $B$ несовместны (не могут произойти одновременно), то $\P(A\cup
B)=\P(A)+\P(B)$; \\
$\E(X)$ --- среднее значение случайной величины $X$, математическое ожидание \\
%Функция $p(t)$ называется функцией плотности св. $X$, если $\P(X\in[a;b])=\int_{a}^{b}p(t)dt$; \\
%Если у св. $X$ есть функция плотности, то $\E(X)=\int_{-\infty}^{+\infty}t\cdot p(t)dt$ \\
$1_{A}$, индикатор события $A$, --- случайная величина, которая равна 
единице, если событие $A$ наступило, и нулю, если событие $A$ не наступило \\
%Число способов выбрать $k$ предметов из $n$ = $C_{n}^{k}=\frac{n!}{k!(n-k)!}$ \\ 

\zad Актёр-дед Мороз зарабатывает 100 тысяч рублей в январе и 10 тысяч рублей в остальные месяцы. Зарплату, приходящуюся на месяц рождения актёра, обозначим $X$. Найдите $\P(X=10)$ и $\E(X)$.

\zad Завтра реализуется одна из трёх возможностей, $\Omega=\{a,b,c\}$. 
Cлучайная величина $X$ --- температура $[^{o}C]$, а $Y$ --- влажность $[\%]$ завтра . \\
$\begin{array}{|c|c|c|c|}
\hline
\Omega & a & b & c \\
\hline
X & 6 & 8 & 10  \\
\hline 
Y & 70 & 50 & 80  \\
\hline
\P() & 0.2 & 0.5 & 0.3  \\
\hline
\end{array}$, \\
Найдите: \\
а) $\P(X>9)$, $\P(X<10)$, $\P(X>0)$, $\P(XY>700)$, $\P(\sqrt{Y}<8)$ \\
б) $\E(X)$, $\E(Y)$, $\E(X^2)$, $\E(0.1\cdot Y)$ \\
в) Сравните $\E(X^{2})$ и $\E(X)^{2}$; $\E(Y^{2})$ и $\E(Y)^{2}$;

\zad Величина $N$ --- количество 
шестерок выпадающих при двух подбрасываниях кубика. Найдите $\P(N=0)$, $\P(N=1)$, $\P(N=2)$, $\P(N=3)$, $\P(N\ge 1)$, $\E(N)$

\zad Две команды равной силы играют до 3-х побед. Ничья невозможна. Св. $N$ -- количество сыгранных партий. Составьте табличку возможных значений $N$ с их вероятностями (такая табличка называется законом распределения случайной величины). Найдите $\P(N$ --- четное$)$, $\E(N)$ 

\zad Какова вероятность того, что у 10 человек не будет ни одного
совпадения дней рождений?

\zad Наугад из четырех тузов разных мастей выбираются два.
$\P($они будут разного цвета$)$?

\zad События $A$  и  $B$ несовместны,  $\P(A)=0,3$, $\P(B)=0,4$. Найдите
$\P(A^{c} \cap B^{c} )$

\zad Вероятность $\P(A)=0,3$,  $\P(B)=0,8$. В каких пределах может лежать $\P(A\cap B)$?

\zad Множество исходов $\Omega =\left\{a,b,c\right\}$, $\P(\left\{a,b\right\})=0,8$,
$\P(\{b,c\})=0,7$. Найдите $\P(\{a\})$, $\P(\{b\})$, $\P(\{c\})$

\zad Мама в среднем получает 40 т.р. в месяц, папа - в среднем 50 т.р. Каков средний доход семьи? Как связаны между собой $\E(X)$, $\E(Y)$ и $\E(X+Y)$?

\zad На бумаге проведена прямая. На бумагу бросают спичку. Какова
вероятность, что острый угол между прямой и спичкой будет меньше
10 градусов?

\zad Вася бегает по кругу длиной 400 метров. В случайный момент
времени он останавливается. Какова вероятность того, что он будет
ближе, чем в 50 м от точки старта? Дальше, чем в 100 м?

\zad Треугольник с вершинами $(0;0)$, $(2;0)$ и $(1;1)$. Внутри него случайным образом выбирается точка, $X$ -- абсцисса точки. Найдите $\P(X>1)$, $\P(X\in [0.5;1])$, $\E(X)$

\zad Треугольник с вершинами $(0;0)$, $(2;0)$ и $(2;1)$. Внутри него случайным образом выбирается точка, $X$ -- абсцисса точки. Найдите $\P(X>1)$, $\P(X\in [0.5;1])$. Что больше, $\E(X)$ или 1?

\zad 
$\begin{array}{|c|c|c|c|}
\hline
X & 6 & 8 & 10  \\
\hline
\P() & 0.2 & 0.5 & 0.3  \\
\hline
\end{array}$, \\
Событие $A$ состоит в том, что $X>9$, $A=\{X>9\}$. Найдите $\E(1_{A})$, $\E(X\cdot 1_{A})$ 

\zad Как связаны $\E(1_{B})$ и $\P(B)$ для произвольного события $B$?

\zad Как связаны $\P(Z\le 9)$ и $\P(Z>9)$ для произвольной случайной величины $Z$?



\setcounter{zadacha}{0}
\newpage 
% семинар 2 -- комбинаторика

\zad Лев собрал 100 зверей. Сколькими способами их можно расставить в очередь ко льву?

\zad Лев собрал 100 зверей и решил их раскрасить, каждого целиком в один цвет. Лев хочет 20 красных, 30 желтых и 50 зеленых зверей. Сколько существует вариантов раскрасок?

\zad Из 50 деталей 4 бракованных. Выбирается наугад 10 на
проверку. Какова вероятность не заметить брак? Сколько в среднем бракованных деталей попадется? 

%\zad Сколькими способами можно расставить 5 человек в очередь?

\zad В клубе 25 человек. Сколькими способами можно выбрать: \\
а) комитет из 4-х человек? б) руководство, состоящее из
директора, зама и кассира?

\zad  Вася играет в преферанс. Он взял прикуп, снес две карты и
выбрал козыря. У Васи на руках четыре козыря. Какова вероятность,
что оставшиеся четыре козыря разделились 4:0, 3:1, 2:2? \\
Правила: из 32-х карт (нет шестерок) две кладут в прикуп, остальные раздают по 10 трем игрокам.

\zad
На столе стоят 4 отличающихся друг от друга чашки, 4 одинаковых граненых стакана,
10 одинаковых кусков сахара, 7 соломинок разных цветов. Сколькими способами можно полностью разложить: \\
а) сахар по чашкам; б) сахар по стаканам; в) соломинки по чашкам;
г) соломинки по стаканам; \\
д) Как изменятся ответы, если требуется, чтобы пустых емкостей не
оставалось? 

\zad В библиотеке Маше выдали 25 книг. Она решила прочесть 4 книги. Сколько вариантов выбора есть у Маши?



\setcounter{zadacha}{0}
\newpage 
% семинар 3 - разложение в сумму и первый шаг
\zad 
Из грота ведут 10 штреков, с длинами 100м, 200м,... 1000м. Самый длинный штрек оканчивается выходом на поверхность. Остальные - тупиком. Вася выбирает штреки наугад (естественно, в тупиковый штрек он два раза не ходит). Какова вероятность того, что Вася посетит самый короткий штрек? Какой в среднем путь он нагуляет прежде чем выберется на поверхность? 

\zad 
У Маши 30 разных пар туфель. И она говорит, что мало! Пес
Шарик утащил (без разбору на левые и правые) 17 туфель. Какова вероятность того, что у Маши останется 13 полных пар? Сколько
полных пар в среднем осталось? Сколько полных пар в среднем
досталось Шарику? 


\zad 
У меня в кармане 3 рубля мелочью. Среди монет всего одна монета достоинством 50 копеек. Я извлекаю монеты по одной наугад до извлечения 50 копеечной монеты. Какую сумму в среднем я извлеку? 

\zad <<Модница>>. 
В шкатулке у Маши 100 пар сережек. Каждый день утром она выбирает одну пару наугад, носит ее, а вечером возвращает в шкатулку. Проходит год. \\
а) Сколько в среднем пар окажутся ни разу не надетыми? \\
б) Сколько в среднем пар окажутся одетыми не менее двух раз? \\
в*) Как изменятся ответы, если каждый день Маша покупает себе новую пару сережек и вечером добавляет ее в шкатулку? 

\zad 
Вовочка получает пятерку с вероятностью 0.1, четверку - с вероятностью 0.2, тройку - с вероятностью - 0.3 и двойку с вероятностью 0.4. В этом четверти он писал 20 контрольных. Какова вероятность того, что все оценки у Вовочки одинаковые? Сколько разных оценок он в среднем получит? 

\zad <<Судьба Дон Жуана>>
У Васи  $n$  знакомых девушек (их всех зовут по-разному). Он пишет
им  $n$  писем, но, по рассеянности, раскладывает их в конверты
наугад. С.в.  $X$  обозначает количество девушек, получивших
письма, написанные лично для них. Найдите  $\E(X)$. 

\zad 
Над озером взлетело 20 уток. Каждый из 10 охотников
стреляет в утку по своему выбору. Каково ожидаемое количество
убитых уток, если охотники стреляют без промаха? Как изменится
ответ, если вероятность попадания равна 0,7? Каким будет ожидаемое
количество охотников, попавших в цель? 


\setcounter{zadacha}{0}
\newpage 
% семинар 3 -  первый шаг


\zad 
Неправильную монетку (вероятность <<орла>> равна $p$) подбрасывают до первого <<орла>>. 
 Чему равно среднее количество подбрасываний?  Орлов? Решек? 

\zad 
Саша и Маша по очереди подбрасывают кубик. Посуду будет
мыть тот, кто первым выбросит шестерку. Маша бросает первой.
Каковы ее шансы отдохнуть за <<Cosmo>>? 

\zad
Вы играете в следующую игру. Кубик подкидывается неограниченное число раз. Если на кубике выпадает 1, 2 или 3, то соответствующее количество монет добавляется на кон. Если выпадает 4 или 5, то игра оканчивается и Вы получаете сумму, лежащую на кону. Если выпадает 6, то игра оканчивается, а Вы не получаете ничего. \\
а) Чему равен ожидаемый выигрыш в эту игру? \\
б) Изменим условие: если выпадает 5, то набранная сумма сгорает, а игра начинается заново. Чему будет равен ожидаемый выигрыш? 

\zad 
Саша и Маша подкидывают монетку до тех пор, пока не выпадет
последовательность РОО или ОOР. Если игра закончится выпадением
РОО, то выигрывает Саша, если ОOР, то - Маша. \\
а) У кого какие шансы выиграть? \\
b) Сколько в среднем времени ждать до появления ООР? \\
с) Сколько в среднем времени ждать до определения победителя? 

\zad <<Amoeba>>
A population starts with a single amoeba. For this one and for the generations thereafter, there is a probability of 3/4 that an individual amoeba will split to create two amoebas, and a 1/4 probability that it will die out without producing offspring. What is the probability that the family tree of the original amoeba will go on for ever? 

\zad 
Вася подкидывает кубик. Если выпадает единица, или Вася говорит
<<стоп>>, то игра оканчивается, если нет, то начинается заново.
Васин выигрыш - последнее выпавшее число. Как выглядит оптимальная
стратегия? Как выглядит оптимальная стратегия, если за каждое
подбрасывание Вася платит 35 копеек?


\zad 
Suppose the probability to get a head when throwing an unfair coin is p, what's the expected number of throwings in order to get two consecutive heads? 



\setcounter{zadacha}{0}
\newpage
% семинар 3-4 (?) Условная вероятность, условное среднее
Условная вероятность $\P(A|B)=\P(A\cap B)/\P(B)$ и условное среднее $\E(X|B)=\E(X\cdot 1_{B})/\P(B)$ \\

\zad Величина $X$ равномерно распределена на $[0;1]$. Найдите $\P(X>0.7)$, $\E(X)$, $\P(X>0.7\mid X<0.9)$, $\E(X\mid X<0.9)$, $\P(X>0.7\mid X>0.5)$, $\E(X\mid X>0.5)$

\zad Изначально известны следующие вероятности исходов: \\
$\begin{array}{|c|c|c|c|}
\hline 
X & 6 & 3 & -5  \\
\hline
\P() & 1/4 & 2/4 & 1/4  \\
\hline
\end{array}$\\
a) После проведения эксперимента дополнительно стало известно, что $X>0$. Рассчитайте новые условные вероятности исходов: $\P(X=6|X>0)$, $\P(X=3|X>0)$, $\P(X=-5|X>0)$ \\
б) С помощью новых вероятностей найдите $\E(X|X>0)$, $\P(X<5|X>0)$, $\E(X^{2}|X>0)$ 

\zad Эксперимент может окончиться одним из шести исходов: \\
$\begin{array}{|c|c|c|c|}
\hline
& X=-1 & X=1 & X=2 \\
\hline 
Y=0 & 0.1 & 0.2 & 0.3  \\
\hline
Y=4 & 0.2 & 0.1 & 0.1  \\
\hline
\end{array}$
Найдите: \\
а) $\P(X=1 \cap Y=0)$, $\P(X=1)$, $\P(Y=0)$ \\
б) $\P(X=1|Y>1)$, $\P(Y=4|X>0)$, $\P(Y=0|X>0)$ \\
в) $\E(Y|X>0)$, $\E(X|Y>1)$, $\E(XY|X>0)$ 

\zad  Имеется три монетки. Две <<правильных>> и одна - с
<<орлами>> по обеим сторонам. Петя выбирает одну монетку наугад и
подкидывает ее два раза. Оба раза выпадает <<орел>>. Какова
условная вероятность того, что монетка <<неправильная>>?

\zad  Два охотника одновременно выстрелили в одну утку. Первый попадает с
вероятностью 0,4, второй - с вероятностью 0,6. В утку попала ровно
одна пуля. Какова условная вероятность того, что утка была убита первым
охотником?

\zad Кубик подбрасывается два раза. Найдите вероятность получить
сумму равную 8, если при первом броске выпало 3.

\zad Игрок получает 13 карт из колоды в 52 карты. 
Какова вероятность, что у него как минимум два туза, если
известно, что у него есть хотя бы один туз?
Какова вероятность того, что у него как минимум два туза, если
известно, что у него есть туз пик? 

\zad В урне 7 красных, 5 желтых и 11 белых шаров. Два шара
выбирают наугад. Какова вероятность, что это красный и белый, если
известно, что они разного цвета?

\zad  В урне 5 белых и 11 черных шаров. Два шара извлекаются по
очереди. Какова вероятность того, что второй шар будет черным?
Какова вероятность того, что первый шар - белый, если известно,
что второй шар - черный?

\zad Примерно 4\% коров заражены <<коровьим бешенством>>.  Имеется тест, который дает ошибочный результат с вероятностью 0,1. Судя по тесту, новая партия мяса заражена. Какова вероятность того, что она действительно заражена?

\zad В школе три девятых класса, «А», «Б» и «В», одинаковые по
численности. В «А» классе 30\% обожают учителя географии, в «Б»
классе – 40\% и в «В» классе – 70\%. Девятиклассник Петя обожает
учителя географии. Какова вероятность того, что он из «Б» класса?

\zad Ген карих глаз доминирует ген синих. Т.е. у носителя пары bb
глаза синие, а у носителя пар BB и Bb – карие. У диплоидных
организмов (а мы такие :)) одна аллель наследуется от папы, а одна
– от мамы. В семье у кареглазых родителей два сына – кареглазый и
синеглазый. Кареглазый женился на синеглазой девушке. Какова
вероятность рождения у них синеглазого ребенка?

\zad Известно, что $\P(A\cap B)=\P(A)\cdot \P(B)$. Найдите $\P(A|B)$ и $\P(B|A)$. 

\zad Почему события $A$ и $B$ такие, что $\P(A\cap B)=\P(A)\cdot \P(B)$ называются независимыми? 

\zad  Из колоды в 52 карты извлекается одна карта наугад. Являются
ли события <<извлечен туз>> и <<извлечена пика>> независимыми?

\zad  Из колоды в 52 карты извлекаются по очереди две карты
наугад. Являются ли события <<первая карта - туз>> и <<вторая
карта - туз>> независимыми?

\zad  Известно, что $\P(A)=0,3$, $\P(B)=0,4$, $\P(C)=0,5$. События
$A$ и $B$ несовместны, события $A$ и $C$ независимы и
$\P(B|C)=0,1$. Найдите $\P(A\cup B\cup C)$.

\zad Чему равно $\P(w|A)$ если $w\notin A$, т.е. исход $w$ и событие $A$ не могут произойти одновременно? 





\newpage
\setcounter{zadacha}{0}
Дисперсия $\Var(X)=E\left((X-\E(X))^2\right)=\E(X^{2})-(\E(X))^{2}$; стандартное отклонение $\sigma_{X}=\sqrt{\Var(X)}$ \\
Ковариация $Cov(X,Y)=\E(XY)-\E(X)\E(Y)$; корреляция $Corr(X,Y)=\frac{Cov(X,Y)}{\sigma_{X}\sigma_{Y}}$ \\


\zad Эксперимент может окончиться одним из шести исходов: \\
$\begin{array}{|c|c|c|c|}
\hline
& X=-1 & X=0 & X=2 \\
\hline 
Y=0 & 0.1 & 0.2 & 0.3  \\
\hline
Y=4 & 0.2 & 0.1 & 0.1  \\
\hline
\end{array}$ \\
Найдите: $\Var(X)$, $\Var(Y)$, $Cov(X,Y)$, $Corr(X,Y)$, $Cov(X,Y|X\ge 0)$ 

\zad Функция плотности случайной величины  $X$ имеет
вид: $p(t)=\left\{
\begin{array}{c}
  \frac{3}{16}t^2,t\in [-2;2] \\
  0, t\notin [-2;2]
\end{array}
\right.$, Найдите:\\
a) $\P(X>1)$, $\E(X)$, $\E(X^{2})$, $\Var(X)$, $\sigma_{X}$ \\
б) $\E(X|X>1)$, $\E(X^{2}|X>1)$, $\Var(X|X>1)$ \\
в) Решите пункты а) и б) для $p(t)=
\begin{cases}
  t/8, t\in [0;4] \\
  0, t\notin [0;4]
\end{cases}$ 

\zad Из коробки с 4 синими и 5 зелеными шарами достают 2 шара. Пусть
$B$  и  $G$  - количество извлеченных синих и зеленых шаров.
Найдите  $\E(B)$,  $\E(G)$,  $\E(B\cdot
G)$,  $\E(B-G)$, $\Var(G)$, $\Var(B)$, $Cov(B,G)$, $Corr(B,G)$ 

\zad Время $T$ между поездами метро распределено равномерно от 2 до 4 минут. Найдите $\E(T)$, $\E(T^{2})$, $\Var(T)$, $\P(T>2.5)$ 

\zad Найдите $\E(X)$, $\E(X^{2})$, $\Var(X)$, $\E(X^{n})$ для случайной величины $X$  заданной таблицей \\
$\begin{array}{|c|c|c|}
\hline
X & 0 & 1 \\
\hline 
\P() & (1-p) & p  \\
\hline
\end{array}$ \\

\zad Время работы принтера до первой поломки -- случайная величина $X$ [месяцев] с функцией плотности  $p(t)=1/20 \exp(-t/20)$ при $t\ge 0$. Найдите $\E(X)$ и $\Var(X)$, $\P(X>20)$, $\P(X>12)$

\zad Цена акции -- случайная величина с функцией плотности  $p(x)=\frac{3}{4}
\max \left\{x(2-x),0\right\}$. \\
а) Постройте график функции плотности \\
б) Рассчитайте средний Васин доход и дисперсию дохода, если: \\
б1) У Васи есть одна акция; б2) У Васи есть 10 акций \\
б3) У Васи есть опцион-пут, дающий ему право продать акции по цене 1,2 рубля. \\
б4) У Васи есть опцион-колл, дающий ему право купить акции по цене 1 рубль. \\
в) В пунктах б3) и б4) найдите вероятность исполнения опциона 

\zad   Прямой убыток от пожара равномерно распределен на
$\left[0;1\right]$. Если убыток оказывается больше 0,7, то
страховая компания выплачивает компенсацию 0,7. Чему равны средние
потери? Дисперсия потерь? 

\zad Вася решает 10 задач по теории вероятностей. Вероятность решения каждой задачи равна 0.4, величина $X$ -- количество решенных задач. Найдите $\E(X)$, $\P(X>8)$, $\P(X>8|X>1)$, $\E(X|X>1)$ 

\zad Известно, что $\Var(X)=1$. Найдите $\Var(2X)$, $\Var(X+5)$, $\Var(3X+16)$, $\Var(14-2X)$ 

\zad При каком условии $Cov(X,Y)=\E(XY)$? $\Var(X)=\E(X^2)$? 

\zad Вася предлагают две игры. В первой монетку подбрасывают один раз и за орла платят Васе 10 рублей. Во второй монетку подбрасывают 10 раз и за каждого орла платят один рубль. Где больше средний выигрыш? Дисперсия выигрыша? 

\zad  Известно, что $\E(X)=5$, $\E(X^2)=28 $\\
а) При каком значении числа  $t$  величина $\E((X-t)^2)$  будет наименьшей?\\
б) Чему равно это минимальное значение?

\zad Пусть $\E(X)=m$. Отметим на плоскости точку $(m,m)$. Построим на плоскости случайный квадрат с вершинами $(m,m)$, $(m,X)$, $(X,m)$ и $(X,X)$. Чему равна средняя площадь этого квадрата? 

%\zad Будем изображать случайные величины как векторы! Изобразите множество констант как прямую. Изобразите произвольную случайную величину $X$. Изобразите $m=\E(X)$ как проекцию $X$ на прямую констант. Как следует называть тождество $\E((X-m)^{2})+m^{2}=\E(X^{2})$? \\
\zad Величина $X$ равномерна на $[-2;1]$, а $Y$ -- расстояние от числа $X$ до числа $(-1)$. Найдите фукнцию плотности $Y$, $\E(Y)$, $\Var(Y)$, $\P(Y>0.5)$, $\P(Y>0.5|Y<1)$, $\E(Y|Y<1)$

\zad Случайная величина $X$ имеет функцию плотности $p(t)$ и $p(4)=9$. Примерно найдите вероятность $\P(X\in [4;4.003])$. 

\zad (*) В коробке 4 синих, 5 зеленых и один красный шар. Шары извлекают до появления красного. Найдите $\E(B)$,  $\E(G)$,  $\E(B\cdot G)$,  $\E(B-G)$, $\Var(G)$, $\Var(B)$, $Cov(B,G)$, $Corr(B,G)$ 


\newpage
\setcounter{zadacha}{0}

Пуассоновская случайная величина, $X\sim Pois(\lambda)$, вероятность $\P(X=k)=e^{-\lambda}\frac{\lambda^k}{k!}$. \\ 
Замена $Bin(n,p)$ на $Pois(\lambda=np)$ дает погрешность не более $\min\{p,np^2\}$ \\
Экспоненциальная случайная величина, $Y\sim Exp(\lambda)$, ф. плотности $p(y)=\lambda \exp(-\lambda y)$ при $y\geq 0$. \\
Пуассоновский поток событий:\\
время между соседними событиями, $Y\sim Exp(\lambda)$; количество событий за время $t$, $X_t\sim Pois(\lambda t)$. \\



\zad Маша и Саша пошли в лес по грибы. Саша собирает все
грибы, а Маша – только подберезовики. Саша в среднем находит один
гриб за одну минуту, Маша – один гриб за десять минут. Какова
вероятность того, за 8 минут они найдут ровно 13 грибов? Какова
вероятность того, что следующий гриб попадется позже, чем через
минуту, если Маша только что нашла подберезовик?

\zad  Пост майора ГИБДД Иванова И.И. в среднем ловит одного
нарушителя в час. Какова вероятность того, что два нарушителя
появятся с интервалом менее 30 минут? Какова вероятность того, что следующего нарушителя 
 ждать еще более 40 минут, если уже целых три часа никто не превышал скорость?

\zad Оля и Юля пишут смс Маше. Оля отправляет Маше в среднем 5 смс
в час. Юля отправляет Маше в среднем 2 смс в час. Какова
вероятность того, что Маша получит ровно 6 смс за час? Сколько
времени в среднем проходит между смс, получаемыми Машей от подруг?

\zad Кузнечики на большой поляне распределены
по пуассоновскому закону, в среднем 3 кузнечика на квадратный метр. Какой
следует взять сторону квадрата, чтобы вероятность найти в нем хотя
бы одного кузнечика была равна  $0,8$? 

\zad В магазине две кассирши (ах, да! две хозяйки кассы). Допустим, что время обслуживания клиента распределено экспоненциально. Тетя Зина обслуживает в среднем $5$ клиентов в час, а тетя Маша - $7$. Два клиента подошли к кассам одновременно. \\
а) Какова вероятность того, что тетя Зина обслужит клиента быстрее? \\
б) Как распределено время обслуживания того клиента, который освободится быстрее? \\
в) Каково условное среднее время обслуживания клиента тетей Зиной, если известно, что она обслужила клиента быстрее тети Маши? 
%Ответы: $\frac{a}{a+b}$, экспоненциально с параметром $a+b$, $\frac{1}{a+b}$ 

\zad Время между приходами студентов в столовую распределено экспоненциально; в среднем за 10 минут приходит 5 студентов. Время обслуживания имеет экспоненциальное распределение; в среднем за 10 минут столовая может обслужить 6 студентов. \\
а) Как распредено количество студентов в очереди? \\
б) Какова средняя длина очереди? \\
Подсказка: если сейчас в очереди $n$ человек, то через малый промежуток времени $dt$... %Ответ: \\
%а) геометрическое распределение \\
%б) $\E(N)=\frac{\lambda_{in}}{\lambda_{capacity}-\lambda_{in}}$\\

\zad The arrival of buses at a given bus stop follows Poisson law with rate 2. The arrival of taxis at the same bus stop is also Poisson, with rate 3. What is the probability that next time I'll go to the bus stop I'll see at least two taxis arriving before a bus? Exactly two taxis? 
%Solution: \\
%The probability of observing a taxi before a bus is given by $3/(3+2)=3/5$ since the 
%waiting times are independent and exponentially distributed. By the memoryless %property both processes then restart and hence the probability of observing (at least) %two taxis before the first bus is $(3/5)^2=9/25$. The probability of observing exactly %two taxis before the first bus is $(3/5)^2*(2/5)=18/125$. 

\zad Время, которое хорошо обученная свинья тратит на поиск трюфеля -- экспоненциальная случайная величина со средним в 10 минут. Какова вероятность того, что свинья за 20 минут не найдет ни одного трюфеля?

\zad Пусть $X$ - распределена экспоненциально с параметром $\lambda$ и $a>0$. \\
Как распределена величина $Y=aX$? 

\zad В гирлянде 25 лампочек. Вероятность брака для отдельной
лампочки равна 0,01. Какова вероятность того, что гирлянда
полностью исправна? Оцените точность ответа.

\zad По некоему предмету незачет получило всего 2\% студентов.
Какова вероятность того, что в группе из 50 студентов будет ровно
1 человек с незачетом? Оцените точность ответа.

\zad Вася испек 40 булочек. В каждую из них он кладет изюминку с
$p=0,02$ . Какова вероятность того, что всего окажется 3 булочки с
изюмом? Оцените точность ответа. 

\zad В офисе два телефона -- зеленый и красный. Входящие звонки на красный -- Пуассоновский поток событий с интенсивностью $\lambda_{1}=4$ звонка в час, входящие на зеленый -- с интенсивностью $\lambda_2=5$ звонка в час. Секретарша Василиса Премудрая одна в офисе. Перед началом рабочего дня она подбрасывает монетку и отключает один из телефонов, зеленый -- если выпала решка, красный -- если орел. Обозначим $Y$ время от начала дня до первого звонка. Найдите функцию плотности $Y$.


\newpage
\setcounter{zadacha}{0}
\zad Докажите следующие свойства ($X$, $Y$, $Z$ - с.в., $a$, $b$ - константы): \\
a) $Cov(aX+b,Y)=aCov(X,Y)$, в частности $Cov(X,a)=0$ \\
b) $Cov(X+Y,Z)=Cov(X,Z)+Cov(Y,Z)$, линейность ковариации \\
c) $\Var(aX+b)=a^{2}\Var(X)$,  дисперсия - это квадрат длины случайной величины \\
d) $\Var(X+Y)=\Var(X)+\Var(Y)+2Cov(X,Y)$ \\
d2) $\Var(X-Y)=\Var(X)+\Var(Y)-2Cov(X,Y)$ \\
e) $Corr(aX,Y)=Corr(X,Y)$ если $a>0$, косинус угла между векторами не меняется \\
f) $\sigma_{aX}=|a|\sigma_{X}$, стандартная ошибка - это длина
случайной величины \\
g) $\sigma_{X}+\sigma_{Y}\ge \sigma_{X+Y}$, неравенство
треугольника

\zad Известно, что $Y=2X-3$, а $Z=6-3X$. Найдите $Corr(X,Y)$, $Corr(X,Z)$ 

\zad Пусть $X$ и $Y$ независимы. \\
а) Найдите $Cov(X,Y)$, $Cov(X^{3},Y^{2}-5Y)$, $Corr(X,Y)$ \\
b) Выразите $\Var(X+Y)$ и $\Var(X-Y)$ через $\Var(X)$ и $\Var(Y)$

\zad Кубик подбрасывается два раза, $X$  - сумма очков, $Y$  - разность очков, число при
первом броске минус число при втором. Найдите $\E(XY)$, $Cov(X,Y)$,
$Corr(X,Y)$

\zad Пусть  $X$  равновероятно принимает значения -1, 0, +1, а $Y=X^{2}$ \\
а) Найдите  $Cov(X,Y)$; б) Верно ли, что $X$ и $Y$ независимы?

\zad Пусть  $X$  равномерно на $[0;1]$, $Y=X^{2}$ \\
а) Найдите  $Cov(X,Y)$; б) Верно ли, что $X$ и $Y$ независимы?

\zad Паук сидит в начале координат. Равновероятно он может сместиться на единицу вверх, вниз, влево или вправо (по диагонали паук не ползает). Пусть $X$ и $Y$ - это абсцисса и ордината паука после первого шага. \\
а) Найдите $Cov(X,Y)$? \\
б) Верно ли, что $X$ и $Y$ независимы? 

\zad  Кубик подбрасывается $n$
раз. Пусть $X_{1}$ - число выпадений 1, а $X_{6}$ - число выпадений 6. \\
Найдите $Corr(X_{1},X_{6})$. Подсказка: $Cov(X_{1},X_{1}+...+X_{6})=...$  

\zad Вероятность дождя в субботу 0.5, вероятность дождя в воскресенье 0.3. Корреляция между наличием дождя в субботу и наличием дождя в воскресенье равна $r$. \\
Какова вероятность того, что в выходные вообще не будет дождя? 

\zad Пусть $\P(A|B)>\P(A)$. Можно ли определить знак $Cov(1_{A},1_{B})$? 

\zad Вася наблюдает значение $X$, но не наблюдает значение $Y$; при этом он знает, что $\Var(X)=3$, $\Var(Y)=8$, $Cov(X,Y)=-3$, $\E(Y)=3$, $\E(X)=2$.  Задача Васи -- спрогнозировать $Y$ с помощью линейной функции от $X$, т.е. построить $\hat{Y}=aX+b$. 
Васю штрафуют за неправильный прогноз на сумму $(Y-\hat{Y})^{2}$. Найдите $a$ и $b$ 

\zad Случайные величины $X$ и $Y$ зависимы, случайные величины $Y$ и $Z$ зависимы. Верно ли, что случайные величины $X$ и $Z$ зависимы?

\zad Пусть $cov(X,Y)>0$, $cov(Y,Z)>0$. Верно ли, что $cov(X,Z)>0$? $cov(X,Z)\geq 0$?

\zad Кольцо задавается системой неравенств: $x^{2}+y^{2}\geq 1$ и $x^{2}+y^{2}\le 4$. Случайным образом, равномерно на этом кольце, выбирается точка, $X$ и $Y$ -- ее координаты. \\
Чему равна корреляция $X$ и $Y$? Зависимы ли $X$ и $Y$?  \\ \\

Случайные величины $X$ и $Y$ \textit{независимы}, если независимы любые\footnote{для тех, кто ходит на стоханализ: $A$ и $B$ должны лежать в борелевской $\sigma$-алгебре} два события, первое из которых формулируется с помощью величины $X$, в второе - с помощью $Y$, т.е. $\forall A, B\subset \mathbb{R}$, $\P(X\in A\cap Y\in B)=\P(X\in A)\cdot \P(Y\in B)$ \\
Если $X$ и $Y$ независимы, то любые\footnote{любые измеримые, т.е. такие, что $\forall A\in \mathcal{B}$, $f^{-1}(A)\in\mathcal{B}$} $f(X)$ и $g(Y)$ независимы. \\
Если $X$ и $Y$ независимы, то $\E(XY)=\E(X)\E(Y)$ \\

%Теоремки: \\
%1. Для независимости необходимо и достаточно проверить, что $\P(X\le x\cap Y\le y)=\P(X\le x)\cdot \P(Y\le y)$ для всех $x$ и $y$ \\
%2. Дискретные величины независимы если и только если $\P(X=x\cap Y=y)=\P(X=x)\cdot \P(Y=y)$ для всех $x$ и $y$ \\
%3. Непрерывные величины независимы если и только если $p_{X,Y}(x,y)=p_{X}(x)\cdot p_{Y}(y)$ для всех $x$ и $y$ \\
%4. 
%5.  \\



\newpage
\setcounter{zadacha}{0}
$\int_{-\infty}^{+\infty} p_{X,Y}(x,y) dy=p_{X}(x)$ \\
$p_{X|Y}(x|y)=\frac{p_{X,Y}(x,y)}{p_{Y}(y)}$ 



\zad Случайные величины $X$ и $Y$ заданы двумерной функцией плотности $p_{X,Y}(t_{1},t_{2})$. Известно, что $p(4,8)=9$. Примерно найдите вероятность $\P(X\in [4;4.003]\cap Y\in [7.999;8])$. 
% $9\cdot 0.003\cdot 0.001$ \\

\zad Как из равномерной случайной величины получить экспоненциальную величину со средним значением $m$? 

\zad Пусть $g(t)$ - возрастающая функция и $Y=g(X)$. Докажите, что функция плотности случайной величины $Y$ имеет вид: $p_{Y}(t)=p_{X}(g^{-1}(t))\frac{dg^{-1}}{dt}$. 

\zad Пусть  $X$ распределена равномерно на $\left[0;1\right]$.
Найдите плотность распределения случайных величин  $Y=\ln
\frac{X}{1-X} $, $Z=-\frac{1}{\lambda}\ln X$ и $\lambda>0$,
$W=X^{3}$, $Q=X-1/X$. Как изменятся ответы, если $X$ не равномерна, а имеет функцию плотности $p(t)=2t$ на отрезке $[0;1]$? 

\zad Совместная функция плотности имеет вид: \\
$p_{X,Y}(x,y)=\left\{
\begin{array}{ll}
  2-x-y, & $ если $x\in[0;1],y\in[0;1] \\
  0, & $ иначе $ \\
\end{array}
\right.$ \\
Найдите $\P(Y>2X)$, $\E(Y)$, $\E(XY)$ $Cov(X,Y)$, $\E(X|Y>0,5)$,
частную (предельную) функцию плотности $p_{Y}(t)$, условную
функцию плотности $p_{X|Y}(x|y)$, $\E(X|Y)$. Верно ли, что величины $X$ и $Y$ являются независимыми? 
%$\P(Y>2X)=7/24$, $\E(XY)=1/6)$, $\E(X|Y>0.5)=\frac{7/48}{3/8}=7/18$ \\
%$X$ и $Y$ зависимы \\


\zad Совместная функция плотности имеет вид \\
$p_{X,Y}(x,y)=\left\{
\begin{array}{ll}
  x+y, & $ если $x\in[0;1],y\in[0;1] \\
  0, & $ иначе $ \\
\end{array}
\right.$ \\
Найдите  $\P(Y>X)$,  $\E(X)$, $\E(X|Y>X)$, $Cov(X,Y)$, частную
(предельную) функцию плотности $p_{Y}(t)$, условную функцию
плотности $p_{X|Y}(x|y)$, $\E(X|Y)$. Верно ли, что величины $X$ и $Y$ являются независимыми? 

\zad  $X$  и  $Y$  независимы и равномерны на отрезке $[0;1]$.
Найдите функцию плотности  $Z=X+Y$. Как изменится ответ, если $X$ и $Y$ независимы и имеют функцию плотности $p(t)=2t$ на отрезке $[0;1]$? 

\zad Пусть  $X_{1} $,  $X_{2}$  и  $X_{3}$  - независимы и
равномерны на отрезке  $[0;1]$. Найдите функцию плотности $Y=\max
\left\{X_{1},X_{2},X_{3} \right\}$, $Z=\min
\left\{X_{1},X_{2},X_{3} \right\}$

\zad $X$ выбирается равномерно на отрезке $[0;1]$. Затем
$Y$ выбирается равномерно на отрезке $[0;X]$. \\
a) Найдите условную функцию плотности $p_{Y|X}(y|x)$ \\
b) Найдите $p_{X,Y}(x,y)$ и $p_{Y}(y)$ \\
c) Найдите $\E(Y)$, $\Var(Y)$, $\P(X+Y>1)$

\zad Пусть $X$ - неотрицательная с.в. с функцией плотности $p(t)$
и $\E(X)<\infty$. При каком $c$ функция $g(t)=c\cdot tp(t)$ также
будет функцией плотности?

\zad Совместная функция плотности $X$ и $Y$ имеет вид $p(x,y)=cx^2+y^2$ на участке $x\in [0;1]$ и $y\in [0;1]$. Найдите $c$. Найдите совместную функцию плотности величин $W$ и $Z$, если a) $W=X+Y$, $Z=X-Y$; б) $W=XY$, $Z=X^{2}Y$. 

\zad Петя сообщает Васе значение случайной величины, равномерной
на отрезке  $[0;4]$. С вероятностью  $\frac{1}{4}$
Вася возводит Петино число в квадрат, а с вероятностью
$\frac{3}{4}$  прибавляет к Петиному числу 2. Обозначим результат
буквой  $Y$. Найдите  $\P(Y<4)$  и функцию плотности случайной
величины  $Y$. \\
Вася выбирает свое действие независимо от Петиного числа. 

\zad Оценка  $X$  за экзамен распределена равномерно на
отрезке  $\left[0;100\right]$. Итоговая оценка  $Y$ рассчитывается
по формуле  
$Y=\left\{\begin{array}{l} 
{0, X<30} \\ 
{X, X\in [30;80]} \\
{100, X>80} 
\end{array}\right. $. \\
Найдите  $\E(Y)$,  $\E(X\cdot Y)$, $\E(Y^{2})$,  $\E(Y|Y>0)$. 

\zad Петя сообщает Васе значение величины $X\sim U[0;1]$. Вася изготавливает неправильную монетку, которая выпадает <<орлом>> с вероятностью  $X$ и подкидывает ее 20 раз. \\
а) Какова вероятность, что выпадет ровно 5 орлов? \\
б) Каково среднее количество выпавших орлов? Дисперсия? \\





\newpage
\setcounter{zadacha}{0}

\zad  Пусть  $X\sim N(0;1)$. Найдите $\P(X>0,5)$, $\P(-1<X<2)$,
$\P(X^{2}
>3)$,  $\P(X<0,3)$,
$\P(|X|<0,8)$

\zad
 Пусть  $X\sim N(4;9)$,  $Y\sim
N(-5;16)$,  $Z\sim N(20;100)$ ;  $X$,  $Y$ и $Z$  - независимы.
Найдите  $\P(X>8)$,  $\P(X\in \left[1;5\right])$, $\P(Y\in
[-10;-3))$, $\P(Z>100)$, $\P(X+Y>3)$, $\P(|Z|>10)$, $\P(4Y+Z>15)$.

\zad Пусть  $X\sim N(\mu ;\sigma ^{2}
)$. Найдите $\P(X-\mu>2\sigma )$, $\P(|X-\mu |>2\sigma )$,  $\P(|X-\mu |>3\sigma )$.

\zad   Монетку подбрасывают 1000 раз,  $S$  -- общее
количество <<орлов>>. Найдите  $\P(S\in [490;520] )$, $\P(492<S<505)$, $\P(S<400)$. По теореме Берри-Эссена оцените погрешность.

\zad   Доход от одной акции компании XXX представляет собой случайную величину $X\sim
N(50;5^{2} )$, а доход от одной акции компании YYY -- величину $Y\sim
N(80;9^{2} )$, при этом $Cor(X,Y)=-0,4$. Определите вероятность того, что суммарный доход от
пакета из восьми акций XXX и двух акций YYY составит не менее 750.

\zad Пусть  $X\sim N(7;16)$. Найдите $\E(X|X>11)$, $\E(X|X<10)$,
$\E(X|X\in[0;10])$.

\zad  В большом-большом городе 80\% аудиокиосков торгуют
контрафактной продукцией. Какова вероятность того, что в наугад
выбранных 90 киосках более 60 будут торговать контрафактной
продукцией? менее 50? от 40 до 80? от 70 до 75?

\zad Количество опечаток в газете -- случайная величина с матожиданием 10 и
дисперсией 25. Какова вероятность того, что по 144 газетам среднее
количество опечаток превысит 11? будет от 10 до 10,5? Будет
больше 9,5? меньше 20?

\zad Случайная величина $X$ имеет функцию плотности  $p_{X} (t)=c\cdot \exp (-8t^{2}
+5t)$. Найдите $\E(X)$,  $\sigma _{X} $, $c$.

\zad Количество смс за сутки, посылаемое каждым из 160 абонентов, имеет пуассоновское распределение со средним значением 5 смс в сутки. Какова вероятность того, что за двое суток абоненты пошлют в сумме более 1700
сообщений?

\zad  Вероятность выпадения монетки <<орлом>> равна 0,63. \\
a) Какова вероятность, что в 100 испытаниях выборочная доля выпадения
орлов будет отличаться от истинной менее, чем на 0,07? \\
b) Каким должно быть минимальное количество испытаний, чтобы
вероятность отличия менее чем на 0,02 была больше 0,95?

\zad Пусть $X \sim N( {0,\sigma ^2 } )$. \\
а) Найдите функцию плотности $|X|$ \\
б) Найдите $\E(|X|)$ (можно найти не решая а).

\zad  Известно, что  $\ln Y\sim N(\mu ;\sigma ^{2} )$. Найдите
$\E(Y)$,  $\Var(Y)$.

\zad Докажите, что случайная величина с функцией плотности
$p(x)=c\cdot \exp(-\frac{1}{2\sigma^{2}}(x-\mu)^{2})$ действительно
имеет матожидание $\mu$ и дисперсию $\sigma^{2}$

\zad Каждый день цена акции равновероятно поднимается или
опускается на один рубль. Сейчас акция стоит 1000 рублей. Найдите
вероятность того, что через сто дней акция будет стоить больше
1030 рублей. % Введем случайную величину  $X_{i} $, обозначающую изменение курса акции
% за  $i$ -ый день. Найдите  $\E(X_{i} )$  и $\Var(X_{i} )$.

\zad В самолете пассажирам предлагают на выбор <<мясо>> или <<курицу>>. В самолете 250 мест. Каждый пассажир с вероятностью 0.6 выбирает курицу, и с вероятностью 0.4 -- мясо. Сколько порций курицы и мяса нужно взять, чтобы с вероятностью 99\% каждый пассажир получил предпочитаемое блюдо, а стоимость <<мяса>> и <<курицы>> для компании одинаковая? \\
Как изменится ответ, если компания берет на борт одинаковое количество <<мяса>> и <<курицы>>? 
%$K=170$, $M=120$ (симметричный интервал) или $K=M=168$ (площадь с одного края можно принять за 0) \\

\zad Величины $X_{i}$ имеют одинаковое среднее $\mu$ и одинаковую дисперсию $\sigma^2$ и попарно некоррелированны, $Corr(X_i,X_j)=0$. Рассмотрим их сумму $S_n=X_1+...+X_n$, и их среднее $\bar{X}_n=S_n/n$. Найдите $\E(S_n)$, $\E(\bar{X}_n)$, $\Var(S_n)$, $\Var(\bar{X}_n)$.



Величина $X$ распределена нормально, $X\sim N(\mu;\sigma^2)$, если $p(x)=c\cdot
\exp\left(-\frac{(x-\mu)^{2}}{2\sigma^{2}}\right)$, где $c=\frac{1}{\sqrt{2\pi}\sigma}$ 

%Если $X\sim N$, то $aX+b\sim N$, если $a\neq 0$ \\
%Если $X\sim N$, $Y\sim N$, $X$ и $Y$ независимы, то $X+Y\sim N$ 

Центрирование: Если $X\sim N$, то $Z=\frac{X-\E(X)}{\sqrt{\Var(X)}}\sim
N(0;1)$ 

Центральная предельная теорема (интуитивно): \\
Если $X_{1}$, ..., $X_n$ независимы, одинаково распределенны, то при больших $n$ их сумма $S_n$
и среднее $\bar{X}_n$ имеют распределение похожее на
нормальное. 

Теорема Берри-Эссена: \\
при расчете вероятности $\P(S_n \in [a;b])$ по ЦПТ абсолютная ошибка не превосходит $\frac{\E(|X_{i}-\mu|^{3})}{\sigma^{3}\sqrt{n}}$

%$\E(\sum_{i}^{n}X_{i})=n\mu$, $\Var(\sum_{i}^{n}X_{i})=n\sigma^{2}$, $\E(\bar{X})=\mu$, $\Var(\bar{X})=\frac{\sigma^{2}}{n}$ \\



\newpage
\setcounter{zadacha}{0}

\zad Может ли ковариационная матрица иметь вид: \\
a) $\left(\begin{array}{cc} {4} & {2} \\ {-1} & {9}
\end{array}\right)$, б) $\left(\begin{array}{cc} {4} & {7} \\ {7} &
{9} \end{array}\right)$, в) $\left(\begin{array}{cc} {9} & {0} \\
{0} & {-1} \end{array}\right)$, г)  $\left(\begin{array}{cc} {9} &
{2} \\ {2} & {1} \end{array}\right)$?

\zad Есть три случайных величины. Все попарные корреляции равны $\rho$. В каких пределах может лежать $\rho$?

\zad Пусть  $X\sim N\left(\left(\begin{array}{l} {2} \\ {3}
\end{array}\right);\left(\begin{array}{cc} {9} & {-1} \\ {-1} &
{16} \end{array}\right)\right)$. Найдите  $\E(X_{1} )$, $\E(X_{1} +2X_{2} )$,
$\Var(X_{1} -X_{2} )$, $\P(X_{1}
>X_{2} )$,  $\P(2X_{1} +X_{2} <5)$. Как распределена
случайная величина  $X_{1}$  при
условии, что  $X_{2} =6$?  $X_{2} =-3$?

\zad В данном регионе кандидата в парламент Обещаева И.И.
поддерживает 60\% населения. Сколько нужно опросить человек, чтобы
с вероятностью 0,99 доля  опрошенных избирателей, поддерживающих
Обещаева И.И.,  отличалась от 0,6 (истинной доли) менее, чем на
0,01? 


\zad Складывают $n$ чисел. Перед сложением каждое число округляют до ближайшего целого. Появляющуюся при этом ошибку можно считать равномерно распределенной на отрезке $[-0.5;0.5]$. Определите, сколько чисел складывают, если с вероятностью $\frac{1}{2}$ получаемая сумма отличается от настоящей больше чем на 3 (в любую сторону). 

%\zad Верно ли, что $\Var(X)=\E(X\cdot X^{T})-\E(X)\cdot (\E(X))^{T}$?

%\zad Верно ли, что  $\Var\left(a_{1} X_{1} +a_{2} X_{2}
%\right)=\left(a_{1};a_{2} \right)\cdot H\cdot \left(\begin{array}{c} {a_{1}} \\
%{a_{2}} \end{array}\right) $, где  $H=\Var(\vec{X})$  - ковариационная
%матрица?

\zad Рассмотрим антагонистическую игру двух игроков. У каждого
игрока две стратегии. Матрица игры определяется случайным образом:
каждый из четырех платежей - это независимая случайная величина
$X_{ij}\sim N(0;1)$. Каково ожидаемое количество равновесий по
Нэшу в чистых стратегиях в получающейся матрице? 

\zad  Внутри упаковки шоколадки находится
наклейка с изображением одного из 30 животных. Предположим, что
все наклейки равновероятны. Большой приз получит каждый, кто
соберет наклейки всех животных. Какое количество шоколадок в
среднем нужно купить, чтобы выиграть приз?

\zad С помощью неравенства Чебышева оцените вероятности\\
а) $\P(-2\sigma<X-\mu<2\sigma)$, если $\E(X)=\mu$, $\Var(X)=\sigma^2$\\
b) $\P(8<Y<12)$, если $\E(Y)=10$, $\Var(Y)=400/12$  \\
c) $\P(-2<Z-\E(Z)<2)$, если $\E(Z)=1$, $\Var(Z)=1$  \\
d) Найдите точные значения, если дополнительно известно, что 
$X\sim N(\mu;\sigma^{2})$, $Y\sim U[0;20]$ и $Z\sim Exp(1)$.
% b $4/20\geq 1-100/12$
% c $1-e^{-3}\geq 0.75$

%\zad Докажите неравенство Чебышева: \\
%а) Сравните $\Var(X)$ и $\E((X-\E(X))^{2}\cdot 1_{|X-\E(X)|>a})$ \\
%б) Сравните $\E((X-\E(X))^{2}\cdot 1_{|X-\E(X)|>a})$ и $a^{2}\P(|X-\E(X)|>a)$ 

\zad Пусть $X\sim N(0;1)$; $Z$ равновероятно принимает значения $1$ и
$-1$; $X$ и $Z$ независимы; $Y=X\cdot Z$. \\
а) Какое распределение имеет св. $Y$? 
б) Чему равна $Cov(X,Y)$? 
г) Верно ли, что $X+Y$ нормально? 

\zad Пусть величины $X$, $Y$, $Z$ - имеют совместное нормальное распределение, с математическим ожиданием $0$ и некоей ковариационной матрицей $B$. Как зависит от $B$ вероятность $\P(XYZ>0)$? 

\zad Пусть $X_{1}$ и $X_{2}$ имеют совместное нормальное распределение, причем каждая $X_{i}\sim N(0;1)$, а корреляция равна $\rho$. \\
а) Выпишите в явном виде (без матриц) совместную функцию плотности \\
б) Пусть $\rho=0.5$. Какое условное распределение имеет $X_{1}$ при условии, что $X_{2}=-1$? 

\zad Допустим, что в городе $N$ рост матери и рост дочери являются совместно нормально распределенными случайными величинами. Причем и рост матери, и рост дочери распределены $N(165;5^2)$ с корреляцией $0.7$. \\
a) Among the daughters of above average height, what percent were shorter than their mothers? \\
b) Amont the daughters of above average height, what percent have an above average mother? \\
c) Дополнительно известно, что рост дочери равен 167 см. Какова вероятность того, что рост матери выше среднего? Чему равен ожидаемый рост матери? \\



\texttt{Минитеория} \\
Неравенство Чебышева $\P(|X-\E(X)|>a)\le \frac{\Var(X)}{a^{2}}$\\
Многомерное нормальное. Пусть $\vec{\mu}$ - вектор столбец средних, а $H$ - ковариационная матрица \\
$\P(\vec{x})=\frac{1}{\sqrt{(2\pi)^n\cdot \det(H)}}\exp(-\frac{1}{2}(\vec{x}-\vec{\mu})^{t}H^{-1}(\vec{x}-\vec{\mu}))$ \\
Для вектора случайных величин $\Var(X)$ означает ковариационную матрицу. \\



\newpage
\setcounter{zadacha}{0}

\zad  $X_{i} \sim iid$ , какая из приведенных оценок для
$\E\left(X_{i} \right)$  является несмещенной? наиболее
эффективной? 

а)  $X_{1} +3X_{2} -2X_{3} $ ; в)  ${\left(X_{1}
+X_{2} \right)\mathord{\left/ {\vphantom {\left(X_{1} +X_{2}
\right) 2}} \right. \kern-\nulldelimiterspace} 2} $ ; г)
${\left(X_{1} +X_{2} +X_{3} \right)\mathord{\left/ {\vphantom
{\left(X_{1} +X_{2} +X_{3} \right) 3}} \right.
\kern-\nulldelimiterspace} 3} $ ; д)  $\left(X_{1} +...+X_{20}
\right)/21$ ; е)  $X_{1} -2X_{2} $


\zad С.в.  $X$  равномерна на  $\left[0;a\right]$ . Придумайте
$\hat{a}=\alpha +\beta X$  так, чтобы  $Y$  была несмещенной оценкой $a$.

\zad Пусть  $X_{i} $  - независимы и одинаково распределены. При
каком значении параметра  $\beta $

а)   $2X_{1} -5X_{2} +\beta X_{3} $  будет несмещенной оценкой для
$\E\left(X_{i} \right)$ ?

б)   $\beta \left(X_{1} +X_{2} -2X_{3} \right)^{2} $  будет
несмещенной оценкой для  $\Var\left(X_{i} \right)$ ?

\zad Пусть  $X_{1} $  и  $X_{2} $  независимы и равномерны на
$\left[0;a\right]$ . При каком  $\beta $  оценка  $Y=\beta \cdot
\min \left\{X_{1} ,X_{2} \right\}$  для параметра  $a$  будет
несмещенной?

\zad  $X_{i} \sim iid$ , какая из приведенных оценок для
$\Var\left(X_{i} \right)$  является несмещенной

а)  $X_{1}^{2} -X_{1} X_{2} $ ; б)  $\frac{\sum
_{i=1}^{n}\left(X_{i} -\bar{X}\right)^{2}  }{n} $ ; в)
$\frac{\sum _{i=1}^{n}\left(X_{i} -\bar{X}\right)^{2}  }{n-1} $ ;
г)  $\frac{1}{2} \left(X_{1} -X_{2} \right)^{2} $ ; д)  $X_{1}
-2X_{2} $ ; е)  $X_{1} X_{2} $


\zad Пусть  $X$  равномерна на  $\left[3a-2;3a+7\right]$ . При
каких  $\alpha $  и  $\beta $  оценка  $Y=\alpha +\beta X$
неизвестного параметра  $a$  будет несмещенной?

\zad Закон распределения с.в.  $X$  имеет вид

а)  $\begin{array}{|c|ccc|}  \hline {x_{i} } & {0} & {1} & {a} \\
\hline {P\left(X=x_{i} \right)} & {{1\mathord{\left/ {\vphantom {1
4}} \right. \kern-\nulldelimiterspace} 4} } & {{1\mathord{\left/
{\vphantom {1 4}} \right. \kern-\nulldelimiterspace} 4} } &
{{2\mathord{\left/ {\vphantom {2 4}} \right.
\kern-\nulldelimiterspace} 4} } \\  \hline  \end{array}$ ; б)
$\begin{array}{|c|ccc|}  \hline {x_{i} } & {0} & {1} & {2} \\
\hline {P\left(X=x_{i} \right)} & {{1\mathord{\left/ {\vphantom {1
4}} \right. \kern-\nulldelimiterspace} 4} } & {a} &
{\left({3\mathord{\left/ {\vphantom {3 4}} \right.
\kern-\nulldelimiterspace} 4} -a\right)} \\  \hline  \end{array}$

Постройте несмещенную оценку вида  $Y=\alpha +\beta X$  для
неизвестного параметра  $a$


\zad Время горения лампочки – экспоненциальная с.в. с
ожиданием равным  $\theta $ . Вася включил одновременно 20
лампочек. С.в. $X$  обозначает время самого первого перегорания.
Как с помощью $X$  построить несмещенную оценку для  $\theta $ ?

\zad Пусть $X_{i}$ независимы и одинаково распределены, причем
$\Var(X_{i})=\sigma^{2}$, а $\E(X_{i})=\frac{\theta}{\theta+1}$, где
$\theta>0$ - неизвестный параметр. С помощью $\bar{X}$ постройте
состоятельную оценку для $\theta$.

\zad Пусть $Y_{i}=\beta X_{i} +\epsilon_{i}$, константа $\beta$ и
случайные величины $\epsilon_{i}$ являются ненаблюдаемыми.
Известно, что $\E(\epsilon_{i})=0$, $\Var(\epsilon_{i})=\sigma^{2}$,
$\epsilon_{i}$ являются независимыми. Константы $X_{i}$
наблюдаемы, и известно, что $20<X_{i}<100$. У исследователя есть
две
оценки для $\beta$: $\hat{\beta}_{1}=\frac{\bar{Y}}{\bar{X}}$ и 
$\hat{\beta}_{2}=\frac{\sum{X_{i}Y_{i}}}{\sum{X_{i}^{2}}}$ \\
a) Проверьте несмещенность, состоятельность. \\
б) Определите, какая оценка является наиболее эффективной.

\zad Есть два золотых слитка, разных по весу. Сначала взвесили первый слиток и получили результат $X$. Затем взвесили второй слиток и получили результат $Y$. Затем взвесили оба слитка и получили результат $Z$. Допустим, что ошибка каждого взвешивания - это случайная величина с нулевым средним и дисперсией $\sigma^{2}$. \\
а) Придумайте наилучшую оценку веса первого слитка. \\
б) В каком смысле оценка, полученная в <<а>> является наилучшей? 

\zad Измерен рост 100 человек. Средний рост оказался равным 160
см. Медиана оказалась равной 155 см. Машин рост в 163 см был
ошибочно внесен как 173 см. Как изменятся медиана и среднее после
исправления ошибки? 

\zad Имеется пять чисел: $x$, $4$, $5$, $7$, $9$. При каком значении
$x$ медиана будет равна среднему? 

\zad Деканат утверждает, что если студента N перевести из группы
А в группу В, то средний рейтинг каждой группы возрастет. Возможно
ли это? 

\zad В среднем в каждой группе учится 25 человек. Мы выбираем одного
студента с курса наугад. Верно ли что, что ожидаемое количество человек в его группе равно 25? \\

\texttt{Минитеория} \\
Оценка $\hat{\theta}$ неизвестного параметра $\theta$ называется \emph{несмещенной}, если  $\E(\hat{\theta})=\theta$.
 
Последовательность оценок $\hat{\theta}_1$, $\hat{\theta}_2$, ..., называется \emph{состоятельной}, \\
если $\lim \P(|\theta -\hat{\theta}_{n}|>\varepsilon )=0$ для всех $\forall \varepsilon
>0$. 

Среднеквадратичная ошибка, $\MSE(\hat{\theta})=\E((\hat{\theta}-\theta)^2)$

Оценка называется \emph{эффективной}, если у нее минимально возможная $\MSE$. \\

Неравенство Йенсена: Если $f$ --- выпуклая вверх (а-ля $y=x^2$) функция, то $\E(f(X))\geq f(\E(X))$ \\

Закон больших чисел без технических деталей: $\bar{X}=\frac{\sum X_i}{n}$ --- состоятельная оценка для $\mu=\E(X_i)$ 



\newpage
\setcounter{zadacha}{0}

%\section{11. ML, MM, байесовский подход}
\zad Допустим, что $X_{i}$ - независимы и имеют закон распределения, заданный табличкой: \\
\begin{tabular}{|c|c|c|c|}
  \hline
  X & -1 & 0 & 2 \\
  \hline
  $\P()$ & $\theta$ & $2\theta-0.2$ & $1.2-3\theta$ \\
  \hline
\end{tabular} \\
Имеется выборка: $X_{1}=0$, $X_{2}=2$. \\
a) Найдите оценки $\hat{\theta}_{ML}$ и $\hat{\theta}_{MM}$ \\
b) Первоначально ничего о $\theta$ не было известно и поэтому
предполагалось, что $\theta$ распределена равномерно на
$[0.1;0.4]$. Как выглядит условное распределение $\theta$, если известно что $X_{1}=0$, $X_{2}=2$? \\
c) Постройте ML и MM оценки для произвольной выборки $X_{1}$, $X_{2}$, ... $X_{n}$ 
% ответы $\hat{\theta}_{ML}=0.25$, $\hat{\theta}_{MM}=0.2$
% $\hat{\theta}_{MM}=\frac{2{,}4-\bar{X}}{7}$

\zad У Васи есть два одинаковых золотых слитка неизвестной массы $m$ и весы, которые взвешивают с некоторой погрешностью. Сначала Вася положил на весы один слиток и получил результат $Y_{1}=m+u_{1}$, где $u_1$ --- случайная величина, ошибка первого взвешивания. Затем Вася положил на весы сразу оба слитка и получил результат $Y_{2}=2m+u_{2}$, где $u_2$ --- случайная величина, ошибка второго взвешивания. Оказалось, что $y_{1}=0.9$, а $y_{2}=2.3$. \\
Используя ML оцените вес слитка $m$ и параметр погрешности весов $b$, если \\
a) $u_i$ --- независимы и $N(0;b)$ b) $u_i$ --- независимы и $U[-b;b]$ 

\zad Пусть $Y_{1}$ и $Y_{2}$ независимы и распределены по Пуассону. Известно также, что $\E(Y_{1})=e^{a}$ и $\E(Y_{2})=e^{a+b}$. Найдите ML оценки для $a$ и $b$. 
%Студенты все время спрашивают ответы... Вот они: $\hat{a}=\ln(Y_{1})$, $\hat{b}=\ln(Y_{2})-\ln(Y_{1})$ 

\zad Пусть  $X_{1}$, ..., $X_{n}$  распределены одинаково и независимо.
Оцените значение $\theta $ с помощью ML (везде) и MM (в <<а>> и <<б>>), оцените дисперсию ML оценки, если функция плотности $X_{i}$, $p(t)$ имеет вид: \\
а) $\theta t^{\theta -1} $  при  $t\in \left[0;1\right]$;
б) $\frac{2t}{\theta ^{2}} $  при  $t\in \left[0;\theta \right]$\\
в) $\frac{\theta e^{-\frac{\theta ^{2} }{2t} } }{\sqrt{2\pi t^{3}
} } $  при  $t\in \left[0;+\infty \right)$;
г) $\frac{\theta \left(\ln ^{\theta -1} t\right)}{t} $  при  $t\in
\left[1;e\right]$;
д)  $\frac{e^{-\left|t\right|} }{2\left(1-e^{-\theta } \right)} $
при  $t\in \left[-\theta ;\theta \right]$ 

\zad <<Про зайцев>>. В темно-синем лесу, где трепещут осины, живут $n$ зайцев. Мы случайным образом отловили $100$ зайцев. Каждому из них на левое ухо мы завязали бант из красной ленточки и потом всех отпустили. Через неделю будет снова отловлено $100$ зайцев. Из них
$Z$ зайцев окажутся с бантами. \\
С помощью величины $Z$ постройте MM и ML оценку для неизвестного параметра $n$. 

\zad Пусть $X_{1}$, ..., $X_{n}$ - независимы и экспоненциальны с параметром $\lambda$. Постройте MM и ML оценки параметра  $\lambda$. Оцените дисперсию ML оценки.

\zad Пусть $X_{1}$, ..., $X_{n}$ - независимы и $N(\mu;\sigma^{2})$. Значение $\sigma^{2}$ известно.  Постройте MM и ML оценки параметра $\mu$.

\zad Пусть $X_{i}$ независимы и одинаково распределены $N(\alpha,2\alpha)$ \\
По выборке $X_{1}$, ..., $X_{n}$ постройте оценку для $\alpha$ с помощью ML и MM. Оцените дисперсию ML оценки.

\zad Пусть $Y_{1}\sim N(0;\frac{1}{1-\theta^{2}})$. \\
a) Найдите ML оценку для $\theta$. b) Оцените дисперсию ML оценки.

\zad Пусть $X_{1}$, $X_{2}$,..., $X_{n}$ независимы и их функции
плотности имеет вид: \\
$ f(x)=
\left\{%
\begin{array}{ll}
    (k+1)x^{k}, & x \in [0;1]; \\
    0, & x \notin [0;1]. \\
\end{array}%
\right.$ \\
Найдите оценки параметра $k$ с помощью ML и MM. Оцените дисперсию ML оценки.

\zad  Пусть $X_{1}$, $X_{2}$,..., $X_{n}$ независимы и
равномерно
распределены на отрезке $[0;\theta]$, $\theta>1$ \\
a) Постройте MM и ML оценки для неизвестного $\theta$. \\
б) Как изменятся ответы на <<a>>, если исследователь не
знает значений самих $X_{i}$, а знает только количество $X_{i}$
оказавшихся больше единицы? \\



\texttt{Минитеория} \\
Метод моментов (MM, method of moments): найти $\theta$ из уравнения $\bar{X_{n}}=\E(X_{i})$ \\
Метод максимального правдоподобия (ML, maximum likelihood): \\
найти $\theta$ при котором вероятность получить имеющиеся наблюдения будет максимальной \\
Наблюдаемая информация Фишера: $\hat{I}=-\frac{\partial^2 l}{\partial^2 \theta}(\hat{\theta})$, $\widehat{\Var}(\hat{\theta}_{ML})=\hat{I}^{-1}$. \\
Байесовский подход (bayesian approach): \\
1. Сделать изначальное предположение о распределении $\hat{\theta}$ \\
2. Обновлять закон распределения  $\hat{\theta}$ по формуле условной вероятности 


\newpage
\setcounter{zadacha}{0}

\zad В озере водятся караси, окуни, щуки и налимы. Вероятности их поймать занесены в табличку \\
\begin{tabular}{c|cccc}
  % after \\: \hline or \cline{col1-col2} \cline{col3-col4} ...
  Fish & Карась & Окунь & Щука & Налим \\
  \hline
  $\P()$ & 0.1 & $p$ & $p$ & $0.9-2p$ \\
\end{tabular} \\
Рыбак поймал 100 рыб и среди пойманных 100 рыб он посчитал количества карасей, окуней, щук и налимов. \\
а) Постройте $\hat{p}_{ML}$ \\
в) Найдите ожидаемую и наблюдаемую информацию Фишера \\
г) Несмещенная оценка $\hat{\theta}$ полученна по 100
наблюдениям: $X_{1}$, ..., $X_{100}$. \\
В каких пределах может лежать $\Var(\hat{\theta})$?

\zad Известно, что $X_{i}$ - независимы и имеют закон распределения, заданный таблицей: \\
\begin{tabular}{c|cc}
  % after \\: \hline or \cline{col1-col2} \cline{col3-col4} ...
  $X_i$ & 0 & 1 \\
  \hline
  $\P()$ & $p$ & $1-p$ \\
\end{tabular}\\
а) Постройте $\hat{p}_{ML}$
б) Найдите ожидаемую и наблюдаемую информацию Фишера. Постройте возможные графики $I(p)$.
в) Пусть $\hat{\theta}$ - несмещенная оценка, полученная по 100 наблюдениям: $X_{1}$, ..., $X_{100}$. \\ 
В каких пределах может лежать $\Var(\hat{\theta})$?

\zad Пусть $X_{i}$ независимы и имеют экспоненциальное распределение с параметром
$\lambda$, т.е. $p(t)=\lambda e^{-\lambda t}$. \\
a) Найдите $I(\lambda)$, если наблюдаются $X_{1}$, ..., $X_{n}$ \\
б) Пусть $\lambda=1/\theta$, т.е. $p(t)=\frac{1}{\theta}
e^{-\frac{1}{\theta} t}$. Найдите $I(\theta)$, если наблюдается $X_{1}$, ..., $X_{n}$ 
%в) Пусть $\hat{\theta}$ - несмещенная оценка, полученная по 100
%наблюдениям. В каких пределах может лежать $\Var(\hat{\theta})$?

\zad Пусть $X_{i}$ - независимы и одинаково распределены. Пусть $I_{X_{i}}(\theta)$ - информация Фишера о $\theta$, получаемая при наблюдении $X_{i}$. \\
а) Верно ли, что $I_{X_{1}}(\theta)=I_{X_{2}}(\theta)$? б) Как найти $I_{X_{1}, ...,X_{n}}(\theta)$ зная $I_{X_{i}}(\theta)$? 

\zad Пусть $X$ - равномерна на участке $[0;2a]$. С какой вероятностью интервал $[0.9X;1.1X]$ накрывает неизвестное $a$? Постройте 95\%-ый доверительный интервал для $a$ вида $[0;kX]$. 

\zad Пусть $X$ - экспоненциальна с параметром $\lambda$ и $\mu=\E(X)$. C какой вероятностью интервал $[0.9X;1.1X]$ накрывает $\mu$? Постройте 90\%-ый доверительный интервал для $\mu$ вида $[0;kX]$. 

\zad Пусть $X_i$ - независимы и нормальны $N(\mu,1)$. Какова вероятность того, что интервал $[\bar{X}_{10}-1;\bar{X}_{10}+1]$ накроет неизвестное $\mu$? Постройте 90\%-ый доверительный интервал для $\mu$ вида $[\bar{X}_{10}-k;\bar{X}_{10}+k]$. 

\zad Величины $X_{1}$, ..., $X_{n}$ - независимы и одинаково распределены с функцией плотности $\frac{\theta \left(\ln ^{\theta -1} t\right)}{t} $  при  $t\in
\left[1;e\right]$. По выборке из 100 наблюдений оказалось, что $\sum{\ln(\ln(X_{i}))}=-30$ \\
а) Найдите ML оценку параметра $\theta$ и ожидаемую и наблюдаемую информацию Фишера \\
г) Постройте 95\% доверительный интервал для $\theta$ 

\zad Величины $X_{1}$, ..., $X_{n}$ - независимы и одинаково распределены с функцией плотности $\frac{\theta e^{-\frac{\theta^{2}}{2t}}}{\sqrt{2\pi t^{3}
}}$ при $t\in \left[0;+\infty \right)$. По выборке из 100 наблюдений оказалось, что $\sum{1/X_{i}}=12$ \\
а) Найдите ML оценку параметра $\theta$ и информацию Фишера $I(\theta)$ \\
в) Пользуясь данными по выборке постройте оценку $\hat{I}$ для информации Фишера \\
г) Постройте 90\% доверительный интервал для $\theta$ 
Hint: $\E(1/X_{i})=1/\theta^{2}$ (интеграл берется заменой $x=\theta^{2}a^{-2}$) 

\zad Время, которое Вася тратит на задачу --- равномерно распределенная случайная величина: на простую - от 1 до 15 минут, на сложную - от 10 до 20 минут. Известно, что на некую задачу Вася потратил 13 минут. 
\begin{enumerate}
\item С помощью метода максимального правдоподобия определите, простая она или трудная. 
\item С помощью байесовского подхода посчитайте вероятности того, что задача была простая, если на экзамене было 7 легких и 3 трудных задачи.
\end{enumerate}

\texttt{Минитеория} \\
Пусть $l(\theta)$ - логарифмическая функция правдоподобия ($l(\theta)=\ln(f(X_{1}, ...,X_{n},\theta))$). \\
Ожидаемая информация Фишера $I(\theta)=\E\left[\left(\frac{\partial
l}{\partial \theta}\right)^{2}\right]=-E\left(\frac{\partial^{2}l}{\partial \theta^{2}}\right)$ \\
Сколько информации о неизвестном $\theta$ содержится в выборке $X_{1}$, ..., $X_{n}$
%Если $X_{i}$ - независимы и одинаково распределены, то $I(\theta)=n %I_{X_{i}}(\theta)$, где $I_{X_{i}}(\theta)$ - информация о неизвестном $\theta$, %содержащаяся в наблюдении одного $X_{i}$ \\

Неравенство Крамера-Рао (Cramer-Rao) (<<слишком хорошей оценки не бывает>>): \\
Если $\hat{\theta}$ - несмещенная оценка и ..., то
$\Var(\hat{\theta})\ge \frac{1}{I(\theta)}$ 

Оценки ML - самые лучшие (асимптотически несмещенные и с минимальной дисперсий): \\
Если $X_{i}$ - iid, ..., и $n\to\infty$ то $\hat{\theta}_{ML}\sim N(\theta,\frac{1}{I(\theta)})$. 

%В качестве оценки $\hat{I}$ берут $I(\hat{\theta})$ \\


\newpage
\setcounter{zadacha}{0}

\zad Известно, что $X_i$ независимы, $\E(X_i)=5$, $\Var(X_i)=4$. Как примерно распределены следующие величины:
а) $\bar{X}_n$, b) $Y_n=(\bar{X}_n+3)/(\bar{X}_n+6)$, c) $Z_n=\bar{X}_n^2$, d) $W_n=1/\bar{X}_n$

\zad Известно, что $X_i$ независимы и равномерны на $[0;1]$.

a) Найдите $\E(\ln(X_i))$, $\Var(\ln(X_i))$, $\E(X_i^2)$, $\Var(X_i^2)$

b) Как примерно распределены величины $X_n=\frac{\sum\ln(X_i)}{n}$, $Y_n=(X_1\cdot X_2\cdots X_n)^{1/n}$, $Z_n=\left(\frac{\sum X_i^2}{n} \right)^3$

\zad Величины $X_i$ независимы и имеют функцию плотности $f(x)=a\cdot x^{a-1}$ на отрезке $[0;1]$.

a) Постройте оценку $\hat{a}$ методом моментов, укажите ее примерный закон распределения

b) По 100 наблюдениям оказалось, что $\sum X_i=25$. Посчитайте численное значение $\hat{a}$ и оцените дисперсию случайной величины $\hat{a}$.

\zad Вася достает из м-энд-эм-сины из большой желтой пачки. Он собрал следующую статистику:
\begin{tabular}{c|cccc}
Цвет & Желтый & Зеленый & Красный & Коричневый \\ 
\hline 
Количество & 25 & 19 & 27 & 33 \\ 
\end{tabular} 
\begin{enumerate}
\item С помощью ML оцените вероятность вытащить м-энд-эм-сину каждого цвета
\item С помощью теста отношения правдоподобия проверьте гипотезу, что все цвета равновероятны на уровне значимости 5\%
\end{enumerate}

\zad 


\zad Пользуясь примером про Васю и эм-энд-эм-сины найдите общую формулу хи-квадрат статистики для проверки гипотезы о том, что настоящие вероятности равны

\begin{tabular}{c|cccc}
Категория & 1 & 2 & 3 & \ldots \\ 
\hline 
Вероятность & $p_1$ & $p_2$ & $p_3$ & \ldots \\ 
\end{tabular},

если имеются данные о фактических частотах

\begin{tabular}{c|cccc}
Категория & 1 & 2 & 3 & \ldots \\ 
\hline 
Количество & $X_1$ & $X_2$ & $X_3$ & \ldots \\ 
\end{tabular}.



\zad Монету подбросили 1000 раз, при этом 519 раз она выпала на
орла. Проверьте гипотезу о том, что монета <<правильная>> на
уровне значимости 5\%. Постройте 95\% доверительный интервал для
вероятности выпадения орла, $p$.

\zad Вася отвечает на 100 тестовых вопросов. В каждом вопросе один
правильный вариант ответа из пяти возможных. На 5\%-ом уровне
значимости проверьте гипотезу о том, что Вася ставит ответы
наугад, если он ответил правильно на 26 вопросов из теста.

\zad Пусть $X_{1}$, ..., $X_{30}$ независимы, $N(\mu,\sigma^{2})$. Известно, что $\sum (X_{i}-\bar{X})^{2}=600$. Проверьте гипотезу о том, что $\sigma^{2}=15$ против альтернативной $\sigma^{2}>15$ на 5\%-ом уровне значимости. 



\texttt{Минитеория} \\
Дельта-метод. Если $\sqrt{n}(\hat{\mu}_n-\mu)\to N(0;\sigma^2)$, то $\sqrt{n}(g(\hat{\mu}_n)-g(\mu))\to N(0;\sigma^2\cdot (g'(\mu))^2)$


Статистика отношения правдоподобия, $LR=-2(\max_{H_0} l(\theta_1,\ldots,\theta_n)-\max l(\theta_1,\ldots,\theta_n))$




\end{document}

\newpage
\setcounter{zadacha}{0}

Попробуйте проверить каждую гипотезу, используя и не используя ML. \\


\texttt{Минитеория} \\
1. Формулируем $H_{0}$, $H_{a}$. \\
2. Выбираем порог редкости, уровень значимости ($\alpha=0.05$) \\
3. Придумываем какую-нибудь случайную величину, так чтобы $H_{0}$ утаскивало ее значения в одну сторону, а $H_{a}$ - в другую. \\
4. Предполагая верной $H_{0}$, считаем вероятность получить такую выборку как имеется или еще более близкую к $H_{a}$. Это P-value. \\
5. Если P-value>порога редкости, значит выборка <<совместима>> с $H_{0}$ и $H_{0}$ не отвергается. \\
Если P-value<порога редкости, значит произошло слишком редкое событие для $H_{0}$ и $H_{0}$ отвергается. \\
