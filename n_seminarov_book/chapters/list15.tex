

Листок 15 по ТВ и МС 2013--2014 [08.03.2014]







1

Кафедра математической экономики и эконометрики НИУ ВШЭ. Борзых Д. А.

Листок 15

$\chi ^{2} {\rm -}$критерий Пирсона



\textbf{Задача 1.} Вася решил проверить известное утверждение о том, что бутерброд падает маслом вниз. Для этого он провел серию из 200 испытаний. Ниже приведена таблица с результатами.

\begin{tabular}{|p{1.1in}|p{0.8in}|p{0.8in}|} \hline 
Бутерброд с маслом & Маслом вниз & Маслом вверх \\ \hline 
Число наблюдений & 105 & 95 \\ \hline 
\end{tabular}

Можно ли утверждать, что бутерброд падает маслом вниз также часто, как и маслом вверх? Уровень значимости $\alpha =0.01$.



\textbf{Решение.} $n=200$; $r=2$.\underbar{}

Пусть $X_{j} =\left\{\begin{array}{l} {1,\; c\; 25@.\; p} \\ {0,\; c\; 25@.\; \left(1-p\right)} \end{array}\right. $, $\left(j=\overline{1,n}\right)$. Если $x_{j} =1$, то будем считать, что в $j-$м испытании бутерброд упал маслом вниз, если же $x_{j} =0$, то будем считать, что в $j-$м испытании бутерброд упал маслом вверх.

Требуется проверить гипотезу $H_{0} :p=0.5$.

Известно, что $W_{n} =\sum _{i=1}^{r}\frac{\left(\nu _{i} -n\cdot p_{i} \right)^{2} }{n\cdot p_{i} }  \mathop{\sim }\limits^{as} \chi ^{2} \left(r-1\right)$. Поэтому считаем, что $W_{200} =\sum _{i=1}^{2}\frac{\left(\nu _{i} -n\cdot p_{i} \right)^{2} }{n\cdot p_{i} }  \sim \chi ^{2} \left(1\right)$, где $p_{1} =P\left(\left\{X_{j} =1\right\}\right)=0.5$ и $p_{2} =P\left(\left\{X_{j} =0\right\}\right)=0.5$.

\[\nu _{1} =105, \nu _{2} =95.\] 

\[W_{200,=01;} =\frac{\left(105-200\cdot 0.5\right)^{2} }{200\cdot 0.5} +\frac{\left(95-200\cdot 0.5\right)^{2} }{200\cdot 0.5} =\frac{50}{100} =0.5.\] 

\[\chi _{:@}^{2} =6.63.\] 



\textbf{Задача 2.} Игральный кубик подбрасывался 1389 раз. В следующей таблице приведено, сколько раз выпала каждая грань.

\begin{tabular}{|p{0.3in}|p{0.3in}|p{0.3in}|p{0.3in}|p{0.3in}|p{0.3in}|} \hline 
1 & 2 & 3 & 4 & 5 & 6 \\ \hline 
234 & 229 & 240 & 219 & 236 & 231 \\ \hline 
\end{tabular}

Проверьте гипотезу о том, что кубик правильный. Уровень значимости $\alpha =0.05$.

\textbf{Решение.} $X_{j} \left(\Omega \right)=S=\left\{1;2;3;4;5;6\right\}$;

\[S_{1} =\left\{1\right\}; S_{2} =\left\{2\right\}; S_{3} =\left\{3\right\}; S_{4} =\left\{4\right\}; S_{5} =\left\{5\right\}; S_{6} =\left\{6\right\}; r=6.\] 

\[n=1389.\] 

Основная гипотеза означает, что

\[p_{i} =P\left(\left\{X_{j} =i\right\}\right)=\frac{1}{6}  \left(i=\overline{1,6}\right).\] 

Известно, что $W_{n} =\sum _{i=1}^{r}\frac{\left(\nu _{i} -n\cdot p_{i} \right)^{2} }{n\cdot p_{i} }  \mathop{\sim }\limits^{as} \chi ^{2} \left(r-1\right)$.

Поэтому считаем, что $W_{1389} =\sum _{i=1}^{6}\frac{\left(\nu _{i} -n\cdot p_{i} \right)^{2} }{n\cdot p_{i} }  \sim \chi ^{2} \left(5\right)$.

\[\nu _{1} =234; \nu _{2} =229; \nu _{3} =240; \nu _{4} =219; \nu _{5} =236; \nu _{6} =231.\] 

\[W_{1389,\; =01;.} =\sum _{i=1}^{6}\frac{\left(\nu _{i} -n\cdot p_{i} \right)^{2} }{n\cdot p_{i} }  =1.12.\] 

\[\chi _{:@}^{2} =11.07.\] 

Поскольку $W_{1389,\; =01;.} =1.12\in \left[0;11.07\right]$, то основная гипотеза $H_{0} :p_{1} =p_{2} =p_{3} =p_{4} =p_{5} =p_{6} ={\tfrac{1}{6}} $ не отвергается.



\textbf{Задача 3.} Вася Сидоров утверждает, что ходит в кино в два раза чаще, чем в спортзал, а в спортзал  в два раза чаще, чем в театр. За последние полгода он 10 раз был в театре, 17 раз -- в спортзале и 39 раз в кино. Правдоподобно ли Васино утверждение?



\textbf{Задача 4.} Монета подбрасывалась 4040 раз, при этом герб выпал 2048 раз. Согласуются ли эти данные с гипотезой о симметричности монеты на уровне значимости $\alpha =0.1?$



\textbf{Задача 5.} Монета подбрасывалась 400 раз, при этом герб выпал 209 раз. Согласуются ли эти данные с гипотезой о симметричности монеты на уровне значимости $\alpha =0.01?$



\textbf{Задача 6.} Пусть задана реализация случайной выборки $x=\left(x_{1} ,...,x_{n} \right)$ из распределения Бернулли с неизвестной вероятностью ``успеха'' $p$, т.е. $P\left(\left\{X_{j} =l\right\}\right)=p^{l} \cdot \left(1-p\right)^{1-l} $, $\left(l=0,1\right)$. Известно, что реализация случайной выборки $x=\left(x_{1} ,...,x_{n} \right)$ содержит 192 нуля и  208 единиц. Используя $\chi ^{2} -$критерий согласия Пирсона, требуется на уровне значимости $\alpha =0.05$ проверить гипотезу $H_{0} :p=0.5$ против альтернативной гипотезы $H_{1} :p\ne 0.5$.



\textbf{Задача 7.} Пусть задана реализация случайной выборки $x=\left(x_{1} ,...,x_{n} \right)$ из распределения Бернулли с неизвестной вероятностью ``успеха'' $p$, т.е. $P\left(\left\{X_{j} =l\right\}\right)=p^{l} \cdot \left(1-p\right)^{1-l} $, $\left(l=0,1\right)$. Известно, что реализация случайной выборки $x=\left(x_{1} ,...,x_{n} \right)$ содержит 251 ноль и  149 единиц. Используя $\chi ^{2} -$критерий согласия Пирсона, требуется на уровне значимости $\alpha =0.01$ проверить гипотезу $H_{0} :p=0.4$ против альтернативной гипотезы $H_{1} :p\ne 0.4$.



\textbf{Задача 8.} Пусть задана реализация случайной выборки $x=\left(x_{1} ,...,x_{n} \right)$ из распределения Бернулли с неизвестной вероятностью ``успеха'' $p$, т.е. $P\left(\left\{X_{j} =l\right\}\right)=p^{l} \cdot \left(1-p\right)^{1-l} $, $\left(l=0,1\right)$. Известно, что реализация случайной выборки $x=\left(x_{1} ,...,x_{n} \right)$ содержит 359 нулей и  41 единицу. Используя $\chi ^{2} -$критерий согласия Пирсона, требуется на уровне значимости $\alpha =0.01$ проверить гипотезу $H_{0} :p=0.2$ против альтернативной гипотезы $H_{1} :p\ne 0.2$.



\textbf{Задача 9.} Пусть задана реализация случайной выборки $x=\left(x_{1} ,...,x_{n} \right)$ из распределения Бернулли с неизвестной вероятностью ``успеха'' $p$, т.е. $P\left(\left\{X_{j} =l\right\}\right)=p^{l} \cdot \left(1-p\right)^{1-l} $, $\left(l=0,1\right)$. Известно, что реализация случайной выборки $x=\left(x_{1} ,...,x_{n} \right)$ содержит 359 нулей и  41 единицу. Используя $\chi ^{2} -$критерий согласия Пирсона, требуется на уровне значимости $\alpha =0.01$ проверить гипотезу $H_{0} :p=0.1$ против альтернативной гипотезы $H_{1} :p\ne 0.1$.



\textbf{Задача 10.} Пусть задана реализация случайной выборки $x=\left(x_{1} ,...,x_{n} \right)$ из биномиального распределения с неизвестной вероятностью ``успеха'' $p$ и числом испытаний $k=3$, т.е. $P\left(\left\{X_{j} =l\right\}\right)=C_{k}^{l} \cdot p^{l} \cdot \left(1-p\right)^{k-l} $, $\left(l=\overline{0,3}\right)$. Известно, что реализация случайной выборки $x=\left(x_{1} ,...,x_{n} \right)$ содержит 47 нулей, 163 единицы, 148 двоек и 42 тройки. Используя $\chi ^{2} -$критерий согласия Пирсона, требуется на уровне значимости $\alpha =0.05$ проверить гипотезу $H_{0} :p=0.5$ против альтернативной гипотезы $H_{1} :p\ne 0.5$.

\textbf{Решение.} $X_{j} \left(\Omega \right)=S=\left\{0;1;2;3\right\}$;

\[S_{1} =\left\{0\right\}; S_{2} =\left\{1\right\}; S_{3} =\left\{2\right\}; S_{4} =\left\{3\right\}; r=4.\] 

\[n=400=47+163+148+42.\] 

Основная гипотеза означает, что

\[p_{1} =P\left(\left\{X_{j} =0\right\}\right)=C_{3}^{0} \cdot 0.5^{0} \cdot \left(1-0.5\right)^{3-0} =0.125;\] 

\[p_{2} =P\left(\left\{X_{j} =1\right\}\right)=C_{3}^{1} \cdot 0.5^{1} \cdot \left(1-0.5\right)^{3-1} =0.375;\] 

\[p_{3} =P\left(\left\{X_{j} =2\right\}\right)=C_{3}^{2} \cdot 0.5^{2} \cdot \left(1-0.5\right)^{3-2} =0.375;\] 

\[p_{4} =P\left(\left\{X_{j} =3\right\}\right)=C_{3}^{3} \cdot 0.5^{3} \cdot \left(1-0.5\right)^{3-3} =0.125.\] 

Известно, что $W_{n} =\sum _{i=1}^{r}\frac{\left(\nu _{i} -n\cdot p_{i} \right)^{2} }{n\cdot p_{i} }  \mathop{\sim }\limits^{as} \chi ^{2} \left(r-1\right)$.

Поэтому считаем, что $W_{400} =\sum _{i=1}^{4}\frac{\left(\nu _{i} -n\cdot p_{i} \right)^{2} }{n\cdot p_{i} }  \sim \chi ^{2} \left(3\right)$.

\[\nu _{1} =47; \nu _{2} =163; \nu _{3} =148; \nu _{4} =42.\] 

\[W_{400,\; =01;.} =\sum _{i=1}^{4}\frac{\left(\nu _{i} -n\cdot p_{i} \right)^{2} }{n\cdot p_{i} }  =\] 

\[=\frac{\left(47-400\cdot 0.125\right)^{2} }{400\cdot 0.125} +\frac{\left(163-400\cdot 0.375\right)^{2} }{400\cdot 0.375} +\frac{\left(148-400\cdot 0.375\right)^{2} }{400\cdot 0.375} +\frac{\left(42-400\cdot 0.125\right)^{2} }{400\cdot 0.125} =2.61;\] 

\[\chi _{:@}^{2} =7.81.\] 

Поскольку $W_{400,\; =01;.} =2.61\in \left[0;7.81\right]$, то основная гипотеза $H_{0} :p=0.5$ не отвергается.



\textbf{Задача 11.} Пусть задана реализация случайной выборки $x=\left(x_{1} ,...,x_{n} \right)$ из биномиального распределения с неизвестной вероятностью ``успеха'' $p$ и числом испытаний $k=3$, т.е. $P\left(\left\{X_{j} =l\right\}\right)=C_{k}^{l} \cdot p^{l} \cdot \left(1-p\right)^{k-l} $, $\left(l=\overline{0,3}\right)$. Известно, что реализация случайной выборки $x=\left(x_{1} ,...,x_{n} \right)$ содержит 55 нулей, 139 единиц, 147 двоек и 59 троек. Используя $\chi ^{2} -$критерий согласия Пирсона, требуется на уровне значимости $\alpha =0.05$ проверить гипотезу $H_{0} :p=0.5$ против альтернативной гипотезы $H_{1} :p\ne 0.5$.



\textbf{Задача 12.} Пусть задана реализация случайной выборки $x=\left(x_{1} ,...,x_{n} \right)$ из биномиального распределения с неизвестной вероятностью ``успеха'' $p$ и числом испытаний $k=3$, т.е. $P\left(\left\{X_{j} =l\right\}\right)=C_{k}^{l} \cdot p^{l} \cdot \left(1-p\right)^{k-l} $, $\left(l=\overline{0,3}\right)$. Известно, что реализация случайной выборки $x=\left(x_{1} ,...,x_{n} \right)$ содержит 10 нулей, 73 единицы, 174 двойки и 143 тройки. Используя $\chi ^{2} -$критерий согласия Пирсона, требуется на уровне значимости $\alpha =0.05$ проверить гипотезу $H_{0} :p=0.5$ против альтернативной гипотезы $H_{1} :p\ne 0.5$.

\textbf{Решение.} $X_{j} \left(\Omega \right)=S=\left\{0;1;2;3\right\}$;

\[S_{1} =\left\{0\right\}; S_{2} =\left\{1\right\}; S_{3} =\left\{2\right\}; S_{4} =\left\{3\right\}; r=4.\] 

\[n=400=10+73+174+143.\] 

Основная гипотеза означает, что

\[p_{1} =P\left(\left\{X_{j} =0\right\}\right)=C_{3}^{0} \cdot 0.5^{0} \cdot \left(1-0.5\right)^{3-0} =0.125;\] 

\[p_{2} =P\left(\left\{X_{j} =1\right\}\right)=C_{3}^{1} \cdot 0.5^{1} \cdot \left(1-0.5\right)^{3-1} =0.375;\] 

\[p_{3} =P\left(\left\{X_{j} =2\right\}\right)=C_{3}^{2} \cdot 0.5^{2} \cdot \left(1-0.5\right)^{3-2} =0.375;\] 

\[p_{4} =P\left(\left\{X_{j} =3\right\}\right)=C_{3}^{3} \cdot 0.5^{3} \cdot \left(1-0.5\right)^{3-3} =0.125.\] 

Известно, что $W_{n} =\sum _{i=1}^{r}\frac{\left(\nu _{i} -n\cdot p_{i} \right)^{2} }{n\cdot p_{i} }  \mathop{\sim }\limits^{as} \chi ^{2} \left(r-1\right)$.

Поэтому считаем, что $W_{400} =\sum _{i=1}^{4}\frac{\left(\nu _{i} -n\cdot p_{i} \right)^{2} }{n\cdot p_{i} }  \sim \chi ^{2} \left(3\right)$.

\[\nu _{1} =10; \nu _{2} =73; \nu _{3} =174; \nu _{4} =143.\] 

\[W_{400,\; =01;.} =\sum _{i=1}^{4}\frac{\left(\nu _{i} -n\cdot p_{i} \right)^{2} }{n\cdot p_{i} }  =\] 

\[=\frac{\left(10-400\cdot 0.125\right)^{2} }{400\cdot 0.125} +\frac{\left(73-400\cdot 0.375\right)^{2} }{400\cdot 0.375} +\frac{\left(174-400\cdot 0.375\right)^{2} }{400\cdot 0.375} +\frac{\left(143-400\cdot 0.125\right)^{2} }{400\cdot 0.125} \approx 248;\] 

\[\chi _{:@}^{2} =7.81.\] 

Поскольку $W_{400,\; =01;.} =248\notin \left[0;7.81\right]$, то основная гипотеза $H_{0} :p=0.5$ отвергается в пользу альтернативной гипотезы $H_{1} :p\ne 0.5$.



\textbf{Задача 13.} Пусть задана реализация случайной выборки $x=\left(x_{1} ,...,x_{n} \right)$ из биномиального распределения с неизвестной вероятностью ``успеха'' $p$ и числом испытаний $k=3$, т.е. $P\left(\left\{X_{j} =l\right\}\right)=C_{k}^{l} \cdot p^{l} \cdot \left(1-p\right)^{k-l} $, $\left(l=\overline{0,3}\right)$. Известно, что реализация случайной выборки $x=\left(x_{1} ,...,x_{n} \right)$ содержит 81 ноль, 162 единицы, 132 двойки и 25 троек. Используя $\chi ^{2} -$критерий согласия Пирсона, требуется на уровне значимости $\alpha =0.05$ проверить гипотезу $H_{0} :p=0.5$ против альтернативной гипотезы $H_{1} :p\ne 0.5$.



\textbf{Задача 14.} Пусть задана реализация случайной выборки $x=\left(x_{1} ,...,x_{n} \right)$ из биномиального распределения с неизвестной вероятностью ``успеха'' $p$ и числом испытаний $k=3$, т.е. $P\left(\left\{X_{j} =l\right\}\right)=C_{k}^{l} \cdot p^{l} \cdot \left(1-p\right)^{k-l} $, $\left(l=\overline{0,3}\right)$. Известно, что реализация случайной выборки $x=\left(x_{1} ,...,x_{n} \right)$ содержит 81 ноль, 162 единицы, 132 двойки и 25 троек. Используя $\chi ^{2} -$критерий согласия Пирсона, требуется на уровне значимости $\alpha =0.05$ проверить гипотезу $H_{0} :p=0.4$ против альтернативной гипотезы $H_{1} :p\ne 0.4$.



\textbf{Задача 15.} [Источник случайных чисел: Шведов А.С., ТВиМС-2, стр. 213] Пусть задана реализация случайной выборки $x=\left(x_{1} ,...,x_{n} \right)$ объема $n=200$ наблюдений. И пусть заданы множества $S_{1} =\left[0;{\tfrac{1}{8}} \right]$; $S_{i} =\left({\tfrac{i-1}{8}} ;{\tfrac{i}{8}} \right]$, $i=\overline{2,8}$. 

В следующей таблице указано число элементов выборки $x=\left(x_{1} ,...,x_{n} \right)$, попавших в каждое множество $S_{i} $, $i=\overline{1,8}$.

\begin{tabular}{|p{0.3in}|p{0.3in}|p{0.3in}|p{0.3in}|p{0.3in}|p{0.3in}|p{0.3in}|p{0.3in}|} \hline 
$S_{1} $ & $S_{2} $ & $S_{3} $ & $S_{4} $ & $S_{5} $ & $S_{6} $ & $S_{7} $ & $S_{8} $ \\ \hline 
29 & 22 & 24 & 22 & 27 & 22 & 26 & 28 \\ \hline 
\end{tabular}



Используя $\chi ^{2} -$критерий согласия Пирсона, требуется на уровне значимости $\alpha =0.01$ проверить гипотезу о том, что элементы случайной выборки имеют равномерное на отрезке $\left[0;1\right]$ распределение.



\textbf{Задача 16.} [Источник случайных чисел: датчик excel] Пусть задана реализация случайной выборки $x=\left(x_{1} ,...,x_{n} \right)$ объема $n=200$ наблюдений. И пусть заданы множества $S_{1} =\left[0;{\tfrac{1}{8}} \right]$; $S_{i} =\left({\tfrac{i-1}{8}} ;{\tfrac{i}{8}} \right]$, $i=\overline{2,8}$.

В следующей таблице указано число элементов выборки $x=\left(x_{1} ,...,x_{n} \right)$, попавших в каждое множество $S_{i} $, $i=\overline{1,8}$.

\begin{tabular}{|p{0.3in}|p{0.3in}|p{0.3in}|p{0.3in}|p{0.3in}|p{0.3in}|p{0.3in}|p{0.3in}|} \hline 
$S_{1} $ & $S_{2} $ & $S_{3} $ & $S_{4} $ & $S_{5} $ & $S_{6} $ & $S_{7} $ & $S_{8} $ \\ \hline 
30 & 34 & 21 & 20 & 24 & 22 & 21 & 28 \\ \hline 
\end{tabular}



Используя $\chi ^{2} -$критерий согласия Пирсона, требуется на уровне значимости $\alpha =0.01$ проверить гипотезу о том, что элементы случайной выборки имеют равномерное на отрезке $\left[0;1\right]$ распределение.

\textbf{Решение.} $r=8$;$n=200$.

Основная гипотеза означает, что

\[p_{i} =P\left(\left\{X_{j} \in S_{i} \right\}\right)={\tfrac{1}{8}}  \left(i=\overline{1,8}\right).\] 

Известно, что $W_{n} =\sum _{i=1}^{r}\frac{\left(\nu _{i} -n\cdot p_{i} \right)^{2} }{n\cdot p_{i} }  \mathop{\sim }\limits^{as} \chi ^{2} \left(r-1\right)$.

Поэтому считаем, что $W_{200} =\sum _{i=1}^{4}\frac{\left(\nu _{i} -n\cdot p_{i} \right)^{2} }{n\cdot p_{i} }  \sim \chi ^{2} \left(7\right)$.

\[\nu _{1} =30; \nu _{2} =34; \nu _{3} =21; \nu _{4} =20; \nu _{5} =24; \nu _{6} =22; \nu _{7} =21; \nu _{8} =28.\] 



\[W_{200,\; =01;.} =\sum _{i=1}^{8}\frac{\left(\nu _{i} -n\cdot p_{i} \right)^{2} }{n\cdot p_{i} }  =\] 

\[\begin{array}{l} {=\frac{\left(30-25\right)^{2} }{25} +\frac{\left(34-25\right)^{2} }{25} +\frac{\left(21-25\right)^{2} }{25} +\frac{\left(20-25\right)^{2} }{25} +} \\ {+\frac{\left(24-25\right)^{2} }{25} +\frac{\left(22-25\right)^{2} }{25} +\frac{\left(21-25\right)^{2} }{25} +\frac{\left(28-25\right)^{2} }{25} =7.28} \end{array}\] 

\[\chi _{:@}^{2} =18.47.\] 

Поскольку $W_{200,\; =01;.} =7.28\in \left[0;18.47\right]$, то основная гипотеза о равномерности распределения на отрезке $\left[0;1\right]$ не может быть отвергнута.



\textbf{Задача 17.} [Источник случайных чисел: Шведов А.С., ТВиМС-2, стр. 213] Пусть задана реализация случайной выборки $x=\left(x_{1} ,...,x_{n} \right)$ объема $n=200$ наблюдений. И пусть заданы множества $S_{1} =\left[0;{\tfrac{1}{13}} \right]$; $S_{i} =\left({\tfrac{i-1}{13}} ;{\tfrac{i}{13}} \right]$, $i=\overline{2,13}$. 

В следующей таблице указано число элементов выборки $x=\left(x_{1} ,...,x_{n} \right)$, попавших в каждое множество $S_{i} $, $i=\overline{1,13}$.

\begin{tabular}{|p{0.3in}|p{0.3in}|p{0.3in}|p{0.3in}|p{0.3in}|p{0.3in}|p{0.3in}|p{0.3in}|p{0.3in}|p{0.3in}|p{0.3in}|p{0.3in}|p{0.3in}|} \hline 
$S_{1} $ & $S_{2} $ & $S_{3} $ & $S_{4} $ & $S_{5} $ & $S_{6} $ & $S_{7} $ & $S_{8} $ & $S_{9} $ & $S_{10} $ & $S_{11} $ & $S_{12} $ & $S_{13} $ \\ \hline 
15 & 21 & 12 & 13 & 16 & 13 & 15 & 19 & 10 & 21 & 11 & 16 & 18 \\ \hline 
\end{tabular}



Используя $\chi ^{2} -$критерий согласия Пирсона, требуется на уровне значимости $\alpha =0.01$ проверить гипотезу о том, что элементы случайной выборки имеют равномерное на отрезке $\left[0;1\right]$ распределение.



\textbf{Задача 18.} [Источник случайных чисел: датчик excel] Пусть задана реализация случайной выборки $x=\left(x_{1} ,...,x_{n} \right)$ объема $n=200$ наблюдений. И пусть заданы множества $S_{1} =\left[0;{\tfrac{1}{13}} \right]$; $S_{i} =\left({\tfrac{i-1}{13}} ;{\tfrac{i}{13}} \right]$, $i=\overline{2,13}$. 

В следующей таблице указано число элементов выборки $x=\left(x_{1} ,...,x_{n} \right)$, попавших в каждое множество $S_{i} $, $i=\overline{1,13}$.

\begin{tabular}{|p{0.3in}|p{0.3in}|p{0.3in}|p{0.3in}|p{0.3in}|p{0.3in}|p{0.3in}|p{0.3in}|p{0.3in}|p{0.3in}|p{0.3in}|p{0.3in}|p{0.3in}|} \hline 
$S_{1} $ & $S_{2} $ & $S_{3} $ & $S_{4} $ & $S_{5} $ & $S_{6} $ & $S_{7} $ & $S_{8} $ & $S_{9} $ & $S_{10} $ & $S_{11} $ & $S_{12} $ & $S_{13} $ \\ \hline 
20 & 20 & 16 & 16 & 14 & 10 & 15 & 16 & 15 & 13 & 9 & 20 & 16 \\ \hline 
\end{tabular}



Используя $\chi ^{2} -$критерий согласия Пирсона, требуется на уровне значимости $\alpha =0.01$ проверить гипотезу о том, что элементы случайной выборки имеют равномерное на отрезке $\left[0;1\right]$ распределение.



\textbf{Задача 19.} [Источник случайных чисел: Шведов А.С., ТВиМС-2, стр. 213] Пусть задана реализация случайной выборки $x=\left(x_{1} ,...,x_{n} \right)$ объема $n=200$ наблюдений. И пусть заданы множества $S_{1} =\left[0;{\tfrac{1}{10}} \right]$; $S_{i} =\left({\tfrac{i-1}{10}} ;{\tfrac{i}{10}} \right]$, $i=\overline{2,10}$. 

В следующей таблице указано число элементов выборки $x=\left(x_{1} ,...,x_{n} \right)$, попавших в каждое множество $S_{i} $, $i=\overline{1,10}$.

\begin{tabular}{|p{0.3in}|p{0.3in}|p{0.3in}|p{0.3in}|p{0.3in}|p{0.3in}|p{0.3in}|p{0.3in}|p{0.3in}|p{0.3in}|} \hline 
$S_{1} $ & $S_{2} $ & $S_{3} $ & $S_{4} $ & $S_{5} $ & $S_{6} $ & $S_{7} $ & $S_{8} $ & $S_{9} $ & $S_{10} $ \\ \hline 
21 & 22 & 14 & 22 & 18 & 25 & 15 & 22 & 21 & 20 \\ \hline 
\end{tabular}



Используя $\chi ^{2} -$критерий согласия Пирсона, требуется на уровне значимости $\alpha =0.01$ проверить гипотезу о том, что элементы случайной выборки имеют равномерное на отрезке $\left[0;1\right]$ распределение.



\textbf{Задача 20.} [Источник случайных чисел: датчик excel] Пусть задана реализация случайной выборки $x=\left(x_{1} ,...,x_{n} \right)$ объема $n=200$ наблюдений. И пусть заданы множества $S_{1} =\left[0;{\tfrac{1}{10}} \right]$; $S_{i} =\left({\tfrac{i-1}{10}} ;{\tfrac{i}{10}} \right]$, $i=\overline{2,10}$. 

В следующей таблице указано число элементов выборки $x=\left(x_{1} ,...,x_{n} \right)$, попавших в каждое множество $S_{i} $, $i=\overline{1,10}$.

\begin{tabular}{|p{0.3in}|p{0.3in}|p{0.3in}|p{0.3in}|p{0.3in}|p{0.3in}|p{0.3in}|p{0.3in}|p{0.3in}|p{0.3in}|} \hline 
$S_{1} $ & $S_{2} $ & $S_{3} $ & $S_{4} $ & $S_{5} $ & $S_{6} $ & $S_{7} $ & $S_{8} $ & $S_{9} $ & $S_{10} $ \\ \hline 
26 & 26 & 18 & 16 & 19 & 19 & 19 & 14 & 19 & 24 \\ \hline 
\end{tabular}



Используя $\chi ^{2} -$критерий согласия Пирсона, требуется на уровне значимости $\alpha =0.01$ проверить гипотезу о том, что элементы случайной выборки имеют равномерное на отрезке $\left[0;1\right]$ распределение.



\textbf{Задача 21.} [Источник случайных чисел: Шведов А.С., ТВиМС-2, стр. 213] Пусть задана реализация случайной выборки $x=\left(x_{1} ,...,x_{n} \right)$ объема $n=200$ наблюдений. И пусть заданы множества $S_{1} =\left[0;{\tfrac{1}{12}} \right]$; $S_{i} =\left({\tfrac{i-1}{12}} ;{\tfrac{i}{12}} \right]$, $i=\overline{2,12}$. 

В следующей таблице указано число элементов выборки $x=\left(x_{1} ,...,x_{n} \right)$, попавших в каждое множество $S_{i} $, $i=\overline{1,12}$.

\begin{tabular}{|p{0.3in}|p{0.3in}|p{0.3in}|p{0.3in}|p{0.3in}|p{0.3in}|p{0.3in}|p{0.3in}|p{0.3in}|p{0.3in}|p{0.3in}|p{0.3in}|} \hline 
$S_{1} $ & $S_{2} $ & $S_{3} $ & $S_{4} $ & $S_{5} $ & $S_{6} $ & $S_{7} $ & $S_{8} $ & $S_{9} $ & $S_{10} $ & $S_{11} $ & $S_{12} $ \\ \hline 
17 & 22 & 12 & 17 & 14 & 15 & 22 & 13 & 14 & 17 & 18 & 19 \\ \hline 
\end{tabular}



Используя $\chi ^{2} -$критерий согласия Пирсона, требуется на уровне значимости $\alpha =0.01$ проверить гипотезу о том, что элементы случайной выборки имеют равномерное на отрезке $\left[0;1\right]$ распределение.



\textbf{Задача 22.} [Источник случайных чисел: датчик excel] Пусть задана реализация случайной выборки $x=\left(x_{1} ,...,x_{n} \right)$ объема $n=200$ наблюдений. И пусть заданы множества $S_{1} =\left[0;{\tfrac{1}{12}} \right]$; $S_{i} =\left({\tfrac{i-1}{12}} ;{\tfrac{i}{12}} \right]$, $i=\overline{2,12}$. 

В следующей таблице указано число элементов выборки $x=\left(x_{1} ,...,x_{n} \right)$, попавших в каждое множество $S_{i} $, $i=\overline{1,12}$.

\begin{tabular}{|p{0.3in}|p{0.3in}|p{0.3in}|p{0.3in}|p{0.3in}|p{0.3in}|p{0.3in}|p{0.3in}|p{0.3in}|p{0.3in}|p{0.3in}|p{0.3in}|} \hline 
$S_{1} $ & $S_{2} $ & $S_{3} $ & $S_{4} $ & $S_{5} $ & $S_{6} $ & $S_{7} $ & $S_{8} $ & $S_{9} $ & $S_{10} $ & $S_{11} $ & $S_{12} $ \\ \hline 
21 & 21 & 22 & 13 & 10 & 18 & 18 & 15 & 13 & 9 & 21 & 19 \\ \hline 
\end{tabular}



Используя $\chi ^{2} -$критерий согласия Пирсона, требуется на уровне значимости $\alpha =0.01$ проверить гипотезу о том, что элементы случайной выборки имеют равномерное на отрезке $\left[0;1\right]$ распределение.



\textbf{Задача 23.} [Источник случайных чисел: датчик excel] Пусть задана реализация случайной выборки $x=\left(x_{1} ,...,x_{n} \right)$ объема $n=500$ наблюдений. И пусть заданы множества $S_{1} =\left(-\infty ;-3\right]$; $S_{10} =\left(3;+\infty \right)$;$S_{i} =\left(-3+6\cdot {\tfrac{i-2}{8}} ;-3+6\cdot {\tfrac{i-1}{8}} \right]$, $i=\overline{2,9}$. 

В следующей таблице указано число элементов выборки $x=\left(x_{1} ,...,x_{n} \right)$, попавших в каждое множество $S_{i} $, $i=\overline{1,10}$.

\begin{tabular}{|p{0.3in}|p{0.3in}|p{0.3in}|p{0.3in}|p{0.3in}|p{0.3in}|p{0.3in}|p{0.3in}|p{0.3in}|p{0.3in}|} \hline 
$S_{1} $ & $S_{2} $ & $S_{3} $ & $S_{4} $ & $S_{5} $ & $S_{6} $ & $S_{7} $ & $S_{8} $ & $S_{9} $ & $S_{10} $ \\ \hline 
0 & 83 & 145 & 132 & 73 & 24 & 3 & 3 & 0 & 3 \\ \hline 
\end{tabular}



Используя $\chi ^{2} -$критерий согласия Пирсона, требуется на уровне значимости $\alpha =0.01$ проверить гипотезу о том, что элементы случайной выборки имеют нормальное стандартное распределение.

Для удобства расчетов ниже приведена следующая информация.

Пусть $p_{i} =P\left(\left\{X_{j} \in S_{i} \right\}\right)$ $\left(i=\overline{1,10}\right)$. Значения гипотетических вероятностей $p_{i} $ $\left(i=\overline{1,10}\right)$ приведены в следующей таблице.

\begin{tabular}{|p{0.4in}|p{0.4in}|p{0.4in}|p{0.4in}|p{0.4in}|p{0.4in}|p{0.4in}|p{0.4in}|p{0.4in}|p{0.4in}|} \hline 
$p_{1} $ & $p_{2} $ & $p_{3} $ & $p_{4} $ & $p_{5} $ & $p_{6} $ & $p_{7} $ & $p_{8} $ & $p_{9} $ & $p_{10} $ \\ \hline 
0.0013 & 0.0108 & 0.0545 & 0.1598 & 0.2733 & 0.2733 & 0.1598 & 0.0545 & 0.0108 & 0.0013 \\ \hline 
\end{tabular}



\textbf{Задача 24.} [Источник случайных чисел: датчик excel] Пусть задана реализация случайной выборки $x=\left(x_{1} ,...,x_{n} \right)$ объема $n=500$ наблюдений. И пусть заданы множества $S_{1} =\left(-\infty ;-3\right]$; $S_{8} =\left(3;+\infty \right)$;$S_{i} =\left(-3+6\cdot {\tfrac{i-2}{6}} ;-3+6\cdot {\tfrac{i-1}{6}} \right]$, $i=\overline{2,7}$. 

В следующей таблице указано число элементов выборки $x=\left(x_{1} ,...,x_{n} \right)$, попавших в каждое множество $S_{i} $, $i=\overline{1,8}$.

\begin{tabular}{|p{0.3in}|p{0.3in}|p{0.3in}|p{0.3in}|p{0.3in}|p{0.3in}|p{0.3in}|p{0.3in}|} \hline 
$S_{1} $ & $S_{2} $ & $S_{3} $ & $S_{4} $ & $S_{5} $ & $S_{6} $ & $S_{7} $ & $S_{8} $ \\ \hline 
0 & 7 & 82 & 176 & 168 & 57 & 7 & 3 \\ \hline 
\end{tabular}



Используя $\chi ^{2} -$критерий согласия Пирсона, требуется на уровне значимости $\alpha =0.01$ проверить гипотезу о том, что элементы случайной выборки имеют нормальное стандартное распределение.

Для удобства расчетов ниже приведена следующая информация.

Пусть $p_{i} =P\left(\left\{X_{j} \in S_{i} \right\}\right)$ $\left(i=\overline{1,8}\right)$. Значения гипотетических вероятностей $p_{i} $ $\left(i=\overline{1,8}\right)$ приведены в следующей таблице.

\begin{tabular}{|p{0.4in}|p{0.4in}|p{0.4in}|p{0.4in}|p{0.4in}|p{0.4in}|p{0.4in}|p{0.4in}|} \hline 
$p_{1} $ & $p_{2} $ & $p_{3} $ & $p_{4} $ & $p_{5} $ & $p_{6} $ & $p_{7} $ & $p_{8} $ \\ \hline 
0.0013 & 0.0214 & 0.1359 & 0.3413 & 0.3413 & 0.1359 & 0.0214 & 0.0013 \\ \hline 
\end{tabular}



\textbf{Задача 25.} [Источник случайных чисел: датчик excel] Пусть задана реализация случайной выборки $x=\left(x_{1} ,...,x_{n} \right)$ объема $n=500$ наблюдений. И пусть заданы множества $S_{1} =\left(-\infty ;-3\right]$; $S_{6} =\left(3;+\infty \right)$;$S_{i} =\left(-3+6\cdot {\tfrac{i-2}{4}} ;-3+6\cdot {\tfrac{i-1}{4}} \right]$, $i=\overline{2,5}$. 

В следующей таблице указано число элементов выборки $x=\left(x_{1} ,...,x_{n} \right)$, попавших в каждое множество $S_{i} $, $i=\overline{1,6}$.

\begin{tabular}{|p{0.3in}|p{0.3in}|p{0.3in}|p{0.3in}|p{0.3in}|p{0.3in}|} \hline 
$S_{1} $ & $S_{2} $ & $S_{3} $ & $S_{4} $ & $S_{5} $ & $S_{6} $ \\ \hline 
0 & 37 & 228 & 205 & 27 & 3 \\ \hline 
\end{tabular}



Используя $\chi ^{2} -$критерий согласия Пирсона, требуется на уровне значимости $\alpha =0.01$ проверить гипотезу о том, что элементы случайной выборки имеют нормальное стандартное распределение.

Для удобства расчетов ниже приведена следующая информация.

Пусть $p_{i} =P\left(\left\{X_{j} \in S_{i} \right\}\right)$ $\left(i=\overline{1,6}\right)$. Значения гипотетических вероятностей $p_{i} $ $\left(i=\overline{1,6}\right)$ приведены в следующей таблице.

\begin{tabular}{|p{0.4in}|p{0.4in}|p{0.4in}|p{0.4in}|p{0.4in}|p{0.4in}|} \hline 
$p_{1} $ & $p_{2} $ & $p_{3} $ & $p_{4} $ & $p_{5} $ & $p_{6} $ \\ \hline 
0.0013 & 0.0654 & 0.4331 & 0.4331 & 0.0654 & 0.0013 \\ \hline 
\end{tabular}

\textbf{Решение. }$r=6$;$n=500$. 

Известно, что $W_{n} =\sum _{i=1}^{r}\frac{\left(\nu _{i} -n\cdot p_{i} \right)^{2} }{n\cdot p_{i} }  \mathop{\sim }\limits^{as} \chi ^{2} \left(r-1\right)$.

Поэтому считаем, что $W_{500} =\sum _{i=1}^{6}\frac{\left(\nu _{i} -n\cdot p_{i} \right)^{2} }{n\cdot p_{i} }  \sim \chi ^{2} \left(5\right)$.

\[\nu _{1} =0; \nu _{2} =37; \nu _{3} =228; \nu _{4} =205; \nu _{5} =27; \nu _{6} =3.\] 



\[W_{500,\; =01;.} =\sum _{i=1}^{6}\frac{\left(\nu _{i} -n\cdot p_{i} \right)^{2} }{n\cdot p_{i} }  ={\rm 11.46}\] 

\[\chi _{:@}^{2} =15.08.\] 

Поскольку $W_{500,\; =01;.} ={\rm 11.46}\in \left[0;15.08\right]$, то основная гипотеза $H_{0} $ не может быть отвергнута.





