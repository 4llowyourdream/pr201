

Листок 2 по ТВ и МС 2013--2014 [08.03.2014]





1

Кафедра математической экономики и эконометрики НИУ ВШЭ. Борзых Д. А.

Листок 2 

Распределение Бернулли. Биномиальное распределение



\textbf{Определение 1.} Случайная величина $X:\Omega \to {\mathbb R}$ имеет \textit{распределение Бернулли} с параметром $p\in (0;1)$, пишут $X\sim Be(p)$, если $X$ принимает значения $0$ и $1$ с вероятностями ${\rm {\mathbb P}}(\{ X=0\} )=1-p$ и ${\rm {\mathbb P}}(\{ X=1\} )=p$.

\textbf{Определение 2.} Случайная величина $X:\Omega \to {\mathbb R}$ имеет \textit{биномиальное распределение} с параметрами $n\in {\mathbb N}$ и $p\in (0;1)$, пишут $X\sim Bi(n,p)$, если $X$ принимает значения $k=0,1,\ldots ,n$ с вероятностями ${\rm {\mathbb P}}(\{ X=k\} )=C_{n}^{k} p^{k} (1-p)^{n-k} $.



\textbf{Задача 1.} Пусть случайная величина $X$ имеет распределение Бернулли с параметром $p$. Найдите

\begin{enumerate}
\item  ${\rm {\mathbb E}}X$,

\item  ${\rm {\mathbb E}}[X^{2} ]$,

\item  $DX$.

\item  Постройте таблицу и функцию распределения случайной величины $X$.
\end{enumerate}

\textbf{Задача 2.} Докажите, что для любых $n\in {\mathbb N}$ и $p\in (0;1)$ имеет место 

\[\sum _{k=0}^{n}C_{n}^{k} p^{k} (1-p)^{n-k}  =1.\] 

\textbf{Задача 3.} Пусть случайная величина $X$ имеет биномиальное распределение с параметрами $n$ и $p$. Найдите

\begin{enumerate}
\item  ${\rm {\mathbb E}}X$,

\item  ${\rm {\mathbb E}}[X\cdot (X-1)]$,

\item  ${\rm {\mathbb E}}[X^{2} ]$,

\item  $DX$,

\item  ${\rm {\mathbb E}}[X\cdot (X-1)\cdot (X-2)]$,

\item  ${\rm {\mathbb E}}[X^{3} ]$,

\item  ${\rm {\mathbb E}}[X\cdot (X-1)\cdot (X-2)\cdot (X-3)]$,

\item  ${\rm {\mathbb E}}[X^{4} ]$.

\item  Постройте таблицу распределения случайной величины $X$.
\end{enumerate}

\textbf{Решение.} (a) ${\rm {\mathbb E}}X=\sum _{k=0}^{n}k\cdot {\rm {\mathbb P}}(\{ X=k\} ) =\sum _{k=1}^{n}k\cdot {\rm {\mathbb P}}(\{ X=k\} ) =\sum _{k=1}^{n}k\cdot C_{n}^{k} p^{k} (1-p)^{n-k}  =$

\[=\sum _{k=1}^{n}k\cdot {\tfrac{n!}{k!(n-k)!}} p^{k} (1-p)^{n-k}  =\sum _{k=1}^{n}{\tfrac{n!}{(k-1)!(n-k)!}} p^{k} (1-p)^{n-k}  =np\sum _{k=1}^{n}{\tfrac{(n-1)!}{(k-1)!(n-k)!}} p^{k-1} (1-p)^{n-k}  =\] 

\[=np\sum _{k=1}^{n}{\tfrac{(n-1)!}{(k-1)!((n-1)-(k-1))!}} p^{k-1} (1-p)^{(n-1)-(k-1)}  =[k-1=l|_{0}^{n-1} ]=np\sum _{l=0}^{n-1}{\tfrac{(n-1)!}{l!((n-1)-l)!}} p^{l} (1-p)^{(n-1)-l}  =\] 

\[=np\sum _{l=0}^{n-1}C_{n-1}^{l} p^{l} (1-p)^{(n-1)-l}  =np(p+(1-p))^{n-1} =np.\] 

(b) ${\rm {\mathbb E}}[X\cdot (X-1)]=\sum _{k=0}^{n}k(k-1)\cdot {\rm {\mathbb P}}(\{ X=k\} ) =\sum _{k=2}^{n}k(k-1)\cdot {\rm {\mathbb P}}(\{ X=k\} ) =$

\[=\sum _{k=2}^{n}k(k-1)\cdot {\tfrac{n!}{k!(n-k)!}} p^{k} (1-p)^{n-k}  =\sum _{k=2}^{n}{\tfrac{n!}{(k-2)!(n-k)!}} p^{k} (1-p)^{n-k}  =\] 

\[=n(n-1)p^{2} \sum _{k=2}^{n}{\tfrac{(n-2)!}{(k-2)!(n-k)!}} p^{k-2} (1-p)^{n-k}  =n(n-1)p^{2} \sum _{k=2}^{n}{\tfrac{(n-2)!}{(k-2)!((n-2)-(k-2))!}} p^{k-2} (1-p)^{(n-2)-(k-2)}  =\] 

\[=[k-2=l|_{0}^{n-2} ]=n(n-1)p^{2} \sum _{l=0}^{n-2}{\tfrac{(n-2)!}{l!((n-2)-l)!}} p^{l} (1-p)^{(n-2)-l}  =n(n-1)p^{2} \sum _{l=0}^{n-2}C_{n-2}^{l} p^{l} (1-p)^{(n-2)-l}  =\] 

\[=n(n-1)p^{2} (p+(1-p))^{n-2} =n(n-1)p^{2} =n^{2} p^{2} -np^{2} .\] 

(c) ${\rm {\mathbb E}}[X^{2} ]={\rm {\mathbb E}}[X\cdot (X-1)]+{\rm {\mathbb E}}X=n^{2} p^{2} -np^{2} +np$.

(d) $DX={\rm {\mathbb E}}[X^{2} ]-[{\rm {\mathbb E}}X]^{2} =n^{2} p^{2} -np^{2} +np-n^{2} p^{2} =np(1-p)$.



\textbf{Задача 4.} Случайная величина $X$ имеет биномиальное распределение с параметрами $n=5$ и $p=1/2$. Найдите 

\begin{enumerate}
\item  ${\rm {\mathbb P}}\left(\left\{\omega \in \Omega :X(\omega )>0\right\}\right)$,

\item  ${\rm {\mathbb P}}\left(\left\{\omega \in \Omega :X(\omega )\in \{ 1,2\} \right\}\right)$;

\item  ${\rm {\mathbb P}}\left(\left\{\omega \in \Omega :X(\omega )=4\right\}\right)$.
\end{enumerate}

\textbf{Задача 5.} Укажите, какие из следующих случайных величин имеют распределение Бернулли, а какие --- биномиальное распределение. Во всех случаях найдите математическое ожидание и дисперсию случайной величины.

\begin{tabular}{|p{0.2in}|p{1.6in}|p{0.5in}|p{0.3in}|p{1.7in}|} \hline 
(a) & $k$ $-1$ $1$ \newline ${\rm {\mathbb P}}_{X} $ $1/2$ $1/2$ \newline   &  & (b) & $k$ $0$ $1$ \newline ${\rm {\mathbb P}}_{X} $ $1/2$ $1/2$ \newline  \\ \hline 
(c) & $k$ $0$ $1$ $2$ \newline ${\rm {\mathbb P}}_{X} $ $1/3$ $1/3$ $1/3$ \newline   &  & (d) & $k$ $0$ $1$ $2$ \newline ${\rm {\mathbb P}}_{X} $ $1/4$ $1/2$ $1/4$ \newline  \\ \hline 
(e) & $k$ $0$ $1$ $2$ $3$ \newline ${\rm {\mathbb P}}_{X} $ $1/8$ $3/8$ $3/8$ $1/8$ \newline  &  & (f) & $k$ $0$ $1$ $2$ $3$ \newline ${\rm {\mathbb P}}_{X} $ $8/27$ $4/9$ $2/9$ $1/27$ \newline  \\ \hline 
\end{tabular}

\textbf{Задача 6.} Найдите наиболее вероятное значение случайной величины $X$, которая имеет биномиальное распределение, если известно, что ${\rm {\mathbb E}}X=1$, а $DX=3/4$.





\textbf{Задача 7.} Может ли случайная величина $X$ иметь биномиальное распределение, если

\begin{enumerate}
\item  ${\rm {\mathbb E}}X=6$, $DX=3$;

\item  ${\rm {\mathbb E}}X=7$, $DX=4$.
\end{enumerate}

\textbf{Задача 8.} Пусть $\Omega =\left\{a,b,c,d\right\}$, ${\mathcal F}=2^{\Omega } $, ${\rm {\mathbb P}}\left(\{ a\} \right)={\rm {\mathbb P}}\left(\{ b\} \right)={\rm {\mathbb P}}\left(\{ c\} \right)={\rm {\mathbb P}}\left(\{ d\} \right)=1/4$ и случайные величины $X_{1} $, $X_{2} $ и $Y$ заданы с помощью таблицы:

\begin{tabular}{|p{0.2in}|p{0.3in}|p{0.3in}|p{0.2in}|p{0.3in}|} \hline 
$\Omega $ & $a$ & $b$ & $c$ & $d$ \\ \hline 
$X_{1} $ & 1 & 1 & 0 & 0 \\ \hline 
$X_{2} $ & 1 & 0 & 1 & 0 \\ \hline 
$Y$ & 2 & 1 & 1 & 0 \\ \hline 
\end{tabular}

\begin{enumerate}
\item  Покажите, что случайные величины $X_{1} $ и $X_{2} $ независимы и имеют распределение Бернулли. 

\item  Убедитесь в том, что случайная величина $Y=X_{1} +X_{2} $ имеет биномиальное распределение с некоторыми параметрами $n$ и $p$. Определите значения этих параметров.
\end{enumerate}

\textbf{Задача 9.} Пусть $\Omega =\left\{a,b,...,h\right\}$, ${\mathcal F}=2^{\Omega } $, ${\rm {\mathbb P}}\left(\{ a\} \right)=...={\rm {\mathbb P}}\left(\{ h\} \right)=1/8$ и случайные величины $X_{1} $, $X_{2} $, $X_{3} $ и $Y$ заданы с помощью таблицы:

\begin{tabular}{|p{0.2in}|p{0.3in}|p{0.3in}|p{0.2in}|p{0.3in}|p{0.3in}|p{0.3in}|p{0.3in}|p{0.3in}|} \hline 
$\Omega $ & $a$ & $b$ & $c$ & $d$ & $e$ & $f$ & $g$ & $h$ \\ \hline 
$X_{1} $ & 1 & 1 & 1 & 1 & 0 & 0 & 0 & 0 \\ \hline 
$X_{2} $ & 1 & 1 & 0 & 0 & 1 & 1 & 0 & 0 \\ \hline 
$X_{3} $ & 1 & 0 & 1 & 0 & 1 & 0 & 1 & 0 \\ \hline 
$Y$ & 3 & 2 & 2 & 1 & 2 & 1 & 1 & 0 \\ \hline 
\end{tabular}

\begin{enumerate}
\item  Покажите, что случайные величины $X_{1} $, $X_{2} $ и $X_{3} $ независимы и имеют распределение Бернулли. 

\item  Убедитесь в том, что случайная величина $Y=X_{1} +X_{2} +X_{3} $ имеет биномиальное распределение с некоторыми параметрами $n$ и $p$. Определите значения этих параметров.
\end{enumerate}

\textbf{Задача 10.} Какова вероятность того, что при бросании десяти монет выпадет семь орлов и три решки?

\textbf{Задача 11. }Бросают пять игральных костей. Чему равна вероятность того, что из пяти выпавших цифр одна -- чётная, а остальные нечётные?

\textbf{Задача 12.} В лифт 9-этажного дома на первом этаже вошли 5 человек. Вычислите вероятность того, что на 6-м этаже:

\begin{enumerate}
\item  не выйдет ни один из них;

\item  выйдет один из них;

\item  выйдут трое из них.
\end{enumerate}

\textbf{Задача 13.} Что вероятнее: выиграть у равносильного партнёра три партии из четырёх или пять партий из восьми? (Ничьи исключаются.)

\textbf{Задача 14.} Бросают 10 монет. Какое число выпавших орлов более вероятно: 5 или 4?

\textbf{Задача 15.} Бросают 19 монет. Какое число выпавших орлов более вероятно: 10 или 9?

\textbf{Задача 16.} Определите вероятность $P_{n} $ того, что при $n$ подбрасываниях монеты орлов выпадет больше, чем решек. Приведите числовые значения этой вероятности при $n=5$ и $n=6$.

\textbf{Задача 17.} Каждый билет лотереи независимо от остальных выигрывает с вероятностью 0.001. У меня 20 билетов. Чему равна вероятность того, что я выиграю:

\begin{enumerate}
\item  хотя бы по одному билету;

\item  не менее чем по двум билетам?
\end{enumerate}

\textbf{Задача 18.} Пусть $\Omega =\left\{a,b,c,d\right\}$, ${\mathcal F}=2^{\Omega } $. Случайные величины $X_{1} $, $X_{2} $ и $Y$ заданы с помощью таблицы:

\begin{tabular}{|p{0.2in}|p{0.3in}|p{0.3in}|p{0.2in}|p{0.3in}|} \hline 
$\Omega $ & $a$ & $b$ & $c$ & $d$ \\ \hline 
${\rm {\mathbb P}}$ & ? & ? & ? & ? \\ \hline 
$X_{1} $ & 1 & 1 & 0 & 0 \\ \hline 
$X_{2} $ & 1 & 0 & 1 & 0 \\ \hline 
$Y$ & 2 & 1 & 1 & 0 \\ \hline 
\end{tabular}

\begin{enumerate}
\item  Подберите вероятности ${\rm {\mathbb P}}\left(\{ a\} \right)$, ${\rm {\mathbb P}}\left(\{ b\} \right)$, ${\rm {\mathbb P}}\left(\{ c\} \right)$ и ${\rm {\mathbb P}}\left(\{ d\} \right)$ так, чтобы случайные величины $X_{1} $ и $X_{2} $ были независимы и имели распределение Бернулли с параметром $p\in (0;1)$.

\item  Убедитесь, что в этом случае случайная величина $Y=X_{1} +X_{2} $ имеет биномиальное распределение с некоторыми параметрами $n$ и $p$. Определите значения этих параметров.
\end{enumerate}

\textbf{Задача 19.} Пусть $\Omega =\left\{a,b,...,h\right\}$, ${\mathcal F}=2^{\Omega } $. Случайные величины $X_{1} $, $X_{2} $, $X_{3} $ и $Y$ заданы с помощью таблицы:

\begin{tabular}{|p{0.2in}|p{0.3in}|p{0.3in}|p{0.2in}|p{0.3in}|p{0.3in}|p{0.3in}|p{0.3in}|p{0.3in}|} \hline 
$\Omega $ & $a$ & $b$ & $c$ & $d$ & $e$ & $f$ & $g$ & $h$ \\ \hline 
${\rm {\mathbb P}}$ & ? & ? & ? & ? & ? & ? & ? & ? \\ \hline 
$X_{1} $ & 1 & 1 & 1 & 1 & 0 & 0 & 0 & 0 \\ \hline 
$X_{2} $ & 1 & 1 & 0 & 0 & 1 & 1 & 0 & 0 \\ \hline 
$X_{3} $ & 1 & 0 & 1 & 0 & 1 & 0 & 1 & 0 \\ \hline 
$Y$ & 3 & 2 & 2 & 1 & 2 & 1 & 1 & 0 \\ \hline 
\end{tabular}

\begin{enumerate}
\item  Подберите вероятности ${\rm {\mathbb P}}\left(\{ a\} \right)$,\dots ,${\rm {\mathbb P}}\left(\{ h\} \right)$ так, чтобы случайные величины $X_{1} $, $X_{2} $ и $X_{3} $ были независимы и имели распределение Бернулли с параметром $p\in (0;1)$.

\item  Убедитесь, что в этом случае случайная величина $Y=X_{1} +X_{2} +X_{3} $ имеет биномиальное распределение с некоторыми параметрами $n$ и $p$. Определите значения этих параметров.
\end{enumerate}

\textbf{Задача 20.} Пусть $\Omega =\left\{\omega =(x_{1} ,x_{2} ):x_{i} \in \{ 0;1\} ,{\rm \; }i=1,2\right\}$, ${\mathcal F}=2^{\Omega } $, 

\[{\rm {\mathbb P}}\left(\left\{(0,0)\right\}\right)=q^{2} , {\rm {\mathbb P}}\left(\left\{(0,1)\right\}\right)={\rm {\mathbb P}}\left(\left\{(1,0)\right\}\right)=pq, {\rm {\mathbb P}}\left(\left\{(1,1)\right\}\right)=p^{2} .\] 

\begin{enumerate}
\item  Покажите, что случайные величины $X_{1} (\omega )=x_{1} $ и $X_{2} (\omega )=x_{2} $ независимы и имеют распределение Бернулли с параметром  $p\in (0;1)$.

\item  Убедитесь в том, что случайная величина $Y=X_{1} +X_{2} $ имеет биномиальное распределение с некоторыми параметрами $n$ и $p$. Определите значения этих параметров.
\end{enumerate}

 \textbf{Задача 21.} Пусть $\Omega =\left\{\omega =(x_{1} ,x_{2} ,x_{3} ):x_{i} \in \{ 0;1\} ,{\rm \; }i=1,3\right\}$, ${\mathcal F}=2^{\Omega } $, 

\[{\rm {\mathbb P}}\left(\left\{(0,0,0)\right\}\right)=q^{3} , {\rm {\mathbb P}}\left(\left\{(0,0,1)\right\}\right)={\rm {\mathbb P}}\left(\left\{(0,1,0)\right\}\right)={\rm {\mathbb P}}\left(\left\{(1,0,0)\right\}\right)=pq^{2} ,\] 

\[{\rm {\mathbb P}}\left(\left\{(0,1,1)\right\}\right)={\rm {\mathbb P}}\left(\left\{(1,0,1)\right\}\right)={\rm {\mathbb P}}\left(\left\{(1,1,0)\right\}\right)=p^{2} q, {\rm {\mathbb P}}\left(\left\{(1,1,1)\right\}\right)=p^{3} .\] 

\begin{enumerate}
\item  Покажите, что случайные величины $X_{1} (\omega )=x_{1} $, $X_{2} (\omega )=x_{2} $ и $X_{3} (\omega )=x_{3} $ независимы и имеют распределение Бернулли с параметром  $p\in (0;1)$.

\item  Убедитесь в том, что случайная величина $Y=X_{1} +X_{2} +X_{3} $ имеет биномиальное распределение с некоторыми параметрами $n$ и $p$. Определите значения этих параметров.
\end{enumerate}

\textbf{Задача 22.} Решив задачи 18 -- 21, постройте пример вероятностного пространства $(\Omega ,{\mathcal F},{\rm {\mathbb P}})$, на котором заданы независимые случайные величины $X_{1} $, $X_{2} $, $X_{3} $ и $X_{4} $, имеющие распределение Бернулли с параметром $p\in (0;1)$. На этом же вероятностном пространстве определите биномиальную случайную величину с параметрами $4$ и $p$.

\textbf{Задача 23.} Найти коэффициент корреляции между числом выпадений единицы и числом выпадений шестёрки при одном подбрасывании игральной кости; при двух подбрасываниях игральной кости.

\textbf{Задача 24.} Пусть случайные величины $X_{1} $ и $X_{2} $ независимы и имеют распределение Бернулли с параметром $p\in (0;1)$. Докажите, что случайная величина $Y=X_{1} +X_{2} $ имеет биномиальное распределение с параметрами $2$ и $p$.

\textbf{Задача 25 (разберите решение задачи).} Пусть случайные величины $X_{1} \sim Bi(2,p)$, $X_{2} \sim Bi(3,p)$ независимы. Докажите, что $Y=X_{1} +X_{2} \sim Bi(5,p)$.

\textbf{Решение.} Докажем сначала вспомогательные формулы:

                                                                    $C_{2}^{0} C_{3}^{0} =C_{5}^{0} $,                                                                (1)

                                                               $C_{2}^{1} C_{3}^{0} +C_{2}^{0} C_{3}^{1} =C_{5}^{1} $,                                                         (2)

                                                         $C_{2}^{2} C_{3}^{0} +C_{2}^{1} C_{3}^{1} +C_{2}^{0} C_{3}^{2} =C_{5}^{2} $,                                                   (3)

                                                        $C_{2}^{2} C_{3}^{1} +C_{2}^{1} C_{3}^{2} +C_{2}^{0} C_{3}^{3} =C_{5}^{3} $,                                                    (4)

                                                               $C_{2}^{2} C_{3}^{2} +C_{2}^{1} C_{3}^{3} =C_{5}^{4} $,                                                         (5)

                                                                     $C_{2}^{2} C_{3}^{3} =C_{5}^{5} $.                                                               (6)

Для этого воспользуемся формулой бинома Ньютона. Имеем

                                                         $(1+x)^{2} =C_{2}^{0} +C_{2}^{1} x+C_{2}^{2} x^{2} $,                                                   (7)

                                                   $(1+x)^{3} =C_{3}^{0} +C_{3}^{1} x+C_{3}^{2} x^{2} +C_{3}^{3} x^{3} $,                                              (8)

                                        $(1+x)^{5} =C_{5}^{0} +C_{5}^{1} x+C_{5}^{2} x^{2} +C_{5}^{3} x^{3} +C_{5}^{4} x^{4} +C_{5}^{5} x^{5} $.                                  (9)

Далее, умножим формулу (7) на формулу (8). Получаем

\[(1+x)^{5} =C_{2}^{0} C_{3}^{0} +(C_{2}^{1} C_{3}^{0} +C_{2}^{0} C_{3}^{1} )x+(C_{2}^{2} C_{3}^{0} +C_{2}^{1} C_{3}^{1} +C_{2}^{0} C_{3}^{2} )x^{2} +\] 

                                 $+(C_{2}^{2} C_{3}^{1} +C_{2}^{1} C_{3}^{2} +C_{2}^{0} C_{3}^{3} )x^{3} +(C_{2}^{2} C_{3}^{2} +C_{2}^{1} C_{3}^{3} )x^{4} +C_{2}^{2} C_{3}^{3} x^{5} $.                          (10)

Из (9) и (10) вытекает справедливость вспомогательных формул (1) -- (6).

Переходим теперь непосредственно к решению задачи. Требуется доказать, что 

                                                           ${\rm {\mathbb P}}\left(\left\{Y=k\right\}\right)=C_{5}^{k} p^{k} q^{5-k} $                                                    (11)

при $k=0,1,...,5$. Последовательно для каждого $k=0,1,...,5$ проверяем истинность формулы (11). Пусть $k=0$. Имеем

\[{\rm {\mathbb P}}\left(\left\{Y=0\right\}\right)={\rm {\mathbb P}}\left(\left\{X_{1} =0\right\}\bigcap \left\{X_{2} =0\right\}\right)=\] 

\[={\rm {\mathbb P}}\left(\left\{X_{1} =0\right\}\right)\cdot {\rm {\mathbb P}}\left(\left\{X_{2} =0\right\}\right)=C_{2}^{0} p^{0} q^{2} \cdot C_{3}^{0} p^{0} q^{3} =C_{5}^{0} p^{0} q^{5} .\] 

Мы воспользовались независимостью случайных величин $X_{1} $ и $X_{2} $, а также формулой (1). При $k=0$ формула (11) доказана. Пусть теперь $k=1$, получаем

\[{\rm {\mathbb P}}\left(\left\{Y=1\right\}\right)={\rm {\mathbb P}}\left(\left\{X_{1} =1\right\}\bigcap \left\{X_{2} =0\right\}\right)+{\rm {\mathbb P}}\left(\left\{X_{1} =0\right\}\bigcap \left\{X_{2} =1\right\}\right)=\] 

\[={\rm {\mathbb P}}\left(\left\{X_{1} =1\right\}\right)\cdot {\rm {\mathbb P}}\left(\left\{X_{2} =0\right\}\right)+{\rm {\mathbb P}}\left(\left\{X_{1} =0\right\}\right)\cdot {\rm {\mathbb P}}\left(\left\{X_{2} =1\right\}\right)=\] 

\[=C_{2}^{1} p^{1} q^{1} \cdot C_{3}^{0} p^{0} q^{3} +C_{2}^{0} p^{0} q^{2} \cdot C_{3}^{1} p^{1} q^{2} =(C_{2}^{1} C_{3}^{0} +C_{2}^{0} C_{3}^{1} )p^{1} q^{4} =C_{5}^{1} p^{1} q^{4} .\] 

Здесь мы снова использовали независимость $X_{1} $ и $X_{2} $, а также формулу (2). При $k=1$ формула (11) также доказана. Для остальных $k=2,...,5$ обоснование формулы (11) проводится аналогичным образом (завершите доказательство\textbf{!!!}). $\square $

\textbf{Задача 26.} Пусть случайные величины $X_{1} \sim Bi(2,p)$, $X_{2} \sim Bi(4,p)$ независимы. Докажите, что $Y=X_{1} +X_{2} \sim Bi(6,p)$.

\textbf{Задача 27.} Вычислите ${\rm {\mathbb P}}\left(\left\{\omega :|X(\omega )-{\rm {\mathbb E}}X|<\sqrt{DX} \right\}\right)$, если $X\sim Bi(10,1/5)$.

\textbf{Задача 28.} Известно, что $X\sim Bi(n,p)$, $n>2$. Найдите 

\begin{enumerate}
\item  ${\rm {\mathbb E}}signX$, 

\item \textbf{ }${\rm {\mathbb E}}\max \{ X-1,0\} $, 

\item \textbf{ }${\rm {\mathbb E}}\max \{ X-2,0\} $. 
\end{enumerate}





