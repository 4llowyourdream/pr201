\documentclass[12pt,a4paper]{article}
\usepackage[utf8]{inputenc}
\usepackage[russian]{babel}

\usepackage{enumitem}
\usepackage{amsmath}
\usepackage{amsfonts}
\usepackage{amssymb}
\usepackage{graphicx}

\newcommand{\E}{\mathrm{E}}
\newcommand{\Var}{\mathrm{Var}}

\usepackage[left=1cm,right=1cm,top=1cm,bottom=1cm]{geometry}
\begin{document}



\textbf{Теория вероятностей}

\begin{enumerate}

\item У тети Маши --- двое детей, один старше другого. Предположим, что вероятности рождения мальчика и девочки равны и не зависят от дня недели, а пол первого и второго ребенка независимы. 
\begin{enumerate}
\item Известно, что старший ребенок --- мальчик. Какова
вероятность того, что у тети Маши есть ребенок-девочка?
\item Известно, что хотя бы один ребенок --- мальчик. Какова
вероятность того, что у тети Маши есть ребенок-девочка?
\item На вопрос: <<А правда ли, что у вас есть сын, родившийся в пятницу?>> тетя Маша ответила: <<Да>>. Какова
вероятность того, что у тети Маши есть ребенок-девочка?
\end{enumerate}


\item Вася решает тест путем проставления каждого ответа наугад. В тесте 5 вопросов. В каждом вопросе 4 варианта ответа. Пусть  $X$  - число правильных ответов,  $Y$  - число неправильных ответов и  $Z=X-Y$ .

\begin{enumerate}
\item Найдите  $\P\left(X>3\right)$ 

\item Найдите  $Var\left(X\right)$  и  $Cov\left(X,Y\right)$ 

\item Найдите  $Corr\left(X,Z\right)$ 
\end{enumerate}


\item Маша подкидывает 300 игральных кубиков. Те, что выпали не на шестёрку, она перекидывает один раз. Обозначим буквой $N$ количество шестёрок на всех кубиках после возможных перекидываний.
\begin{enumerate}
\item Найдите $\E(N)$, $\Var(N)$
\item Какова примерно вероятность того, величина $N$ лежит от 50 до 70?
\item Укажите любой интервал, в который величина $N$ попадает с вероятностью 0.9
\end{enumerate} 

\end{enumerate}

\textbf{Математическая статистика}

\begin{enumerate}[resume]


\item Карл Магнусен сыграл 100 партий в шахматы. Из них он 40 выиграл, 30 проиграл и 30 раз сыграл вничью. Используя метод максимального правдоподобия или критерий Пирсона проверьте гипотезу о том, что все три исхода равновероятны на уровне значимости 5\%.


\item Случайные величины $X_1$, $X_2$, \ldots, $X_{100}$ независимы и имеют пуассоновское распределение с неизвестным параметром $\lambda$. Известно, что $\sum X_i = 150$.
\begin{enumerate}
\item С помощью метода максимального правдоподобия постройте оценку для $\lambda$ и 95\%-ый доверительный интервал.
\item Предположим, что сумма  $X_i$ неизвестна, зато известно, что количество ненулевых $X_i$ равно $20$.  С помощью метода максимального правдоподобия постройте оценку для $\lambda$ и 95\%-ый доверительный интервал.
\item Являются ли полученные оценки несмещенными?
\end{enumerate}


\item  Известно, что  $X_{i}$ независимы и нормальны, $N\left(\mu ;900\right)$.
Исследователь проверяет гипотезу $H_{0}$: $\mu =10$  против
$H_{A}$: $\mu =30$  по выборке из 20 наблюдений. Критерий выглядит
следующим образом: если  $\bar{X}>c$ , то выбрать  $H_{A} $ ,
иначе выбрать  $H_{0} $.
\begin{enumerate}
\item  Рассчитайте вероятности ошибок
первого и второго рода, мощность критерия для $c=25$. 
\item Что произойдет с указанными вероятностями при росте количества
наблюдений если известно что $c\in(10;30)$?
\item Каким должно быть $c$, чтобы вероятность ошибки второго рода
равнялась $0,15$?
\end{enumerate}



\end{enumerate}


\end{document}